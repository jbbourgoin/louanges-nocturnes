%%% Local Variables: 
%%% coding: utf-8
%%% mode: latex
%%% TeX-engine: xetex
%%% End: 

\documentclass[twoside, 10pt, draft]{book}
% format lulu
\usepackage[margin=1.5cm, paperwidth=15.24cm, paperheight=22.86cm]{geometry}
\usepackage{fontspec}
\setmainfont[Mapping=tex-text]{Linux Libertine O}
\usepackage[latin, french]{babel}
\usepackage{at}
\usepackage{textcomp}
\usepackage{url}
\usepackage{verse}
\usepackage{tocbibind}
\usepackage{multind}
\usepackage{hyperref}
\usepackage{color}
\usepackage{xcolor}
\usepackage{fontspec}
\usepackage{pdfcolparallel}
\usepackage{paralist}
\usepackage{parcolumns}
\usepackage{lettrine}
\usepackage{yfonts}
\usepackage{multind}
\usepackage[rm,sc,big,center]{titlesec}
\titlelabel{}
\titleformat*{\section}{\LARGE\scshape\centering}
%% césures
%\usepackage{hyphenat}
%\hyphenpenalty=1000
\usepackage{microtype}
%\usepackage{ragged2e}
\makeindex{antiphonae}
\makeindex{responsoria}
\makeindex{orationes}
\makeindex{hymni}
\makeindex{psalmi}
\makeindex{vetus}
\makeindex{novum}
\makeindex{lectiones}
\bibliographystyle{plain}
%\hfuzz=1pt


\setromanfont[Mapping=tex-text, Numbers=OldStyle, Ligatures=Historical, Alternate=1]{Linux Libertine O}
\newfontfamily\freeserif{FreeSerif}
\newfontfamily\biolinum{Linux Biolinum O}
%\let\guillemotleft«
%\let\guillemotright»
\newcommand{\enluminure}[2]{\lettrine[lines=3]{\small \initfamily #1}{#2}}

\newcommand{\LG}{fr}
\newcommand{\LGredef}{fr}
\parindent = 5pt
\newcommand{\AVANTJXXIII}{
\rule{5.2cm}{0.2cm} \emph{avant 1960} \rule{5.2cm}{0.2cm}
}

\newcommand{\EAVANTJXXIII}{
\rule{5.1cm}{0.2cm} FIN \emph{avant 1960} \rule{5.1cm}{0.2cm}
}

\newcommand{\Respons}{{\freeserif \char"211F}}
\newcommand{\Versicle}{{\freeserif \char"2123}}

% fontes
\newcommand{\fontlat}[1]{\begin{otherlanguage}{latin} {\biolinum #1} \end{otherlanguage}}
\newcommand{\fontfr}[1]{\begin{otherlanguage}{french} #1 \end{otherlanguage}}

\newcommand{\CHAPTER}[1]
{
  \chapter*{#1}
  \addcontentsline{toc}{chapter}{#1}
}

\newcommand{\SECTION}[1]
{
  \section*{#1}
  \addcontentsline{toc}{section}{#1}
}

\newcommand{\SUBSECTION}[1]
{
  \subsection*{#1}
  \addcontentsline{toc}{subsection}{#1}
}

%%%% PARALLEL %%%

\newcommand{\BEGINPARA}{\begin{Parallel}[c]{0.44\textwidth}{0.51\textwidth}}
\newcommand{\ENDPARA}{\end{Parallel}}
\newcommand{\BLEFT}{\selectlanguage{french}\ParallelLText}
\newcommand{\ELEFT}{}
\newcommand{\BRIGHT}{\selectlanguage{french}\ParallelRText}
\newcommand{\ERIGHT}{}

%% Insertion sur deux colonnes simplifiée

\newcommand{\CCOL}[2]{ \BEGINPARA \BLEFT { \fontlat #1 \ELEFT } \BRIGHT { \fontfr #2 \ERIGHT } \ENDPARA}
\newcommand{\CCOLT}[3]{ \subsection*{#3} \BEGINPARA \BLEFT { \fontlat #1 \ELEFT } \BRIGHT { \fontfr #2 \ERIGHT } \ENDPARA}

%% VERSETS

\newcommand{\VERSE}{\item}
\newcommand{\EVERSE}{\\}


%% MACROS INSERTION

% \HYMNUS{\cmdlat}{\cmdfr}{Nom}
\newcommand\HYMNUS[3]
{
\subsection*{Hymne}
  \subsubsection*{#3\index{hymni}{#3}}
  \BEGINPARA
  \BLEFT
  {%para
	\fontlat{#1}
    \ELEFT
  }%para
  \BRIGHT
  {%para
	\fontfr{#2}
    \ERIGHT
  }%para
  \ENDPARA
}

% ANTIPHONÆ
\newcommand{\ANTIPHON}[3]
{
  \BEGINPARA
  \BLEFT
  {%para
    \fontlat{#1}\index{antiphonae}{#3}
    \ELEFT
  }%para
  \BRIGHT
  {%para
    \fontfr{#2}
    \ERIGHT
  }%para
  \ENDPARA
}

% \PSALMUS{counter}{\cmdlat}{\cmdfr}{Nom}
\newcommand{\BVERSELIST}{\begin{inparaenum}[\bfseries 1\/]}
\newcommand{\EVERSELIST}{\end{inparaenum}}
\newcommand{\PSALMUS}[4][0]
{
  \subsection*{Psaume #4\index{psalmi}{#4}}
  \BEGINPARA
  \BLEFT
  {%para
    \BVERSELIST
    \addtocounter{enumi}{#1}
    \fontlat{#2}
    \EVERSELIST
    \ELEFT
  }%para
  \BRIGHT
  {%para
    \BVERSELIST
    \addtocounter{enumi}{#1}
    \fontfr{#3}
    \EVERSELIST
    \ERIGHT
  }%para
  \ENDPARA
  \gloriapatri
}


% VETUS TESTAMENTUM
\newcommand{\VETUS}[5][0]
{
\subsection*{#5}
  \renewcommand{\EVERSE}{}
  \subsubsection*{#4\index{vetus}{#4}}
  \BEGINPARA
  \BLEFT
  {%para
    \BVERSELIST
    \addtocounter{enumi}{#1}
    \fontlat{#2}
    \EVERSELIST
    \ELEFT
  }%para
  \BRIGHT
  {%para
    \BVERSELIST
    \addtocounter{enumi}{#1}
    \fontfr{#3}
    \EVERSELIST
    \ERIGHT
  }%para
  \ENDPARA
  \renewcommand{\EVERSE}{\\}
  \CCOL{\tuautemdomine}{\tuautemdominefr}
}

% NOVUM TESTAMENTUM
\newcommand{\NOVUM}[5][0]
{
\subsection*{#5}
  \renewcommand{\EVERSE}{}
  \subsubsection*{#4\index{novum}{#4}}
  \BEGINPARA
  \BLEFT
  {%para
    \BVERSELIST
    \addtocounter{enumi}{#1}
    \fontlat{#2}
    \EVERSELIST
    \ELEFT
  }%para
  \BRIGHT
  {%para
    \BVERSELIST
    \addtocounter{enumi}{#1}
    \fontfr{#3}
    \EVERSELIST
    \ERIGHT
  }%para
  \ENDPARA
  \renewcommand{\EVERSE}{\\}
}

% LECTIONES
\newcommand{\LECTIO}[3]
{
  \subsubsection*{#3\index{lectiones}{#3}}
  \BEGINPARA
  \BLEFT
  {%para
    \fontlat{#1}
    \ELEFT
  }%para
  \BRIGHT
  {%para
    \fontfr{#2}
    \ERIGHT
  }%para
  \ENDPARA
  \CCOL{\tuautemdomine}{\tuautemdominefr}
}

\newcommand{\ORATIO}[3]
{\subsection*{Collecte}
\BEGINPARA \BLEFT {
\fontlat{\Versicle. Dómine exáudi oratiónem meam. \\
\Respons. Et clamor meus ad te véniat. \\
Oremus.}
 \ELEFT } \BRIGHT {
\fontfr{\Versicle. Seigneur, veuillez exaucer ma prière. \\
\Respons. Et que mon cri parvienne jusqu'à Vous. \\
Prions.}
\ERIGHT } \ENDPARA
\BEGINPARA
\BLEFT
{ \fontlat #1 \ELEFT }
\BRIGHT
{ \fontfr #2 \ERIGHT }
\ENDPARA
\begin{quote}Amen\end{quote}
}

\newcommand{\VERSI}[1]{\begin{quotation}\noindent#1\end{quotation}}

% TITRE

\makeatletter
\def\maketitle{
  \begin{titlepage}
    \begin{center}
      {\Large \@author}
      \null\vfill
    \end{center}
    \begin{center}
      {\huge \@title}\\
      \vskip 0.5cm
      {\Large \emph{\@subtitle}}
      % \vskip 1cm
      % {\texttt{\@email}}
      \vskip 2cm
      % {\Large \@date}
      {\normalsize {\it
          \emph{1\iere{} édition bilingue} latin-français.}}
      \vfill\null
      {\small Creative Commons BY-NC-SA}
    \end{center}
  \end{titlepage}
}
\def\email#1{\def\@email{#1}}
\def\subtitle#1{\def\@subtitle{#1}}
\makeatother


\title{Louanges Nocturnes}
\subtitle{}
\author{\textsc{}}
\email{jeanbaptiste.bourgoin@gmail.com}
\date{2013}

\begin{document}

\newcommand{\psalmi}{
\VERSE Beatus vir qui non abiit in consilio impiorum, et in via peccatorum non stetit,  et in cathedra pestilentiæ non sedit ; \EVERSE
\VERSE sed in lege Domini voluntas ejus, et in lege ejus meditabitur die ac nocte. \EVERSE
\VERSE Et erit tamquam lignum quod plantatum est secus decursus aquarum, quod fructum suum dabit in tempore suo : et folium ejus non defluet ; et omnia quæcumque faciet prosperabuntur. \EVERSE
\VERSE Non sic impii,  non sic ; sed tamquam pulvis quem projicit ventus a facie terræ. \EVERSE
\VERSE Ideo non resurgent impii in judicio, neque peccatores in concilio justorum : \EVERSE
\VERSE quoniam novit Dominus viam justorum, et iter impiorum peribit.

}
\newcommand{\psalmii}{
\VERSE Quare fremuerunt gentes, et populi meditati sunt inania ? \EVERSE
\VERSE Astiterunt reges terræ, et principes convenerunt in unum adversus Dominum,  et adversus christum ejus. \EVERSE
\VERSE Dirumpamus vincula eorum, et projiciamus a nobis jugum ipsorum. \EVERSE
\VERSE Qui habitat in cælis irridebit eos, et Dominus subsannabit eos. \EVERSE
\VERSE Tunc loquetur ad eos in ira sua, et in furore suo conturbabit eos. \EVERSE
\VERSE Ego autem constitutus sum rex ab eosuper Sion,  montem sanctum ejus, 
prædicans præceptum ejus. \EVERSE
\VERSE Dominus dixit ad me : Filius meus es tu ; ego hodie genui te. \EVERSE
\VERSE Postula a me,  et dabo tibi gentes hæreditatem tuam, et possessionem tuam terminos terræ. \EVERSE
\VERSE Reges eos in virga ferrea, et tamquam vas figuli confringes eos. \EVERSE
\VERSE Et nunc,  reges,  intelligite ; erudimini,  qui judicatis terram. \EVERSE
\VERSE Servite Domino in timore, et exsultate ei cum tremore. \EVERSE
\VERSE Apprehendite disciplinam,  nequando irascatur Dominus, et pereatis de via justa. \EVERSE
\VERSE Cum exarserit in brevi ira ejus, beati omnes qui confidunt in eo.

}
\newcommand{\psalmiii}{
\VERSE Psalmus David,  cum fugeret a facie Absalom filii sui.\VERSE Domine,  quid multiplicati sunt qui tribulant me ?Multi insurgunt adversum me ; \EVERSE
\VERSE multi dicunt animæ meæ :Non est salus ipsi in Deo ejus. \EVERSE
\VERSE Tu autem Domine,  susceptor meus es, gloria mea,  et exaltans caput meum. \EVERSE
\VERSE Voce mea ad Dominum clamavi ; et exaudivit me de monte sancto suo. \EVERSE
\VERSE Ego dormivi,  et soporatus sum ; et exsurrexi,  quia Dominus suscepit me. \EVERSE
\VERSE Non timebo millia populi circumdantis me.Exsurge,  Domine ; salvum me fac,  Deus meus. \EVERSE
\VERSE Quoniam tu percussisti omnes adversantes mihi sine causa ; dentes peccatorum contrivisti. \EVERSE
\VERSE Domini est salus ; et super populum tuum benedictio tua.

}
\newcommand{\psalmiv}{
\VERSE In finem,  in carminibus. Psalmus David.\VERSE Cum invocarem exaudivit me Deus justitiæ meæ, in tribulatione dilatasti mihi.
Miserere mei,  et exaudi orationem meam. \EVERSE
\VERSE Filii hominum,  usquequo gravi corde ?ut quid diligitis vanitatem,  et quæritis mendacium ? \EVERSE
\VERSE Et scitote quoniam mirificavit Dominus sanctum suum ; Dominus exaudiet me cum clamavero ad eum. \EVERSE
\VERSE Irascimini,  et nolite peccare ; quæ dicitis in cordibus vestris,  in cubilibus vestris compungimini. \EVERSE
\VERSE Sacrificate sacrificium justitiæ,  et sperate in Domino.Multi dicunt : Quis ostendit nobis bona ? \EVERSE
\VERSE Signatum est super nos lumen vultus tui,  Domine :dedisti lætitiam in corde meo. \EVERSE
\VERSE A fructu frumenti,  vini,  et olei sui,  multiplicati sunt.\VERSE In pace in idipsum dormiam,  et requiescam ; \VERSE quoniam tu,  Domine,  singulariter in spe constituisti me.}
\newcommand{\psalmv}{
\VERSE In finem,  pro ea quæ hæreditatem consequitur. Psalmus David.\VERSE Verba mea auribus percipe,  Domine ; intellige clamorem meum. \EVERSE
\VERSE Intende voci orationis meæ, rex meus et Deus meus. \EVERSE
\VERSE Quoniam ad te orabo,  Domine :mane exaudies vocem meam. \EVERSE
\VERSE Mane astabo tibi,  et videboquoniam non Deus volens iniquitatem tu es. \EVERSE
\VERSE Neque habitabit juxta te malignus, neque permanebunt injusti ante oculos tuos. \EVERSE
\VERSE Odisti omnes qui operantur iniquitatem ; perdes omnes qui loquuntur mendacium.
Virum sanguinum et dolosum abominabitur Dominus. \EVERSE
\VERSE Ego autem in multitudine misericordiæ tuæintroibo in domum tuam ; 
adorabo ad templum sanctum tuum in timore tuo. \EVERSE
\VERSE Domine,  deduc me in justitia tua :propter inimicos meos dirige in conspectu tuo viam meam. \EVERSE
\VERSE Quoniam non est in ore eorum veritas ; cor eorum vanum est. \EVERSE
\VERSE Sepulchrum patens est guttur eorum ; linguis suis dolose agebant :
judica illos,  Deus.
Decidant a cogitationibus suis ; 
secundum multitudinem impietatum eorum expelle eos, 
quoniam irritaverunt te,  Domine. \EVERSE
\VERSE Et lætentur omnes qui sperant in te ; in æternum exsultabunt,  et habitabis in eis.
Et gloriabuntur in te omnes qui diligunt nomen tuum,  \EVERSE
\VERSE quoniam tu benedices justo.Domine,  ut scuto bonæ voluntatis tuæ coronasti nos.

}
\newcommand{\psalmvi}{
\VERSE In finem,  in carminibus. Psalmus David. Pro octava.\VERSE Domine,  ne in furore tuo arguas me, neque in ira tua corripias me. \EVERSE
\VERSE Miserere mei,  Domine,  quoniam infirmus sum ; sana me,  Domine,  quoniam conturbata sunt ossa mea. \EVERSE
\VERSE Et anima mea turbata est valde ; sed tu,  Domine,  usquequo ? \EVERSE
\VERSE Convertere,  Domine,  et eripe animam meam ; salvum me fac propter misericordiam tuam. \EVERSE
\VERSE Quoniam non est in morte qui memor sit tui ; in inferno autem quis confitebitur tibi ? \EVERSE
\VERSE Laboravi in gemitu meo ; lavabo per singulas noctes lectum meum :
lacrimis meis stratum meum rigabo. \EVERSE
\VERSE Turbatus est a furore oculus meus ; inveteravi inter omnes inimicos meos. \EVERSE
\VERSE Discedite a me omnes qui operamini iniquitatem, quoniam exaudivit Dominus vocem fletus mei. \EVERSE
\VERSE Exaudivit Dominus deprecationem meam ; Dominus orationem meam suscepit. \EVERSE
\VERSE Erubescant,  et conturbentur vehementer,  omnes inimici mei ; convertantur,  et erubescant valde velociter.

}
\newcommand{\psalmvii}{
\VERSE Psalmus David,  quem cantavit Domino pro verbis Chusi,  filii Jemini.\VERSE Domine Deus meus,  in te speravi ; salvum me fac ex omnibus persequentibus me,  et libera me : \EVERSE
\VERSE nequando rapiat ut leo animam meam, dum non est qui redimat,  neque qui salvum faciat. \EVERSE
\VERSE Domine Deus meus,  si feci istud, si est iniquitas in manibus meis,  \EVERSE
\VERSE si reddidi retribuentibus mihi mala, decidam merito ab inimicis meis inanis. \EVERSE
\VERSE Persequatur inimicus animam meam,  et comprehendat ; et conculcet in terra vitam meam, 
et gloriam meam in pulverem deducat. \EVERSE
\VERSE Exsurge,  Domine,  in ira tua, et exaltare in finibus inimicorum meorum :
et exsurge,  Domine Deus meus,  in præcepto quod mandasti,  \EVERSE
\VERSE et synagoga populorum circumdabit te :et propter hanc in altum regredere : \EVERSE
\VERSE Dominus judicat populos.Judica me,  Domine,  secundum justitiam meam, 
et secundum innocentiam meam super me. \EVERSE
\VERSE Consumetur nequitia peccatorum,  et diriges justum, scrutans corda et renes,  Deus. \EVERSE
\VERSE Justum adjutorium meum a Domino, qui salvos facit rectos corde. \EVERSE
\VERSE Deus judex justus,  fortis,  et patiens ; numquid irascitur per singulos dies ? \EVERSE
\VERSE Nisi conversi fueritis,  gladium suum vibrabit ; arcum suum tetendit,  et paravit illum. \EVERSE
\VERSE Et in eo paravit vasa mortis, sagittas suas ardentibus effecit. \EVERSE
\VERSE Ecce parturiit injustitiam ; concepit dolorem,  et peperit iniquitatem. \EVERSE
\VERSE Lacum aperuit,  et effodit eum ; et incidit in foveam quam fecit. \EVERSE
\VERSE Convertetur dolor ejus in caput ejus, et in verticem ipsius iniquitas ejus descendet. \EVERSE
\VERSE Confitebor Domino secundum justitiam ejus, et psallam nomini Domini altissimi.

}
\newcommand{\psalmviii}{
\VERSE In finem,  pro torcularibus. Psalmus David.\VERSE Domine,  Dominus noster, quam admirabile est nomen tuum in universa terra !
quoniam elevata est magnificentia tua super cælos. \EVERSE
\VERSE Ex ore infantium et lactentium perfecisti laudem propter inimicos tuos, ut destruas inimicum et ultorem. \EVERSE
\VERSE Quoniam videbo cælos tuos,  opera digitorum tuorum, lunam et stellas quæ tu fundasti. \EVERSE
\VERSE Quid est homo,  quod memor es ejus ?aut filius hominis,  quoniam visitas eum ? \EVERSE
\VERSE Minuisti eum paulominus ab angelis ; gloria et honore coronasti eum ;  \EVERSE
\VERSE et constituisti eum super opera manuum tuarum.\VERSE Omnia subjecisti sub pedibus ejus, oves et boves universas, 
insuper et pecora campi,  \EVERSE
\VERSE volucres cæli,  et pisces marisqui perambulant semitas maris. \EVERSE
\VERSE Domine,  Dominus noster, quam admirabile est nomen tuum in universa terra !

}
\newcommand{\psalmixa}{
\VERSE In finem,  pro occultis filii. Psalmus David.\VERSE Confitebor tibi,  Domine,  in toto corde meo ; narrabo omnia mirabilia tua. \EVERSE
\VERSE Lætabor et exsultabo in te ; psallam nomini tuo,  Altissime. \EVERSE
\VERSE In convertendo inimicum meum retrorsum ; infirmabuntur,  et peribunt a facie tua. \EVERSE
\VERSE Quoniam fecisti judicium meum et causam meam ; sedisti super thronum,  qui judicas justitiam. \EVERSE
\VERSE Increpasti gentes,  et periit impius :nomen eorum delesti in æternum,  et in sæculum sæculi. \EVERSE
\VERSE Inimici defecerunt frameæ in finem, et civitates eorum destruxisti.
Periit memoria eorum cum sonitu ;  \EVERSE
\VERSE et Dominus in æternum permanet.Paravit in judicio thronum suum,  \EVERSE
\VERSE et ipse judicabit orbem terræ in æquitate :judicabit populos in justitia. \EVERSE
\VERSE Et factus est Dominus refugium pauperi ; adjutor in opportunitatibus,  in tribulatione. \EVERSE
\VERSE Et sperent in te qui noverunt nomen tuum, quoniam non dereliquisti quærentes te,  Domine.
}

\newcommand{\psalmixb}{
\VERSE Psallite Domino qui habitat in Sion ; annuntiate inter gentes studia ejus : \EVERSE
\VERSE quoniam requirens sanguinem eorum recordatus est ; non est oblitus clamorem pauperum.\EVERSE
\VERSE Miserere mei,  Domine :vide humilitatem meam de inimicis meis,  \EVERSE
\VERSE qui exaltas me de portis mortis, ut annuntiem omnes laudationes tuas in portis filiæ Sion : \EVERSE
\VERSE exultabo in salutari tuo.Infixæ sunt gentes in interitu quem fecerunt ; 
in laqueo isto quem absconderunt
comprehensus est pes eorum. \EVERSE
\VERSE Cognoscetur Dominus judicia faciens ; in operibus manuum suarum comprehensus est peccator. \EVERSE
\VERSE Convertantur peccatores in infernum, omnes gentes quæ obliviscuntur Deum. \EVERSE
\VERSE Quoniam non in finem oblivio erit pauperis ; patientia pauperum non peribit in finem. \EVERSE
\VERSE Exsurge,  Domine ;  non confortetur homo :judicentur gentes in conspectu tuo. \EVERSE
\VERSE Constitue,  Domine,  legislatorem super eos, ut sciant gentes quoniam homines sunt.
}

\newcommand{\psalmixc}{
\VERSE Ut quid, Dómine, recessísti longe, * déspicis in opportunitátibus, in tribulatióne? \EVERSE
\VERSE Dum supérbit ímpius, incénditur pauper: * comprehendúntur in consíliis quibus cógitant. \EVERSE
\VERSE Quóniam laudátur peccátor in desidériis ánimæ suæ: * et iníquus benedícitur. \EVERSE
\VERSE Exacerbávit Dóminum peccátor, * secúndum multitúdinem iræ suæ non quæret. \EVERSE
\VERSE Non est Deus in conspéctu eius: * inquinátæ sunt viæ illíus in omni témpore. \EVERSE
\VERSE Auferúntur iudícia tua a fácie eius: * ómnium inimicórum suórum dominábitur. \EVERSE
\VERSE Dixit enim in corde suo: * Non movébor a generatióne in generatiónem sine malo. \EVERSE
\VERSE Cuius maledictióne os plenum est, et amaritúdine, et dolo: * sub lingua eius labor et dolor. \EVERSE
\VERSE Sedet in insídiis cum divítibus in occúltis: * ut interfíciat innocéntem. \EVERSE
\VERSE Óculi eius in páuperem respíciunt: * insidiátur in abscóndito, quasi leo in spelúnca sua. \EVERSE
\VERSE Insidiátur ut rápiat páuperem: * rápere páuperem, dum áttrahit eum. \EVERSE
\VERSE In láqueo suo humiliábit eum: * inclinábit se, et cadet, cum dominátus fúerit páuperum.\EVERSE
\VERSE Dixit enim in corde suo: Oblítus est Deus, * avértit fáciem suam ne vídeat in finem.
}

\newcommand{\psalmixd}{
\VERSE Exsúrge, Dómine Deus, exaltétur manus tua: * ne obliviscáris páuperum. \EVERSE
\VERSE Propter quid irritávit ímpius Deum? * dixit enim in corde suo: Non requíret.\EVERSE
\VERSE Vides quóniam tu labórem et dolórem consíderas: * ut tradas eos in manus tuas.\EVERSE
\VERSE Tibi derelíctus est pauper: * órphano tu eris adiútor.\EVERSE
\VERSE Cóntere bráchium peccatóris et malígni: * quærétur peccátum illíus, et non inveniétur.\EVERSE
\VERSE Dóminus regnábit in ætérnum, et in sæculum sæculi: * períbitis, Gentes, de terra illíus.\EVERSE
\VERSE Desidérium páuperum exaudívit Dóminus: * præparatiónem cordis eórum audívit auris tua.\EVERSE
\VERSE Iudicáre pupíllo et húmili, * ut non appónat ultra magnificáre se homo super terram.
}

\newcommand{\psalmx}{
\VERSE In finem. Psalmus David.\VERSE In Domino confido ;  quomodo dicitis animæ meæ :Transmigra in montem sicut passer ? \EVERSE
\VERSE Quoniam ecce peccatores intenderunt arcum ; paraverunt sagittas suas in pharetra, 
ut sagittent in obscuro rectos corde : \EVERSE
\VERSE quoniam quæ perfecisti destruxerunt ; justus autem,  quid fecit ? \EVERSE
\VERSE Dominus in templo sancto suo ; Dominus in cælo sedes ejus.
Oculi ejus in pauperem respiciunt ; 
palpebræ ejus interrogant filios hominum. \EVERSE
\VERSE Dominus interrogat justum et impium ; qui autem diligit iniquitatem,  odit animam suam. \EVERSE
\VERSE Pluet super peccatores laqueos ; ignis et sulphur,  et spiritus procellarum,  pars calicis eorum. \EVERSE
\VERSE Quoniam justus Dominus,  et justitias dilexit :æquitatem vidit vultus ejus.

}
\newcommand{\psalmxi}{
\VERSE In finem,  pro octava. Psalmus David.\VERSE Salvum me fac,  Domine,  quoniam defecit sanctus, quoniam diminutæ sunt veritates a filiis hominum. \EVERSE
\VERSE Vana locuti sunt unusquisque ad proximum suum ; labia dolosa,  in corde et corde locuti sunt. \EVERSE
\VERSE Disperdat Dominus universa labia dolosa, et linguam magniloquam. \EVERSE
\VERSE Qui dixerunt : Linguam nostram magnificabimus ; labia nostra a nobis sunt.
Quis noster dominus est ? \EVERSE
\VERSE Propter miseriam inopum,  et gemitum pauperum, nunc exsurgam,  dicit Dominus.
Ponam in salutari ; 
fiducialiter agam in eo. \EVERSE
\VERSE Eloquia Domini,  eloquia casta ;  argentum igne examinatum, probatum terræ,  purgatum septuplum. \EVERSE
\VERSE Tu,  Domine,  servabis nos, et custodies nos a generatione hac in æternum. \EVERSE
\VERSE In circuitu impii ambulant :secundum altitudinem tuam multiplicasti filios hominum.

}
\newcommand{\psalmxii}{
\VERSE In finem. Psalmus David.Usquequo,  Domine,  oblivisceris me in finem ?
usquequo avertis faciem tuam a me ? \EVERSE
\VERSE quamdiu ponam consilia in anima mea ; dolorem in corde meo per diem ? \EVERSE
\VERSE usquequo exaltabitur inimicus meus super me ?\VERSE Respice,  et exaudi me,  Domine Deus meus.Illumina oculos meos,  ne umquam obdormiam in morte ;  \EVERSE
\VERSE nequando dicat inimicus meus : Prævalui adversus eum.Qui tribulant me exsultabunt si motus fuero ;  \EVERSE
\VERSE ego autem in misericordia tua speravi.Exsultabit cor meum in salutari tuo.
Cantabo Domino qui bona tribuit mihi ; 
et psallam nomini Domini altissimi.

}
\newcommand{\psalmxiii}{
\VERSE In finem. Psalmus David.Dixit insipiens in corde suo : Non est Deus.
Corrupti sunt,  et abominabiles facti sunt in studiis suis ; 
non est qui faciat bonum,  non est usque ad unum. \EVERSE
\VERSE Dominus de cælo prospexit super filios hominum, ut videat si est intelligens,  aut requirens Deum. \EVERSE
\VERSE Omnes declinaverunt,  simul inutiles facti sunt.Non est qui faciat bonum,  non est usque ad unum.
Sepulchrum patens est guttur eorum ; 
linguis suis dolose agebant.
Venenum aspidum sub labiis eorum, 
quorum os maledictione et amaritudine plenum est ; 
veloces pedes eorum ad effundendum sanguinem.
Contritio et infelicitas in viis eorum, 
et viam pacis non cognoverunt ; 
non est timor Dei ante oculos eorum. \EVERSE
\VERSE Nonne cognoscent omnes qui operantur iniquitatem, qui devorant plebem meam sicut escam panis ? \EVERSE
\VERSE Dominum non invocaverunt ; illic trepidaverunt timore,  ubi non erat timor. \EVERSE
\VERSE Quoniam Dominus in generatione justa est :consilium inopis confudistis, 
quoniam Dominus spes ejus est. \EVERSE
\VERSE Quis dabit ex Sion salutare Israël ?Cum averterit Dominus captivitatem plebis suæ, 
exsultabit Jacob,  et lætabitur Israël.

}
\newcommand{\psalmxiv}{
\VERSE Psalmus David.Domine,  quis habitabit in tabernaculo tuo ?
aut quis requiescet in monte sancto tuo ? \EVERSE
\VERSE Qui ingreditur sine macula, et operatur justitiam ;  \EVERSE
\VERSE qui loquitur veritatem in corde suo :qui non egit dolum in lingua sua, 
nec fecit proximo suo malum, 
et opprobrium non accepit adversus proximos suos. \EVERSE
\VERSE Ad nihilum deductus est in conspectu ejus malignus ; timentes autem Dominum glorificat.
Qui jurat proximo suo,  et non decipit ;  \EVERSE
\VERSE qui pecuniam suam non dedit ad usuram, et munera super innocentem non accepit :
qui facit hæc non movebitur in æternum.

}
\newcommand{\psalmxv}{
\VERSE Tituli inscriptio,  ipsi David.Conserva me,  Domine,  quoniam speravi in te. \EVERSE
\VERSE Dixi Domino : Deus meus es tu, quoniam bonorum meorum non eges. \EVERSE
\VERSE Sanctis qui sunt in terra ejus, mirificavit omnes voluntates meas in eis. \EVERSE
\VERSE Multiplicatæ sunt infirmitates eorum :postea acceleraverunt.
Non congregabo conventicula eorum de sanguinibus, 
nec memor ero nominum eorum per labia mea. \EVERSE
\VERSE Dominus pars hæreditatis meæ,  et calicis mei :tu es qui restitues hæreditatem meam mihi. \EVERSE
\VERSE Funes ceciderunt mihi in præclaris ; etenim hæreditas mea præclara est mihi. \EVERSE
\VERSE Benedicam Dominum qui tribuit mihi intellectum ; insuper et usque ad noctem increpuerunt me renes mei. \EVERSE
\VERSE Providebam Dominum in conspectu meo semper :quoniam a dextris est mihi,  ne commovear. \EVERSE
\VERSE Propter hoc lætatum est cor meum,  et exsultavit lingua mea ; insuper et caro mea requiescet in spe. \EVERSE
\VERSE Quoniam non derelinques animam meam in inferno, nec dabis sanctum tuum videre corruptionem.
Notas mihi fecisti vias vitæ ; 
adimplebis me lætitia cum vultu tuo :
delectationes in dextera tua usque in finem.

}
\newcommand{\psalmxvi}{
\VERSE Oratio David.Exaudi,  Domine,  justitiam meam ; 
intende deprecationem meam.
Auribus percipe orationem meam, 
non in labiis dolosis. \EVERSE
\VERSE De vultu tuo judicium meum prodeat ; oculi tui videant æquitates. \EVERSE
\VERSE Probasti cor meum,  et visitasti nocte ; igne me examinasti,  et non est inventa in me iniquitas. \EVERSE
\VERSE Ut non loquatur os meum opera hominum :propter verba labiorum tuorum,  ego custodivi vias duras. \EVERSE
\VERSE Perfice gressus meos in semitis tuis, ut non moveantur vestigia mea. \EVERSE
\VERSE Ego clamavi,  quoniam exaudisti me,  Deus ; inclina aurem tuam mihi,  et exaudi verba mea. \EVERSE
\VERSE Mirifica misericordias tuas, qui salvos facis sperantes in te. \EVERSE
\VERSE A resistentibus dexteræ tuæ custodi meut pupillam oculi.
Sub umbra alarum tuarum protege me \EVERSE
\VERSE a facie impiorum qui me afflixerunt.Inimici mei animam meam circumdederunt ;  \EVERSE
\VERSE adipem suum concluserunt :os eorum locutum est superbiam. \EVERSE
\VERSE Projicientes me nunc circumdederunt me ; oculos suos statuerunt declinare in terram. \EVERSE
\VERSE Susceperunt me sicut leo paratus ad prædam, et sicut catulus leonis habitans in abditis. \EVERSE
\VERSE Exsurge,  Domine : præveni eum,  et supplanta eum :eripe animam meam ab impio ; 
frameam tuam 14 ab inimicis manus tuæ.
Domine,  a paucis de terra divide eos in vita eorum ; 
de absconditis tuis adimpletus est venter eorum.
Saturati sunt filiis, 
et dimiserunt reliquias suas parvulis suis. \EVERSE
\VERSE Ego autem in justitia apparebo conspectui tuo ; satiabor cum apparuerit gloria tua.

}
\newcommand{\psalmxviia}{
\VERSE In finem. Puero Domini David,  qui locutus est Domino verba cantici hujus,  in die qua eripuit eum Dominus de manu omnium inimicorum ejus,  et de manu Saul,  et dixit :\VERSE Diligam te,  Domine,  fortitudo mea.\VERSE Dominus firmamentum meum,  et refugium meum,  et liberator meus.Deus meus adjutor meus,  et sperabo in eum ; 
protector meus,  et cornu salutis meæ,  et susceptor meus. \EVERSE
\VERSE Laudans invocabo Dominum, et ab inimicis meis salvus ero. \EVERSE
\VERSE Circumdederunt me dolores mortis, et torrentes iniquitatis conturbaverunt me. \EVERSE
\VERSE Dolores inferni circumdederunt me ; præoccupaverunt me laquei mortis. \EVERSE
\VERSE In tribulatione mea invocavi Dominum, et ad Deum meum clamavi :
et exaudivit de templo sancto suo vocem meam ; 
et clamor meus in conspectu ejus introivit in aures ejus. \EVERSE
\VERSE Commota est,  et contremuit terra ; fundamenta montium conturbata sunt,  et commota sunt :
quoniam iratus est eis. \EVERSE
\VERSE Ascendit fumus in ira ejus, et ignis a facie ejus exarsit ; 
carbones succensi sunt ab eo. \EVERSE
\VERSE Inclinavit cælos,  et descendit, et caligo sub pedibus ejus. \EVERSE
\VERSE Et ascendit super cherubim,  et volavit ; volavit super pennas ventorum. \EVERSE
\VERSE Et posuit tenebras latibulum suum ; in circuitu ejus tabernaculum ejus, 
tenebrosa aqua in nubibus aëris. \EVERSE
\VERSE Præ fulgore in conspectu ejus nubes transierunt ; grando et carbones ignis. \EVERSE
\VERSE Et intonuit de cælo Dominus, et Altissimus dedit vocem suam :
grando et carbones ignis. \EVERSE
\VERSE Et misit sagittas suas,  et dissipavit eos ; fulgura multiplicavit,  et conturbavit eos. \EVERSE
\VERSE Et apparuerunt fontes aquarum, et revelata sunt fundamenta orbis terrarum, 
ab increpatione tua,  Domine, 
ab inspiratione spiritus iræ tuæ. \EVERSE
\VERSE Misit de summo,  et accepit me ; et assumpsit me de aquis multis. \EVERSE
\VERSE Eripuit me de inimicis meis fortissimis,  et ab his qui oderunt me.Quoniam confortati sunt super me ;  \EVERSE
\VERSE prævenerunt me in die afflictionis meæ :et factus est Dominus protector meus. \EVERSE
\VERSE Et eduxit me in latitudinem ; salvum me fecit,  quoniam voluit me,  \EVERSE
\VERSE et retribuet mihi Dominus secundum justitiam meam, et secundum puritatem manuum mearum retribuet mihi : \EVERSE
\VERSE quia custodivi vias Domini, nec impie gessi a Deo meo ;  \EVERSE
\VERSE quoniam omnia judicia ejus in conspectu meo, et justitias ejus non repuli a me. \EVERSE
\VERSE Et ero immaculatus cum eo ; et observabo me ab iniquitate mea. \EVERSE
\VERSE Et retribuet mihi Dominus secundum justitiam meam, et secundum puritatem manuum mearum in conspectu oculorum ejus. \EVERSE
\VERSE Cum sancto sanctus eris, et cum viro innocente innocens eris,  \EVERSE
\VERSE et cum electo electus eris, et cum perverso perverteris. \EVERSE
\VERSE Quoniam tu populum humilem salvum facies, et oculos superborum humiliabis. \EVERSE
\VERSE Quoniam tu illuminas lucernam meam,  Domine ; Deus meus,  illumina tenebras meas. \EVERSE
\VERSE Quoniam in te eripiar a tentatione ; et in Deo meo transgrediar murum. \EVERSE
\VERSE Deus meus,  impolluta via ejus ; eloquia Domini igne examinata :
protector est omnium sperantium in se.

}
\newcommand{\psalmxviib}{
\VERSE Quoniam quis deus præter Dominum ?aut quis deus præter Deum nostrum ? \EVERSE
\VERSE Deus qui præcinxit me virtute, et posuit immaculatam viam meam ;  \EVERSE
\VERSE qui perfecit pedes meos tamquam cervorum, et super excelsa statuens me ;  \EVERSE
\VERSE qui docet manus meas ad prælium.Et posuisti,  ut arcum æreum,  brachia mea,  \EVERSE
\VERSE et dedisti mihi protectionem salutis tuæ :et dextera tua suscepit me, 
et disciplina tua correxit me in finem, 
et disciplina tua ipsa me docebit. \EVERSE
\VERSE Dilatasti gressus meos subtus me, et non sunt infirmata vestigia mea. \EVERSE
\VERSE Persequar inimicos meos,  et comprehendam illos ; et non convertar donec deficiant. \EVERSE
\VERSE Confringam illos,  nec poterunt stare ; cadent subtus pedes meos. \EVERSE
\VERSE Et præcinxisti me virtute ad bellum, et supplantasti insurgentes in me subtus me. \EVERSE
\VERSE Et inimicos meos dedisti mihi dorsum, et odientes me disperdidisti. \EVERSE
\VERSE Clamaverunt,  nec erat qui salvos faceret ; ad Dominum,  nec exaudivit eos. \EVERSE
\VERSE Et comminuam eos ut pulverem ante faciem venti ; ut lutum platearum delebo eos. \EVERSE
\VERSE Eripies me de contradictionibus populi ; constitues me in caput gentium. \EVERSE
\VERSE Populus quem non cognovi servivit mihi ; in auditu auris obedivit mihi. \EVERSE
\VERSE Filii alieni mentiti sunt mihi, filii alieni inveterati sunt, 
et claudicaverunt a semitis suis. \EVERSE
\VERSE Vivit Dominus,  et benedictus Deus meus, et exaltetur Deus salutis meæ. \EVERSE
\VERSE Deus qui das ndictas mihi, et subdis populos sub me ; 
liberator meus de inimicis meis iracundis. \EVERSE
\VERSE Et ab insurgentibus in me exaltabis me ; a viro iniquo eripies me. \EVERSE
\VERSE Propterea confitebor tibi in nationibus,  Domine, et nomini tuo psalmum dicam ;  \EVERSE
\VERSE magnificans salutes regis ejus, et faciens misericordiam christo suo David, 
et semini ejus usque in sæculum.

}
\newcommand{\psalmxviii}{
\VERSE In finem. Psalmus David.\VERSE Cæli enarrant gloriam Dei, et opera manuum ejus annuntiat firmamentum. \EVERSE
\VERSE Dies diei eructat verbum, et nox nocti indicat scientiam. \EVERSE
\VERSE Non sunt loquelæ,  neque sermones, quorum non audiantur voces eorum. \EVERSE
\VERSE In omnem terram exivit sonus eorum, et in fines orbis terræ verba eorum. \EVERSE
\VERSE In sole posuit tabernaculum suum ; et ipse tamquam sponsus procedens de thalamo suo.
Exsultavit ut gigas ad currendam viam ;  \EVERSE
\VERSE a summo cælo egressio ejus.Et occursus ejus usque ad summum ejus ; 
nec est qui se abscondat a calore ejus. \EVERSE
\VERSE Lex Domini immaculata,  convertens animas ; testimonium Domini fidele,  sapientiam præstans parvulis. \EVERSE
\VERSE Justitiæ Domini rectæ,  lætificantes corda ; præceptum Domini lucidum,  illuminans oculos. \EVERSE
\VERSE Timor Domini sanctus,  permanens in sæculum sæculi ; judicia Domini vera,  justificata in semetipsa,  \EVERSE
\VERSE desiderabilia super aurum et lapidem pretiosum multum, et dulciora super mel et favum. \EVERSE
\VERSE Etenim servus tuus custodit ea ; in custodiendis illis retributio multa. \EVERSE
\VERSE Delicta quis intelligit ?ab occultis meis munda me ;  \EVERSE
\VERSE et ab alienis parce servo tuo.Si mei non fuerint dominati,  tunc immaculatus ero, 
et emundabor a delicto maximo. \EVERSE
\VERSE Et erunt ut complaceant eloquia oris mei, et meditatio cordis mei in conspectu tuo semper.
Domine,  adjutor meus,  et redemptor meus.

}
\newcommand{\psalmxix}{
\VERSE In finem. Psalmus David.\VERSE Exaudiat te Dominus in die tribulationis ; protegat te nomen Dei Jacob. \EVERSE
\VERSE Mittat tibi auxilium de sancto, et de Sion tueatur te. \EVERSE
\VERSE Memor sit omnis sacrificii tui, et holocaustum tuum pingue fiat. \EVERSE
\VERSE Tribuat tibi secundum cor tuum, et omne consilium tuum confirmet. \EVERSE
\VERSE Lætabimur in salutari tuo ; et in nomine Dei nostri magnificabimur. \EVERSE
\VERSE Impleat Dominus omnes petitiones tuas ; nunc cognovi quoniam salvum fecit Dominus christum suum.
Exaudiet illum de cælo sancto suo, 
in potentatibus salus dexteræ ejus. \EVERSE
\VERSE Hi in curribus,  et hi in equis ; nos autem in nomine Domini Dei nostri invocabimus. \EVERSE
\VERSE Ipsi obligati sunt,  et ceciderunt ; nos autem surreximus,  et erecti sumus. \EVERSE
\VERSE Domine,  salvum fac regem, et exaudi nos in die qua invocaverimus te.

}
\newcommand{\psalmxx}{
\VERSE In finem. Psalmus David.\VERSE Domine,  in virtute tua lætabitur rex, et super salutare tuum exsultabit vehementer. \EVERSE
\VERSE Desiderium cordis ejus tribuisti ei, et voluntate labiorum ejus non fraudasti eum. \EVERSE
\VERSE Quoniam prævenisti eum in benedictionibus dulcedinis ; posuisti in capite ejus coronam de lapide pretioso. \EVERSE
\VERSE Vitam petiit a te,  et tribuisti ei longitudinem dierum, in sæculum,  et in sæculum sæculi. \EVERSE
\VERSE Magna est gloria ejus in salutari tuo ; gloriam et magnum decorem impones super eum. \EVERSE
\VERSE Quoniam dabis eum in benedictionem in sæculum sæculi ; lætificabis eum in gaudio cum vultu tuo. \EVERSE
\VERSE Quoniam rex sperat in Domino, et in misericordia Altissimi non commovebitur. \EVERSE
\VERSE Inveniatur manus tua omnibus inimicis tuis ; dextera tua inveniat omnes qui te oderunt. \EVERSE
\VERSE Pones eos ut clibanum ignis in tempore vultus tui :Dominus in ira sua conturbabit eos, 
et devorabit eos ignis. \EVERSE
\VERSE Fructum eorum de terra perdes, et semen eorum a filiis hominum,  \EVERSE
\VERSE quoniam declinaverunt in te mala ; cogitaverunt consilia quæ non potuerunt stabilire. \EVERSE
\VERSE Quoniam pones eos dorsum ; in reliquiis tuis præparabis vultum eorum. \EVERSE
\VERSE Exaltare,  Domine,  in virtute tua ; cantabimus et psallemus virtutes tuas.

}
\newcommand{\psalmxxi}{
\VERSE In finem,  pro susceptione matutina. Psalmus David.\VERSE Deus,  Deus meus,  respice in me : quare me dereliquisti ?longe a salute mea verba delictorum meorum. \EVERSE
\VERSE Deus meus,  clamabo per diem,  et non exaudies ; et nocte,  et non ad insipientiam mihi. \EVERSE
\VERSE Tu autem in sancto habitas,  laus Israël.\VERSE In te speraverunt patres nostri ; speraverunt,  et liberasti eos. \EVERSE
\VERSE Ad te clamaverunt,  et salvi facti sunt ; in te speraverunt,  et non sunt confusi. \EVERSE
\VERSE Ego autem sum vermis,  et non homo ; opprobrium hominum,  et abjectio plebis. \EVERSE
\VERSE Omnes videntes me deriserunt me ; locuti sunt labiis,  et moverunt caput. \EVERSE
\VERSE Speravit in Domino,  eripiat eum :salvum faciat eum,  quoniam vult eum. \EVERSE
\VERSE Quoniam tu es qui extraxisti me de ventre, spes mea ab uberibus matris meæ. \EVERSE
\VERSE In te projectus sum ex utero ; de ventre matris meæ Deus meus es tu : \EVERSE
\VERSE ne discesseris a me, quoniam tribulatio proxima est, 
quoniam non est qui adjuvet. \EVERSE
\VERSE Circumdederunt me vituli multi ; tauri pingues obsederunt me. \EVERSE
\VERSE Aperuerunt super me os suum, sicut leo rapiens et rugiens. \EVERSE
\VERSE Sicut aqua effusus sum, et dispersa sunt omnia ossa mea :
factum est cor meum tamquam cera liquescens in medio ventris mei. \EVERSE
\VERSE Aruit tamquam testa virtus mea, et lingua mea adhæsit faucibus meis :
et in pulverem mortis deduxisti me. \EVERSE
\VERSE Quoniam circumdederunt me canes multi ; concilium malignantium obsedit me.
Foderunt manus meas et pedes meos ;  \EVERSE
\VERSE dinumeraverunt omnia ossa mea.Ipsi vero consideraverunt et inspexerunt me. \EVERSE
\VERSE Diviserunt sibi vestimenta mea, et super vestem meam miserunt sortem. \EVERSE
\VERSE Tu autem,  Domine,  ne elongaveris auxilium tuum a me ; ad defensionem meam conspice. \EVERSE
\VERSE Erue a framea,  Deus,  animam meam, et de manu canis unicam meam. \EVERSE
\VERSE Salva me ex ore leonis, et a cornibus unicornium humilitatem meam. \EVERSE
\VERSE Narrabo nomen tuum fratribus meis ; in medio ecclesiæ laudabo te. \EVERSE
\VERSE Qui timetis Dominum,  laudate eum ; universum semen Jacob,  glorificate eum. \EVERSE
\VERSE Timeat eum omne semen Israël, quoniam non sprevit,  neque despexit deprecationem pauperis, 
nec avertit faciem suam a me :
et cum clamarem ad eum,  exaudivit me. \EVERSE
\VERSE Apud te laus mea in ecclesia magna ; vota mea reddam in conspectu timentium eum. \EVERSE
\VERSE Edent pauperes,  et saturabuntur, et laudabunt Dominum qui requirunt eum :
vivent corda eorum in sæculum sæculi. \EVERSE
\VERSE Reminiscentur et convertentur ad Dominum universi fines terræ ; et adorabunt in conspectu ejus universæ familiæ gentium : \EVERSE
\VERSE quoniam Domini est regnum, et ipse dominabitur gentium. \EVERSE
\VERSE Manducaverunt et adoraverunt omnes pingues terræ ; in conspectu ejus cadent omnes qui descendunt in terram. \EVERSE
\VERSE Et anima mea illi vivet ; et semen meum serviet ipsi. \EVERSE
\VERSE Annuntiabitur Domino generatio ventura ; et annuntiabunt cæli justitiam ejus
populo qui nascetur,  quem fecit Dominus.

}
\newcommand{\psalmxxii}{
\VERSE Psalmus David.Dominus regit me,  et nihil mihi deerit : \EVERSE
\VERSE in loco pascuæ,  ibi me collocavit.Super aquam refectionis educavit me ;  \EVERSE
\VERSE animam meam convertit.Deduxit me super semitas justitiæ
propter nomen suum. \EVERSE
\VERSE Nam etsi ambulavero in medio umbræ mortis, non timebo mala,  quoniam tu mecum es.
Virga tua,  et baculus tuus, 
ipsa me consolata sunt. \EVERSE
\VERSE Parasti in conspectu meo mensamadversus eos qui tribulant me ; 
impinguasti in oleo caput meum :
et calix meus inebrians,  quam præclarus est ! \EVERSE
\VERSE Et misericordia tua subsequetur meomnibus diebus vitæ meæ ; 
et ut inhabitem in domo Domini
in longitudinem dierum.

}
\newcommand{\psalmxxiii}{
\VERSE Prima sabbati. Psalmus David.Domini est terra,  et plenitudo ejus ; 
orbis terrarum,  et universi qui habitant in eo. \EVERSE
\VERSE Quia ipse super maria fundavit eum, et super flumina præparavit eum. \EVERSE
\VERSE Quis ascendet in montem Domini ?aut quis stabit in loco sancto ejus ? \EVERSE
\VERSE Innocens manibus et mundo corde, qui non accepit in vano animam suam, 
nec juravit in dolo proximo suo : \EVERSE
\VERSE hic accipiet benedictionem a Domino, et misericordiam a Deo salutari suo. \EVERSE
\VERSE Hæc est generatio quærentium eum, quærentium faciem Dei Jacob. \EVERSE
\VERSE Attollite portas,  principes,  vestras, et elevamini,  portæ æternales, 
et introibit rex gloriæ. \EVERSE
\VERSE Quis est iste rex gloriæ ?Dominus fortis et potens, 
Dominus potens in prælio. \EVERSE
\VERSE Attollite portas,  principes,  vestras, et elevamini,  portæ æternales, 
et introibit rex gloriæ. \EVERSE
\VERSE Quis est iste rex gloriæ ?Dominus virtutum ipse est rex gloriæ.

}
\newcommand{\psalmxxiv}{
\VERSE In finem. Psalmus David.Ad te,  Domine,  levavi animam meam : \EVERSE
\VERSE Deus meus,  in te confido ;  non erubescam.\VERSE Neque irrideant me inimici mei :etenim universi qui sustinent te,  non confundentur. \EVERSE
\VERSE Confundantur omnes iniqua agentes supervacue.Vias tuas,  Domine,  demonstra mihi, 
et semitas tuas edoce me. \EVERSE
\VERSE Dirige me in veritate tua,  et doce me, quia tu es Deus salvator meus, 
et te sustinui tota die. \EVERSE
\VERSE Reminiscere miserationum tuarum,  Domine, et misericordiarum tuarum quæ a sæculo sunt. \EVERSE
\VERSE Delicta juventutis meæ,  et ignorantias meas,  ne memineris.Secundum misericordiam tuam memento mei tu, 
propter bonitatem tuam,  Domine. \EVERSE
\VERSE Dulcis et rectus Dominus ; propter hoc legem dabit delinquentibus in via. \EVERSE
\VERSE Diriget mansuetos in judicio ; docebit mites vias suas. \EVERSE
\VERSE Universæ viæ Domini,  misericordia et veritas, requirentibus testamentum ejus et testimonia ejus. \EVERSE
\VERSE Propter nomen tuum,  Domine, propitiaberis peccato meo ;  multum est enim. \EVERSE
\VERSE Quis est homo qui timet Dominum ?legem statuit ei in via quam elegit. \EVERSE
\VERSE Anima ejus in bonis demorabitur, et semen ejus hæreditabit terram. \EVERSE
\VERSE Firmamentum est Dominus timentibus eum ; et testamentum ipsius ut manifestetur illis. \EVERSE
\VERSE Oculi mei semper ad Dominum, quoniam ipse evellet de laqueo pedes meos. \EVERSE
\VERSE Respice in me,  et miserere mei, quia unicus et pauper sum ego. \EVERSE
\VERSE Tribulationes cordis mei multiplicatæ sunt :de necessitatibus meis erue me. \EVERSE
\VERSE Vide humilitatem meam et laborem meum, et dimitte universa delicta mea. \EVERSE
\VERSE Respice inimicos meos,  quoniam multiplicati sunt, et odio iniquo oderunt me. \EVERSE
\VERSE Custodi animam meam,  et erue me :non erubescam,  quoniam speravi in te. \EVERSE
\VERSE Innocentes et recti adhæserunt mihi, quia sustinui te. \EVERSE
\VERSE Libera,  Deus,  Israëlex omnibus tribulationibus suis.

}
\newcommand{\psalmxxv}{
\VERSE In finem. Psalmus David.Judica me,  Domine,  quoniam ego in innocentia mea ingressus sum, 
et in Domino sperans non infirmabor. \EVERSE
\VERSE Proba me,  Domine,  et tenta me ; ure renes meos et cor meum. \EVERSE
\VERSE Quoniam misericordia tua ante oculos meos est, et complacui in veritate tua. \EVERSE
\VERSE Non sedi cum concilio vanitatis, et cum iniqua gerentibus non introibo. \EVERSE
\VERSE Odivi ecclesiam malignantium, et cum impiis non sedebo. \EVERSE
\VERSE Lavabo inter innocentes manus meas, et circumdabo altare tuum,  Domine : \EVERSE
\VERSE ut audiam vocem laudis, et enarrem universa mirabilia tua. \EVERSE
\VERSE Domine,  dilexi decorem domus tuæ, et locum habitationis gloriæ tuæ. \EVERSE
\VERSE Ne perdas cum impiis,  Deus,  animam meam, et cum viris sanguinum vitam meam : \EVERSE
\VERSE in quorum manibus iniquitates sunt ; dextera eorum repleta est muneribus. \EVERSE
\VERSE Ego autem in innocentia mea ingressus sum ; redime me,  et miserere mei. \EVERSE
\VERSE Pes meus stetit in directo ; in ecclesiis benedicam te,  Domine.

}
\newcommand{\psalmxxvi}{
\VERSE Psalmus David,  priusquam liniretur.Dominus illuminatio mea et salus mea : quem timebo ?
Dominus protector vitæ meæ :
a quo trepidabo ? \EVERSE
\VERSE Dum appropiant super me nocentes ut edant carnes meas, qui tribulant me inimici mei, 
ipsi infirmati sunt et ceciderunt. \EVERSE
\VERSE Si consistant adversum me castra,  non timebit cor meum ; si exsurgat adversum me prælium,  in hoc ego sperabo. \EVERSE
\VERSE Unam petii a Domino,  hanc requiram, ut inhabitem in domo Domini omnibus diebus vitæ meæ ; 
ut videam voluptatem Domini,  et visitem templum ejus. \EVERSE
\VERSE Quoniam abscondit me in tabernaculo suo ; in die malorum protexit me in abscondito tabernaculi sui. \EVERSE
\VERSE In petra exaltavit me, et nunc exaltavit caput meum super inimicos meos.
Circuivi,  et immolavi in tabernaculo ejus hostiam vociferationis ; 
cantabo,  et psalmum dicam Domino. \EVERSE
\VERSE Exaudi,  Domine,  vocem meam,  qua clamavi ad te ; miserere mei,  et exaudi me. \EVERSE
\VERSE Tibi dixit cor meum : Exquisivit te facies mea ; faciem tuam,  Domine,  requiram. \EVERSE
\VERSE Ne avertas faciem tuam a me ; ne declines in ira a servo tuo.
Adjutor meus esto ;  ne derelinquas me, 
neque despicias me,  Deus salutaris meus. \EVERSE
\VERSE Quoniam pater meus et mater mea dereliquerunt me ; Dominus autem assumpsit me. \EVERSE
\VERSE Legem pone mihi,  Domine,  in via tua, et dirige me in semitam rectam,  propter inimicos meos. \EVERSE
\VERSE Ne tradideris me in animas tribulantium me, quoniam insurrexerunt in me testes iniqui, 
et mentita est iniquitas sibi. \EVERSE
\VERSE Credo videre bona Domini in terra viventium.\VERSE Expecta Dominum,  viriliter age :et confortetur cor tuum,  et sustine Dominum.

}
\newcommand{\psalmxxvii}{
\VERSE Psalmus ipsi David.Ad te,  Domine,  clamabo ;  Deus meus,  ne sileas a me :
nequando taceas a me,  et assimilabor descendentibus in lacum. \EVERSE
\VERSE Exaudi,  Domine,  vocem deprecationis meæ dum oro ad te ; dum extollo manus meas ad templum sanctum tuum. \EVERSE
\VERSE Ne simul trahas me cum peccatoribus, et cum operantibus iniquitatem ne perdas me ; 
qui loquuntur pacem cum proximo suo, 
mala autem in cordibus eorum. \EVERSE
\VERSE Da illis secundum opera eorum, et secundum nequitiam adinventionum ipsorum.
Secundum opera manuum eorum tribue illis ; 
redde retributionem eorum ipsis. \EVERSE
\VERSE Quoniam non intellexerunt opera Domini, et in opera manuum ejus destrues illos, 
et non ædificabis eos. \EVERSE
\VERSE Benedictus Dominus, quoniam exaudivit vocem deprecationis meæ. \EVERSE
\VERSE Dominus adjutor meus et protector meus ; in ipso speravit cor meum,  et adjutus sum :
et refloruit caro mea, 
et ex voluntate mea confitebor ei. \EVERSE
\VERSE Dominus fortitudo plebis suæ, et protector salvationum christi sui est. \EVERSE
\VERSE Salvum fac populum tuum,  Domine,  et benedic hæreditati tuæ ; et rege eos,  et extolle illos usque in æternum.

}
\newcommand{\psalmxxviii}{
\VERSE Psalmus David,  in consummatione tabernaculi.Afferte Domino,  filii Dei, 
afferte Domino filios arietum. \EVERSE
\VERSE Afferte Domino gloriam et honorem ; afferte Domino gloriam nomini ejus ; 
adorate Dominum in atrio sancto ejus. \EVERSE
\VERSE Vox Domini super aquas ; Deus majestatis intonuit :
Dominus super aquas multas. \EVERSE
\VERSE Vox Domini in virtute ; vox Domini in magnificentia. \EVERSE
\VERSE Vox Domini confringentis cedros, et confringet Dominus cedros Libani : \EVERSE
\VERSE et comminuet eas,  tamquam vitulum Libani, et dilectus quemadmodum filius unicornium. \EVERSE
\VERSE Vox Domini intercidentis flammam ignis ; \VERSE vox Domini concutientis desertum :et commovebit Dominus desertum Cades. \EVERSE
\VERSE Vox Domini præparantis cervos :et revelabit condensa, 
et in templo ejus omnes dicent gloriam. \EVERSE
\VERSE Dominus diluvium inhabitare facit, et sedebit Dominus rex in æternum. \EVERSE
\VERSE Dominus virtutem populo suo dabit ; Dominus benedicet populo suo in pace.

}
\newcommand{\psalmxxix}{
\VERSE Psalmus cantici,  in dedicatione domus David.\VERSE Exaltabo te,  Domine,  quoniam suscepisti me, nec delectasti inimicos meos super me. \EVERSE
\VERSE Domine Deus meus,  clamavi ad te,  et sanasti me.\VERSE Domine,  eduxisti ab inferno animam meam ; salvasti me a descendentibus in lacum. \EVERSE
\VERSE Psallite Domino,  sancti ejus ; et confitemini memoriæ sanctitatis ejus. \EVERSE
\VERSE Quoniam ira in indignatione ejus, et vita in voluntate ejus :
ad vesperum demorabitur fletus, 
et ad matutinum lætitia. \EVERSE
\VERSE Ego autem dixi in abundantia mea :Non movebor in æternum. \EVERSE
\VERSE Domine,  in voluntate tua præstitisti decori meo virtutem ; avertisti faciem tuam a me,  et factus sum conturbatus. \EVERSE
\VERSE Ad te,  Domine,  clamabo, et ad Deum meum deprecabor. \EVERSE
\VERSE Quæ utilitas in sanguine meo, dum descendo in corruptionem ?
numquid confitebitur tibi pulvis, 
aut annuntiabit veritatem tuam ? \EVERSE
\VERSE Audivit Dominus,  et misertus est mei ; Dominus factus est adjutor meus. \EVERSE
\VERSE Convertisti planctum meum in gaudium mihi ; conscidisti saccum meum,  et circumdedisti me lætitia : \EVERSE
\VERSE ut cantet tibi gloria mea,  et non compungar.Domine Deus meus,  in æternum confitebor tibi.

}
\newcommand{\psalmxxx}{
\VERSE In finem. Psalmus David,  pro extasi.\VERSE In te,  Domine,  speravi ; non confundar in æternum :
in justitia tua libera me. \EVERSE
\VERSE Inclina ad me aurem tuam ; accelera ut eruas me.
Esto mihi in Deum protectorem, 
et in domum refugii,  ut salvum me facias : \EVERSE
\VERSE quoniam fortitudo mea et refugium meum es tu ; et propter nomen tuum deduces me et enutries me. \EVERSE
\VERSE Educes me de laqueo hoc quem absconderunt mihi, quoniam tu es protector meus. \EVERSE
\VERSE In manus tuas commendo spiritum meum ; redemisti me,  Domine Deus veritatis. \EVERSE
\VERSE Odisti observantes vanitates supervacue ; ego autem in Domino speravi. \EVERSE
\VERSE Exsultabo,  et lætabor in misericordia tua, quoniam respexisti humilitatem meam ; 
salvasti de necessitatibus animam meam. \EVERSE
\VERSE Nec conclusisti me in manibus inimici :statuisti in loco spatioso pedes meos. \EVERSE
\VERSE Miserere mei,  Domine,  quoniam tribulor ; conturbatus est in ira oculus meus,  anima mea,  et venter meus. \EVERSE
\VERSE Quoniam defecit in dolore vita mea, et anni mei in gemitibus.
Infirmata est in paupertate virtus mea, 
et ossa mea conturbata sunt. \EVERSE
\VERSE Super omnes inimicos meos factus sum opprobrium, et vicinis meis valde,  et timor notis meis ; 
qui videbant me foras fugerunt a me. \EVERSE
\VERSE Oblivioni datus sum,  tamquam mortuus a corde ; factus sum tamquam vas perditum : \EVERSE
\VERSE quoniam audivi vituperationem multorum commorantium in circuitu.In eo dum convenirent simul adversum me, 
accipere animam meam consiliati sunt. \EVERSE
\VERSE Ego autem in te speravi,  Domine ; dixi : Deus meus es tu ;  \EVERSE
\VERSE in manibus tuis sortes meæ :eripe me de manu inimicorum meorum,  et a persequentibus me. \EVERSE
\VERSE Illustra faciem tuam super servum tuum ; salvum me fac in misericordia tua. \EVERSE
\VERSE Domine,  non confundar,  quoniam invocavi te.Erubescant impii,  et deducantur in infernum ;  \EVERSE
\VERSE muta fiant labia dolosa, quæ loquuntur adversus justum iniquitatem, 
in superbia,  et in abusione. \EVERSE
\VERSE Quam magna multitudo dulcedinis tuæ,  Domine, quam abscondisti timentibus te ; 
perfecisti eis qui sperant in te in conspectu filiorum hominum ! \EVERSE
\VERSE Abscondes eos in abscondito faciei tuæ a conturbatione hominum ; proteges eos in tabernaculo tuo,  a contradictione linguarum. \EVERSE
\VERSE Benedictus Dominus, quoniam mirificavit misericordiam suam mihi in civitate munita. \EVERSE
\VERSE Ego autem dixi in excessu mentis meæ :Projectus sum a facie oculorum tuorum :
ideo exaudisti vocem orationis meæ,  dum clamarem ad te. \EVERSE
\VERSE Diligite Dominum,  omnes sancti ejus, quoniam veritatem requiret Dominus, 
et retribuet abundanter facientibus superbiam. \EVERSE
\VERSE Viriliter agite,  et confortetur cor vestrum, omnes qui speratis in Domino.

}
\newcommand{\psalmxxxi}{
\VERSE Ipsi David intellectus.Beati quorum remissæ sunt iniquitates, 
et quorum tecta sunt peccata. \EVERSE
\VERSE Beatus vir cui non imputavit Dominus peccatum, nec est in spiritu ejus dolus. \EVERSE
\VERSE Quoniam tacui,  inveteraverunt ossa mea, dum clamarem tota die. \EVERSE
\VERSE Quoniam die ac nocte gravata est super me manus tua, conversus sum in ærumna mea,  dum configitur spina. \EVERSE
\VERSE Delictum meum cognitum tibi feci, et injustitiam meam non abscondi.
Dixi : Confitebor adversum me injustitiam meam Domino ; 
et tu remisisti impietatem peccati mei. \EVERSE
\VERSE Pro hac orabit ad te omnis sanctusin tempore opportuno.
Verumtamen in diluvio aquarum multarum, 
ad eum non approximabunt. \EVERSE
\VERSE Tu es refugium meum a tribulatione quæ circumdedit me ; exsultatio mea,  erue me a circumdantibus me. \EVERSE
\VERSE Intellectum tibi dabo,  et instruam te in via hac qua gradieris ; firmabo super te oculos meos. \EVERSE
\VERSE Nolite fieri sicut equus et mulus, quibus non est intellectus.
In camo et freno maxillas eorum constringe, 
qui non approximant ad te. \EVERSE
\VERSE Multa flagella peccatoris ; sperantem autem in Domino misericordia circumdabit. \EVERSE
\VERSE Lætamini in Domino,  et exsultate,  justi ; et gloriamini,  omnes recti corde.

}
\newcommand{\psalmxxxii}{
\VERSE Psalmus David.Exsultate,  justi,  in Domino ; 
rectos decet collaudatio. \EVERSE
\VERSE Confitemini Domino in cithara ; in psalterio decem chordarum psallite illi. \EVERSE
\VERSE Cantate ei canticum novum ; bene psallite ei in vociferatione. \EVERSE
\VERSE Quia rectum est verbum Domini, et omnia opera ejus in fide. \EVERSE
\VERSE Diligit misericordiam et judicium ; misericordia Domini plena est terra. \EVERSE
\VERSE Verbo Domini cæli firmati sunt, et spiritu oris ejus omnis virtus eorum. \EVERSE
\VERSE Congregans sicut in utre aquas maris ; ponens in thesauris abyssos. \EVERSE
\VERSE Timeat Dominum omnis terra ; ab eo autem commoveantur omnes inhabitantes orbem. \EVERSE
\VERSE Quoniam ipse dixit,  et facta sunt ; ipse mandavit et creata sunt. \EVERSE
\VERSE Dominus dissipat consilia gentium ; reprobat autem cogitationes populorum, 
et reprobat consilia principum. \EVERSE
\VERSE Consilium autem Domini in æternum manet ; cogitationes cordis ejus in generatione et generationem. \EVERSE
\VERSE Beata gens cujus est Dominus Deus ejus ; populus quem elegit in hæreditatem sibi. \EVERSE
\VERSE De cælo respexit Dominus ; vidit omnes filios hominum. \EVERSE
\VERSE De præparato habitaculo suorespexit super omnes qui habitant terram : \EVERSE
\VERSE qui finxit sigillatim corda eorum ; qui intelligit omnia opera eorum. \EVERSE
\VERSE Non salvatur rex per multam virtutem, et gigas non salvabitur in multitudine virtutis suæ. \EVERSE
\VERSE Fallax equus ad salutem ; in abundantia autem virtutis suæ non salvabitur. \EVERSE
\VERSE Ecce oculi Domini super metuentes eum, et in eis qui sperant super misericordia ejus : \EVERSE
\VERSE ut eruat a morte animas eorum, et alat eos in fame. \EVERSE
\VERSE Anima nostra sustinet Dominum, quoniam adjutor et protector noster est. \EVERSE
\VERSE Quia in eo lætabitur cor nostrum, et in nomine sancto ejus speravimus. \EVERSE
\VERSE Fiat misericordia tua,  Domine,  super nos, quemadmodum speravimus in te.

}
\newcommand{\psalmxxxiii}{
\VERSE Davidi,  cum immutavit vultum suum coram Achimelech,  et dimisit eum,  et abiit.\VERSE Benedicam Dominum in omni tempore ; semper laus ejus in ore meo. \EVERSE
\VERSE In Domino laudabitur anima mea :audiant mansueti,  et lætentur. \EVERSE
\VERSE Magnificate Dominum mecum, et exaltemus nomen ejus in idipsum. \EVERSE
\VERSE Exquisivi Dominum,  et exaudivit me ; et ex omnibus tribulationibus meis eripuit me. \EVERSE
\VERSE Accedite ad eum,  et illuminamini ; et facies vestræ non confundentur. \EVERSE
\VERSE Iste pauper clamavit,  et Dominus exaudivit eum, et de omnibus tribulationibus ejus salvavit eum. \EVERSE
\VERSE Immittet angelus Domini in circuitu timentium eum, et eripiet eos. \EVERSE
\VERSE Gustate et videte quoniam suavis est Dominus ; beatus vir qui sperat in eo. \EVERSE
\VERSE Timete Dominum,  omnes sancti ejus, quoniam non est inopia timentibus eum. \EVERSE
\VERSE Divites eguerunt,  et esurierunt ; inquirentes autem Dominum non minuentur omni bono. \EVERSE
\VERSE Venite,  filii ;  audite me :timorem Domini docebo vos. \EVERSE
\VERSE Quis est homo qui vult vitam ; diligit dies videre bonos ? \EVERSE
\VERSE Prohibe linguam tuam a malo, et labia tua ne loquantur dolum. \EVERSE
\VERSE Diverte a malo,  et fac bonum ; inquire pacem,  et persequere eam. \EVERSE
\VERSE Oculi Domini super justos, et aures ejus in preces eorum. \EVERSE
\VERSE Vultus autem Domini super facientes mala, ut perdat de terra memoriam eorum. \EVERSE
\VERSE Clamaverunt justi,  et Dominus exaudivit eos ; et ex omnibus tribulationibus eorum liberavit eos. \EVERSE
\VERSE Juxta est Dominus iis qui tribulato sunt corde, et humiles spiritu salvabit. \EVERSE
\VERSE Multæ tribulationes justorum ; et de omnibus his liberabit eos Dominus. \EVERSE
\VERSE Custodit Dominus omnia ossa eorum :unum ex his non conteretur. \EVERSE
\VERSE Mors peccatorum pessima ; et qui oderunt justum delinquent. \EVERSE
\VERSE Redimet Dominus animas servorum suorum, et non delinquent omnes qui sperant in eo.

}
\newcommand{\psalmxxxiv}{
\VERSE Ipsi David.Judica,  Domine,  nocentes me ; 
expugna impugnantes me. \EVERSE
\VERSE Apprehende arma et scutum, et exsurge in adjutorium mihi. \EVERSE
\VERSE Effunde frameam,  et conclude adversus eos qui persequuntur me ; dic animæ meæ : Salus tua ego sum. \EVERSE
\VERSE Confundantur et revereantur quærentes animam meam ; avertantur retrorsum et confundantur cogitantes mihi mala. \EVERSE
\VERSE Fiant tamquam pulvis ante faciem venti, et angelus Domini coarctans eos. \EVERSE
\VERSE Fiat via illorum tenebræ et lubricum, et angelus Domini persequens eos. \EVERSE
\VERSE Quoniam gratis absconderunt mihi interitum laquei sui ;  supervacue exprobraverunt animam meam.\VERSE Veniat illi laqueus quem ignorat, et captio quam abscondit apprehendat eum, 
et in laqueum cadat in ipsum. \EVERSE
\VERSE Anima autem mea exsultabit in Domino, et delectabitur super salutari suo. \EVERSE
\VERSE Omnia ossa mea dicent :Domine,  quis similis tibi ?
eripiens inopem de manu fortiorum ejus ; 
egenum et pauperem a diripientibus eum. \EVERSE
\VERSE Surgentes testes iniqui, quæ ignorabam interrogabant me. \EVERSE
\VERSE Retribuebant mihi mala pro bonis, sterilitatem animæ meæ. \EVERSE
\VERSE Ego autem,  cum mihi molesti essent,  induebar cilicio ; humiliabam in jejunio animam meam, 
et oratio mea in sinu meo convertetur. \EVERSE
\VERSE Quasi proximum et quasi fratrem nostrum sic complacebam ; quasi lugens et contristatus sic humiliabar. \EVERSE
\VERSE Et adversum me lætati sunt,  et convenerunt ; congregata sunt super me flagella,  et ignoravi. \EVERSE
\VERSE Dissipati sunt,  nec compuncti ; tentaverunt me,  subsannaverunt me subsannatione ; 
frenduerunt super me dentibus suis. \EVERSE
\VERSE Domine,  quando respicies ?Restitue animam meam a malignitate eorum ; 
a leonibus unicam meam. \EVERSE
\VERSE Confitebor tibi in ecclesia magna ;  in populo gravi laudabo te.\VERSE Non supergaudeant mihi qui adversantur mihi inique, qui oderunt me gratis,  et annuunt oculis. \EVERSE
\VERSE Quoniam mihi quidem pacifice loquebantur ; et in iracundia terræ loquentes,  dolos cogitabant. \EVERSE
\VERSE Et dilataverunt super me os suum ; dixerunt : Euge,  euge ! viderunt oculi nostri. \EVERSE
\VERSE Vidisti,  Domine : ne sileas ; Domine,  ne discedas a me. \EVERSE
\VERSE Exsurge et intende judicio meo,  Deus meus ; et Dominus meus,  in causam meam. \EVERSE
\VERSE Judica me secundum justitiam tuam,  Domine Deus meus, et non supergaudeant mihi. \EVERSE
\VERSE Non dicant in cordibus suis : Euge,  euge,  animæ nostræ ; nec dicant : Devoravimus eum. \EVERSE
\VERSE Erubescant et revereantur simul qui gratulantur malis meis ; induantur confusione et reverentia qui magna loquuntur super me. \EVERSE
\VERSE Exsultent et lætentur qui volunt justitiam meam ; et dicant semper : Magnificetur Dominus,  qui volunt pacem servi ejus. \EVERSE
\VERSE Et lingua mea meditabitur justitiam tuam ; tota die laudem tuam.

}
\newcommand{\psalmxxxv}{
\VERSE In finem. Servo Domini ipsi David.\VERSE Dixit injustus ut delinquat in semetipso :non est timor Dei ante oculos ejus. \EVERSE
\VERSE Quoniam dolose egit in conspectu ejus, ut inveniatur iniquitas ejus ad odium. \EVERSE
\VERSE Verba oris ejus iniquitas,  et dolus ; noluit intelligere ut bene ageret. \EVERSE
\VERSE Iniquitatem meditatus est in cubili suo ; astitit omni viæ non bonæ :
malitiam autem non odivit. \EVERSE
\VERSE Domine,  in cælo misericordia tua, et veritas tua usque ad nubes. \EVERSE
\VERSE Justitia tua sicut montes Dei ; judicia tua abyssus multa.
Homines et jumenta salvabis,  Domine,  \EVERSE
\VERSE quemadmodum multiplicasti misericordiam tuam,  Deus.Filii autem hominum in tegmine alarum tuarum sperabunt. \EVERSE
\VERSE Inebriabuntur ab ubertate domus tuæ, et torrente voluptatis tuæ potabis eos : \EVERSE
\VERSE quoniam apud te est fons vitæ, et in lumine tuo videbimus lumen. \EVERSE
\VERSE Prætende misericordiam tuam scientibus te, et justitiam tuam his qui recto sunt corde. \EVERSE
\VERSE Non veniat mihi pes superbiæ, et manus peccatoris non moveat me. \EVERSE
\VERSE Ibi ceciderunt qui operantur iniquitatem ; expulsi sunt,  nec potuerunt stare.

}
\newcommand{\psalmxxxvi}{
\VERSE Psalmus ipsi David.Noli æmulari in malignantibus, 
neque zelaveris facientes iniquitatem : \EVERSE
\VERSE quoniam tamquam fœnum velociter arescent, et quemadmodum olera herbarum cito decident. \EVERSE
\VERSE Spera in Domino,  et fac bonitatem ; et inhabita terram,  et pasceris in divitiis ejus. \EVERSE
\VERSE Delectare in Domino, et dabit tibi petitiones cordis tui. \EVERSE
\VERSE Revela Domino viam tuam, et spera in eo,  et ipse faciet. \EVERSE
\VERSE Et educet quasi lumen justitiam tuam, et judicium tuum tamquam meridiem. \EVERSE
\VERSE Subditus esto Domino,  et ora eum.Noli æmulari in eo qui prosperatur in via sua ; 
in homine faciente injustitias. \EVERSE
\VERSE Desine ab ira,  et derelinque furorem ; noli æmulari ut maligneris. \EVERSE
\VERSE Quoniam qui malignantur exterminabuntur ; sustinentes autem Dominum,  ipsi hæreditabunt terram. \EVERSE
\VERSE Et adhuc pusillum,  et non erit peccator ; et quæres locum ejus,  et non invenies. \EVERSE
\VERSE Mansueti autem hæreditabunt terram, et delectabuntur in multitudine pacis. \EVERSE
\VERSE Observabit peccator justum, et stridebit super eum dentibus suis. \EVERSE
\VERSE Dominus autem irridebit eum, quoniam prospicit quod veniet dies ejus. \EVERSE
\VERSE Gladium evaginaverunt peccatores ; intenderunt arcum suum :
ut dejiciant pauperem et inopem, 
ut trucident rectos corde. \EVERSE
\VERSE Gladius eorum intret in corda ipsorum, et arcus eorum confringatur. \EVERSE
\VERSE Melius est modicum justo, super divitias peccatorum multas : \EVERSE
\VERSE quoniam brachia peccatorum conterentur :confirmat autem justos Dominus. \EVERSE
\VERSE Novit Dominus dies immaculatorum, et hæreditas eorum in æternum erit. \EVERSE
\VERSE Non confundentur in tempore malo, et in diebus famis saturabuntur : \EVERSE
\VERSE quia peccatores peribunt.Inimici vero Domini mox ut honorificati fuerint et exaltati, 
deficientes quemadmodum fumus deficient. \EVERSE
\VERSE Mutuabitur peccator,  et non solvet ; justus autem miseretur et tribuet : \EVERSE
\VERSE quia benedicentes ei hæreditabunt terram ; maledicentes autem ei disperibunt. \EVERSE
\VERSE Apud Dominum gressus hominis dirigentur, et viam ejus volet. \EVERSE
\VERSE Cum ceciderit,  non collidetur, quia Dominus supponit manum suam. \EVERSE
\VERSE Junior fui,  etenim senui ; et non vidi justum derelictum, 
nec semen ejus quærens panem. \EVERSE
\VERSE Tota die miseretur et commodat ; et semen illius in benedictione erit. \EVERSE
\VERSE Declina a malo,  et fac bonum, et inhabita in sæculum sæculi : \EVERSE
\VERSE quia Dominus amat judicium, et non derelinquet sanctos suos :
in æternum conservabuntur.
Injusti punientur, 
et semen impiorum peribit. \EVERSE
\VERSE Justi autem hæreditabunt terram, et inhabitabunt in sæculum sæculi super eam. \EVERSE
\VERSE Os justi meditabitur sapientiam, et lingua ejus loquetur judicium. \EVERSE
\VERSE Lex Dei ejus in corde ipsius, et non supplantabuntur gressus ejus. \EVERSE
\VERSE Considerat peccator justum, et quærit mortificare eum. \EVERSE
\VERSE Dominus autem non derelinquet eum in manibus ejus, nec damnabit eum cum judicabitur illi. \EVERSE
\VERSE Exspecta Dominum,  et custodi viam ejus, et exaltabit te ut hæreditate capias terram :
cum perierint peccatores,  videbis. \EVERSE
\VERSE Vidi impium superexaltatum, et elevatum sicut cedros Libani : \EVERSE
\VERSE et transivi,  et ecce non erat ; et quæsivi eum,  et non est inventus locus ejus. \EVERSE
\VERSE Custodi innocentiam,  et vide æquitatem, quoniam sunt reliquiæ homini pacifico. \EVERSE
\VERSE Injusti autem disperibunt simul ; reliquiæ impiorum interibunt. \EVERSE
\VERSE Salus autem justorum a Domino ; et protector eorum in tempore tribulationis. \EVERSE
\VERSE Et adjuvabit eos Dominus,  et liberabit eos ; et eruet eos a peccatoribus,  et salvabit eos, 
quia speraverunt in eo.

}
\newcommand{\psalmxxxvii}{
\VERSE Psalmus David,  in rememorationem de sabbato.\VERSE Domine,  ne in furore tuo arguas me, neque in ira tua corripias me : \EVERSE
\VERSE quoniam sagittæ tuæ infixæ sunt mihi, et confirmasti super me manum tuam. \EVERSE
\VERSE Non est sanitas in carne mea,  a facie iræ tuæ ; non est pax ossibus meis,  a facie peccatorum meorum : \EVERSE
\VERSE quoniam iniquitates meæ supergressæ sunt caput meum, et sicut onus grave gravatæ sunt super me. \EVERSE
\VERSE Putruerunt et corruptæ sunt cicatrices meæ, a facie insipientiæ meæ. \EVERSE
\VERSE Miser factus sum et curvatus sum usque in finem ; tota die contristatus ingrediebar. \EVERSE
\VERSE Quoniam lumbi mei impleti sunt illusionibus, et non est sanitas in carne mea. \EVERSE
\VERSE Afflictus sum,  et humiliatus sum nimis ; rugiebam a gemitu cordis mei. \EVERSE
\VERSE Domine,  ante te omne desiderium meum, et gemitus meus a te non est absconditus. \EVERSE
\VERSE Cor meum conturbatum est ; dereliquit me virtus mea,  et lumen oculorum meorum, 
et ipsum non est mecum. \EVERSE
\VERSE Amici mei et proximi mei adversum me appropinquaverunt,  et steterunt ; et qui juxta me erant,  de longe steterunt :
et vim faciebant qui quærebant animam meam. \EVERSE
\VERSE Et qui inquirebant mala mihi,  locuti sunt vanitates, et dolos tota die meditabantur. \EVERSE
\VERSE Ego autem,  tamquam surdus,  non audiebam ; et sicut mutus non aperiens os suum. \EVERSE
\VERSE Et factus sum sicut homo non audiens, et non habens in ore suo redargutiones. \EVERSE
\VERSE Quoniam in te,  Domine,  speravi ; tu exaudies me,  Domine Deus meus. \EVERSE
\VERSE Quia dixi : Nequando supergaudeant mihi inimici mei ; et dum commoventur pedes mei,  super me magna locuti sunt. \EVERSE
\VERSE Quoniam ego in flagella paratus sum, et dolor meus in conspectu meo semper. \EVERSE
\VERSE Quoniam iniquitatem meam annuntiabo, et cogitabo pro peccato meo. \EVERSE
\VERSE Inimici autem mei vivunt,  et confirmati sunt super me :et multiplicati sunt qui oderunt me inique. \EVERSE
\VERSE Qui retribuunt mala pro bonis detrahebant mihi, quoniam sequebar bonitatem. \EVERSE
\VERSE Ne derelinquas me,  Domine Deus meus ; ne discesseris a me. \EVERSE
\VERSE Intende in adjutorium meum, Domine Deus salutis meæ.

}
\newcommand{\psalmxxxviii}{
\VERSE In finem,  ipsi Idithun. Canticum David.\VERSE Dixi : Custodiam vias meas :ut non delinquam in lingua mea.
Posui ori meo custodiam, 
cum consisteret peccator adversum me. \EVERSE
\VERSE Obmutui,  et humiliatus sum,  et silui a bonis ; et dolor meus renovatus est. \EVERSE
\VERSE Concaluit cor meum intra me ; et in meditatione mea exardescet ignis. \EVERSE
\VERSE Locutus sum in lingua mea :Notum fac mihi,  Domine,  finem meum, 
et numerum dierum meorum quis est, 
ut sciam quid desit mihi. \EVERSE
\VERSE Ecce mensurabiles posuisti dies meos, et substantia mea tamquam nihilum ante te.
Verumtamen universa vanitas,  omnis homo vivens. \EVERSE
\VERSE Verumtamen in imagine pertransit homo ; sed et frustra conturbatur :
thesaurizat,  et ignorat cui congregabit ea. \EVERSE
\VERSE Et nunc quæ est exspectatio mea : nonne Dominus ?et substantia mea apud te est. \EVERSE
\VERSE Ab omnibus iniquitatibus meis erue me :opprobrium insipienti dedisti me. \EVERSE
\VERSE Obmutui,  et non aperui os meum, quoniam tu fecisti ;  \EVERSE
\VERSE amove a me plagas tuas.\VERSE A fortitudine manus tuæ ego defeci in increpationibus :propter iniquitatem corripuisti hominem.
Et tabescere fecisti sicut araneam animam ejus :
verumtamen vane conturbatur omnis homo. \EVERSE
\VERSE Exaudi orationem meam,  Domine,  et deprecationem meam ; auribus percipe lacrimas meas.
Ne sileas,  quoniam advena ego sum apud te, 
et peregrinus sicut omnes patres mei. \EVERSE
\VERSE Remitte mihi,  ut refrigererpriusquam abeam et amplius non ero.

}
\newcommand{\psalmxxxix}{
\VERSE In finem. Psalmus ipsi David.\VERSE Exspectans exspectavi Dominum, et intendit mihi. \EVERSE
\VERSE Et exaudivit preces meas, et eduxit me de lacu miseriæ et de luto fæcis.
Et statuit super petram pedes meos, 
et direxit gressus meos. \EVERSE
\VERSE Et immisit in os meum canticum novum, carmen Deo nostro.
Videbunt multi,  et timebunt, 
et sperabunt in Domino. \EVERSE
\VERSE Beatus vir cujus est nomen Domini spes ejus, et non respexit in vanitates et insanias falsas. \EVERSE
\VERSE Multa fecisti tu,  Domine Deus meus,  mirabilia tua ; et cogitationibus tuis non est qui similis sit tibi.
Annuntiavi et locutus sum :
multiplicati sunt super numerum. \EVERSE
\VERSE Sacrificium et oblationem noluisti ; aures autem perfecisti mihi.
Holocaustum et pro peccato non postulasti ;  \EVERSE
\VERSE tunc dixi : Ecce venio.In capite libri scriptum est de me,  \EVERSE
\VERSE ut facerem voluntatem tuam.Deus meus,  volui, 
et legem tuam in medio cordis mei. \EVERSE
\VERSE Annuntiavi justitiam tuam in ecclesia magna ; ecce labia mea non prohibebo : Domine,  tu scisti. \EVERSE
\VERSE Justitiam tuam non abscondi in corde meo ; veritatem tuam et salutare tuum dixi ; 
non abscondi misericordiam tuam et veritatem tuam a concilio multo. \EVERSE
\VERSE Tu autem,  Domine,  ne longe facias miserationes tuas a me ; misericordia tua et veritas tua semper susceperunt me. \EVERSE
\VERSE Quoniam circumdederunt me mala quorum non est numerus ; comprehenderunt me iniquitates meæ,  et non potui ut viderem.
Multiplicatæ sunt super capillos capitis mei, 
et cor meum dereliquit me. \EVERSE
\VERSE Complaceat tibi,  Domine,  ut eruas me ; Domine,  ad adjuvandum me respice. \EVERSE
\VERSE Confundantur et revereantur simul, qui quærunt animam meam ut auferant eam ; 
convertantur retrorsum et revereantur, 
qui volunt mihi mala. \EVERSE
\VERSE Ferant confestim confusionem suam, qui dicunt mihi : Euge,  euge ! \EVERSE
\VERSE Exsultent et lætentur super te omnes quærentes te ; et dicant semper : Magnificetur Dominus,  qui diligunt salutare tuum. \EVERSE
\VERSE Ego autem mendicus sum et pauper ; Dominus sollicitus est mei.
Adjutor meus et protector meus tu es ; 
Deus meus,  ne tardaveris.

}
\newcommand{\psalmxl}{
\VERSE In finem. Psalmus ipsi David.\VERSE Beatus qui intelligit super egenum et pauperem :in die mala liberabit eum Dominus. \EVERSE
\VERSE Dominus conservet eum,  et vivificet eum, et beatum faciat eum in terra, 
et non tradat eum in animam inimicorum ejus. \EVERSE
\VERSE Dominus opem ferat illi super lectum doloris ejus ; universum stratum ejus versasti in infirmitate ejus. \EVERSE
\VERSE Ego dixi : Domine,  miserere mei ; sana animam meam,  quia peccavi tibi. \EVERSE
\VERSE Inimici mei dixerunt mala mihi :Quando morietur,  et peribit nomen ejus ? \EVERSE
\VERSE Et si ingrediebatur ut videret,  vana loquebatur ; cor ejus congregavit iniquitatem sibi.
Egrediebatur foras et loquebatur. \EVERSE
\VERSE In idipsum adversum me susurrabant omnes inimici mei ; adversum me cogitabant mala mihi. \EVERSE
\VERSE Verbum iniquum constituerunt adversum me :Numquid qui dormit non adjiciet ut resurgat ? \EVERSE
\VERSE Etenim homo pacis meæ in quo speravi, qui edebat panes meos, 
magnificavit super me supplantationem. \EVERSE
\VERSE Tu autem,  Domine,  miserere mei, et resuscita me ;  et retribuam eis. \EVERSE
\VERSE In hoc cognovi quoniam voluisti me, quoniam non gaudebit inimicus meus super me. \EVERSE
\VERSE Me autem propter innocentiam suscepisti ; et confirmasti me in conspectu tuo in æternum. \EVERSE
\VERSE Benedictus Dominus Deus Israëla sæculo et usque in sæculum. Fiat,  fiat.

}
\newcommand{\psalmxli}{
\VERSE In finem. Intellectus filiis Core.\VERSE Quemadmodum desiderat cervus ad fontes aquarum, ita desiderat anima mea ad te,  Deus. \EVERSE
\VERSE Sitivit anima mea ad Deum fortem,  vivum ; quando veniam,  et apparebo ante faciem Dei ? \EVERSE
\VERSE Fuerunt mihi lacrimæ meæ panes die ac nocte, dum dicitur mihi quotidie : Ubi est Deus tuus ? \EVERSE
\VERSE Hæc recordatus sum,  et effudi in me animam meam, quoniam transibo in locum tabernaculi admirabilis,  usque ad domum Dei, 
in voce exsultationis et confessionis,  sonus epulantis. \EVERSE
\VERSE Quare tristis es,  anima mea ?et quare conturbas me ?
Spera in Deo,  quoniam adhuc confitebor illi, 
salutare vultus mei,  \EVERSE
\VERSE et Deus meus.Ad meipsum anima mea conturbata est :
propterea memor ero tui de terra Jordanis et Hermoniim a monte modico. \EVERSE
\VERSE Abyssus abyssum invocat,  in voce cataractarum tuarum ; omnia excelsa tua,  et fluctus tui super me transierunt. \EVERSE
\VERSE In die mandavit Dominus misericordiam suam, et nocte canticum ejus ; 
apud me oratio Deo vitæ meæ. \EVERSE
\VERSE Dicam Deo : Susceptor meus es ; quare oblitus es mei ?
et quare contristatus incedo,  dum affligit me inimicus ? \EVERSE
\VERSE Dum confringuntur ossa mea, exprobraverunt mihi qui tribulant me inimici mei, 
dum dicunt mihi per singulos dies : Ubi est Deus tuus ? \EVERSE
\VERSE Quare tristis es,  anima mea ?et quare conturbas me ?
Spera in Deo,  quoniam adhuc confitebor illi, 
salutare vultus mei,  et Deus meus.

}
\newcommand{\psalmxlii}{
\VERSE Psalmus David.Judica me,  Deus,  et discerne causam meam de gente non sancta :
ab homine iniquo et doloso erue me. \EVERSE
\VERSE Quia tu es,  Deus,  fortitudo mea : quare me repulisti ?et quare tristis incedo,  dum affligit me inimicus ? \EVERSE
\VERSE Emitte lucem tuam et veritatem tuam :ipsa me deduxerunt,  et adduxerunt
in montem sanctum tuum,  et in tabernacula tua. \EVERSE
\VERSE Et introibo ad altare Dei, ad Deum qui lætificat juventutem meam.
Confitebor tibi in cithara,  Deus,  Deus meus. \EVERSE
\VERSE Quare tristis es,  anima mea ?et quare conturbas me ?
Spera in Deo,  quoniam adhuc confitebor illi, 
salutare vultus mei,  et Deus meus.

}
\newcommand{\psalmxliii}{
\VERSE In finem. Filiis Core ad intellectum.\VERSE Deus,  auribus nostris audivimus, patres nostri annuntiaverunt nobis, 
opus quod operatus es in diebus eorum, 
et in diebus antiquis. \EVERSE
\VERSE Manus tua gentes disperdidit,  et plantasti eos ; afflixisti populos,  et expulisti eos. \EVERSE
\VERSE Nec enim in gladio suo possederunt terram, et brachium eorum non salvavit eos :
sed dextera tua et brachium tuum, 
et illuminatio vultus tui,  quoniam complacuisti in eis. \EVERSE
\VERSE Tu es ipse rex meus et Deus meus, qui mandas salutes Jacob. \EVERSE
\VERSE In te inimicos nostros ventilabimus cornu, et in nomine tuo spernemus insurgentes in nobis. \EVERSE
\VERSE Non enim in arcu meo sperabo, et gladius meus non salvabit me : \EVERSE
\VERSE salvasti enim nos de affligentibus nos, et odientes nos confudisti. \EVERSE
\VERSE In Deo laudabimur tota die, et in nomine tuo confitebimur in sæculum. \EVERSE
\VERSE Nunc autem repulisti et confudisti nos, et non egredieris,  Deus,  in virtutibus nostris. \EVERSE
\VERSE Avertisti nos retrorsum post inimicos nostros, et qui oderunt nos diripiebant sibi. \EVERSE
\VERSE Dedisti nos tamquam oves escarum, et in gentibus dispersisti nos. \EVERSE
\VERSE Vendidisti populum tuum sine pretio, et non fuit multitudo in commutationibus eorum. \EVERSE
\VERSE Posuisti nos opprobrium vicinis nostris ; subsannationem et derisum his qui sunt in circuitu nostro. \EVERSE
\VERSE Posuisti nos in similitudinem gentibus ; commotionem capitis in populis. \EVERSE
\VERSE Tota die verecundia mea contra me est, et confusio faciei meæ cooperuit me : \EVERSE
\VERSE a voce exprobrantis et obloquentis, a facie inimici et persequentis. \EVERSE
\VERSE Hæc omnia venerunt super nos ;  nec obliti sumus te, et inique non egimus in testamento tuo. \EVERSE
\VERSE Et non recessit retro cor nostrum ; et declinasti semitas nostras a via tua : \EVERSE
\VERSE quoniam humiliasti nos in loco afflictionis, et cooperuit nos umbra mortis. \EVERSE
\VERSE Si obliti sumus nomen Dei nostri, et si expandimus manus nostras ad deum alienum,  \EVERSE
\VERSE nonne Deus requiret ista ?ipse enim novit abscondita cordis.
Quoniam propter te mortificamur tota die ; 
æstimati sumus sicut oves occisionis. \EVERSE
\VERSE Exsurge ;  quare obdormis,  Domine ?exsurge,  et ne repellas in finem. \EVERSE
\VERSE Quare faciem tuam avertis ?oblivisceris inopiæ nostræ et tribulationis nostræ ? \EVERSE
\VERSE Quoniam humiliata est in pulvere anima nostra ; conglutinatus est in terra venter noster. \EVERSE
\VERSE Exsurge,  Domine,  adjuva nos, et redime nos propter nomen tuum.

}
\newcommand{\psalmxliv}{
\VERSE In finem,  pro iis qui commutabuntur. Filiis Core,  ad intellectum. Canticum pro dilecto.\VERSE Eructavit cor meum verbum bonum :dico ego opera mea regi.
Lingua mea calamus scribæ
velociter scribentis. \EVERSE
\VERSE Speciosus forma præ filiis hominum, diffusa est gratia in labiis tuis :
propterea benedixit te Deus in æternum. \EVERSE
\VERSE Accingere gladio tuo super femur tuum,  potentissime.\VERSE Specie tua et pulchritudine tuaintende,  prospere procede,  et regna, 
propter veritatem,  et mansuetudinem,  et justitiam ; 
et deducet te mirabiliter dextera tua. \EVERSE
\VERSE Sagittæ tuæ acutæ :populi sub te cadent, 
in corda inimicorum regis. \EVERSE
\VERSE Sedes tua,  Deus,  in sæculum sæculi ; virga directionis virga regni tui. \EVERSE
\VERSE Dilexisti justitiam,  et odisti iniquitatem ; propterea unxit te Deus,  Deus tuus, 
oleo lætitiæ,  præ consortibus tuis. \EVERSE
\VERSE Myrrha,  et gutta,  et casia a vestimentis tuis, a domibus eburneis ;  ex quibus delectaverunt te \EVERSE
\VERSE filiæ regum in honore tuo.Astitit regina a dextris tuis
in vestitu deaurato,  circumdata varietate. \EVERSE
\VERSE Audi,  filia,  et vide,  et inclina aurem tuam ; et obliviscere populum tuum,  et domum patris tui. \EVERSE
\VERSE Et concupiscet rex decorem tuum, quoniam ipse est Dominus Deus tuus,  et adorabunt eum. \EVERSE
\VERSE Et filiæ Tyri in muneribus vultum tuum deprecabuntur ; omnes divites plebis. \EVERSE
\VERSE Omnis gloria ejus filiæ regis ab intus, in fimbriis aureis,  \EVERSE
\VERSE circumamicta varietatibus.Adducentur regi virgines post eam ; 
proximæ ejus afferentur tibi. \EVERSE
\VERSE Afferentur in lætitia et exsultatione ; adducentur in templum regis. \EVERSE
\VERSE Pro patribus tuis nati sunt tibi filii ; constitues eos principes super omnem terram. \EVERSE
\VERSE Memores erunt nominis tui in omni generatione et generationem :propterea populi confitebuntur tibi in æternum,  et in sæculum sæculi.

}
\newcommand{\psalmxlv}{
\VERSE In finem,  filiis Core,  pro arcanis. Psalmus.\VERSE Deus noster refugium et virtus ; adjutor in tribulationibus quæ invenerunt nos nimis. \EVERSE
\VERSE Propterea non timebimus dum turbabitur terra, et transferentur montes in cor maris. \EVERSE
\VERSE Sonuerunt,  et turbatæ sunt aquæ eorum ; conturbati sunt montes in fortitudine ejus. \EVERSE
\VERSE Fluminis impetus lætificat civitatem Dei :sanctificavit tabernaculum suum Altissimus. \EVERSE
\VERSE Deus in medio ejus,  non commovebitur ; adjuvabit eam Deus mane diluculo. \EVERSE
\VERSE Conturbatæ sunt gentes,  et inclinata sunt regna :dedit vocem suam,  mota est terra. \EVERSE
\VERSE Dominus virtutum nobiscum ; susceptor noster Deus Jacob. \EVERSE
\VERSE Venite,  et videte opera Domini, quæ posuit prodigia super terram,  \EVERSE
\VERSE auferens bella usque ad finem terræ.Arcum conteret,  et confringet arma, 
et scuta comburet igni. \EVERSE
\VERSE Vacate,  et videte quoniam ego sum Deus ; exaltabor in gentibus,  et exaltabor in terra. \EVERSE
\VERSE Dominus virtutum nobiscum ; susceptor noster Deus Jacob.

}
\newcommand{\psalmxlvi}{
\VERSE In finem,  pro filiis Core. Psalmus.\VERSE Omnes gentes,  plaudite manibus ; jubilate Deo in voce exsultationis : \EVERSE
\VERSE quoniam Dominus excelsus,  terribilis, rex magnus super omnem terram. \EVERSE
\VERSE Subjecit populos nobis, et gentes sub pedibus nostris. \EVERSE
\VERSE Elegit nobis hæreditatem suam ; speciem Jacob quam dilexit. \EVERSE
\VERSE Ascendit Deus in jubilo, et Dominus in voce tubæ. \EVERSE
\VERSE Psallite Deo nostro,  psallite ; psallite regi nostro,  psallite : \EVERSE
\VERSE quoniam rex omnis terræ Deus, psallite sapienter. \EVERSE
\VERSE Regnabit Deus super gentes ; Deus sedet super sedem sanctam suam. \EVERSE
\VERSE Principes populorum congregati sunt cum Deo Abraham, quoniam dii fortes terræ vehementer elevati sunt.

}
\newcommand{\psalmxlvii}{
\VERSE Psalmus cantici. Filiis Core,  secunda sabbati.\VERSE Magnus Dominus et laudabilis nimis, in civitate Dei nostri,  in monte sancto ejus. \EVERSE
\VERSE Fundatur exsultatione universæ terræ mons Sion ; latera aquilonis,  civitas regis magni. \EVERSE
\VERSE Deus in domibus ejus cognosceturcum suscipiet eam. \EVERSE
\VERSE Quoniam ecce reges terræ congregati sunt ; convenerunt in unum. \EVERSE
\VERSE Ipsi videntes,  sic admirati sunt, conturbati sunt,  commoti sunt. \EVERSE
\VERSE Tremor apprehendit eos ; ibi dolores ut parturientis : \EVERSE
\VERSE in spiritu vehementi conteres naves Tharsis.\VERSE Sicut audivimus,  sic vidimus, in civitate Domini virtutum,  in civitate Dei nostri :
Deus fundavit eam in æternum. \EVERSE
\VERSE Suscepimus,  Deus,  misericordiam tuamin medio templi tui. \EVERSE
\VERSE Secundum nomen tuum,  Deus, sic et laus tua in fines terræ ; 
justitia plena est dextera tua. \EVERSE
\VERSE Lætetur mons Sion, et exsultent filiæ Judæ, 
propter judicia tua,  Domine. \EVERSE
\VERSE Circumdate Sion,  et complectimini eam ; narrate in turribus ejus. \EVERSE
\VERSE Ponite corda vestra in virtute ejus, et distribuite domos ejus,  ut enarretis in progenie altera. \EVERSE
\VERSE Quoniam hic est Deus, Deus noster in æternum,  et in sæculum sæculi :
ipse reget nos in sæcula.

}
\newcommand{\psalmxlviii}{
\VERSE In finem,  filiis Core. Psalmus.\VERSE Audite hæc,  omnes gentes ; auribus percipite,  omnes qui habitatis orbem : \EVERSE
\VERSE quique terrigenæ et filii hominum, simul in unum dives et pauper. \EVERSE
\VERSE Os meum loquetur sapientiam, et meditatio cordis mei prudentiam. \EVERSE
\VERSE Inclinabo in parabolam aurem meam ; aperiam in psalterio propositionem meam. \EVERSE
\VERSE Cur timebo in die mala ?iniquitas calcanei mei circumdabit me. \EVERSE
\VERSE Qui confidunt in virtute sua, et in multitudine divitiarum suarum,  gloriantur. \EVERSE
\VERSE Frater non redimit,  redimet homo :non dabit Deo placationem suam,  \EVERSE
\VERSE et pretium redemptionis animæ suæ.Et laborabit in æternum ;  \EVERSE
\VERSE et vivet adhuc in finem.\VERSE Non videbit interitum, cum viderit sapientes morientes :
simul insipiens et stultus peribunt.
Et relinquent alienis divitias suas,  \EVERSE
\VERSE et sepulchra eorum domus illorum in æternum ; tabernacula eorum in progenie et progenie :
vocaverunt nomina sua in terris suis. \EVERSE
\VERSE Et homo,  cum in honore esset,  non intellexit.Comparatus est jumentis insipientibus, 
et similis factus est illis. \EVERSE
\VERSE Hæc via illorum scandalum ipsis ; et postea in ore suo complacebunt. \EVERSE
\VERSE Sicut oves in inferno positi sunt :mors depascet eos.
Et dominabuntur eorum justi in matutino ; 
et auxilium eorum veterascet in inferno a gloria eorum. \EVERSE
\VERSE Verumtamen Deus redimet animam meam de manu inferi, cum acceperit me. \EVERSE
\VERSE Ne timueris cum dives factus fuerit homo, et cum multiplicata fuerit gloria domus ejus : \EVERSE
\VERSE quoniam,  cum interierit,  non sumet omnia, neque descendet cum eo gloria ejus. \EVERSE
\VERSE Quia anima ejus in vita ipsius benedicetur ; confitebitur tibi cum benefeceris ei. \EVERSE
\VERSE Introibit usque in progenies patrum suorum ; et usque in æternum non videbit lumen. \EVERSE
\VERSE Homo,  cum in honore esset,  non intellexit.Comparatus est jumentis insipientibus, 
et similis factus est illis.

}
\newcommand{\psalmxlix}{
\VERSE Psalmus Asaph.Deus deorum Dominus locutus est, 
et vocavit terram a solis ortu usque ad occasum. \EVERSE
\VERSE Ex Sion species decoris ejus :\VERSE Deus manifeste veniet ; Deus noster,  et non silebit.
Ignis in conspectu ejus exardescet ; 
et in circuitu ejus tempestas valida. \EVERSE
\VERSE Advocabit cælum desursum,  et terram, discernere populum suum. \EVERSE
\VERSE Congregate illi sanctos ejus, qui ordinant testamentum ejus super sacrificia. \EVERSE
\VERSE Et annuntiabunt cæli justitiam ejus, quoniam Deus judex est. \EVERSE
\VERSE Audi,  populus meus,  et loquar ; Israël,  et testificabor tibi :
Deus,  Deus tuus ego sum. \EVERSE
\VERSE Non in sacrificiis tuis arguam te ; holocausta autem tua in conspectu meo sunt semper. \EVERSE
\VERSE Non accipiam de domo tua vitulos, neque de gregibus tuis hircos : \EVERSE
\VERSE quoniam meæ sunt omnes feræ silvarum, jumenta in montibus,  et boves. \EVERSE
\VERSE Cognovi omnia volatilia cæli, et pulchritudo agri mecum est. \EVERSE
\VERSE Si esuriero,  non dicam tibi :meus est enim orbis terræ et plenitudo ejus. \EVERSE
\VERSE Numquid manducabo carnes taurorum ?aut sanguinem hircorum potabo ? \EVERSE
\VERSE Immola Deo sacrificium laudis, et redde Altissimo vota tua. \EVERSE
\VERSE Et invoca me in die tribulationis :eruam te,  et honorificabis me. \EVERSE
\VERSE Peccatori autem dixit Deus : Quare tu enarras justitias meas ?et assumis testamentum meum per os tuum ? \EVERSE
\VERSE Tu vero odisti disciplinam, et projecisti sermones meos retrorsum. \EVERSE
\VERSE Si videbas furem,  currebas cum eo ; et cum adulteris portionem tuam ponebas. \EVERSE
\VERSE Os tuum abundavit malitia, et lingua tua concinnabat dolos. \EVERSE
\VERSE Sedens adversus fratrem tuum loquebaris, et adversus filium matris tuæ ponebas scandalum. \EVERSE
\VERSE Hæc fecisti,  et tacui.Existimasti inique quod ero tui similis :
arguam te,  et statuam contra faciem tuam. \EVERSE
\VERSE Intelligite hæc,  qui obliviscimini Deum, nequando rapiat,  et non sit qui eripiat. \EVERSE
\VERSE Sacrificium laudis honorificabit me, et illic iter quo ostendam illi salutare Dei.

}
\newcommand{\psalml}{
\VERSE In finem. Psalmus David, \VERSE cum venit ad eum Nathan propheta,  quando intravit ad Bethsabee.\VERSE Miserere mei,  Deus,  secundum magnam misericordiam tuam ; et secundum multitudinem miserationum tuarum,  dele iniquitatem meam. \EVERSE
\VERSE Amplius lava me ab iniquitate mea, et a peccato meo munda me. \EVERSE
\VERSE Quoniam iniquitatem meam ego cognosco, et peccatum meum contra me est semper. \EVERSE
\VERSE Tibi soli peccavi,  et malum coram te feci ; ut justificeris in sermonibus tuis, 
et vincas cum judicaris. \EVERSE
\VERSE Ecce enim in iniquitatibus conceptus sum, et in peccatis concepit me mater mea. \EVERSE
\VERSE Ecce enim veritatem dilexisti ; incerta et occulta sapientiæ tuæ manifestasti mihi. \EVERSE
\VERSE Asperges me hyssopo,  et mundabor ; lavabis me,  et super nivem dealbabor. \EVERSE
\VERSE Auditui meo dabis gaudium et lætitiam, et exsultabunt ossa humiliata. \EVERSE
\VERSE Averte faciem tuam a peccatis meis, et omnes iniquitates meas dele. \EVERSE
\VERSE Cor mundum crea in me,  Deus, et spiritum rectum innova in visceribus meis. \EVERSE
\VERSE Ne projicias me a facie tua, et spiritum sanctum tuum ne auferas a me. \EVERSE
\VERSE Redde mihi lætitiam salutaris tui, et spiritu principali confirma me. \EVERSE
\VERSE Docebo iniquos vias tuas, et impii ad te convertentur. \EVERSE
\VERSE Libera me de sanguinibus,  Deus,  Deus salutis meæ, et exsultabit lingua mea justitiam tuam. \EVERSE
\VERSE Domine,  labia mea aperies, et os meum annuntiabit laudem tuam. \EVERSE
\VERSE Quoniam si voluisses sacrificium,  dedissem utique ; holocaustis non delectaberis. \EVERSE
\VERSE Sacrificium Deo spiritus contribulatus ; cor contritum et humiliatum,  Deus,  non despicies. \EVERSE
\VERSE Benigne fac,  Domine,  in bona voluntate tua Sion, ut ædificentur muri Jerusalem. \EVERSE
\VERSE Tunc acceptabis sacrificium justitiæ,  oblationes et holocausta ; tunc imponent super altare tuum vitulos.

}
\newcommand{\psalmli}{
\VERSE In finem. Intellectus David, \VERSE cum venit Doëg Idumæus,  et nuntiavit Sauli : Venit David in domum Achimelech.\VERSE Quid gloriaris in malitia, qui potens es in iniquitate ? \EVERSE
\VERSE Tota die injustitiam cogitavit lingua tua ; sicut novacula acuta fecisti dolum. \EVERSE
\VERSE Dilexisti malitiam super benignitatem ; iniquitatem magis quam loqui æquitatem. \EVERSE
\VERSE Dilexisti omnia verba præcipitationis ; lingua dolosa. \EVERSE
\VERSE Propterea Deus destruet te in finem ; evellet te,  et emigrabit te de tabernaculo tuo, 
et radicem tuam de terra viventium. \EVERSE
\VERSE Videbunt justi,  et timebunt ; et super eum ridebunt,  et dicent : \EVERSE
\VERSE Ecce homo qui non posuit Deum adjutorem suum ; sed speravit in multitudine divitiarum suarum, 
et prævaluit in vanitate sua. \EVERSE
\VERSE Ego autem,  sicut oliva fructifera in domo Dei ; speravi in misericordia Dei,  in æternum et in sæculum sæculi. \EVERSE
\VERSE Confitebor tibi in sæculum,  quia fecisti ; et exspectabo nomen tuum, 
quoniam bonum est in conspectu sanctorum tuorum.

}
\newcommand{\psalmlii}{
\VERSE In finem,  pro Maëleth intelligentiæ David.Dixit insipiens in corde suo : Non est Deus. \EVERSE
\VERSE Corrupti sunt,  et abominabiles facti sunt in iniquitatibus ; non est qui faciat bonum. \EVERSE
\VERSE Deus de cælo prospexit super filios hominum, ut videat si est intelligens,  aut requirens Deum. \EVERSE
\VERSE Omnes declinaverunt ;  simul inutiles facti sunt :non est qui faciat bonum,  non est usque ad unum. \EVERSE
\VERSE Nonne scient omnes qui operantur iniquitatem, qui devorant plebem meam ut cibum panis ? \EVERSE
\VERSE Deum non invocaverunt ; illic trepidaverunt timore,  ubi non erat timor.
Quoniam Deus dissipavit ossa eorum qui hominibus placent :
confusi sunt,  quoniam Deus sprevit eos. \EVERSE
\VERSE Quis dabit ex Sion salutare Israël ?cum converterit Deus captivitatem plebis suæ, 
exsultabit Jacob,  et lætabitur Israël.

}
\newcommand{\psalmliii}{
\VERSE In finem,  in carminibus. Intellectus David, \VERSE cum venissent Ziphæi,  et dixissent ad Saul : Nonne David absconditus est apud nos ?\VERSE Deus,  in nomine tuo salvum me fac, et in virtute tua judica me. \EVERSE
\VERSE Deus,  exaudi orationem meam ; auribus percipe verba oris mei. \EVERSE
\VERSE Quoniam alieni insurrexerunt adversum me, et fortes quæsierunt animam meam, 
et non proposuerunt Deum ante conspectum suum. \EVERSE
\VERSE Ecce enim Deus adjuvat me, et Dominus susceptor est animæ meæ. \EVERSE
\VERSE Averte mala inimicis meis ; et in veritate tua disperde illos. \EVERSE
\VERSE Voluntarie sacrificabo tibi, et confitebor nomini tuo,  Domine,  quoniam bonum est. \EVERSE
\VERSE Quoniam ex omni tribulatione eripuisti me, et super inimicos meos despexit oculus meus.

}
\newcommand{\psalmliv}{
\VERSE In finem,  in carminibus. Intellectus David.\VERSE Exaudi,  Deus,  orationem meam, et ne despexeris deprecationem meam : \EVERSE
\VERSE intende mihi,  et exaudi me.Contristatus sum in exercitatione mea, 
et conturbatus sum \EVERSE
\VERSE a voce inimici,  et a tribulatione peccatoris.Quoniam declinaverunt in me iniquitates, 
et in ira molesti erant mihi. \EVERSE
\VERSE Cor meum conturbatum est in me, et formido mortis cecidit super me. \EVERSE
\VERSE Timor et tremor venerunt super me, et contexerunt me tenebræ. \EVERSE
\VERSE Et dixi : Quis dabit mihi pennas sicut columbæ, et volabo,  et requiescam ? \EVERSE
\VERSE Ecce elongavi fugiens, et mansi in solitudine. \EVERSE
\VERSE Exspectabam eum qui salvum me fecita pusillanimitate spiritus,  et tempestate. \EVERSE
\VERSE Præcipita,  Domine ;  divide linguas eorum :quoniam vidi iniquitatem et contradictionem in civitate. \EVERSE
\VERSE Die ac nocte circumdabit eam super muros ejus iniquitas ; et labor in medio ejus,  \EVERSE
\VERSE et injustitia :et non defecit de plateis ejus usura et dolus. \EVERSE
\VERSE Quoniam si inimicus meus maledixisset mihi, sustinuissem utique.
Et si is qui oderat me super me magna locutus fuisset, 
abscondissem me forsitan ab eo. \EVERSE
\VERSE Tu vero homo unanimis, dux meus,  et notus meus : \EVERSE
\VERSE qui simul mecum dulces capiebas cibos ; in domo Dei ambulavimus cum consensu. \EVERSE
\VERSE Veniat mors super illos, et descendant in infernum viventes :
quoniam nequitiæ in habitaculis eorum,  in medio eorum. \EVERSE
\VERSE Ego autem ad Deum clamavi, et Dominus salvabit me. \EVERSE
\VERSE Vespere,  et mane,  et meridie,  narrabo,  et annuntiabo ; et exaudiet vocem meam. \EVERSE
\VERSE Redimet in pace animam meam ab his qui appropinquant mihi :quoniam inter multos erant mecum. \EVERSE
\VERSE Exaudiet Deus,  et humiliabit illos, qui est ante sæcula.
Non enim est illis commutatio, 
et non timuerunt Deum. \EVERSE
\VERSE Extendit manum suam in retribuendo ; contaminaverunt testamentum ejus : \EVERSE
\VERSE divisi sunt ab ira vultus ejus, et appropinquavit cor illius.
Molliti sunt sermones ejus super oleum ; 
et ipsi sunt jacula. \EVERSE
\VERSE Jacta super Dominum curam tuam,  et ipse te enutriet ; non dabit in æternum fluctuationem justo. \EVERSE
\VERSE Tu vero,  Deus,  deduces eos in puteum interitus.Viri sanguinum et dolosi non dimidiabunt dies suos ; 
ego autem sperabo in te,  Domine.

}
\newcommand{\psalmlv}{
\VERSE In finem,  pro populo qui a sanctis longe factus est. David in tituli inscriptionem,  cum tenuerunt eum Allophyli in Geth.\VERSE Miserere mei,  Deus,  quoniam conculcavit me homo ; tota die impugnans,  tribulavit me. \EVERSE
\VERSE Conculcaverunt me inimici mei tota die, quoniam multi bellantes adversum me. \EVERSE
\VERSE Ab altitudine diei timebo :ego vero in te sperabo. \EVERSE
\VERSE In Deo laudabo sermones meos ; in Deo speravi :
non timebo quid faciat mihi caro. \EVERSE
\VERSE Tota die verba mea execrabantur ; adversum me omnes cogitationes eorum in malum. \EVERSE
\VERSE Inhabitabunt,  et abscondent ; ipsi calcaneum meum observabunt.
Sicut sustinuerunt animam meam,  \EVERSE
\VERSE pro nihilo salvos facies illos ; in ira populos confringes. \EVERSE
\VERSE Deus,  vitam meam annuntiavi tibi ; posuisti lacrimas meas in conspectu tuo, 
sicut et in promissione tua : \EVERSE
\VERSE tunc convertentur inimici mei retrorsum.In quacumque die invocavero te, 
ecce cognovi quoniam Deus meus es. \EVERSE
\VERSE In Deo laudabo verbum ; in Domino laudabo sermonem.
In Deo speravi :
non timebo quid faciat mihi homo. \EVERSE
\VERSE In me sunt,  Deus,  vota tua, quæ reddam,  laudationes tibi : \EVERSE
\VERSE quoniam eripuisti animam meam de morte, et pedes meos de lapsu, 
ut placeam coram Deo in lumine viventium.

}
\newcommand{\psalmlvi}{
\VERSE In finem,  ne disperdas. David in tituli inscriptionem,  cum fugeret a facie Saul in speluncam.\VERSE Miserere mei,  Deus,  miserere mei, quoniam in te confidit anima mea.
Et in umbra alarum tuarum sperabo, 
donec transeat iniquitas. \EVERSE
\VERSE Clamabo ad Deum altissimum, Deum qui benefecit mihi. \EVERSE
\VERSE Misit de cælo,  et liberavit me ; dedit in opprobrium conculcantes me.
Misit Deus misericordiam suam et veritatem suam,  \EVERSE
\VERSE et eripuit animam meam de medio catulorum leonum.Dormivi conturbatus.
Filii hominum dentes eorum arma et sagittæ, 
et lingua eorum gladius acutus. \EVERSE
\VERSE Exaltare super cælos,  Deus, et in omnem terram gloria tua. \EVERSE
\VERSE Laqueum paraverunt pedibus meis, et incurvaverunt animam meam.
Foderunt ante faciem meam foveam, 
et inciderunt in eam. 8 Paratum cor meum,  Deus,  paratum cor meum ; 
cantabo,  et psalmum dicam. \EVERSE
\VERSE Exsurge,  gloria mea ; exsurge,  psalterium et cithara :
exsurgam diluculo. \EVERSE
\VERSE Confitebor tibi in populis,  Domine, et psalmum dicam tibi in gentibus : \EVERSE
\VERSE quoniam magnificata est usque ad cælos misericordia tua, et usque ad nubes veritas tua. \EVERSE
\VERSE Exaltare super cælos,  Deus, et super omnem terram gloria tua.

}
\newcommand{\psalmlvii}{
\VERSE In finem,  ne disperdas. David in tituli inscriptionem.\VERSE Si vere utique justitiam loquimini, recta judicate,  filii hominum. \EVERSE
\VERSE Etenim in corde iniquitates operamini ; in terra injustitias manus vestræ concinnant. \EVERSE
\VERSE Alienati sunt peccatores a vulva ; erraverunt ab utero :
locuti sunt falsa. \EVERSE
\VERSE Furor illis secundum similitudinem serpentis, sicut aspidis surdæ et obturantis aures suas,  \EVERSE
\VERSE quæ non exaudiet vocem incantantium, et venefici incantantis sapienter. \EVERSE
\VERSE Deus conteret dentes eorum in ore ipsorum ; molas leonum confringet Dominus. \EVERSE
\VERSE Ad nihilum devenient tamquam aqua decurrens ; intendit arcum suum donec infirmentur. \EVERSE
\VERSE Sicut cera quæ fluit auferentur ; supercecidit ignis,  et non viderunt solem. \EVERSE
\VERSE Priusquam intelligerent spinæ vestræ rhamnum, sicut viventes sic in ira absorbet eos. \EVERSE
\VERSE Lætabitur justus cum viderit vindictam ; manus suas lavabit in sanguine peccatoris. \EVERSE
\VERSE Et dicet homo : Si utique est fructus justo, utique est Deus judicans eos in terra.

}
\newcommand{\psalmlviii}{
\VERSE In finem,  ne disperdas. David in tituli inscriptionem,  quando misit Saul et custodivit domum ejus ut eum interficeret.\VERSE Eripe me de inimicis meis,  Deus meus, et ab insurgentibus in me libera me. \EVERSE
\VERSE Eripe me de operantibus iniquitatem, et de viris sanguinum salva me. \EVERSE
\VERSE Quia ecce ceperunt animam meam ; irruerunt in me fortes. \EVERSE
\VERSE Neque iniquitas mea,  neque peccatum meum,  Domine ; sine iniquitate cucurri,  et direxi. \EVERSE
\VERSE Exsurge in occursum meum,  et vide :et tu,  Domine Deus virtutum,  Deus Israël, 
intende ad visitandas omnes gentes :
non miserearis omnibus qui operantur iniquitatem. \EVERSE
\VERSE Convertentur ad vesperam,  et famem patientur ut canes :et circuibunt civitatem. \EVERSE
\VERSE Ecce loquentur in ore suo, et gladius in labiis eorum : quoniam quis audivit ? \EVERSE
\VERSE Et tu,  Domine,  deridebis eos ; ad nihilum deduces omnes gentes. \EVERSE
\VERSE Fortitudinem meam ad te custodiam, quia,  Deus,  susceptor meus es : \EVERSE
\VERSE Deus meus misericordia ejus præveniet me.\VERSE Deus ostendet mihi super inimicos meos :ne occidas eos,  nequando obliviscantur populi mei.
Disperge illos in virtute tua, 
et depone eos,  protector meus,  Domine : \EVERSE
\VERSE delictum oris eorum,  sermonem labiorum ipsorum ; et comprehendantur in superbia sua.
Et de execratione et mendacio annuntiabuntur 14 in consummatione :
in ira consummationis,  et non erunt.
Et scient quia Deus dominabitur Jacob,  et finium terræ. \EVERSE
\VERSE Convertentur ad vesperam,  et famem patientur ut canes :et circuibunt civitatem. 16 Ipsi dispergentur ad manducandum ; 
si vero non fuerint saturati,  et murmurabunt. \EVERSE
\VERSE Ego autem cantabo fortitudinem tuam, et exsultabo mane misericordiam tuam :
quia factus es susceptor meus, 
et refugium meum in die tribulationis meæ. \EVERSE
\VERSE Adjutor meus,  tibi psallam, quia Deus susceptor meus es ; 
Deus meus,  misericordia mea.

}
\newcommand{\psalmlix}{
\VERSE In finem,  pro his qui immutabuntur,  in tituli inscriptionem ipsi David,  in doctrinam, \VERSE cum succendit Mesopotamiam Syriæ et Sobal,  et convertit Joab,  et percussit Idumæam in valle Salinarum duodecim millia.\VERSE Deus,  repulisti nos,  et destruxisti nos ; iratus es,  et misertus es nobis. \EVERSE
\VERSE Commovisti terram,  et conturbasti eam ; sana contritiones ejus,  quia commota est. \EVERSE
\VERSE Ostendisti populo tuo dura ; potasti nos vino compunctionis. \EVERSE
\VERSE Dedisti metuentibus te significationem, ut fugiant a facie arcus ; 
ut liberentur dilecti tui. \EVERSE
\VERSE Salvum fac dextera tua,  et exaudi me.\VERSE Deus locutus est in sancto suo :lætabor,  et partibor Sichimam ; 
et convallem tabernaculorum metibor. \EVERSE
\VERSE Meus est Galaad,  et meus est Manasses ; et Ephraim fortitudo capitis mei.
Juda rex meus ;  \EVERSE
\VERSE Moab olla spei meæ.In Idumæam extendam calceamentum meum :
mihi alienigenæ subditi sunt. \EVERSE
\VERSE Quis deducet me in civitatem munitam ?quis deducet me usque in Idumæam ? \EVERSE
\VERSE nonne tu,  Deus,  qui repulisti nos ?et non egredieris,  Deus,  in virtutibus nostris ? \EVERSE
\VERSE Da nobis auxilium de tribulatione, quia vana salus hominis. \EVERSE
\VERSE In Deo faciemus virtutem ; et ipse ad nihilum deducet tribulantes nos.

}
\newcommand{\psalmlx}{
\VERSE In finem. In hymnis David.\VERSE Exaudi,  Deus,  deprecationem meam ; intende orationi meæ. \EVERSE
\VERSE A finibus terræ ad te clamavi,  dum anxiaretur cor meum ; in petra exaltasti me.
Deduxisti me,  4 quia factus es spes mea :
turris fortitudinis a facie inimici. \EVERSE
\VERSE Inhabitabo in tabernaculo tuo in sæcula ; protegar in velamento alarum tuarum. \EVERSE
\VERSE Quoniam tu,  Deus meus,  exaudisti orationem meam ; dedisti hæreditatem timentibus nomen tuum. \EVERSE
\VERSE Dies super dies regis adjicies ; annos ejus usque in diem generationis et generationis. \EVERSE
\VERSE Permanet in æternum in conspectu Dei :misericordiam et veritatem ejus quis requiret ? \EVERSE
\VERSE Sic psalmum dicam nomini tuo in sæculum sæculi, ut reddam vota mea de die in diem.

}
\newcommand{\psalmlxi}{
\VERSE In finem,  pro Idithun. Psalmus David.\VERSE Nonne Deo subjecta erit anima mea ?ab ipso enim salutare meum. \EVERSE
\VERSE Nam et ipse Deus meus et salutaris meus ; susceptor meus,  non movebor amplius. \EVERSE
\VERSE Quousque irruitis in hominem ?interficitis universi vos, 
tamquam parieti inclinato et maceriæ depulsæ. \EVERSE
\VERSE Verumtamen pretium meum cogitaverunt repellere ; cucurri in siti :
ore suo benedicebant, 
et corde suo maledicebant. \EVERSE
\VERSE Verumtamen Deo subjecta esto,  anima mea, quoniam ab ipso patientia mea : \EVERSE
\VERSE quia ipse Deus meus et salvator meus, adjutor meus,  non emigrabo. \EVERSE
\VERSE In Deo salutare meum et gloria mea ; Deus auxilii mei,  et spes mea in Deo est. \EVERSE
\VERSE Sperate in eo,  omnis congregatio populi ; effundite coram illo corda vestra :
Deus adjutor noster in æternum. \EVERSE
\VERSE Verumtamen vani filii hominum, mendaces filii hominum in stateris, 
ut decipiant ipsi de vanitate in idipsum. \EVERSE
\VERSE Nolite sperare in iniquitate, et rapinas nolite concupiscere ; 
divitiæ si affluant,  nolite cor apponere. \EVERSE
\VERSE Semel locutus est Deus ; duo hæc audivi :
quia potestas Dei est,  13 et tibi,  Domine,  misericordia :
quia tu reddes unicuique juxta opera sua.

}
\newcommand{\psalmlxii}{
\VERSE Psalmus David,  cum esset in deserto Idumææ.
\VERSE Deus,  Deus meus,  ad te de luce vigilo.Sitivit in te anima mea ;  quam multipliciter tibi caro mea ! \EVERSE
\VERSE In terra deserta,  et invia,  et inaquosa, sic in sancto apparui tibi, 
ut viderem virtutem tuam et gloriam tuam. \EVERSE
\VERSE Quoniam melior est misericordia tua super vitas, labia mea laudabunt te. \EVERSE
\VERSE Sic benedicam te in vita mea, et in nomine tuo levabo manus meas. \EVERSE
\VERSE Sicut adipe et pinguedine repleatur anima mea, et labiis exsultationis laudabit os meum. \EVERSE
\VERSE Si memor fui tui super stratum meum, in matutinis meditabor in te. \EVERSE
\VERSE Quia fuisti adjutor meus, et in velamento alarum tuarum exsultabo. \EVERSE
\VERSE Adhæsit anima mea post te ;  me suscepit dextera tua. \EVERSE
\VERSE Ipsi vero in vanum quæsierunt animam meam :introibunt in inferiora terræ ;  \EVERSE
\VERSE tradentur in manus gladii : partes vulpium erunt. \EVERSE
\VERSE Rex vero lætabitur in Deo ;  laudabuntur omnes qui jurant in eo :
quia obstructum est os loquentium iniqua.

}
\newcommand{\psalmlxiii}{
\VERSE In finem. Psalmus David.\VERSE Exaudi,  Deus,  orationem meam cum deprecor ; a timore inimici eripe animam meam. \EVERSE
\VERSE Protexisti me a conventu malignantium, a multitudine operantium iniquitatem. \EVERSE
\VERSE Quia exacuerunt ut gladium linguas suas ; intenderunt arcum rem amaram,  \EVERSE
\VERSE ut sagittent in occultis immaculatum.\VERSE Subito sagittabunt eum,  et non timebunt ; firmaverunt sibi sermonem nequam.
Narraverunt ut absconderent laqueos ; 
dixerunt : Quis videbit eos ? \EVERSE
\VERSE Scrutati sunt iniquitates ;  defecerunt scrutantes scrutinio.
Accedet homo ad cor altum,  \EVERSE
\VERSE et exaltabitur Deus.Sagittæ parvulorum factæ sunt plagæ eorum,  \EVERSE
\VERSE et infirmatæ sunt contra eos linguæ eorum.Conturbati sunt omnes qui videbant eos,  \EVERSE
\VERSE et timuit omnis homo.Et annuntiaverunt opera Dei, 
et facta ejus intellexerunt. \EVERSE
\VERSE Lætabitur justus in Domino,  et sperabit in eo, et laudabuntur omnes recti corde.

}
\newcommand{\psalmlxiv}{
\VERSE In finem. Psalmus David,  canticum Jeremiæ et Ezechielis populo transmigrationis,  cum inciperent exire.\VERSE Te decet hymnus,  Deus,  in Sion, et tibi reddetur votum in Jerusalem. \EVERSE
\VERSE Exaudi orationem meam ; ad te omnis caro veniet. \EVERSE
\VERSE Verba iniquorum prævaluerunt super nos, et impietatibus nostris tu propitiaberis. \EVERSE
\VERSE Beatus quem elegisti et assumpsisti :inhabitabit in atriis tuis.
Replebimur in bonis domus tuæ ; 
sanctum est templum tuum,  \EVERSE
\VERSE mirabile in æquitate.Exaudi nos,  Deus,  salutaris noster, 
spes omnium finium terræ,  et in mari longe. \EVERSE
\VERSE Præparans montes in virtute tua, accinctus potentia ;  \EVERSE
\VERSE qui conturbas profundum maris, sonum fluctuum ejus.
Turbabuntur gentes,  \EVERSE
\VERSE et timebunt qui habitant terminos a signis tuis ; exitus matutini et vespere delectabis. \EVERSE
\VERSE Visitasti terram,  et inebriasti eam ; multiplicasti locupletare eam.
Flumen Dei repletum est aquis ;  parasti cibum illorum :
quoniam ita est præparatio ejus. \EVERSE
\VERSE Rivos ejus inebria ; multiplica genimina ejus :
in stillicidiis ejus lætabitur germinans. \EVERSE
\VERSE Benedices coronæ anni benignitatis tuæ, et campi tui replebuntur ubertate. \EVERSE
\VERSE Pinguescent speciosa deserti, et exsultatione colles accingentur. \EVERSE
\VERSE Induti sunt arietes ovium, et valles abundabunt frumento ; 
clamabunt,  etenim hymnum dicent.

}
\newcommand{\psalmlxv}{
\VERSE In finem. Canticum psalmi resurrectionis.Jubilate Deo,  omnis terra ;  \EVERSE
\VERSE psalmum dicite nomini ejus ; date gloriam laudi ejus. \EVERSE
\VERSE Dicite Deo : Quam terribilia sunt opera tua,  Domine !in multitudine virtutis tuæ mentientur tibi inimici tui. \EVERSE
\VERSE Omnis terra adoret te,  et psallat tibi ; psalmum dicat nomini tuo. \EVERSE
\VERSE Venite,  et videte opera Dei :terribilis in consiliis super filios hominum. \EVERSE
\VERSE Qui convertit mare in aridam ; in flumine pertransibunt pede :
ibi lætabimur in ipso. \EVERSE
\VERSE Qui dominatur in virtute sua in æternum ; oculi ejus super gentes respiciunt :
qui exasperant non exaltentur in semetipsis. \EVERSE
\VERSE Benedicite,  gentes,  Deum nostrum, et auditam facite vocem laudis ejus : \EVERSE
\VERSE qui posuit animam meam ad vitam, et non dedit in commotionem pedes meos. \EVERSE
\VERSE Quoniam probasti nos,  Deus ; igne nos examinasti,  sicut examinatur argentum. \EVERSE
\VERSE Induxisti nos in laqueum ; posuisti tribulationes in dorso nostro ;  \EVERSE
\VERSE imposuisti homines super capita nostra.Transivimus per ignem et aquam, 
et eduxisti nos in refrigerium. \EVERSE
\VERSE Introibo in domum tuam in holocaustis ; reddam tibi vota mea 14 quæ distinxerunt labia mea :
et locutum est os meum in tribulatione mea. \EVERSE
\VERSE Holocausta medullata offeram tibi,  cum incenso arietum ; offeram tibi boves cum hircis. \EVERSE
\VERSE Venite,  audite,  et narrabo,  omnes qui timetis Deum, quanta fecit animæ meæ. \EVERSE
\VERSE Ad ipsum ore meo clamavi, et exaltavi sub lingua mea. \EVERSE
\VERSE Iniquitatem si aspexi in corde meo, non exaudiet Dominus. \EVERSE
\VERSE Propterea exaudivit Deus, et attendit voci deprecationis meæ. \EVERSE
\VERSE Benedictus Deus,  qui non amovit orationem meam, et misericordiam suam a me.

}
\newcommand{\psalmlxvi}{
\VERSE In finem,  in hymnis. Psalmus cantici David.\VERSE Deus misereatur nostri,  et benedicat nobis ; illuminet vultum suum super nos,  et misereatur nostri : \EVERSE
\VERSE ut cognoscamus in terra viam tuam, in omnibus gentibus salutare tuum. \EVERSE
\VERSE Confiteantur tibi populi,  Deus :confiteantur tibi populi omnes. \EVERSE
\VERSE Lætentur et exsultent gentes, quoniam judicas populos in æquitate, 
et gentes in terra dirigis. \EVERSE
\VERSE Confiteantur tibi populi,  Deus :confiteantur tibi populi omnes. \EVERSE
\VERSE Terra dedit fructum suum :benedicat nos Deus,  Deus noster ! \EVERSE
\VERSE Benedicat nos Deus, et metuant eum omnes fines terræ.

}
\newcommand{\psalmlxvii}{
\VERSE In finem. Psalmus cantici ipsi David.\VERSE Exsurgat Deus,  et dissipentur inimici ejus ; et fugiant qui oderunt eum a facie ejus. \EVERSE
\VERSE Sicut deficit fumus,  deficiant ; sicut fluit cera a facie ignis,  sic pereant peccatores a facie Dei. \EVERSE
\VERSE Et justi epulentur,  et exsultent in conspectu Dei, et delectentur in lætitia. \EVERSE
\VERSE Cantate Deo ;  psalmum dicite nomini ejus :iter facite ei qui ascendit super occasum.
Dominus nomen illi ;  exsultate in conspectu ejus.
Turbabuntur a facie ejus,  \EVERSE
\VERSE patris orphanorum,  et judicis viduarum ; Deus in loco sancto suo. \EVERSE
\VERSE Deus qui inhabitare facit unius moris in domo ; qui educit vinctos in fortitudine, 
similiter eos qui exasperant,  qui habitant in sepulchris. \EVERSE
\VERSE Deus,  cum egredereris in conspectu populi tui, cum pertransires in deserto,  \EVERSE
\VERSE terra mota est,  etenim cæli distillaverunt, a facie Dei Sinai,  a facie Dei Israël. \EVERSE
\VERSE Pluviam voluntariam segregabis,  Deus,  hæreditati tuæ ; et infirmata est,  tu vero perfecisti eam. \EVERSE
\VERSE Animalia tua habitabunt in ea ; parasti in dulcedine tua pauperi,  Deus. \EVERSE
\VERSE Dominus dabit verbum evangelizantibus,  virtute multa.\VERSE Rex virtutum dilecti,  dilecti ; et speciei domus dividere spolia. \EVERSE
\VERSE Si dormiatis inter medios cleros, pennæ columbæ deargentatæ, 
et posteriora dorsi ejus in pallore auri. \EVERSE
\VERSE Dum discernit cælestis reges super eam, nive dealbabuntur in Selmon. \EVERSE
\VERSE Mons Dei,  mons pinguis :mons coagulatus,  mons pinguis. \EVERSE
\VERSE Ut quid suspicamini,  montes coagulatos ?mons in quo beneplacitum est Deo habitare in eo ; 
etenim Dominus habitabit in finem. \EVERSE
\VERSE Currus Dei decem millibus multiplex,  millia lætantium ; Dominus in eis in Sina,  in sancto. \EVERSE
\VERSE Ascendisti in altum,  cepisti captivitatem, accepisti dona in hominibus ; 
etenim non credentes inhabitare Dominum Deum. \EVERSE
\VERSE Benedictus Dominus die quotidie :prosperum iter faciet nobis Deus salutarium nostrorum. \EVERSE
\VERSE Deus noster,  Deus salvos faciendi ; et Domini,  Domini exitus mortis. \EVERSE
\VERSE Verumtamen Deus confringet capita inimicorum suorum, verticem capilli perambulantium in delictis suis. \EVERSE
\VERSE Dixit Dominus : Ex Basan convertam, convertam in profundum maris : \EVERSE
\VERSE ut intingatur pes tuus in sanguine ; lingua canum tuorum ex inimicis,  ab ipso. \EVERSE
\VERSE Viderunt ingressus tuos,  Deus, ingressus Dei mei,  regis mei,  qui est in sancto. \EVERSE
\VERSE Prævenerunt principes conjuncti psallentibus, in medio juvencularum tympanistriarum. \EVERSE
\VERSE In ecclesiis benedicite Deo Domino de fontibus Israël.\VERSE Ibi Benjamin adolescentulus,  in mentis excessu ; principes Juda,  duces eorum ; 
principes Zabulon,  principes Nephthali. \EVERSE
\VERSE Manda,  Deus,  virtuti tuæ ; confirma hoc,  Deus,  quod operatus es in nobis. \EVERSE
\VERSE A templo tuo in Jerusalem, tibi offerent reges munera. \EVERSE
\VERSE Increpa feras arundinis ; congregatio taurorum in vaccis populorum :
ut excludant eos qui probati sunt argento.
Dissipa gentes quæ bella volunt. \EVERSE
\VERSE Venient legati ex Ægypto ; Æthiopia præveniet manus ejus Deo. \EVERSE
\VERSE Regna terræ,  cantate Deo ; psallite Domino ;  psallite Deo. \EVERSE
\VERSE Qui ascendit super cælum cæli,  ad orientem :ecce dabit voci suæ vocem virtutis. \EVERSE
\VERSE Date gloriam Deo super Israël ; magnificentia ejus et virtus ejus in nubibus. \EVERSE
\VERSE Mirabilis Deus in sanctis suis ; Deus Israël ipse dabit virtutem et fortitudinem plebi suæ.
Benedictus Deus !

}
\newcommand{\psalmlxviii}{
\VERSE In finem,  pro iis qui commutabuntur. David.\VERSE Salvum me fac,  Deus, quoniam intraverunt aquæ usque ad animam meam. \EVERSE
\VERSE Infixus sum in limo profundi et non est substantia.Veni in altitudinem maris,  et tempestas demersit me. \EVERSE
\VERSE Laboravi clamans,  raucæ factæ sunt fauces meæ ; defecerunt oculi mei,  dum spero in Deum meum. \EVERSE
\VERSE Multiplicati sunt super capillos capitis meiqui oderunt me gratis.
Confortati sunt qui persecuti sunt me
inimici mei injuste ; 
quæ non rapui,  tunc exsolvebam. \EVERSE
\VERSE Deus,  tu scis insipientiam meam ; et delicta mea a te non sunt abscondita. \EVERSE
\VERSE Non erubescant in me qui exspectant te,  Domine,  Domine virtutum ; non confundantur super me qui quærunt te,  Deus Israël. \EVERSE
\VERSE Quoniam propter te sustinui opprobrium ; operuit confusio faciem meam. \EVERSE
\VERSE Extraneus factus sum fratribus meis, et peregrinus filiis matris meæ. \EVERSE
\VERSE Quoniam zelus domus tuæ comedit me, et opprobria exprobrantium tibi ceciderunt super me. \EVERSE
\VERSE Et operui in jejunio animam meam, et factum est in opprobrium mihi. \EVERSE
\VERSE Et posui vestimentum meum cilicium ; et factus sum illis in parabolam. \EVERSE
\VERSE Adversum me loquebantur qui sedebant in porta, et in me psallebant qui bibebant vinum. \EVERSE
\VERSE Ego vero orationem meam ad te,  Domine ; tempus beneplaciti,  Deus.
In multitudine misericordiæ tuæ, 
exaudi me in veritate salutis tuæ. \EVERSE
\VERSE Eripe me de luto,  ut non infigar ; libera me ab iis qui oderunt me,  et de profundis aquarum. \EVERSE
\VERSE Non me demergat tempestas aquæ, neque absorbeat me profundum, 
neque urgeat super me puteus os suum. \EVERSE
\VERSE Exaudi me,  Domine,  quoniam benigna est misericordia tua ; secundum multitudinem miserationum tuarum respice in me. \EVERSE
\VERSE Et ne avertas faciem tuam a puero tuo ; quoniam tribulor,  velociter exaudi me. \EVERSE
\VERSE Intende animæ meæ,  et libera eam ; propter inimicos meos,  eripe me. \EVERSE
\VERSE Tu scis improperium meum,  et confusionem meam,  et reverentiam meam ; \VERSE in conspectu tuo sunt omnes qui tribulant me.Improperium exspectavit cor meum et miseriam :
et sustinui qui simul contristaretur,  et non fuit ; 
et qui consolaretur,  et non inveni. \EVERSE
\VERSE Et dederunt in escam meam fel, et in siti mea potaverunt me aceto. \EVERSE
\VERSE Fiat mensa eorum coram ipsis in laqueum, et in retributiones,  et in scandalum. \EVERSE
\VERSE Obscurentur oculi eorum,  ne videant, et dorsum eorum semper incurva. \EVERSE
\VERSE Effunde super eos iram tuam, et furor iræ tuæ comprehendat eos. \EVERSE
\VERSE Fiat habitatio eorum deserta, et in tabernaculis eorum non sit qui inhabitet. \EVERSE
\VERSE Quoniam quem tu percussisti persecuti sunt, et super dolorem vulnerum meorum addiderunt. \EVERSE
\VERSE Appone iniquitatem super iniquitatem eorum, et non intrent in justitiam tuam. \EVERSE
\VERSE Deleantur de libro viventium, et cum justis non scribantur. \EVERSE
\VERSE Ego sum pauper et dolens ; salus tua,  Deus,  suscepit me. \EVERSE
\VERSE Laudabo nomen Dei cum cantico, et magnificabo eum in laude : \EVERSE
\VERSE et placebit Deo super vitulum novellum, cornua producentem et ungulas. \EVERSE
\VERSE Videant pauperes,  et lætentur ; quærite Deum,  et vivet anima vestra : \EVERSE
\VERSE quoniam exaudivit pauperes Dominus, et vinctos suos non despexit. \EVERSE
\VERSE Laudent illum cæli et terra ; mare,  et omnia reptilia in eis. \EVERSE
\VERSE Quoniam Deus salvam faciet Sion, et ædificabuntur civitates Juda, 
et inhabitabunt ibi,  et hæreditate acquirent eam. \EVERSE
\VERSE Et semen servorum ejus possidebit eam ; et qui diligunt nomen ejus habitabunt in ea.

}
\newcommand{\psalmlxix}{
\VERSE In finem. Psalmus David in rememorationem,  quod salvum fecerit eum Dominus.\VERSE Deus,  in adjutorium meum intende ; Domine,  ad adjuvandum me festina. \EVERSE
\VERSE Confundantur,  et revereantur,  qui quærunt animam meam.\VERSE Avertantur retrorsum,  et erubescant,  qui volunt mihi mala ; avertantur statim erubescentes qui dicunt mihi : Euge,  euge ! \EVERSE
\VERSE Exsultent et lætentur in te omnes qui quærunt te ; et dicant semper : Magnificetur Dominus,  qui diligunt salutare tuum. \EVERSE
\VERSE Ego vero egenus et pauper sum ; Deus,  adjuva me.
Adjutor meus et liberator meus es tu ; 
Domine,  ne moreris.

}
\newcommand{\psalmlxx}{
\VERSE Psalmus David,  filiorum Jonadab,  et priorum captivorum.In te,  Domine,  speravi ;  non confundar in æternum. \EVERSE
\VERSE In justitia tua libera me,  et eripe me :inclina ad me aurem tuam,  et salva me. \EVERSE
\VERSE Esto mihi in Deum protectorem, et in locum munitum,  ut salvum me facias :
quoniam firmamentum meum et refugium meum es tu. \EVERSE
\VERSE Deus meus,  eripe me de manu peccatoris, et de manu contra legem agentis,  et iniqui : \EVERSE
\VERSE quoniam tu es patientia mea,  Domine ; Domine,  spes mea a juventute mea. \EVERSE
\VERSE In te confirmatus sum ex utero ; de ventre matris meæ tu es protector meus ; 
in te cantatio mea semper. \EVERSE
\VERSE Tamquam prodigium factus sum multis ; et tu adjutor fortis. \EVERSE
\VERSE Repleatur os meum laude, ut cantem gloriam tuam, 
tota die magnitudinem tuam. \EVERSE
\VERSE Ne projicias me in tempore senectutis ; cum defecerit virtus mea,  ne derelinquas me. \EVERSE
\VERSE Quia dixerunt inimici mei mihi, et qui custodiebant animam meam consilium fecerunt in unum,  \EVERSE
\VERSE dicentes : Deus dereliquit eum :persequimini et comprehendite eum, 
quia non est qui eripiat. \EVERSE
\VERSE Deus,  ne elongeris a me ; Deus meus,  in auxilium meum respice. \EVERSE
\VERSE Confundantur et deficiant detrahentes animæ meæ ; operiantur confusione et pudore qui quærunt mala mihi. \EVERSE
\VERSE Ego autem semper sperabo, et adjiciam super omnem laudem tuam. \EVERSE
\VERSE Os meum annuntiabit justitiam tuam, tota die salutare tuum.
Quoniam non cognovi litteraturam,  \EVERSE
\VERSE introibo in potentias Domini ; Domine,  memorabor justitiæ tuæ solius. \EVERSE
\VERSE Deus,  docuisti me a juventute mea ; et usque nunc pronuntiabo mirabilia tua. \EVERSE
\VERSE Et usque in senectam et senium, Deus,  ne derelinquas me, 
donec annuntiem brachium tuum generationi omni quæ ventura est, 
potentiam tuam,  19 et justitiam tuam,  Deus,  usque in altissima ; 
quæ fecisti magnalia,  Deus : quis similis tibi ? \EVERSE
\VERSE Quantas ostendisti mihi tribulationes multas et malas !et conversus vivificasti me, 
et de abyssis terræ iterum reduxisti me. \EVERSE
\VERSE Multiplicasti magnificentiam tuam ; et conversus consolatus es me. \EVERSE
\VERSE Nam et ego confitebor tibi in vasis psalmi veritatem tuam,  Deus ; psallam tibi in cithara,  sanctus Israël. \EVERSE
\VERSE Exsultabunt labia mea cum cantavero tibi ; et anima mea quam redemisti. \EVERSE
\VERSE Sed et lingua mea tota die meditabitur justitiam tuam, cum confusi et reveriti fuerint qui quærunt mala mihi.

}
\newcommand{\psalmlxxi}{
\VERSE Psalmus,  in Salomonem.\VERSE Deus,  judicium tuum regi da, et justitiam tuam filio regis ; 
judicare populum tuum in justitia, 
et pauperes tuos in judicio. \EVERSE
\VERSE Suscipiant montes pacem populo, et colles justitiam. \EVERSE
\VERSE Judicabit pauperes populi, et salvos faciet filios pauperum, 
et humiliabit calumniatorem. \EVERSE
\VERSE Et permanebit cum sole,  et ante lunam, in generatione et generationem. \EVERSE
\VERSE Descendet sicut pluvia in vellus, et sicut stillicidia stillantia super terram. \EVERSE
\VERSE Orietur in diebus ejus justitia,  et abundantia pacis, donec auferatur luna. \EVERSE
\VERSE Et dominabitur a mari usque ad mare, et a flumine usque ad terminos orbis terrarum. \EVERSE
\VERSE Coram illo procident Æthiopes, et inimici ejus terram lingent. \EVERSE
\VERSE Reges Tharsis et insulæ munera offerent ; reges Arabum et Saba dona adducent : \EVERSE
\VERSE et adorabunt eum omnes reges terræ ; omnes gentes servient ei. \EVERSE
\VERSE Quia liberabit pauperem a potente, et pauperem cui non erat adjutor. \EVERSE
\VERSE Parcet pauperi et inopi, et animas pauperum salvas faciet. \EVERSE
\VERSE Ex usuris et iniquitate redimet animas eorum, et honorabile nomen eorum coram illo. \EVERSE
\VERSE Et vivet,  et dabitur ei de auro Arabiæ ; et adorabunt de ipso semper, 
tota die benedicent ei. \EVERSE
\VERSE Et erit firmamentum in terra in summis montium ; superextolletur super Libanum fructus ejus, 
et florebunt de civitate sicut fœnum terræ. \EVERSE
\VERSE Sit nomen ejus benedictum in sæcula ; ante solem permanet nomen ejus.
Et benedicentur in ipso omnes tribus terræ ; 
omnes gentes magnificabunt eum. \EVERSE
\VERSE Benedictus Dominus Deus Israël, qui facit mirabilia solus. \EVERSE
\VERSE Et benedictum nomen majestatis ejus in æternum, et replebitur majestate ejus omnis terra. Fiat,  fiat. \EVERSE
\VERSE Defecerunt laudes David,  filii Jesse.}
\newcommand{\psalmlxxii}{
\VERSE Psalmus Asaph.Quam bonus Israël Deus, 
his qui recto sunt corde ! \EVERSE
\VERSE Mei autem pene moti sunt pedes, pene effusi sunt gressus mei : \EVERSE
\VERSE quia zelavi super iniquos, pacem peccatorum videns. \EVERSE
\VERSE Quia non est respectus morti eorum, et firmamentum in plaga eorum. \EVERSE
\VERSE In labore hominum non sunt, et cum hominibus non flagellabuntur. \EVERSE
\VERSE Ideo tenuit eos superbia ; operti sunt iniquitate et impietate sua. \EVERSE
\VERSE Prodiit quasi ex adipe iniquitas eorum ; transierunt in affectum cordis. \EVERSE
\VERSE Cogitaverunt et locuti sunt nequitiam ; iniquitatem in excelso locuti sunt. \EVERSE
\VERSE Posuerunt in cælum os suum, et lingua eorum transivit in terra. \EVERSE
\VERSE Ideo convertetur populus meus hic, et dies pleni invenientur in eis. \EVERSE
\VERSE Et dixerunt : Quomodo scit Deus, et si est scientia in excelso ? \EVERSE
\VERSE Ecce ipsi peccatores,  et abundantes in sæculoobtinuerunt divitias. \EVERSE
\VERSE Et dixi : Ergo sine causa justificavi cor meum, et lavi inter innocentes manus meas,  \EVERSE
\VERSE et fui flagellatus tota die, et castigatio mea in matutinis. \EVERSE
\VERSE Si dicebam : Narrabo sic ; ecce nationem filiorum tuorum reprobavi. \EVERSE
\VERSE Existimabam ut cognoscerem hoc ; labor est ante me : \EVERSE
\VERSE donec intrem in sanctuarium Dei, et intelligam in novissimis eorum. \EVERSE
\VERSE Verumtamen propter dolos posuisti eis ; dejecisti eos dum allevarentur. \EVERSE
\VERSE Quomodo facti sunt in desolationem ?subito defecerunt : perierunt propter iniquitatem suam. \EVERSE
\VERSE Velut somnium surgentium,  Domine, in civitate tua imaginem ipsorum ad nihilum rediges. \EVERSE
\VERSE Quia inflammatum est cor meum, et renes mei commutati sunt ;  \EVERSE
\VERSE et ego ad nihilum redactus sum,  et nescivi :\VERSE ut jumentum factus sum apud te, et ego semper tecum. \EVERSE
\VERSE Tenuisti manum dexteram meam, et in voluntate tua deduxisti me, 
et cum gloria suscepisti me. \EVERSE
\VERSE Quid enim mihi est in cælo ?et a te quid volui super terram ? \EVERSE
\VERSE Defecit caro mea et cor meum ; Deus cordis mei,  et pars mea,  Deus in æternum. \EVERSE
\VERSE Quia ecce qui elongant se a te peribunt ; perdidisti omnes qui fornicantur abs te. \EVERSE
\VERSE Mihi autem adhærere Deo bonum est ; ponere in Domino Deo spem meam :
ut annuntiem omnes prædicationes tuas
in portis filiæ Sion.

}
\newcommand{\psalmlxxiii}{
\VERSE Intellectus Asaph.Ut quid,  Deus,  repulisti in finem, 
iratus est furor tuus super oves pascuæ tuæ ? \EVERSE
\VERSE Memor esto congregationis tuæ, quam possedisti ab initio.
Redemisti virgam hæreditatis tuæ, 
mons Sion,  in quo habitasti in eo. \EVERSE
\VERSE Leva manus tuas in superbias eorum in finem :quanta malignatus est inimicus in sancto ! \EVERSE
\VERSE Et gloriati sunt qui oderunt te in medio solemnitatis tuæ ; posuerunt signa sua,  signa : \EVERSE
\VERSE et non cognoverunt sicut in exitu super summum.Quasi in silva lignorum securibus \EVERSE
\VERSE exciderunt januas ejus in idipsum ; in securi et ascia dejecerunt eam. \EVERSE
\VERSE Incenderunt igni sanctuarium tuum ; in terra polluerunt tabernaculum nominis tui. \EVERSE
\VERSE Dixerunt in corde suo cognatio eorum simul :Quiescere faciamus omnes dies festos Dei a terra. \EVERSE
\VERSE Signa nostra non vidimus ; jam non est propheta ; 
et nos non cognoscet amplius. \EVERSE
\VERSE Usquequo,  Deus,  improperabit inimicus ?irritat adversarius nomen tuum in finem ? \EVERSE
\VERSE Ut quid avertis manum tuam, et dexteram tuam de medio sinu tuo in finem ? \EVERSE
\VERSE Deus autem rex noster ante sæcula :operatus est salutem in medio terræ. \EVERSE
\VERSE Tu confirmasti in virtute tua mare ; contribulasti capita draconum in aquis. \EVERSE
\VERSE Tu confregisti capita draconis ; dedisti eum escam populis Æthiopum. \EVERSE
\VERSE Tu dirupisti fontes et torrentes ; tu siccasti fluvios Ethan. \EVERSE
\VERSE Tuus est dies,  et tua est nox ; tu fabricatus es auroram et solem. \EVERSE
\VERSE Tu fecisti omnes terminos terræ ; æstatem et ver tu plasmasti ea. \EVERSE
\VERSE Memor esto hujus : inimicus improperavit Domino, et populus insipiens incitavit nomen tuum. \EVERSE
\VERSE Ne tradas bestiis animas confitentes tibi, et animas pauperum tuorum ne obliviscaris in finem. \EVERSE
\VERSE Respice in testamentum tuum, quia repleti sunt qui obscurati sunt terræ domibus iniquitatum. \EVERSE
\VERSE Ne avertatur humilis factus confusus ; pauper et inops laudabunt nomen tuum. \EVERSE
\VERSE Exsurge,  Deus,  judica causam tuam ; memor esto improperiorum tuorum, 
eorum quæ ab insipiente sunt tota die. \EVERSE
\VERSE Ne obliviscaris voces inimicorum tuorum :superbia eorum qui te oderunt ascendit semper.

}
\newcommand{\psalmlxxiv}{
\VERSE In finem,  ne corrumpas. Psalmus cantici Asaph.\VERSE Confitebimur tibi,  Deus,  confitebimur, et invocabimus nomen tuum ; 
narrabimus mirabilia tua. \EVERSE
\VERSE Cum accepero tempus, ego justitias judicabo. \EVERSE
\VERSE Liquefacta est terra et omnes qui habitant in ea :ego confirmavi columnas ejus. \EVERSE
\VERSE Dixi iniquis : Nolite inique agere :et delinquentibus : Nolite exaltare cornu : \EVERSE
\VERSE nolite extollere in altum cornu vestrum ; nolite loqui adversus Deum iniquitatem. \EVERSE
\VERSE Quia neque ab oriente,  neque ab occidente, neque a desertis montibus : \EVERSE
\VERSE quoniam Deus judex est.Hunc humiliat,  et hunc exaltat : \EVERSE
\VERSE quia calix in manu Domini vini meri,  plenus misto.Et inclinavit ex hoc in hoc ; 
verumtamen fæx ejus non est exinanita :
bibent omnes peccatores terræ. \EVERSE
\VERSE Ego autem annuntiabo in sæculum ; cantabo Deo Jacob : \EVERSE
\VERSE et omnia cornua peccatorum confringam, et exaltabuntur cornua justi.

}
\newcommand{\psalmlxxv}{
\VERSE In finem,  in laudibus. Psalmus Asaph,  canticum ad Assyrios.\VERSE Notus in Judæa Deus ; in Israël magnum nomen ejus. \EVERSE
\VERSE Et factus est in pace locus ejus, et habitatio ejus in Sion. \EVERSE
\VERSE Ibi confregit potentias arcuum, scutum,  gladium,  et bellum. \EVERSE
\VERSE Illuminans tu mirabiliter a montibus æternis ; \VERSE turbati sunt omnes insipientes corde.Dormierunt somnum suum, 
et nihil invenerunt omnes viri divitiarum in manibus suis. \EVERSE
\VERSE Ab increpatione tua,  Deus Jacob, dormitaverunt qui ascenderunt equos. \EVERSE
\VERSE Tu terribilis es ;  et quis resistet tibi ?ex tunc ira tua. \EVERSE
\VERSE De cælo auditum fecisti judicium :terra tremuit et quievit 10 cum exsurgeret in judicium Deus, 
ut salvos faceret omnes mansuetos terræ. \EVERSE
\VERSE Quoniam cogitatio hominis confitebitur tibi, et reliquiæ cogitationis diem festum agent tibi. \EVERSE
\VERSE Vovete et reddite Domino Deo vestro, omnes qui in circuitu ejus affertis munera :
terribili,  13 et ei qui aufert spiritum principum :
terribili apud reges terræ.

}
\newcommand{\psalmlxxvi}{
\VERSE In finem,  pro Idithun. Psalmus Asaph.\VERSE Voce mea ad Dominum clamavi ; voce mea ad Deum,  et intendit mihi. \EVERSE
\VERSE In die tribulationis meæ Deum exquisivi ; manibus meis nocte contra eum,  et non sum deceptus.
Renuit consolari anima mea ;  \EVERSE
\VERSE memor fui Dei,  et delectatus sum, et exercitatus sum,  et defecit spiritus meus. \EVERSE
\VERSE Anticipaverunt vigilias oculi mei ; turbatus sum,  et non sum locutus. \EVERSE
\VERSE Cogitavi dies antiquos, et annos æternos in mente habui. \EVERSE
\VERSE Et meditatus sum nocte cum corde meo, et exercitabar,  et scopebam spiritum meum. \EVERSE
\VERSE Numquid in æternum projiciet Deus ?aut non apponet ut complacitior sit adhuc ? \EVERSE
\VERSE aut in finem misericordiam suam abscindet, a generatione in generationem ? \EVERSE
\VERSE aut obliviscetur misereri Deus ?aut continebit in ira sua misericordias suas ? \EVERSE
\VERSE Et dixi : Nunc cœpi ; hæc mutatio dexteræ Excelsi. \EVERSE
\VERSE Memor fui operum Domini, quia memor ero ab initio mirabilium tuorum : \EVERSE
\VERSE et meditabor in omnibus operibus tuis, et in adinventionibus tuis exercebor. \EVERSE
\VERSE Deus,  in sancto via tua :quis deus magnus sicut Deus noster ? \EVERSE
\VERSE Tu es Deus qui facis mirabilia :notam fecisti in populis virtutem tuam. \EVERSE
\VERSE Redemisti in brachio tuo populum tuum, filios Jacob et Joseph. \EVERSE
\VERSE Viderunt te aquæ,  Deus ; viderunt te aquæ,  et timuerunt :
et turbatæ sunt abyssi. \EVERSE
\VERSE Multitudo sonitus aquarum ; vocem dederunt nubes.
Etenim sagittæ tuæ transeunt ;  \EVERSE
\VERSE vox tonitrui tui in rota.Illuxerunt coruscationes tuæ orbi terræ ; 
commota est,  et contremuit terra. \EVERSE
\VERSE In mari via tua,  et semitæ tuæ in aquis multis, et vestigia tua non cognoscentur. \EVERSE
\VERSE Deduxisti sicut oves populum tuum, in manu Moysi et Aaron.

}
\newcommand{\psalmlxxvii}{
\VERSE Intellectus Asaph.Attendite,  popule meus,  legem meam ; 
inclinate aurem vestram in verba oris mei. \EVERSE
\VERSE Aperiam in parabolis os meum ; loquar propositiones ab initio. \EVERSE
\VERSE Quanta audivimus,  et cognovimus ea, et patres nostri narraverunt nobis. \EVERSE
\VERSE Non sunt occultata a filiis eorum in generatione altera, narrantes laudes Domini et virtutes ejus, 
et mirabilia ejus quæ fecit. \EVERSE
\VERSE Et suscitavit testimonium in Jacob, et legem posuit in Israël, 
quanta mandavit patribus nostris
nota facere ea filiis suis : \EVERSE
\VERSE ut cognoscat generatio altera :filii qui nascentur et exsurgent, 
et narrabunt filiis suis,  \EVERSE
\VERSE ut ponant in Deo spem suam, et non obliviscantur operum Dei, 
et mandata ejus exquirant : \EVERSE
\VERSE ne fiant,  sicut patres eorum, generatio prava et exasperans ; 
generatio quæ non direxit cor suum, 
et non est creditus cum Deo spiritus ejus. \EVERSE
\VERSE Filii Ephrem,  intendentes et mittentes arcum, conversi sunt in die belli. \EVERSE
\VERSE Non custodierunt testamentum Dei, et in lege ejus noluerunt ambulare. \EVERSE
\VERSE Et obliti sunt benefactorum ejus, et mirabilium ejus quæ ostendit eis. \EVERSE
\VERSE Coram patribus eorum fecit mirabiliain terra Ægypti,  in campo Taneos. \EVERSE
\VERSE Interrupit mare,  et perduxit eos, et statuit aquas quasi in utre : \EVERSE
\VERSE et deduxit eos in nube diei, et tota nocte in illuminatione ignis. \EVERSE
\VERSE Interrupit petram in eremo, et adaquavit eos velut in abysso multa. \EVERSE
\VERSE Et eduxit aquam de petra, et deduxit tamquam flumina aquas. \EVERSE
\VERSE Et apposuerunt adhuc peccare ei ; in iram excitaverunt Excelsum in inaquoso. \EVERSE
\VERSE Et tentaverunt Deum in cordibus suis, ut peterent escas animabus suis. \EVERSE
\VERSE Et male locuti sunt de Deo ; dixerunt : Numquid poterit Deus parare mensam in deserto ? \EVERSE
\VERSE quoniam percussit petram,  et fluxerunt aquæ, et torrentes inundaverunt.
Numquid et panem poterit dare, 
aut parare mensam populo suo ? \EVERSE
\VERSE Ideo audivit Dominus et distulit ; et ignis accensus est in Jacob, 
et ira ascendit in Israël : \EVERSE
\VERSE quia non crediderunt in Deo, nec speraverunt in salutari ejus. \EVERSE
\VERSE Et mandavit nubibus desuper, et januas cæli aperuit. \EVERSE
\VERSE Et pluit illis manna ad manducandum, et panem cæli dedit eis. \EVERSE
\VERSE Panem angelorum manducavit homo ; cibaria misit eis in abundantia. \EVERSE
\VERSE Transtulit austrum de cælo, et induxit in virtute sua africum. \EVERSE
\VERSE Et pluit super eos sicut pulverem carnes, et sicut arenam maris volatilia pennata. \EVERSE
\VERSE Et ceciderunt in medio castrorum eorum, circa tabernacula eorum. \EVERSE
\VERSE Et manducaverunt,  et saturati sunt nimis, et desiderium eorum attulit eis : \EVERSE
\VERSE non sunt fraudati a desiderio suo.Adhuc escæ eorum erant in ore ipsorum,  \EVERSE
\VERSE et ira Dei ascendit super eos :et occidit pingues eorum, 
et electos Israël impedivit. \EVERSE
\VERSE In omnibus his peccaverunt adhuc, et non crediderunt in mirabilibus ejus. \EVERSE
\VERSE Et defecerunt in vanitate dies eorum, et anni eorum cum festinatione. \EVERSE
\VERSE Cum occideret eos,  quærebant eum et revertebantur, et diluculo veniebant ad eum. \EVERSE
\VERSE Et rememorati sunt quia Deus adjutor est eorum, et Deus excelsus redemptor eorum est. \EVERSE
\VERSE Et dilexerunt eum in ore suo, et lingua sua mentiti sunt ei ;  \EVERSE
\VERSE cor autem eorum non erat rectum cum eo, nec fideles habiti sunt in testamento ejus. \EVERSE
\VERSE Ipse autem est misericors, et propitius fiet peccatis eorum, 
et non disperdet eos.
Et abundavit ut averteret iram suam, 
et non accendit omnem iram suam. \EVERSE
\VERSE Et recordatus est quia caro sunt, spiritus vadens et non rediens. \EVERSE
\VERSE Quoties exacerbaverunt eum in deserto ; in iram concitaverunt eum in inaquoso ? \EVERSE
\VERSE Et conversi sunt,  et tentaverunt Deum, et sanctum Israël exacerbaverunt. \EVERSE
\VERSE Non sunt recordati manus ejus, die qua redemit eos de manu tribulantis : \EVERSE
\VERSE sicut posuit in Ægypto signa sua, et prodigia sua in campo Taneos ;  \EVERSE
\VERSE et convertit in sanguinem flumina eorum, et imbres eorum,  ne biberent. \EVERSE
\VERSE Misit in eos cœnomyiam,  et comedit eos, et ranam,  et disperdidit eos ;  \EVERSE
\VERSE et dedit ærugini fructus eorum, et labores eorum locustæ ;  \EVERSE
\VERSE et occidit in grandine vineas eorum, et moros eorum in pruina ;  \EVERSE
\VERSE et tradidit grandini jumenta eorum, et possessionem eorum igni ;  \EVERSE
\VERSE misit in eos iram indignationis suæ, indignationem,  et iram,  et tribulationem, 
immissiones per angelos malos. \EVERSE
\VERSE Viam fecit semitæ iræ suæ :non pepercit a morte animabus eorum, 
et jumenta eorum in morte conclusit : \EVERSE
\VERSE et percussit omne primogenitum in terra Ægypti ; primitias omnis laboris eorum in tabernaculis Cham : \EVERSE
\VERSE et abstulit sicut oves populum suum, et perduxit eos tamquam gregem in deserto : \EVERSE
\VERSE et deduxit eos in spe,  et non timuerunt, et inimicos eorum operuit mare. \EVERSE
\VERSE Et induxit eos in montem sanctificationis suæ, montem quem acquisivit dextera ejus ; 
et ejecit a facie eorum gentes, 
et sorte divisit eis terram in funiculo distributionis ;  \EVERSE
\VERSE et habitare fecit in tabernaculis eorum tribus Israël.\VERSE Et tentaverunt,  et exacerbaverunt Deum excelsum, et testimonia ejus non custodierunt. \EVERSE
\VERSE Et averterunt se,  et non servaverunt pactum :quemadmodum patres eorum,  conversi sunt in arcum pravum. \EVERSE
\VERSE In iram concitaverunt eum in collibus suis, et in sculptilibus suis ad æmulationem eum provocaverunt. \EVERSE
\VERSE Audivit Deus,  et sprevit, et ad nihilum redegit valde Israël. \EVERSE
\VERSE Et repulit tabernaculum Silo, tabernaculum suum,  ubi habitavit in hominibus. \EVERSE
\VERSE Et tradidit in captivitatem virtutem eorum, et pulchritudinem eorum in manus inimici. \EVERSE
\VERSE Et conclusit in gladio populum suum, et hæreditatem suam sprevit. \EVERSE
\VERSE Juvenes eorum comedit ignis, et virgines eorum non sunt lamentatæ. \EVERSE
\VERSE Sacerdotes eorum in gladio ceciderunt, et viduæ eorum non plorabantur. \EVERSE
\VERSE Et excitatus est tamquam dormiens Dominus, tamquam potens crapulatus a vino. \EVERSE
\VERSE Et percussit inimicos suos in posteriora ; opprobrium sempiternum dedit illis. \EVERSE
\VERSE Et repulit tabernaculum Joseph, et tribum Ephraim non elegit : \EVERSE
\VERSE sed elegit tribum Juda, montem Sion,  quem dilexit. \EVERSE
\VERSE Et ædificavit sicut unicornium sanctificium suum, in terra quam fundavit in sæcula. \EVERSE
\VERSE Et elegit David,  servum suum, et sustulit eum de gregibus ovium ; 
de post fœtantes accepit eum : \EVERSE
\VERSE pascere Jacob servum suum, et Israël hæreditatem suam. \EVERSE
\VERSE Et pavit eos in innocentia cordis sui, et in intellectibus manuum suarum deduxit eos.

}
\newcommand{\psalmlxxviii}{
\VERSE Psalmus Asaph.Deus,  venerunt gentes in hæreditatem tuam ; 
polluerunt templum sanctum tuum ; 
posuerunt Jerusalem in pomorum custodiam. \EVERSE
\VERSE Posuerunt morticina servorum tuorum escas volatilibus cæli ; carnes sanctorum tuorum bestiis terræ. \EVERSE
\VERSE Effuderunt sanguinem eorum tamquam aquam in circuitu Jerusalem, et non erat qui sepeliret. \EVERSE
\VERSE Facti sumus opprobrium vicinis nostris ; subsannatio et illusio his qui in circuitu nostro sunt. \EVERSE
\VERSE Usquequo,  Domine,  irasceris in finem ?accendetur velut ignis zelus tuus ? \EVERSE
\VERSE Effunde iram tuam in gentes quæ te non noverunt, et in regna quæ nomen tuum non invocaverunt : \EVERSE
\VERSE quia comederunt Jacob, et locum ejus desolaverunt. \EVERSE
\VERSE Ne memineris iniquitatum nostrarum antiquarum ; cito anticipent nos misericordiæ tuæ, 
quia pauperes facti sumus nimis. \EVERSE
\VERSE Adjuva nos,  Deus salutaris noster, et propter gloriam nominis tui,  Domine,  libera nos :
et propitius esto peccatis nostris,  propter nomen tuum. \EVERSE
\VERSE Ne forte dicant in gentibus : Ubi est Deus eorum ?et innotescat in nationibus coram oculis nostris
ultio sanguinis servorum tuorum qui effusus est. \EVERSE
\VERSE Introëat in conspectu tuo gemitus compeditorum ; secundum magnitudinem brachii tui posside filios mortificatorum : \EVERSE
\VERSE et redde vicinis nostris septuplum in sinu eorum ; improperium ipsorum quod exprobraverunt tibi,  Domine. \EVERSE
\VERSE Nos autem populus tuus,  et oves pascuæ tuæ, confitebimur tibi in sæculum ; 
in generationem et generationem annuntiabimus laudem tuam.

}
\newcommand{\psalmlxxix}{
\VERSE In finem,  pro iis qui commutabuntur. Testimonium Asaph,  psalmus.\VERSE Qui regis Israël,  intende ; qui deducis velut ovem Joseph.
Qui sedes super cherubim,  manifestare \EVERSE
\VERSE coram Ephraim,  Benjamin,  et Manasse.Excita potentiam tuam,  et veni, 
ut salvos facias nos. \EVERSE
\VERSE Deus,  converte nos,  et ostende faciem tuam, et salvi erimus. \EVERSE
\VERSE Domine Deus virtutum, quousque irasceris super orationem servi tui ? \EVERSE
\VERSE cibabis nos pane lacrimarum, et potum dabis nobis in lacrimis in mensura ? \EVERSE
\VERSE Posuisti nos in contradictionem vicinis nostris, et inimici nostri subsannaverunt nos. \EVERSE
\VERSE Deus virtutum,  converte nos,  et ostende faciem tuam, et salvi erimus. \EVERSE
\VERSE Vineam de Ægypto transtulisti :ejecisti gentes,  et plantasti eam. \EVERSE
\VERSE Dux itineris fuisti in conspectu ejus ; plantasti radices ejus,  et implevit terram. \EVERSE
\VERSE Operuit montes umbra ejus, et arbusta ejus cedros Dei. \EVERSE
\VERSE Extendit palmites suos usque ad mare, et usque ad flumen propagines ejus. \EVERSE
\VERSE Ut quid destruxisti maceriam ejus, et vindemiant eam omnes qui prætergrediuntur viam ? \EVERSE
\VERSE Exterminavit eam aper de silva, et singularis ferus depastus est eam. \EVERSE
\VERSE Deus virtutum,  convertere, respice de cælo,  et vide, 
et visita vineam istam : \EVERSE
\VERSE et perfice eam quam plantavit dextera tua, et super filium hominis quem confirmasti tibi. \EVERSE
\VERSE Incensa igni et suffossa, ab increpatione vultus tui peribunt. \EVERSE
\VERSE Fiat manus tua super virum dexteræ tuæ, et super filium hominis quem confirmasti tibi. \EVERSE
\VERSE Et non discedimus a te :vivificabis nos,  et nomen tuum invocabimus. \EVERSE
\VERSE Domine Deus virtutum,  converte nos,  et ostende faciem tuam, et salvi erimus.

}
\newcommand{\psalmlxxx}{
\VERSE In finem,  pro torcularibus. Psalmus ipsi Asaph.\VERSE Exsultate Deo adjutori nostro ; jubilate Deo Jacob. \EVERSE
\VERSE Sumite psalmum,  et date tympanum ; psalterium jucundum cum cithara. \EVERSE
\VERSE Buccinate in neomenia tuba, in insigni die solemnitatis vestræ : \EVERSE
\VERSE quia præceptum in Israël est, et judicium Deo Jacob. \EVERSE
\VERSE Testimonium in Joseph posuit illud, cum exiret de terra Ægypti ; 
linguam quam non noverat,  audivit. \EVERSE
\VERSE Divertit ab oneribus dorsum ejus ; manus ejus in cophino servierunt. \EVERSE
\VERSE In tribulatione invocasti me,  et liberavi te.Exaudivi te in abscondito tempestatis ; 
probavi te apud aquam contradictionis. \EVERSE
\VERSE Audi,  populus meus,  et contestabor te.Israël,  si audieris me,  \EVERSE
\VERSE non erit in te deus recens, neque adorabis deum alienum. \EVERSE
\VERSE Ego enim sum Dominus Deus tuus, qui eduxi te de terra Ægypti.
Dilata os tuum,  et implebo illud. \EVERSE
\VERSE Et non audivit populus meus vocem meam, et Israël non intendit mihi. \EVERSE
\VERSE Et dimisi eos secundum desideria cordis eorum ; ibunt in adinventionibus suis. \EVERSE
\VERSE Si populus meus audisset me, Israël si in viis meis ambulasset,  \EVERSE
\VERSE pro nihilo forsitan inimicos eorum humiliassem, et super tribulantes eos misissem manum meam. \EVERSE
\VERSE Inimici Domini mentiti sunt ei, et erit tempus eorum in sæcula. \EVERSE
\VERSE Et cibavit eos ex adipe frumenti, et de petra melle saturavit eos.

}
\newcommand{\psalmlxxxi}{
\VERSE Psalmus Asaph.Deus stetit in synagoga deorum ; 
in medio autem deos dijudicat. \EVERSE
\VERSE Usquequo judicatis iniquitatem, et facies peccatorum sumitis ? \EVERSE
\VERSE Judicate egeno et pupillo ; humilem et pauperem justificate. \EVERSE
\VERSE Eripite pauperem, et egenum de manu peccatoris liberate. \EVERSE
\VERSE Nescierunt,  neque intellexerunt ; in tenebris ambulant :
movebuntur omnia fundamenta terræ. \EVERSE
\VERSE Ego dixi : Dii estis, et filii Excelsi omnes. \EVERSE
\VERSE Vos autem sicut homines moriemini, et sicut unus de principibus cadetis. \EVERSE
\VERSE Surge,  Deus,  judica terram, quoniam tu hæreditabis in omnibus gentibus.

}
\newcommand{\psalmlxxxii}{
\VERSE Canticum Psalmi Asaph.\VERSE Deus,  quis similis erit tibi ?ne taceas,  neque compescaris,  Deus : \EVERSE
\VERSE quoniam ecce inimici tui sonuerunt, et qui oderunt te extulerunt caput. \EVERSE
\VERSE Super populum tuum malignaverunt consilium, et cogitaverunt adversus sanctos tuos. \EVERSE
\VERSE Dixerunt : Venite,  et disperdamus eos de gente, et non memoretur nomen Israël ultra. \EVERSE
\VERSE Quoniam cogitaverunt unanimiter ; simul adversum te testamentum disposuerunt : \EVERSE
\VERSE tabernacula Idumæorum et Ismahelitæ, Moab et Agareni,  \EVERSE
\VERSE Gebal,  et Ammon,  et Amalec ; alienigenæ cum habitantibus Tyrum. \EVERSE
\VERSE Etenim Assur venit cum illis :facti sunt in adjutorium filiis Lot. \EVERSE
\VERSE Fac illis sicut Madian et Sisaræ, sicut Jabin in torrente Cisson. \EVERSE
\VERSE Disperierunt in Endor ; facti sunt ut stercus terræ. \EVERSE
\VERSE Pone principes eorum sicut Oreb, et Zeb,  et Zebee,  et Salmana :
omnes principes eorum,  \EVERSE
\VERSE qui dixerunt : Hæreditate possideamus sanctuarium Dei.\VERSE Deus meus,  pone illos ut rotam, et sicut stipulam ante faciem venti. \EVERSE
\VERSE Sicut ignis qui comburit silvam, et sicut flamma comburens montes,  \EVERSE
\VERSE ita persequeris illos in tempestate tua, et in ira tua turbabis eos. \EVERSE
\VERSE Imple facies eorum ignominia, et quærent nomen tuum,  Domine. \EVERSE
\VERSE Erubescant,  et conturbentur in sæculum sæculi, et confundantur,  et pereant. \EVERSE
\VERSE Et cognoscant quia nomen tibi Dominus :tu solus Altissimus in omni terra.

}
\newcommand{\psalmlxxxiii}{
\VERSE In finem,  pro torcularibus filiis Core. Psalmus.\VERSE Quam dilecta tabernacula tua,  Domine virtutum !\VERSE Concupiscit,  et deficit anima mea in atria Domini ; cor meum et caro mea exsultaverunt in Deum vivum. \EVERSE
\VERSE Etenim passer invenit sibi domum, et turtur nidum sibi,  ubi ponat pullos suos :
altaria tua,  Domine virtutum, 
rex meus,  et Deus meus. \EVERSE
\VERSE Beati qui habitant in domo tua,  Domine ; in sæcula sæculorum laudabunt te. \EVERSE
\VERSE Beatus vir cujus est auxilium abs te :ascensiones in corde suo disposuit,  \EVERSE
\VERSE in valle lacrimarum,  in loco quem posuit.\VERSE Etenim benedictionem dabit legislator ; ibunt de virtute in virtutem :
videbitur Deus deorum in Sion. \EVERSE
\VERSE Domine Deus virtutum,  exaudi orationem meam ; auribus percipe,  Deus Jacob. \EVERSE
\VERSE Protector noster,  aspice,  Deus, et respice in faciem christi tui. \EVERSE
\VERSE Quia melior est dies una in atriis tuis super millia ; elegi abjectus esse in domo Dei mei
magis quam habitare in tabernaculis peccatorum. \EVERSE
\VERSE Quia misericordiam et veritatem diligit Deus :gratiam et gloriam dabit Dominus. \EVERSE
\VERSE Non privabit bonis eos qui ambulant in innocentia :Domine virtutum,  beatus homo qui sperat in te.

}
\newcommand{\psalmlxxxiv}{
\VERSE In finem,  filiis Core. Psalmus.\VERSE Benedixisti,  Domine,  terram tuam ; avertisti captivitatem Jacob. \EVERSE
\VERSE Remisisti iniquitatem plebis tuæ ; operuisti omnia peccata eorum. \EVERSE
\VERSE Mitigasti omnem iram tuam ; avertisti ab ira indignationis tuæ. \EVERSE
\VERSE Converte nos,  Deus salutaris noster, et averte iram tuam a nobis. \EVERSE
\VERSE Numquid in æternum irasceris nobis ?aut extendes iram tuam a generatione in generationem ? \EVERSE
\VERSE Deus,  tu conversus vivificabis nos, et plebs tua lætabitur in te. \EVERSE
\VERSE Ostende nobis,  Domine,  misericordiam tuam, et salutare tuum da nobis. \EVERSE
\VERSE Audiam quid loquatur in me Dominus Deus, quoniam loquetur pacem in plebem suam, 
et super sanctos suos, 
et in eos qui convertuntur ad cor. \EVERSE
\VERSE Verumtamen prope timentes eum salutare ipsius, ut inhabitet gloria in terra nostra. \EVERSE
\VERSE Misericordia et veritas obviaverunt sibi ; justitia et pax osculatæ sunt. \EVERSE
\VERSE Veritas de terra orta est, et justitia de cælo prospexit. \EVERSE
\VERSE Etenim Dominus dabit benignitatem, et terra nostra dabit fructum suum. \EVERSE
\VERSE Justitia ante eum ambulabit, et ponet in via gressus suos.

}
\newcommand{\psalmlxxxv}{
\VERSE Oratio ipsi David.Inclina,  Domine,  aurem tuam et exaudi me, 
quoniam inops et pauper sum ego. \EVERSE
\VERSE Custodi animam meam,  quoniam sanctus sum ; salvum fac servum tuum,  Deus meus,  sperantem in te. \EVERSE
\VERSE Miserere mei,  Domine, quoniam ad te clamavi tota die ;  \EVERSE
\VERSE lætifica animam servi tui, quoniam ad te,  Domine,  animam meam levavi. \EVERSE
\VERSE Quoniam tu,  Domine,  suavis et mitis, et multæ misericordiæ omnibus invocantibus te. \EVERSE
\VERSE Auribus percipe,  Domine,  orationem meam, et intende voci deprecationis meæ. \EVERSE
\VERSE In die tribulationis meæ clamavi ad te, quia exaudisti me. \EVERSE
\VERSE Non est similis tui in diis,  Domine, et non est secundum opera tua. \EVERSE
\VERSE Omnes gentes quascumque fecisti venient, et adorabunt coram te,  Domine, 
et glorificabunt nomen tuum. \EVERSE
\VERSE Quoniam magnus es tu,  et faciens mirabilia ; tu es Deus solus. \EVERSE
\VERSE Deduc me,  Domine,  in via tua,  et ingrediar in veritate tua ; lætetur cor meum,  ut timeat nomen tuum. \EVERSE
\VERSE Confitebor tibi,  Domine Deus meus,  in toto corde meo, et glorificabo nomen tuum in æternum : \EVERSE
\VERSE quia misericordia tua magna est super me, et eruisti animam meam ex inferno inferiori. \EVERSE
\VERSE Deus,  iniqui insurrexerunt super me, et synagoga potentium quæsierunt animam meam :
et non proposuerunt te in conspectu suo. \EVERSE
\VERSE Et tu,  Domine Deus,  miserator et misericors ; patiens,  et multæ misericordiæ,  et verax. \EVERSE
\VERSE Respice in me,  et miserere mei ; da imperium tuum puero tuo, 
et salvum fac filium ancillæ tuæ. \EVERSE
\VERSE Fac mecum signum in bonum, ut videant qui oderunt me,  et confundantur :
quoniam tu,  Domine,  adjuvisti me,  et consolatus es me.

}
\newcommand{\psalmlxxxvi}{
\VERSE Filiis Core. Psalmus cantici.Fundamenta ejus in montibus sanctis ;  \EVERSE
\VERSE diligit Dominus portas Sion super omnia tabernacula Jacob.\VERSE Gloriosa dicta sunt de te,  civitas Dei !\VERSE Memor ero Rahab et Babylonis,  scientium me ; ecce alienigenæ,  et Tyrus,  et populus Æthiopum, 
hi fuerunt illic. \EVERSE
\VERSE Numquid Sion dicet : Homo et homo natus est in ea, et ipse fundavit eam Altissimus ? \EVERSE
\VERSE Dominus narrabit in scripturis populorum et principum, horum qui fuerunt in ea. \EVERSE
\VERSE Sicut lætantium omniumhabitatio est in te.

}
\newcommand{\psalmlxxxvii}{
\VERSE Canticum Psalmi,  filiis Core,  in finem,  pro Maheleth ad respondendum. Intellectus Eman Ezrahitæ.\VERSE Domine,  Deus salutis meæ, in die clamavi et nocte coram te. \EVERSE
\VERSE Intret in conspectu tuo oratio mea, inclina aurem tuam ad precem meam. \EVERSE
\VERSE Quia repleta est malis anima mea, et vita mea inferno appropinquavit. \EVERSE
\VERSE Æstimatus sum cum descendentibus in lacum, factus sum sicut homo sine adjutorio,  \EVERSE
\VERSE inter mortuos liber ; sicut vulnerati dormientes in sepulchris, 
quorum non es memor amplius, 
et ipsi de manu tua repulsi sunt. \EVERSE
\VERSE Posuerunt me in lacu inferiori, in tenebrosis,  et in umbra mortis. \EVERSE
\VERSE Super me confirmatus est furor tuus, et omnes fluctus tuos induxisti super me. \EVERSE
\VERSE Longe fecisti notos meos a me ; posuerunt me abominationem sibi.
Traditus sum,  et non egrediebar ;  \EVERSE
\VERSE oculi mei languerunt præ inopia.Clamavi ad te,  Domine,  tota die ; 
expandi ad te manus meas. \EVERSE
\VERSE Numquid mortuis facies mirabilia ?aut medici suscitabunt,  et confitebuntur tibi ? \EVERSE
\VERSE Numquid narrabit aliquis in sepulchro misericordiam tuam, et veritatem tuam in perditione ? \EVERSE
\VERSE Numquid cognoscentur in tenebris mirabilia tua ?et justitia tua in terra oblivionis ? \EVERSE
\VERSE Et ego ad te,  Domine,  clamavi, et mane oratio mea præveniet te. \EVERSE
\VERSE Ut quid,  Domine,  repellis orationem meam ; avertis faciem tuam a me ? \EVERSE
\VERSE Pauper sum ego,  et in laboribus a juventute mea ; exaltatus autem,  humiliatus sum et conturbatus. \EVERSE
\VERSE In me transierunt iræ tuæ, et terrores tui conturbaverunt me : \EVERSE
\VERSE circumdederunt me sicut aqua tota die ; circumdederunt me simul. \EVERSE
\VERSE Elongasti a me amicum et proximum, et notos meos a miseria.

}
\newcommand{\psalmlxxxviii}{
\VERSE Intellectus Ethan Ezrahitæ.\VERSE Misericordias Domini in æternum cantabo ; in generationem et generationem annuntiabo veritatem tuam in ore meo. \EVERSE
\VERSE Quoniam dixisti : In æternum misericordia ædificabitur in cælis ; præparabitur veritas tua in eis. \EVERSE
\VERSE Disposui testamentum electis meis ; juravi David servo meo : \EVERSE
\VERSE Usque in æternum præparabo semen tuum, et ædificabo in generationem et generationem sedem tuam. \EVERSE
\VERSE Confitebuntur cæli mirabilia tua,  Domine ; etenim veritatem tuam in ecclesia sanctorum. \EVERSE
\VERSE Quoniam quis in nubibus æquabitur Domino ; similis erit Deo in filiis Dei ? \EVERSE
\VERSE Deus,  qui glorificatur in consilio sanctorum, magnus et terribilis super omnes qui in circuitu ejus sunt. \EVERSE
\VERSE Domine Deus virtutum,  quis similis tibi ?potens es,  Domine,  et veritas tua in circuitu tuo. \EVERSE
\VERSE Tu dominaris potestati maris ; motum autem fluctuum ejus tu mitigas. \EVERSE
\VERSE Tu humiliasti,  sicut vulneratum,  superbum ; in brachio virtutis tuæ dispersisti inimicos tuos. \EVERSE
\VERSE Tui sunt cæli,  et tua est terra :orbem terræ,  et plenitudinem ejus tu fundasti ;  \EVERSE
\VERSE aquilonem et mare tu creasti.Thabor et Hermon in nomine tuo exsultabunt : \EVERSE
\VERSE tuum brachium cum potentia.Firmetur manus tua,  et exaltetur dextera tua : \EVERSE
\VERSE justitia et judicium præparatio sedis tuæ :misericordia et veritas præcedent faciem tuam. \EVERSE
\VERSE Beatus populus qui scit jubilationem :Domine,  in lumine vultus tui ambulabunt,  \EVERSE
\VERSE et in nomine tuo exsultabunt tota die, et in justitia tua exaltabuntur. \EVERSE
\VERSE Quoniam gloria virtutis eorum tu es, et in beneplacito tuo exaltabitur cornu nostrum. \EVERSE
\VERSE Quia Domini est assumptio nostra, et sancti Israël regis nostri. \EVERSE
\VERSE Tunc locutus es in visione sanctis tuis, et dixisti : Posui adjutorium in potente, 
et exaltavi electum de plebe mea. \EVERSE
\VERSE Inveni David,  servum meum ; oleo sancto meo unxi eum. \EVERSE
\VERSE Manus enim mea auxiliabitur ei, et brachium meum confortabit eum. \EVERSE
\VERSE Nihil proficiet inimicus in eo, et filius iniquitatis non apponet nocere ei. \EVERSE
\VERSE Et concidam a facie ipsius inimicos ejus, et odientes eum in fugam convertam. \EVERSE
\VERSE Et veritas mea et misericordia mea cum ipso, et in nomine meo exaltabitur cornu ejus. \EVERSE
\VERSE Et ponam in mari manum ejus, et in fluminibus dexteram ejus. \EVERSE
\VERSE Ipse invocabit me : Pater meus es tu, Deus meus,  et susceptor salutis meæ. \EVERSE
\VERSE Et ego primogenitum ponam illum, excelsum præ regibus terræ. \EVERSE
\VERSE In æternum servabo illi misericordiam meam, et testamentum meum fidele ipsi. \EVERSE
\VERSE Et ponam in sæculum sæculi semen ejus, et thronum ejus sicut dies cæli. \EVERSE
\VERSE Si autem dereliquerint filii ejus legem meam, et in judiciis meis non ambulaverint ;  \EVERSE
\VERSE si justitias meas profanaverint, et mandata mea non custodierint : \EVERSE
\VERSE visitabo in virga iniquitates eorum, et in verberibus peccata eorum ;  \EVERSE
\VERSE misericordiam autem meam non dispergam ab eo, neque nocebo in veritate mea,  \EVERSE
\VERSE neque profanabo testamentum meum :et quæ procedunt de labiis meis non faciam irrita. \EVERSE
\VERSE Semel juravi in sancto meo,  si David mentiar :\VERSE semen ejus in æternum manebit.Et thronus ejus sicut sol in conspectu meo,  \EVERSE
\VERSE et sicut luna perfecta in æternum, et testis in cælo fidelis. \EVERSE
\VERSE Tu vero repulisti et despexisti ; distulisti christum tuum. \EVERSE
\VERSE Evertisti testamentum servi tui ; profanasti in terra sanctuarium ejus. \EVERSE
\VERSE Destruxisti omnes sepes ejus ; posuisti firmamentum ejus formidinem. \EVERSE
\VERSE Diripuerunt eum omnes transeuntes viam ; factus est opprobrium vicinis suis. \EVERSE
\VERSE Exaltasti dexteram deprimentium eum ; lætificasti omnes inimicos ejus. \EVERSE
\VERSE Avertisti adjutorium gladii ejus, et non es auxiliatus ei in bello. \EVERSE
\VERSE Destruxisti eum ab emundatione, et sedem ejus in terram collisisti. \EVERSE
\VERSE Minorasti dies temporis ejus ; perfudisti eum confusione. \EVERSE
\VERSE Usquequo,  Domine,  avertis in finem ?exardescet sicut ignis ira tua ? \EVERSE
\VERSE Memorare quæ mea substantia :numquid enim vane constituisti omnes filios hominum ? \EVERSE
\VERSE Quis est homo qui vivet et non videbit mortem ?eruet animam suam de manu inferi ? \EVERSE
\VERSE Ubi sunt misericordiæ tuæ antiquæ,  Domine, sicut jurasti David in veritate tua ? \EVERSE
\VERSE Memor esto,  Domine,  opprobrii servorum tuorum, quod continui in sinu meo,  multarum gentium : \EVERSE
\VERSE quod exprobraverunt inimici tui,  Domine ; quod exprobraverunt commutationem christi tui. \EVERSE
\VERSE Benedictus Dominus in æternum. Fiat,  fiat.}
\newcommand{\psalmlxxxix}{
\VERSE Oratio Moysi,  hominis Dei.Domine,  refugium factus es nobis
a generatione in generationem. \EVERSE
\VERSE Priusquam montes fierent, aut formaretur terra et orbis, 
a sæculo et usque in sæculum tu es,  Deus. \EVERSE
\VERSE Ne avertas hominem in humilitatem :et dixisti : Convertimini,  filii hominum. \EVERSE
\VERSE Quoniam mille anni ante oculos tuostamquam dies hesterna quæ præteriit :
et custodia in nocte 5 quæ pro nihilo habentur, 
eorum anni erunt. \EVERSE
\VERSE Mane sicut herba transeat ; mane floreat,  et transeat ; 
vespere decidat,  induret,  et arescat. \EVERSE
\VERSE Quia defecimus in ira tua, et in furore tuo turbati sumus. \EVERSE
\VERSE Posuisti iniquitates nostras in conspectu tuo ; sæculum nostrum in illuminatione vultus tui. \EVERSE
\VERSE Quoniam omnes dies nostri defecerunt, et in ira tua defecimus.
Anni nostri sicut aranea meditabuntur ;  \EVERSE
\VERSE dies annorum nostrorum in ipsis septuaginta anni.Si autem in potentatibus octoginta anni, 
et amplius eorum labor et dolor ; 
quoniam supervenit mansuetudo,  et corripiemur. \EVERSE
\VERSE Quis novit potestatem iræ tuæ, et præ timore tuo iram tuam 12 dinumerare ?
Dexteram tuam sic notam fac, 
et eruditos corde in sapientia. \EVERSE
\VERSE Convertere,  Domine ;  usquequo ?et deprecabilis esto super servos tuos. \EVERSE
\VERSE Repleti sumus mane misericordia tua ; et exsultavimus,  et delectati sumus omnibus diebus nostris. \EVERSE
\VERSE Lætati sumus pro diebus quibus nos humiliasti ; annis quibus vidimus mala. \EVERSE
\VERSE Respice in servos tuos et in opera tua, et dirige filios eorum. \EVERSE
\VERSE Et sit splendor Domini Dei nostri super nos, et opera manuum nostrarum dirige super nos, 
et opus manuum nostrarum dirige.

}
\newcommand{\psalmxc}{
\VERSE Laus cantici David.Qui habitat in adjutorio Altissimi, 
in protectione Dei cæli commorabitur. \EVERSE
\VERSE Dicet Domino : Susceptor meus es tu,  et refugium meum ; Deus meus,  sperabo in eum. \EVERSE
\VERSE Quoniam ipse liberavit me de laqueo venantium, et a verbo aspero. \EVERSE
\VERSE Scapulis suis obumbrabit tibi, et sub pennis ejus sperabis. \EVERSE
\VERSE Scuto circumdabit te veritas ejus :non timebis a timore nocturno ;  \EVERSE
\VERSE a sagitta volante in die, a negotio perambulante in tenebris, 
ab incursu,  et dæmonio meridiano. \EVERSE
\VERSE Cadent a latere tuo mille,  et decem millia a dextris tuis ; ad te autem non appropinquabit. \EVERSE
\VERSE Verumtamen oculis tuis considerabis, et retributionem peccatorum videbis. \EVERSE
\VERSE Quoniam tu es,  Domine,  spes mea ; Altissimum posuisti refugium tuum. \EVERSE
\VERSE Non accedet ad te malum, et flagellum non appropinquabit tabernaculo tuo. \EVERSE
\VERSE Quoniam angelis suis mandavit de te, ut custodiant te in omnibus viis tuis. \EVERSE
\VERSE In manibus portabunt te, ne forte offendas ad lapidem pedem tuum. \EVERSE
\VERSE Super aspidem et basiliscum ambulabis, et conculcabis leonem et draconem. \EVERSE
\VERSE Quoniam in me speravit,  liberabo eum ; protegam eum,  quoniam cognovit nomen meum. \EVERSE
\VERSE Clamabit ad me,  et ego exaudiam eum ; cum ipso sum in tribulatione :
eripiam eum,  et glorificabo eum. \EVERSE
\VERSE Longitudine dierum replebo eum, et ostendam illi salutare meum.

}
\newcommand{\psalmxci}{
\VERSE Psalmus cantici,  in die sabbati.\VERSE Bonum est confiteri Domino, et psallere nomini tuo,  Altissime : \EVERSE
\VERSE ad annuntiandum mane misericordiam tuam, et veritatem tuam per noctem,  \EVERSE
\VERSE in decachordo,  psalterio ; cum cantico,  in cithara. \EVERSE
\VERSE Quia delectasti me,  Domine,  in factura tua ; et in operibus manuum tuarum exsultabo. \EVERSE
\VERSE Quam magnificata sunt opera tua,  Domine !nimis profundæ factæ sunt cogitationes tuæ. \EVERSE
\VERSE Vir insipiens non cognoscet, et stultus non intelliget hæc. \EVERSE
\VERSE Cum exorti fuerint peccatores sicut fœnum, et apparuerint omnes qui operantur iniquitatem, 
ut intereant in sæculum sæculi : \EVERSE
\VERSE tu autem Altissimus in æternum,  Domine.\VERSE Quoniam ecce inimici tui,  Domine, quoniam ecce inimici tui peribunt ; 
et dispergentur omnes qui operantur iniquitatem. \EVERSE
\VERSE Et exaltabitur sicut unicornis cornu meum, et senectus mea in misericordia uberi. \EVERSE
\VERSE Et despexit oculus meus inimicos meos, et in insurgentibus in me malignantibus audiet auris mea. \EVERSE
\VERSE Justus ut palma florebit ; sicut cedrus Libani multiplicabitur. \EVERSE
\VERSE Plantati in domo Domini, in atriis domus Dei nostri florebunt. \EVERSE
\VERSE Adhuc multiplicabuntur in senecta uberi, et bene patientes erunt : \EVERSE
\VERSE ut annuntient quoniam rectus Dominus Deus noster, et non est iniquitas in eo.

}
\newcommand{\psalmxcii}{
\VERSE Laus cantici ipsi David,  in die ante sabbatum,  quando fundata est terra.Dominus regnavit,  decorem indutus est :
indutus est Dominus fortitudinem,  et præcinxit se.
Etenim firmavit orbem terræ,  qui non commovebitur. \EVERSE
\VERSE Parata sedes tua ex tunc ; a sæculo tu es. \EVERSE
\VERSE Elevaverunt flumina,  Domine, elevaverunt flumina vocem suam ; 
elevaverunt flumina fluctus suos,  \EVERSE
\VERSE a vocibus aquarum multarum.Mirabiles elationes maris ; 
mirabilis in altis Dominus. \EVERSE
\VERSE Testimonia tua credibilia facta sunt nimis ; domum tuam decet sanctitudo,  Domine, 
in longitudinem dierum.

}
\newcommand{\psalmxciii}{
\VERSE Psalmus ipsi David,  quarta sabbati.Deus ultionum Dominus ; 
Deus ultionum libere egit. \EVERSE
\VERSE Exaltare,  qui judicas terram ; redde retributionem superbis. \EVERSE
\VERSE Usquequo peccatores,  Domine, usquequo peccatores gloriabuntur ;  \EVERSE
\VERSE effabuntur et loquentur iniquitatem ; loquentur omnes qui operantur injustitiam ? \EVERSE
\VERSE Populum tuum,  Domine,  humiliaverunt, et hæreditatem tuam vexaverunt. \EVERSE
\VERSE Viduam et advenam interfecerunt, et pupillos occiderunt. \EVERSE
\VERSE Et dixerunt : Non videbit Dominus, nec intelliget Deus Jacob. \EVERSE
\VERSE Intelligite,  insipientes in populo ; et stulti,  aliquando sapite. \EVERSE
\VERSE Qui plantavit aurem non audiet ?aut qui finxit oculum non considerat ? \EVERSE
\VERSE Qui corripit gentes non arguet, qui docet hominem scientiam ? \EVERSE
\VERSE Dominus scit cogitationes hominum, quoniam vanæ sunt. \EVERSE
\VERSE Beatus homo quem tu erudieris,  Domine, et de lege tua docueris eum : \EVERSE
\VERSE ut mitiges ei a diebus malis, donec fodiatur peccatori fovea. \EVERSE
\VERSE Quia non repellet Dominus plebem suam, et hæreditatem suam non derelinquet,  \EVERSE
\VERSE quoadusque justitia convertatur in judicium :et qui juxta illam,  omnes qui recto sunt corde. \EVERSE
\VERSE Quis consurget mihi adversus malignantes ?aut quis stabit mecum adversus operantes iniquitatem ? \EVERSE
\VERSE Nisi quia Dominus adjuvit me, paulominus habitasset in inferno anima mea. \EVERSE
\VERSE Si dicebam : Motus est pes meus :misericordia tua,  Domine,  adjuvabat me. \EVERSE
\VERSE Secundum multitudinem dolorum meorum in corde meo, consolationes tuæ lætificaverunt animam meam. \EVERSE
\VERSE Numquid adhæret tibi sedes iniquitatis, qui fingis laborem in præcepto ? \EVERSE
\VERSE Captabunt in animam justi, et sanguinem innocentem condemnabunt. \EVERSE
\VERSE Et factus est mihi Dominus in refugium, et Deus meus in adjutorium spei meæ. \EVERSE
\VERSE Et reddet illis iniquitatem ipsorum, et in malitia eorum disperdet eos :
disperdet illos Dominus Deus noster.

}
\newcommand{\psalmxciv}{
\VERSE Laus cantici ipsi David.Venite,  exsultemus Domino ; 
jubilemus Deo salutari nostro ;  \EVERSE
\VERSE præoccupemus faciem ejus in confessione, et in psalmis jubilemus ei : \EVERSE
\VERSE quoniam Deus magnus Dominus, et rex magnus super omnes deos. \EVERSE
\VERSE Quia in manu ejus sunt omnes fines terræ, et altitudines montium ipsius sunt ;  \EVERSE
\VERSE quoniam ipsius est mare,  et ipse fecit illud, et siccam manus ejus formaverunt. \EVERSE
\VERSE Venite,  adoremus,  et procidamus, et ploremus ante Dominum qui fecit nos : \EVERSE
\VERSE quia ipse est Dominus Deus noster, et nos populus pascuæ ejus,  et oves manus ejus. \EVERSE
\VERSE Hodie si vocem ejus audieritis, nolite obdurare corda vestra 9 sicut in irritatione, 
secundum diem tentationis in deserto, 
ubi tentaverunt me patres vestri :
probaverunt me,  et viderunt opera mea. \EVERSE
\VERSE Quadraginta annis offensus fui generationi illi, et dixi : Semper hi errant corde. \EVERSE
\VERSE Et isti non cognoverunt vias meas :ut juravi in ira mea :
Si introibunt in requiem meam.

}
\newcommand{\psalmxcv}{
\VERSE Canticum ipsi David,  quando domus ædificabatur post captivitatem.Cantate Domino canticum novum ; 
cantate Domino omnis terra. \EVERSE
\VERSE Cantate Domino,  et benedicite nomini ejus ; annuntiate de die in diem salutare ejus. \EVERSE
\VERSE Annuntiate inter gentes gloriam ejus ; in omnibus populis mirabilia ejus. \EVERSE
\VERSE Quoniam magnus Dominus,  et laudabilis nimis :terribilis est super omnes deos. \EVERSE
\VERSE Quoniam omnes dii gentium dæmonia ; Dominus autem cælos fecit. \EVERSE
\VERSE Confessio et pulchritudo in conspectu ejus ; sanctimonia et magnificentia in sanctificatione ejus. \EVERSE
\VERSE Afferte Domino,  patriæ gentium, afferte Domino gloriam et honorem ;  \EVERSE
\VERSE afferte Domino gloriam nomini ejus.Tollite hostias,  et introite in atria ejus ;  \EVERSE
\VERSE adorate Dominum in atrio sancto ejus.Commoveatur a facie ejus universa terra ;  \EVERSE
\VERSE dicite in gentibus,  quia Dominus regnavit.Etenim correxit orbem terræ,  qui non commovebitur ; 
judicabit populos in æquitate. \EVERSE
\VERSE Lætentur cæli,  et exsultet terra ; commoveatur mare et plenitudo ejus ;  \EVERSE
\VERSE gaudebunt campi,  et omnia quæ in eis sunt.Tunc exsultabunt omnia ligna silvarum \EVERSE
\VERSE a facie Domini,  quia venit, quoniam venit judicare terram.
Judicabit orbem terræ in æquitate, 
et populos in veritate sua.

}
\newcommand{\psalmxcvi}{
\VERSE Huic David,  quando terra ejus restituta est.Dominus regnavit : exsultet terra ; 
lætentur insulæ multæ. \EVERSE
\VERSE Nubes et caligo in circuitu ejus ; justitia et judicium correctio sedis ejus. \EVERSE
\VERSE Ignis ante ipsum præcedet, et inflammabit in circuitu inimicos ejus. \EVERSE
\VERSE Illuxerunt fulgura ejus orbi terræ ; vidit,  et commota est terra. \EVERSE
\VERSE Montes sicut cera fluxerunt a facie Domini ; a facie Domini omnis terra. \EVERSE
\VERSE Annuntiaverunt cæli justitiam ejus, et viderunt omnes populi gloriam ejus. \EVERSE
\VERSE Confundantur omnes qui adorant sculptilia, et qui gloriantur in simulacris suis.
Adorate eum omnes angeli ejus. \EVERSE
\VERSE Audivit,  et lætata est Sion, et exsultaverunt filiæ Judæ
propter judicia tua,  Domine. \EVERSE
\VERSE Quoniam tu Dominus altissimus super omnem terram ; nimis exaltatus es super omnes deos. \EVERSE
\VERSE Qui diligitis Dominum,  odite malum :custodit Dominus animas sanctorum suorum ; 
de manu peccatoris liberabit eos. \EVERSE
\VERSE Lux orta est justo, et rectis corde lætitia. \EVERSE
\VERSE Lætamini,  justi,  in Domino, et confitemini memoriæ sanctificationis ejus.

}
\newcommand{\psalmxcvii}{
\VERSE Psalmus ipsi David.Cantate Domino canticum novum,  quia mirabilia fecit. Salvavit sibi dextera ejus,  et brachium sanctum ejus. \EVERSE
\VERSE Notum fecit Dominus salutare suum ; in conspectu gentium revelavit justitiam suam. \EVERSE
\VERSE Recordatus est misericordiæ suæ, et veritatis suæ domui Israël. Viderunt omnes termini terræ salutare Dei nostri. \EVERSE
\VERSE Jubilate Deo,  omnis terra ; cantate,  et exsultate,  et psallite. \EVERSE
\VERSE Psallite Domino in cithara ; in cithara et voce psalmi ;  \EVERSE
\VERSE in tubis ductilibus,  et voce tubæ corneæ.Jubilate in conspectu regis Domini : \EVERSE
\VERSE moveatur mare,  et plenitudo ejus ; orbis terrarum,  et qui habitant in eo. \EVERSE
\VERSE Flumina plaudent manu ; simul montes exsultabunt 9 a conspectu Domini : quoniam venit judicare terram. Judicabit orbem terrarum in justitia,  et populos in æquitate.

}
\newcommand{\psalmxcviii}{
\VERSE Psalmus ipsi David.Dominus regnavit : irascantur populi ; 
qui sedet super cherubim : moveatur terra. \EVERSE
\VERSE Dominus in Sion magnus, et excelsus super omnes populos. \EVERSE
\VERSE Confiteantur nomini tuo magno, quoniam terribile et sanctum est,  \EVERSE
\VERSE et honor regis judicium diligit.Tu parasti directiones ; 
judicium et justitiam in Jacob tu fecisti. \EVERSE
\VERSE Exaltate Dominum Deum nostrum, et adorate scabellum pedum ejus, 
quoniam sanctum est. \EVERSE
\VERSE Moyses et Aaron in sacerdotibus ejus, et Samuel inter eos qui invocant nomen ejus :
invocabant Dominum,  et ipse exaudiebat eos ;  \EVERSE
\VERSE in columna nubis loquebatur ad eos.Custodiebant testimonia ejus, 
et præceptum quod dedit illis. \EVERSE
\VERSE Domine Deus noster,  tu exaudiebas eos ; Deus,  tu propitius fuisti eis, 
et ulciscens in omnes adinventiones eorum. \EVERSE
\VERSE Exaltate Dominum Deum nostrum, et adorate in monte sancto ejus, 
quoniam sanctus Dominus Deus noster.

}
\newcommand{\psalmxcix}{
\VERSE Psalmus in confessione.\VERSE Jubilate Deo,  omnis terra ; servite Domino in lætitia.
Introite in conspectu ejus in exsultatione. \EVERSE
\VERSE Scitote quoniam Dominus ipse est Deus ; ipse fecit nos,  et non ipsi nos :
populus ejus,  et oves pascuæ ejus. \EVERSE
\VERSE Introite portas ejus in confessione ; atria ejus in hymnis :
confitemini illi.
Laudate nomen ejus,  5 quoniam suavis est Dominus, 
in æternum misericordia ejus, 
et usque in generationem et generationem veritas ejus.

}
\newcommand{\psalmc}{
\VERSE Psalmus ipsi David.Misericordiam et judicium cantabo tibi,  Domine ; 
psallam,  \EVERSE
\VERSE et intelligam in via immaculata : quando venies ad me ?Perambulabam in innocentia cordis mei,  in medio domus meæ. \EVERSE
\VERSE Non proponebam ante oculos meos rem injustam ; facientes prævaricationes odivi ; 
non adhæsit mihi 4 cor pravum ; 
declinantem a me malignum non cognoscebam. \EVERSE
\VERSE Detrahentem secreto proximo suo,  hunc persequebar :superbo oculo,  et insatiabili corde,  cum hoc non edebam. \EVERSE
\VERSE Oculi mei ad fideles terræ,  ut sedeant mecum ; ambulans in via immaculata,  hic mihi ministrabat. \EVERSE
\VERSE Non habitabit in medio domus meæ qui facit superbiam ; qui loquitur iniqua non direxit in conspectu oculorum meorum. \EVERSE
\VERSE In matutino interficiebam omnes peccatores terræ, ut disperderem de civitate Domini omnes operantes iniquitatem.

}
\newcommand{\psalmci}{
\VERSE Oratio pauperis,  cum anxius fuerit,  et in conspectu Domini effuderit precem suam.\VERSE Domine,  exaudi orationem meam, et clamor meus ad te veniat. \EVERSE
\VERSE Non avertas faciem tuam a me :in quacumque die tribulor,  inclina ad me aurem tuam ; 
in quacumque die invocavero te,  velociter exaudi me. \EVERSE
\VERSE Quia defecerunt sicut fumus dies mei, et ossa mea sicut cremium aruerunt. \EVERSE
\VERSE Percussus sum ut fœnum,  et aruit cor meum, quia oblitus sum comedere panem meum. \EVERSE
\VERSE A voce gemitus meiadhæsit os meum carni meæ. \EVERSE
\VERSE Similis factus sum pellicano solitudinis ; factus sum sicut nycticorax in domicilio. \EVERSE
\VERSE Vigilavi,  et factus sum sicut passer solitarius in tecto.\VERSE Tota die exprobrabant mihi inimici mei, et qui laudabant me adversum me jurabant : \EVERSE
\VERSE quia cinerem tamquam panem manducabam, et potum meum cum fletu miscebam,  \EVERSE
\VERSE a facie iræ et indignationis tuæ :quia elevans allisisti me. \EVERSE
\VERSE Dies mei sicut umbra declinaverunt, et ego sicut fœnum arui. \EVERSE
\VERSE Tu autem,  Domine,  in æternum permanes, et memoriale tuum in generationem et generationem. \EVERSE
\VERSE Tu exsurgens misereberis Sion, quia tempus miserendi ejus,  quia venit tempus : \EVERSE
\VERSE quoniam placuerunt servis tuis lapides ejus, et terræ ejus miserebuntur. \EVERSE
\VERSE Et timebunt gentes nomen tuum,  Domine, et omnes reges terræ gloriam tuam : \EVERSE
\VERSE quia ædificavit Dominus Sion, et videbitur in gloria sua. \EVERSE
\VERSE Respexit in orationem humiliumet non sprevit precem eorum. \EVERSE
\VERSE Scribantur hæc in generatione altera, et populus qui creabitur laudabit Dominum. \EVERSE
\VERSE Quia prospexit de excelso sancto suo ; Dominus de cælo in terram aspexit : \EVERSE
\VERSE ut audiret gemitus compeditorum ; ut solveret filios interemptorum : \EVERSE
\VERSE ut annuntient in Sion nomen Domini, et laudem ejus in Jerusalem : \EVERSE
\VERSE in conveniendo populos in unum,  et reges, ut serviant Domino. \EVERSE
\VERSE Respondit ei in via virtutis suæ :Paucitatem dierum meorum nuntia mihi : \EVERSE
\VERSE ne revoces me in dimidio dierum meorum, in generationem et generationem anni tui. \EVERSE
\VERSE Initio tu,  Domine,  terram fundasti, et opera manuum tuarum sunt cæli. \EVERSE
\VERSE Ipsi peribunt,  tu autem permanes ; et omnes sicut vestimentum veterascent.
Et sicut opertorium mutabis eos,  et mutabuntur ;  \EVERSE
\VERSE tu autem idem ipse es,  et anni tui non deficient.\VERSE Filii servorum tuorum habitabunt, et semen eorum in sæculum dirigetur.

}
\newcommand{\psalmcii}{
\VERSE Ipsi David.Benedic,  anima mea,  Domino, 
et omnia quæ intra me sunt nomini sancto ejus. \EVERSE
\VERSE Benedic,  anima mea,  Domino, et noli oblivisci omnes retributiones ejus. \EVERSE
\VERSE Qui propitiatur omnibus iniquitatibus tuis ; qui sanat omnes infirmitates tuas : \EVERSE
\VERSE qui redimit de interitu vitam tuam ; qui coronat te in misericordia et miserationibus : \EVERSE
\VERSE qui replet in bonis desiderium tuum ; renovabitur ut aquilæ juventus tua : \EVERSE
\VERSE faciens misericordias Dominus, et judicium omnibus injuriam patientibus. \EVERSE
\VERSE Notas fecit vias suas Moysi ; filiis Israël voluntates suas. \EVERSE
\VERSE Miserator et misericors Dominus :longanimis,  et multum misericors. \EVERSE
\VERSE Non in perpetuum irascetur, neque in æternum comminabitur. \EVERSE
\VERSE Non secundum peccata nostra fecit nobis, neque secundum iniquitates nostras retribuit nobis. \EVERSE
\VERSE Quoniam secundum altitudinem cæli a terra, corroboravit misericordiam suam super timentes se ;  \EVERSE
\VERSE quantum distat ortus ab occidente, longe fecit a nobis iniquitates nostras. \EVERSE
\VERSE Quomodo miseretur pater filiorum, misertus est Dominus timentibus se. \EVERSE
\VERSE Quoniam ipse cognovit figmentum nostrum ; recordatus est quoniam pulvis sumus. \EVERSE
\VERSE Homo,  sicut fœnum dies ejus ; tamquam flos agri,  sic efflorebit : \EVERSE
\VERSE quoniam spiritus pertransibit in illo,  et non subsistet, et non cognoscet amplius locum suum. \EVERSE
\VERSE Misericordia autem Domini ab æterno, et usque in æternum super timentes eum.
Et justitia illius in filios filiorum,  \EVERSE
\VERSE his qui servant testamentum ejus, et memores sunt mandatorum ipsius ad faciendum ea. \EVERSE
\VERSE Dominus in cælo paravit sedem suam, et regnum ipsius omnibus dominabitur. \EVERSE
\VERSE Benedicite Domino,  omnes angeli ejus :potentes virtute,  facientes verbum illius, 
ad audiendam vocem sermonum ejus. \EVERSE
\VERSE Benedicite Domino,  omnes virtutes ejus ; ministri ejus,  qui facitis voluntatem ejus. \EVERSE
\VERSE Benedicite Domino,  omnia opera ejus :in omni loco dominationis ejus, 
benedic,  anima mea,  Domino.

}
\newcommand{\psalmciii}{
\VERSE Ipsi David.Benedic,  anima mea,  Domino :
Domine Deus meus,  magnificatus es vehementer.
Confessionem et decorem induisti,  \EVERSE
\VERSE amictus lumine sicut vestimento.Extendens cælum sicut pellem,  \EVERSE
\VERSE qui tegis aquis superiora ejus :qui ponis nubem ascensum tuum ; 
qui ambulas super pennas ventorum : \EVERSE
\VERSE qui facis angelos tuos spiritus, et ministros tuos ignem urentem. \EVERSE
\VERSE Qui fundasti terram super stabilitatem suam :non inclinabitur in sæculum sæculi. \EVERSE
\VERSE Abyssus sicut vestimentum amictus ejus ; super montes stabunt aquæ. \EVERSE
\VERSE Ab increpatione tua fugient ; a voce tonitrui tui formidabunt. \EVERSE
\VERSE Ascendunt montes,  et descendunt campi, in locum quem fundasti eis. \EVERSE
\VERSE Terminum posuisti quem non transgredientur, neque convertentur operire terram. \EVERSE
\VERSE Qui emittis fontes in convallibus ; inter medium montium pertransibunt aquæ. \EVERSE
\VERSE Potabunt omnes bestiæ agri ; expectabunt onagri in siti sua. \EVERSE
\VERSE Super ea volucres cæli habitabunt ; de medio petrarum dabunt voces. \EVERSE
\VERSE Rigans montes de superioribus suis ; de fructu operum tuorum satiabitur terra : \EVERSE
\VERSE producens fœnum jumentis, et herbam servituti hominum, 
ut educas panem de terra,  \EVERSE
\VERSE et vinum lætificet cor hominis :ut exhilaret faciem in oleo, 
et panis cor hominis confirmet. \EVERSE
\VERSE Saturabuntur ligna campi, et cedri Libani quas plantavit : \EVERSE
\VERSE illic passeres nidificabunt :herodii domus dux est eorum. \EVERSE
\VERSE Montes excelsi cervis ; petra refugium herinaciis. \EVERSE
\VERSE Fecit lunam in tempora ; sol cognovit occasum suum. \EVERSE
\VERSE Posuisti tenebras,  et facta est nox ; in ipsa pertransibunt omnes bestiæ silvæ : \EVERSE
\VERSE catuli leonum rugientes ut rapiant, et quærant a Deo escam sibi. \EVERSE
\VERSE Ortus est sol,  et congregati sunt, et in cubilibus suis collocabuntur. \EVERSE
\VERSE Exibit homo ad opus suum, et ad operationem suam usque ad vesperum. \EVERSE
\VERSE Quam magnificata sunt opera tua,  Domine !omnia in sapientia fecisti ; 
impleta est terra possessione tua. \EVERSE
\VERSE Hoc mare magnum et spatiosum manibus ; illic reptilia quorum non est numerus :
animalia pusilla cum magnis. \EVERSE
\VERSE Illic naves pertransibunt ; draco iste quem formasti ad illudendum ei. \EVERSE
\VERSE Omnia a te expectantut des illis escam in tempore. \EVERSE
\VERSE Dante te illis,  colligent ; aperiente te manum tuam,  omnia implebuntur bonitate. \EVERSE
\VERSE Avertente autem te faciem,  turbabuntur ; auferes spiritum eorum,  et deficient, 
et in pulverem suum revertentur. \EVERSE
\VERSE Emittes spiritum tuum,  et creabuntur, et renovabis faciem terræ. \EVERSE
\VERSE Sit gloria Domini in sæculum ; lætabitur Dominus in operibus suis. \EVERSE
\VERSE Qui respicit terram,  et facit eam tremere ; qui tangit montes,  et fumigant. \EVERSE
\VERSE Cantabo Domino in vita mea ; psallam Deo meo quamdiu sum. \EVERSE
\VERSE Jucundum sit ei eloquium meum ; ego vero delectabor in Domino. \EVERSE
\VERSE Deficiant peccatores a terra, et iniqui,  ita ut non sint.
Benedic,  anima mea,  Domino.

}
\newcommand{\psalmciv}{
\VERSE Alleluja.Confitemini Domino,  et invocate nomen ejus ; 
annuntiate inter gentes opera ejus. \EVERSE
\VERSE Cantate ei,  et psallite ei ; narrate omnia mirabilia ejus. \EVERSE
\VERSE Laudamini in nomine sancto ejus ; lætetur cor quærentium Dominum. \EVERSE
\VERSE Quærite Dominum,  et confirmamini ; quærite faciem ejus semper. \EVERSE
\VERSE Mementote mirabilium ejus quæ fecit ; prodigia ejus,  et judicia oris ejus : \EVERSE
\VERSE semen Abraham servi ejus ; filii Jacob electi ejus. \EVERSE
\VERSE Ipse Dominus Deus noster ; in universa terra judicia ejus. \EVERSE
\VERSE Memor fuit in sæculum testamenti sui ; verbi quod mandavit in mille generationes : \EVERSE
\VERSE quod disposuit ad Abraham, et juramenti sui ad Isaac : \EVERSE
\VERSE et statuit illud Jacob in præceptum, et Israël in testamentum æternum,  \EVERSE
\VERSE dicens : Tibi dabo terram Chanaan, funiculum hæreditatis vestræ : \EVERSE
\VERSE cum essent numero brevi, paucissimi,  et incolæ ejus. \EVERSE
\VERSE Et pertransierunt de gente in gentem, et de regno ad populum alterum. \EVERSE
\VERSE Non reliquit hominem nocere eis :et corripuit pro eis reges. \EVERSE
\VERSE Nolite tangere christos meos, et in prophetis meis nolite malignari. \EVERSE
\VERSE Et vocavit famem super terram, et omne firmamentum panis contrivit. \EVERSE
\VERSE Misit ante eos virum :in servum venundatus est,  Joseph. \EVERSE
\VERSE Humiliaverunt in compedibus pedes ejus ; ferrum pertransiit animam ejus : \EVERSE
\VERSE donec veniret verbum ejus.Eloquium Domini inflammavit eum. \EVERSE
\VERSE Misit rex,  et solvit eum ; princeps populorum,  et dimisit eum. \EVERSE
\VERSE Constituit eum dominum domus suæ, et principem omnis possessionis suæ : \EVERSE
\VERSE ut erudiret principes ejus sicut semetipsum, et senes ejus prudentiam doceret. \EVERSE
\VERSE Et intravit Israël in Ægyptum, et Jacob accola fuit in terra Cham. \EVERSE
\VERSE Et auxit populum suum vehementer, et firmavit eum super inimicos ejus. \EVERSE
\VERSE Convertit cor eorum,  ut odirent populum ejus, et dolum facerent in servos ejus. \EVERSE
\VERSE Misit Moysen servum suum ; Aaron quem elegit ipsum. \EVERSE
\VERSE Posuit in eis verba signorum suorum, et prodigiorum in terra Cham. \EVERSE
\VERSE Misit tenebras,  et obscuravit ; et non exacerbavit sermones suos. \EVERSE
\VERSE Convertit aquas eorum in sanguinem, et occidit pisces eorum. \EVERSE
\VERSE Edidit terra eorum ranasin penetralibus regum ipsorum. \EVERSE
\VERSE Dixit,  et venit cœnomyia et ciniphesin omnibus finibus eorum. \EVERSE
\VERSE Posuit pluvias eorum grandinem :ignem comburentem in terra ipsorum. \EVERSE
\VERSE Et percussit vineas eorum,  et ficulneas eorum, et contrivit lignum finium eorum. \EVERSE
\VERSE Dixit,  et venit locusta, et bruchus cujus non erat numerus : \EVERSE
\VERSE et comedit omne fœnum in terra eorum, et comedit omnem fructum terræ eorum. \EVERSE
\VERSE Et percussit omne primogenitum in terra eorum, primitias omnis laboris eorum. \EVERSE
\VERSE Et eduxit eos cum argento et auro, et non erat in tribubus eorum infirmus. \EVERSE
\VERSE Lætata est Ægyptus in profectione eorum, quia incubuit timor eorum super eos. \EVERSE
\VERSE Expandit nubem in protectionem eorum, et ignem ut luceret eis per noctem. \EVERSE
\VERSE Petierunt,  et venit coturnix, et pane cæli saturavit eos. \EVERSE
\VERSE Dirupit petram,  et fluxerunt aquæ :abierunt in sicco flumina. \EVERSE
\VERSE Quoniam memor fuit verbi sancti sui, quod habuit ad Abraham puerum suum. \EVERSE
\VERSE Et eduxit populum suum in exsultatione, et electos suos in lætitia. \EVERSE
\VERSE Et dedit illis regiones gentium, et labores populorum possederunt : \EVERSE
\VERSE ut custodiant justificationes ejus, et legem ejus requirant.

}
\newcommand{\psalmcv}{
\VERSE Alleluja.Confitemini Domino,  quoniam bonus, 
quoniam in sæculum misericordia ejus. \EVERSE
\VERSE Quis loquetur potentias Domini ; auditas faciet omnes laudes ejus ? \EVERSE
\VERSE Beati qui custodiunt judicium, et faciunt justitiam in omni tempore. \EVERSE
\VERSE Memento nostri,  Domine,  in beneplacito populi tui ; visita nos in salutari tuo : \EVERSE
\VERSE ad videndum in bonitate electorum tuorum ; ad lætandum in lætitia gentis tuæ :
ut lauderis cum hæreditate tua. \EVERSE
\VERSE Peccavimus cum patribus nostris :injuste egimus ;  iniquitatem fecimus. \EVERSE
\VERSE Patres nostri in Ægypto non intellexerunt mirabilia tua ; non fuerunt memores multitudinis misericordiæ tuæ.
Et irritaverunt ascendentes in mare,  mare Rubrum ;  \EVERSE
\VERSE et salvavit eos propter nomen suum, ut notam faceret potentiam suam. \EVERSE
\VERSE Et increpuit mare Rubrum et exsiccatum est, et deduxit eos in abyssis sicut in deserto. \EVERSE
\VERSE Et salvavit eos de manu odientium, et redemit eos de manu inimici. \EVERSE
\VERSE Et operuit aqua tribulantes eos ; unus ex eis non remansit. \EVERSE
\VERSE Et crediderunt verbis ejus, et laudaverunt laudem ejus. \EVERSE
\VERSE Cito fecerunt ;  obliti sunt operum ejus :et non sustinuerunt consilium ejus. \EVERSE
\VERSE Et concupierunt concupiscentiam in deserto, et tentaverunt Deum in inaquoso. \EVERSE
\VERSE Et dedit eis petitionem ipsorum, et misit saturitatem in animas eorum. \EVERSE
\VERSE Et irritaverunt Moysen in castris ; Aaron,  sanctum Domini. \EVERSE
\VERSE Aperta est terra,  et deglutivit Dathan, et operuit super congregationem Abiron. \EVERSE
\VERSE Et exarsit ignis in synagoga eorum :flamma combussit peccatores. \EVERSE
\VERSE Et fecerunt vitulum in Horeb, et adoraverunt sculptile. \EVERSE
\VERSE Et mutaverunt gloriam suamin similitudinem vituli comedentis fœnum. \EVERSE
\VERSE Obliti sunt Deum qui salvavit eos ; qui fecit magnalia in Ægypto,  \EVERSE
\VERSE mirabilia in terra Cham, terribilia in mari Rubro. \EVERSE
\VERSE Et dixit ut disperderet eos, si non Moyses,  electus ejus, 
stetisset in confractione in conspectu ejus, 
ut averteret iram ejus,  ne disperderet eos. \EVERSE
\VERSE Et pro nihilo habuerunt terram desiderabilem ; non crediderunt verbo ejus. \EVERSE
\VERSE Et murmuraverunt in tabernaculis suis ; non exaudierunt vocem Domini. \EVERSE
\VERSE Et elevavit manum suam super eosut prosterneret eos in deserto : \EVERSE
\VERSE et ut dejiceret semen eorum in nationibus, et dispergeret eos in regionibus. \EVERSE
\VERSE Et initiati sunt Beelphegor, et comederunt sacrificia mortuorum. \EVERSE
\VERSE Et irritaverunt eum in adinventionibus suis, et multiplicata est in eis ruina. \EVERSE
\VERSE Et stetit Phinees,  et placavit, et cessavit quassatio. \EVERSE
\VERSE Et reputatum est ei in justitiam, in generationem et generationem usque in sempiternum. \EVERSE
\VERSE Et irritaverunt eum ad aquas contradictionis, et vexatus est Moyses propter eos : \EVERSE
\VERSE quia exacerbaverunt spiritum ejus, et distinxit in labiis suis. \EVERSE
\VERSE Non disperdiderunt gentesquas dixit Dominus illis : \EVERSE
\VERSE et commisti sunt inter gentes, et didicerunt opera eorum ;  \EVERSE
\VERSE et servierunt sculptilibus eorum, et factum est illis in scandalum. \EVERSE
\VERSE Et immolaverunt filios suos et filias suas dæmoniis.\VERSE Et effuderunt sanguinem innocentem, sanguinem filiorum suorum et filiarum suarum, 
quas sacrificaverunt sculptilibus Chanaan.
Et infecta est terra in sanguinibus,  \EVERSE
\VERSE et contaminata est in operibus eorum :et fornicati sunt in adinventionibus suis. \EVERSE
\VERSE Et iratus est furore Dominus in populum suum, et abominatus est hæreditatem suam. \EVERSE
\VERSE Et tradidit eos in manus gentium ; et dominati sunt eorum qui oderunt eos. \EVERSE
\VERSE Et tribulaverunt eos inimici eorum, et humiliati sunt sub manibus eorum ;  \EVERSE
\VERSE sæpe liberavit eos.Ipsi autem exacerbaverunt eum in consilio suo, 
et humiliati sunt in iniquitatibus suis. \EVERSE
\VERSE Et vidit cum tribularentur, et audivit orationem eorum. \EVERSE
\VERSE Et memor fuit testamenti sui, et pœnituit eum secundum multitudinem misericordiæ suæ : \EVERSE
\VERSE et dedit eos in misericordias, in conspectu omnium qui ceperant eos. \EVERSE
\VERSE Salvos nos fac,  Domine Deus noster, et congrega nos de nationibus :
ut confiteamur nomini sancto tuo, 
et gloriemur in laude tua. \EVERSE
\VERSE Benedictus Dominus Deus Israël,  a sæculo et usque in sæculum ; et dicet omnis populus : Fiat,  fiat.

}
\newcommand{\psalmcvi}{
\VERSE Alleluja.Confitemini Domino,  quoniam bonus, 
quoniam in sæculum misericordia ejus. \EVERSE
\VERSE Dicant qui redempti sunt a Domino, quos redemit de manu inimici, 
et de regionibus congregavit eos,  \EVERSE
\VERSE a solis ortu,  et occasu,  ab aquilone,  et mari.\VERSE Erraverunt in solitudine,  in inaquoso ; viam civitatis habitaculi non invenerunt. \EVERSE
\VERSE Esurientes et sitientes, anima eorum in ipsis defecit. \EVERSE
\VERSE Et clamaverunt ad Dominum cum tribularentur, et de necessitatibus eorum eripuit eos ;  \EVERSE
\VERSE et deduxit eos in viam rectam, ut irent in civitatem habitationis. \EVERSE
\VERSE Confiteantur Domino misericordiæ ejus, et mirabilia ejus filiis hominum. \EVERSE
\VERSE Quia satiavit animam inanem, et animam esurientem satiavit bonis. \EVERSE
\VERSE Sedentes in tenebris et umbra mortis ; vinctos in mendicitate et ferro. \EVERSE
\VERSE Quia exacerbaverunt eloquia Dei, et consilium Altissimi irritaverunt. \EVERSE
\VERSE Et humiliatum est in laboribus cor eorum ; infirmati sunt,  nec fuit qui adjuvaret. \EVERSE
\VERSE Et clamaverunt ad Dominum cum tribularentur ; et de necessitatibus eorum liberavit eos. \EVERSE
\VERSE Et eduxit eos de tenebris et umbra mortis, et vincula eorum dirupit. \EVERSE
\VERSE Confiteantur Domino misericordiæ ejus, et mirabilia ejus filiis hominum. \EVERSE
\VERSE Quia contrivit portas æreas, et vectes ferreos confregit. \EVERSE
\VERSE Suscepit eos de via iniquitatis eorum ; propter injustitias enim suas humiliati sunt. \EVERSE
\VERSE Omnem escam abominata est anima eorum, et appropinquaverunt usque ad portas mortis. \EVERSE
\VERSE Et clamaverunt ad Dominum cum tribularentur, et de necessitatibus eorum liberavit eos. \EVERSE
\VERSE Misit verbum suum,  et sanavit eos, et eripuit eos de interitionibus eorum. \EVERSE
\VERSE Confiteantur Domino misericordiæ ejus, et mirabilia ejus filiis hominum. \EVERSE
\VERSE Et sacrificent sacrificium laudis, et annuntient opera ejus in exsultatione. \EVERSE
\VERSE Qui descendunt mare in navibus, facientes operationem in aquis multis : \EVERSE
\VERSE ipsi viderunt opera Domini, et mirabilia ejus in profundo. \EVERSE
\VERSE Dixit,  et stetit spiritus procellæ, et exaltati sunt fluctus ejus. \EVERSE
\VERSE Ascendunt usque ad cælos,  et descendunt usque ad abyssos ; anima eorum in malis tabescebat. \EVERSE
\VERSE Turbati sunt,  et moti sunt sicut ebrius, et omnis sapientia eorum devorata est. \EVERSE
\VERSE Et clamaverunt ad Dominum cum tribularentur ; et de necessitatibus eorum eduxit eos. \EVERSE
\VERSE Et statuit procellam ejus in auram, et siluerunt fluctus ejus. \EVERSE
\VERSE Et lætati sunt quia siluerunt ; et deduxit eos in portum voluntatis eorum. \EVERSE
\VERSE Confiteantur Domino misericordiæ ejus, et mirabilia ejus filiis hominum. \EVERSE
\VERSE Et exaltent eum in ecclesia plebis, et in cathedra seniorum laudent eum. \EVERSE
\VERSE Posuit flumina in desertum, et exitus aquarum in sitim ;  \EVERSE
\VERSE terram fructiferam in salsuginem, a malitia inhabitantium in ea. \EVERSE
\VERSE Posuit desertum in stagna aquarum, et terram sine aqua in exitus aquarum. \EVERSE
\VERSE Et collocavit illic esurientes, et constituerunt civitatem habitationis : \EVERSE
\VERSE et seminaverunt agros et plantaverunt vineas, et fecerunt fructum nativitatis. \EVERSE
\VERSE Et benedixit eis,  et multiplicati sunt nimis ; et jumenta eorum non minoravit. \EVERSE
\VERSE Et pauci facti sunt et vexati sunt, a tribulatione malorum et dolore. \EVERSE
\VERSE Effusa est contemptio super principes :et errare fecit eos in invio,  et non in via. \EVERSE
\VERSE Et adjuvit pauperem de inopia, et posuit sicut oves familias. \EVERSE
\VERSE Videbunt recti,  et lætabuntur ; et omnis iniquitas oppilabit os suum. \EVERSE
\VERSE Quis sapiens,  et custodiet hæc, et intelliget misericordias Domini ?

}
\newcommand{\psalmcvii}{
\VERSE Canticum Psalmi,  ipsi David.\VERSE Paratum cor meum,  Deus,  paratum cor meum ; cantabo,  et psallam in gloria mea. \EVERSE
\VERSE Exsurge,  gloria mea ;  exsurge,  psalterium et cithara ; exsurgam diluculo. \EVERSE
\VERSE Confitebor tibi in populis,  Domine, et psallam tibi in nationibus : \EVERSE
\VERSE quia magna est super cælos misericordia tua, et usque ad nubes veritas tua. \EVERSE
\VERSE Exaltare super cælos,  Deus, et super omnem terram gloria tua : \EVERSE
\VERSE ut liberentur dilecti tui.Salvum fac dextera tua,  et exaudi me. \EVERSE
\VERSE Deus locutus est in sancto suo :Exsultabo,  et dividam Sichimam ; 
et convallem tabernaculorum dimetiar. \EVERSE
\VERSE Meus est Galaad,  et meus est Manasses, et Ephraim susceptio capitis mei.
Juda rex meus ;  10 Moab lebes spei meæ :
in Idumæam extendam calceamentum meum ; 
mihi alienigenæ amici facti sunt. \EVERSE
\VERSE Quis deducet me in civitatem munitam ?quis deducet me usque in Idumæam ? \EVERSE
\VERSE nonne tu,  Deus,  qui repulisti nos ?et non exibis,  Deus,  in virtutibus nostris ? \EVERSE
\VERSE Da nobis auxilium de tribulatione, quia vana salus hominis. \EVERSE
\VERSE In Deo faciemus virtutem ; et ipse ad nihilum deducet inimicos nostros.

}
\newcommand{\psalmcviii}{
\VERSE In finem. Psalmus David.\VERSE Deus,  laudem meam ne tacueris, quia os peccatoris et os dolosi super me apertum est. \EVERSE
\VERSE Locuti sunt adversum me lingua dolosa, et sermonibus odii circumdederunt me :
et expugnaverunt me gratis. \EVERSE
\VERSE Pro eo ut me diligerent,  detrahebant mihi ; ego autem orabam. \EVERSE
\VERSE Et posuerunt adversum me mala pro bonis, et odium pro dilectione mea. \EVERSE
\VERSE Constitue super eum peccatorem, et diabolus stet a dextris ejus. \EVERSE
\VERSE Cum judicatur,  exeat condemnatus ; et oratio ejus fiat in peccatum. \EVERSE
\VERSE Fiant dies ejus pauci, et episcopatum ejus accipiat alter. \EVERSE
\VERSE Fiant filii ejus orphani, et uxor ejus vidua. \EVERSE
\VERSE Nutantes transferantur filii ejus et mendicent, et ejiciantur de habitationibus suis. \EVERSE
\VERSE Scrutetur fœnerator omnem substantiam ejus, et diripiant alieni labores ejus. \EVERSE
\VERSE Non sit illi adjutor, nec sit qui misereatur pupillis ejus. \EVERSE
\VERSE Fiant nati ejus in interitum ; in generatione una deleatur nomen ejus. \EVERSE
\VERSE In memoriam redeat iniquitas patrum ejus in conspectu Domini, et peccatum matris ejus non deleatur. \EVERSE
\VERSE Fiant contra Dominum semper, et dispereat de terra memoria eorum : \EVERSE
\VERSE pro eo quod non est recordatus facere misericordiam, \VERSE et persecutus est hominem inopem et mendicum, et compunctum corde,  mortificare. \EVERSE
\VERSE Et dilexit maledictionem,  et veniet ei ; et noluit benedictionem,  et elongabitur ab eo.
Et induit maledictionem sicut vestimentum ; 
et intravit sicut aqua in interiora ejus, 
et sicut oleum in ossibus ejus. \EVERSE
\VERSE Fiat ei sicut vestimentum quo operitur, et sicut zona qua semper præcingitur. \EVERSE
\VERSE Hoc opus eorum qui detrahunt mihi apud Dominum, et qui loquuntur mala adversus animam meam. \EVERSE
\VERSE Et tu,  Domine,  Domine,  fac mecum propter nomen tuum, quia suavis est misericordia tua. \EVERSE
\VERSE Libera me,  quia egenus et pauper ego sum, et cor meum conturbatum est intra me. \EVERSE
\VERSE Sicut umbra cum declinat ablatus sum, et excussus sum sicut locustæ. \EVERSE
\VERSE Genua mea infirmata sunt a jejunio, et caro mea immutata est propter oleum. \EVERSE
\VERSE Et ego factus sum opprobrium illis ; viderunt me,  et moverunt capita sua. \EVERSE
\VERSE Adjuva me,  Domine Deus meus ; salvum me fac secundum misericordiam tuam. \EVERSE
\VERSE Et sciant quia manus tua hæc, et tu,  Domine,  fecisti eam. \EVERSE
\VERSE Maledicent illi,  et tu benedices :qui insurgunt in me confundantur ; 
servus autem tuus lætabitur. \EVERSE
\VERSE Induantur qui detrahunt mihi pudore, et operiantur sicut diploide confusione sua. \EVERSE
\VERSE Confitebor Domino nimis in ore meo, et in medio multorum laudabo eum : \EVERSE
\VERSE quia astitit a dextris pauperis, ut salvam faceret a persequentibus animam meam.

}
\newcommand{\psalmcix}{
\VERSE Psalmus David.Dixit Dominus Domino meo :
Sede a dextris meis, 
donec ponam inimicos tuos scabellum pedum tuorum. \EVERSE
\VERSE Virgam virtutis tuæ emittet Dominus ex Sion :dominare in medio inimicorum tuorum. \EVERSE
\VERSE Tecum principium in die virtutis tuæin splendoribus sanctorum :
ex utero,  ante luciferum,  genui te. \EVERSE
\VERSE Juravit Dominus,  et non pœnitebit eum :Tu es sacerdos in æternum
secundum ordinem Melchisedech. \EVERSE
\VERSE Dominus a dextris tuis ; confregit in die iræ suæ reges. \EVERSE
\VERSE Judicabit in nationibus,  implebit ruinas ; conquassabit capita in terra multorum. \EVERSE
\VERSE De torrente in via bibet ; propterea exaltabit caput.

}
\newcommand{\psalmcx}{
\VERSE Alleluja.Confitebor tibi,  Domine,  in toto corde meo, 
in consilio justorum,  et congregatione. \EVERSE
\VERSE Magna opera Domini :exquisita in omnes voluntates ejus. \EVERSE
\VERSE Confessio et magnificentia opus ejus, et justitia ejus manet in sæculum sæculi. \EVERSE
\VERSE Memoriam fecit mirabilium suorum, misericors et miserator Dominus. \EVERSE
\VERSE Escam dedit timentibus se ; memor erit in sæculum testamenti sui. \EVERSE
\VERSE Virtutem operum suorum annuntiabit populo suo, \VERSE ut det illis hæreditatem gentium.Opera manuum ejus veritas et judicium. \EVERSE
\VERSE Fidelia omnia mandata ejus, confirmata in sæculum sæculi, 
facta in veritate et æquitate. \EVERSE
\VERSE Redemptionem misit populo suo ; mandavit in æternum testamentum suum.
Sanctum et terribile nomen ejus. \EVERSE
\VERSE Initium sapientiæ timor Domini ; intellectus bonus omnibus facientibus eum :
laudatio ejus manet in sæculum sæculi.

}
\newcommand{\psalmcxi}{
\VERSE Alleluja,  reversionis Aggæi et Zachariæ.Beatus vir qui timet Dominum :
in mandatis ejus volet nimis. \EVERSE
\VERSE Potens in terra erit semen ejus ; generatio rectorum benedicetur. \EVERSE
\VERSE Gloria et divitiæ in domo ejus, et justitia ejus manet in sæculum sæculi. \EVERSE
\VERSE Exortum est in tenebris lumen rectis :misericors,  et miserator,  et justus. \EVERSE
\VERSE Jucundus homo qui miseretur et commodat ; disponet sermones suos in judicio : \EVERSE
\VERSE quia in æternum non commovebitur.\VERSE In memoria æterna erit justus ; ab auditione mala non timebit.
Paratum cor ejus sperare in Domino,  \EVERSE
\VERSE confirmatum est cor ejus ; non commovebitur donec despiciat inimicos suos. \EVERSE
\VERSE Dispersit,  dedit pauperibus ; justitia ejus manet in sæculum sæculi :
cornu ejus exaltabitur in gloria. \EVERSE
\VERSE Peccator videbit,  et irascetur ; dentibus suis fremet et tabescet :
desiderium peccatorum peribit.

}
\newcommand{\psalmcxii}{
\VERSE Alleluja.Laudate,  pueri,  Dominum ; 
laudate nomen Domini. \EVERSE
\VERSE Sit nomen Domini benedictumex hoc nunc et usque in sæculum. \EVERSE
\VERSE A solis ortu usque ad occasumlaudabile nomen Domini. \EVERSE
\VERSE Excelsus super omnes gentes Dominus, et super cælos gloria ejus. \EVERSE
\VERSE Quis sicut Dominus Deus noster,  qui in altis habitat, \VERSE et humilia respicit in cælo et in terra ?\VERSE Suscitans a terra inopem, et de stercore erigens pauperem : \EVERSE
\VERSE ut collocet eum cum principibus, cum principibus populi sui. \EVERSE
\VERSE Qui habitare facit sterilem in domo, matrem filiorum lætantem.

}
\newcommand{\psalmcxiii}{
\VERSE Alleluja.In exitu Israël de Ægypto, 
domus Jacob de populo barbaro,  \EVERSE
\VERSE facta est Judæa sanctificatio ejus ; Israël potestas ejus. \EVERSE
\VERSE Mare vidit,  et fugit ; Jordanis conversus est retrorsum. \EVERSE
\VERSE Montes exsultaverunt ut arietes, et colles sicut agni ovium. \EVERSE
\VERSE Quid est tibi,  mare,  quod fugisti ?et tu,  Jordanis,  quia conversus es retrorsum ? \EVERSE
\VERSE montes,  exsultastis sicut arietes ?et colles,  sicut agni ovium ? \EVERSE
\VERSE A facie Domini mota est terra, a facie Dei Jacob : \EVERSE
\VERSE qui convertit petram in stagna aquarum, et rupem in fontes aquarum. \EVERSE
\VERSE Non nobis,  Domine,  non nobis, sed nomini tuo da gloriam : \EVERSE
\VERSE super misericordia tua et veritate tua ; nequando dicant gentes :
Ubi est Deus eorum ? \EVERSE
\VERSE Deus autem noster in cælo ; omnia quæcumque voluit fecit. \EVERSE
\VERSE Simulacra gentium argentum et aurum, opera manuum hominum. \EVERSE
\VERSE Os habent,  et non loquentur ; oculos habent,  et non videbunt. \EVERSE
\VERSE Aures habent,  et non audient ; nares habent,  et non odorabunt. \EVERSE
\VERSE Manus habent,  et non palpabunt ; pedes habent,  et non ambulabunt ; 
non clamabunt in gutture suo. \EVERSE
\VERSE Similes illis fiant qui faciunt ea, et omnes qui confidunt in eis. \EVERSE
\VERSE Domus Israël speravit in Domino ; adjutor eorum et protector eorum est. \EVERSE
\VERSE Domus Aaron speravit in Domino ; adjutor eorum et protector eorum est. \EVERSE
\VERSE Qui timent Dominum speraverunt in Domino ;  adjutor eorum et protector eorum est.\VERSE Dominus memor fuit nostri, et benedixit nobis.
Benedixit domui Israël ; 
benedixit domui Aaron. \EVERSE
\VERSE Benedixit omnibus qui timent Dominum, pusillis cum majoribus. \EVERSE
\VERSE Adjiciat Dominus super vos, super vos et super filios vestros. \EVERSE
\VERSE Benedicti vos a Domino, qui fecit cælum et terram. \EVERSE
\VERSE Cælum cæli Domino ; terram autem dedit filiis hominum. \EVERSE
\VERSE Non mortui laudabunt te,  Domine, neque omnes qui descendunt in infernum : \EVERSE
\VERSE sed nos qui vivimus,  benedicimus Domino, ex hoc nunc et usque in sæculum.

}
\newcommand{\psalmcxiv}{
\VERSE Alleluja.Dilexi,  quoniam exaudiet Dominus
vocem orationis meæ. \EVERSE
\VERSE Quia inclinavit aurem suam mihi, et in diebus meis invocabo. \EVERSE
\VERSE Circumdederunt me dolores mortis ; et pericula inferni invenerunt me.
Tribulationem et dolorem inveni,  \EVERSE
\VERSE et nomen Domini invocavi :o Domine,  libera animam meam. \EVERSE
\VERSE Misericors Dominus et justus, et Deus noster miseretur. \EVERSE
\VERSE Custodiens parvulos Dominus ; humiliatus sum,  et liberavit me. \EVERSE
\VERSE Convertere,  anima mea,  in requiem tuam, quia Dominus benefecit tibi : \EVERSE
\VERSE quia eripuit animam meam de morte, oculos meos a lacrimis, 
pedes meos a lapsu. \EVERSE
\VERSE Placebo Domino in regione vivorum.}
\newcommand{\psalmcxv}{
\VERSE Alleluja.Credidi,  propter quod locutus sum ; 
ego autem humiliatus sum nimis. \EVERSE
\VERSE Ego dixi in excessu meo :Omnis homo mendax. \EVERSE
\VERSE Quid retribuam Dominopro omnibus quæ retribuit mihi ? \EVERSE
\VERSE Calicem salutaris accipiam, et nomen Domini invocabo. \EVERSE
\VERSE Vota mea Domino reddamcoram omni populo ejus. \EVERSE
\VERSE Pretiosa in conspectu Dominimors sanctorum ejus. \EVERSE
\VERSE O Domine,  quia ego servus tuus ; ego servus tuus,  et filius ancillæ tuæ.
Dirupisti vincula mea : \EVERSE
\VERSE tibi sacrificabo hostiam laudis, et nomen Domini invocabo. \EVERSE
\VERSE Vota mea Domino reddamin conspectu omnis populi ejus ;  \EVERSE
\VERSE in atriis domus Domini, in medio tui,  Jerusalem.

}
\newcommand{\psalmcxvi}{
\VERSE Alleluja.Laudate Dominum,  omnes gentes ; 
laudate eum,  omnes populi. \EVERSE
\VERSE Quoniam confirmata est super nos misericordia ejus, et veritas Domini manet in æternum.

}
\newcommand{\psalmcxvii}{
\VERSE Alleluja. Confitemini Domino,  quoniam bonus,  quoniam in sæculum misericordia ejus. \EVERSE
\VERSE Dicat nunc Israël : Quoniam bonus, quoniam in sæculum misericordia ejus. \EVERSE
\VERSE Dicat nunc domus Aaron :Quoniam in sæculum misericordia ejus. \EVERSE
\VERSE Dicant nunc qui timent Dominum :Quoniam in sæculum misericordia ejus. \EVERSE
\VERSE De tribulatione invocavi Dominum, et exaudivit me in latitudine Dominus. \EVERSE
\VERSE Dominus mihi adjutor ; non timebo quid faciat mihi homo. \EVERSE
\VERSE Dominus mihi adjutor, et ego despiciam inimicos meos. \EVERSE
\VERSE Bonum est confidere in Domino, quam confidere in homine. \EVERSE
\VERSE Bonum est sperare in Domino, quam sperare in principibus. \EVERSE
\VERSE Omnes gentes circuierunt me, et in nomine Domini,  quia ultus sum in eos. \EVERSE
\VERSE Circumdantes circumdederunt me, et in nomine Domini,  quia ultus sum in eos. \EVERSE
\VERSE Circumdederunt me sicut apes, et exarserunt sicut ignis in spinis : et in nomine Domini,  quia ultus sum in eos. \EVERSE
\VERSE Impulsus eversus sum,  ut caderem, et Dominus suscepit me. \EVERSE
\VERSE Fortitudo mea et laus mea Dominus, et factus est mihi in salutem. \EVERSE
\VERSE Vox exsultationis et salutisin tabernaculis justorum. \EVERSE
\VERSE Dextera Domini fecit virtutem ; dextera Domini exaltavit me : dextera Domini fecit virtutem. \EVERSE
\VERSE Non moriar,  sed vivam, et narrabo opera Domini. \EVERSE
\VERSE Castigans castigavit me Dominus, et morti non tradidit me. \EVERSE
\VERSE Aperite mihi portas justitiæ :ingressus in eas confitebor Domino. \EVERSE
\VERSE Hæc porta Domini :justi intrabunt in eam. \EVERSE
\VERSE Confitebor tibi quoniam exaudisti me, et factus es mihi in salutem. \EVERSE
\VERSE Lapidem quem reprobaverunt ædificantes, hic factus est in caput anguli. \EVERSE
\VERSE A Domino factum est istud, et est mirabile in oculis nostris. \EVERSE
\VERSE Hæc est dies quam fecit Dominus ; exsultemus,  et lætemur in ea. \EVERSE
\VERSE O Domine,  salvum me fac ; o Domine,  bene prosperare. \EVERSE
\VERSE Benedictus qui venit in nomine Domini :benediximus vobis de domo Domini. \EVERSE
\VERSE Deus Dominus,  et illuxit nobis.Constituite diem solemnem in condensis,  usque ad cornu altaris. \EVERSE
\VERSE Deus meus es tu,  et confitebor tibi ; Deus meus es tu,  et exaltabo te. Confitebor tibi quoniam exaudisti me,  et factus es mihi in salutem. \EVERSE
\VERSE Confitemini Domino,  quoniam bonus, quoniam in sæculum misericordia ejus.

}
\newcommand{\psalmcxviiia}{
\VERSE Alleluja. Aleph. Beati immaculati in via,  qui ambulant in lege Domini. \EVERSE
\VERSE Beati qui scrutantur testimonia ejus ; in toto corde exquirunt eum. \EVERSE
\VERSE Non enim qui operantur iniquitatemin viis ejus ambulaverunt. \EVERSE
\VERSE Tu mandasti mandata tuacustodiri nimis. \EVERSE
\VERSE Utinam dirigantur viæ meæad custodiendas justificationes tuas. \EVERSE
\VERSE Tunc non confundar, cum perspexero in omnibus mandatis tuis. \EVERSE
\VERSE Confitebor tibi in directione cordis, in eo quod didici judicia justitiæ tuæ. \EVERSE
\VERSE Justificationes tuas custodiam ; non me derelinquas usquequaque. \EVERSE
\VERSE Beth. In quo corrigit adolescentior viam suam ?in custodiendo sermones tuos. \EVERSE
\VERSE In toto corde meo exquisivi te ; ne repellas me a mandatis tuis. \EVERSE
\VERSE In corde meo abscondi eloquia tua, ut non peccem tibi. \EVERSE
\VERSE Benedictus es,  Domine ; doce me justificationes tuas. \EVERSE
\VERSE In labiis meis pronuntiavi omnia judicia oris tui.\VERSE In via testimoniorum tuorum delectatus sum, sicut in omnibus divitiis. \EVERSE
\VERSE In mandatis tuis exercebor, et considerabo vias tuas. \EVERSE
\VERSE In justificationibus tuis meditabor :non obliviscar sermones tuos. \EVERSE
\VERSE Ghimel. Retribue servo tuo,  vivifica me, et custodiam sermones tuos. \EVERSE
\VERSE Revela oculos meos, et considerabo mirabilia de lege tua. \EVERSE
\VERSE Incola ego sum in terra :non abscondas a me mandata tua. \EVERSE
\VERSE Concupivit anima meadesiderare justificationes tuas in omni tempore. \EVERSE
\VERSE Increpasti superbos ; maledicti qui declinant a mandatis tuis. \EVERSE
\VERSE Aufer a me opprobrium et contemptum, quia testimonia tua exquisivi. \EVERSE
\VERSE Etenim sederunt principes,  et adversum me loquebantur ; servus autem tuus exercebatur in justificationibus tuis. \EVERSE
\VERSE Nam et testimonia tua meditatio mea est, et consilium meum justificationes tuæ. \EVERSE
\VERSE Daleth. Adhæsit pavimento anima mea :vivifica me secundum verbum tuum. \EVERSE
\VERSE Vias meas enuntiavi,  et exaudisti me ; doce me justificationes tuas. \EVERSE
\VERSE Viam justificationum tuarum instrue me, et exercebor in mirabilibus tuis. \EVERSE
\VERSE Dormitavit anima mea præ tædio :confirma me in verbis tuis. \EVERSE
\VERSE Viam iniquitatis amove a me, et de lege tua miserere mei. \EVERSE
\VERSE Viam veritatis elegi ; judicia tua non sum oblitus. \EVERSE
\VERSE Adhæsi testimoniis tuis,  Domine ; noli me confundere. \EVERSE
\VERSE Viam mandatorum tuorum cucurri, cum dilatasti cor meum.

}
\newcommand{\psalmcxviiib}{
\VERSE He. Legem pone mihi,  Domine,  viam justificationum tuarum, et exquiram eam semper. \EVERSE
\VERSE Da mihi intellectum,  et scrutabor legem tuam, et custodiam illam in toto corde meo. \EVERSE
\VERSE Deduc me in semitam mandatorum tuorum, quia ipsam volui. \EVERSE
\VERSE Inclina cor meum in testimonia tua, et non in avaritiam. \EVERSE
\VERSE Averte oculos meos,  ne videant vanitatem ; in via tua vivifica me. \EVERSE
\VERSE Statue servo tuo eloquium tuumin timore tuo. \EVERSE
\VERSE Amputa opprobrium meum quod suspicatus sum, quia judicia tua jucunda. \EVERSE
\VERSE Ecce concupivi mandata tua :in æquitate tua vivifica me. \EVERSE
\VERSE Vau. Et veniat super me misericordia tua,  Domine ; salutare tuum secundum eloquium tuum. \EVERSE
\VERSE Et respondebo exprobrantibus mihi verbum, quia speravi in sermonibus tuis. \EVERSE
\VERSE Et ne auferas de ore meo verbum veritatis usquequaque, quia in judiciis tuis supersperavi. \EVERSE
\VERSE Et custodiam legem tuam semper, in sæculum et in sæculum sæculi. \EVERSE
\VERSE Et ambulabam in latitudine, quia mandata tua exquisivi. \EVERSE
\VERSE Et loquebar in testimoniis tuis in conspectu regum, et non confundebar. \EVERSE
\VERSE Et meditabar in mandatis tuis, quæ dilexi. \EVERSE
\VERSE Et levavi manus meas ad mandata tua,  quæ dilexi, et exercebar in justificationibus tuis. \EVERSE
\VERSE Zain. Memor esto verbi tui servo tuo, in quo mihi spem dedisti. \EVERSE
\VERSE Hæc me consolata est in humilitate mea, quia eloquium tuum vivificavit me. \EVERSE
\VERSE Superbi inique agebant usquequaque ; a lege autem tua non declinavi. \EVERSE
\VERSE Memor fui judiciorum tuorum a sæculo,  Domine, et consolatus sum. \EVERSE
\VERSE Defectio tenuit me, pro peccatoribus derelinquentibus legem tuam. \EVERSE
\VERSE Cantabiles mihi erant justificationes tuæin loco peregrinationis meæ. \EVERSE
\VERSE Memor fui nocte nominis tui,  Domine, et custodivi legem tuam. \EVERSE
\VERSE Hæc facta est mihi, quia justificationes tuas exquisivi.

}
\newcommand{\psalmcxviiic}{
\VERSE Heth. Portio mea,  Domine, dixi custodire legem tuam. \EVERSE
\VERSE Deprecatus sum faciem tuam in toto corde meo ; miserere mei secundum eloquium tuum. \EVERSE
\VERSE Cogitavi vias meas, et converti pedes meos in testimonia tua. \EVERSE
\VERSE Paratus sum,  et non sum turbatus, ut custodiam mandata tua. \EVERSE
\VERSE Funes peccatorum circumplexi sunt me, et legem tuam non sum oblitus. \EVERSE
\VERSE Media nocte surgebam ad confitendum tibi, super judicia justificationis tuæ. \EVERSE
\VERSE Particeps ego sum omnium timentium te, et custodientium mandata tua. \EVERSE
\VERSE Misericordia tua,  Domine,  plena est terra ; justificationes tuas doce me. \EVERSE
\VERSE Teth. Bonitatem fecisti cum servo tuo,  Domine, secundum verbum tuum. \EVERSE
\VERSE Bonitatem,  et disciplinam,  et scientiam doce me, quia mandatis tuis credidi. \EVERSE
\VERSE Priusquam humiliarer ego deliqui :propterea eloquium tuum custodivi. \EVERSE
\VERSE Bonus es tu,  et in bonitate tuadoce me justificationes tuas. \EVERSE
\VERSE Multiplicata est super me iniquitas superborum ; ego autem in toto corde meo scrutabor mandata tua. \EVERSE
\VERSE Coagulatum est sicut lac cor eorum ; ego vero legem tuam meditatus sum. \EVERSE
\VERSE Bonum mihi quia humiliasti me, ut discam justificationes tuas. \EVERSE
\VERSE Bonum mihi lex oris tui, super millia auri et argenti. \EVERSE
\VERSE Jod. Manus tuæ fecerunt me,  et plasmaverunt me :da mihi intellectum,  et discam mandata tua. \EVERSE
\VERSE Qui timent te videbunt me et lætabuntur, quia in verba tua supersperavi. \EVERSE
\VERSE Cognovi,  Domine,  quia æquitas judicia tua, et in veritate tua humiliasti me. \EVERSE
\VERSE Fiat misericordia tua ut consoletur me, secundum eloquium tuum servo tuo. \EVERSE
\VERSE Veniant mihi miserationes tuæ,  et vivam, quia lex tua meditatio mea est. \EVERSE
\VERSE Confundantur superbi,  quia injuste iniquitatem fecerunt in me ; ego autem exercebor in mandatis tuis. \EVERSE
\VERSE Convertantur mihi timentes te, et qui noverunt testimonia tua. \EVERSE
\VERSE Fiat cor meum immaculatum in justificationibus tuis, ut non confundar.

}
\newcommand{\psalmcxviiid}{
\VERSE Caph. Defecit in salutare tuum anima mea, et in verbum tuum supersperavi. \EVERSE
\VERSE Defecerunt oculi mei in eloquium tuum, dicentes : Quando consolaberis me ? \EVERSE
\VERSE Quia factus sum sicut uter in pruina ; justificationes tuas non sum oblitus. \EVERSE
\VERSE Quot sunt dies servi tui ?quando facies de persequentibus me judicium ? \EVERSE
\VERSE Narraverunt mihi iniqui fabulationes, sed non ut lex tua. \EVERSE
\VERSE Omnia mandata tua veritas :inique persecuti sunt me,  adjuva me. \EVERSE
\VERSE Paulominus consummaverunt me in terra ; ego autem non dereliqui mandata tua. \EVERSE
\VERSE Secundum misericordiam tuam vivifica me, et custodiam testimonia oris tui. \EVERSE
\VERSE Lamed. In æternum,  Domine, verbum tuum permanet in cælo. \EVERSE
\VERSE In generationem et generationem veritas tua ; fundasti terram,  et permanet. \EVERSE
\VERSE Ordinatione tua perseverat dies, quoniam omnia serviunt tibi. \EVERSE
\VERSE Nisi quod lex tua meditatio mea est, tunc forte periissem in humilitate mea. \EVERSE
\VERSE In æternum non obliviscar justificationes tuas, quia in ipsis vivificasti me. \EVERSE
\VERSE Tuus sum ego ;  salvum me fac :quoniam justificationes tuas exquisivi. \EVERSE
\VERSE Me exspectaverunt peccatores ut perderent me ; testimonia tua intellexi. \EVERSE
\VERSE Omnis consummationis vidi finem, latum mandatum tuum nimis. \EVERSE
\VERSE Mem. Quomodo dilexi legem tuam,  Domine !tota die meditatio mea est. \EVERSE
\VERSE Super inimicos meos prudentem me fecisti mandato tuo, quia in æternum mihi est. \EVERSE
\VERSE Super omnes docentes me intellexi, quia testimonia tua meditatio mea est. \EVERSE
\VERSE Super senes intellexi, quia mandata tua quæsivi. \EVERSE
\VERSE Ab omni via mala prohibui pedes meos, ut custodiam verba tua. \EVERSE
\VERSE A judiciis tuis non declinavi, quia tu legem posuisti mihi. \EVERSE
\VERSE Quam dulcia faucibus meis eloquia tua !super mel ori meo. \EVERSE
\VERSE A mandatis tuis intellexi ; propterea odivi omnem viam iniquitatis.

}
\newcommand{\psalmcxviiie}{
\VERSE Nun. Lucerna pedibus meis verbum tuum, et lumen semitis meis. \EVERSE
\VERSE Juravi et statuicustodire judicia justitiæ tuæ. \EVERSE
\VERSE Humiliatus sum usquequaque,  Domine ; vivifica me secundum verbum tuum. \EVERSE
\VERSE Voluntaria oris mei beneplacita fac,  Domine, et judicia tua doce me. \EVERSE
\VERSE Anima mea in manibus meis semper, et legem tuam non sum oblitus. \EVERSE
\VERSE Posuerunt peccatores laqueum mihi, et de mandatis tuis non erravi. \EVERSE
\VERSE Hæreditate acquisivi testimonia tua in æternum, quia exsultatio cordis mei sunt. \EVERSE
\VERSE Inclinavi cor meum ad faciendas justificationes tuas in æternum, propter retributionem. \EVERSE
\VERSE Samech. Iniquos odio habui, et legem tuam dilexi. \EVERSE
\VERSE Adjutor et susceptor meus es tu, et in verbum tuum supersperavi. \EVERSE
\VERSE Declinate a me,  maligni, et scrutabor mandata Dei mei. \EVERSE
\VERSE Suscipe me secundum eloquium tuum,  et vivam, et non confundas me ab exspectatione mea. \EVERSE
\VERSE Adjuva me,  et salvus ero, et meditabor in justificationibus tuis semper. \EVERSE
\VERSE Sprevisti omnes discedentes a judiciis tuis, quia injusta cogitatio eorum. \EVERSE
\VERSE Prævaricantes reputavi omnes peccatores terræ ; ideo dilexi testimonia tua. \EVERSE
\VERSE Confige timore tuo carnes meas ; a judiciis enim tuis timui. \EVERSE
\VERSE Ain. Feci judicium et justitiam :non tradas me calumniantibus me. \EVERSE
\VERSE Suscipe servum tuum in bonum :non calumnientur me superbi. \EVERSE
\VERSE Oculi mei defecerunt in salutare tuum, et in eloquium justitiæ tuæ. \EVERSE
\VERSE Fac cum servo tuo secundum misericordiam tuam, et justificationes tuas doce me. \EVERSE
\VERSE Servus tuus sum ego : da mihi intellectum, ut sciam testimonia tua. \EVERSE
\VERSE Tempus faciendi,  Domine :dissipaverunt legem tuam. \EVERSE
\VERSE Ideo dilexi mandata tuasuper aurum et topazion. \EVERSE
\VERSE Propterea ad omnia mandata tua dirigebar ; omnem viam iniquam odio habui.

}
\newcommand{\psalmcxviiif}{
\VERSE Phe. Mirabilia testimonia tua :ideo scrutata est ea anima mea. \EVERSE
\VERSE Declaratio sermonum tuorum illuminat, et intellectum dat parvulis. \EVERSE
\VERSE Os meum aperui,  et attraxi spiritum :quia mandata tua desiderabam. \EVERSE
\VERSE Aspice in me,  et miserere mei, secundum judicium diligentium nomen tuum. \EVERSE
\VERSE Gressus meos dirige secundum eloquium tuum, et non dominetur mei omnis injustitia. \EVERSE
\VERSE Redime me a calumniis hominumut custodiam mandata tua. \EVERSE
\VERSE Faciem tuam illumina super servum tuum, et doce me justificationes tuas. \EVERSE
\VERSE Exitus aquarum deduxerunt oculi mei, quia non custodierunt legem tuam. \EVERSE
\VERSE Sade. Justus es,  Domine, et rectum judicium tuum. \EVERSE
\VERSE Mandasti justitiam testimonia tua, et veritatem tuam nimis. \EVERSE
\VERSE Tabescere me fecit zelus meus, quia obliti sunt verba tua inimici mei. \EVERSE
\VERSE Ignitum eloquium tuum vehementer, et servus tuus dilexit illud. \EVERSE
\VERSE Adolescentulus sum ego et contemptus ; justificationes tuas non sum oblitus. \EVERSE
\VERSE Justitia tua,  justitia in æternum, et lex tua veritas. \EVERSE
\VERSE Tribulatio et angustia invenerunt me ; mandata tua meditatio mea est. \EVERSE
\VERSE Æquitas testimonia tua in æternum :intellectum da mihi,  et vivam. \EVERSE
\VERSE Coph. Clamavi in toto corde meo : exaudi me,  Domine ; justificationes tuas requiram. \EVERSE
\VERSE Clamavi ad te ;  salvum me fac :ut custodiam mandata tua. \EVERSE
\VERSE Præveni in maturitate,  et clamavi :quia in verba tua supersperavi. \EVERSE
\VERSE Prævenerunt oculi mei ad te diluculo, ut meditarer eloquia tua. \EVERSE
\VERSE Vocem meam audi secundum misericordiam tuam,  Domine, et secundum judicium tuum vivifica me. \EVERSE
\VERSE Appropinquaverunt persequentes me iniquitati :a lege autem tua longe facti sunt. \EVERSE
\VERSE Prope es tu,  Domine, et omnes viæ tuæ veritas. \EVERSE
\VERSE Initio cognovi de testimoniis tuis, quia in æternum fundasti ea.

}
\newcommand{\psalmcxviiig}{
\VERSE Res. Vide humilitatem meam,  et eripe me, quia legem tuam non sum oblitus. \EVERSE
\VERSE Judica judicium meum,  et redime me :propter eloquium tuum vivifica me. \EVERSE
\VERSE Longe a peccatoribus salus, quia justificationes tuas non exquisierunt. \EVERSE
\VERSE Misericordiæ tuæ multæ,  Domine ; secundum judicium tuum vivifica me. \EVERSE
\VERSE Multi qui persequuntur me,  et tribulant me ; a testimoniis tuis non declinavi. \EVERSE
\VERSE Vidi prævaricantes et tabescebam, quia eloquia tua non custodierunt. \EVERSE
\VERSE Vide quoniam mandata tua dilexi,  Domine :in misericordia tua vivifica me. \EVERSE
\VERSE Principium verborum tuorum veritas ; in æternum omnia judicia justitiæ tuæ. \EVERSE
\VERSE Sin. Principes persecuti sunt me gratis, et a verbis tuis formidavit cor meum. \EVERSE
\VERSE Lætabor ego super eloquia tua, sicut qui invenit spolia multa. \EVERSE
\VERSE Iniquitatem odio habui,  et abominatus sum, legem autem tuam dilexi. \EVERSE
\VERSE Septies in die laudem dixi tibi, super judicia justitiæ tuæ. \EVERSE
\VERSE Pax multa diligentibus legem tuam, et non est illis scandalum. \EVERSE
\VERSE Exspectabam salutare tuum,  Domine, et mandata tua dilexi. \EVERSE
\VERSE Custodivit anima mea testimonia tua, et dilexit ea vehementer. \EVERSE
\VERSE Servavi mandata tua et testimonia tua, quia omnes viæ meæ in conspectu tuo. \EVERSE
\VERSE Tau. Appropinquet deprecatio mea in conspectu tuo,  Domine ; juxta eloquium tuum da mihi intellectum. \EVERSE
\VERSE Intret postulatio mea in conspectu tuo ; secundum eloquium tuum eripe me. \EVERSE
\VERSE Eructabunt labia mea hymnum, cum docueris me justificationes tuas. \EVERSE
\VERSE Pronuntiabit lingua mea eloquium tuum, quia omnia mandata tua æquitas. \EVERSE
\VERSE Fiat manus tua ut salvet me, quoniam mandata tua elegi. \EVERSE
\VERSE Concupivi salutare tuum,  Domine, et lex tua meditatio mea est. \EVERSE
\VERSE Vivet anima mea,  et laudabit te, et judicia tua adjuvabunt me. \EVERSE
\VERSE Erravi sicut ovis quæ periit : quære servum tuum, quia mandata tua non sum oblitus.

}
\newcommand{\psalmcxix}{
\VERSE Canticum graduum.Ad Dominum cum tribularer clamavi, 
et exaudivit me. \EVERSE
\VERSE Domine,  libera animam meam a labiis iniquiset a lingua dolosa. \EVERSE
\VERSE Quid detur tibi,  aut quid apponatur tibiad linguam dolosam ? \EVERSE
\VERSE Sagittæ potentis acutæ, cum carbonibus desolatoriis. \EVERSE
\VERSE Heu mihi,  quia incolatus meus prolongatus est !habitavi cum habitantibus Cedar ;  \EVERSE
\VERSE multum incola fuit anima mea.\VERSE Cum his qui oderunt pacem eram pacificus ; cum loquebar illis,  impugnabant me gratis.

}
\newcommand{\psalmcxx}{
\VERSE Canticum graduum.Levavi oculos meos in montes, 
unde veniet auxilium mihi. \EVERSE
\VERSE Auxilium meum a Domino, qui fecit cælum et terram. \EVERSE
\VERSE Non det in commotionem pedem tuum, neque dormitet qui custodit te. \EVERSE
\VERSE Ecce non dormitabit neque dormietqui custodit Israël. \EVERSE
\VERSE Dominus custodit te ; Dominus protectio tua super manum dexteram tuam. \EVERSE
\VERSE Per diem sol non uret te, neque luna per noctem. \EVERSE
\VERSE Dominus custodit te ab omni malo ; custodiat animam tuam Dominus. \EVERSE
\VERSE Dominus custodiat introitum tuum et exitum tuum, ex hoc nunc et usque in sæculum.

}
\newcommand{\psalmcxxi}{
\VERSE Canticum graduum.Lætatus sum in his quæ dicta sunt mihi :
In domum Domini ibimus. \EVERSE
\VERSE Stantes erant pedes nostriin atriis tuis,  Jerusalem. \EVERSE
\VERSE Jerusalem,  quæ ædificatur ut civitas, cujus participatio ejus in idipsum. \EVERSE
\VERSE Illuc enim ascenderunt tribus,  tribus Domini :testimonium Israël, 
ad confitendum nomini Domini. \EVERSE
\VERSE Quia illic sederunt sedes in judicio, sedes super domum David. \EVERSE
\VERSE Rogate quæ ad pacem sunt Jerusalem, et abundantia diligentibus te. \EVERSE
\VERSE Fiat pax in virtute tua, et abundantia in turribus tuis. \EVERSE
\VERSE Propter fratres meos et proximos meos, loquebar pacem de te. \EVERSE
\VERSE Propter domum Domini Dei nostri, quæsivi bona tibi.

}
\newcommand{\psalmcxxii}{
\VERSE Canticum graduum.Ad te levavi oculos meos, 
qui habitas in cælis. \EVERSE
\VERSE Ecce sicut oculi servorumin manibus dominorum suorum ; 
sicut oculi ancillæ
in manibus dominæ suæ :
ita oculi nostri ad Dominum Deum nostrum, 
donec misereatur nostri. \EVERSE
\VERSE Miserere nostri,  Domine,  miserere nostri, quia multum repleti sumus despectione ;  \EVERSE
\VERSE quia multum repleta est anima nostraopprobrium abundantibus,  et despectio superbis.

}
\newcommand{\psalmcxxiii}{
\VERSE Canticum graduum.Nisi quia Dominus erat in nobis, 
dicat nunc Israël,  \EVERSE
\VERSE nisi quia Dominus erat in nobis :cum exsurgerent homines in nos,  \EVERSE
\VERSE forte vivos deglutissent nos ; cum irasceretur furor eorum in nos,  \EVERSE
\VERSE forsitan aqua absorbuisset nos ; \VERSE torrentem pertransivit anima nostra ; forsitan pertransisset anima nostra
aquam intolerabilem. \EVERSE
\VERSE Benedictus Dominus,  qui non dedit nosin captionem dentibus eorum. \EVERSE
\VERSE Anima nostra sicut passer erepta estde laqueo venantium ; 
laqueus contritus est, 
et nos liberati sumus. \EVERSE
\VERSE Adjutorium nostrum in nomine Domini, qui fecit cælum et terram.

}
\newcommand{\psalmcxxiv}{
\VERSE Canticum graduum.Qui confidunt in Domino,  sicut mons Sion :
non commovebitur in æternum, 
qui habitat \EVERSE
\VERSE in Jerusalem.Montes in circuitu ejus ; 
et Dominus in circuitu populi sui, 
ex hoc nunc et usque in sæculum. \EVERSE
\VERSE Quia non relinquet Dominus virgam peccatorumsuper sortem justorum :
ut non extendant justi
ad iniquitatem manus suas,  \EVERSE
\VERSE benefac,  Domine,  bonis, et rectis corde. \EVERSE
\VERSE Declinantes autem in obligationes, adducet Dominus cum operantibus iniquitatem.
Pax super Israël !

}
\newcommand{\psalmcxxv}{
\VERSE Canticum graduum.In convertendo Dominus captivitatem Sion, 
facti sumus sicut consolati. \EVERSE
\VERSE Tunc repletum est gaudio os nostrum, et lingua nostra exsultatione.
Tunc dicent inter gentes :
Magnificavit Dominus facere cum eis. \EVERSE
\VERSE Magnificavit Dominus facere nobiscum ; facti sumus lætantes. \EVERSE
\VERSE Converte,  Domine,  captivitatem nostram, sicut torrens in austro. \EVERSE
\VERSE Qui seminant in lacrimis, in exsultatione metent. \EVERSE
\VERSE Euntes ibant et flebant, mittentes semina sua.
Venientes autem venient cum exsultatione, 
portantes manipulos suos.

}
\newcommand{\psalmcxxvi}{
\VERSE Canticum graduum Salomonis.Nisi Dominus ædificaverit domum, 
in vanum laboraverunt qui ædificant eam.
Nisi Dominus custodierit civitatem, 
frustra vigilat qui custodit eam. \EVERSE
\VERSE Vanum est vobis ante lucem surgere :surgite postquam sederitis, 
qui manducatis panem doloris.
Cum dederit dilectis suis somnum,  \EVERSE
\VERSE ecce hæreditas Domini,  filii ; merces,  fructus ventris. \EVERSE
\VERSE Sicut sagittæ in manu potentis, ita filii excussorum. \EVERSE
\VERSE Beatus vir qui implevit desiderium suum ex ipsis :non confundetur cum loquetur inimicis suis in porta.

}
\newcommand{\psalmcxxvii}{
\VERSE Canticum graduum.Beati omnes qui timent Dominum, 
qui ambulant in viis ejus. \EVERSE
\VERSE Labores manuum tuarum quia manducabis :beatus es,  et bene tibi erit. \EVERSE
\VERSE Uxor tua sicut vitis abundansin lateribus domus tuæ ; 
filii tui sicut novellæ olivarum
in circuitu mensæ tuæ. \EVERSE
\VERSE Ecce sic benedicetur homoqui timet Dominum. \EVERSE
\VERSE Benedicat tibi Dominus ex Sion, et videas bona Jerusalem omnibus diebus vitæ tuæ. \EVERSE
\VERSE Et videas filios filiorum tuorum :pacem super Israël.

}
\newcommand{\psalmcxxviii}{
\VERSE Canticum graduum.Sæpe expugnaverunt me a juventute mea, 
dicat nunc Israël ;  \EVERSE
\VERSE sæpe expugnaverunt me a juventute mea :etenim non potuerunt mihi. \EVERSE
\VERSE Supra dorsum meum fabricaverunt peccatores ; prolongaverunt iniquitatem suam. \EVERSE
\VERSE Dominus justusconcidit cervices peccatorum. \EVERSE
\VERSE Confundantur,  et convertantur retrorsumomnes qui oderunt Sion. \EVERSE
\VERSE Fiant sicut fœnum tectorum, quod priusquam evellatur exaruit : \EVERSE
\VERSE de quo non implevit manum suam qui metit, et sinum suum qui manipulos colligit. \EVERSE
\VERSE Et non dixerunt qui præteribant :Benedictio Domini super vos.
Benediximus vobis in nomine Domini.

}
\newcommand{\psalmcxxix}{
\VERSE Canticum graduum.De profundis clamavi ad te,  Domine ;  \EVERSE
\VERSE Domine,  exaudi vocem meam.Fiant aures tuæ intendentes in vocem deprecationis meæ. \EVERSE
\VERSE Si iniquitates observaveris,  Domine, Domine,  quis sustinebit ? \EVERSE
\VERSE Quia apud te propitiatio est ; et propter legem tuam sustinui te,  Domine.
Sustinuit anima mea in verbo ejus : \EVERSE
\VERSE speravit anima mea in Domino.\VERSE A custodia matutina usque ad noctem, speret Israël in Domino. \EVERSE
\VERSE Quia apud Dominum misericordia, et copiosa apud eum redemptio. \EVERSE
\VERSE Et ipse redimet Israëlex omnibus iniquitatibus ejus.

}
\newcommand{\psalmcxxx}{
\VERSE Canticum graduum David.Domine,  non est exaltatum cor meum, 
neque elati sunt oculi mei, 
neque ambulavi in magnis, 
neque in mirabilibus super me. \EVERSE
\VERSE Si non humiliter sentiebam, sed exaltavi animam meam :
sicut ablactatus est super matre sua, 
ita retributio in anima mea. \EVERSE
\VERSE Speret Israël in Domino, ex hoc nunc et usque in sæculum.

}
\newcommand{\psalmcxxxi}{
\VERSE Canticum graduum.Memento,  Domine,  David, 
et omnis mansuetudinis ejus : \EVERSE
\VERSE sicut juravit Domino ; votum vovit Deo Jacob : \EVERSE
\VERSE Si introiero in tabernaculum domus meæ ; si ascendero in lectum strati mei ;  \EVERSE
\VERSE si dedero somnum oculis meis, et palpebris meis dormitationem,  \EVERSE
\VERSE et requiem temporibus meis, donec inveniam locum Domino, 
tabernaculum Deo Jacob. \EVERSE
\VERSE Ecce audivimus eam in Ephrata ; invenimus eam in campis silvæ. \EVERSE
\VERSE Introibimus in tabernaculum ejus ; adorabimus in loco ubi steterunt pedes ejus. \EVERSE
\VERSE Surge,  Domine,  in requiem tuam, tu et arca sanctificationis tuæ. \EVERSE
\VERSE Sacerdotes tui induantur justitiam, et sancti tui exsultent. \EVERSE
\VERSE Propter David servum tuumnon avertas faciem christi tui. \EVERSE
\VERSE Juravit Dominus David veritatem, et non frustrabitur eam :
De fructu ventris tui
ponam super sedem tuam. \EVERSE
\VERSE Si custodierint filii tui testamentum meum, et testimonia mea hæc quæ docebo eos, 
et filii eorum usque in sæculum
sedebunt super sedem tuam. \EVERSE
\VERSE Quoniam elegit Dominus Sion :elegit eam in habitationem sibi. \EVERSE
\VERSE Hæc requies mea in sæculum sæculi ; hic habitabo,  quoniam elegi eam. \EVERSE
\VERSE Viduam ejus benedicens benedicam ; pauperes ejus saturabo panibus. \EVERSE
\VERSE Sacerdotes ejus induam salutari, et sancti ejus exsultatione exsultabunt. \EVERSE
\VERSE Illuc producam cornu David ; paravi lucernam christo meo. \EVERSE
\VERSE Inimicos ejus induam confusione ; super ipsum autem efflorebit sanctificatio mea.

}
\newcommand{\psalmcxxxii}{
\VERSE Canticum graduum David.Ecce quam bonum et quam jucundum, 
habitare fratres in unum ! \EVERSE
\VERSE Sicut unguentum in capite, quod descendit in barbam,  barbam Aaron, 
quod descendit in oram vestimenti ejus ;  \EVERSE
\VERSE sicut ros Hermon, qui descendit in montem Sion.
Quoniam illic mandavit Dominus benedictionem, 
et vitam usque in sæculum.

}
\newcommand{\psalmcxxxiii}{
\VERSE Canticum graduum.Ecce nunc benedicite Dominum,  omnes servi Domini :
qui statis in domo Domini, 
in atriis domus Dei nostri. \EVERSE
\VERSE In noctibus extollite manus vestras in sancta, et benedicite Dominum. \EVERSE
\VERSE Benedicat te Dominus ex Sion, qui fecit cælum et terram.

}
\newcommand{\psalmcxxxiv}{
\VERSE Alleluja.Laudate nomen Domini ; 
laudate,  servi,  Dominum : \EVERSE
\VERSE qui statis in domo Domini, in atriis domus Dei nostri. \EVERSE
\VERSE Laudate Dominum,  quia bonus Dominus ; psallite nomini ejus,  quoniam suave. \EVERSE
\VERSE Quoniam Jacob elegit sibi Dominus ; Israël in possessionem sibi. \EVERSE
\VERSE Quia ego cognovi quod magnus est Dominus, et Deus noster præ omnibus diis. \EVERSE
\VERSE Omnia quæcumque voluit Dominus fecit, in cælo,  in terra,  in mari et in omnibus abyssis. \EVERSE
\VERSE Educens nubes ab extremo terræ, fulgura in pluviam fecit ; 
qui producit ventos de thesauris suis. \EVERSE
\VERSE Qui percussit primogenita Ægypti, ab homine usque ad pecus. \EVERSE
\VERSE Et misit signa et prodigia in medio tui,  Ægypte :in Pharaonem,  et in omnes servos ejus. \EVERSE
\VERSE Qui percussit gentes multas, et occidit reges fortes : \EVERSE
\VERSE Sehon,  regem Amorrhæorum,  et Og,  regem Basan, et omnia regna Chanaan : \EVERSE
\VERSE et dedit terram eorum hæreditatem, hæreditatem Israël populo suo. \EVERSE
\VERSE Domine,  nomen tuum in æternum ; Domine,  memoriale tuum in generationem et generationem. \EVERSE
\VERSE Quia judicabit Dominus populum suum, et in servis suis deprecabitur. \EVERSE
\VERSE Simulacra gentium argentum et aurum, opera manuum hominum. \EVERSE
\VERSE Os habent,  et non loquentur ; oculos habent,  et non videbunt. \EVERSE
\VERSE Aures habent,  et non audient ; neque enim est spiritus in ore ipsorum. \EVERSE
\VERSE Similes illis fiant qui faciunt ea, et omnes qui confidunt in eis. \EVERSE
\VERSE Domus Israël,  benedicite Domino ; domus Aaron,  benedicite Domino. \EVERSE
\VERSE Domus Levi,  benedicite Domino ; qui timetis Dominum,  benedicite Domino. \EVERSE
\VERSE Benedictus Dominus ex Sion, qui habitat in Jerusalem.

}
\newcommand{\psalmcxxxv}{
\VERSE Alleluja.Confitemini Domino,  quoniam bonus, 
quoniam in æternum misericordia ejus. \EVERSE
\VERSE Confitemini Deo deorum, quoniam in æternum misericordia ejus. \EVERSE
\VERSE Confitemini Domino dominorum, quoniam in æternum misericordia ejus. \EVERSE
\VERSE Qui facit mirabilia magna solus, quoniam in æternum misericordia ejus. \EVERSE
\VERSE Qui fecit cælos in intellectu, quoniam in æternum misericordia ejus. \EVERSE
\VERSE Qui firmavit terram super aquas, quoniam in æternum misericordia ejus. \EVERSE
\VERSE Qui fecit luminaria magna, quoniam in æternum misericordia ejus : \EVERSE
\VERSE solem in potestatem diei, quoniam in æternum misericordia ejus ;  \EVERSE
\VERSE lunam et stellas in potestatem noctis, quoniam in æternum misericordia ejus. \EVERSE
\VERSE Qui percussit Ægyptum cum primogenitis eorum, quoniam in æternum misericordia ejus. \EVERSE
\VERSE Qui eduxit Israël de medio eorum, quoniam in æternum misericordia ejus,  \EVERSE
\VERSE in manu potenti et brachio excelso, quoniam in æternum misericordia ejus. \EVERSE
\VERSE Qui divisit mare Rubrum in divisiones, quoniam in æternum misericordia ejus ;  \EVERSE
\VERSE et eduxit Israël per medium ejus, quoniam in æternum misericordia ejus ;  \EVERSE
\VERSE et excussit Pharaonem et virtutem ejus in mari Rubro, quoniam in æternum misericordia ejus. \EVERSE
\VERSE Qui traduxit populum suum per desertum, quoniam in æternum misericordia ejus. \EVERSE
\VERSE Qui percussit reges magnos, quoniam in æternum misericordia ejus ;  \EVERSE
\VERSE et occidit reges fortes, quoniam in æternum misericordia ejus : \EVERSE
\VERSE Sehon,  regem Amorrhæorum, quoniam in æternum misericordia ejus ;  \EVERSE
\VERSE et Og,  regem Basan, quoniam in æternum misericordia ejus : \EVERSE
\VERSE et dedit terram eorum hæreditatem, quoniam in æternum misericordia ejus ;  \EVERSE
\VERSE hæreditatem Israël,  servo suo, quoniam in æternum misericordia ejus. \EVERSE
\VERSE Quia in humilitate nostra memor fuit nostri, quoniam in æternum misericordia ejus ;  \EVERSE
\VERSE et redemit nos ab inimicis nostris, quoniam in æternum misericordia ejus. \EVERSE
\VERSE Qui dat escam omni carni, quoniam in æternum misericordia ejus. \EVERSE
\VERSE Confitemini Deo cæli, quoniam in æternum misericordia ejus.
Confitemini Domino dominorum, 
quoniam in æternum misericordia ejus.

}
\newcommand{\psalmcxxxvi}{
\VERSE Psalmus David,  Jeremiæ.Super flumina Babylonis illic sedimus et flevimus, 
cum recordaremur Sion. \EVERSE
\VERSE In salicibus in medio ejussuspendimus organa nostra : \EVERSE
\VERSE quia illic interrogaverunt nos,  qui captivos duxerunt nos, verba cantionum ; 
et qui abduxerunt nos :
Hymnum cantate nobis de canticis Sion. \EVERSE
\VERSE Quomodo cantabimus canticum Dominiin terra aliena ? \EVERSE
\VERSE Si oblitus fuero tui,  Jerusalem, oblivioni detur dextera mea. \EVERSE
\VERSE Adhæreat lingua mea faucibus meis, si non meminero tui ; 
si non proposuero Jerusalem
in principio lætitiæ meæ. \EVERSE
\VERSE Memor esto,  Domine,  filiorum Edom, in die Jerusalem :
qui dicunt : Exinanite,  exinanite
usque ad fundamentum in ea. \EVERSE
\VERSE Filia Babylonis misera ! beatus qui retribuet tibiretributionem tuam quam retribuisti nobis. \EVERSE
\VERSE Beatus qui tenebit, et allidet parvulos tuos ad petram.

}
\newcommand{\psalmcxxxvii}{
\VERSE Ipsi David.Confitebor tibi,  Domine,  in toto corde meo, 
quoniam audisti verba oris mei.
In conspectu angelorum psallam tibi ;  \EVERSE
\VERSE adorabo ad templum sanctum tuum, et confitebor nomini tuo :
super misericordia tua et veritate tua ; 
quoniam magnificasti super omne,  nomen sanctum tuum. \EVERSE
\VERSE In quacumque die invocavero te,  exaudi me ; multiplicabis in anima mea virtutem. \EVERSE
\VERSE Confiteantur tibi,  Domine,  omnes reges terræ, quia audierunt omnia verba oris tui. \EVERSE
\VERSE Et cantent in viis Domini, quoniam magna est gloria Domini ;  \EVERSE
\VERSE quoniam excelsus Dominus,  et humilia respicit, et alta a longe cognoscit. \EVERSE
\VERSE Si ambulavero in medio tribulationis,  vivificabis me ; et super iram inimicorum meorum extendisti manum tuam, 
et salvum me fecit dextera tua. \EVERSE
\VERSE Dominus retribuet pro me.Domine,  misericordia tua in sæculum ; 
opera manuum tuarum ne despicias.

}
\newcommand{\psalmcxxxviii}{
\VERSE In finem,  psalmus David.Domine,  probasti me,  et cognovisti me ;  \EVERSE
\VERSE tu cognovisti sessionem meam et resurrectionem meam.\VERSE Intellexisti cogitationes meas de longe ; semitam meam et funiculum meum investigasti : \EVERSE
\VERSE et omnes vias meas prævidisti, quia non est sermo in lingua mea. \EVERSE
\VERSE Ecce,  Domine,  tu cognovisti omnia, novissima et antiqua.
Tu formasti me,  et posuisti super me manum tuam. \EVERSE
\VERSE Mirabilis facta est scientia tua ex me ; confortata est,  et non potero ad eam. \EVERSE
\VERSE Quo ibo a spiritu tuo ?et quo a facie tua fugiam ? \EVERSE
\VERSE Si ascendero in cælum,  tu illic es ; si descendero in infernum,  ades. \EVERSE
\VERSE Si sumpsero pennas meas diluculo, et habitavero in extremis maris,  \EVERSE
\VERSE etenim illuc manus tua deducet me, et tenebit me dextera tua. \EVERSE
\VERSE Et dixi : Forsitan tenebræ conculcabunt me ; et nox illuminatio mea in deliciis meis. \EVERSE
\VERSE Quia tenebræ non obscurabuntur a te, et nox sicut dies illuminabitur :
sicut tenebræ ejus,  ita et lumen ejus. \EVERSE
\VERSE Quia tu possedisti renes meos ; suscepisti me de utero matris meæ. \EVERSE
\VERSE Confitebor tibi,  quia terribiliter magnificatus es ; mirabilia opera tua,  et anima mea cognoscit nimis. \EVERSE
\VERSE Non est occultatum os meum a te,  quod fecisti in occulto ; et substantia mea in inferioribus terræ. \EVERSE
\VERSE Imperfectum meum viderunt oculi tui, et in libro tuo omnes scribentur.
Dies formabuntur,  et nemo in eis. \EVERSE
\VERSE Mihi autem nimis honorificati sunt amici tui,  Deus ; nimis confortatus est principatus eorum. \EVERSE
\VERSE Dinumerabo eos,  et super arenam multiplicabuntur.Exsurrexi,  et adhuc sum tecum. \EVERSE
\VERSE Si occideris,  Deus,  peccatores, viri sanguinum,  declinate a me : \EVERSE
\VERSE quia dicitis in cogitatione :Accipient in vanitate civitates tuas. \EVERSE
\VERSE Nonne qui oderunt te,  Domine,  oderam, et super inimicos tuos tabescebam ? \EVERSE
\VERSE Perfecto odio oderam illos, et inimici facti sunt mihi. \EVERSE
\VERSE Proba me,  Deus,  et scito cor meum ; interroga me,  et cognosce semitas meas. \EVERSE
\VERSE Et vide si via iniquitatis in me est, et deduc me in via æterna.

}
\newcommand{\psalmcxxxix}{
\VERSE In finem. Psalmus David.\VERSE Eripe me,  Domine,  ab homine malo ; a viro iniquo eripe me. \EVERSE
\VERSE Qui cogitaverunt iniquitates in corde, tota die constituebant prælia. \EVERSE
\VERSE Acuerunt linguas suas sicut serpentis ; venenum aspidum sub labiis eorum. \EVERSE
\VERSE Custodi me,  Domine,  de manu peccatoris, et ab hominibus iniquis eripe me.
Qui cogitaverunt supplantare gressus meos : \EVERSE
\VERSE absconderunt superbi laqueum mihi.Et funes extenderunt in laqueum ; 
juxta iter,  scandalum posuerunt mihi. \EVERSE
\VERSE Dixi Domino : Deus meus es tu ; exaudi,  Domine,  vocem deprecationis meæ. \EVERSE
\VERSE Domine,  Domine,  virtus salutis meæ, obumbrasti super caput meum in die belli. \EVERSE
\VERSE Ne tradas me,  Domine,  a desiderio meo peccatori :cogitaverunt contra me ; 
ne derelinquas me,  ne forte exaltentur. \EVERSE
\VERSE Caput circuitus eorum :labor labiorum ipsorum operiet eos. \EVERSE
\VERSE Cadent super eos carbones ; in ignem dejicies eos :
in miseriis non subsistent. \EVERSE
\VERSE Vir linguosus non dirigetur in terra ; virum injustum mala capient in interitu. \EVERSE
\VERSE Cognovi quia faciet Dominus judicium inopis, et vindictam pauperum. \EVERSE
\VERSE Verumtamen justi confitebuntur nomini tuo, et habitabunt recti cum vultu tuo.

}
\newcommand{\psalmcxl}{
\VERSE Psalmus David.Domine,  clamavi ad te : exaudi me ; 
intende voci meæ,  cum clamavero ad te. \EVERSE
\VERSE Dirigatur oratio mea sicut incensum in conspectu tuo ; elevatio manuum mearum sacrificium vespertinum. \EVERSE
\VERSE Pone,  Domine,  custodiam ori meo, et ostium circumstantiæ labiis meis. \EVERSE
\VERSE Non declines cor meum in verba malitiæ, ad excusandas excusationes in peccatis ; 
cum hominibus operantibus iniquitatem, 
et non communicabo cum electis eorum. \EVERSE
\VERSE Corripiet me justus in misericordia,  et increpabit me :oleum autem peccatoris non impinguet caput meum.
Quoniam adhuc et oratio mea in beneplacitis eorum : \EVERSE
\VERSE absorpti sunt juncti petræ judices eorum.Audient verba mea,  quoniam potuerunt. \EVERSE
\VERSE Sicut crassitudo terræ erupta est super terram, dissipata sunt ossa nostra secus infernum. \EVERSE
\VERSE Quia ad te,  Domine,  Domine,  oculi mei ; in te speravi,  non auferas animam meam. \EVERSE
\VERSE Custodi me a laqueo quem statuerunt mihi, et a scandalis operantium iniquitatem. \EVERSE
\VERSE Cadent in retiaculo ejus peccatores :singulariter sum ego,  donec transeam.

}
\newcommand{\psalmcxli}{
\VERSE Intellectus David,  cum esset in spelunca,  oratio.\VERSE Voce mea ad Dominum clamavi, voce mea ad Dominum deprecatus sum. \EVERSE
\VERSE Effundo in conspectu ejus orationem meam, et tribulationem meam ante ipsum pronuntio : \EVERSE
\VERSE in deficiendo ex me spiritum meum, et tu cognovisti semitas meas.
In via hac qua ambulabam
absconderunt laqueum mihi. \EVERSE
\VERSE Considerabam ad dexteram,  et videbam, et non erat qui cognosceret me :
periit fuga a me, 
et non est qui requirat animam meam. \EVERSE
\VERSE Clamavi ad te,  Domine ; dixi : Tu es spes mea, 
portio mea in terra viventium. \EVERSE
\VERSE Intende ad deprecationem meam, quia humiliatus sum nimis.
Libera me a persequentibus me, 
quia confortati sunt super me. \EVERSE
\VERSE Educ de custodia animam meamad confitendum nomini tuo ; 
me exspectant justi donec retribuas mihi.

}
\newcommand{\psalmcxlii}{
\VERSE Psalmus David,  quando persequebatur eum Absalom filius ejus.Domine,  exaudi orationem meam ; 
auribus percipe obsecrationem meam in veritate tua ; 
exaudi me in tua justitia. \EVERSE
\VERSE Et non intres in judicium cum servo tuo, quia non justificabitur in conspectu tuo omnis vivens. \EVERSE
\VERSE Quia persecutus est inimicus animam meam ; humiliavit in terra vitam meam ; 
collocavit me in obscuris,  sicut mortuos sæculi. \EVERSE
\VERSE Et anxiatus est super me spiritus meus ; in me turbatum est cor meum. \EVERSE
\VERSE Memor fui dierum antiquorum ; meditatus sum in omnibus operibus tuis :
in factis manuum tuarum meditabar. \EVERSE
\VERSE Expandi manus meas ad te ; anima mea sicut terra sine aqua tibi. \EVERSE
\VERSE Velociter exaudi me,  Domine ; defecit spiritus meus.
Non avertas faciem tuam a me, 
et similis ero descendentibus in lacum. \EVERSE
\VERSE Auditam fac mihi mane misericordiam tuam, quia in te speravi.
Notam fac mihi viam in qua ambulem, 
quia ad te levavi animam meam. \EVERSE
\VERSE Eripe me de inimicis meis,  Domine :ad te confugi. \EVERSE
\VERSE Doce me facere voluntatem tuam, quia Deus meus es tu.
Spiritus tuus bonus deducet me in terram rectam. \EVERSE
\VERSE Propter nomen tuum,  Domine,  vivificabis me :in æquitate tua,  educes de tribulatione animam meam,  \EVERSE
\VERSE et in misericordia tua disperdes inimicos meos, et perdes omnes qui tribulant animam meam, 
quoniam ego servus tuus sum.

}
\newcommand{\psalmcxliii}{
\VERSE Psalmus David. Adversus Goliath.Benedictus Dominus Deus meus, 
qui docet manus meas ad prælium, 
et digitos meos ad bellum. \EVERSE
\VERSE Misericordia mea et refugium meum ; susceptor meus et liberator meus ; 
protector meus,  et in ipso speravi, 
qui subdit populum meum sub me. \EVERSE
\VERSE Domine,  quid est homo,  quia innotuisti ei ?aut filius hominis,  quia reputas eum ? \EVERSE
\VERSE Homo vanitati similis factus est ; dies ejus sicut umbra prætereunt. \EVERSE
\VERSE Domine,  inclina cælos tuos,  et descende ; tange montes,  et fumigabunt. \EVERSE
\VERSE Fulgura coruscationem,  et dissipabis eos ; emitte sagittas tuas,  et conturbabis eos. \EVERSE
\VERSE Emitte manum tuam de alto : eripe me, et libera me de aquis multis, 
de manu filiorum alienorum : \EVERSE
\VERSE quorum os locutum est vanitatem, et dextera eorum dextera iniquitatis. \EVERSE
\VERSE Deus,  canticum novum cantabo tibi ; in psalterio decachordo psallam tibi. \EVERSE
\VERSE Qui das salutem regibus, qui redemisti David servum tuum de gladio maligno,  \EVERSE
\VERSE eripe me,  et erue me de manu filiorum alienorum, quorum os locutum est vanitatem, 
et dextera eorum dextera iniquitatis. \EVERSE
\VERSE Quorum filii sicut novellæ plantationes in juventute sua ; filiæ eorum compositæ, 
circumornatæ ut similitudo templi. \EVERSE
\VERSE Promptuaria eorum plena,  eructantia ex hoc in illud ; oves eorum fœtosæ,  abundantes in egressibus suis ;  \EVERSE
\VERSE boves eorum crassæ.Non est ruina maceriæ,  neque transitus, 
neque clamor in plateis eorum. \EVERSE
\VERSE Beatum dixerunt populum cui hæc sunt ; beatus populus cujus Dominus Deus ejus.

}
\newcommand{\psalmcxliva}{
\VERSE Laudatio ipsi David.Exaltabo te,  Deus meus rex, 
et benedicam nomini tuo in sæculum,  et in sæculum sæculi. \EVERSE
\VERSE Per singulos dies benedicam tibi, et laudabo nomen tuum in sæculum,  et in sæculum sæculi. \EVERSE
\VERSE Magnus Dominus,  et laudabilis nimis, et magnitudinis ejus non est finis. \EVERSE
\VERSE Generatio et generatio laudabit opera tua, et potentiam tuam pronuntiabunt. \EVERSE
\VERSE Magnificentiam gloriæ sanctitatis tuæ loquentur, et mirabilia tua narrabunt. \EVERSE
\VERSE Et virtutem terribilium tuorum dicent, et magnitudinem tuam narrabunt. \EVERSE
\VERSE Memoriam abundantiæ suavitatis tuæ eructabunt, et justitia tua exsultabunt. \EVERSE
\VERSE Miserator et misericors Dominus :patiens,  et multum misericors. \EVERSE
\VERSE Suavis Dominus universis, et miserationes ejus super omnia opera ejus.

}
\newcommand{\psalmcxlivb}{
\VERSE Confiteantur tibi,  Domine,  omnia opera tua, et sancti tui benedicant tibi. \EVERSE
\VERSE Gloriam regni tui dicent, et potentiam tuam loquentur : \EVERSE
\VERSE ut notam faciant filiis hominum potentiam tuam, et gloriam magnificentiæ regni tui. \EVERSE
\VERSE Regnum tuum regnum omnium sæculorum ; et dominatio tua in omni generatione et generationem.
Fidelis Dominus in omnibus verbis suis, 
et sanctus in omnibus operibus suis. \EVERSE
\VERSE Allevat Dominus omnes qui corruunt, et erigit omnes elisos. \EVERSE
\VERSE Oculi omnium in te sperant,  Domine, et tu das escam illorum in tempore opportuno. \EVERSE
\VERSE Aperis tu manum tuam, et imples omne animal benedictione. \EVERSE
\VERSE Justus Dominus in omnibus viis suis, et sanctus in omnibus operibus suis. \EVERSE
\VERSE Prope est Dominus omnibus invocantibus eum, omnibus invocantibus eum in veritate. \EVERSE
\VERSE Voluntatem timentium se faciet, et deprecationem eorum exaudiet,  et salvos faciet eos. \EVERSE
\VERSE Custodit Dominus omnes diligentes se, et omnes peccatores disperdet. \EVERSE
\VERSE Laudationem Domini loquetur os meum ; et benedicat omnis caro nomini sancto ejus in sæculum,  et in sæculum sæculi.

}
\newcommand{\psalmcxlv}{
\VERSE Alleluja,  Aggæi et Zachariæ.\VERSE Lauda,  anima mea,  Dominum.Laudabo Dominum in vita mea ; 
psallam Deo meo quamdiu fuero.
Nolite confidere in principibus,  \EVERSE
\VERSE in filiis hominum,  in quibus non est salus.\VERSE Exibit spiritus ejus,  et revertetur in terram suam ; in illa die peribunt omnes cogitationes eorum. \EVERSE
\VERSE Beatus cujus Deus Jacob adjutor ejus, spes ejus in Domino Deo ipsius : \EVERSE
\VERSE qui fecit cælum et terram, mare,  et omnia quæ in eis sunt. \EVERSE
\VERSE Qui custodit veritatem in sæculum ; facit judicium injuriam patientibus ; 
dat escam esurientibus.
Dominus solvit compeditos ;  \EVERSE
\VERSE Dominus illuminat cæcos.Dominus erigit elisos ; 
Dominus diligit justos. \EVERSE
\VERSE Dominus custodit advenas, pupillum et viduam suscipiet, 
et vias peccatorum disperdet. \EVERSE
\VERSE Regnabit Dominus in sæcula ; Deus tuus,  Sion,  in generationem et generationem.

}
\newcommand{\psalmcxlvi}{
\VERSE Alleluja.Laudate Dominum,  quoniam bonus est psalmus ; 
Deo nostro sit jucunda,  decoraque laudatio. \EVERSE
\VERSE Ædificans Jerusalem Dominus, dispersiones Israëlis congregabit : \EVERSE
\VERSE qui sanat contritos corde, et alligat contritiones eorum ;  \EVERSE
\VERSE qui numerat multitudinem stellarum, et omnibus eis nomina vocat. \EVERSE
\VERSE Magnus Dominus noster,  et magna virtus ejus, et sapientiæ ejus non est numerus. \EVERSE
\VERSE Suscipiens mansuetos Dominus ; humilians autem peccatores usque ad terram. \EVERSE
\VERSE Præcinite Domino in confessione ; psallite Deo nostro in cithara. \EVERSE
\VERSE Qui operit cælum nubibus, et parat terræ pluviam ; 
qui producit in montibus fœnum, 
et herbam servituti hominum ;  \EVERSE
\VERSE qui dat jumentis escam ipsorum, et pullis corvorum invocantibus eum. \EVERSE
\VERSE Non in fortitudine equi voluntatem habebit, nec in tibiis viri beneplacitum erit ei. \EVERSE
\VERSE Beneplacitum est Domino super timentes eum, et in eis qui sperant super misericordia ejus.

}
\newcommand{\psalmcxlvii}{
\VERSE Alleluja.Lauda,  Jerusalem,  Dominum ; 
lauda Deum tuum,  Sion. \EVERSE
\VERSE Quoniam confortavit seras portarum tuarum ; benedixit filiis tuis in te. \EVERSE
\VERSE Qui posuit fines tuos pacem, et adipe frumenti satiat te. \EVERSE
\VERSE Qui emittit eloquium suum terræ :velociter currit sermo ejus. \EVERSE
\VERSE Qui dat nivem sicut lanam ; nebulam sicut cinerem spargit. \EVERSE
\VERSE Mittit crystallum suam sicut buccellas :ante faciem frigoris ejus quis sustinebit ? \EVERSE
\VERSE Emittet verbum suum,  et liquefaciet ea ; flabit spiritus ejus,  et fluent aquæ. \EVERSE
\VERSE Qui annuntiat verbum suum Jacob, justitias et judicia sua Israël. \EVERSE
\VERSE Non fecit taliter omni nationi, et judicia sua non manifestavit eis.
Alleluja.

}
\newcommand{\psalmcxlviii}{
\VERSE Alleluja.Laudate Dominum de cælis ; 
laudate eum in excelsis. \EVERSE
\VERSE Laudate eum,  omnes angeli ejus ; laudate eum,  omnes virtutes ejus. \EVERSE
\VERSE Laudate eum,  sol et luna ; laudate eum,  omnes stellæ et lumen. \EVERSE
\VERSE Laudate eum,  cæli cælorum ; et aquæ omnes quæ super cælos sunt,  \EVERSE
\VERSE laudent nomen Domini.Quia ipse dixit,  et facta sunt ; 
ipse mandavit,  et creata sunt. \EVERSE
\VERSE Statuit ea in æternum,  et in sæculum sæculi ; præceptum posuit,  et non præteribit. \EVERSE
\VERSE Laudate Dominum de terra, dracones et omnes abyssi ;  \EVERSE
\VERSE ignis,  grando,  nix,  glacies,  spiritus procellarum, quæ faciunt verbum ejus ;  \EVERSE
\VERSE montes,  et omnes colles ; ligna fructifera,  et omnes cedri ;  \EVERSE
\VERSE bestiæ,  et universa pecora ; serpentes,  et volucres pennatæ ;  \EVERSE
\VERSE reges terræ et omnes populi ; principes et omnes judices terræ ;  \EVERSE
\VERSE juvenes et virgines ;  senes cum junioribus, laudent nomen Domini : \EVERSE
\VERSE quia exaltatum est nomen ejus solius.\VERSE Confessio ejus super cælum et terram ; et exaltavit cornu populi sui.
Hymnus omnibus sanctis ejus ; 
filiis Israël,  populo appropinquanti sibi.
Alleluja.

}
\newcommand{\psalmcxlix}{
\VERSE Alleluja.Cantate Domino canticum novum ; 
laus ejus in ecclesia sanctorum. \EVERSE
\VERSE Lætetur Israël in eo qui fecit eum, et filii Sion exsultent in rege suo. \EVERSE
\VERSE Laudent nomen ejus in choro ; in tympano et psalterio psallant ei. \EVERSE
\VERSE Quia beneplacitum est Domino in populo suo, et exaltabit mansuetos in salutem. \EVERSE
\VERSE Exsultabunt sancti in gloria ; lætabuntur in cubilibus suis. \EVERSE
\VERSE Exaltationes Dei in gutture eorum, et gladii ancipites in manibus eorum : \EVERSE
\VERSE ad faciendam vindictam in nationibus, increpationes in populis ;  \EVERSE
\VERSE ad alligandos reges eorum in compedibus, et nobiles eorum in manicis ferreis ;  \EVERSE
\VERSE ut faciant in eis judicium conscriptum :gloria hæc est omnibus sanctis ejus.
Alleluja.

}
\newcommand{\psalmcl}{
\VERSE Alleluja.Laudate Dominum in sanctis ejus ; 
laudate eum in firmamento virtutis ejus. \EVERSE
\VERSE Laudate eum in virtutibus ejus ; laudate eum secundum multitudinem magnitudinis ejus. \EVERSE
\VERSE Laudate eum in sono tubæ ;  laudate eum in psalterio et cithara. \EVERSE
\VERSE Laudate eum in tympano et choro ; laudate eum in chordis et organo. \EVERSE
\VERSE Laudate eum in cymbalis benesonantibus ; laudate eum in cymbalis jubilationis. \EVERSE
\VERSE Omnis spiritus laudet Dominum !Alleluja.

}

 
%%%%%%% PSAUME 1 %%%%%%%
\newcommand{\psalmifr}{
\VERSE Heureux l'homme qui n'a point marché dans le conseil des impies, qui ne s'est pas arrêté dans la voie des pécheurs, et qui ne s'est point assis dans la chaire de pestilence; \EVERSE
\VERSE mais qui a ses affections dans la loi du Seigneur, et qui médite cette loi jour et nuit. \EVERSE
\VERSE Il sera comme un arbre planté près d'un cours d'eau, et qui donne son fruit en son temps, et son feuillage ne tombera pas; et tout ce qu'il fera réussira. \EVERSE
\VERSE Il n'en est pas ainsi des impies, il n'en est pas ainsi; mais ils sont comme la poussière que le vent disperse de dessus la surface du sol. \EVERSE
\VERSE C'est pourquoi les impies ne ressusciteront point dans le jugement, ni les pécheurs dans l'assemblée des justes. \EVERSE
\VERSE Car le Seigneur connaît la voie des justes, et le chemin des impies périra.

}
%%%%%%% PSAUME 2 %%%%%%%
\newcommand{\psalmiifr}{
\VERSE Pourquoi les nations ont-elles frémi, et les peuples ont-ils formé de vains desseins? \EVERSE
\VERSE Les rois de la terre se sont levés, et les princes se sont assemblés contre le Seigneur et contre Son Christ. \EVERSE
\VERSE Rompons leurs liens, et jetons loin de nous leur joug. \EVERSE
\VERSE Celui qui habite dans les cieux Se rira d'eux, et le Seigneur Se moquera d'eux. \EVERSE
\VERSE Alors Il leur parlera dans Sa colère, et Il les épouvantera dans Sa fureur. \EVERSE
\VERSE Pour Moi, J'ai été établi Roi par Lui sur Sion, Sa montagne sainte, afin d'annoncer Son décret. \EVERSE
\VERSE Le Seigneur M'a dit: Tu es Mon Fils; Je T'ai engendré aujourd'hui. \EVERSE
\VERSE Demande-Moi et Je Te donnerai les nations pour Ton héritage, et pour Ton domaine les extrémités de la terre. \EVERSE
\VERSE Tu les gouverneras avec une verge de fer, et Tu les briseras comme le vase du potier. \EVERSE
\VERSE Et maintenant, ô rois, comprenez; instruisez-vous, juges de la terre. \EVERSE
\VERSE Servez le Seigneur avec crainte, et réjouissez-vous en Lui avec tremblement. \EVERSE
\VERSE Attachez-vous à la doctrine, de peur que le Seigneur ne S'irrite, et que vous ne périssiez hors de la voie droite. \EVERSE
\VERSE Lorsque bientôt s'enflammera Sa colère, heureux tous ceux qui ont confiance en Lui.

}
%%%%%%% PSAUME 3 %%%%%%%
\newcommand{\psalmiiifr}{
\VERSE Psaume de David lorsqu'il fuyait devant Absalom son fils. \EVERSE
\VERSE Seigneur, pourquoi ceux qui me persécutent se sont-ils multipliés? Une multitude s'élève contre moi. \EVERSE
\VERSE Beaucoup disent à mon âme: Il n'y a pas de salut pour elle dans son Dieu. \EVERSE
\VERSE Mais Vous, Seigneur, Vous êtes mon protecteur et ma gloire, et Vous relevez ma tête. \EVERSE
\VERSE De ma voix j'ai crié vers le Seigneur, et Il m'a exaucé du haut de Sa montagne sainte. \EVERSE
\VERSE Je me suis endormi, et j'ai été assoupi; et je me suis levé, parce que le Seigneur a été mon soutien. \EVERSE
\VERSE Je ne craindrai point les milliers d'hommes du peuple qui m'environnent. Levez-Vous, Seigneur; sauvez-moi, mon Dieu. \EVERSE
\VERSE Car Vous avez frappé tous ceux qui s'opposaient à moi sans raison; Vous avez brisé les dents des pécheurs. \EVERSE
\VERSE Le salut vient du Seigneur; et c'est Vous qui bénissez Votre peuple.

}
%%%%%%% PSAUME 4 %%%%%%%
\newcommand{\psalmivfr}{
\VERSE Pour la fin, parmi les cantiques, psaume de David. \EVERSE
\VERSE Lorsque je L'ai invoqué, le Dieu de ma justice m'a exaucé; Vous m'avez mis au large dans la tribulation. Ayez pitié de moi, et exaucez ma prière. \EVERSE
\VERSE Enfants des hommes, jusques à quand aurez-vous le coeur appesanti? Pourquoi aimez-vous la vanité, et cherchez-vous le mensonge? \EVERSE
\VERSE Sachez donc que le Seigneur a merveilleusement glorifié Son Saint. Le Seigneur m'exaucera quand j'aurai crié vers Lui. \EVERSE
\VERSE Irritez-vous, mais ne péchez point. Ce que vous dites contre moi au fond de vos coeurs, répétez-le avec componction sur vos couches. \EVERSE
\VERSE Offrez un sacrifice de justice, et espérez au Seigneur. Beaucoup disent: Qui nous fera voir le bonheur? \EVERSE
\VERSE La lumière de Votre visage est gravée sur nous, Seigneur; Vous avez mis la joie dans mon coeur. \EVERSE
\VERSE Ils se sont multipliés par l'abondance de leur froment, de leur vin et de leur huile. \EVERSE
\VERSE Et moi je dormirai et me reposerai en paix; \EVERSE
\VERSE parce que Vous, Seigneur, m'avez affermi dans une espérance singulière.

}
%%%%%%% PSAUME 5 %%%%%%%
\newcommand{\psalmvfr}{
\VERSE Pour la fin, pour celle qui obtient l'héritage, psaume de David. \EVERSE
\VERSE Seigneur, prêtez l'oreille à mes paroles, comprenez mon cri. \EVERSE
\VERSE Soyez attentif à la voix de ma prière, mon Roi et mon Dieu. \EVERSE
\VERSE Car c'est Vous que je prierai, Seigneur; dès le matin Vous exaucerez ma voix. \EVERSE
\VERSE Dès le matin je me tiendrai devant Vous, et je verrai que Vous n'êtes pas un Dieu qui aime l'iniquité. \EVERSE
\VERSE Le méchant n'habitera pas auprès de Vous, et les injustes ne subsisteront point devant Vos yeux. \EVERSE
\VERSE Vous haïssez tous ceux qui commettent l'iniquité; Vous perdrez tous ceux qui profèrent le mensonge. Le Seigneur aura en abomination l'homme sanguinaire et trompeur; \EVERSE
\VERSE mais moi, grâce à l'abondance de Votre miséricorde, J'entrerai dans Votre maison; j'adorerai dans Votre saint temple, pénétré de Votre crainte. \EVERSE
\VERSE Seigneur, conduisez-moi dans Votre justice; à cause de mes ennemis, rendez droite ma voie en Votre présence. \EVERSE
\VERSE Car la vérité n'est point dans leur bouche; leur coeur est vain. \EVERSE
\VERSE Leur gosier est un sépulcre ouvert; ils se sont servis de leurs langues pour tromper: jugez-les, ô Dieu! Qu'ils échouent dans leurs desseins; repoussez-les selon la multitude de leurs impiétés, parce qu'ils Vous ont irrité, Seigneur. \EVERSE
\VERSE Mais que tous ceux qui espèrent en Vous se réjouissent; ils seront éternellement dans l'allégresse, et Vous habiterez en eux. Et tous ceux qui aiment Votre Nom se glorifieront en Vous, \EVERSE
\VERSE parce que Vous bénirez le juste. Seigneur, Vous nous avez entourés de Votre amour comme d'un bouclier.

}
%%%%%%% PSAUME 6 %%%%%%%
\newcommand{\psalmvifr}{
\VERSE Pour la fin, parmi les cantiques, psaume de David, pour l'octave. \EVERSE
\VERSE Seigneur, ne me reprenez pas dans Votre fureur, et ne me châtiez pas dans Votre colère. \EVERSE
\VERSE Ayez pitié de moi, Seigneur, car je suis sans force; guérissez-moi, Seigneur, car mes os sont ébranlés. \EVERSE
\VERSE Et mon âme est toute troublée; mais Vous, Seigneur, jusques à quand...? \EVERSE
\VERSE Revenez, Seigneur, et délivrez mon âme: sauvez-moi à cause de Votre miséricorde. \EVERSE
\VERSE Car il n'y a personne qui se souvienne de Vous dans la mort; et qui donc Vous louera dans le séjour des morts? \EVERSE
\VERSE Je suis épuisé à force de gémir; je laverai toutes les nuits mon lit de mes pleurs; j'arroserai ma couche de mes larmes. \EVERSE
\VERSE Mon oeil a été troublé par la fureur; j'ai vieilli au milieu de tous mes ennemis. \EVERSE
\VERSE Eloignez-vous de moi, vous tous qui commettez l'iniquité, car le Seigneur a exaucé la voix de mes larmes. \EVERSE
\VERSE Le Seigneur a exaucé ma supplication; le Seigneur a agréé ma prière. \EVERSE
\VERSE Que tous mes ennemis rougissent et soient saisis d'une vive épouvante; qu'ils reculent promptement, et qu'ils soient bientôt confondus.

}
%%%%%%% PSAUME 7 %%%%%%%
\newcommand{\psalmviifr}{
\VERSE Psaume de David, qu'il chanta au Seigneur à cause des paroles de Chus, fils de Jémini. \EVERSE
\VERSE Seigneur mon Dieu, j'ai espéré en Vous; sauvez-moi de tous ceux qui me persécutent, et délivrez-moi; \EVERSE
\VERSE de peur qu'il ne ravisse mon âme comme un lion, s'il n'y a personne pour me délivrer et me sauver. \EVERSE
\VERSE Seigneur mon Dieu, si j'ai fait cela, s'il y a de l'iniquité dans mes mains, \EVERSE
\VERSE si j'ai rendu le mal à ceux qui m'en avaient fait, que je succombe, justement et dénué de tout, devant mes ennemis. \EVERSE
\VERSE Que l'ennemi poursuive mon âme et s'en rende maître; qu'il foule à terre ma vie, et qu'il traîne ma gloire dans la poussière. \EVERSE
\VERSE Levez-vous, Seigneur, dans Votre colère, et soyez exalté au milieu de mes ennemis. Levez-Vous, Seigneur mon Dieu, suivant le précepte que Vous avez établi; \EVERSE
\VERSE et l'assemblée des peuples Vous environnera. A cause d'elle remontez en haut. \EVERSE
\VERSE Le Seigneur juge les peuples. Jugez-moi, Seigneur, selon ma justice, et selon l'innocence qui est en moi. \EVERSE
\VERSE La malice des pécheurs prendra fin, et Vous conduirez le juste, ô Dieu, qui sondez les coeurs et les reins. \EVERSE
\VERSE Mon légitime secours me viendra du Seigneur, qui sauve ceux qui ont le coeur droit. \EVERSE
\VERSE Dieu est un juge équitable, fort et patient; est-ce qu'Il S'irrite tous les jours? \EVERSE
\VERSE Si vous ne vous convertissez, Il brandira Son glaive; Il a déjà tendu Son arc, et le tient tout prêt. \EVERSE
\VERSE Et Il y a préparé des instruments de mort; Il a rendu Ses flèches brûlantes. \EVERSE
\VERSE Voici que l'ennemi a mis au monde l'injustice; il a conçu la douleur, et a enfanté l'iniquité. \EVERSE
\VERSE Il a ouvert une fosse, et l'a creusée; et il est tombé dans cette fosse qu'il avait faite. \EVERSE
\VERSE La douleur qu'il a causée reviendra sur sa tête, et son iniquité retombera sur son front. \EVERSE
\VERSE Je rendrai gloire au Seigneur selon Sa justice, et je chanterai le Nom du Seigneur très haut.

}
%%%%%%% PSAUME 8 %%%%%%%
\newcommand{\psalmviiifr}{
\VERSE Pour la fin, pour les pressoirs, psaume de David. \EVERSE
\VERSE Seigneur, notre Maître, que Votre Nom est admirable dans toute la terre! Car Votre magnificence est élevée au-dessus des cieux. \EVERSE
\VERSE De la bouche des enfants et de ceux qui sont à la mamelle Vous avez tiré une louange parfaite contre Vos adversaires, pour détruire l'ennemi, et celui qui veut se venger. \EVERSE
\VERSE Quand je considère Vos cieux, qui sont l'ouvrage de Vos doigts, la lune et les étoiles que Vous avez créées, \EVERSE
\VERSE je m'écrie: Qu'est-ce que l'homme, pour que Vous Vous souveniez de lui? ou le fils de l'homme, pour que Vous le visitiez? \EVERSE
\VERSE Vous ne l'avez mis qu'un peu au-dessous des Anges; Vous l'avez couronné de gloire et d'honneur, \EVERSE
\VERSE et Vous l'avez établi sur les ouvrages de Vos mains. \EVERSE
\VERSE Vous avez mis toutes choses sous ses pieds, toutes les brebis, et tous les boefs, et même les animaux des champs, \EVERSE
\VERSE les oiseaux du ciel, et les poissons de la mer, qui parcourent les sentiers de l'océan. \EVERSE
\VERSE Seigneur, notre Maître, que Votre Nom est admirable dans toute la terre!

}
%%%%%%% PSAUME 9 %%%%%%%
\newcommand{\psalmixafr}{
\VERSE Pour la fin, pour les secrets du fils, psaume de David. \EVERSE
\VERSE Je Vous louerai, Seigneur, de tout mon coeur; je raconterai toutes Vos merveilles. \EVERSE
\VERSE En Vous je me réjouirai, et me livrerai à l'allégresse; je chanterai Votre Nom, ô Très-Haut; \EVERSE
\VERSE parce que Vous avez fait retourner mon ennemi en arrière. Ils vont être épuisés, et ils périront devant Votre face. \EVERSE
\VERSE Car Vous m'avez rendu justice, et Vous avez soutenu ma cause; Vous Vous êtes assis sur Votre trône, Vous qui jugez selon le droit. \EVERSE
\VERSE Vous avez châtié les nations, et l'impie a péri; Vous avez effacé leur Nom à jamais, et pour les siècles des siècles. \EVERSE
\VERSE Les glaives de l'ennemi ont perdu leur force pour toujours, et Vous avez détruit leurs villes. Leur mémoire a péri avec fracas; \EVERSE
\VERSE mais le Seigneur demeure éternellement. Il a préparé Son trône pour le jugement; \EVERSE
\VERSE et Il jugera Lui-même l'univers avec équité; Il jugera les peuples avec justice. \EVERSE
\VERSE Le Seigneur est devenu le refuge du pauvre, et Son secours au temps du besoin et de l'affliction. \EVERSE
\VERSE Qu'ils espèrent en Vous, ceux qui connaissent Votre Nom; car Vous n'avez pas abandonné ceux qui Vous cherchent, Seigneur. 

}
\newcommand{\psalmixbfr}{
\VERSE Chantez au Seigneur qui habite dans Sion: annoncez parmi les nations Ses desseins; \EVERSE
\VERSE car Celui qui recherche le sang versé S'est souvenu de Ses serviteurs; Il n'a pas oublié le cri des pauvres.\EVERSE
\VERSE Ayez pitié de moi, Seigneur; voyez l'humiliation où mes ennemis m'ont réduit, \EVERSE
\VERSE Vous qui me retirez des portes de la mort, pour que j'annonce toutes Vos louanges aux portes de la fille de Sion. \EVERSE
\VERSE Je serai transporté de joie à cause du salut que Vous m'aurez procuré. Les nations se sont enfoncées dans la fosse qu'elles avaient faite. Leur pied a été pris dans le piège qu'elles avaient caché. \EVERSE
\VERSE On reconnaîtra le Seigneur qui rend justice; le pécheur a été pris dans les oeuvres de ses mains. \EVERSE
\VERSE Que les pécheurs soient précipités dans l'enfer, et toutes les nations qui oublient Dieu. \EVERSE
\VERSE Car le pauvre ne sera pas en oubli pour toujours; la patience des pauvres ne périra pas à jamais. \EVERSE
\VERSE Levez-Vous, Seigneur; que l'homme ne triomphe pas; que les nations soient jugées devant Votre face. \EVERSE
\VERSE Seigneur, imposez-leur un maître, afin que les peuples sachent qu'ils sont hommes.
}
\newcommand{\psalmixcfr}{}
\newcommand{\psalmixdfr}{}

%%%%%%% PSAUME 10 %%%%%%%
\newcommand{\psalmxfr}{
\VERSE Pour la fin, psaume de David. \EVERSE
\VERSE Je me confie au Seigneur; comment dites-vous à mon âme: Emigrez sur la montagne comme un passereau? \EVERSE
\VERSE Car voici que les pécheurs ont tendu leur arc; ils ont préparé leurs flèches dans leur carquois, pour tirer dans l'ombre contre ceux qui ont le coeur droit. \EVERSE
\VERSE Car ce que vous aviez établi, ils l'ont détruit; mais le juste, qu'a-t-il fait? \EVERSE
\VERSE Le Seigneur est dans Son saint temple; le Seigneur a Son trône dans le Ciel. Ses yeux regardent le pauvre; Ses paupières examinent les enfants des hommes. \EVERSE
\VERSE Le Seigneur examine le juste et l'impie; or celui qui aime l'iniquité hait son âme. \EVERSE
\VERSE Il fera pleuvoir des pièges sur les pécheurs; le feu, et le soufre, et le vent des tempêtes, sont la part de leur calice. \EVERSE
\VERSE Car le Seigneur est juste, et Il aime la justice; Son visage contemple l'équité.

}
%%%%%%% PSAUME 11 %%%%%%%
\newcommand{\psalmxifr}{
\VERSE Pour la fin, pour l'octave, psaume de David. \EVERSE
\VERSE Sauvez-moi, Seigneur, car il n'y a plus de saint, car les vérités ont été diminuées par les enfants des hommes. \EVERSE
\VERSE Chacun ne dit à son prochain que des choses vaines; leurs lèvres sont trompeuses, et ils parlent avec un coeur double. \EVERSE
\VERSE Que le Seigneur détruise toutes les lèvres trompeuses, et la langue qui se vante avec jactance. \EVERSE
\VERSE Ils ont dit: Nous ferons de grandes choses par notre langue; nos lèvres ne dépendent que de nous. Qui est notre maître? \EVERSE
\VERSE A cause de la misère des malheureux et du gémissement des pauvres, Je Me lèverai maintenant, dit le Seigneur. Je procurerai leur salut; J'agirai en cela avec une entière puissance. \EVERSE
\VERSE Les paroles du Seigneur sont des paroles pures: c'est un argent éprouvé au feu, purifié dans la terre, et raffiné sept fois. \EVERSE
\VERSE C'est Vous, Seigneur, qui nous garderez, et qui nous préserverez à jamais de cette génération. \EVERSE
\VERSE Les impies vont et viennent à l'entour. Selon la profondeur de Votre sagesse, Vous avez multiplié les enfants des hommes.

}
%%%%%%% PSAUME 12 %%%%%%%
\newcommand{\psalmxiifr}{
\VERSE Pour la fin, psaume de David. Jusques à quand, Seigneur, m'oublierez-Vous sans cesse? Jusques à quand détournerez-Vous de moi Votre face? \EVERSE
\VERSE Jusques à quand remplirai-je mon âme de projets, et mon coeur chaque jour de chagrin? \EVERSE
\VERSE Jusques à quand mon ennemi sera-t-il élevé au-dessus de moi? \EVERSE
\VERSE Regardez, et exaucez-moi, Seigneur mon Dieu. Eclairez mes yeux, afin que je ne m'endorme jamais dans la mort; \EVERSE
\VERSE de peur que mon ennemi me dise: J'ai eu l'avantage contre lui. Ceux qui me persécutent seront dans l'allégresse si je suis ébranlé; \EVERSE
\VERSE mais j'ai espéré en Votre miséricorde. Mon coeur sera transporté de joie à cause de Votre salut. Je chanterai le Seigneur qui m'a comblé de biens, et je célébrerai le Nom du Seigneur Très-Haut.

}
%%%%%%% PSAUME 13 %%%%%%%
\newcommand{\psalmxiiifr}{
\VERSE Pour la fin, psaume de David. L'insensé a dit dans son coeur: Il n'y a point de Dieu. Ils se sont corrompus, et sont devenus abominables dans leurs tendances. Il n'y en a point qui fasse le bien, il n'y en a pas un seul. \EVERSE
\VERSE Le Seigneur a regardé du haut du Ciel sur les enfants des hommes, pour voir s'il y a quelqu'un qui soit intelligent ou qui cherche Dieu. \EVERSE
\VERSE Tous se sont détournés, ils sont tous devenus inutiles. Il n'y en a point qui fasse le bien, il n'y en a pas un seul. Leur gosier est un sépulcre ouvert; ils se servent de leurs langues pour tromper; le venin des aspics est sous leurs lèvres. Leur bouche est remplie de malédiction et d'amertume; leurs pieds sont agiles pour répandre le sang. L'affliction et le malheur sont dans leurs voies, et ils n'ont pas connu la voie de la paix; la crainte de Dieu n'est pas devant leurs yeux. \EVERSE
\VERSE Ne comprendront-ils pas, tous ces hommes qui commettent l'iniquité, qui dévorent Mon peuple comme un morceau de pain? \EVERSE
\VERSE Ils n'ont pas invoqué le Seigneur; ils ont tremblé de frayeur là où il n'y avait rien à craindre. \EVERSE
\VERSE Car le Seigneur est avec la race des justes; vous vous êtes moqués du dessein du pauvre, parce que le Seigneur est son espérance. \EVERSE
\VERSE Qui procurera de Sion le salut d'Israël? Quand le Seigneur aura mis fin à la captivité de Son peuple, Jacob sera dans l'allégresse, et Israël dans la joie.

}
%%%%%%% PSAUME 14 %%%%%%%
\newcommand{\psalmxivfr}{
\VERSE Psaume de David. Seigneur, qui habitera dans Votre tabernacle? ou qui reposera sur Votre montagne sainte? \EVERSE
\VERSE Celui qui vit sans tache, et qui pratique la justice; \EVERSE
\VERSE qui dit la vérité dans son coeur; qui n'a point usé de tromperie dans ses paroles; qui n'a pas fait de mal à son prochain, et qui n'a point accueilli de calomnie contre ses frères. \EVERSE
\VERSE Le méchant est compté pour rien à ses yeux; mais il honore ceux qui craignent le Seigneur. Il fait serment à son prochain et ne le trompe pas; \EVERSE
\VERSE il ne donne point son argent à usure, et ne reçoit pas de présents contre l'innocent. Celui qui se conduit ainsi ne sera jamais ébranlé.

}
%%%%%%% PSAUME 15 %%%%%%%
\newcommand{\psalmxvfr}{
\VERSE Inscription du titre, de David. Conservez-moi, Seigneur, car j'ai espéré en Vous. \EVERSE
\VERSE J'ai dit au Seigneur: Vous êtes mon Dieu, et Vous n'avez nul besoin de mes biens. \EVERSE
\VERSE Il a fait éclater toutes mes dispositions bienveillantes envers les Saints qui sont sur Sa terre. \EVERSE
\VERSE Leurs infirmités se sont multipliées, et ensuite ils ont couru avec vitesse. Je ne les réunirai point dans des assemblées de sang, et je ne me souviendrai plus de leurs noms pour les prononcer. \EVERSE
\VERSE Le Seigneur est la part de mon héritage et de ma coupe; c'est Vous, Seigneur, qui me rendrez mon héritage. \EVERSE
\VERSE Le cordeau est tombé pour moi en des lieux magnifiques, car mon héritage est excellent. \EVERSE
\VERSE Je bénirai le Seigneur qui m'a donné l'intelligence; de plus, jusque dans la nuit même mes reins m'y ont excité. \EVERSE
\VERSE Je prenais soin d'avoir toujours le Seigneur devant mes yeux; car Il est à ma droite, pour que je ne sois pas ébranlé. \EVERSE
\VERSE C'est pourquoi mon coeur s'est réjoui, et ma langue a tressailli d'allégresse; de plus, ma chair même se reposera dans l'espérance. \EVERSE
\VERSE Car Vous n'abandonnerez pas mon âme dans l'enfer, et Vous ne souffrirez pas que Votre Saint voie la corruption. \EVERSE
\VERSE Vous m'avez fait connaître les voies de la vie; Vous me comblerez de joie par Votre visage: il y a des délices sans fin à Votre droite.

}
%%%%%%% PSAUME 16 %%%%%%%
\newcommand{\psalmxvifr}{
\VERSE Prière de David. Exaucez, Seigneur, ma justice; soyez attentif à ma supplication. Prêtez l'oreille à ma prière, qui ne part point de lèvres trompeuses. \EVERSE
\VERSE Que mon jugement procède de Votre visage; que Vos yeux regardent l'équité. \EVERSE
\VERSE Vous avez éprouvé mon coeur, et Vous l'avez visité durant la nuit; Vous m'avez éprouvé par le feu, et l'iniquité ne s'est point trouvée en moi. \EVERSE
\VERSE Afin que ma bouche ne célèbre point les oeuvres des hommes, j'ai eu soin, à cause des paroles de Vos lèvres, de marcher par des voies rudes. \EVERSE
\VERSE Affermissez mes pas dans Vos sentiers, afin que mes pieds ne soient point ébranlés. \EVERSE
\VERSE J'ai crié, mon Dieu, parce que Vous m'avez exaucé: inclinez vers moi Votre oreille, et exaucez mes paroles. \EVERSE
\VERSE Faites éclater Vos miséricordes, Vous qui sauvez ceux qui espèrent en Vous. \EVERSE
\VERSE Contre ceux qui résistent à Votre droite, défendez-moi comme la prunelle de l'oeil. Protégez-moi à l'ombre de Vos ailes, \EVERSE
\VERSE contre les impies qui m'affligent. Mes ennemis ont environné mon âme; \EVERSE
\VERSE ils ont fermé leurs entrailles; leur bouche a parlé avec orgueil. \EVERSE
\VERSE Après m'avoir repoussé, maintenant ils m'assaillent; ils fixent leurs yeux sur moi pour me renverser à terre. \EVERSE
\VERSE Ils m'ont saisi comme un lion prêt à ravir sa proie, et comme un lionceau qui habite dans les fourrés. \EVERSE
\VERSE Levez-Vous, Seigneur; prévenez-le, et faites-le tomber; délivrez mon âme de l'impie, et arrachez Votre glaive \EVERSE
\VERSE aux ennemis de Votre main. Seigneur, séparez-les dès leur vie même du petit nombre de Vos fidèles qui sont sur la terre; leur ventre est rempli de Vos trésors. Ils sont rassasiés d'enfants, et ils laissent le reste de leurs biens à leurs petits enfants. \EVERSE
\VERSE Pour moi, c'est par la justice que je serai admis en Votre présence; je serai rassasié lorsque apparaîtra Votre gloire.

}
%%%%%%% PSAUME 17 %%%%%%%
\newcommand{\psalmxviiafr}{
\VERSE Pour la fin, de David, serviteur du Seigneur, qui adressa au Seigneur les paroles de ce cantique, au jour où le Seigneur le délivra de la main de tous ses ennemis et de la main de Saül; et il dit: \EVERSE
\VERSE Je Vous aimerai, Seigneur, Vous qui êtes ma force. \EVERSE
\VERSE Le Seigneur est mon ferme appui, mon refuge et mon libérateur. Mon Dieu est mon secours, et j'espérerai en Lui. Il est mon protecteur, et la corne de mon salut, et mon défenseur. \EVERSE
\VERSE J'invoquerai le Seigneur en Le louant, et je serai délivré de mes ennemis. \EVERSE
\VERSE Les douleurs de la mort m'ont environné, et les torrents de l'iniquité m'ont rempli de trouble. \EVERSE
\VERSE Les douleurs de l'enfer m'ont entouré, les filets de la mort m'ont saisi. \EVERSE
\VERSE Dans mon affliction j'ai invoqué le Seigneur, et j'ai crié vers mon Dieu. Et de Son saint temple Il a entendu ma voix; et mon cri a pénétré en Sa présence jusqu'à Ses oreilles. \EVERSE
\VERSE La terre a été ébranlée et a tremblé; les fondements des montagnes ont été secoués et agités, parce qu'Il S'est irrité contre elles. \EVERSE
\VERSE La fumée a monté à cause de Sa colère, et le feu s'est allumé par Ses regards; des charbons en ont été embrasés. \EVERSE
\VERSE Il a abaissé les cieux, et est descendu; un nuage obscur était sous Ses pieds. \EVERSE
\VERSE Et Il est monté sur les chérubins, et Il S'est envolé; Il a volé sur les ailes des vents. \EVERSE
\VERSE Et Il a fait des ténèbres le lieu de Sa retraite; Sa tente était tout autour de Lui, l'eau ténébreuse des nuées de l'air. \EVERSE
\VERSE Devant l'éclat de Sa présence, les nuées se sont élancées; de la grêle et des charbons de feu. \EVERSE
\VERSE Et le Seigneur a tonné du haut du Ciel, et le Très-Haut a fait ntendre Sa voix; de la grêle et des charbons de feu. \EVERSE
\VERSE Et Il a tiré Ses flèches, et Il les a dispersés; Il a ultiplié les éclairs, et Il les a mis en déroute. \EVERSE
\VERSE Alors les sources des eaux ont paru, et les fondements de la erre ont été mis à nu Votre menace, Seigneur, et par le souffle impétueux de Votre olère. \EVERSE
\VERSE Il a tendu d'en haut Sa main et Il m'a pris, et Il m'a tiré e l'inondation des eaux. \EVERSE
\VERSE Il m'a arraché à mes très puissant ennemis, et à ceux qui me aïssaient, car ils étaient plus forts que moi. \EVERSE
\VERSE Ils m'ont attaqué les premiers au jour de mon affliction, et e Seigneur S'est fait mon protecteur. \EVERSE
\VERSE Il m'a retiré et mis au large; Il m'a sauvé parce qu'Il 'aimait. \EVERSE
\VERSE Et le Seigneur me rendra selon ma justice; Il me écompensera selon la pureté de mes mains. \EVERSE
\VERSE Car j'ai gardé les voies du Seigneur, et n'ai rien fait 'impie qui m'éloignât de mon Dieu. \EVERSE
\VERSE Car tous Ses jugements sont présents devant moi, et je n'ai oint rejeté Ses préceptes loin de moi. \EVERSE
\VERSE Et je serai sans tache envers Lui, et je me garderai de mon niquité. \EVERSE
\VERSE Et le Seigneur me rendra selon ma justice, et selon la ureté de mes mains qui est présente à Ses yeux. \EVERSE
\VERSE Avec celui qui est saint Vous serez saint, et avec l'homme ui est innocent Vous serez innocent. \EVERSE
\VERSE Avec celui qui est pur Vous serez pur, et avec le pervers ous agirez avec détours. \EVERSE
\VERSE Car Vous sauverez le peuple qui est humble, et Vous humilierez les yeux des superbes. \EVERSE
\VERSE Car c'est Vous, Seigneur, qui allumez ma lampe; mon Dieu, éclairez mes ténèbres. \EVERSE
\VERSE Car par Vous je serai arraché à la tentation, et par mon Dieu je franchirai le mur. \EVERSE
\VERSE La voie de mon Dieu est pure; les paroles du Seigneur sont éprouvées au feu; Il est le protecteur de tous ceux qui espèrent en Lui.
}
\newcommand{\psalmxviibfr}{
\VERSE Car qui est Dieu, si ce n'est le Seigneur? et qui est Dieu, si ce n'est notre Dieu? \EVERSE
\VERSE Le Dieu qui m'a ceint de force, et qui a rendu ma voie immaculée; \EVERSE
\VERSE qui a fait mes pieds agiles comme ceux des cerfs, et m'a établi sur les hauts lieux; \EVERSE
\VERSE qui enseigne à mes mains le combat, et c'est Vous qui avez fait de mes bras comme un arc d'airain; \EVERSE
\VERSE et Vous m'avez donné Votre protection pour me sauver, et Votre droite m'a soutenu; et Vos leçons m'ont corrigé jusqu'à la fin, et ces leçons continuent de m'instruire. \EVERSE
\VERSE Vous avez élargi la voie sous mes pas, et mes pieds ne se sont point affaiblis. \EVERSE
\VERSE Je poursuivrai mes ennemis, et je les atteindrai; et je ne m'en retournerai pas qu'ils ne soient anéantis. \EVERSE
\VERSE Je les briserai, et ils ne pourront se tenir debout; ils tomberont sous mes pieds. \EVERSE
\VERSE Car Vous m'avez ceint de force pour la guerre, et Vous avez abattu sous moi ceux qui s'élevaient contre moi. \EVERSE
\VERSE Et Vous avez fait tourner le dos à mes ennemis devant moi, et Vous avez exterminé ceux qui me haïssaient. \EVERSE
\VERSE Ils ont crié, et il n'y avait personne pour les sauver; ils ont appelé le Seigneur, et Il ne les a pas exaucés. \EVERSE
\VERSE Et je les briserai comme la poussière que le vent emporte; je les écraserai comme la boue des rues. \EVERSE
\VERSE Vous me délivrerez des dissensions du peuple; Vous m'établirez chef des nations. \EVERSE
\VERSE Un peuple que je ne connaissais pas m'a été assujetti; il m'a obéi au premier ordre. \EVERSE
\VERSE Les fils de l'étranger m'ont menti; les fils de l'étranger sont en défaillance, et ils sont sortis en chancelant de leurs sentiers. \EVERSE
\VERSE Vive le Seigneur, et béni soit mon Dieu! et que le Dieu de mon salut soit exalté! \EVERSE
\VERSE O Dieu, qui prenez soin de me venger, et qui me soumettez les peuples; Vous qui me délivrez de mes ennemis furieux. \EVERSE
\VERSE Et Vous m'élèverez au-dessus de ceux qui se dressent contre moi; Vous m'arracherez des mains de l'homme inique. \EVERSE
\VERSE C'est pourquoi je Vous louerai, Seigneur, parmi les nations, et je chanterai un cantique à la gloire de Votre Nom; \EVERSE
\VERSE à la gloire d'un Dieu qui procure de merveilleuses délivrances à Son Roi, et qui fait miséricorde à David Son oint, et à sa postérité jusqu'à la fin des siècles.

}
%%%%%%% PSAUME 18 %%%%%%%
\newcommand{\psalmxviiifr}{
\VERSE Pour la fin, psaume de David. \EVERSE
\VERSE Les cieux racontent la gloire de Dieu, et le firmament publie les oeuvres de Ses mains. \EVERSE
\VERSE Le jour proclame ce message au jour, et la nuit en donne connaissance à la nuit. \EVERSE
\VERSE Ce ne sont point des paroles, ce n'est pas un langage dont la voix ne soit pas entendue. \EVERSE
\VERSE Leur bruit s'est répandu dans toute la terre, et leurs accents jusqu'aux extrémités du monde. \EVERSE
\VERSE Il a établi Sa tente dans le soleil, qui est lui-même semblable à un époux sortant de sa chambre nuptiale. Il s'est élancé comme un géant pour fournir sa carrière. \EVERSE
\VERSE Il sort de l'extrémité du ciel, et sa course va jusqu'à l'autre extrémité, et il n'y a personne qui se dérobe à sa chaleur. \EVERSE
\VERSE La loi du Seigneur est sans tache, elle restaure les âmes; le témoignage du Seigneur est fidèle, il donne la sagesse aux petits. \EVERSE
\VERSE Les justices du Seigneur sont droites, elles réjouissent les coeurs; le précepte du Seigneur est lumineux, il éclaire les yeux. \EVERSE
\VERSE La crainte du Seigneur est sainte, elle subsiste à jamais; les jugements du Seigneur sont vrais, ils se justifient par eux-mêmes. \EVERSE
\VERSE Ils sont plus désirables que l'or et que beaucoup de pierres précieuses;ils sont plus doux que le miel, et qu'un rayon plein de miel. \EVERSE
\VERSE Aussi Votre serviteur les observe; à les garder, on trouve une grande récompense. \EVERSE
\VERSE Qui connaît ses fautes? Purifiez-moi de celles qui sont cachées en moi, \EVERSE
\VERSE et préservez Votre serviteur de la corruption des étrangers. S'ils ne me dominent point, alors je serai sans tache, et purifié d'un très grand péché. \EVERSE
\VERSE Et alors les paroles de ma bouche pourront Vous plaire, et la méditation de mon coeur sera toujours en Votre présence. Seigneur, Vous êtes mon secours et mon rédempteur.

}
%%%%%%% PSAUME 19 %%%%%%%
\newcommand{\psalmxixfr}{
\VERSE Pour la fin, psaume de David. \EVERSE
\VERSE Que le Seigneur vous exauce au jour de l'affliction; que le Nom du Dieu de Jacob vous protège. \EVERSE
\VERSE Qu'Il vous envoie du secours de Son sanctuaire, et que de Sion Il vous défende. \EVERSE
\VERSE Qu'Il Se souvienne de tous vos sacrifices, et que votre holocauste Lui soit agréable. \EVERSE
\VERSE Qu'Il vous donne ce que votre coeur désire, et qu'Il accomplisse tous vos desseins. \EVERSE
\VERSE Nous nous réjouirons de votre salut, et nous nous glorifierons au Nom de notre Dieu. \EVERSE
\VERSE Que le Seigneur exauce toutes vos demandes. J'ai reconnu maintenant que le Seigneur a sauvé Son oint. Il l'exaucera du Ciel, Sa sainte demeure; Sa droite toute-puissante produira le salut. \EVERSE
\VERSE Ceux-là se confient dans leurs chars, et ceux-ci dans leurs chevaux; mais nous, nous invoquerons le Nom du Seigneur notre Dieu. \EVERSE
\VERSE Eux ils ont été comme liés, et ils sont tombés; mais nous, nous nous sommes relevés, et nous restons debout. \EVERSE
\VERSE Seigneur, sauvez le roi, et exaucez-nous au jour où nous Vous invoquerons.

}
%%%%%%% PSAUME 20 %%%%%%%
\newcommand{\psalmxxfr}{
\VERSE Pour la fin, psaume de David. \EVERSE
\VERSE Seigneur, le roi se réjouira dans Votre force, et il tressaillira d'une vive allégresse, parce que Vous l'aurez sauvé. \EVERSE
\VERSE Vous lui avez accordé le désir de son coeur, et Vous ne l'avez point frustré de la demande de ses lèvres. \EVERSE
\VERSE Car Vous l'avez prévenu des plus douces bénédictions; Vous avez mis sur sa tête une couronne de pierres précieuses. \EVERSE
\VERSE Il vous a demandé la vie, et Vous lui avez accordé des jours qui dureront dans les siècles des siècles. \EVERSE
\VERSE Sa gloire est grande, grâce à Votre salut; Vous le couvrirez de gloire et d'un honneur immense. \EVERSE
\VERSE Car Vous ferez de lui une source de bénédictions perpétuelles; Vous le comblerez de joie en lui montrant Votre visage. \EVERSE
\VERSE Car le roi espère au Seigneur, et la miséricorde du Très-Haut le rendra inébranlable. \EVERSE
\VERSE Que votre main atteigne tous Vos ennemis; que votre droite trouve tous ceux qui vous haïssent. \EVERSE
\VERSE Vous en ferez comme une fournaise ardente, au temps où vous montrerez votre visage irrité; le Seigneur dans Sa colère les remplira de trouble, et le feu les dévorera. \EVERSE
\VERSE Vous exterminerez leur fruit de dessus la terre, et leur race d'entre les enfants des hommes. \EVERSE
\VERSE Car ils ont fait tomber des maux sur vous; ils ont formé des desseins qu'ils n'ont pu exécuter. \EVERSE
\VERSE Car vous leur ferez tourner le dos; vous préparerez leur visage à recevoir les traits qui vous restent. \EVERSE
\VERSE Levez-Vous, Seigneur, dans Votre force; nous chanterons et nous célébrerons Vos actions d'éclat.

}
%%%%%%% PSAUME 21 %%%%%%%
\newcommand{\psalmxxifr}{
\VERSE Pour la fin, pour le secours du matin, psaume de David. \EVERSE
\VERSE O Dieu, mon Dieu, regardez-moi; pourquoi m'avez-Vous abandonné? La voix de mes péchés éloigne de moi le salut. \EVERSE
\VERSE Mon Dieu, je crierai pendant le jour, et Vous ne m'exaucerez pas; et pendant la nuit, et l'on ne me l'imputera point à folie. \EVERSE
\VERSE Mais Vous, Vous habitez dans le sanctuaire; Vous qui êtes la louange d'Israël. \EVERSE
\VERSE Nos pères ont espéré en Vous; ils ont espéré, et Vous les avez délivrés. \EVERSE
\VERSE Ils ont crié vers Vous, et ils ont été sauvés; ils ont espéré en Vous, et ils n'ont point été confondus. \EVERSE
\VERSE Mais moi, je suis un ver, et non un homme; l'opprobre des hommes, et le rebut du peuple. \EVERSE
\VERSE Tous ceux qui m'ont vu se sont moqués de moi; de leurs lèvres ils ont proféré l'outrage, et ils ont branlé la tête. \EVERSE
\VERSE Il a espéré au Seigneur, qu'Il le délivre; qu'Il le sauve, puisqu'Il l'aime. \EVERSE
\VERSE Oui, c'est Vous qui m'avez tiré du ventre de ma mère; Vous êtes mon espérance depuis le temps où je suçais ses mamelles. \EVERSE
\VERSE Au sortir de son sein, j'ai été jeté sur Vos genoux; depuis que j'ai quitté ses entrailles, c'est Vous qui êtes mon Dieu. \EVERSE
\VERSE Ne Vous retirez pas de moi, car la tentation est proche, et il n'y a personne qui me secoure. \EVERSE
\VERSE Des jeunes taureaux nombreux m'ont environné; des taureaux gras m'ont assiégé. \EVERSE
\VERSE Ils ont ouvert leur bouche sur moi, comme un lion ravisseur et rugissant. \EVERSE
\VERSE Je me suis répandu comme l'eau, et tous mes os se sont disloqués. Mon coeur est devenu comme de la cire fondue au milieu de mes entrailles. \EVERSE
\VERSE Ma force s'est desséchée comme un tesson, et ma langue s'est attachée à mon palais; et Vous m'avez conduit à la poussière du tombeau. \EVERSE
\VERSE Car des chiens nombreux m'ont environné; une bande de scélérats m'a assiégé. Ils ont percé mes mains et mes pieds, \EVERSE
\VERSE ils ont compté tous mes os. Ils m'ont considéré et contemplé. \EVERSE
\VERSE Ils se sont partagé mes vêtements, et ils ont jeté le sort sur ma tunique. \EVERSE
\VERSE Mais Vous, Seigneur, n'éloignez pas de moi Votre secours; prenez soin de ma défense. \EVERSE
\VERSE Délivrez, ô Dieu, mon âme du glaive, et mon unique du pouvoir du chien. \EVERSE
\VERSE Sauvez-moi de la gueule du lion, et sauvez ma faiblesse des cornes des licornes. \EVERSE
\VERSE J'annoncerai Votre Nom à mes frères; je Vous louerai au milieu de l'assemblée. \EVERSE
\VERSE Vous qui craignez le Seigneur, louez-Le; toute la race de Jacob, glorifiez-Le. \EVERSE
\VERSE Que toute la race d'Israël Le craigne, parce qu'Il n'a pas méprisé ni dédaigné la supplication du pauvre, et qu'Il n'a point détourné de moi Son visage; mais qu'Il m'a exaucé lorsque je criais vers Lui. \EVERSE
\VERSE Je Vous adresserai ma louange dans une grande assemblée; j'acquitterai mes voeux en présence de ceux qui Le craignent. \EVERSE
\VERSE Les pauvres mangeront et seront rassassiés, et ceux qui cherchent le Seigneur Le loueront; leurs coeurs vivront dans les siècles des siècles. \EVERSE
\VERSE Toutes les extrémités de la terre se souviendront du Seigneur et se convertiront à Lui; et toutes les familles des nations L'adoreront en Sa présence; \EVERSE
\VERSE car le règne appartient au Seigneur, et Il dominera sur les nations. \EVERSE
\VERSE Tous les riches de la terre ont mangé et adoré; tous ceux qui descendent dans la terre se prosterneront devant Lui. \EVERSE
\VERSE Et mon âme vivra pour Lui, et ma race Le servira. \EVERSE
\VERSE La postérité qui doit venir sera annoncée au Seigneur, et les cieux annonceront Sa justice au peuple qui doit naître, et que le Seigneur a fait.

}
%%%%%%% PSAUME 22 %%%%%%%
\newcommand{\psalmxxiifr}{
\VERSE Psaume de David. C'est le Seigneur qui me conduit, et rien ne pourra me manquer. \EVERSE
\VERSE Il m'a établi dans un lieu de pâturages. Il m'a amené près d'une eau foritifiante, \EVERSE
\VERSE Il a fait revenir mon âme. Il m'a conduit par les sentiers de la justice, à cause de Son Nom. \EVERSE
\VERSE Aussi, quand même je marcherais au milieu de l'ombre de la mort, je ne craindrais aucun mal, car Vous êtes avec moi. Votre houlette et Votre bâton m'ont consolé. \EVERSE
\VERSE Vous avez préparé devant moi une table contre ceux qui me persécutent. Vous avez oint ma tête d'huile, et que mon calice enivrant est admirable! \EVERSE
\VERSE Et Votre miséricorde me suivra tous les jours de ma vie, pour que j'habite dans la maison du Seigneur durant de longs jours.

}
%%%%%%% PSAUME 23 %%%%%%%
\newcommand{\psalmxxiiifr}{
\VERSE Pour le premier jour de la semaine, psaume de David. Au Seigneur est la terre et tout ce qu'elle renferme, le monde et tous ceux qui l'habitent. \EVERSE
\VERSE Car c'est Lui qui l'a fondé sur les mers, et qui l'a établi sur les fleuves. \EVERSE
\VERSE Qui montera sur la montagne du Seigneur? ou qui se tiendra dans Son lieu saint? \EVERSE
\VERSE Celui qui a les mains innocentes et le coeur pur, qui n'a pas livré son âme à la vanité, ni fait à son prochain un serment trompeur. \EVERSE
\VERSE Celui-là recevra la bénédiction du Seigneur, et la miséricorde de Dieu, son Sauveur. \EVERSE
\VERSE Telle est la race de ceux qui Le cherchent, de ceux qui cherchent la face du Dieu de Jacob. \EVERSE
\VERSE Levez vos portes, ô princes, et élevez-vous, portes éternelles, et le roi de gloire entrera. \EVERSE
\VERSE Qui est ce roi de gloire? C'est le Seigneur fort et puissant, le Seigneur puissant dans les combats. \EVERSE
\VERSE Levez vos portes, ô princes, et élevez-vous, portes éternelles, et le roi de gloire entrera. \EVERSE
\VERSE Quel est ce roi de gloire? Le Seigneur des armées est Lui-même ce roi de gloire.

}
%%%%%%% PSAUME 24 %%%%%%%
\newcommand{\psalmxxivfr}{
\VERSE Pour la fin, psaume de David. Vers Vous, Seigneur, j'ai élevé mon âme; \EVERSE
\VERSE mon Dieu, je mets ma confiance en Vous; que je n'aie pas à rougir. \EVERSE
\VERSE Et que mes ennemis ne se moquent point de moi; car tous ceux qui espèrent en Vous ne seront pas confondus. \EVERSE
\VERSE Qu'ils soient confondus, tous ceux qui commettent l'iniquité sans raison. Seigneur, montrez-moi Vos voies, et enseignez-moi Vos sentiers. \EVERSE
\VERSE Conduisez-moi dans Votre vérité, et instruisez-moi, car Vous êtes le Dieu mon Sauveur, et j'ai espéré en Vous tout le jour. \EVERSE
\VERSE Souvenez-Vous de Vos bontés, Seigneur, et de Vos miséricordes qui datent des siècles passés. \EVERSE
\VERSE Ne Vous souvenez pas des fautes de ma jeunesse, ni de mes ignorances. Souvenez-Vous de moi selon Votre miséricorde, à cause de Votre bonté, Seigneur. \EVERSE
\VERSE Le Seigneur est doux et droit; c'est pour cela qu'Il montrera aux pécheurs leur voie. \EVERSE
\VERSE Il conduira dans la justice ceux qui sont dociles; Il enseignera Ses voies à ceux qui sont doux. \EVERSE
\VERSE Toutes les voies du Seigneur sont miséricorde et vérité, pour ceux qui recherchent Son testament et Ses préceptes. \EVERSE
\VERSE A cause de Votre Nom, Seigneur, Vous me pardonnerez mon péché; car il est grand. \EVERSE
\VERSE Quel est l'homme qui craint le Seigneur? Il lui fixe une loi dans la voie qu'il a choisie. \EVERSE
\VERSE Son âme se reposera parmi les biens, et sa race aura la terre en héritage. \EVERSE
\VERSE Le Seigneur est le ferme appui de ceux qui Le craignent, et Il leur manifestera Son alliance. \EVERSE
\VERSE Mes yeux sont constamment tournés vers le Seigneur; car c'est Lui qui retirera mes pieds du filet. \EVERSE
\VERSE Regardez-moi, et ayez pitié de moi; car je suis délaissé et pauvre. \EVERSE
\VERSE Les tribulations de mon coeur se sont multipliées; tirez-moi de mes angoisses. \EVERSE
\VERSE Voyez mon humiliation et ma peine,et remettez-moi tous mes péchés. \EVERSE
\VERSE Voyez combien mes ennemis se sont multipliés, et de quelle haine injuste ils me haïssent. \EVERSE
\VERSE Gardez mon âme, et délivrez-moi; que je n'ai pas à rougir pour avoir espéré en Vous. \EVERSE
\VERSE Les hommes innocents et droits se sont attachés à moi, parce que j'ai eu confiance en Vous. \EVERSE
\VERSE O Dieu, délivrez Israël de toutes ses tribulations.

}
%%%%%%% PSAUME 25 %%%%%%%
\newcommand{\psalmxxvfr}{
\VERSE Pour la fin, psaume de David. Jugez-moi, Seigneur, parce que j'ai marché dans mon innocence; et comme j'espère au Seigneur, je ne serai point affaibli. \EVERSE
\VERSE Eprouvez-moi, Seigneur, et sondez-moi; passez au feu mes reins et mon coeur. \EVERSE
\VERSE Car Votre miséricorde est devant mes yeux, et je me suis complu dans Votre vérité. \EVERSE
\VERSE Je ne me suis point assis dans l'assemblée de la vanité, et je n'entrerai pas avec les artisans d'iniquité. \EVERSE
\VERSE Je hais l'assemblée des méchants, et je ne m'assoierai point avec les impies. \EVERSE
\VERSE Je laverai mes mains parmi les innocents; et je me tiendrai autour de Votre autel, Seigneur, \EVERSE
\VERSE pour entendre la voix de Vos louanges, et pour raconter toutes Vos merveilles. \EVERSE
\VERSE Seigneur, j'ai aimé la beauté de Votre maison, et le lieu où habite Votre gloire. \EVERSE
\VERSE Ne perdez pas, ô Dieu, mon âme avec les impies, ni ma vie avec les hommes de sang; \EVERSE
\VERSE qui ont l'iniquité dans les mains, et dont la droite est remplie de présents. \EVERSE
\VERSE Pour moi j'ai marché dans mon innocence; délivrez-moi et ayez pitié de moi. \EVERSE
\VERSE Mon pied s'est tenu dans le droit chemin; je Vous bénirai, Seigneur, dans les assemblées.

}
%%%%%%% PSAUME 26 %%%%%%%
\newcommand{\psalmxxvifr}{
\VERSE Psaume de David, avant qu'il fût oint. Le Seigneur est ma lumière et mon salut; qui craindrai-je? Le Seigneur est le défenser de ma vie; devant qui tremblerai-je? \EVERSE
\VERSE Lorsque les méchants s'approchent de moi pour dévorer ma chair, ces ennemis qui me persécutent ont été eux-mêmes affaiblis et sont tombés. \EVERSE
\VERSE Qu'une armée campe contre moi, mon coeur ne craindra pas. Que le combat s'engage contre moi, c'est alors même que j'espérerai. \EVERSE
\VERSE Il est une chose que j'ai demandée au Seigneur, et je la rechercherai uniquement; c'est d'habiter dans la maison du Seigneur tous les jours de ma vie, pour contempler les délices du Seigneur et visiter Son temple. \EVERSE
\VERSE Car Il m'a caché dans Son tabernacle; au jour de l'affliction Il m'a protégé dans le secret de Son tabernacle. \EVERSE
\VERSE Il m'a élevé sur la pierre, et maintenant Il a élevé ma tête au-dessus de mes ennemis. J'ai entouré l'autel et j'ai immolé dans Son tabernacle une victime avec des cris de joie; je chanterai et je dirai une hymne au Seigneur. \EVERSE
\VERSE Exaucez, Seigneur, ma voix, qui a crié vers Vous; ayez pitié de moi, et exaucez-moi. \EVERSE
\VERSE Mon coeur Vous a dit: Mes yeux Vous ont cherché; Votre visage, Seigneur, je le chercherai. \EVERSE
\VERSE Ne détournez pas de moi Votre face; ne Vous retirez pas de Votre serviteur, dans Votre colère. Soyez mon aide; ne m'abandonnez pas, et ne me méprisez pas, ô Dieu mon Sauveur. \EVERSE
\VERSE Car mon père et ma mère m'ont abandonné; mais le Seigneur m'a recueilli. \EVERSE
\VERSE Seigneur, enseignez-moi Votre voie, et conduisez-moi dans le droit sentier à cause de mes ennemis. \EVERSE
\VERSE Ne me livrez pas à la merci de ceux qui me persécutent; des témoins iniques se sont élevés contre moi, et l'iniquité a menti contre elle-même. \EVERSE
\VERSE Je crois que je verrai les biens du Seigneur dans la terre des vivants. \EVERSE
\VERSE Attends le Seigneur, agis avec courage; que ton coeur soit ferme, et espère au Seigneur.

}
%%%%%%% PSAUME 27 %%%%%%%
\newcommand{\psalmxxviifr}{
\VERSE Je crierai vers Vous, Seigneur; mon Dieu, ne gardez pas le silence à mon égard, de peur que, si Vous ne me répondez pas, je ne sois semblable à ceux qui descendent dans la fosse. \EVERSE
\VERSE Exaucez, Seigneur, la voix de ma supplication, quand je Vous prie, quand je lève mes mains vers Votre saint temple. \EVERSE
\VERSE Ne m'entraînez pas avec les pécheurs; et ne me perdez pas avec ceux qui commettent l'iniquité; qui parlent de paix avec leur prochain, et qui ont la méchanceté dans leurs coeurs. \EVERSE
\VERSE Rendez-leur selon leurs oeuvres, et selon la malignité de leurs desseins. Traitez-les selon les oeuvres de leurs mains; donnez-leur le salaire qu'ils méritent. \EVERSE
\VERSE Car ils n'ont pas compris les oeuvres du Seigneur et les oeuvres de Ses mains; Vous les détruirez, et ne les rétablirez pas. \EVERSE
\VERSE Béni soit le Seigneur, car Il a exaucé la voix de ma supplication. \EVERSE
\VERSE Le Seigneur est mon aide et mon protecteur; mon coeur a espéré en Lui, et j'ai été secouru. Et ma chair a refleuri; aussi Le louerai-je de tout mon coeur. \EVERSE
\VERSE Le Seigneur est la force de Son peuple, et le protecteur qui ménage les délivrances à Son oint. \EVERSE
\VERSE Sauvez Votre peuple, Seigneur, et bénissez Votre héritage; conduisez-les, et exaltez-les à jamais.

}
%%%%%%% PSAUME 28 %%%%%%%
\newcommand{\psalmxxviiifr}{
\VERSE Pour la fin de la fête des tabernacles. Offrez au Seigneur, enfants de Dieu, offrez au Seigneur les petits des béliers. \EVERSE
\VERSE Offrez au Seigneur la gloire et l'honneur; offrez au Seigneur la gloire due à Son Nom; adorez le Seigneur dans Son saint parvis. \EVERSE
\VERSE La voix du Seigneur est au-dessus des eaux; le Dieu de majesté a tonné; le Seigneur est au-dessus des grandes eaux. \EVERSE
\VERSE La voix du Seigneur est puissante; la voix du Seigneur est majestueuse. \EVERSE
\VERSE La voix du Seigneur brise les cèdres, et le Seigneur brisera les cèdres du Liban. \EVERSE
\VERSE Il les mettrez en pièces comme un jeune taureau du Liban, et le bien-aimé est comme le petit des licornes. \EVERSE
\VERSE La voix du Seigneur fait jaillir des flammes de feu. \EVERSE
\VERSE La voix du Seigneur ébranle le désert, et le Seigneur fera tressaillir le désert de Cadès. \EVERSE
\VERSE La voix du Seigneur prépare les cerfs, et découvre les lieux sombres; et dans Son temple, tous publieront Sa gloire. \EVERSE
\VERSE Le Seigneur fait persister le déluge, et le Seigneur siège en Roi à jamais. \EVERSE
\VERSE Le Seigneur donnera la force à Son peuple; le Seigneur bénira Son peuple dans la paix.

}
%%%%%%% PSAUME 29 %%%%%%%
\newcommand{\psalmxxixfr}{
\VERSE pour la dédicace de la maison de David. \EVERSE
\VERSE Je Vous exalterai, Seigneur, parce que Vous m'avez relevé, et que Vous n'avez pas réjoui mes ennemis à mon sujet. \EVERSE
\VERSE Seigneur mon Dieu, j'ai crié vers Vous, et Vous m'avez guéri. \EVERSE
\VERSE Seigneur, Vous avez retiré mon âme du séjour des morts; Vous m'avez sauvé du milieu de ceux qui descendent dans la fosse. \EVERSE
\VERSE Chantez au Seigneur, Vous qui êtes Ses saints, et célébrez Sa sainte mémoire. \EVERSE
\VERSE Car le châtiment provient de Son indignation, et la vie de Sa bienveillance. Les pleurs se répandent le soir, et le matin viendra la joie. \EVERSE
\VERSE Pour moi j'ai dit dans ma prospérité: Je ne serai jamais ébranlé. \EVERSE
\VERSE Seigneur, c'est par Votre volonté que Vous m'avez affermi dans ma gloire. Vous avez détourné de moi Votre visage, et j'ai été tout troublé. \EVERSE
\VERSE Je crierai vers Vous, Seigneur, et j'implorerai mon Dieu. \EVERSE
\VERSE Quelle utilité retirerez-Vous de ma mort, lorsque je descendrai dans la pourriture? Est-ce que la poussière chantera Vos louanges? ou publiera-t-elle Votre vérité? \EVERSE
\VERSE Le Seigneur a entendu, et Il a eu pitié de moi; le Seigneur S'est fait mon protecteur. \EVERSE
\VERSE Vous avez changé mes lamentations en allégresse; Vous avez déchiré mon sac, et Vous m'avez environné de joie, \EVERSE
\VERSE afin que mon âme Vous chante, et que je ne ressente plus la douleur. Seigneur mon Dieu, je Vous louerai éternellement.

}
%%%%%%% PSAUME 30 %%%%%%%
\newcommand{\psalmxxxfr}{
\VERSE Pour la fin, psaume de David, pour l'extase. \EVERSE
\VERSE J'ai espéré en Vous, Seigneur; que je ne sois jamais confondu; dans Votre justice délivrez-moi. \EVERSE
\VERSE Inclinez vers moi Votre oreille; hâtez-Vous de me délivrer. Soyez-moi un Dieu protecteur et une maison de refuge, afin que Vous me sauviez. \EVERSE
\VERSE Car Vous êtes ma force et mon refuge, et à cause de Votre Nom, Vous me conduirez et me nourrirez. \EVERSE
\VERSE Vous me tirerez de ce piège qu'ils ont caché contre moi, car Vous êtes mon protecteur. \EVERSE
\VERSE Je remets mon âme entre Vos mains; Vous m'avez racheté, Seigneur, Dieu de vérité. \EVERSE
\VERSE Vous haïssez ceux qui s'attachent sans aucun fruit à des choses vaines. Pour moi, j'ai mis mon espérance dans le Seigneur. \EVERSE
\VERSE Je tressaillirai de joie et d'allégresse dans Votre miséricorde. Car Vous avez regardé mon état humilié; Vous avez sauvé mon âme des angoisses. \EVERSE
\VERSE Et Vous ne m'avez pas livré aux mains de l'ennemi; Vous avez mis mes pieds au large. \EVERSE
\VERSE Ayez pitié de moi, Seigneur, car je suis très affligé; mon oeil, mon âme et mes entrailles sont troublés par la colère. \EVERSE
\VERSE Car ma vie se consume dans la douleur, et mes années dans les gémissements. Ma force s'est affaiblie par la pauvreté, et mes os sont ébranlés. \EVERSE
\VERSE Plus que tous mes ennemis, je suis devenu un objet d'opprobre, surtout à mes voisins, et l'effroi de ceux qui me connaissent. Ceux qui me voyaient dehors fuyaient loin de moi. \EVERSE
\VERSE J'ai été oublié des coeurs, comme un mort. J'ai été comme un vase brisé; \EVERSE
\VERSE car j'ai entendu les propos injurieux de ceux qui demeurent alentour. Quand ils se réunissaient ensemble contre moi, ils ont tenu conseil pour m'ôter la vie. \EVERSE
\VERSE Mais j'ai espéré en Vous, Seigneur. J'ai dit: Vous êtes mon Dieu; \EVERSE
\VERSE mes destinées sont entre Vos mains. Arrachez-moi de la main de mes ennemis et de mes persécuteurs. \EVERSE
\VERSE Faites luire Votre visage sur Votre serviteur; sauvez-moi par Votre miséricorde. \EVERSE
\VERSE Seigneur, que je ne sois pas confondu, car je Vous ai invoqué. Que les impies rougissent, et qu'ils soient conduits dans l'enfer;  \EVERSE
\VERSE que les lèvres trompeuses deviennent muettes, elles qui profèrent l'iniquité contre le juste, avec orgueil et insolence. \EVERSE
\VERSE Qu'elle est grande, Seigneur, l'abondance de Votre douceur, que Vous avez mise en réserve pour ceux qui Vous craignent! Vous l'exercez envers ceux qui espèrent en Vous, à la vue des enfants des hommes. \EVERSE
\VERSE Vous les cacherez dans le secret de Votre face, à l'abri du tumulte des hommes. Vous les protégerez dans Votre tabernacle contre les langues qui les attaquent. \EVERSE
\VERSE Béni soit le Seigneur, car Il a signalé envers moi Sa miséricorde dans une ville fortifiée. \EVERSE
\VERSE Pour moi j'ai dit dans le transport de mon esprit: J'ai été rejeté de devant Vos yeux. C'est pour cela que Vous avez exaucé la voix de ma prière, lorsque je criais vers Vous. \EVERSE
\VERSE Aimez le Seigneur, vous tous Ses saints; car le Seigneur recherchera la vérité, et Il châtiera largement ceux qui se livrent à l'orgueil. \EVERSE
\VERSE Agissez avec courage, et que votre coeur s'affermisse, vous tous qui espérez au Seigneur.

}
%%%%%%% PSAUME 31 %%%%%%%
\newcommand{\psalmxxxifr}{
\VERSE De David, instruction: Heureux ceux dont les iniquités ont été remises, et dont les péchés sont couverts. \EVERSE
\VERSE Heureux l'homme à qui le Seigneur n'a pas imputé de péché, et dont l'esprit est exempt de fraude. \EVERSE
\VERSE Parce que je me suis tu, mes os ont vieilli, tandis que je criais tout le jour. \EVERSE
\VERSE Car jour et nuit Votre main s'est appesantie sur moi; je me suis retourné dans ma douleur, pendant que l'épine s'enfonçait. \EVERSE
\VERSE Je Vous ai fait connaître mon péché, et je n'ai pas caché mon injustice. J'ai dit: Je confesserai au Seigneur contre moi-même mon injustice; et Vous m'avez remis l'impiété de mon péché. \EVERSE
\VERSE C'est pour cela que tout homme saint Vous priera au temps favorable. Et quand les grandes eaux fondront comme un déluge, elles n'approcheront pas de lui. \EVERSE
\VERSE Vous êtes mon refuge dans la tribulation qui m'a entouré; Vous qui êtes ma joie, délivrez-moi de ceux qui m'environnent. \EVERSE
\VERSE Je vous donnerai l'intelligence, et Je vous enseignerai la voie par où vous devez marcher; J'arrêterai Mes yeux sur vous. \EVERSE
\VERSE Ne soyez pas comme le cheval et le mulet, qui n'ont pas d'intelligence. Resserrez leur bouche avec le mors et le frein, quand ils ne veulent point s'approcher de Vous. \EVERSE
\VERSE Le pécheur sera exposé à des peines nombreuses; mais celui qui espère au Seigneur sera environné de miséricorde. \EVERSE
\VERSE Justes, réjouissez-vous dans le Seigneur, et soyez dans l'allégresse; et glorifiez-vous en Lui, vous tous qui avez le coeur droit.

}
%%%%%%% PSAUME 32 %%%%%%%
\newcommand{\psalmxxxiifr}{
\VERSE Psaume de David: Justes, réjouissez-vous dans le Seigneur; c'est aux hommes droits que sied la louange. \EVERSE
\VERSE Célébrez le Seigneur avec la harpe; chantez Sa gloire sur la lyre à dix cordes. \EVERSE
\VERSE Chantez-Lui un cantique nouveau; louez-Le avec art par vos instruments et vos acclamations. \EVERSE
\VERSE Car la parole du Seigneur est droite, et dans toutes Ses oeuvres éclate Sa fidélité. \EVERSE
\VERSE Il aime la miséricorde et la justice; la terre est remplie de la miséricorde du Seigneur. \EVERSE
\VERSE Les cieux ont été affermis par la parole du Seigneur, et toute leur armée par le souffle de Sa bouche. \EVERSE
\VERSE Il rassemble les eaux de la mer comme dans une outre; Il renferme les océans dans Ses trésors. \EVERSE
\VERSE Que toute la terre craigne le Seigneur; et que tous ceux qui habitent l'univers tremblent devant Lui. \EVERSE
\VERSE Car Il a dit, et tout a été fait; Il a commandé, et tout a été créé. \EVERSE
\VERSE Le Seigneur dissipe les desseins des nations; Il renverse les pensées des peuples, et Il renverse les conseils des princes. \EVERSE
\VERSE Mais le conseil du Seigneur demeure éternellement, et les pensées de Son coeur subsistent de race en race. \EVERSE
\VERSE Heureuse la nation qui a le Seigneur pour son Dieu; heureux le peuple qu'Il a choisi pour Son héritage. \EVERSE
\VERSE Le Seigneur a regardé du haut du Ciel; Il a vu tous les enfants des hommes. \EVERSE
\VERSE De la demeure qu'Il S'est préparée Il a jeté les yeux sur tous ceux qui habitent la terre; \EVERSE
\VERSE Lui qui a formé le coeur de chacun d'eux, et qui connaît toutes leurs oeuvres. \EVERSE
\VERSE Ce n'est point dans Sa grande puissance qu'un roi trouve le salut, et le géant ne se sauvera point par sa force extraordinaire. \EVERSE
\VERSE Le cheval trompe celui qui attend de lui son salut; et sa force, quelque grande qu'elle soit, ne le sauvera pas. \EVERSE
\VERSE Voici! les yeux du Seigneur sont sur ceux qui Le craignent, et sur ceux qui espèrent en Sa miséricorde: \EVERSE
\VERSE pour délivrer leurs âmes de la mort, et les nourrir dans la famine. \EVERSE
\VERSE Notre âme attend le Seigneur; car Il est notre secours et notre protecteur. \EVERSE
\VERSE Car c'est en Lui que notre coeur se réjouira, et c'est en Son saint Nom que nous avons espéré. \EVERSE
\VERSE Faites paraître Votre miséricorde sur nous, Seigneur, selon l'espérance que nous avons eue en Vous.

}
%%%%%%% PSAUME 33 %%%%%%%
\newcommand{\psalmxxxiiifr}{
\VERSE De David, lorsqu'il changea son visage devant Achimélech, qui le renvoya, et qu'il s'en alla. \EVERSE
\VERSE Je bénirai le Seigneur en tout temps; toujours Sa louange sera dans ma bouche. \EVERSE
\VERSE Mon âme mettra sa gloire dans le Seigneur. Que ceux qui sont doux entendent et se réjouissent. \EVERSE
\VERSE Célébrez le Seigneur avec moi, et exaltons tous ensemble Son Nom. \EVERSE
\VERSE J'ai cherché le Seigneur, et Il m'a exaucé; et Il m'a tiré de toutes mes tribulations. \EVERSE
\VERSE Approchez-vous de Lui, et vous serez éclairés; et vos visages ne seront pas couverts de confusion. \EVERSE
\VERSE Ce pauvre a crié, et le Seigneur l'a exaucé; et Il l'a sauvé de toutes ses tribulations. \EVERSE
\VERSE L'Ange du Seigneur environnera ceux qui Le craignent, et Il les délivrera. \EVERSE
\VERSE Goûtez et voyez combien le Seigneur est doux. Heureux est l'homme qui espère en Lui. \EVERSE
\VERSE Craignez le Seigneur, vous tous ses saints, car Il n'y a pas d'indigence pour ceux qui Le craignent. \EVERSE
\VERSE Les riches ont été dans le besoin, et ont eu faim; mais ceux qui cherchent le Seigneur ne seront privés d'aucun bien. \EVERSE
\VERSE Venez, mes fils, écoutez-moi; je vous enseignerai la crainte du Seigneur. \EVERSE
\VERSE Quel est l'homme qui désire la vie, et qui aime à voir d'heureux jours? \EVERSE
\VERSE Préservez votre langue du mal, et que vos lèvres ne profèrent pas la tromperie. \EVERSE
\VERSE Détournez-vous du mal, et faites le bien; recherchez la paix et poursuivez-la. \EVERSE
\VERSE Les yeux du Seigneur sont sur les justes, et Ses oreilles sont ouvertes à leurs prières. \EVERSE
\VERSE Mais le visage du Seigneur est sur ceux qui font le mal, pour exterminer leur mémoire de dessus la terre. \EVERSE
\VERSE Les justes ont crié, et le Seigneur les a exaucés; et Il les a délivrés de toutes leurs tribulations. \EVERSE
\VERSE Le Seigneur est près de ceux qui ont le coeur affligé, et Il sauvera les humbles d'esprit. \EVERSE
\VERSE Les tribulations des justes sont nombreuses, et le Seigneur les délivrera de toutes ces peines. \EVERSE
\VERSE Le Seigneur préserve tous leurs os; il n'y en aura pas un seul de brisé. \EVERSE
\VERSE La mort des pécheurs est affreuse, et ceux qui haïssent le juste sont coupables. \EVERSE
\VERSE Le Seigneur rachètera les âmes de Ses serviteurs, et tous ceux qui mettent leur espérance en Lui ne seront point frustrés.

}
%%%%%%% PSAUME 34 %%%%%%%
\newcommand{\psalmxxxivfr}{
\VERSE De David. Jugez, Seigneur, ceux qui me font du mal; combattez ceux qui me combattent. \EVERSE
\VERSE Prenez Vos armes et Votre bouclier, et levez-Vous pour me secourir. \EVERSE
\VERSE Tirez Votre épée et barrez le passage à ceux qui me persécutent; dites à mon âme: Je suis ton salut. \EVERSE
\VERSE Qu'ils soient couverts de honte et de confusion, ceux qui en veulent à ma vie. Qu'il reculent et soient confondus, ceux qui méditent le mal contre moi. \EVERSE
\VERSE Qu'ils deviennent comme la poussière emportée par le vent, et que l'Ange du Seigneur les serre de près. \EVERSE
\VERSE Que leur chemin soit ténébreux et glissant, et que l'ange du Seigneur les poursuive. \EVERSE
\VERSE Car sans raison ils ont caché un piège pour me perdre; ils ont sans motif outragé mon âme. \EVERSE
\VERSE Qu'un piège dont il ne se doute pas tombe sur lui; que le rets qu'il a caché le saisisse, et qu'il tombe dans son propre filet. \EVERSE
\VERSE Mais mon âme se réjouira dans le Seigneur, et mettra ses délices dans son Sauveur. \EVERSE
\VERSE Tous mes os diront: Seigneur, qui Vous est semblable, à Vous, qui arrachez le pauvre des mains de ceux qui sont plus forts que lui; l'indigent et le pauvre à ceux qui le dépouillent? \EVERSE
\VERSE Des témoins iniques se sont élevés; ils m'ont interrogé sur ce que j'ignorais. \EVERSE
\VERSE Ils n'ont rendu le mal pour le bien; c'était la stérilité pour mon âme. \EVERSE
\VERSE Mais moi, quand ils me tourmentaient, je me revêtais d'un cilice. J'humiliais mon âme par le jeûne, et ma prière retournait dans mon sein. \EVERSE
\VERSE J'avais pour eux la même compassion que pour un proche ou un frère; je me courbais comme dans le deuil et la tristesse. \EVERSE
\VERSE Et ils se sont réjouis, et se sont assemblés contre moi; les malheurs se sont réunis sur moi, sans que j'en connusse la raison. \EVERSE
\VERSE Ils ont été dispersés; mais, sans componction, ils m'ont de nouveau mis à l'épreuve; ils m'ont accablé d'insultes; ils ont grincé des dents contre moi. \EVERSE
\VERSE Seigneur, quand regarderez-Vous? Sauvez mon âme de leur malignité; arrachez mon unique à ces lions. \EVERSE
\VERSE Je Vous célébrerai dans une grande assemblée; je Vous louerai au milieu d'un peuple nombreux. \EVERSE
\VERSE Qu'ils ne se réjouissent point à mon sujet, ceux qui m'attaquent injustement, qui me haïssent sans raison et qui clignent des yeux. \EVERSE
\VERSE Car ils me disaient des paroles de paix; mais, parlant dans le pays avec colère, ils méditaient des tromperies. \EVERSE
\VERSE Et ils ont ouvert au grand large leur bouche contre moi, et ils ont dit: Courage, couage! nos yeux ont vu. \EVERSE
\VERSE Vous avez vu, Seigneur; ne restez pas en silence; Seigneur, ne Vous éloignez pas de moi. \EVERSE
\VERSE Levez-Vous et prenez soin de mon droit; mon Dieu et mon Seigneur, défendez ma cause. \EVERSE
\VERSE Jugez-moi selon Votre justice, Seigneur mon Dieu, et qu'ils ne se réjouissent pas à mon sujet. \EVERSE
\VERSE Qu'ils ne disent pas dans leur coeurs: Courage, courage! réjouissons-nous. Qu'ils ne disent pas: Nous l'avons dévoré. \EVERSE
\VERSE Qu'ils rougissent et soient confondus, ceux qui se félicitent de mes maux. Qu'ils soient couverts de confusion et de honte, ceux qui parlent avec orgueil contre moi. \EVERSE
\VERSE Qu'il soient dans l'allégresse et la joie, ceux qui veulent ma justification; et qu'ils disent sans cesse: Gloire au Seigneur, ceux qui désirent la paix de Son serviteur. \EVERSE
\VERSE Et ma langue célébrera Votre justice, et Votre louange tout le jour.

}
%%%%%%% PSAUME 35 %%%%%%%
\newcommand{\psalmxxxvfr}{
\VERSE Pour la fin, de David, serviteur du Seigneur. \EVERSE
\VERSE L'injuste a dit en lui-même qu'il voulait pécher; la crainte de Dieu n'est point devant ses yeux. \EVERSE
\VERSE Car il a agi avec tromperie en Sa présence, afin que son iniquité se trouvât digne de haine. \EVERSE
\VERSE Les paroles de sa bouche sont iniquité et tromperie; il n'a point voulu devenir intelligent pour faire le bien. \EVERSE
\VERSE Il a médité l'iniquité sur sa couche; il s'est arrêté sur toute voie mauvaise, et il n'a pas eu de haine pour la malice. \EVERSE
\VERSE Seigneur, Votre miséricorde est dans le Ciel, et Votre vérité s'élève jusqu'aux nues. \EVERSE
\VERSE Votre justice est comme les montagnes de Dieu; Vos jugements sont un profond abîme. Vous sauverez, Seigneur, les hommes et les bêtes. \EVERSE
\VERSE Comme Vous avez multiplié Votre miséricorde, ô Dieu! Mais les enfants des hommes espéreront, à couvert sous Vos ailes. \EVERSE
\VERSE Ils seront enivrés de l'abondance de Votre maison, et Vous les ferez boire au torrent de Vos délices. \EVERSE
\VERSE Car en Vous est la source de la vie, et dans Votre lumière nous verrons la lumière. \EVERSE
\VERSE Etendez Votre miséricorde sur ceux qui Vous connaissent, et Votre justice sur ceux qui ont le coeur droit. \EVERSE
\VERSE Que le pied du superbe ne vienne point jusqu'à moi, et que la main du pécheur ne m'ébranle pas. \EVERSE
\VERSE C'est là que sont tombés ceux qui commettent l'iniquité; ils ont été chassés, et ils n'ont pu se tenir debout.

}
%%%%%%% PSAUME 36 %%%%%%%
\newcommand{\psalmxxxvifr}{
\VERSE Psaume de David. Ne porte pas envie aux méchants, et ne sois pas jaloux de ceux qui commettent l'iniquité; \EVERSE
\VERSE car ils se dessécheront aussi vite que l'herbe, et, comme les tiges des plantes, ils se faneront promptement. \EVERSE
\VERSE Espère au Seigneur, et fais le bien; alors tu habiteras la terre, et tu te nourriras de ses richesses. \EVERSE
\VERSE Mets tes délices dans le Seigneur, et il t'accordera ce que ton coeur demande. \EVERSE
\VERSE Découvre au Seigneur ta voie, et espère en Lui, et Lui-même Il agira. \EVERSE
\VERSE Et Il fera éclater ta justice comme la lumière, et ton droit comme le soleil à son midi. \EVERSE
\VERSE Sois soumis au Seigneur, et prie-Le. Ne porte pas envie à celui qui réussit dans sa voie, à l'homme qui commet des injustices. \EVERSE
\VERSE Laisse la colère, et abandonne la fureur; n'aie pas d'envie, ce serait mal faire. \EVERSE
\VERSE Car les méchants seront exterminés; mais ceux qui attendent patiemment le Seigneur auront la terre en héritage. \EVERSE
\VERSE Encore un peu de temps, et le pécheur ne sera plus; et tu chercheras sa place, et tu ne la trouveras pas. \EVERSE
\VERSE Mais les doux posséderont la terre, et ils se délecteront dans l'abondance de la paix. \EVERSE
\VERSE Le pécheur observera le juste, et il grincera des dents contre lui. \EVERSE
\VERSE Mais le Seigneur Se rira de lui, parce qu'il voit que Son jour viendra. \EVERSE
\VERSE Les pécheurs ont tiré le glaive, ils ont tendu leur arc, pour renverser le pauvre et l'indigent, pour égorger ceux qui ont le coeur droit. \EVERSE
\VERSE Que leur glaive perce leur propre coeur, et que leur arc soit brisé. \EVERSE
\VERSE Mieux vaut le peu du juste que les grandes richesses des pécheurs; \EVERSE
\VERSE car les bras des pécheurs seront brisés, mais le Seigneur affermit les justes. \EVERSE
\VERSE Le Seigneur connaît les jours des hommes sans tache, et leur héritage sera éternel. \EVERSE
\VERSE Ils ne seront pas confondus au temps du malheur, et aux jours de famine ils seront rassasiés, \EVERSE
\VERSE parce que les pécheurs périront. Mais les ennemis du Seigneur n'auront pas plus tôt été honorés et élevés, qu'ils tomberont et s'évanouiront comme la fumée. \EVERSE
\VERSE Le pécheur empruntera et ne payera point; mais le juste est compatissant et il donne. \EVERSE
\VERSE Car ceux qui bénissent Dieu posséderont la terre; mais ceux qui Le maudissent périront. \EVERSE
\VERSE Les pas de l'homme seront dirigés par le Seigneur, et il prendra plaisir à sa voie. \EVERSE
\VERSE Lorsqu'il tombera, il ne se brisera pas, car le Seigneur le soutient de Sa main. \EVERSE
\VERSE J'ai été jeune, et j'ai vieilli; mais je n'ai pas vu le juste abandonné, ni sa race mendiant du pain. \EVERSE
\VERSE Tout le jour il est compatissant et il prête, et sa race sera en bénédiction. \EVERSE
\VERSE Détourne-toi du mal et fais le bien, et possède une demeure éternelle. \EVERSE
\VERSE Car le Seigneur aime l'équité, et Il n'abandonnera pas Ses saints; ils seront gardés éternellement. Les méchants seront punis, et la race des impies périra. \EVERSE
\VERSE Mais les justes posséderont la terre, et ils y habiteront à jamais. \EVERSE
\VERSE La bouche du juste méditera la sagesse, et sa langue proférera l'équité. \EVERSE
\VERSE La loi de son Dieu est dans son coeur, et on ne le renversera point. \EVERSE
\VERSE Le pécheur observe le juste, et il cherche à le mettre à mort. \EVERSE
\VERSE Mais le Seigneur ne l'abandonnera point entre ses mains, et ne le condamnera pas lorsqu'il sera jugé. \EVERSE
\VERSE Attends le Seigneur et garde Sa voie; et Il t'élèvera, pour que tu possèdes la terre en héritage. Quand les pécheurs périront, tu verras. \EVERSE
\VERSE J'ai vu l'impie grandement exalté, et élevé comme les cèdres du Liban. \EVERSE
\VERSE Et j'ai passé, et déjà il n'était plus; et je l'ai cherché, mais on n'a pu trouver sa place. \EVERSE
\VERSE Garde l'innocence, et n'aie en vue que l'équité, car des biens resteront à l'homme pacifique. \EVERSE
\VERSE Mais les injustes périront tous ensemble; ce que les impies auront laissé disparaîtra. \EVERSE
\VERSE Mais le salut des justes vient du Seigneur, et Il est leur protecteur au temps de la tribulation. \EVERSE
\VERSE Le Seigneur les assistera, et les délivrera; Il les arrachera des mains des pécheurs, et Il les sauvera, parce qu'ils ont espéré en Lui.

}
%%%%%%% PSAUME 37 %%%%%%%
\newcommand{\psalmxxxviifr}{
\VERSE Psaume de David, pour faire souvenir, pour le sabbat. \EVERSE
\VERSE Seigneur, ne me reprenez dans Votre fureur, et ne me punissez pas dans Votre colère. \EVERSE
\VERSE Car j'ai été percé de Vos flèches, et Vous avez appesanti sur moi Votre main. \EVERSE
\VERSE Il n'est rien resté de sain dans ma chair à la vue de Votre colère; il n'y a plus de paix dans mes os à la vue de mes péchés. \EVERSE
\VERSE Car mes iniquités se sont élevés au-dessus de ma tête, et comme un lourd fardeau elles se sont appesanties sur moi. \EVERSE
\VERSE Mes plaies ont été remplies de corruption et de pourriture, par l'effet de ma folie. \EVERSE
\VERSE Je suis devenu misérable, et continuellement tout courbé; je marchais triste tout le jour. \EVERSE
\VERSE Car mes reins ont été remplis d'illusions, et il n'y a rien de sain dans ma chair. \EVERSE
\VERSE J'ai été affligé et humilié outre mesure, et le gémissement de mon coeur m'arrachait des rugissements. \EVERSE
\VERSE Seigneur, tout mon désir est devant Vous, et mon gémissement ne Vous est point caché. \EVERSE
\VERSE Mon coeur est troublé, ma force m'a quitté, et la lumière même de mes yeux n'est plus avec moi. \EVERSE
\VERSE Mes amis et mes proches se sont avancés jusqu'à moi, et se sont arrêtés. Ceux qui étaient près de moi se sont arrêtés à distance. \EVERSE
\VERSE Et ceux qui en voulaient à ma vie usaient de violence. Ceux qui cherchaient à me faire du mal ont proféré des mensonges, et tout le jour ils méditaient la tromperie. \EVERSE
\VERSE Mais moi, comme si j'eusse été sourd, je n'entendais pas; et comme si j'eusse été muet, je n'ouvrais pas la bouche. \EVERSE
\VERSE Je suis devenu comme un homme qui n'entend pas, et qui n'a pas de répliques dans sa bouche. \EVERSE
\VERSE Car c'est en Vous, Seigneur, que j'ai espéré; Vous m'exaucerez, Seigneur mon Dieu. \EVERSE
\VERSE Car j'ai dit: Que mes ennemis ne se réjouissent pas à mon sujet, eux qui, ayant vu mes pieds ébranlés, ont parlé insolemment de moi. \EVERSE
\VERSE Car je suis préparé aux châtiments, et ma douleur est toujours devant mes yeux. \EVERSE
\VERSE Car je proclamerai mon iniquité, et je serai toujours occupé de la pensée de mon péché. \EVERSE
\VERSE Cependant mes ennemis vivent, et sont devenus plus puissants que moi, et ceux qui me haïssent injustement se sont multipliés. \EVERSE
\VERSE Ceux qui rendent le mal pour le bien me décriaient, parce que je m'attachais au bien. \EVERSE
\VERSE Ne m'abandonnez pas, Seigneur mon Dieu; ne Vous éloignez pas de moi. \EVERSE
\VERSE Hâtez-Vous de me secourir, Seigneur, Dieu de mon salut.

}
%%%%%%% PSAUME 38 %%%%%%%
\newcommand{\psalmxxxviiifr}{
\VERSE Pour la fin, à Idithun lui-même, cantique de David. \EVERSE
\VERSE J'ai dit: je veillerai sur mes voies, pour ne point pécher par ma langue. J'ai mis une garde à ma bouche, pendant que le pécheur s'élevait devant moi. \EVERSE
\VERSE Je me suis tu, et je me suis humilié, et je me suis abstenu de dire même de bonnes choses; et ma douleur a été renouvelée. \EVERSE
\VERSE Mon coeur s'est échauffé au dedans de moi, et tandis que je méditais, un feu s'est embrasé. \EVERSE
\VERSE La parole est venue sur ma langue: Faites-moi connaître ma fin, Seigneur, et quel est le nombre de mes jours, afin que je sache combien peu il m'en reste. \EVERSE
\VERSE Voici que Vous avez soumis mes jours à une mesure bornée, et mon être est comme un néant devant Vous. Oui, tout homme vivant n'est qu'entière vanité. \EVERSE
\VERSE Oui, l'homme passe comme un fantôme, et c'est en vain qu'il se tourmente. Il amasse des trésors, et il ignore pour qui il les aura entassés. \EVERSE
\VERSE Et maintenant quelle est mon attente? N'est-ce pas le Seigneur? Tous mes biens sont en Vous. \EVERSE
\VERSE Délivrez-moi de toutes mes iniquités. Vous m'avez rendu l'opprobre de l'insensé. \EVERSE
\VERSE Je me suis tu, et je n'ai pas ouvert la bouche, parce que c'est Vous qui l'avez fait. \EVERSE
\VERSE Détournez de moi Vos coups. \EVERSE
\VERSE Sous la puissance de Votre main, j'ai défailli, quand Vous m'avez repris. Vous avez puni l'homme à cause de son iniquité. Et Vous avez fait dessécher son âme comme l'araignée. Oui, c'est en vain que tout homme s'inquiète. \EVERSE
\VERSE Exaucez, Seigneur, ma prière et ma supplication; soyez attentif à mes larmes. Ne gardez pas le silence, car je suis auprès de Vous un étranger et un voyageur, comme tous mes pères. \EVERSE
\VERSE Accordez-moi quelque relâche, afin que je sois rafraîchi avant de partir et de disparaître.

}
%%%%%%% PSAUME 39 %%%%%%%
\newcommand{\psalmxxxixfr}{
\VERSE Pour la fin, Psaume de David lui-même. \EVERSE
\VERSE J'ai attendu, et encore attendu le Seigneur, et Il a fait attention à moi. \EVERSE
\VERSE Il a exaucé mes prières, et Il m'a tiré de l'abîme de misère et de la boue profonde. Et Il a placé mes pieds sur la pierre, et Il a conduit mes pas. \EVERSE
\VERSE Et Il a mis dans ma bouche un cantique nouveau, un hymne à notre Dieu. Beaucoup Le verront, et craindront, et espéreront dans le Seigneur. \EVERSE
\VERSE Heureux l'homme qui a mis son espérance dans le Nom du Seigneur, et qui n'a point arrêté son regard sur des vanités et des folies mensongères. \EVERSE
\VERSE Vous avez fait, Seigneur mon Dieu, un grand nombre d'oeuvres admirables, et il n'y a personne qui Vous soit semblable dans Vos pensées. J'ai voulu les annoncer et en parler, mais leur multitude est sans nombre. \EVERSE
\VERSE Vous n'avez voulu ni sacrifice ni oblation, mais Vous m'avez façonné des oreilles. Vous n'avez pas demandé d'holocauste ni de sacrifice pour le péché; \EVERSE
\VERSE alors j'ai dit: Voici que je viens. En tête de Son livre il est écrit de Moi \EVERSE
\VERSE que je dois faire Votre volonté. Mon Dieu, je l'ai voulu, et Votre loi est au fond de mon coeur. \EVERSE
\VERSE J'ai publié Votre justice dans une grande assemblée: je ne fermerai pas mes lèvres, Seigneur, Vous le savez. \EVERSE
\VERSE Je n'ai pas caché Votre justice dans mon coeur; j'ai proclamé Votre vérité et Votre salut. Je n'ai point caché Votre miséricorde et Votre vérité devant l'assemblée nombreuse. \EVERSE
\VERSE Pour Vous, Seigneur, n'éloignez pas de moi Vos miséricordes; Votre bonté et Votre vérité m'ont toujours soutenu. \EVERSE
\VERSE Car des maux sans nombre m'environnent; mes iniquités m'ont saisi, et je n'ai pu les voir toutes. Elles sont plus nombreuses que les cheveux de ma tête, et mon coeur m'a manqué. \EVERSE
\VERSE Qu'il Vous plaise, Seigneur, de me délivrer; Seigneur, regardez vers moi pour me secourir. \EVERSE
\VERSE Qu'ils soient confondus et couverts de honte, ceux qui cherchent ma vie pour me l'ôter. Qu'ils reculent en arrière et soient dans la confusion, ceux qui me veulent du mal. \EVERSE
\VERSE Qu'ils soient à l'instant couverts de honte, ceux qui me disent: Va! va! \EVERSE
\VERSE Mais que tous ceux qui Vous cherchent tressaillent en Vous d'allégresse et de joie, et que ceux qui aiment Votre salut disent sans cesse: Que le Seigneur soit glorifié! \EVERSE
\VERSE Pour moi, je suis pauvre et indigent; mais le Seigneur prend soin de moi. Vous êtes mon aide et mon protecteur. Mon Dieu, ne tardez pas.

}
%%%%%%% PSAUME 40 %%%%%%%
\newcommand{\psalmxlfr}{
\VERSE Pour la fin, psaume de David lui-même. \EVERSE
\VERSE Heureux celui qui a l'intelligence de l'indigent et du pauvre: le Seigneur le délivrera au jour mauvais. \EVERSE
\VERSE Que le Seigneur le conserve, et le fasse vivre, et qu'Il le rende heureux sur la terre, et qu'Il ne le livre pas au désir de ses ennemis. \EVERSE
\VERSE Que le Seigneur lui porte secours sur son lit de douleur. Vous avez retourné toute sa couche dans sa maladie. \EVERSE
\VERSE J'ai dit: Seigneur, ayez pitié de moi; guérissez mon âme, car j'ai péché contre Vous. \EVERSE
\VERSE Mes ennemis ont dit du mal contre moi: Quand mourra-t-il, et quand périra son Nom? \EVERSE
\VERSE Si l'un d'eux entrait pour me voir, il me tenait de vains discours; son coeur amassait l'iniquité en lui-même. Il sortait dehors, et parlait. \EVERSE
\VERSE Tous mes ennemis ensemble chuchotaient contre moi; ils tramaient des maux contre moi. \EVERSE
\VERSE Ils se sont arrêtés contre moi à une parole inique: Est-ce que celui qui dort ne pourra jamais se lever? \EVERSE
\VERSE Même l'homme de mon intimité, en qui je me suis confié, et qui mangeait mon pain, a fait éclater sa trahison contre moi. \EVERSE
\VERSE Mais Vous, Seigneur, ayez compassion de moi, et ressuscitez-moi; et je leur rendrai ce qu'ils méritent. \EVERSE
\VERSE J'ai connu quel a été Votre amour pour moi, en ce que mon ennemi ne se réjouira point à mon sujet. \EVERSE
\VERSE Vous m'avez accueilli à cause de mon innocence, et Vous m'avez affermi pour toujours en Votre présence. \EVERSE
\VERSE Béni soit le Seigneur, le Dieu d'Israël, dans les siècles des siècles. Ainsi soit-il. Ainsi soit-il.

}
%%%%%%% PSAUME 41 %%%%%%%
\newcommand{\psalmxlifr}{
\VERSE Pour la fin, instruction des fils de Coré. \EVERSE
\VERSE Comme le cerf soupire après les sources des eaux, ainsi mon âme soupire vers Vous, mon Dieu. \EVERSE
\VERSE Mon âme a soif du Dieu fort et vivant. Quand viendrai-je, et paraîtrai-je devant la face de Dieu? \EVERSE
\VERSE Mes larmes ont été ma nourriture le jour et la nuit, pendant qu'on me dit tous les jours: Où est ton Dieu? \EVERSE
\VERSE Je me suis souvenu de ces choses, et j'ai répandu mon âme au dedans de moi-même; car je passerai dans le lieu du tabernacle admirable jusqu'à la maison de Dieu, parmi les chants d'allégresse et de louange, pareils au bruit d'un festin. \EVERSE
\VERSE Pourquoi es-tu triste, mon âme? et pourquoi me troubles-tu? Espère en Dieu, car je Le louerai encore, Lui le salut de mon visage \EVERSE
\VERSE et mon Dieu. Mon âme a été toute troublée en moi-même; c'est pourquoi je me souviendrai de Vous, du pays du Jourdain, de l'Hermon, et de la petite montagne. \EVERSE
\VERSE L'abîme appelle l'abîme, au bruit de Vos cataractes. Toutes Vos vagues amoncelées et Vos flots ont passé sur moi. \EVERSE
\VERSE Pendant le jour le Seigneur a envoyé Sa miséricorde, et la nuit Son cantique. Au dedans de moi est une prière pour le Dieu de ma vie. \EVERSE
\VERSE Je dirai à Dieu: Vous êtes mon défenseur; pourquoi m'avez-Vous oublié? et pourquoi faut-il que je marche attristé, tandis que l'ennemi m'afflige? \EVERSE
\VERSE Pendant que mes os sont brisés, mes ennemis qui me persécutent m'accablent par leurs reproches, me disant tous les jours: Où est son Dieu?  \EVERSE
\VERSE Pourquoi es-tu triste, mon âme? et pourquoi me troubles-tu? Espère en Dieu, car je Le louerai encore, Lui le salut de mon visage et mon Dieu.

}
%%%%%%% PSAUME 42 %%%%%%%
\newcommand{\psalmxliifr}{
\VERSE Psaume de David. Jugez-moi, ô Dieu, et séparez ma cause de celle d'une nation qui n'est pas sainte; délivrez-moi de l'homme méchant et trompeur. \EVERSE
\VERSE Car Vous êtes ma force, ô Dieu; pourquoi m'avez-vous repoussé, et pourquoi dois-je marcher attristé, pendant que l'ennemi m'afflige? \EVERSE
\VERSE Envoyez Votre lumière et Votre vérité: elles me conduiront et m'amèneront à Votre montagne sainte et à Vos tabernacles. \EVERSE
\VERSE Et j'entrerai à l'autel de Dieu, au Dieu qui réjouit ma jeunesse. Je Vous louerai sur la harpe, ô Dieu, mon Dieu.  \EVERSE
\VERSE Pourquoi es-tu triste, mon âme? et pourquoi me troubles-tu? Espère en Dieu, car je Le louerai encore, Lui, le salut de mon visage et mon Dieu.

}
%%%%%%% PSAUME 43 %%%%%%%
\newcommand{\psalmxliiifr}{
\VERSE Pour la fin, des fils de Coré, pour l'instruction. \EVERSE
\VERSE O Dieu, nous avons entendu de nos oreilles; nos pères nous ont annoncé l'oeuvre que Vous avez faite en leurs jours, et aux jours anciens. \EVERSE
\VERSE Votre main a exterminé les nations, et Vous les avez établis à leur place;Vous avez affligé les peuples, et Vous les avez chassés. \EVERSE
\VERSE Car ce n'est point par leur glaive qu'ils ont conquis ce pays, et ce n'est pas leur bras qui les a sauvés, mais c'est Votre droite et Votre bras, et la lumière de Votre visage, parce que Vous les aimiez. \EVERSE
\VERSE Vous êtes mon roi et mon Dieu, Vous qui ordonnez le salut de Jacob. \EVERSE
\VERSE Par Vous nous renverserons nos ennemis, et en Votre Nom nous mépriserons ceux qui se lèvent contre nous. \EVERSE
\VERSE Car ce n'est pas dans mon arc que je me confierai, et ce n'est pas mon glaive qui me sauvera. \EVERSE
\VERSE Mais c'est Vous qui nous avez sauvés de ceux qui nous affligeaient, et qui avez confondu ceux qui nous haïssaient. \EVERSE
\VERSE En Dieu nous nous glorifierons tout le jour, et nous célébrerons à jamais Votre Nom. \EVERSE
\VERSE Mais maintenant Vous nous avez repoussés et couverts de honte, et Vous ne sortez plus, ô Dieu, avec nos armées. \EVERSE
\VERSE Vous nous avez fait tourner le dos à nos ennemis, et ceux qui nous haïssaient nous mettaient au pillage. \EVERSE
\VERSE Vous nous avez livrés comme des brebis de boucherie, et Vous nous avez dispersés parmi les nations. \EVERSE
\VERSE Vous avez vendu Votre peuple à vil prix, et il n'y a pas eu foule dans l'achat qui s'en est fait. \EVERSE
\VERSE Vous nous avez rendus l'opprobre de nos voisins, et un objet d'insulte et de moquerie pour ceux qui nous entourent. \EVERSE
\VERSE Vous nous avez rendus la fable des nations; les peuples branlent la tête à notre sujet. \EVERSE
\VERSE Tout le jour ma honte est devant mes yeux, et la confusion de mon visage me couvre tout entier, \EVERSE
\VERSE à la voix de celui qui m'outrage et m'injurie, à la vue de l'ennemi et du persécuteur. \EVERSE
\VERSE Tous ces maux sont venus sur nous; et pourtant nous ne Vous avons pas oublié, et nous n'avons pas agi avec iniquité contre Votre alliance. \EVERSE
\VERSE Et notre coeur ne s'est point retiré en arrière; et Vous avez détourné nos pas de Votre voie. \EVERSE
\VERSE Car Vous nous avez humiliés dans un lieu d'affliction, et l'ombre de la mort nous a recouverts. \EVERSE
\VERSE Si nous avons oublié le Nom de notre Dieu, et si nous avons étendu nos mains vers un dieu étranger, \EVERSE
\VERSE Dieu n'en redemandera-t-Il pas compte? Car Il connaît les secrets du coeur. Car c'est à cause de Vous que nous sommes tous les jours livrés à la mort, et qu'on nous regarde comme des brebis de boucherie. \EVERSE
\VERSE Levez-Vous; pourquoi dormez-Vous, Seigneur? Levez-Vous, et ne nous repoussez pas à jamais. \EVERSE
\VERSE Pourquoi détournez-Vous Votre visage? Pourquoi oubliez-Vous notre misère et notre tribulation? \EVERSE
\VERSE Car notre âme est humiliée dans la poussière, et notre sein est comme collé à la terre. \EVERSE
\VERSE Levez-Vous, Seigneur; secourez-nous, et rachetez-nous à cause de Votre Nom.

}
%%%%%%% PSAUME 44 %%%%%%%
\newcommand{\psalmxlivfr}{
\VERSE Pour la fin, pour ceux qui seront changés, instruction des fils de Coré, cantique pour le Bien-aimé. \EVERSE
\VERSE De mon coeur a jailli une excellent parole; c'est que j'adresse mes oeuvres à un Roi. Ma langue est comme le roseau du scribe qui écrit rapidement. \EVERSE
\VERSE Vous surpassez en beauté les enfants des hommes; la grâce est répandue sur Vos lèvres; c'est pourquoi Dieu Vous a béni à jamais. \EVERSE
\VERSE Ceignez-Vous de Votre glaive sur Votre hanche, ô très puissant. \EVERSE
\VERSE Avec Votre gloire et Votre majesté, avancez, marchez victorieusement, et régnez, pour la vérité, la douceur et la justice; et Votre droite Vous conduira merveilleusement. \EVERSE
\VERSE Vos flèches sont aiguës; les peuples tomberont sous Vous; elles perceront le coeur des ennemis du Roi. \EVERSE
\VERSE Votre trône, ô Dieu, est éternel; le sceptre de Votre règne est un sceptre d'équité. \EVERSE
\VERSE Vous avez aimé la justice, et haï l'iniquité; c'est pourquoi, ô Dieu, Votre Dieu Vous a oint d'une huile d'allégresse d'une manière plus excellente que tous Vos compagnons. \EVERSE
\VERSE La myrrhe, l'aloès et la casse s'exhalent de Vos vêtements, des palais d'ivoire; de là Vous réjouissent \EVERSE
\VERSE les filles des rois dans Votre gloire. La reine se tient à Votre droite, en vêtements tissus d'or, couverte de broderies. \EVERSE
\VERSE Ecoutez, ma fille, voyez, et prêtez l'oreille, et oubliez votre peuple et la maison de votre père. \EVERSE
\VERSE Et le Roi sera épris de votre beauté; car Il est le Seigneur votre Dieu, et on L'adorera. \EVERSE
\VERSE Et les filles de Tyr, avec des présents, vous offriront leurs humbles prières, ainsi que tous les riches d'entre le peuple. \EVERSE
\VERSE Toute la gloire de la fille du Roi est au dedans, quand elle est ornée de franges d'or, \EVERSE
\VERSE couverte de broderies. Des vierges seront amenées au Roi après elle; ses compagnes Vous seront présentées. \EVERSE
\VERSE Elles seront présentées au milieu de la joie et de l'allégresse; on les conduira au temple du Roi. \EVERSE
\VERSE A la place de Vos pères, des fils Vous sont nés; Vous les établirez princes sur toute la terre. \EVERSE
\VERSE Ils se souviendront de Votre Nom de génération en génération. C'est pourquoi les peuples Vous loueront éternellement, et dans les siècles des siècles.

}
%%%%%%% PSAUME 45 %%%%%%%
\newcommand{\psalmxlvfr}{
\VERSE Pour la fin, des fils de Coré, sur les mystères, Psaume. \EVERSE
\VERSE Dieu est notre refuge et notre force; notre secours dans les tribulations qui nous ont enveloppés de toutes parts. \EVERSE
\VERSE C'est pourquoi nous ne craindrons point quand la terre sera ébranlée, et que les montagnes seront transportées au coeur de la mer. \EVERSE
\VERSE Ses eaux ont fait un grand bruit, et ont été agitées; les montagnes ont été ébranlées par Sa puissance. \EVERSE
\VERSE Un fleuve réjouit la cité de Dieu par ses flots abondants; le Très-Haut a sanctifié Son tabernacle. \EVERSE
\VERSE Dieu est au milieu d'elle, elle ne sera pas ébranlée; Dieu la protégera le matin dès l'aurore. \EVERSE
\VERSE Les nations ont été troublées, et les royaumes se sont affaissés; Il a fait entendre Sa voix, la terre a été ébranleé. \EVERSE
\VERSE Le Seigneur des armées est avec nous; le Dieu de Jacob est notre défenseur. \EVERSE
\VERSE Venez, et voyez les oeuvres du Seigneur, les prodiges qu'Il a opérés sur la terre,  \EVERSE
\VERSE en faisant cesser la guerre jusqu'à l'extrémité du monde. Il brisera l'arc, et mettra les armes en pièces, et Il brûlera les boucliers par le feu. \EVERSE
\VERSE Arrêtez, et considérez que c'est Moi qui suis Dieu. Je serai exalté parmi les nations, et Je serai exalté sur la terre. \EVERSE
\VERSE Le Seigneur des armées est avec nous; le Dieu de Jacob est notre défenseur.

}
%%%%%%% PSAUME 46 %%%%%%%
\newcommand{\psalmxlvifr}{
\VERSE Pour la fin, des fils de Coré, Psaume. \EVERSE
\VERSE Nations, frappez toutes des mains; célébrez Dieu par des cris d'allégresse. \EVERSE
\VERSE Car le Seigneur est très haut et terrible, Roi suprême sur toute la terre. \EVERSE
\VERSE Il nous a assujetti les peuples, et a mis les nations sous nos pieds. \EVERSE
\VERSE Il nous a choisis pour Son héritage; la beauté de Jacob qu'Il a aimée. \EVERSE
\VERSE Dieu est monté au milieu des cris de joie, et le Seigneur au son de la trompette. \EVERSE
\VERSE Chantez à notre Dieu, chantez; chantez à notre Roi, chantez. \EVERSE
\VERSE Car Dieu est le Roi de toute la terre; chantez avec sagesse. \EVERSE
\VERSE Dieu régnera sur les nations; Dieu est assis sur Son saint trône. \EVERSE
\VERSE Les princes des peuples se sont unis au Dieu d'Abraham; car les dieux puissants de la terre ont été extraordinairement élevés.

}
%%%%%%% PSAUME 47 %%%%%%%
\newcommand{\psalmxlviifr}{
\VERSE Psaume, cantique des fils de Coré, pour le second jour de la semaine. \EVERSE
\VERSE Le Seigneur est grand et digne de toute louange, dans la cité de notre Dieu, sur Sa sainte montagne. \EVERSE
\VERSE C'est pour l'allégresse de toute la terre qu'a été fondé le mont Sion, le côté de l'aquilon, la cité du grand Roi. \EVERSE
\VERSE Dieu Se fera connaître dans ses maisons, lorsqu'Il la défendra. \EVERSE
\VERSE Car voici que les rois de la terre se sont ligués et se sont avancés ensemble. \EVERSE
\VERSE Eux-mêmes, en la voyant, ont été dans la stupeur, troublés et vivement émus;  \EVERSE
\VERSE un tremblement les a saisis. Il y a eu là des douleurs comme celles de la femme qui enfante.  \EVERSE
\VERSE Par un vent impétueux Vous briserez les vaisseaux de Tharsis. \EVERSE
\VERSE Ce que nous avions entendu dire, nous l'avons vu dans la cité du Seigneur des armées, dans la cité de notre Dieu. Dieu l'a établie à jamais. \EVERSE
\VERSE Nous avons reçu, ô Dieu, Votre miséricorde au milieu de Votre temple. \EVERSE
\VERSE Comme Votre Nom, ô Dieu, ainsi Votre louange s'étend jusqu'aux extrémités de la terre. Votre droite est pleine de justice. \EVERSE
\VERSE Que le mont Sion se réjouisse, et que les filles de Juda soient dans l'allégresse, à cause de Vos jugements, Seigneur. \EVERSE
\VERSE Faites le tour de Sion, et environnez-la; racontez ces merveilles du haut de ses tours. \EVERSE
\VERSE Appliquez-vous à considérer sa force, et faites le dénombrement de ses maisons, pour en faire le récit à la génération future. \EVERSE
\VERSE Car c'est là notre Dieu, notre Dieu pour l'éternité et les siècles des siècles; Il régnera sur nous à jamais.

}
%%%%%%% PSAUME 48 %%%%%%%
\newcommand{\psalmxlviiifr}{
\VERSE Pour la fin, des fils de Coré, Psaume. \EVERSE
\VERSE Ecoutez tous ceci, ô peuples; prêtez l'oreille, vous tous qui habitez l'univers; \EVERSE
\VERSE et vous, enfants de la terre et fils des hommes, le riche aussi bien que le pauvre. \EVERSE
\VERSE Ma bouche proférera la sagesse, et de la méditation de mon coeur sortira la prudence. \EVERSE
\VERSE J'inclinerai mon oreille à la parabole; je révélerai au son de la harpe ce que j'ai à proposer. \EVERSE
\VERSE Pourquoi craindrais-je au jour mauvais? L'iniquité de ceux qui me talonnent m'environnera. \EVERSE
\VERSE Ils se confient dans leur force, et ils se glorifient dans l'abondance de leurs richesses. \EVERSE
\VERSE Le frère ne rachète point, un homme rachètera-t-il? Il ne pourra pas donner à Dieu de quoi L'apaiser, \EVERSE
\VERSE ni un prix capable de racheter son âme. Il sera éternellement dans la peine; \EVERSE
\VERSE et il vivra encore jusqu'à la fin. \EVERSE
\VERSE Il ne verra pas la mort, lorsqu'il verra les sages mourir. Ensemble l'insensé et le fou périront; et ils abandonneront leurs richesses à des étrangers;  \EVERSE
\VERSE et leurs sépulcres seront à jamais leurs demeures. Leurs demeures subsisteront de génération en génération; ils ont donné leurs noms à leurs domaines. \EVERSE
\VERSE Et l'homme, quoique élevé en honneur, n'a pas compris. Il a été comparé aux bêtes sans raison, et il leur est devenu semblable. \EVERSE
\VERSE Telle est leur voie, qui leur est une occasion de chute; et néanmoins ils se complaisent dans leurs discours. \EVERSE
\VERSE Ils ont été mis dans l'enfer comme un troupeau de brebis; la mort les dévorera. Et, au matin, les justes auront l'empire sur eux; et leur appui sera détruit dans l'enfer, après qu'ils auront été dépouillés de leur gloire. \EVERSE
\VERSE Mais Dieu rachètera mon âme de la puissance de l'enfer, lorsqu'Il m'aura pris auprès de Lui. \EVERSE
\VERSE Ne crains pas, quand un homme sera devenu riche, et quand la gloire de sa maison se sera agrandie; \EVERSE
\VERSE car, lorsqu'il sera mort, il n'emportera pas tout, et sa gloire ne descendra point avec lui. \EVERSE
\VERSE Car, pendant sa vie, son âme sera bénie; il te louera quand tu lui auras fait du bien. \EVERSE
\VERSE Il entrera jusqu'auprès des générations de ses pères; et durant toute l'éternité il ne verra pas la lumière. \EVERSE
\VERSE L'homme, quoique élevé en honneur, n'a point compris; il a été comparé aux bêtes sans raison, et il leur est devenu semblable.

}
%%%%%%% PSAUME 49 %%%%%%%
\newcommand{\psalmxlixfr}{
\VERSE Psaume d'Asaph. Le Dieu des dieux, le Seigneur a parlé, et Il a appelé la terre du lever du soleil au couchant.  \EVERSE
\VERSE De Sion apparaît l'éclat de Sa beauté. \EVERSE
\VERSE Dieu viendra visiblement; Lui, notre Dieu, et Il ne Se taira point. Le feu s'enflammera en Sa présence, et une tempête violente L'environnera. \EVERSE
\VERSE Il appellera d'en haut le Ciel et la terre, pour faire le discernement de Son peuple. \EVERSE
\VERSE Rassemblez devant Lui Ses saints, qui scellent Son alliance par des sacrifices. \EVERSE
\VERSE Et les cieux annonceront Sa justice, car c'est Dieu qui est juge. \EVERSE
\VERSE Ecoute, Mon peuple, et Je parlerai; Israël, et Je te rendrai témoignage. C'est Moi qui suis Dieu, ton Dieu. \EVERSE
\VERSE Ce n'est pas pour tes sacrifices que Je te reprendrai, car tes holocaustes sont toujours devant Moi. \EVERSE
\VERSE Je ne prendrai pas les veaux de ta maison, ni les boucs de tes troupeaux; \EVERSE
\VERSE car toutes les bêtes des forêts sont à Moi, ainsi que les animaux des montagnes, et les boeufs. \EVERSE
\VERSE Je connais tous les oiseaux du ciel, et la beauté des champs est en Ma présence. \EVERSE
\VERSE Si J'ai faim, Je ne te le dirai pas; car l'univers est à Moi, avec tout ce qu'il renferme. \EVERSE
\VERSE Est-ce que Je mangerai la chair des taureaux? ou boirai-Je le sang des boucs? \EVERSE
\VERSE Immole à Dieu un sacrifice de louange, et rends tes voeux au Très-Haut. \EVERSE
\VERSE Puis invoque-Moi au jour de la tribulation; Je te délivrerai, et tu Me glorifieras. \EVERSE
\VERSE Mais Dieu a dit au pécheur: Pourquoi énumères-tu Mes lois, et pourquoi as-tu constamment Mon alliance à la bouche? \EVERSE
\VERSE Toi qui hais la discipline, et qui as rejeté derrière toi Mes paroles. \EVERSE
\VERSE Si tu voyais un voleur, tu courais avec lui, et tu mettais ta part avec les adultères. \EVERSE
\VERSE Ta bouche a été remplie de malice, et ta langue ourdissait la fraude. \EVERSE
\VERSE Tu t'asseyais pour parler contre ton frère, et tu tendais des pièges contre le fils de ta mère.  \EVERSE
\VERSE Voilà ce que tu as fait, et Je me suis tu. Tu as cru d'une manière impie que Je te serais semblable. Je te reprendrai, et Je mettrai tout sous tes yeux. \EVERSE
\VERSE Comprenez ces choses, vous qui oubliez Dieu; de peur qu'Il ne déchire, sans que personne puisse délivrer. \EVERSE
\VERSE Le sacrifice de louange est celui qui M'honorera, et là est la voie par laquelle Je montrerai à l'homme le salut de Dieu.

}
%%%%%%% PSAUME 50 %%%%%%%
\newcommand{\psalmlfr}{
\VERSE Pour la fin, psaume de David, \EVERSE
\VERSE lorsque le prophète Nathan vint le trouver après qu'il eut été avec Bethsabée. \EVERSE
\VERSE Ayez pitié de moi, ô Dieu, selon Votre grande miséricorde; et selon la multitude de Vos bontés, effacez mon iniquité. \EVERSE
\VERSE Lavez-moi de plus en plus de mon iniquité, et purifiez-moi de mon péché. \EVERSE
\VERSE Car je connais mon iniquité, et mon péché est toujours devant moi. \EVERSE
\VERSE J'ai péché contre Vous seul, et j'ai fait ce qui est mal à Vos yeux, afin que Vous soyez trouvé juste dans Vos paroles, et victorieux lorsqu'on Vous jugera. \EVERSE
\VERSE Car j'ai été conçu dans l'iniquité, et ma mère m'a conçu dans le péché. \EVERSE
\VERSE Car Vous avez aimé la vérité; Vous m'avez révélé les secrets et les mystères de Votre sagesse. \EVERSE
\VERSE Vous m'arroserez avec l'hysope, et je serai purifié; Vous me laverez, et je deviendrai plus blanc que la neige. \EVERSE
\VERSE Vous me ferez entendre une parole de joie et de bonheur, et mes os, qui sont brisés et humiliés, tressailliront d'allégresse. \EVERSE
\VERSE Détournez Votre face de mes péchés, et effacez toutes mes iniquités. \EVERSE
\VERSE O Dieu, créez en moi un coeur pur, et renouvelez un esprit droit dans mon sein. \EVERSE
\VERSE Ne me rejetez pas de devant Votre face, et ne retirez pas de moi Votre Esprit-Saint. \EVERSE
\VERSE Rendez-moi la joie de Votre salut, et affermissez-moi par un esprit généreux. \EVERSE
\VERSE J'enseignerai Vos voies aux méchants, et les impies se convertiront à Vous. \EVERSE
\VERSE Délivrez-moi du sang que j'ai versé, ô Dieu, Dieu de mon salut, et ma langue célébrera avec joie Votre justice. \EVERSE
\VERSE Seigneur, Vous ouvrirez mes lèvres, et ma bouche publiera Vos louanges. \EVERSE
\VERSE Car si Vous aviez désiré un sacrifice, je Vous l'aurais offert; mais Vous ne prenez pas plaisir aux holocaustes. \EVERSE
\VERSE le sacrifice digne de Dieu, c'est un esprit brisé; Vous ne mépriserez pas, ô Dieu, un coeur contrit et humilié. \EVERSE
\VERSE Seigneur, traitez favorablement Sion dans Votre bonté, afin que les murs de Jérusalem soient bâtis. \EVERSE
\VERSE Alors Vous agréerez un sacrifice de justice, les oblations et les holocaustes; alors on offrira de jeunes taureaux sur Votre autel.

}
%%%%%%% PSAUME 51 %%%%%%%
\newcommand{\psalmlifr}{
\VERSE Pour la fin, instruction de David, \EVERSE
\VERSE lorsque Doëg l'Iduméen vint annoncer cette nouvelle à Saül: David est venu dans la maison d'Achimélech. \EVERSE
\VERSE Pourquoi te glorifies-tu dans le mal, toi qui est vaillant pour commettre l'iniquité? \EVERSE
\VERSE Tout le jour ta langue a médité l'injustice; comme un rasoir affilé tu pratiques la tromperie. \EVERSE
\VERSE Tu as plus aimé la malice que la bonté, l'iniquité plus que les paroles de justice. \EVERSE
\VERSE Tu as aimé toutes les paroles de ruine, ô langue trompeuse. \EVERSE
\VERSE C'est pourquoi Dieu te détruira pour toujours; Il t'arrachera et te fera sortir de ta tente, et Il enlèvera ta racine de la terre des vivants. \EVERSE
\VERSE Les justes le verront, et craindront; et ils se riront de lui, en disant:  \EVERSE
\VERSE Voilà l'homme qui n'a point pris Dieu pour son protecteur, mais qui s'est confié dans la multitude de ses richesses, et qui s'est prévalu de sa vanité. \EVERSE
\VERSE Mais moi, je suis comme un olivier fertile dans la maison de Dieu. J'espère en la miséricorde de Dieu éternellement et à jamais. \EVERSE
\VERSE Je Vous louerai sans fin, parce que Vous avez fait cela; et j'attendrai Votre Nom, parce qu'il est bon, en présence de Vos saints.

}
%%%%%%% PSAUME 52 %%%%%%%
\newcommand{\psalmliifr}{
\VERSE Pour la fin, sur Maéleth, instruction de David. L'insensé a dit dans son coeur: Il n'y a point de Dieu. \EVERSE
\VERSE Ils se sont corrompus et sont devenus abominables dans leurs iniquités: il n'y en a point qui fasse le bien. \EVERSE
\VERSE Dieu a regardé du haut du Ciel sur les enfants des hommes, pour voir s'il y a quelqu'un qui soit intelligent et qui cherche Dieu. \EVERSE
\VERSE Tous se sont détournés, ils sont tous devenus inutiles; il n'y en a point qui fasse le bien, il n'y en a pas un seul. \EVERSE
\VERSE Ne comprendront-ils pas, tous ces hommes qui commettent l'iniquité, qui dévorent Mon peuple comme un morceau de pain? \EVERSE
\VERSE Ils n'ont pas invoqué Dieu; ils ont tremblé de frayeur là où il n'y avait rien à craindre. Car Dieu a brisé les os de ceux qui cherchent à plaire aux hommes; ils ont été confondus, parce que Dieu les a méprisés. \EVERSE
\VERSE Qui procurera de Sion le salut d'Israël? Quand Dieu aura mis fin à la captivité de Son peuple, Jacob sera dans l'allégresse et Israël dans la joie.

}
%%%%%%% PSAUME 53 %%%%%%%
\newcommand{\psalmliiifr}{
\VERSE Pour la fin, parmi les cantiques, instruction de David, \EVERSE
\VERSE lorsque les habitants de Ziph vinrent dire à Saül: David n'est-il pas caché parmi nous? \EVERSE
\VERSE O Dieu, sauvez-moi par Votre Nom, et rendez-moi justice par Votre puissance. \EVERSE
\VERSE O Dieu, exaucez ma prière; prêtez l'oreille aux paroles de ma bouche. \EVERSE
\VERSE Car des étrangers se sont élevés contre moi, et des hommes puissants ont cherché à m'ôter la vie; et ils n'ont point placé Dieu devant leurs yeux. \EVERSE
\VERSE Mais voici que Dieu vient à mon aide, et que le Seigneur est le protecteur de ma vie. \EVERSE
\VERSE Faites retomber les maux sur mes ennemis, et exterminez-les dans Votre vérité. \EVERSE
\VERSE Je Vous offrirai volontairement des sacrifices; et je célébrerai Votre Nom, Seigneur, parce qu'il est bon. \EVERSE
\VERSE Car Vous m'avez délivré de toute affliction, et mon oeil a regardé mes ennemis avec assurance.

}
%%%%%%% PSAUME 54 %%%%%%%
\newcommand{\psalmlivfr}{
\VERSE Pour la fin, parmi les cantiques, instruction de David. \EVERSE
\VERSE Exaucez, ô Dieu, ma prière, et ne méprisez pas ma supplication.  \EVERSE
\VERSE Ecoutez-moi, et exaucez-moi. J'ai été rempli de tristesse dans mon épreuve, et le trouble m'a saisi \EVERSE
\VERSE à la voix de l'ennemi, et devant l'oppression du pécheur. Car ils m'ont accusé de crimes, et dans leur colère ils m'ont affligé. \EVERSE
\VERSE Mon coeur s'est troublé au dedans de moi, et les terreurs de la mort sont tombées sur moi. \EVERSE
\VERSE La crainte et le tremblement m'ont saisi, et les ténèbres m'ont enveloppé. \EVERSE
\VERSE Et j'ai dit: Qui me donnera des ailes comme à la colombe, pour que je puisse m'envoler et me reposer? \EVERSE
\VERSE Voici que je me suis éloigné en fuyant, et j'ai demeuré au désert. \EVERSE
\VERSE J'attendais là Celui qui m'a sauvé de l'abattement de l'esprit et de la tempête. \EVERSE
\VERSE Perdez-les, Seigneur, divisez leurs langues; car j'ai vu l'iniquité et la contradiction dans la ville. \EVERSE
\VERSE Jour et nuit l'iniquité fait le tour de ses murs; au milieu d'elle sont le travail \EVERSE
\VERSE et l'injustice. L'usure et la tromperie ne quittent point ses places publiques. \EVERSE
\VERSE Car, si mon ennemi m'avait maudit, je l'aurais supporté. Et si celui qui me haïssait avait parlé de moi avec insolence, peut-être me serais-je caché de lui. \EVERSE
\VERSE Mais toi, qui ne faisais qu'un avec moi, mon conseiller et mon ami; \EVERSE
\VERSE toi qui avec moi partageais les doux mets de ma table: nous marchions avec tant d'union dans la maison de Dieu! \EVERSE
\VERSE Que la mort fonde sur eux, et qu'ils descendent tout vivants dans l'enfer. Car l'iniquité est dans leurs demeures, en eux-mêmes. \EVERSE
\VERSE Mais moi j'ai crié vers Dieu, et le Seigneur me sauvera. \EVERSE
\VERSE Le soir, le matin et à midi, je raconterai et j'annoncerai mes misères, et Il exaucera ma voix. \EVERSE
\VERSE Il délivrera en paix mon âme de ceux qui s'approchent pour me perdre; car ils étaient en grand nombre contre moi. \EVERSE
\VERSE Dieu m'exaucera, et Il les humiliera, Lui qui est avant tous les siècles. Car il n'y a point de changement en eux, et ils ne craignent pas Dieu.  \EVERSE
\VERSE Il a étendu Sa main pour leur rendre ce qu'ils méritaient. Ils ont souillé Son alliance; \EVERSE
\VERSE ils ont été dissipés par la colère de Son visage, et Son coeur s'est approché. Ses discours sont plus doux que l'huile; mais ils sont en même temps comme des flèches. \EVERSE
\VERSE Jette ton souci sur le Seigneur, et Lui-même Il te nourrira; Il ne laissera pas le juste dans une éternelle agitation. \EVERSE
\VERSE Mais Vous, ô Dieu, Vous les conduirez jusque dans l'abîme de la mort. Les hommes sanguinaires et trompeurs n'arriveront point à la moitié de leurs jours; mais moi, j'espérerai en Vous, Seigneur.

}
%%%%%%% PSAUME 55 %%%%%%%
\newcommand{\psalmlvfr}{
\VERSE Pour la fin, pour le peuple qui a été éloigné des Saints, de David, inscription du titre, lorsque les Philistins l'eurent arrêté à Geth. \EVERSE
\VERSE Ayez pitié de moi, ô Dieu, car l'homme m'a foulé aux pieds; m'attaquant tout le jour, il m'a tourmenté. \EVERSE
\VERSE Mes ennemis m'ont foulè aux pieds tout le jour; car il y en a beaucoup qui me font la guerre. \EVERSE
\VERSE La hauteur du jour me donnera de la crainte; mais j'espérerai en Vous. \EVERSE
\VERSE Je louerai en Dieu les paroles qu'Il m'a fait entendre; j'espère en Dieu; je ne craindrai point ce que la chair peut me faire. \EVERSE
\VERSE Tout le jour ils avaient mes paroles en exécration; toutes leurs pensées tendaient à me faire du mal. \EVERSE
\VERSE Ils s'assembleront et se cacheront; ils observeront mes démarches. De même qu'ils en ont voulu à ma vie, \EVERSE
\VERSE Vous ne les sauverez nullement; dans Votre colère Vous briserez les peuples. O Dieu, \EVERSE
\VERSE je Vous ai exposé toute ma vie; Vous avez mis mes larmes devant Vous, selon Votre promesse.  \EVERSE
\VERSE Alors mes ennemis devront retourner en arrière. En quelque jour que je Vous invoque, je connais que Vous êtes mon Dieu. \EVERSE
\VERSE Je louerai en Dieu la parole qu'Il m'a donnée; je louerai dans le Seigneur Sa promesse. J'espère en Dieu; je ne craindrai point ce que l'homme peut me faire. \EVERSE
\VERSE Je connais, ô Dieu, les voeux que je Vous ai faits, et les louanges, dont j'ai à m'acquitter envers Vous. \EVERSE
\VERSE Car Vous avez délivré mon âme de la mort, et mes pieds de la chute, afin que je me rende agréable devant Dieu à la lumière des vivants.

}
%%%%%%% PSAUME 56 %%%%%%%
\newcommand{\psalmlvifr}{
\VERSE n'exterminez pas; de David, inscription du titre, lorsqu'il s'enfuit de devant Saül dans une caverne. \EVERSE
\VERSE Ayez pitié de moi, ô Dieu, ayez pitié de moi, car mon âme a confiance en Vous. Et j'espérerai à l'ombre de Vos ailes, jusqu'à ce que l'iniquité ait passé. \EVERSE
\VERSE Je crierai vers le Dieu très haut, le Dieu qui m'a fait du bien. \EVERSE
\VERSE Il a envoyé du Ciel Son secours, et Il m'a délivré; Il a couvert d'opprobre ceux qui me foulaient aux pieds. Dieu a envoyé Sa miséricorde et Sa vérité,  \EVERSE
\VERSE et Il a arraché mon âme du milieu des petits des lions; j'ai dormi plein de trouble. Les enfants des hommes ont pour dents des armes et des flèches, et leur langue est un glaive acéré. \EVERSE
\VERSE Soyez exalté au-dessus des cieux, ô Dieu, et que votre gloire brille par toute la terre. \EVERSE
\VERSE Ils ont préparé un filet pour mes pieds, et ils ont courbé mon âme. Ils ont creusé une fosse devant moi, et ils y sont eux-mêmes tombés. \EVERSE
\VERSE Mon coeur est préparé, ô Dieu, mon coeur est préparé; je chanterai, et je psalmodierai. \EVERSE
\VERSE Levez-vous, ma gloire; levez-vous, mon luth et ma harpe; je me lèverai dès l'aurore. \EVERSE
\VERSE Je Vous célébrerai, Seigneur, au milieu des peuples, et je Vous chanterai parmi les nations; \EVERSE
\VERSE car Votre miséricorde s'est élevé jusqu'aux cieux, et Votre vérité jusqu'aux nues. \EVERSE
\VERSE Soyez exalté, ô Dieu, au-dessus des cieux, et que Votre gloire brille par toute la terre.

}
%%%%%%% PSAUME 57 %%%%%%%
\newcommand{\psalmlviifr}{
\VERSE n'exterminez pas; de David, inscription du titre. \EVERSE
\VERSE Parlez-vous vraiment selon la justice? Jugez avec droiture, fils des hommes. \EVERSE
\VERSE Mais dans votre coeur vous formez des desseins d'iniquité; dans le pays vos mains ourdissent des injustices. \EVERSE
\VERSE Les pécheurs sont pervertis dès le sein maternel, ils se sont égarés dès leur naissance; ils ont dit des choses fausses. \EVERSE
\VERSE Leur fureur est semblable à celle du serpent, et de l'aspic sourd, qui ferme ses oreilles, \EVERSE
\VERSE et qui n'entend pas la voix des enchanteurs, et du magicien qui use d'adresse pour le charmer. \EVERSE
\VERSE Dieu brisera leurs dents dans leur bouche; le Seigneur mettra en pièces les mâchoires des lions. \EVERSE
\VERSE Ils seront réduits à rien, comme une eau qui s'écoule; Il a tendu Son arc jusqu'à ce qu'ils devinssent impuissants. \EVERSE
\VERSE Comme la cire qui coule, ils seront enlevés; le feu est tombé d'en haut sur eux, et ils n'ont plus vu le soleil. \EVERSE
\VERSE Avant qu'ils connaissent que leurs épines sont devenues un buisson, Il les engloutit comme tout vivants dans Sa colère. \EVERSE
\VERSE Le juste se réjouira en voyant la vengeance; il lavera ses mains dans le sang du pécheur. \EVERSE
\VERSE Et les hommes diront: Oui, il y a une récompense pour le juste; oui, il y a un Dieu qui les juge sur la terre.

}
%%%%%%% PSAUME 58 %%%%%%%
\newcommand{\psalmlviiifr}{
\VERSE n'exterminez pas; de David, pour l'inscription du titre, quand Saül envoya garder sa maison pour le tuer. \EVERSE
\VERSE Sauvez-moi des mains de mes ennemis, ô mon Dieu, et délivrez-moi de ceux qui se lèvent contre moi. \EVERSE
\VERSE Délivrez-moi de ceux qui commettent l'iniquité, et sauvez-moi des hommes de sang. \EVERSE
\VERSE Car voici qu'ils se sont rendus maîtres de ma vie; des hommes puissants se sont précipités sur moi. \EVERSE
\VERSE Il n'y a eu ni faute ni péché de ma part, Seigneur; j'ai couru et j'ai conduit mes pas sans injustice. \EVERSE
\VERSE Levez-Vous au-devant de moi, et voyez. Et Vous, Seigneur, Dieu des armées, Seigneur d'Israël, appliquez-Vous à visiter toutes les nations; n'ayez pas pitié de tous ceux qui commettent l'iniquité. \EVERSE
\VERSE Ils reviendront le soir, et ils seront affamés comme des chiens, et ils feront le tour de la ville. \EVERSE
\VERSE Voici qu'ils parleront de leur bouche, et un glaive sera sur leurs lèvres; car qui est-ce qui a entendu? \EVERSE
\VERSE Et Vous, Seigneur, Vous Vous rirez d'eux; Vous réduirez à néant toutes les nations. \EVERSE
\VERSE C'est en Vous que je conserverai ma force; car, ô Dieu, Vous êtes mon défenseur.  \EVERSE
\VERSE La miséricorde de mon Dieu me préviendra. \EVERSE
\VERSE Dieu me fera regarder par-dessus mes ennemis. Ne les tuez pas, de peur qu'on n'oublie, mon peuple. Dispersez-les par Votre puissance, et renversez-les, Seigneur, Vous qui êtes mon protecteur, \EVERSE
\VERSE à cause du crime de leur bouche, des paroles de leurs lèvres; et qu'ils soient pris dans leur orgueil. Et l'on publiera leurs malédictions et leurs mensonges, \EVERSE
\VERSE au jour de la consommation, dans la colère de la consommation; et ils ne seront plus. Et ils sauront que Dieu règnera sur Jacob et jusqu'aux extrémités de la terre. \EVERSE
\VERSE Ils reviendront le soir, et ils seront affamés comme des chiens, et ils feront le tour de la ville. \EVERSE
\VERSE Ils se disperseront pour manger; mais, s'ils ne sont point rassasiés, ils murmureront. \EVERSE
\VERSE Mais moi, je chanterai Votre puissance, et le matin je célébrerai avec joie Votre miséricorde. Car Vous Vous êtes fait mon protecteur et mon refuge au jour de ma tribulation. \EVERSE
\VERSE O mon défenseur, je Vous célébrerai, parce que Vous êtes le Dieu qui me protégez, mon Dieu, ma miséricorde.

}
%%%%%%% PSAUME 59 %%%%%%%
\newcommand{\psalmlixfr}{
\VERSE pour ceux qui seront changés, inscription du titre, instruction de David, \EVERSE
\VERSE lorsqu'il brûla la Mésopotamie de Syrie et Sobal, et que Joab revint et frappa l'Idumée dans la vallée des Salines, tuant douze mille hommes. \EVERSE
\VERSE O Dieu, Vous nous avez repoussés et Vous nous avez détruits; Vous Vous êtes irrité, et Vous avez eu pitié de nous. \EVERSE
\VERSE Vous avez ébranlé la terre, et Vous l'avez troublée. Guérissez ses brisures, car elle est ébranlée. \EVERSE
\VERSE Vous avez fait voir à Votre peuple des choses dures; Vous nous avez abreuvés d'un vin de douleur. \EVERSE
\VERSE Vous avez donné à ceux qui Vous craignent un signal, afin qu'ils fuient de devant l'arc. Pour que Vos bien-aimés soient délivrés,  \EVERSE
\VERSE sauvez-nous par Votre droite, et exaucez-moi. \EVERSE
\VERSE Dieu a parlé dans Son sanctuaire: Je me réjouirai, et je partagerai Sichem, et je mesurerai la vallée des Tentes. \EVERSE
\VERSE Galaad est à moi, et à moi Manassé; et Ephraïm est la force de ma tête. Juda est mon roi. \EVERSE
\VERSE Moab est comme le vase de mon espérance. J'étendrai ma chaussure sur l'Idumée; les étrangers me sont assujettis. \EVERSE
\VERSE Qui me conduira à la ville fortifiée? Qui me conduira jusqu'en Idumée? \EVERSE
\VERSE N'est-ce pas Vous, ô Dieu, qui nous avez repoussés? et ne sortirez-Vous pas, ô Dieu, à la tête de nos armées? \EVERSE
\VERSE Donnez-nous du secours contre la tribulation, car la protection de l'homme est vaine. \EVERSE
\VERSE Avec Dieu nous ferons des actes de courage, et Lui-même réduira à néant ceux qui nous persécutent.

}
%%%%%%% PSAUME 60 %%%%%%%
\newcommand{\psalmlxfr}{
\VERSE Pour la fin, sur les cantiques, de David. \EVERSE
\VERSE Exaucez, ô Dieu, ma supplication; soyez attentif à ma prière. \EVERSE
\VERSE Des extrémités de la terre j'ai crié vers Vous, lorsque mon coeur était dans l'angoisse; Vous m'avez élevé sur la pierre. Vous m'avez conduit, \EVERSE
\VERSE parce que Vous êtes devenu mon espérance, une tour solide contre l'ennemi. \EVERSE
\VERSE J'habiterai à jamais dans Vos tabernacles; je trouverai un abri à l'ombre de Vos ailes. \EVERSE
\VERSE Car Vous, mon Dieu, Vous avez exaucé ma prière; Vous avez donné un héritage à ceux qui craignent Votre nom. \EVERSE
\VERSE Vous ajouterez des jours aux jours du roi; Vous étendrez Ses années de génération en génération. \EVERSE
\VERSE Il demeure éternellement en présence de Dieu. Qui scrutera Sa miséricorde et Sa vérité? \EVERSE
\VERSE Ainsi je chanterai un cantique à Votre nom dans les siècles des siècles, pour m'acquitter chaque jour de mes voeux.

}
%%%%%%% PSAUME 61 %%%%%%%
\newcommand{\psalmlxifr}{
\VERSE Pour la fin, pour Idithun, psaume de David. \EVERSE
\VERSE Mon âme ne sera-t-elle pas soumise à Dieu? car c'est de Lui que vient mon salut. \EVERSE
\VERSE Car c'est Lui qui est mon Dieu et mon sauveur; Il est mon protecteur, je ne serai plus ébranlé. \EVERSE
\VERSE Jusques à quand vous jetterez-vous sur un homme? Vous le tuez tous ensemble, comme une muraille qui penche, et une masure tout ébranlée. \EVERSE
\VERSE Cependant ils ont entrepris de me dépouiller de ma dignité; j'ai couru altéré; de leur bouche ils bénissaient, et dans leur coeur ils maudissaient. \EVERSE
\VERSE Cependant sois soumise à Dieu, mon âme, car c'est de Lui que vient ma patience. \EVERSE
\VERSE Car c'est Lui qui est mon Dieu et mon sauveur; Il est mon protecteur, et je ne fuirai point. \EVERSE
\VERSE En Dieu est mon salut et ma gloire; il est le Dieu qui me secourt, et mon espérance est en Dieu. \EVERSE
\VERSE Espérez en Lui, Vous tous qui composez le peuple; répandez devant Lui vos coeurs; Dieu est notre défenseur à jamais. \EVERSE
\VERSE Mais les fils des hommes sont vains; les fils des hommes sont des menteurs dans leurs balances, afin de tromper ensemble pour des choses vaines. \EVERSE
\VERSE Ne mettez pas votre espérance dans l'iniquité, et ne désirez point les rapines. Si les richesses affluent, n'y attachez pas votre coeur. \EVERSE
\VERSE Dieu a parlé une fois; j'ai entendu ces deux choses: la puissance est à Dieu, \EVERSE
\VERSE et à Vous, Seigneur, la miséricorde; car Vous rendrez à chacun selon ses oeuvres.

}
%%%%%%% PSAUME 62 %%%%%%%
\newcommand{\psalmlxiifr}{
\VERSE Psaume de David, lorsqu'il était dans le désert d'Idumée. \EVERSE
\VERSE O Dieu, mon Dieu, je veille aspirant à Vous dès l'aurore. Mon âme a soif de Vous. Et combien ma chair aussi est altérée de Vous! \EVERSE
\VERSE Dans cette terre déserte, et sans chemin, et sans eau, c'est ainsi que je me suis présenté devant Vous dans le sanctuaire, pour contempler Votre puissance et Votre gloire. \EVERSE
\VERSE Car Votre miséricorde est meilleure que toutes les vies; mes lèvres Vous loueront. \EVERSE
\VERSE Ainsi je Vous bénirai toute ma vie, et je lèverai mes mains en Votre nom. \EVERSE
\VERSE Que mon âme soit comme rassasiée et engraissée, et ma bouche Vous louera avec des lèvres d'allégresse. \EVERSE
\VERSE Si je me souviens de Vous sur ma couche, dès le matin je méditerai sur Vous.  \EVERSE
\VERSE Car Vous avez été mon défenseur, et je me réjouirai à l'ombre de Vos ailes. \EVERSE
\VERSE Mon âme s'est attachée à Votre suite, et Votre droite m'a soutenu. \EVERSE
\VERSE Quant à eux, c'est en vain qu'ils ont cherché à m'ôter la vie. Ils entreront dans les profondeurs de la terre;  \EVERSE
\VERSE ils seront livrés au pouvoir du glaive; ils deviendront la proie des renards. \EVERSE
\VERSE Mais le roi se réjouira en Dieu; tous ceux qui jurent par lui se féliciteront, car la bouche de ceux qui profèrent l'iniquité a été fermée.

}
%%%%%%% PSAUME 63 %%%%%%%
\newcommand{\psalmlxiiifr}{
\VERSE Pour la fin, psaume de David. \EVERSE
\VERSE Exaucez, ô Dieu, ma prière lorsque je Vous implore; délivrez mon âme de la crainte de l'ennemi. \EVERSE
\VERSE Vous m'avez protégé contre l'assemblée des méchants, contre la multitude de ceux qui commettent l'iniquité. \EVERSE
\VERSE Car ils ont aiguisé leurs langues comme un glaive, et ils ont tendu leur arc, chose amère, \EVERSE
\VERSE pour percer de flèches l'innocent dans l'obscurité. \EVERSE
\VERSE Ils le perceront soudain, et ils n'éprouveront aucune crainte; ils se sont affermis dans leur résolution perverse. Ils se sont concertés pour cacher des pièges; ils ont dit: Qui les verra? \EVERSE
\VERSE Ils ont inventé des crimes; ils se sont épuisés dans une profonde recherche. L'homme pénétrera au fond de son coeur,  \EVERSE
\VERSE et Dieu sera exalté. Les blessures qu'ils font sont comme celles des flèches des petits enfants,  \EVERSE
\VERSE et leurs langues ont perdu leur force en se tournant contre eux-mêmes. Tous ceux qui les voyaient ont été remplis de trouble,  \EVERSE
\VERSE et tout homme a été saisi de frayeur. Et ils ont annoncé les oeuvres de Dieu, et ils ont compris Ses actes. \EVERSE
\VERSE Le juste se réjouira dans le Seigneur, et espérera en Lui; et tous ceux qui ont le coeur droit se féliciteront.

}
%%%%%%% PSAUME 64 %%%%%%%
\newcommand{\psalmlxivfr}{
\VERSE Pour la fin, psaume de David, cantique de Jérémie et d'Ezéchiel, pour le peuple de la captivité, lorsqu'il commençait à partir. \EVERSE
\VERSE L'hymne de louange Vous est due, ô Dieu, dans Sion, et on Vous rendra des voeux dans Jérusalem. \EVERSE
\VERSE Exaucez ma prière; à Vous viendra toute chair. \EVERSE
\VERSE Les paroles des méchants ont prévalu sur nous, mais Vous nous pardonnerez nos impiétés. \EVERSE
\VERSE Heureux celui que Vous avez choisi et pris avec Vous; il habitera dans Vos parvis. Nous serons remplis de biens de Votre maison; Votre temple est saint, \EVERSE
\VERSE il est admirable en équité. Exaucez-nous, ô Dieu, notre sauveur, espérance de tous les confins de la terre et des lointains rivages de la mer. \EVERSE
\VERSE Vous affermissez les montagnes par Votre force, Vous qui êtes ceint de puissance,  \EVERSE
\VERSE qui troublez les profondeurs de la mer, et qui faites retentir le bruit de ses flots. Les nations seront troublées, \EVERSE
\VERSE et ceux qui habitent les extrémités de la terre seront effrayés par Vos prodiges; Vous réjouirez les contrées de l'orient et de l'occident. \EVERSE
\VERSE Vous avez visité la terre, et Vous l'avez enivrée de Vos pluies; Vous l'avez comblée de richesses. Le fleuve de Dieu a été rempli d'eaux; Vous avez préparé la nourriture de Votre peuple; car c'est ainsi que Vous préparez la terre. \EVERSE
\VERSE Enivrez d'eau ses ruisseaux, multipliez ses germes; sous ses ondées elle se réjouira, donnant ses fruits. \EVERSE
\VERSE Vous bénirez la couronne de l'année de Votre bonté, et Vos champs seront remplis d'abondantes récoltes. \EVERSE
\VERSE Les gracieux pâturages du désert seront engraissés, et les collines seront ceintes d'allégresse. \EVERSE
\VERSE Les béliers des brebis se revêtiront, et les vallées seront pleines de blé; tout chantera et fera entendre des hymnes.

}
%%%%%%% PSAUME 65 %%%%%%%
\newcommand{\psalmlxvfr}{
\VERSE Pour la fin, cantique, psaume de la résurrection. Poussez vers Dieu des cris de joie, ô terre entière; \EVERSE
\VERSE chantez un hymne à Son Nom; rendez glorieuse Sa louange. \EVERSE
\VERSE Dites à Dieu: Que Vos oeuvres sont terribles, Seigneur! A cause de la grandeur de Votre puissance, Vos ennemis Vous adressent des hommages menteurs. \EVERSE
\VERSE Que la terre Vous adore et chante en Votre honneur, qu'elle dise un hymne à Votre nom. \EVERSE
\VERSE Venez et voyez les oeuvres de Dieu; Il est terrible dans Ses desseins sur les enfants des hommes. \EVERSE
\VERSE Il a changé la mer en une terre sèche; ils ont passé le fleuve à pied, c'est là que nous nous réjouirons en Lui. \EVERSE
\VERSE Il règne à jamais par Sa puissance, Ses yeux contemplent les nations; que ceux-là qui l'irritent ne s'élèvent point en eux-mêmes. \EVERSE
\VERSE Nations, bénissez notre Dieu, et faites entendre les accents de Sa louange. \EVERSE
\VERSE C'est Lui qui a conservé la vie à mon âme, et qui n'a point permis que mes pieds soient ébranlés. \EVERSE
\VERSE Car Vous nous avez éprouvés, ô Dieu; Vous nous avez fait passer par le feu, comme on y fait passer l'argent. \EVERSE
\VERSE Vous nous avez fait tomber dans le piège; Vous avez chargé nos épaules de tribulations;  \EVERSE
\VERSE Vous avez mis des hommes sur nos têtes. Nous avons passé par le feu et par l'eau; et Vous nous en avez tirés pour nous mettre en un lieu de rafraîchissement. \EVERSE
\VERSE J'entrerai dans Votre maison avec des holocaustes; je m'acquitterai envers Vous de mes voeux \EVERSE
\VERSE que mes lèvres ont proférés, et que ma bouche a prononcés pendant ma tribulation. \EVERSE
\VERSE Je Vous offrirai de gras holocaustes, avec la fumée des béliers; je Vous offrirai des boeufs avec des boucs. \EVERSE
\VERSE Venez, entendez, vous tous qui craignez Dieu, et je vous raconterai tout ce qu'Il a fait à mon âme. \EVERSE
\VERSE Ma bouche a crié vers Lui, et ma langue L'a exalté. \EVERSE
\VERSE Si j'avais vu l'iniquité dans mon coeur, le Seigneur ne m'aurait pas exaucé. \EVERSE
\VERSE C'est pourquoi Dieu m'a exaucé, et a été attentif à la voix de ma supplication. \EVERSE
\VERSE Béni soit Dieu, qui n'a pas rejeté ma prière, ni éloigné de moi Sa miséricorde.

}
%%%%%%% PSAUME 66 %%%%%%%
\newcommand{\psalmlxvifr}{
\VERSE Pour la fin, parmi les hymnes, Psaume, cantique de David. \EVERSE
\VERSE Que Dieu ait pitié de nous, et nous bénisse; qu'Il fasse briller Son visage sur nous, et qu'Il ait pitié de nous. \EVERSE
\VERSE Afin que nous connaissions Votre voie sur la terre, et Votre salut parmi toutes les nations. \EVERSE
\VERSE Que les peuples Vous glorifient, ô Dieu; que tous les peuples Vous glorifient! \EVERSE
\VERSE Que les nations soient dans la joie et l'allégresse, parce que Vous jugez les peuples dans l'équité, et que Vous dirigez les nations sur la terre. \EVERSE
\VERSE Que les peuples Vous glorifient, ô Dieu; que tous les peuples Vous glorifient!  \EVERSE
\VERSE La terre a donné son fruit. Que Dieu, notre Dieu, nous bénisse! \EVERSE
\VERSE Que Dieu nous bénisse, et que tous les confins de la terre Le craignent!

}
%%%%%%% PSAUME 67 %%%%%%%
\newcommand{\psalmlxviifr}{
\VERSE Pour la fin, Psaume, cantique de David lui-même. \EVERSE
\VERSE Que Dieu Se lève, et que Ses ennemis soient dissipés: et que ceux qui Le haïssent fuient devant Sa face. \EVERSE
\VERSE Comme la fumée disparaît, qu'ils disparaissent; comme la cire se fond devant le feu, qu'ainsi périssent les pécheurs devant la face de Dieu. \EVERSE
\VERSE Mais que les justes soient comme dans un festin, et qu'ils tressaillent en la présence de Dieu, et qu'ils soient dans des transports de joie. \EVERSE
\VERSE Chantez à Dieu, célébrez Son Nom par un cantique; frayez le chemin à Celui qui monte vers le couchant. Le Seigneur est Son Nom. Tressaillez de joie en Sa présence. On tremblera devant Lui. \EVERSE
\VERSE Il est le père des orphelins et le juge des veuves. Dieu est dans Son lieu saint.  \EVERSE
\VERSE C'est le Dieu qui fait habiter dans une même maison ceux qui ont un même esprit; qui délivre les captifs par Sa puissance, aussi bien que ceux qui L'irritent, qui habitent dans les sépulcres. \EVERSE
\VERSE O Dieu, quand Vous marchiez à la tête de Votre peuple, quand Vous traversiez le désert, \EVERSE
\VERSE la terre fut ébranlée, les cieux eux-mêmes se fondirent devant le Dieu du Sinaï, devant le Dieu d'Israël. \EVERSE
\VERSE Vous avez mis en réserve une pluie toute volontaire, ô Dieu, pour Votre héritage; et lorsqu'il a été affaibli, Vous l'avez réconforté. \EVERSE
\VERSE Vos animaux y habiteront: Vous avez dans Votre bonté, ô Dieu, préparé de la nourriture pour le pauvre. \EVERSE
\VERSE Le Seigneur donne Ses ordres à Ses messagers avec une grande puissance. \EVERSE
\VERSE Le roi des armées est au pouvoir du bien-aimé, du bien-aimé; et celle qui est l'ornement de la maison partage les dépouilles. \EVERSE
\VERSE Quand Vous dormez au milieu de Vos héritages, les ailes de la colombe sont argentées, et l'extrémité de son dos a le pâle éclat de l'or. \EVERSE
\VERSE Lorsque le Très-Haut disperse les rois dans le pays, tout est blanchi par les neiges sur le Selmon.  \EVERSE
\VERSE La montagne de Dieu est une grasse montagne. C'est une montagne massive, une grasse montagne. \EVERSE
\VERSE Pourquoi regardez-vous avec admiration les montagnes massives? Il est une montagne où il a plu à Dieu d'habiter; et le Seigneur y habitera à jamais. \EVERSE
\VERSE Le char de Dieu est environné de plus de dix mille; ce sont des milliers d'Anges qui se réjouissent; le Seigneur est au milieu d'eux dans Son sanctuaire, comme au Sinaï. \EVERSE
\VERSE Vous êtes monté en haut; Vous avez emmené des captifs; Vous avez reçu des présents parmi les hommes, et même de ceux qui ne croient pas que le Seigneur Dieu habite avec nous. \EVERSE
\VERSE Que le Seigneur soit béni chaque jour! Le Dieu qui nous a si souvent sauvés rendra notre voie prospère. \EVERSE
\VERSE Notre Dieu est le Dieu qui a la vertu de sauver; au Seigneur, au Seigneur appartiennent les issues de la mort. \EVERSE
\VERSE Mais Dieu brisera la tête de Ses ennemis, le front superbe de ceux qui marchent dans leurs iniquités. \EVERSE
\VERSE Le Seigneur a dit: Je les ramènerai de Basan, et Je les ramènerai du fond de la mer; \EVERSE
\VERSE afin que ton pied trempe dans le sang, et que la langue de tes chiens ait aussi sa part des ennemis. \EVERSE
\VERSE Ils ont vu Votre entrée, ô Dieu, l'entrée de mon Dieu, de mon Roi, qui réside dans le sanctuaire. \EVERSE
\VERSE En avant marchaient les princes, associés aux chanteurs, au milieu des jeunes filles qui jouaient du tambourin. \EVERSE
\VERSE Bénissez le Seigneur Dieu dans les assemblées, vous qui sortez des sources d'Israël. \EVERSE
\VERSE Là est Benjamin, le plus jeune, en de saints transports; là sont les princes de Juda, leurs chefs; les princes de Zabulon, les princes de Nephthali. \EVERSE
\VERSE O Dieu, commandez à Votre puissance; affermissez, ô Dieu, ce que Vous avez fait parmi nous. \EVERSE
\VERSE Dans Votre temple de Jérusalem, les rois Vous offriront des présents. \EVERSE
\VERSE Réprimez les bêtes sauvages des roseaux, la troupe des tauraux et les troupeaux des peuples, pour chasser ceux qui ont été éprouvés comme l'argent. Dissipez les nations qui veulent la guerre. \EVERSE
\VERSE Des ambassadeurs viendront de l'Egypte; l'Ethiopie s'empressera de tendre ses mains vers Dieu. \EVERSE
\VERSE Royaumes de la terre, chantez à Dieu; célébrez le Seigneur, célébrez Dieu, \EVERSE
\VERSE qui S'élève au plus haut des cieux, vers l'orient. Voici qu'Il va donner à Sa voix un puissant éclat. \EVERSE
\VERSE Rendez gloire à Dieu au sujet d'Israël. Sa magnificence et Sa force paraissent dans les nuées. \EVERSE
\VERSE Dieu est admirable dans Ses saints; le Dieu d'Israël donnera Lui-même à Son peuple la puissance et la force. Dieu soit béni!

}
%%%%%%% PSAUME 68 %%%%%%%
\newcommand{\psalmlxviiifr}{
\VERSE Pour la fin, pour ceux qui seront changés, Psaume de David. \EVERSE
\VERSE Sauvez-moi, ô Dieu, car les eaux sont entrées jusqu'à mon âme. \EVERSE
\VERSE Je suis enfoncé dans une boue profonde, où il n'y a pas de consistance. Je suis descendu au fond de la mer, et la tempête m'a submergé. \EVERSE
\VERSE Je me suis fatigué à crier, ma gorge en a été enrouée; mes yeux se sont épuisés, tandis que j'attends mon Dieu. \EVERSE
\VERSE Ils sont devenus plus nombreux que les cheveux de ma tête, ceux qui me haïssent sans cause. Ils sont devenus forts, mes ennemis qui me persécutent injustement; j'ai dû payer ce que je n'avais pas pris. \EVERSE
\VERSE O Dieu, Vous connaissez ma folie, et mes péchés ne Vous sont point cachés. \EVERSE
\VERSE Que ceux qui espèrent en Vous ne rougissent pas à cause de moi, Seigneur, Seigneur des armées. Qu'ils ne soient pas confondus à mon sujet, ceux qui Vous cherchent, Dieu d'Israël. \EVERSE
\VERSE Car c'est à cause de Vous que j'ai souffert l'opprobre, et que la confusion a couvert mon visage. \EVERSE
\VERSE Je suis devenu un étranger pour mes frères, et un inconnu pour les fils de ma mère. \EVERSE
\VERSE Car le zèle de Votre maison m'a dévoré, et les outrages de ceux qui Vous insultaient sont tombés sur moi. \EVERSE
\VERSE J'ai affligé mon âme par le jeûne, et l'on m'en a fait un sujet d'opprobre. \EVERSE
\VERSE J'ai pris pour vêtement un cilice, et je suis devenu leur fable. \EVERSE
\VERSE Ceux qui étaient assis à la porte parlaient contre moi, et ceux qui buvaient du vin me raillaient par leurs chansons. \EVERSE
\VERSE Mais moi je Vous adresse, Seigneur, ma prière. Voici le temps favorable, ô Dieu. Selon la grandeur de Votre miséricorde exaucez-moi, selon la vérité de Vos promesses de salut. \EVERSE
\VERSE Retirez-moi de la boue, afin que je n'y enfonce pas; délivrez-moi de ceux qui me haïssent et des eaux profondes. \EVERSE
\VERSE Que les flots en fureur ne me submergent point; que l'abîme ne m'engloutisse pas, et que le puits ne ferme pas sa bouche sur moi. \EVERSE
\VERSE Exaucez-moi, Seigneur, car Votre miséricorde est toute suave; regardez- moi selon l'abondance de Vos bontés. \EVERSE
\VERSE Et ne détournez pas Votre visage de Votre serviteur; parce que je suis dans l'angoisse, exaucez-moi promptement. \EVERSE
\VERSE Soyez attentif sur mon âme, et délivrez-la à cause de mes ennemis. \EVERSE
\VERSE Vous connaissez mon opprobre, et ma confusion, et ma honte. \EVERSE
\VERSE Tous ceux qui me persécutent sont devant Vous; mon coeur s'attend à l'insulte et à la misère. Et j'ai attendu que quelqu'un s'attristât avec moi, mais nul ne l'a fait; et que quelqu'un me consolât, mais je n'ai trouvé personne. \EVERSE
\VERSE Et ils m'ont donné du fiel pour nourriture, et dans ma soif ils m'ont abreuvé de vinaigre. \EVERSE
\VERSE Que leur table soit devant eux comme un filet, un juste châtiment et une pierre de scandale. \EVERSE
\VERSE Que leurs yeux soient obscurcis, pour qu'ils cessent de voir, et courbez à jamais leur dos. \EVERSE
\VERSE Déversez sur eux Votre colère, et que la fureur de Votre courroux les saisisse. \EVERSE
\VERSE Que leur demeure devienne déserte, et qu'il n'y ait personne qui habite dans leurs tentes. \EVERSE
\VERSE Parce qu'ils ont persécuté celui que Vous avez frappé, et qu'ils ont ajouté à la douleur de mes blessures. \EVERSE
\VERSE Ajoutez l'iniquité à leur iniquité, et qu'ils n'entrent pas dans Votre justice. \EVERSE
\VERSE Qu'ils soient effacés du livre des vivants, et qu'ils ne soient point inscrits avec les justes. \EVERSE
\VERSE Pour moi, je suis pauvre et dans la douleur; Votre salut, ô Dieu, m'a relevé. \EVERSE
\VERSE Je louerai le Nom de Dieu par des cantiques, et je le glorifierai par des louanges; \EVERSE
\VERSE et ce sera plus agréable à Dieu que le jeune veau, à qui poussent les cornes et les ongles. \EVERSE
\VERSE Que les pauvres le voient et se réjouissent. Cherchez Dieu, et votre âme vivra; \EVERSE
\VERSE car le Seigneur a exaucé les pauvres, et Il n'a pas méprisé Ses captifs. \EVERSE
\VERSE Que les cieux et la terre Le louent; la mer, et tout ce qui s'y meut. \EVERSE
\VERSE Car Dieu sauvera Sion, et les villes de Juda seront bâties. Ils y habiteront, et ils l'acquerront en héritage. \EVERSE
\VERSE Et la race de Ses serviteurs la possédera, et ceux qui aiment Son Nom y habiteront.

}
%%%%%%% PSAUME 69 %%%%%%%
\newcommand{\psalmlxixfr}{
\VERSE Pour la fin, psaume de David, en souvenir de ce que Dieu l'avait sauvé. \EVERSE
\VERSE O Dieu, venez à mon aide; Seigneur, hâtez-Vous de me secourir. \EVERSE
\VERSE Qu'ils soient confondus et couverts de honte, ceux qui cherchent à m'ôter la vie. \EVERSE
\VERSE Qu'ils reculent en arrière et soient dans la confusion, ceux qui me veulent du mal. Qu'ils reculent aussitôt, rougissant de honte, ceux qui me disent: Va! va! \EVERSE
\VERSE Mais que tous ceux qui Vous cherchent tressaillent d'allégresse et de joie; et que ceux qui aiment Votre salut disent sans cesse: Que le Seigneur soit glorifié! \EVERSE
\VERSE Pour moi, je suis pauvre et indigent; ô Dieu, aidez-moi. Vous êtes mon aide et mon libérateur. Seigneur, ne tardez pas.

}
%%%%%%% PSAUME 70 %%%%%%%
\newcommand{\psalmlxxfr}{
\VERSE Psaume de David, des fils de Jonadab, et des premiers captifs. C'est en Vous, Seigneur, que j'ai espéré; que je sois pas à jamais confondu. \EVERSE
\VERSE Dans Votre justice, délivrez-moi et secourez-moi Inclinez vers moi Votre oreille, et sauvez-moi. \EVERSE
\VERSE Soyez-moi un Dieu protecteur et un asile fortifié, afin de me sauver; car Vous êtes ma force et mon refuge. \EVERSE
\VERSE Mon Dieu, tirez-moi de la main du pécheur, et de la main de celui qui agit contre la loi, et du pervers; \EVERSE
\VERSE car Vous êtes mon attente, Seigneur; Seigneur, Vous êtes mon espérance depuis ma jeunesse. \EVERSE
\VERSE Sur Vous je me suis appuyé dès ma naissance; dès le sein de ma mère Vous êtes mon protecteur. Vous serez toujours le sujet de mes chants.  \EVERSE
\VERSE Je suis devenu pour beaucoup comme un prodige; et Vous, Vous êtes un puissant secours. \EVERSE
\VERSE Que ma bouche soit remplie de louanges, pour que je chante Votre gloire, et chaque jour Votre grandeur. \EVERSE
\VERSE Ne me rejetez pas au temps de la vieillesse; lorsque ma force se sera épuisée, ne m'abandonnez pas. \EVERSE
\VERSE Car mes ennemis ont parlé contre moi, et ceux qui épiaient ma vie ont tenu conseil ensemble, \EVERSE
\VERSE disant: Dieu l'a abandonné; poursuivez-le et saisissez-le; il n'y a personne pour le délivrer. \EVERSE
\VERSE O Dieu, ne Vous éloignez pas de moi; mon Dieu, voyez à me secourir. \EVERSE
\VERSE Qu'ils soient confondus et réduits à néant, ceux qui en veulent à ma vie;qu'ils soient couverts de confusion et de honte, ceux qui cherchent mon mal. \EVERSE
\VERSE Mais moi, j'espérerai toujours, et j'ajouterai à toutes Vos louanges. \EVERSE
\VERSE Ma bouche publiera Votre justice, et tout le jour Votre assistance salutaire. Ne connaissant pas la science humaine, \EVERSE
\VERSE je contemplerai les oeuvres puissantes du Seigneur; Seigneur, je me rappellerai Votre justice, la Vôtre seule. \EVERSE
\VERSE O Dieu, Vous m'avez instruit dès ma jeunesse, et je publierai Vos merveilles que j'ai éprouvées jusqu'à présent. \EVERSE
\VERSE Et jusqu'à la vieillesse et aux cheveux blancs, ô Dieu, ne m'abandonnez pas, jusqu'à ce que j'aie annoncé la force de Votre bras à toutes les générations à venir; Votre puissance \EVERSE
\VERSE et Votre justice qui atteint, ô Dieu, jusqu'aux cieux. Dans les grandes choses que Vous avez faites, ô Dieu, qui est semblable à Vous? \EVERSE
\VERSE Que de tribulations nombreuses et cruelles Vous m'avez fait éprouver! Et Vous retournant, Vous m'avez rendu la vie, et Vous m'avez retiré des abîmes de la terre. \EVERSE
\VERSE Vous avez fait éclater Votre magnificence, et, Vous retournant, Vous m'avez consolé. \EVERSE
\VERSE Car je célébrerai encore, ô Dieu, Votre vérité au son des instruments; je Vous chanterai sur la harpe, ô Saint d'Israël. \EVERSE
\VERSE L'allégresse sera sur mes lèvres lorsque je Vous chanterai, et dans mon âme, que Vous avez rachetée. \EVERSE
\VERSE Et ma langue annoncera tout le jour Votre justice, lorsque ceux qui cherchent mon mal seront couverts de confusion et de honte.

}
%%%%%%% PSAUME 71 %%%%%%%
\newcommand{\psalmlxxifr}{
\VERSE Psaume sur Solomon. \EVERSE
\VERSE O Dieu, donnez au Roi Votre jugement, et au Fils du roi Votre justice; pour qu'Il juge Votre peuple avec justice, et Vos pauvres selon l'équité. \EVERSE
\VERSE Que les montagnes reçoivent la paix pour le peuple, et les collines la justice! \EVERSE
\VERSE Il jugera les pauvres du peuple, et sauvera les enfants des pauvres, et humiliera le calomniateur. \EVERSE
\VERSE Et Il durera autant que le soleil et que la lune, de génération en génération. \EVERSE
\VERSE Il descendra comme la pluie sur une toison, et comme les eaux qui tombent goutte à goutte sur la terre. \EVERSE
\VERSE En Ses jours apparaîtra la justice et l'abondance de la paix, jusqu'à ce que la lune soit détruite. \EVERSE
\VERSE Et Il dominera de la mer à la mer, et depuis le fleuve jusqu'aux extrémités de la terre. \EVERSE
\VERSE Devant Lui se prosterneront les Ethiopiens, et Ses ennemis lécheront la terre. \EVERSE
\VERSE Les rois de Tharsis et les îles Lui offriront des présents; les rois d'Arabie et de Saba apporteront des dons; \EVERSE
\VERSE et tous les rois de la terre L'adoreront, toutes les nations Lui seront assujetties. \EVERSE
\VERSE Car Il délivrera le pauvre des mains du puissant, et l'indigent qui n'avait personne pour l'assister. \EVERSE
\VERSE Il aura compassion du pauvre et de l'indigent, et Il sauvera les âmes des pauvres. \EVERSE
\VERSE Il affranchira leurs âmes de l'usure et de l'iniquité, et leur nom sera en honneur devant Lui. \EVERSE
\VERSE Et Il vivra, et on Lui donnera de l'or d'Arabie; on L'adorera sans cesse, tout le jour on Le bénira. \EVERSE
\VERSE Et le blé sera sur la terre au sommet des montagnes; son fruit s'élèvera plus haut que le Liban, et on fleurira dans la cité comme l'herbe des champs. \EVERSE
\VERSE Que Son Nom soit béni dans tous les siècles: Son Nom durera autant que le soleil. Et toutes les tribus de la terre seront bénies en Lui; toutes les nations Le glorifieront. \EVERSE
\VERSE Béni soit le Seigneur, Dieu d'Israël, qui opère seul des merveilles. \EVERSE
\VERSE Et béni soit éternellement le Nom de Sa majesté, et que toute la terre soit remplie de Sa majesté. Ainsi soit-il, ainsi soit-il. \EVERSE
\VERSE Ici finissent les louanges de David, fils de Jessé.

}
%%%%%%% PSAUME 72 %%%%%%%
\newcommand{\psalmlxxiifr}{
\VERSE Psaume d'Asaph. Que Dieu est bon pour Israël, pour ceux qui ont le coeur droit! \EVERSE
\VERSE Mes pieds ont été presque ébranlés, mes pas presque renversés, \EVERSE
\VERSE parce que j'ai porté envie aux méchants, en voyant la paix des pécheurs. \EVERSE
\VERSE Car la mort paraît les oublier, et leurs blessures ne durent pas. \EVERSE
\VERSE Ils n'ont point de part au labeur des mortels, et ils ne sont pas frappés comme les autres hommes. \EVERSE
\VERSE Aussi l'orgueil les a-t-il saisis; ils sont couverts de leur iniquité et de leur impiété. \EVERSE
\VERSE L'iniquité sort comme de leur graisse; ils se sont abandonnés aux passions de leur coeur. \EVERSE
\VERSE Leurs pensées et leurs paroles n'ont été que malice; ils ont proféré hautement l'iniquité. \EVERSE
\VERSE Ils ont ouvert leur bouche contre le Ciel, et leur langue a parcouru la terre. \EVERSE
\VERSE C'est pourquoi mon peuple se tourne de ce côté, et on trouve en eux des jours pleins. \EVERSE
\VERSE Et ils ont dit: Comment Dieu le sait-Il? et le Très-Haut en a-t-Il connaissance? \EVERSE
\VERSE Voyez ces pécheurs qui abondent en tout en ce monde: ils ont acquis de nouvelles richesses. \EVERSE
\VERSE Et j'ai dit: C'est en vain que j'ai purifié mon coeur, et que j'ai lavé mes mains parmi les innocents, \EVERSE
\VERSE puisque j'ai été affligé tout le jour, et châtié dès le matin. \EVERSE
\VERSE Si j'avais dit: Je parlerai en ce sens, j'aurais condamné la race de Vos enfants. \EVERSE
\VERSE Je songeais à pénétrer ce secret; la difficulté fut grande devant moi, \EVERSE
\VERSE jusqu'à ce que je fusse entré dans le sanctuaire de Dieu, et que j'eusse compris ce que sera leur fin. \EVERSE
\VERSE En vérité, ce sont des pièges que Vous avez placés devant eux; Vous les avez renversés au moment même où ils s'élevaient. \EVERSE
\VERSE Comment sont-ils tombés dans la désolation? Ils ont disparu soudain; ils ont péri à cause de leur iniquité. \EVERSE
\VERSE Comme le songe de ceux qui s'éveillent, Seigneur, Vous réduirez au néant dans Votre cité leur image. \EVERSE
\VERSE Parce que mon coeur s'est enflammé, et que mes reins ont été altérés,  \EVERSE
\VERSE j'ai été réduit au néant, et plongé dans l'ignorance. \EVERSE
\VERSE Je suis devenu devant Vous comme une bête de somme, et cependant je suis toujours avec Vous. \EVERSE
\VERSE Vous avez tenu ma main droite, et Vous m'avez conduit selon Votre volonté, et Vous m'avez reçu avec gloire. \EVERSE
\VERSE Car qu'y a-t-il pour moi dans le Ciel? et qu'ai-je désiré de Vous sur la terre? \EVERSE
\VERSE Ma chair et mon coeur ont défailli, ô Dieu, qui êtes le Dieu de mon coeur, et mon partage pour l'éternité. \EVERSE
\VERSE Car voici que ceux qui s'éloignent de Vous périront; Vous avez résolu de perdre tous ceux qui se prostituent en s'éloignant de Vous. \EVERSE
\VERSE Pour moi, c'est mon bonheur de m'attacher à Dieu, de mettre mon espérance dans le Seigneur Dieu; afin de publier toutes Vos louanges aux portes de la fille de Sion.

}
%%%%%%% PSAUME 73 %%%%%%%
\newcommand{\psalmlxxiiifr}{
\VERSE Instruction d'Asaph. Pourquoi, ô Dieu, nous avez-Vous rejetés pour toujours? pourquoi Votre fureur s'est-elle allumée contre les brebis de Votre pâturage? \EVERSE
\VERSE Souvenez-Vous de Votre famille, que Vous avez possédé dès le commencement. Vous avez racheté le sceptre de Votre héritage: c'est le mont Sion, où Vous avez habité. \EVERSE
\VERSE Levez Vos mains contre leur insolence sans bornes. Que de forfaits l'ennemi a commis dans le sanctuaire! \EVERSE
\VERSE Ceux qui Vous haïssent ont fait leur gloire de Vous insulter au milieu de Votre solennité. Ils ont placé leurs étendards comme étendards,  \EVERSE
\VERSE et ils n'ont pas plus respecté le sommet que les issues. Comme dans une forêt d'arbres, à coups de hache, \EVERSE
\VERSE ils ont brisé les portes à l'envi. Avec la hache et la cognée ils ont tout renversé. \EVERSE
\VERSE Ils ont mis le feu à Votre sanctuaire; ils ont renversé et profané le tabernacle de Votre Nom. \EVERSE
\VERSE Ils ont dit dans leur coeur, eux et toute leur bande: Faisons cesser dans le pays tous les jours de fête consacrés à Dieu. \EVERSE
\VERSE Nous ne voyons plus nos étendards; il n'y a plus de prophète, et on ne nous connaîtra plus. \EVERSE
\VERSE Jusques à quand, ô Dieu, l'ennemi insultera-t-il? L'adversaire outragera-t-il sans fin Votre Nom? \EVERSE
\VERSE Pourquoi retirez-Vous sans cesse Votre main et Votre droite de Votre sein? \EVERSE
\VERSE Cependant Dieu est notre Roi depuis des siècles; Il a opéré notre salut au milieu de la terre. \EVERSE
\VERSE C'est Vous qui avez affermi la mer par Votre puissance, qui avez brisé les têtes des dragons dans les eaux. \EVERSE
\VERSE C'est Vous qui avez écrasé les têtes du dragon, qui l'avez donné en nourriture aux peuples d'Ethiopie. \EVERSE
\VERSE C'est Vous qui avez fait jaillir des fontaines et des torrents, qui avez desséché les fleuves intarissables. \EVERSE
\VERSE A Vous est le jour, et à Vous est la nuit; c'est Vous qui avez créé l'aurore et le soleil. \EVERSE
\VERSE C'est Vous qui avez établi toutes les limites de la terre, Vous qui avez formé l'été et le printemps. \EVERSE
\VERSE Souvenez-Vous-en: l'ennemi a outragé le Seigneur, et un peuple insensé a irrité Votre Nom. \EVERSE
\VERSE Ne livrez pas aux bêtes les âmes qui Vous louent, et n'oubliez pas pour toujours les âmes de Vos pauvres. \EVERSE
\VERSE Ayez égard à Votre alliance, car les lieux sombres du pays sont remplis de repaires d'iniquité. \EVERSE
\VERSE Que l'humble ne s'en retourne pas couvert de confusion; le pauvre et l'indigent loueront Votre Nom. \EVERSE
\VERSE Levez-Vous, ô Dieu, jugez Votre cause; souvenez-Vous des outrages qui Vous viennent tout le jour de l'insensé. \EVERSE
\VERSE N'oubliez pas les clameurs de Vos ennemis. L'orgueil de ceux qui Vous haïssent monte toujours.

}
%%%%%%% PSAUME 74 %%%%%%%
\newcommand{\psalmlxxivfr}{
\VERSE Pour la fin, Ne détruis pas, Psaume cantique d'Asaph. \EVERSE
\VERSE Nous Vous louerons, ô Dieu, nous Vous louerons, et nous invoquerons Votre Nom; nous raconterons Vos merveilles.  \EVERSE
\VERSE Au temps que J'aurai fixé, Je ferai parfaite justice. \EVERSE
\VERSE La terre s'est dissoute, avec tous ceux qui l'habitent. Moi J'ai affermi ses colonnes. \EVERSE
\VERSE J'ai dit aux méchants: Ne commettez plus l'iniquité; et aux pécheurs: N'élevez plus un front superbe. \EVERSE
\VERSE Ne levez plus si haut la tête; cessez de proférer des blasphèmes contre Dieu. \EVERSE
\VERSE Car ce n'est ni de l'orient, ni de l'occident, ni des montagnes désertes, que vous viendra le secours,  \EVERSE
\VERSE parce que c'est Dieu qui est juge. Il humilie celui-ci, et Il élève celui-là;  \EVERSE
\VERSE car il y a dans la main du Seigneur une coupe de vin pur, pleine d'aromates. Il en verse de côté et d'autre, et pourtant la lie n'en est pas encore épuisée; tous les pécheurs de la terre en boiront. \EVERSE
\VERSE Pour moi, j'annoncerai ces choses à jamais; je chanterai à la gloire du Dieu de Jacob. \EVERSE
\VERSE Et Je briserai toutes les cornes des pécheurs, et les cornes du juste se redresseront.

}
%%%%%%% PSAUME 75 %%%%%%%
\newcommand{\psalmlxxvfr}{
\VERSE Pour la fin, parmi les louanges, Psaume d'Asaph, cantique sur les Assyriens. \EVERSE
\VERSE Dieu S'est fait connaître en Judée; Son Nom est grand dans Israël. \EVERSE
\VERSE Il a fixé Son séjour dans la ville de paix, et Sa demeure dans Sion. \EVERSE
\VERSE C'est là qu'Il a brisé toute la force des arcs, le bouclier, le glaive et la guerre. \EVERSE
\VERSE Vous projetez un merveilleux éclat du haut des montagnes éternelles;  \EVERSE
\VERSE tous ceux dont le coeur était rempli de folie ont été consternés. Ils ont dormi leur sommeil, et tous ces hommes de richesses n'ont rien trouvé dans leurs mains. \EVERSE
\VERSE A Votre menace, ô Dieu de Jacob, se sont endormis ceux qui étaient montés sur des chevaux. \EVERSE
\VERSE Vous êtes terrible, et qui pourra Vous résister au moment de Votre colère?  \EVERSE
\VERSE Du Ciel, Vous avez fait entendre la sentence; la terre a tremblé et s'est tue, \EVERSE
\VERSE lorsque Dieu S'est levé pour rendre justice, afin de sauver tous ceux qui sont doux sur la terre. \EVERSE
\VERSE Aussi la pensée de l'homme Vous louera, et le souvenir qui lui restera Vous fera fête. \EVERSE
\VERSE Faites des voeux, et acquittez-les au Seigneur votre Dieu, vous tous qui des alentours apportez des présents à ce Dieu terrible, \EVERSE
\VERSE qui ôte la vie aux princes, qui est terrible aux rois de la terre.

}
%%%%%%% PSAUME 76 %%%%%%%
\newcommand{\psalmlxxvifr}{
\VERSE Pour la fin, à Idithun, Psaume d'Asaph. \EVERSE
\VERSE J'ai élevé ma voix, et j'ai crié vers le Seigneur; j'ai élevé ma voix vers Dieu, et Il m'a entendu. \EVERSE
\VERSE Au jour de ma tribulation, j'ai cherché Dieu; la nuit, j'ai tendu mes mains vers Lui, et je n'ai pas été déçu. Mon âme a refusé toute consolation;  \EVERSE
\VERSE je me suis souvenu de Dieu, et j'en ai été ravi; je me suis troublé, et mon esprit a défailli. \EVERSE
\VERSE Mes yeux ont devancé les veilles de la nuit; j'ai été dans le trouble, et je ne pouvais parler. \EVERSE
\VERSE Je pensais aux jours anciens, et j'avais dans l'esprit les années éternelles. \EVERSE
\VERSE Et je méditais la nuit dans mon coeur, et je réfléchissais, et je tourmentais mon esprit. \EVERSE
\VERSE Dieu nous rejettera-t-Il pour toujours? ou ne pourra-t-Il plus nous être favorable? \EVERSE
\VERSE Nous privera-t-Il à jamais de Sa miséricorde, de génération en génération? \EVERSE
\VERSE Dieu oubliera-t-Il d'avoir pitié? et, dans Sa colère, arrêtera-t-Il Ses miséricordes? \EVERSE
\VERSE Et j'ai dit: Maintenant je commence. Ce changement vient de la droite du Très-Haut. \EVERSE
\VERSE Je me suis souvenu des oeuvres du Seigneur; car je me souviendrai de Vos merveilles d'autrefois. \EVERSE
\VERSE Et je méditerai sur toutes Vos oeuvres, et je réfléchirai sur Vos desseins. \EVERSE
\VERSE O Dieu, Votre voie est sainte. Quel Dieu est grand comme notre Dieu?  \EVERSE
\VERSE Vous êtes le Dieu qui opérez des merveilles. Vous avez fait connaître parmi les peuples Votre puissance.  \EVERSE
\VERSE Vous avez racheté par Votre bras Votre peuple, les fils de Jacob et de Joseph. \EVERSE
\VERSE Les eaux Vous ont vu, ô Dieu; les eaux Vous ont vu, et elles ont eu peur, et les abîmes ont été troublés. \EVERSE
\VERSE Redoublement du fracas des eaux; les nuées ont fait retentir leur voix. Vos flèches aussi ont été lancées;  \EVERSE
\VERSE voix de Votre tonnerre tout autour. Vos éclairs ont illuminé le monde; la terre a été émue et a tremblé. \EVERSE
\VERSE La mer fut Votre chemin, les grandes eaux furent Vos sentiers, et Vos traces ne seront point connues. \EVERSE
\VERSE Vous avez conduit Votre peuple comme des brebis, par la main de Moïse et d'Aaron.

}
%%%%%%% PSAUME 77 %%%%%%%
\newcommand{\psalmlxxviifr}{
\VERSE Instruction d'Asaph. Mon peuple, écoutez Ma loi; prêtez l'oreille aux paroles de Ma bouche. \EVERSE
\VERSE Je vais ouvrir la bouche pour parler en paraboles; je dirai ce qui s'est fait dès le commencement; \EVERSE
\VERSE ce que nous avons entendu et appris, et ce que nos pères nous ont raconté. \EVERSE
\VERSE Ils ne l'ont point caché à leurs enfants, ni à leur postérité. Ils ont publié les louanges du Seigneur, les actes de Sa puissance, et les merveilles qu'Il a accomplies. \EVERSE
\VERSE Il a fait une ordonnance dans Jacob, et établi une loi dans Israël; c'est ce qu'Il a commandé à nos pères de faire connaître à leurs enfants,  \EVERSE
\VERSE afin que la génération suivante l'apprît; les enfants qui naîtront, et s'élèveront après eux, le raconteront aussi à leurs enfants, \EVERSE
\VERSE pour qu'ils mettent en Dieu leur espérance, qu'ils n'oublient pas les oeuvres de Dieu, et qu'ils recherchent Ses commandements; \EVERSE
\VERSE de peur qu'ils ne deviennent, comme leurs pères, une race mauvaise et exaspérante; une race qui n'a pas gardé son coeur droit, et dont l'esprit n'est pas resté fidèle à Dieu. \EVERSE
\VERSE Les fils d'Ephraïm, habiles à tendre l'arc et à en tirer, ont tourné le dos au jour du combat. \EVERSE
\VERSE Ils n'ont point gardé l'alliance faite avec Dieu, et n'ont pas voulu marcher dans Sa loi. \EVERSE
\VERSE Ils ont oublié Ses bienfaits, et les merveilles qu'Il leur avait manifestées. \EVERSE
\VERSE Devant leurs pères Il a fait des merveilles dans la terre d'Egypte, dans la plaine de Tanis. \EVERSE
\VERSE Il divisa la mer et les fit passer, et Il tint les eaux immobiles comme dans une outre. \EVERSE
\VERSE Il les conduisit le jour avec la nuée, et toute la nuit avec un feu brillant. \EVERSE
\VERSE Il fendit le rocher dans le désert, et Il les abreuva, comme s'il y avait eu là des abîmes d'eaux. \EVERSE
\VERSE Il fit sortir l'eau du rocher, et la fit couler comme des fleuves. \EVERSE
\VERSE Et ils continuèrent de pécher encore contre Lui, et ils excitèrent la colère du Très-Haut dans ce lieu aride. \EVERSE
\VERSE Et ils tentèrent Dieu dans leurs coeurs, en Lui demandant des viandes selon leur convoitise. \EVERSE
\VERSE Et ils parlèrent mal de Dieu, et ils dirent: Dieu pourra-t-Il bien préparer une table dans le désert? \EVERSE
\VERSE Sans doute Il a frappé la pierre, et les eaux ont coulé, et des torrents ont inondé la terre. Pourra-t-Il aussi donner du pain, ou préparer une table à Son peuple? \EVERSE
\VERSE Lorsque le Seigneur eut entendu, Il attendit; et un feu s'alluma contre Jacob, et la colère monta contre Israël; \EVERSE
\VERSE parce qu'ils n'avaient pas eu foi en Dieu, et qu'ils n'avaient pas espéré en Son secours. \EVERSE
\VERSE Et Il commanda aux nuées d'en haut, et Il ouvrit les portes du ciel. \EVERSE
\VERSE Et Il fit pleuvoir sur eux la manne pour les nourrir, et Il leur donna un pain du ciel. \EVERSE
\VERSE L'homme mangea le pain des anges; Il leur envoya des vivres en abondance. \EVERSE
\VERSE Il fit tourner dans le ciel le vent du midi, et Il envoya par Sa puissance le vent d'Afrique. \EVERSE
\VERSE Et Il fit pleuvoir sur eux des viandes comme la poussière, et les oiseaux ailés comme le sable de la mer. \EVERSE
\VERSE Ils tombèrent au milieu de leur camp, autour de leurs tentes. \EVERSE
\VERSE Et ils mangèrent, et furent rassasiés à l'excès, et Il leur accorda ce qu'ils désiraient:  \EVERSE
\VERSE ils ne furent point frustrés de leur désir. Les viandes étaient encore dans leur bouche,  \EVERSE
\VERSE lorsque la colère de Dieu s'éleva contre eux. Et Il tua les plus robustes d'entre eux, et Il fit tomber l'élite d'Israël. \EVERSE
\VERSE Après tout cela ils péchèrent encore, et ils n'eurent pas foi en Ses merveilles. \EVERSE
\VERSE Alors leurs jours passèrent comme un souffle, et leurs années précipitèrent leur cours. \EVERSE
\VERSE Lorsqu'Il les faisait mourir, ils Le cherchaient, et ils se retournaient, et ils se hâtaient de revenir à Lui. \EVERSE
\VERSE Ils se souvenaient que Dieu était leur défenseur, et que le Dieu très haut était leur sauveur. \EVERSE
\VERSE Mais ils ne L'aimaient que de bouche, et de leur langue ils Lui mentaient. \EVERSE
\VERSE Car leur coeur n'était pas droit avec Lui, et ils ne furent pas fidèles à Son alliance. \EVERSE
\VERSE Mais Il est miséricordieux; Il pardonnait leurs péchés, et ne les anéantissait pas. Et très souvent Il détourna Son courroux, et n'alluma point toute Sa colère. \EVERSE
\VERSE Il Se souvint qu'ils n'étaient que chair, un souffle qui passe et ne revient plus. \EVERSE
\VERSE Combien de fois ils L'irritèrent dans le désert, et excitèrent Son courroux dans la plaine aride! \EVERSE
\VERSE Et ils recommençaient à tenter Dieu, et à irriter le Saint d'Israël. \EVERSE
\VERSE Ils ne se souvinrent point de ce que Sa main avait fait au jour où Il les délivra des mains de l'oppresseur, \EVERSE
\VERSE lorsqu'Il fit éclater Ses signes en Egypte, et Ses prodiges dans la plaine de Tanis. \EVERSE
\VERSE Il changea en sang leurs fleuves et leurs eaux, afin qu'ils n'en pussent boire. \EVERSE
\VERSE Il envoya contre eux des mouches qui les dévorèrent, et des grenouilles qui les détruisirent. \EVERSE
\VERSE Il livra leurs récoltes à la rouille, et leurs travaux aux sauterelles. \EVERSE
\VERSE Il fit périr leurs vignes par la grêle, et leurs mûriers par le givre. \EVERSE
\VERSE Il livra leurs bétail à la grêle, et leurs possessions au feu. \EVERSE
\VERSE Il lança contre eux la fureur de Sa colère, l'indignation, et le courroux, et les tribulations, les fléaux envoyés par des anges de malheur. \EVERSE
\VERSE Il ouvrit un large chemin à Sa colère; Il n'épargna pas leur vie, et Il enveloppa leurs troupeaux dans une mort commune. \EVERSE
\VERSE Il frappa tous les premiers-nés dans la terre d'Egypte, et les prémices de toute leur peine dans les tentes de Cham. \EVERSE
\VERSE Et Il enleva Son peuple comme des brebis, et Il les conduisit comme un troupeau dans le désert, \EVERSE
\VERSE et Il les mena pleins d'espérance et leur ôta toute crainte, et la mer engloutit leurs ennemis. \EVERSE
\VERSE Et Il les amena sur la montagne de Sa sainteté, sur la montagne que Sa droite avait acquise. Et Il chassa les nations devant eux, et Il leur distribua au sort la terre promise, après l'avoir partagée avec le cordeau; \EVERSE
\VERSE et Il fit habiter dans leurs tentes les tribus d'Israël. \EVERSE
\VERSE Mais ils tentèrent et irritèrent le Dieu très haut, et ils ne gardèrent point Ses préceptes. \EVERSE
\VERSE Ils se détournèrent, et n'observèrent point l'alliance; comme leurs pères, ils devinrent un arc mauvais. \EVERSE
\VERSE Ils irritèrent Sa colère sur leurs collines, et ils provoquèrent Sa jalousie par leurs idoles. \EVERSE
\VERSE Dieu entendit, et Il méprisa Israël, et Il le réduisit à la dernière humiliation. \EVERSE
\VERSE Et Il rejeta le tabernacle de Silo, Son tabernacle où Il avait habité parmi les hommes. \EVERSE
\VERSE Et Il livra leur force à la captivité, et leur gloire aux mains de l'ennemi. \EVERSE
\VERSE Et Il livra Son peuple au glaive, et Il méprisa Son héritage. \EVERSE
\VERSE Le feu dévora leurs jeunes hommes, et leurs vierges ne furent point pleurées. \EVERSE
\VERSE Leurs prêtres tombèrent par le glaive, et on ne versa pas de larmes sur leurs veuves. \EVERSE
\VERSE Et le Seigneur Se réveilla comme un homme endormi, et comme un héros surexcité par le vin. \EVERSE
\VERSE Il frappa Ses ennemis par derrière, et les couvrit d'une honte éternelle. \EVERSE
\VERSE Et il rejeta le tabernacle de Joseph, et ne choisit point la tribu d'Ephraïm. \EVERSE
\VERSE Mais Il choisit la tribu de Juda, la montagne de Sion qu'Il a aimée. \EVERSE
\VERSE Et Il bâtit Son sanctuaire pareil à la licorne, dans la terre qu'Il a affermie pour toujours. \EVERSE
\VERSE Il a choisi David Son serviteur, et l'a tiré du milieu des troupeaux de brebis; Il l'a pris de derrière les brebis mères, \EVERSE
\VERSE pour qu'il fût le pasteur de Son serviteur Jacob, et d'Israël Son héritage. \EVERSE
\VERSE Et Il les fit paître dans l'innocence de son coeur, et les conduisit avec des mains intelligentes.

}
%%%%%%% PSAUME 78 %%%%%%%
\newcommand{\psalmlxxviiifr}{
\VERSE Psaume d'Asaph. O Dieu, les nations sont venues dans Votre héritage; elles ont souillé Votre saint temple; elles ont fait de Jérusalem une cabane à garder les fruits. \EVERSE
\VERSE Elles ont exposé les cadavres de Vos serviteurs en pâture aux oiseaux du ciel, les chairs de Vos saints aux bêtes de la terre. \EVERSE
\VERSE Elles ont répandu leur sang comme l'eau autour de Jérusalem, et il n'y avait personne pour les ensevelir. \EVERSE
\VERSE Nous sommes devenus un sujet d'opprobre pour nos voisins, la risée et la moquerie de ceux qui nous environnent. \EVERSE
\VERSE Jusques à quand, Seigneur, serez-Vous irrité pour toujours? jusques à quand Votre fureur s'allumera-t-elle comme un feu? \EVERSE
\VERSE Répandez Votre colère sur les nations qui ne Vous connaissent pas, et sur les royaumes qui n'invoquent point Votre Nom; \EVERSE
\VERSE car ils ont dévoré Jacob, et désolé sa demeure. \EVERSE
\VERSE Ne Vous souvenez plus de nos anciennes iniquités; que Vos miséricordes viennent en hâte au-devant de nous, car nous sommes réduits à la dernière misère. \EVERSE
\VERSE Aidez-nous, ô Dieu, notre sauveur, et pour la gloire de Votre nom, Seigneur, délivrez-nous, et pardonnez-nous nos péchés, à cause de Votre Nom. \EVERSE
\VERSE De peur qu'on ne dise parmi les nations: Où est leur Dieu? Faites éclater parmi les nations, sous nos yeux, la vengeance pour le sang de Vos serviteurs qui a été répandu.  \EVERSE
\VERSE Que le gémissement des captifs pénétre jusqu'à Vous. Selon la puissance de Votre bras, gardez les enfants de ceux qu'on a fait mourir. \EVERSE
\VERSE Et faites retomber dans le sein de nos voisins sept fois l'opprobre qu'ils Vous ont fait, Seigneur. \EVERSE
\VERSE Mais nous, Votre peuple et les brebis de Votre pâturage, nous Vous louerons à jamais; nous publierons Vos louanges de génération en génération.

}
%%%%%%% PSAUME 79 %%%%%%%
\newcommand{\psalmlxxixfr}{
\VERSE Pour la fin, pour ceux qui seront changés, témoignage d'Asaph, Psaume. \EVERSE
\VERSE Vous qui conduisez Israël, prêtez l'oreille; Vous qui menez Joseph comme une brebis. Vous qui êtes assis sur les chérubins, manifestez-Vous \EVERSE
\VERSE devant Ephraïm, Benjamin et Manassé. Excitez Votre puissance, et venez pour nous sauver. \EVERSE
\VERSE O Dieu, rétablissez-nous; montrez Votre visage, et nous serons sauvés. \EVERSE
\VERSE Seigneur, Dieu des armées, jusques à quand serez-Vous irrité contre la prière de Votre serviteur? \EVERSE
\VERSE Jusques à quand nous nourrirez-Vous d'un pain de larmes, et nous abreuverez-Vous de pleurs à pleine mesure? \EVERSE
\VERSE Vous avez fait de nous un sujet de dispute pour nos voisins, et nos ennemis se sont moqués de nous. \EVERSE
\VERSE Dieu des armées, rétablissez-nous; montrez-nous Votre visage, et nous serons sauvés. \EVERSE
\VERSE Vous avez transporté Votre vigne de l'Egypte; Vous avez chassé les nations, et Vous l'avez plantée. \EVERSE
\VERSE Vous avez été un guide devant elle dans le chemin; Vous avez planté ses racines, et elle a rempli la terre. \EVERSE
\VERSE Son ombre a couvert les montagnes, et ses rameaux les cèdres de Dieu. \EVERSE
\VERSE Elle a étendu ses branches jusqu'à la mer, et ses rejetons jusqu'au fleuve. \EVERSE
\VERSE Pourquoi avez-Vous détruit sa clôture, de sorte que tous ceux qui passent dans le chemin la pillent? \EVERSE
\VERSE Le sanglier de la forêt l'a ravagée, et la bête sauvage l'a dévorée. \EVERSE
\VERSE Dieu des armées, retournez-Vous; regardez du haut du Ciel, et voyez, et visitez cette vigne, \EVERSE
\VERSE et protégez celle que Votre droite a plantée, et le fils de l'homme que Vous avez établi pour Vous. \EVERSE
\VERSE Elle a été brûlée par le feu, et arrachée; devant Votre visage menaçant l'on va périr. \EVERSE
\VERSE Etendez Votre main sur l'homme de Votre droite, et sur le fils de l'homme que Vous avez établi pour Vous. \EVERSE
\VERSE Et nous ne nous éloignerons plus de Vous; Vous nous rendrez la vie, et nous invoquerons Votre Nom. \EVERSE
\VERSE Seigneur, Dieu des armées, rétablissez-nous, et montrez-nous Votre visage, et nous serons sauvés.

}
%%%%%%% PSAUME 80 %%%%%%%
\newcommand{\psalmlxxxfr}{
\VERSE Pour la fin, pour les pressoirs, psaume de Asaph. \EVERSE
\VERSE Tressaillez d'allégresse en Dieu notre protecteur; chantez avec transport en l'honneur du Dieu de Jacob. \EVERSE
\VERSE Entonnez le cantique, et faites résonner le tambourin, le psaltérion harmonieux, avec la harpe. \EVERSE
\VERSE Sonnez de la trompette à la néoménie, au jour insigne de Votre solennité. \EVERSE
\VERSE Car c'est un précepte pour Israël, et une ordonnance du Dieu de Jacob. \EVERSE
\VERSE Il en fit un statut pour Joseph, lorsqu'il sortait de la terre d'Egypte;il entendit une langue qu'il ne connaissait pas. \EVERSE
\VERSE Il a déchargé ses épaules des fardeaux; ses mains portèrent la corbeille. \EVERSE
\VERSE Dans la tribulation tu M'as invoqué, et Je t'ai délivré. Je t'ai exaucé du sein de la tempête; Je t'ai éprouvé auprès des eaux de contradiction. \EVERSE
\VERSE Ecoute, Mon peuple, et Je t'avertirai. Israël, si tu M'écoutes, \EVERSE
\VERSE il n'y aura pas chez toi de dieu nouveau, et tu n'adoreras pas de dieu étranger. \EVERSE
\VERSE Car Je suis le Seigneur ton Dieu, qui t'ai fait sortir de la terre d'Egypte. Elargis ta bouche, et Je la remplirai. \EVERSE
\VERSE Mais Mon peuple n'a pas écouté Ma voix, et Israël ne M'a point obéi. \EVERSE
\VERSE Et Je les ai abandonnés aux désirs de leur coeur; ils marcheront au gré de leurs conseils. \EVERSE
\VERSE Si Mon peuple M'avait écouté, si Israël avait marché dans Mes voies, \EVERSE
\VERSE J'aurais pu facilement humilier leurs ennemis, et J'aurais appesanti Ma main sur leurs oppresseurs. \EVERSE
\VERSE Les ennemis du Seigneur lui ont menti, et le temps de leur misère durera sans fin. \EVERSE
\VERSE Et cependant Il les a nourris de la fleur du froment, et Il les a rassasiés du miel sorti du rocher.

}
%%%%%%% PSAUME 81 %%%%%%%
\newcommand{\psalmlxxxifr}{
\VERSE Psaume de Asaph. Dieu S'est tenu dans l'assemblée des dieux, et au milieu d'eux Il juge les dieux. \EVERSE
\VERSE Jusques à quand jugerez-vous injustement, et aurez-vous égard à la personne des pécheurs? \EVERSE
\VERSE Faites droit à l'indigent et à l'orphelin; rendez justice au petit et au pauvre. \EVERSE
\VERSE Arrachez le pauvre, et délivrez l'indigent des mains du pécheur. \EVERSE
\VERSE Ils n'ont ni savoir ni intelligence; ils marchent dans les ténèbres; tous les fondements de la terre seront ébranlés. \EVERSE
\VERSE J'ai dit: Vous êtes des dieux; vous êtes tous fils du Très-Haut. \EVERSE
\VERSE Cependant vous mourrez comme des hommes, et vous tomberez comme un prince quelconque. \EVERSE
\VERSE Levez-Vous, ô Dieu, jugez la terre; car Vous devez avoir toutes les nations pour héritage.

}
%%%%%%% PSAUME 82 %%%%%%%
\newcommand{\psalmlxxxiifr}{
\VERSE Cantique psaume d'Asaph. \EVERSE
\VERSE O Dieu, qui sera semblable à Vous? Ne Vous taisez pas, ô Dieu, et ne Vous reposez pas. \EVERSE
\VERSE Car voici que Vos ennemis font un grand bruit, et ceux qui Vous haïssent ont levé la tête. \EVERSE
\VERSE Ils ont formé un dessein plein de malice contre Votre peuple, et ils ont conspiré contre Vos saints. \EVERSE
\VERSE Ils ont dit: Venez et exterminons-les du milieu des nations, et qu'on ne se souvienne plus jamais du nom d'Israël. \EVERSE
\VERSE Ils ont comploté d'un même coeur, et ensemble ils ont fait alliance contre Vous: \EVERSE
\VERSE les tentes des Iduméens et les Ismaélites; Moab et les Agaréniens; \EVERSE
\VERSE Gébal, et Ammon, et Amalec; les étrangers avec les habitants de Tyr. \EVERSE
\VERSE Assur aussi est venu avec eux, et s'est fait l'auxiliaire des fils de Lot. \EVERSE
\VERSE Traitez-les comme Madian et Sisara, comme Jabin au torrent de Cisson. \EVERSE
\VERSE Ils ont été détruits à Endor, ils sont devenus comme le fumier de la terre. \EVERSE
\VERSE Traitez leurs princes comme Oreb, et Zeb, et Zébée, et Salama; tous leurs princes  \EVERSE
\VERSE qui avaient dit: Emparons-nous du sanctuaire de Dieu comme de notre héritage. \EVERSE
\VERSE Mon Dieu, rendez-les semblables à une roue, et à la paille emportée par le vent. \EVERSE
\VERSE Comme le feu qui brûle la forêt, et comme la flamme qui consume les montagnes, \EVERSE
\VERSE ainsi Vous les poursuivrez par Votre tempête, et Vous les épouvanterez dans Votre colère. \EVERSE
\VERSE Couvrez leurs visages de confusion, et ils chercheront Votre Nom, Seigneur. \EVERSE
\VERSE Qu'ils rougissent et soient dans le trouble à jamais; qu'ils soient confondus et qu'ils périssent. \EVERSE
\VERSE Et qu'ils connaissent que Votre Nom est le Seigneur, et que Vous êtes seul le Très-Haut dans toute la terre.

}
%%%%%%% PSAUME 83 %%%%%%%
\newcommand{\psalmlxxxiiifr}{
\VERSE Pour la fin, pour les pressoirs, Psaume des fils de Coré. \EVERSE
\VERSE Que Vos tabernacles sont aimables, Seigneur des armées!  \EVERSE
\VERSE Mon âme soupire et languit après les parvis du Seigneur. Mon coeur et ma chair tressaillent d'amour pour le Dieu vivant. \EVERSE
\VERSE Car le passereau se trouve une maison, et la tourterelle un nid pour y placer ses petits. Vos autels, Seigneur des armées, mon Roi et mon Dieu! \EVERSE
\VERSE Heureux ceux qui habitent dans Votre maison, Seigneur; ils Vous loueront dans les siècles des siècles. \EVERSE
\VERSE Heureux l'homme qui attend de Vous son secours; en son coeur il a disposé des ascensions, \EVERSE
\VERSE dans la vallée des larmes, jusqu'au lieu qu'il a déterminé. \EVERSE
\VERSE Car le divin législateur donnera Sa bénédiction; ils iront de vertu en vertu, et ils verront le Dieu des dieux dans Sion. \EVERSE
\VERSE Seigneur, Dieu des armées, exaucez ma prière; prêtez l'oreille, ô Dieu de Jacob. \EVERSE
\VERSE Vous qui êtes notre protecteur, regardez, ô Dieu, et jetez les yeux sur le visage de Votre christ. \EVERSE
\VERSE Car un seul jour passé dans Vos tabernacles vaut mieux que mille. J'ai choisi d'être des derniers dans la maison de mon Dieu, plutôt que d'habiter dans les tentes des pécheurs. \EVERSE
\VERSE Car Dieu aime la miséricorde et la vérité; le Seigneur donnera la grâce et la gloire. \EVERSE
\VERSE Il ne privera pas de Ses biens ceux qui marchent dans l'innocence. Seigneur des armées, heureux l'homme qui espère en Vous.

}
%%%%%%% PSAUME 84 %%%%%%%
\newcommand{\psalmlxxxivfr}{
\VERSE Pour la fin, Psaume des fils de Coré. \EVERSE
\VERSE Vous avez béni, Seigneur, Votre terre; Vous avez délivré Jacob de la captivité. \EVERSE
\VERSE Vous avez remis l'iniquité de Votre peuple; Vous avez couvert tous leurs péchés. \EVERSE
\VERSE Vous avez adouci toute Votre colère, Vous êtes revenu de l'ardeur de Votre indignation. \EVERSE
\VERSE Rétablissez-nous, ô Dieu, notre sauveur, et détournez de nous Votre colère. \EVERSE
\VERSE Serez-Vous éternellement irrité contre nous? ou étendrez-Vous Votre colère de génération en génération? \EVERSE
\VERSE O Dieu, Vous nous donnerez de nouveau la vie, et Votre peuple se réjouira en Vous. \EVERSE
\VERSE Montrez-nous, Seigneur, Votre miséricorde, et accordez-nous Votre salut. \EVERSE
\VERSE J'écouterai ce que dira au dedans de moi le Seigneur Dieu; car Il annoncera la paix pour Son peuple. et pour Ses saints, et pour ceux qui rentrent au fond de leur coeur. \EVERSE
\VERSE Oui, Son salut est près de ceux qui Le craignent, et la gloire habitera dans notre terre. \EVERSE
\VERSE La miséricorde et la vérité se sont rencontrées; la justice et la paix se sont donné le baiser. \EVERSE
\VERSE La vérité a germé de la terre, et la justice a regardé du haut du Ciel. \EVERSE
\VERSE Car le Seigneur donnera Sa faveur, et notre terre donnera son fruit. \EVERSE
\VERSE La justice marchera devant Lui, et Il imprimera Ses pas sur le chemin.

}
%%%%%%% PSAUME 85 %%%%%%%
\newcommand{\psalmlxxxvfr}{
\VERSE Prière de David: Penchez, Seigneur, Votre oreille, et exaucez-moi, car je suis indigent et pauvre. \EVERSE
\VERSE Gardez mon âme, car je suis saint; sauvez, mon Dieu, Votre serviteur qui espère en Vous. \EVERSE
\VERSE Ayez pitié de moi, Seigneur, car j'ai crié vers Vous tout le jour;  \EVERSE
\VERSE réjouissez l'âme de Votre serviteur, car j'ai élevé mon âme vers Vous, Seigneur. \EVERSE
\VERSE Car Vous êtes, Seigneur, suave et doux, et plein de miséricorde pour tous ceux qui Vous invoquent. \EVERSE
\VERSE Prêtez l'oreille, Seigneur, à ma prière, et soyez attentif à la voix de ma supplication. \EVERSE
\VERSE Au jour de ma tribulation j'ai crié vers Vous, parce que Vous m'avez exaucé. \EVERSE
\VERSE Seigneur, parmi les dieux nul ne Vous est semblable, et rien n'est comparable à Vos oeuvres. \EVERSE
\VERSE Toutes les nations que Vous avez créées viendront, et se prosterneront devant Vous, Seigneur, et elles rendront gloire à Votre Nom. \EVERSE
\VERSE Car Vous êtes grand, et Vous faites des prodiges; Vous seul êtes Dieu. \EVERSE
\VERSE Conduisez-moi, Seigneur, dans Votre voie, et faites que j'entre dans Votre vérité; que mon coeur mette sa joie à craindre Votre Nom. \EVERSE
\VERSE Je Vous louerai, Seigneur mon Dieu, de tout mon coeur, et je glorifierai éternellement Votre Nom; \EVERSE
\VERSE car Votre miséricorde est grande envers moi, et Vous avez retiré mon âme de l'enfer le plus profond. \EVERSE
\VERSE O Dieu, les méchants se sont élevés contre moi, et une troupe d'hommes puissants en a voulu à ma vie, sans qu'ils Vous aient eu présent devant leurs yeux. \EVERSE
\VERSE Mais Vous, Seigneur Dieu, Vous êtes compatissant et clément, patient, plein de miséricorde, et fidèle. \EVERSE
\VERSE Regardez-moi, et ayez pitié de moi; donnez Votre force à Votre serviteur, et sauvez le fils de Votre servante. \EVERSE
\VERSE Opérez un signe en ma faveur, afin que ceux qui me haïssent le voient et soient confondus; car c'est Vous, Seigneur, qui m'avez aidé et consolé.

}
%%%%%%% PSAUME 86 %%%%%%%
\newcommand{\psalmlxxxvifr}{
\VERSE Des fils de Coré, Psaume cantique. Ses fondements sont sur les saintes montagnes.  \EVERSE
\VERSE Le Seigneur aime les portes de Sion plus que toutes les tentes de Jacob. \EVERSE
\VERSE On a dit de toi des choses glorieuses, ô cité de Dieu. \EVERSE
\VERSE Je me souviendrai de Rahab et de Babylone, qui Me connaissent. Voici que les étrangers, et Tyr, et le peuple d'Ethiopie sont là, eux aussi. \EVERSE
\VERSE Ne dira-t-on pas à Sion: Un grand nombre d'hommes sont nés en elle, et le Très-Haut Lui-même l'a fondée? \EVERSE
\VERSE Le Seigneur notera dans la description des peuples et des princes ceux qui auront été en elle. \EVERSE
\VERSE Ils sont tous dans la joie, ceux qui habitent en toi.

}
%%%%%%% PSAUME 87 %%%%%%%
\newcommand{\psalmlxxxviifr}{
\VERSE Cantique psaume des fils de Coré, Pour la fin, sur Mahéleth, pour répondre, instruction d'Eman l'Ezrahite. \EVERSE
\VERSE Seigneur, Dieu de mon salut, devant Vous, la nuit, j'ai crié. \EVERSE
\VERSE Que ma prière pénètre jusqu'à Vous; prêtez l'oreille à ma supplication. \EVERSE
\VERSE Car mon âme est remplie de maux, et ma vie s'approche du séjour des morts. \EVERSE
\VERSE On me compte parmi ceux qui descendent dans la fosse; je suis devenu comme un homme dénué de tout secours, \EVERSE
\VERSE abandonné parmi les morts; comme les blessés qui dorment dans les sépulcres, dont Vous ne Vous souvenez plus, et qui ont été repoussés de Votre main. \EVERSE
\VERSE Ils m'ont mis dans une fosse profonde, dans des lieux ténébreux et à l'ombre de la mort. \EVERSE
\VERSE Votre fureur s'est appesantie sur moi, et Vous avez fait passer sur moi tous Vos flots. \EVERSE
\VERSE Vous avez éloigné de moi ceux qui me connaissaient; ils ont fait de moi l'objet de leur abomination. J'ai été livré, et sans pouvoir sortir;  \EVERSE
\VERSE mes yeux se sont affaiblis par l'affliction. J'ai crié vers Vous, Seigneur, tout le jour; j'ai étendu vers Vous mes mains. \EVERSE
\VERSE Ferez-Vous des miracles pour les morts? ou les médecins les ressusciteront-ils, afin qu'ils Vous louent? \EVERSE
\VERSE Quelqu'un racontera-t-il dans le sépulcre Votre miséricorde, et Votre vérité dans le tombeau? \EVERSE
\VERSE Vos merveilles seront-elles connues dans les ténèbres, et Votre justice dans la terre de l'oubli? \EVERSE
\VERSE Et moi, Seigneur, je crie vers Vous, et le matin ma prière va au-devant de Vous. \EVERSE
\VERSE Pourquoi, Seigneur, rejetez-Vous ma prière, et détournez-Vous de moi Votre visage? \EVERSE
\VERSE Je suis pauvre et dans les travaux depuis ma jeunesse; et, après avoir été exalté, j'ai été humilié et troublé. \EVERSE
\VERSE Votre colère a passé sur moi, et Vos terreurs m'ont épouvanté. \EVERSE
\VERSE Elles m'ont environné comme l'eau tout le jour; elles m'ont environné toutes ensemble. \EVERSE
\VERSE Vous avez éloigné de moi mes amis et mes proches, et ceux qui me connaissaient, à cause de ma misère.

}
%%%%%%% PSAUME 88 %%%%%%%
\newcommand{\psalmlxxxviiifr}{
\VERSE Instruction d'Ethan l'Ezrahite. \EVERSE
\VERSE Je chanterai éternellement les miséricordes du Seigneur; de génération en génération ma bouche annoncera Votre vérité. \EVERSE
\VERSE Car Vous avez dit: La miséricorde s'élèvera comme un édifice éternel dans les Cieux; Votre vérité y sera solidement établie. \EVERSE
\VERSE J'ai contracté une alliance avec Mes élus; J'ai fait ce serment à David, Mon serviteur:  \EVERSE
\VERSE Je conserverai éternellement ta race, et J'affermirai ton trône pour toute les générations. \EVERSE
\VERSE Les Cieux publieront Vos merveilles, Seigneur, et Votre vérité dans l'assemblée des saints. \EVERSE
\VERSE Car qui, dans les Cieux, sera égal au Seigneur? et qui sera semblable à Dieu parmi les fils de Dieu? \EVERSE
\VERSE Dieu, qui est glorifié dans l'assemblée des saints, est plus grand et plus redoutable que tous ceux qui L'environnent. \EVERSE
\VERSE Seigneur, Dieu des armées, qui est semblable à Vous? Vous êtes puissant, Seigneur, et Votre vérité Vous environne. \EVERSE
\VERSE Vous dominez sur la puisssance de la mer, et Vous apaisez le mouvement de ses flots. \EVERSE
\VERSE Vous avez humilié l'orgueilleux, comme un blessé; Vous avez, par la force de Votre bras, dispersé Vos ennemis. \EVERSE
\VERSE A Vous sont les cieux, et à Vous la terre; c'est Vous qui avez fondé l'univers et tout ce qu'il contient;  \EVERSE
\VERSE Vous avez créé l'aquilon et la mer. Le Thabor et l'Hermon tressaillent d'allégresse à Votre Nom;  \EVERSE
\VERSE Votre bras est armé de puissance. Que Votre main s'affermisse, et que Votre droite s'élève.  \EVERSE
\VERSE La justice et l'équité sont l'appui de Votre trône. La miséricorde et la vérité marcheront devant Votre face.  \EVERSE
\VERSE Heureux le peuple qui connaît les acclamations joyeuses. Seigneur, ils marcheront à la lumière de Votre visage; \EVERSE
\VERSE ils se réjouiront tout le jour en Votre Nom, et ils seront élevés par Votre justice. \EVERSE
\VERSE Car Vous êtes la gloire de leur force, et c'est sur Votre bonté que s'élèvera notre puissance. \EVERSE
\VERSE Car c'est le Seigneur qui nous soutient; c'est le Saint d'Israël, notre Roi. \EVERSE
\VERSE Alors Vous avez parlé dans une vision à Vos saints, et Vous avez dit: J'ai prêté Mon secours à un homme puissant, et J'ai élevé celui que J'ai choisi du milieu de Mon peuple. \EVERSE
\VERSE J'ai trouvé David, Mon serviteur; Je l'ai oint de Mon huile sainte. \EVERSE
\VERSE Car Ma main l'assistera, et Mon bras le fortifiera. \EVERSE
\VERSE L'ennemi n'aura jamais l'avantage sur lui, et le fils d'iniquité ne pourra lui nuire. \EVERSE
\VERSE Et Je taillerai ses ennemis en pièces devant lui, et Je mettrai en fuite ceux qui le haïssent. \EVERSE
\VERSE Ma vérité et Ma miséricorde seront avec lui, et par Mon Nom s'élèvera sa puissance. \EVERSE
\VERSE Et j'étendrai sa main sur la mer, et sa droite sur les fleuves. \EVERSE
\VERSE Il m'invoquera: Vous êtes mon Père, mon Dieu, et l'auteur de mon salut. \EVERSE
\VERSE Et Moi, Je ferai de lui le premier-né, le plus élevé des rois de la terre. \EVERSE
\VERSE Je lui conserverai éternellement Ma miséricorde, et Mon alliance avec lui sera inviolable. \EVERSE
\VERSE Et Je ferai subsister sa race durant tous les siècles, et son trône autant que les cieux. \EVERSE
\VERSE Que si ses enfants abandonnent Ma loi, et s'ils ne marchent point dans Mes préceptes; \EVERSE
\VERSE s'ils violent Mes ordonnances, et ne gardent point Mes commandements: \EVERSE
\VERSE Je visiterai avec la verge leurs iniquités, et leurs péchés par des coups; \EVERSE
\VERSE mais Je ne lui retirerai pas Ma miséricorde, et Je ne trahirai pas Ma vérité. \EVERSE
\VERSE Et Je ne violerai pas Mon alliance, et Je ne rendrai pas vaines les paroles sorties de Mes lèvres. \EVERSE
\VERSE Je l'ai une fois juré par Ma sainteté, et Je ne mentirai point à David:  \EVERSE
\VERSE Sa race demeurera éternellement. \EVERSE
\VERSE Et son trône sera comme le soleil en Ma présence, et comme la lune qui subsistera à jamais, et le Témoin qui est au Ciel est fidèle. \EVERSE
\VERSE Et pourtant Vous avez rejeté et méprisé; Vous avez repoussé Votre oint. \EVERSE
\VERSE Vous avez détruit l'alliance faite avec Votre serviteur; Vous avez profané en le jetant à terre son diadème sacré. \EVERSE
\VERSE Vous avez abattu toutes ses clôtures; Vous avez rempli de frayeur ses forteresses. \EVERSE
\VERSE Tous ceux qui passaient par le chemin l'ont pillé, et il est devenu l'opprobre de ses voisins. \EVERSE
\VERSE Vous avez élevé la droite de ses oppresseurs; Vous avez réjoui tous ses ennemis. \EVERSE
\VERSE Vous avez enlevé toute force à son glaive, et Vous ne l'avez pas secouru dans la guerre. \EVERSE
\VERSE Vous l'avez dépouillé de son éclat, et Vous avez brisé son trône contre la terre. \EVERSE
\VERSE Vous avez abrégé les jours de son règne; Vous l'avez couvert d'ignominie. \EVERSE
\VERSE Jusques à quand, Seigneur, Vous détournerez-Vous à jamais? Jusques à quand Votre colère s'embrasera-t-elle comme le feu? \EVERSE
\VERSE Rappelez-Vous ce qu'est ma vie; car est-ce pour le néant que Vous avez créé tous les enfants des hommes? \EVERSE
\VERSE Quel est l'homme qui pourra vivre sans voir la mort, et qui arrachera son âme à la puissance de l'enfer? \EVERSE
\VERSE Où sont, Seigneur, Vos anciennes miséricordes, que Vous avez jurées à David au nom de Votre vérité? \EVERSE
\VERSE Souvenez-Vous, Seigneur, de l'opprobre de Vos serviteurs; je l'ai tenu caché dans mon sein; il venait de nations nombreuses. \EVERSE
\VERSE Souvenez-Vous au reproche de Vos ennemis, Seigneur, du reproche qu'ils ont fait au sujet de Votre changement à l'égard de Votre oint. \EVERSE
\VERSE Béni soit le Seigneur à jamais. Ainsi soit-il, ainsi soit-il.

}
%%%%%%% PSAUME 89 %%%%%%%
\newcommand{\psalmlxxxixfr}{
\VERSE Prière de Moïse, homme de Dieu. Seigneur, Vous avez été pour nous un refuge, de génération en génération. \EVERSE
\VERSE Avant que les montagnes eussent été faites, ou que la terre et le monde eussent été formés, Vous êtes Dieu de toute éternité, et dans tous les siècles. \EVERSE
\VERSE Ne réduisez pas l'homme à l'abaissement, Vous qui avez dit: Revenez, enfants des hommes. \EVERSE
\VERSE Car mille ans sont à Vos yeux comme le jour d'hier qui n'est plus, et comme une veille de la nuit; \EVERSE
\VERSE on les compte pour rien; tel est le cas que l'on fait de leurs années. \EVERSE
\VERSE Comme l'herbe, il passe en un matin; le matin elle fleurit, et elle passe; le soir elle tombe, se durcit et se dessèche. \EVERSE
\VERSE Car nous sommes consumés par Votre colère, et nous avons été troublés par Votre fureur. \EVERSE
\VERSE Vous avez mis nos iniquités en Votre présence, et notre vie à la lumière de Votre visage. \EVERSE
\VERSE C'est pourquoi tous nos jours se sont évanouis, et nous avons été consumés par Votre colère. Nos années se passent en de vains soucis, comme pour l'araignée.  \EVERSE
\VERSE Les jours de nos années sont en tout de soixante-dix ans; pour les plus forts, de quatre-vingts ans. Le surplus n'est que peine et que douleur; car alors survient la faiblesse, et nous sommes affligés. \EVERSE
\VERSE Qui connaît la puissance de Votre colère, et qui comprend combien Votre colère est redoutable? \EVERSE
\VERSE Apprenez-nous à reconnaître Votre droite, et instruisez notre coeur dans la sagesse. \EVERSE
\VERSE Revenz, Seigneur; jusques à quand nous rejetterez-Vous? Laissez-Vous fléchir en faveur de Vos serviteurs. \EVERSE
\VERSE Nous avons été comblés, dès le matin, de Votre miséricorde; nous avons tressailli d'allégresse et de bonheur tous les jours de notre vie. \EVERSE
\VERSE Nous nous sommes réjouis à proportion des jours où Vous nous avez humiliés, et des années où nous avons vu le malheur. \EVERSE
\VERSE Jetez un regard sur Vos serviteurs et sur Vos oeuvres, et guidez leurs enfants. \EVERSE
\VERSE Que la lumière du Seigneur notre Dieu brille sur nous; dirigez d'en haut les ouvrages de nos mains; oui, dirigez l'oeuvre de nos mains.

}
%%%%%%% PSAUME 90 %%%%%%%
\newcommand{\psalmxcfr}{
\VERSE Cantique de louange de David. Celui qui habite sous l'assistance du Très-Haut demeurera sous la protection du Dieu du Ciel. \EVERSE
\VERSE Il dira au Seigneur: Vous êtes mon défenseur et mon refuge. Il est mon Dieu; j'espérerai en Lui. \EVERSE
\VERSE Car c'est Lui qui m'a délivré du piège du chasseur, et de la parole âpre et piquante. \EVERSE
\VERSE Il te mettra à l'ombre sous Ses épaules, et sous Ses ailes tu seras plein d'espoir. \EVERSE
\VERSE Sa vérité t'environnera comme un bouclier; tu ne craindras pas les frayeurs de la nuit, \EVERSE
\VERSE ni la flèche qui vole pendant le jour, ni les maux qui s'avancent dans les ténèbres, ni les attaques du démon de midi. \EVERSE
\VERSE Mille tomberont à ton côté, et dix mille à ta droite; mais la mort n'approchera pas de toi. \EVERSE
\VERSE Et même tu contempleras de tes yeux, et tu verras le châtiment des pécheurs. \EVERSE
\VERSE Car tu as dit: Vous êtes, Seigneur, mon espérance. Tu as fait du Très-Haut ton refuge. \EVERSE
\VERSE Le mal ne viendra pas jusqu'à toi, et les fléaux ne s'approcheront pas de ta tente. \EVERSE
\VERSE Car Il a commandé pour toi à Ses Anges de te garder dans toutes tes voies. \EVERSE
\VERSE Ils te porteront dans leurs mains, de peur que tu heurtes le pied contre la pierre. \EVERSE
\VERSE Tu marcheras sur l'aspic et sur le basilic, et tu fouleras aux pieds le lion et le dragon. \EVERSE
\VERSE Parce qu'il a espéré en Moi, Je le délivrerai; Je le protégerai, parce qu'il a connu Mon Nom.  \EVERSE
\VERSE Il criera vers Moi, et Je l'exaucerai; Je suis avec lui dans la tribulation; Je le sauverai et Je le glorifierai. \EVERSE
\VERSE Je le comblerai de jours, et Je lui ferai voir Mon salut.

}
%%%%%%% PSAUME 91 %%%%%%%
\newcommand{\psalmxcifr}{
\VERSE Psaume cantique, pour le jour du sabbat. \EVERSE
\VERSE Il est bon de louer le Seigneur et de chanter Votre Nom, ô Très-Haut; \EVERSE
\VERSE pour annoncer le matin Votre miséricorde, et Votre vérité durant la nuit, \EVERSE
\VERSE sur l'instrument à dix cordes, joint au chant, et sur la harpe. \EVERSE
\VERSE Car Vous m'avez réjoui, Seigneur, par Vos oeuvres, et je tressaille d'allégresse au sujet des ouvrages de Vos mains. \EVERSE
\VERSE Que Vos oeuvres sont magnifiques, Seigneur! que Vos pensées sont profondes et impénétrables! \EVERSE
\VERSE L'homme stupide ne les connaîtra pas, et l'insensé ne les comprendra pas. \EVERSE
\VERSE Lorsque les pécheurs auront germé comme l'herbe, et que tous ceux qui commettent l'iniquité se seront manifestés, ce sera pour périr à jamais.  \EVERSE
\VERSE Mais Vous, Seigneur, Vous êtes éternellement le Très-Haut.  \EVERSE
\VERSE Car voici, Seigneur, que Vos ennemis, voici que Vos ennemis vont périr, et tous ceux qui commettent l'iniquité seront dispersés. \EVERSE
\VERSE Et ma corne s'élèvera comme celle de la licorne, et ma vieillesse se renouvellera par Votre abondante miséricorde. \EVERSE
\VERSE Et mon oeil a regardé mes ennemis avec mépris, et mon oreille entendra les cris d'angoisse des méchants qui s'élèvent contre moi. \EVERSE
\VERSE Le juste fleurira comme le palmier, et il se multipliera comme le cèdre du Liban. \EVERSE
\VERSE Plantés dans la maison du Seigneur, ils fleuriront dans les parvis de la maison de notre Dieu. \EVERSE
\VERSE Ils se multiplieront de nouveau dans une vieillesse comblée de biens, et ils seront remplis de vigueur, \EVERSE
\VERSE pour publier que le Seigneur notre Dieu est juste, et qu'il n'y a point d'iniquité en Lui.

}
%%%%%%% PSAUME 92 %%%%%%%
\newcommand{\psalmxciifr}{
\VERSE Cantique de louange, de David, pour le jour qui précède le sabbat, lorsque la terre fut entièrement créée. Le Seigneur a régné, et a été revêtu de gloire; le Seigneur a été revêtu et S'est ceint de force. \EVERSE
\VERSE Car Il a affermi le globe de la terre, qui ne sera point ébranlé. \EVERSE
\VERSE Votre trône, ô Dieu, est établi depuis longtemps; Vous êtes de toute éternité. Les fleuves, Seigneur, ont élevé, les fleuves ont élevé leur voix. Les fleuves ont élevé leurs flots,  \EVERSE
\VERSE plus retentisssants que la voix des grandes eaux. Les soulèvements de la mer sont admirables; plus admirable est le Seigneur dans les hauteurs des cieux. \EVERSE
\VERSE Vos témoignages sont tout à fait dignes de créance. La sainteté convient à Votre maison, Seigneur, dans toute la durée des jours.

}
%%%%%%% PSAUME 93 %%%%%%%
\newcommand{\psalmxciiifr}{
\VERSE Psaume de David, pour le quatrième jour après le sabbat: Le Seigneur est le Dieu des vengeances; le Dieu des vengeances a agi avec une entière liberté. \EVERSE
\VERSE Levez-Vous, ô Dieu, qui jugez la terre; rendez aux superbes ce qui leur est dû. \EVERSE
\VERSE Jusques à quand, Seigneur, les pécheurs, jusques à quand les pécheurs se glorifieront-ils? \EVERSE
\VERSE Jusques à quand tous ceux qui commettent des injustices se répandront-ils en des discours insolents, et proféreront-ils l'iniquité? \EVERSE
\VERSE Ils ont humilié Votre peuple, Seigneur; ils ont opprimé Votre héritage. \EVERSE
\VERSE Ils ont mis à mort la veuve et l'étranger, et ils ont tué les orphelins. \EVERSE
\VERSE Et ils ont dit: Le Seigneur ne le verra pas, et le Dieu de Jacob n'en saura rien. \EVERSE
\VERSE Comprenez, vous qui êtes stupides parmi le peuple; insensés, apprenez enfin la sagesse. \EVERSE
\VERSE Celui qui a planté l'oreille n'entendrait-Il pas? ou celui qui a formé l'oeil ne verrait-Il pas? \EVERSE
\VERSE Celui qui reprend les nations ne vous convaincra-t-Il pas de péché, Lui qui enseigne la science à l'homme? \EVERSE
\VERSE Le Seigneur connaît les pensées des hommes; Il sait qu'elles sont vaines. \EVERSE
\VERSE Heureux l'homme que Vous avez Vous-même instruit, Seigneur, et à qui Vous avez enseigné Votre loi, \EVERSE
\VERSE pour lui adoucir les jours mauvais, jusqu'à ce qu'on ait creusé une fosse pour le pécheur. \EVERSE
\VERSE Car le Seigneur ne rejettera pas Son peuple, et Il n'abandonnera pas Son héritage; \EVERSE
\VERSE jusqu'à ce que la justice fasse éclater son jugement, et que tous ceux qui ont le coeur droit se tiennent auprès d'elle. \EVERSE
\VERSE Qui se lèvera pour moi contre les méchants? ou qui se tiendra auprès de moi contre ceux qui commettent l'iniquité? \EVERSE
\VERSE Si Dieu ne m'eût assisté, il s'en serait peu fallu que mon âme n'habitât le séjour des morts. \EVERSE
\VERSE Si je disais: Mon pied a été ébranlé, Votre miséricorde, Seigneur, me soutenait. \EVERSE
\VERSE Selon la multitude des douleurs de mon coeur, Vos consolations ont rempli de joie mon âme. \EVERSE
\VERSE Le trône de l'iniquité Vous est-il attaché, à Vous qui rendez Vos conmmandements pénibles? \EVERSE
\VERSE Le méchants tendront des pièges à l'âme du juste, et condamneront le sang innocent. \EVERSE
\VERSE Mais le Seigneur S'est fait mon refuge, et mon Dieu l'appui de mon espérance. \EVERSE
\VERSE Et Il fera retomber sur eux leur iniquité, et Il les perdra par leur propre malice; le Seigneur notre Dieu les perdra.

}
%%%%%%% PSAUME 94 %%%%%%%
\newcommand{\psalmxcivfr}{
\VERSE Cantique de louange, de David: Venez, réjouissons-nous devant le Seigneur; poussons des cris de joie vers Dieu, notre Sauveur. \EVERSE
\VERSE Allons au-devant de Lui avec des louanges, et chantons des cantiques à Sa gloire. \EVERSE
\VERSE Car le Seigneur est le grand Dieu, et le grand Roi au-dessus de tous les dieux. \EVERSE
\VERSE Dans Sa main sont tous les confins de la terre, et les sommets des montagnes Lui appartiennent. \EVERSE
\VERSE A Lui est la mer, et c'est Lui qui l'a faite, et Ses mains ont formé le continent. \EVERSE
\VERSE Venez, adorons et prosternons-nous, et pleurons devant le Seigneur qui nous a faits; \EVERSE
\VERSE car Il est le Seigneur notre Dieu, et nous, nous sommes le peuple de Son pâturage, et les brebis de Sa main. \EVERSE
\VERSE Aujourd'hui, si vous entendez Sa voix, gardez-vous d'endurcir vos coeurs, \EVERSE
\VERSE comme lorsqu'ils excitèrent Ma colère, au jour de la tentation dans le désert, où vos pères M'ont tenté, M'ont mis à l'épreuve, et ont vu Mes oeuvres. \EVERSE
\VERSE Pendant quarante ans Je fus irrité contre cette génération; et Je dis: Leur coeur ne cesse de s'égarer. \EVERSE
\VERSE Et ils n'ont point connu Mes voies; de sorte que J'ai juré dans Ma colère: Ils n'entreront point dans Mon repos.

}
%%%%%%% PSAUME 95 %%%%%%%
\newcommand{\psalmxcvfr}{
\VERSE Cantique de David: Lorsqu'on bâtissait la maison après la captivité. Chantez au Seigneur un cantique nouveau; chantez au Seigneur, toute la terre. \EVERSE
\VERSE Chantez au Seigneur, et bénissez Son Nom, annoncez de jour en jour Son salut. \EVERSE
\VERSE Annoncez Sa gloire parmi les nations, Ses merveilles parmi tous les peuples. \EVERSE
\VERSE Car le Seigneur est grand et infiniment louable; Il est plus redoutable que tous les dieux. \EVERSE
\VERSE Car tous les dieux des nations sont des démons; mais le Seigneur a fait les cieux. \EVERSE
\VERSE La louange et la splendeur sont devant Lui; la sainteté et la magnificence dans Son sanctuaire. \EVERSE
\VERSE Offrez au Seigneur, familles des nations, offrez au Seigneur la gloire et l'honneur;  \EVERSE
\VERSE offrez au Seigneur la gloire due à Son Nom. Prenez des victimes et entrez dans Ses parvis;  \EVERSE
\VERSE adorez le Seigneur dans Son saint tabernacle. Que toute la terre tremble devant Sa face.  \EVERSE
\VERSE Dites parmi les nations que le Seigneur a établi Son règne. Car Il a affermi toute la terre, qui ne sera point ébranlée; Il jugera les peuples selon l'équité. \EVERSE
\VERSE Que les Cieux se réjouissent, et que la terre tressaille d'allégresse; que la mer s'agite avec ce qu'elle renferme. \EVERSE
\VERSE Les champs seront dans la joie avec tout ce qu'ils contiennent. Alors tous les arbres des forêts tressailliront \EVERSE
\VERSE en présence du Seigneur, car Il vient; Il vient pour juger la terre. Il jugera toute la terre avec équité et les peuples selon Sa vérité.

}
%%%%%%% PSAUME 96 %%%%%%%
\newcommand{\psalmxcvifr}{
\VERSE De David, quand sa terre lui fut rendue. Le Seigneur est Roi: que la terre tressaille de joie, que toutes les îles se réjouissent. \EVERSE
\VERSE La nuée et l'obscurité sont autour de Lui; la justice et l'équité sont le soutien de Son trône. \EVERSE
\VERSE Le feu marche devant Lui, et embrase autour de Lui Ses ennemis. \EVERSE
\VERSE Ses éclairs ont brillé sur le monde; la terre a vu, et a tremblé. \EVERSE
\VERSE Les montagnes se sont fondues comme la cire à la face du Seigneur; à la face du Seigneur, toute la terre. \EVERSE
\VERSE Les cieux ont proclamé Sa justice, et tous les peuples ont vu Sa gloire. \EVERSE
\VERSE Qu'ils soient confondus tous ceux qui adorent les images sculptées, et qui se glorifient dans leurs idoles. Adorez-Le, vous tous Ses Anges.  \EVERSE
\VERSE Sion a entendu et s'est réjouie, et les filles de Juda ont tressailli de joie, à cause de Vos jugements, Seigneur. \EVERSE
\VERSE Car Vous êtes le Seigneur Très-Haut sur toute la terre; Vous êtes infiniment élevé au-dessus de tous les dieux. \EVERSE
\VERSE Vous qui aimez le Seigneur, haïssez le mal; le Seigneur garde les âmes de Ses saints; Il les délivrera de la main du pécheur. \EVERSE
\VERSE La lumière s'est levée pour le juste, et la joie pour ceux qui ont le coeur droit. \EVERSE
\VERSE Rêjouissez-vous, justes, dans le Seigneur, et célébrez la mémoire de Sa sainteté.

}
%%%%%%% PSAUME 97 %%%%%%%
\newcommand{\psalmxcviifr}{
\VERSE Psaume de David. Chantez au Seigneur un cantique nouveau, car Il a opéré des merveilles. Sa droite et Son saint bras L'ont fait triompher. \EVERSE
\VERSE Le Seigneur a fait connaître Son salut; Il a révélé Sa justice aux yeux des nations. \EVERSE
\VERSE Il S'est souvenu de Sa miséricorde et de Sa fidélité envers la maison d'Israël. Tous les confins de la terre ont vu le salut de notre Dieu. \EVERSE
\VERSE Acclamez Dieu, terre entière; chantez, et tressaillez de joie, et jouez des instruments. \EVERSE
\VERSE Jouez sur la harpe au Seigneur; sur la harpe, et en chantant des hymnes; \EVERSE
\VERSE avec les trompettes de métal, et avec la trompette de corne. Poussez des cris de joie en présence du Seigneur votre Roi.  \EVERSE
\VERSE Que la mer se soulève avec ce qu'elle renferme; le globe de la terre, et ceux qui l'habitent. \EVERSE
\VERSE Les fleuves battront des mains; en même temps les montagnes tressailliront de joie \EVERSE
\VERSE à la présence du Seigneur, parce qu'Il vient juger la terre. Il jugera toute la terre avec justice, et les peuples avec équité.

}
%%%%%%% PSAUME 98 %%%%%%%
\newcommand{\psalmxcviiifr}{
\VERSE Psaume de David. Le Seigneur règne: que les peuples s'irritent. Il est assis sur les chérubins: que la terre soit ébranlée. \EVERSE
\VERSE Le Seigneur est grand dans Sion, et Il est élevé au-dessus de tous les peuples. \EVERSE
\VERSE Qu'on rende gloire à Votre grand Nom, car il est terrible et saint,  \EVERSE
\VERSE et l'honneur du roi est d'aimer la justice. Vous avez marqué les directions à suivre; Vous avez exercé la justice et le jugement dans Jacob. \EVERSE
\VERSE Exaltez le Seigneur notre Dieu, et adorez l'escabeau de Ses pieds, car il est saint. \EVERSE
\VERSE Moïse et Aaron étaient parmi Ses prêtres, et Samuel parmi ceux qui invoquent Son Nom. Ils invoquaient le Seigneur, et Il les exauçait;  \EVERSE
\VERSE Il leur parlait dans la colonne de nuée. Ils gardaient Ses ordonnances, et le précepte qu'Il leur avait donné. \EVERSE
\VERSE Seigneur notre Dieu, Vous les exauciez; ô Dieu, Vous leur avez été propice, et Vous punissiez toutes leurs fautes. \EVERSE
\VERSE Exaltez le Seigneur notre Dieu, et adorez-Le sur Sa montagne sainte, car le Seigneur notre Dieu est saint.

}
%%%%%%% PSAUME 99 %%%%%%%
\newcommand{\psalmxcixfr}{
\VERSE Psaume pour la louange. \EVERSE
\VERSE Acclamez Dieu, toute la terre; servez le Seigneur avec joie. Entrez en Sa présence avec allégresse. \EVERSE
\VERSE Sachez que c'est le Seigneur qui est Dieu; c'est Lui qui nous a faits, et non pas nous-mêmes. Nous sommes Son peuple, et les brebis de Son pâturage.  \EVERSE
\VERSE Franchissez Ses portes avec des louanges, Ses parvis en chantant des hymnes; célébrez-Le. Louez Son Nom, \EVERSE
\VERSE car le Seigneur est suave; Sa miséricorde est éternelle, et Sa vérité demeure de génération en génération.

}
%%%%%%% PSAUME 100 %%%%%%%
\newcommand{\psalmcfr}{
\VERSE Psaume de David lui-même. Je chanterai, Seigneur, devant Vous Votre miséricorde et Votre justice. Je les chanterai au son des instruments, \EVERSE
\VERSE et je m'appliquerai à connaître la voie sans tache. Quand viendrez-Vous à moi? Je marchais dans l'innocence de mon coeur, au milieu de ma maison. \EVERSE
\VERSE Je ne plaçais devant mes yeux rien d'injuste; je haïssais ceux qui commettaient la prévarication. J'éloignais de moi \EVERSE
\VERSE le coeur corrompu; le méchant s'écartait de moi, et je ne le connaissais pas. \EVERSE
\VERSE Celui qui médisait en secret de son prochain, je le poursuivais. Celui dont l'oeil est superbe et le coeur insatiable, je ne mangeais pas avec lui. \EVERSE
\VERSE Mes yeux se tournaient vers les hommes fidèles de la terre, pour les faire asseoir près de moi; celui qui marchait dans une voie innocente était mon serviteur. \EVERSE
\VERSE Celui qui agit avec orgueil n'habitera point dans ma maison. Celui qui profère des choses injustes n'a pu se rendre agréable à mes yeux. \EVERSE
\VERSE Je mettais à mort dès le matin tous les pécheurs de la terre, afin d'extirper de la ville du Seigneur tous ceux qui commettent l'iniquité.

}
%%%%%%% PSAUME 101 %%%%%%%
\newcommand{\psalmcifr}{
\VERSE Prière du pauvre, lorsqu'il sera dans l'affliction, et qu'il répandra sa supplication en présence du Seigneur. \EVERSE
\VERSE Seigneur, exaucez ma prière, et que mon cri aille jusqu'à Vous. \EVERSE
\VERSE Ne détournez pas de moi Votre visage; en quelque jour que je sois affligé, inclinez vers moi Votre oreille. En quelque jour que je Vous invoque, exaucez-moi promptement. \EVERSE
\VERSE Car mes jours se sont évanouis comme la fumée, et mes os se sont desséchés comme le bois du foyer. \EVERSE
\VERSE J'ai été frappé comme l'herbe, et mon coeur s'est desséché, parce que j'ai oublié de manger mon pain. \EVERSE
\VERSE A force de pousser des gémissements, mes os se sont attachés à ma peau. \EVERSE
\VERSE Je suis devenu semblable au pélican du désert; je suis devenu comme le hibou des maisons. \EVERSE
\VERSE J'ai veillé, et je suis devenu comme le passereau qui se tient seul sur le toit. \EVERSE
\VERSE Tout le jour mes ennemis me faisaient des reproches, et ceux qui me louaient conspiraient avec serment contre moi. \EVERSE
\VERSE Parce que je mangeais la cendre comme du pain, et que je mêlais mon breuvage avec mes larmes; \EVERSE
\VERSE à cause de Votre colère et de Votre indignation, car après m'avoir élevé Vous m'avez écrasé. \EVERSE
\VERSE Mes jours se sont évanouis comme l'ombre, et je me suis desséché comme l'herbe. \EVERSE
\VERSE Mais Vous, Seigneur, Vous subsistez éternellement, et la mémoire de Votre Nom s'étend de race en race. \EVERSE
\VERSE Vous Vous lèverez, et Vous aurez pitié de Sion, car il est temps d'avoir pitié d'elle, et le temps est venu. \EVERSE
\VERSE Car ses pierres sont aimées de Vos serviteurs, et sa terre les attendrit. \EVERSE
\VERSE Et les nations craindront Votre Nom, Seigneur, et tous les rois de la terre Votre gloire, \EVERSE
\VERSE parce que le Seigneur a bâti Sion, et qu'Il sera vu dans Sa gloire. \EVERSE
\VERSE Il a regardé la prière des humbles, et Il n'a point méprisé leur prière. \EVERSE
\VERSE Que ces choses soient écrites pour la génération future, et le peuple qui sera créé louera le Seigneur. \EVERSE
\VERSE parce qu'Il a regardé du haut de Son lieu saint. Le Seigneur a regardé du Ciel sur la terre, \EVERSE
\VERSE pour entendre les gémissements des captifs, pour délivrer les fils de ceux qui avaient été tués, \EVERSE
\VERSE afin qu'ils annoncent dans Sion le Nom du Seigneur, et Sa louange dans Jérusalem, \EVERSE
\VERSE lorsque les peuples et les rois s'assembleront pour servir conjointement le Seigneur. \EVERSE
\VERSE Il Lui dit dans sa force: Faites-moi connaître le petit nombre de mes jours. \EVERSE
\VERSE Ne me rappelez pas au milieu de mes jours; Vos années durent d'âge en âge. \EVERSE
\VERSE Dès le commencement, Seigneur, Vous avez fondé la terre, et les cieux sont l'oeuvre de Vos mains. \EVERSE
\VERSE Ils périront, mais Vous, Vous demeurez, et ils vieilliront tous comme un vêtement. Vous les changerez comme un manteau, et ils seront changés;  \EVERSE
\VERSE mais Vous, Vous êtes toujours le même, et Vos années ne passeront point. \EVERSE
\VERSE Les fils de Vos serviteurs auront une demeure permanente, et leur postérité sera stable à jamais.

}
%%%%%%% PSAUME 102 %%%%%%%
\newcommand{\psalmciifr}{
\VERSE De David lui-même. Mon âme, bénis le Seigneur, et que tout ce qui est au dedans de moi bénisse Son saint Nom. \EVERSE
\VERSE Mon âme, bénis le Seigneur, et n'oublie jamais tous Ses bienfaits. \EVERSE
\VERSE C'est Lui qui pardonne toutes tes iniquités, et qui guérit toutes tes maladies. \EVERSE
\VERSE C'est Lui qui rachète ta vie de la mort, qui te couronne de miséricorde et de grâces. \EVERSE
\VERSE C'est Lui qui remplit tes désirs en te comblant de biens; ta jeunesse sera renouvelée comme celle de l'aigle. \EVERSE
\VERSE Le Seigneur fait miséricorde, et Il rend justice à tous ceux qui souffrent la violence. \EVERSE
\VERSE Il a fait connaître Ses voies à Moïse, et Ses volontés aux enfants d'Israël. \EVERSE
\VERSE Le Seigneur est compatissant et miséricordieux, patient et très miséricordieux. \EVERSE
\VERSE Il ne S'irritera pas perpétuellement, et ne menacera pas sans fin. \EVERSE
\VERSE Il ne nous a pas traités selon nos péchés, et Il ne nous a pas punis selon nos iniquités. \EVERSE
\VERSE Car autant le Ciel est élevé au-dessus de la terre, autant Il a affermi Sa miséricorde sur ceux qui Le craignent. \EVERSE
\VERSE Autant l'orient est éloigné du couchant, autant Il a éloigné de nous nos iniquités. \EVERSE
\VERSE Comme un père a compassion de ses enfants, ainsi le Seigneur a compassion de ceux qui Le craignent.  \EVERSE
\VERSE Car Il sait de quoi nous sommes formés; Il S'est souvenu que nous ne sommes que poussière.  \EVERSE
\VERSE Les jours de l'homme passent comme l'herbe; il fleurit comme la fleur des champs. \EVERSE
\VERSE Qu'un souffle passe sur lui, et il n'est plus, et le lieu qu'il occupait ne le reconnaît plus. \EVERSE
\VERSE Mais la miséricorde du Seigneur s'étend de l'éternité à l'éternité sur ceux qui Le craignent. Et Sa justice se répand sur les enfants des enfants  \EVERSE
\VERSE de ceux qui gardent Son alliance, et qui se souviennent de Ses préceptes, pour les accomplir. \EVERSE
\VERSE Le Seigneur a préparé Son trône dans le Ciel, et tout sera assujetti à Son empire. \EVERSE
\VERSE Bénissez le Seigneur, vous tous, Ses Anges, qui êtes puissants et forts; qui exécutez Sa parole, pour obéir à la voix de Ses ordres. \EVERSE
\VERSE Bénissez le Seigneur, vous toutes, Ses armées; vous, Ses ministres, qui faites Sa volonté. \EVERSE
\VERSE Bénissez le Seigneur, vous toutes, Ses oeuvres, dans tous les lieux de Sa domination. Mon âme, bénis le Seigneur.

}
%%%%%%% PSAUME 103 %%%%%%%
\newcommand{\psalmciiifr}{
\VERSE De David. Mon âme, bénis le Seigneur. Seigneur mon Dieu, Vous avez fait paraître magnifiquement Votre grandeur. Vous Vous êtes revêtu de majesté et de splendeur,  \EVERSE
\VERSE enveloppé de lumière comme d'un vêtement. Vous étendez le ciel comme une tente;  \EVERSE
\VERSE Vous couvrez d'eaux les parties supérieures; Vous montez sur les nuées, et Vous marchez sur les ailes des vents; \EVERSE
\VERSE Vous faites de Vos Anges des vents rapides, et de Vos ministres un feu brûlant. \EVERSE
\VERSE Vous avez fondé la terre sur sa base solide, elle ne sera jamais renversée. \EVERSE
\VERSE L'abîme l'enveloppe comme un vêtement; les eaux s'élèvent au-dessus des montagnes. \EVERSE
\VERSE Mais devant Votre menace elles fuiront; la voix de Votre tonnerre les épouvantera. \EVERSE
\VERSE Les montagnes s'élèvent, et les vallées descendent au lieu que Vous leur avez fixé. \EVERSE
\VERSE Vous leur avez prescrit des bornes qu'elles ne passeront point, et elles ne reviendront pas couvrir la terre. \EVERSE
\VERSE Vous faites jaillir les sources dans les vallées; les eaux s'écoulent entre les montagnes. \EVERSE
\VERSE Toutes les bêtes des champs s'y abreuvent; les ânes sauvages soupirent après elles dans leur soif. \EVERSE
\VERSE Au-dessus d'elles habitent les oiseaux du ciel; ils font entendre leurs voix du milieu des rochers. \EVERSE
\VERSE Vous arrosez les montagnes des eaux qui tombent d'en haut; la terre sera rassasiée du fruit de Vos oeuvres. \EVERSE
\VERSE Vous faites croître l'herbe pour les bêtes, et les plantes pour l'usage de l'homme. Vous faites sortir le pain de la terre,  \EVERSE
\VERSE et le vin qui réjouit le coeur de l'homme. Vous lui donnez l'huile, pour qu'elle répande la joie sur son visage; et le pain, pour qu'il fortifie son coeur. \EVERSE
\VERSE Les arbres de la campagne se rassasient, aussi bien que les cèdres du Liban, qu'Il a plantés.  \EVERSE
\VERSE C'est là que les oiseaux font leurs nids. La demeure du héron domine les autres.  \EVERSE
\VERSE Les hautes montagnes sont pour les cerfs, et les rochers pour les hérissons. \EVERSE
\VERSE Il a fait la lune pour marquer les temps; le soleil connaît l'heure de son coucher. \EVERSE
\VERSE Vous avez répandu les ténèbres, et la nuit est venue; c'est alors que toutes les bêtes de la forêt se mettent en mouvement. \EVERSE
\VERSE Les petits des lions rugissent après leur proie, et demandent à Dieu leur nourriture. \EVERSE
\VERSE Le soleil se lève, et ils se rassemblent, et vont se coucher dans leurs tanières. \EVERSE
\VERSE L'homme sort pour son ouvrage et pour son travail jusqu'au soir. \EVERSE
\VERSE Que Vos oeuvres sont grandes, Seigneur! Vous avez fait toutes choses avec sagesse; la terre est toute remplie de Vos biens. \EVERSE
\VERSE Voici la vaste mer, aux bras immenses: là sont les reptiles sans nombre, les animaux grands et petits.  \EVERSE
\VERSE C'est là que passent les navires, ce monstre que Vous avez formé pour s'y jouer.  \EVERSE
\VERSE Tous attendent de Vous que Vous leur donniez leur nourriture en son temps. \EVERSE
\VERSE Lorsque Vous la leur donnez, ils la recueillent; lorsque Vous ouvrez Votre main, ils sont tous remplis de Vos biens. \EVERSE
\VERSE Mais si Vous détournez Votre visage, ils seront troublés; Vous leur retirerez le souffle, et ils tomberont en défaillance et retourneront dans leur poussière. \EVERSE
\VERSE Vous enverrez Votre souffle, et ils seront créés, et Vous renouvellerez la face de la terre. \EVERSE
\VERSE Que la gloire du Seigneur soit célébrée à jamais; le Seigneur Se réjouira dans Ses oeuvres. \EVERSE
\VERSE Il regarde la terre et la fait trembler; Il touche les montagnes et elles fument. \EVERSE
\VERSE Je chanterai le Seigneur toute ma vie; je célébrerai mon Dieu tant que je serai. \EVERSE
\VERSE Puissent mes paroles Lui être agréables; pour moi je me délecterai dans le Seigneur. \EVERSE
\VERSE Que les pécheurs et les impies disparaissent de la terre, en sorte qu'ils ne soient plus. Mon âme, bénis le Seigneur. Alleluia.

}
%%%%%%% PSAUME 104 %%%%%%%
\newcommand{\psalmcivfr}{
\VERSE Alleluia: Célébrez le Seigneur et invoquez Son Nom; annoncez Ses oeuvres parmi les nations. \EVERSE
\VERSE Chantez et jouez des instruments en Son honneur; racontez toutes Ses merveilles. \EVERSE
\VERSE Glorifiez-vous dans Son saint Nom; que le coeur de ceux qui cherchent le Seigneur se réjouisse. \EVERSE
\VERSE Cherchez le Seigneur, et soyez remplis de force, cherchez sans cesse Son visage. \EVERSE
\VERSE Souvenez-vous des merveilles qu'Il a accomplies, de Ses prodiges et des jugements sortis de Sa bouche; \EVERSE
\VERSE ô vous, race d'Abraham, Son serviteur; vous, enfants de Jacob, Ses élus. \EVERSE
\VERSE C'est Lui qui est le Seigneur notre Dieu; Ses jugements s'exercent dans toute la terre. \EVERSE
\VERSE Il S'est souvenu pour toujours de Son alliance, de la parole qu'Il a prononcée pour mille générations; \EVERSE
\VERSE de ce qu'Il a promis à Abraham, et de Son serment à Isaac; \EVERSE
\VERSE et Il en a fait une loi pour Jacob, et une alliance éternelle pour Israël, \EVERSE
\VERSE en disant: Je te donnerai la terre de Chanaan, pour la part de ton héritage. \EVERSE
\VERSE Et ils étaient alors en petit nombre, et étrangers dans le pays. \EVERSE
\VERSE Et ils voyageaient de nation en nation, et d'un royaume à un autre peuple. \EVERSE
\VERSE Il ne permit point qu'aucun homme leur fît du mal,et Il réprimanda des rois à cause d'eux. \EVERSE
\VERSE Gardez-vous de toucher à Mes oints, et ne maltraitez pas Mes prophètes. \EVERSE
\VERSE Et Il appela la famine sur la terre, et Il brisa toute la force que procure le pain. \EVERSE
\VERSE Il envoya devant eux un homme; Joseph fut vendu comme esclave. \EVERSE
\VERSE On l'humilia en enchaînant ses pieds; le fer transperça son âme,  \EVERSE
\VERSE jusqu'à ce que Sa parole fût accomplie. La parole du Seigneur l'enflamma.  \EVERSE
\VERSE Le roi envoya et le délia; le prince des peuples le renvoya libre. \EVERSE
\VERSE Il l'établit le maître de sa maison, et le prince de tout ce qu'il possédait, \EVERSE
\VERSE afin qu'il instruisît ses princes comme lui-même, et qu'il apprît la sagesse à ses vieillards. \EVERSE
\VERSE Et Israël entra en Egypte, et Jacob séjourna dans la terre de Cham. \EVERSE
\VERSE Et Dieu multiplia extraordinairement Son peuple, et le rendit plus puissant que Ses ennemis. \EVERSE
\VERSE Il changea leur coeur, de sorte qu'ils haïrent Son peuple, et qu'ils usèrent de perfidie envers Ses serviteurs. \EVERSE
\VERSE Il envoya Moïse Son serviteur, et Aaron qu'Il avait choisi. \EVERSE
\VERSE Il mit en eux Sa puissance, pour accomplir des signes et des prodiges dans la terre de Cham. \EVERSE
\VERSE Il envoya les ténèbres, et fit l'obscurité; et ils ne résistèrent point à Ses ordres. \EVERSE
\VERSE Il changea leurs eaux en sang, et fit périr leurs poissons. \EVERSE
\VERSE Leur terre produisit des grenouilles jusque dans les chambres des rois eux-mêmes. \EVERSE
\VERSE Il parla, et les mouches et les moucherons envahirent tout leur territoire. \EVERSE
\VERSE Il leur donna pour pluies de la grêle, et un feu qui brûlait tout dans leur pays. \EVERSE
\VERSE Et Il frappa leurs vignes et leurs figuiers, et Il brisa tous les arbres de leurs contrées. \EVERSE
\VERSE Il parla, et la sauterelle arriva, des sauterelles sans nombre; \EVERSE
\VERSE et elles mangèrent toute l'herbe de leur terre, et elles dévorèrent tous les fruits de leur pays. \EVERSE
\VERSE Et Il frappa tous les premiers-nés de leur contrée, les prémices de tout leur travail. \EVERSE
\VERSE Et Il fit sortir les Hébreux avec de l'argent et de l'or, et il n'y avait pas de malades dans leurs tribus. \EVERSE
\VERSE L'Egypte fut réjouie de leur départ, car la frayeur qu'elle avait d'eux l'avait saisie. \EVERSE
\VERSE Il étendit une nuée pour les mettre à couvert, et un feu pour les éclairer pendant la nuit. \EVERSE
\VERSE Ils demandèrent, et les cailles arrivèrent, et Il les rassasia du pain du ciel. \EVERSE
\VERSE Il fendit la pierre, et les eaux jaillirent; des fleuves se répandirent dans le désert. \EVERSE
\VERSE Car Il Se souvint de Sa sainte parole, qu'Il avait donnée à Abraham Son serviteur. \EVERSE
\VERSE Et Il fit sortir Son peuple avec allégresse, et Ses élus avec des transports de joie. \EVERSE
\VERSE Il leur donna les pays des nations, et ils possédèrent les travaux des peuples, \EVERSE
\VERSE afin qu'ils gardassent Ses préceptes, et qu'ils recherchassent Sa loi.

}
%%%%%%% PSAUME 105 %%%%%%%
\newcommand{\psalmcvfr}{
\VERSE Alleluia: Célébrez le Seigneur, parce qu'Il est bon et que Sa miséricorde est éternelle. \EVERSE
\VERSE Qui racontera les oeuvres de puissance du Seigneur? Qui fera entendre toutes Ses louanges? \EVERSE
\VERSE Heureux ceux qui gardent l'équité, et qui pratiquent la justice en tout temps. \EVERSE
\VERSE Souvenez-Vous de nous, Seigneur, dans Votre bienveillance pour Votre peuple; visitez-nous par Votre salut: \EVERSE
\VERSE afin que nous voyions le bonheur de Vos élus, que nous nous réjouissions de la joie de Votre peuple, et que Vous soyiez loué avec Votre héritage. \EVERSE
\VERSE Nous avons péché avec nos pères, nous avons agi injustement, nous avons commis l'iniquité. \EVERSE
\VERSE Nos pères n'ont pas compris Vos merveilles en Egypte; ils ne se sont pas souvenus de la multitude de Vos miséricordes. Et ils Vous ont irrité lorsqu'ils montèrent vers la mer, la mer Rouge. \EVERSE
\VERSE Mais Dieu les sauva à cause de Son Nom, afin de faire connaître Sa puissance. \EVERSE
\VERSE Il menaça la mer Rouge, et elle se dessécha; Il les fit marcher au milieu des abîmes, comme dans le désert. \EVERSE
\VERSE Et Il les sauva des mains de ceux qui les haïssaient, et Il les délivra des mains de l'ennemi. \EVERSE
\VERSE Et l'eau engloutit leurs oppresseurs, il n'en resta pas un seul. \EVERSE
\VERSE Alors ils crurent à Ses paroles, et ils firent retentir Sa louange. \EVERSE
\VERSE Mais bientôt ils oublièrent Ses oeuvres, et ils n'attendirent pas l'accomplissement de Ses desseins. \EVERSE
\VERSE Ils se livrèrent à la convoitise dans le désert, et tentèrent Dieu dans la contrée sans eau. \EVERSE
\VERSE Il leur accorda leur demande, et envoya de quoi se rassasier. \EVERSE
\VERSE Et ils irritèrent Moïse dans le camp, et Aaron, le saint du Seigneur. \EVERSE
\VERSE La terre s'entr'ouvrit et engloutit Dathan, et couvrit la troupe d'Abiron. \EVERSE
\VERSE Un feu s'alluma contre leur bande; la flamme consuma les méchants. \EVERSE
\VERSE Et ils firent un veau à Horeb, et adorèrent une image sculptée. \EVERSE
\VERSE Et ils échangèrent leur gloire contre la figure d'un veau qui broute l'herbe. \EVERSE
\VERSE Ils oublièrent le Dieu qui les avait sauvés, qui avait fait de grandes choses en Egypte, \EVERSE
\VERSE des prodiges dans la terre de Cham, des choses terribles dans la mer Rouge. \EVERSE
\VERSE Et Il parlait de les exterminer, si Moïse, Son élu, ne se fût tenu sur la brèche, devant Lui, pour détourner Sa colère, et empêcher qu'Il ne les exterminât.  \EVERSE
\VERSE Et ils n'eurent que du mépris pour sa terre si désirable. Ils ne crurent point à Sa parole. \EVERSE
\VERSE Ils murmurèrent dans leurs tentes, et n'écoutèrent point la voix du Seigneur. \EVERSE
\VERSE Et Il leva Sa main sur eux, pour les exterminer dans le désert, \EVERSE
\VERSE pour rejeter leur race parmi les nations, et les disperser en divers pays. \EVERSE
\VERSE Ils se consacrèrent à Béelphegor, et mangèrent des sacrifices offerts à des dieux sans vie. \EVERSE
\VERSE Et ils irritèrent le Seigneur par leurs oeuvres criminelles, et la ruine s'accumula parmi eux. \EVERSE
\VERSE Phinées se leva et apaisa le Seigneur, et le fléau cessa. \EVERSE
\VERSE Et ce zèle lui a été imputé à justice, de génération en génération à jamais. \EVERSE
\VERSE Ils irritèrent le Seigneur aux Eaux de contradiction, et Moïse fut affligé à cause d'eux, \EVERSE
\VERSE car ils aigrirent son esprit, et il fit paraître de la défiance dans ses paroles.  \EVERSE
\VERSE Ils n'exterminèrent pas les peuples que le Seigneur leur avait marqués; \EVERSE
\VERSE mais ils se mêlèrent aux nations, et ils apprirent leurs oeuvres. \EVERSE
\VERSE Et ils adorèrent leurs idoles sculptées, qui leur devinrent une occasion de chute. \EVERSE
\VERSE Ils immolèrent leurs fils et leurs filles aux démons. \EVERSE
\VERSE Ils répandirent le sang innocent, le sang de leurs fils et de leurs filles, qu'ils sacrifièrent aux idoles de Chanaan. \EVERSE
\VERSE Et la terre fut infectée de meurtres, \EVERSE
\VERSE et elle fut souillée par leurs oeuvres, et ils se prostituèrent à leurs passions. \EVERSE
\VERSE Et le Seigneur entra dans une violente colère contre Son peuple, et Il eut en abomination Son héritage. \EVERSE
\VERSE Et Il les livra aux mains des nations, et ceux qui les haïssaient les assujettirent. \EVERSE
\VERSE Leurs ennemis les tourmentèrent, et ils furent humiliés sous leurs mains. \EVERSE
\VERSE Souvent Dieu les délivra; mais ils L'irritèrent par l'impiété de leurs desseins, et ils furent humiliés par leurs iniquités mêmes. \EVERSE
\VERSE Et Il les vit dans leur détresse, et Il écouta leur prière. \EVERSE
\VERSE Il se souvint de Son alliance, et Se repentit selon la grandeur de Sa miséricorde, \EVERSE
\VERSE et Il fit d'eux l'objet de Ses miséricordes, à la vue de tous ceux qui les avaient asservis. \EVERSE
\VERSE Sauvez-nous, Seigneur notre Dieu, et rassemblez-nous du milieu des nations, afin que nous célébrions Votre saint Nom, et que nous mettions notre gloire à Vous louer. \EVERSE
\VERSE Béni soit le Seigneur, le Dieu d'Israël, dans les siècles des siècles. Et tout le peuple dira: Ainsi soit-il, ainsi soit-il.

}
%%%%%%% PSAUME 106 %%%%%%%
\newcommand{\psalmcvifr}{
\VERSE Alleluia: Célébrez le Seigneur, parce qu'Il est bon et parce que Sa miséricorde est éternelle. \EVERSE
\VERSE Qu'ils le disent ceux qui ont été rachetés par le Seigneur, ceux qu'Il a rachetés de la main de l'ennemi et rassemblés de tous les pays, \EVERSE
\VERSE de l'orient et du couchant, du nord et de la mer. \EVERSE
\VERSE Ils ont erré dans le désert, dans les lieux arides, sans trouver une ville où ils pourraient habiter. \EVERSE
\VERSE Souffrant de la faim et de la soif, leur âme était tombée en défaillance. \EVERSE
\VERSE Ils crièrent au Seigneur dans leurs tribulations, et Il les tira de leurs nécessités, \EVERSE
\VERSE et Il les conduisit dans le droit chemin, pour les faire arriver à une ville qu'ils pussent habiter \EVERSE
\VERSE Qu'ils célèbrent le Seigneur pour Sa miséricorde, et pour Ses merveilles en faveur des enfants des hommes, \EVERSE
\VERSE car Il a rassasié l'âme épuisée, et Il a rempli de biens l'âme affamée. \EVERSE
\VERSE Ils étaient assis dans les ténèbres et dans l'ombre de la mort, captifs, dans l'indigence et dans les fers, \EVERSE
\VERSE parce qu'ils s'étaient révoltés contre les ordres de Dieu, et avaient méprisé le conseil du Très-Haut. \EVERSE
\VERSE Leur coeur fut humilié par les travaux; ils furent épuisés, et il n'y avait personne qui les secourût. \EVERSE
\VERSE Ils crièrent au Seigneur dans leur tribulation, et Il les tira de leurs nécessités, \EVERSE
\VERSE et Il les fit sortir des ténèbres et de l'ombre de la mort, et Il rompit leurs liens. \EVERSE
\VERSE Qu'ils célèbrent le Seigneur pour Sa miséricorde, et pour Ses merveilles en faveur des enfants des hommes; \EVERSE
\VERSE car Il a brisé les portes d'airain, et rompu les verrous de fer. \EVERSE
\VERSE Il les a retirés de la voie de leur iniquité; car ils avaient été humiliés à cause de leurs injustices. \EVERSE
\VERSE Leur âme avait en horreur toute nourriture, et ils étaient près des portes de la mort. \EVERSE
\VERSE Et ils crièrent au Seigneur dans leur tribulation, et Il les tira de leurs nécessités. \EVERSE
\VERSE Il envoya Sa parole, et Il les guérit, et les arracha à la mort. \EVERSE
\VERSE Qu'ils louent le Seigneur pour Sa miséricorde, et pour Ses merveilles en faveur des enfants des hommes. \EVERSE
\VERSE Qu'ils Lui offrent un sacrifice de louange, et qu'ils publient Ses oeuvres avec allégresse. \EVERSE
\VERSE Ceux qui descendent sur la mer dans des navires, et qui travaillent sur les vastes eaux, \EVERSE
\VERSE ceux-là ont vu les oeuvres du Seigneur, et Ses merveilles au milieu de l'abîme. \EVERSE
\VERSE Il dit, et le souffle de la tempête se leva, et les flots de la mer furent soulevés. \EVERSE
\VERSE Ils montent jusqu'au ciel, et descendent jusqu'aux abîmes; leur âme défaillait parmi leurs maux. \EVERSE
\VERSE Ils étaient troublés et agités comme un homme ivre, et toute leur sagesse était anéantie. \EVERSE
\VERSE Et ils crièrent au Seigneur dans leur tribulation, et Il les tira de leurs nécessités. \EVERSE
\VERSE Il changea la tempête en un vent léger, et les flots de la mer s'apaisèrent. \EVERSE
\VERSE Ils se réjouirent de les voir apaisés, et Dieu les conduisit au port où ils voulaient arriver. \EVERSE
\VERSE Qu'ils louent le Seigneur pour Ses miséricordes, et pour Ses merveilles en faveur des enfants des hommes. \EVERSE
\VERSE Qu'ils L'exaltent dans l'assemblée du peuple, et qu'ils Le louent dans le conseil des vieillards. \EVERSE
\VERSE Il a changé les fleuves en désert, et les sources d'eaux en un sol desséché, \EVERSE
\VERSE et la terre fertile en plaine de sel, à cause de la malice de ses habitants. \EVERSE
\VERSE Il a changé les déserts en nappes d'eaux, et la terre aride en eaux courantes. \EVERSE
\VERSE Et Il y a établi les affamés, et ils y ont bâti une ville pour y habiter; \EVERSE
\VERSE ils ont semé des champs et planté des vignes, et recueilli des fruits abondants. \EVERSE
\VERSE Ils les a bénis, et ils se sont multipliés extrêmement; Il n'a pas laissé amoindrir leurs troupeaux. \EVERSE
\VERSE Puis ils ont été réduits à un petit nombre, et accablés par l'affliction de leurs maux et la douleur. \EVERSE
\VERSE Le mépris a été répandu sur les princes, et Il les a fait errer hors de la voie et en des lieux sans chemin. \EVERSE
\VERSE Et Il a secouru le pauvre dans son indigence et multiplié les familles comme des troupeaux. \EVERSE
\VERSE Les justes le verront et se réjouiront, et l'iniquité devra fermer la bouche. \EVERSE
\VERSE Qui est sage pour prendre garde à ces choses, et pour comprendre les miséricordes du Seigneur?

}
%%%%%%% PSAUME 107 %%%%%%%
\newcommand{\psalmcviifr}{
\VERSE Cantique psaume, de David. \EVERSE
\VERSE Mon coeur est préparé, ô Dieu, mon coeur est préparé; je chanterai et je psalmodierai dans ma gloire. \EVERSE
\VERSE Levez-vous, ma gloire; levez-vous, mon luth et ma harpe; je me lèverai dès l'aurore. \EVERSE
\VERSE Je Vous célébrerai, Seigneur, au milieu des peuples, et je Vous chanterai parmi les nations; \EVERSE
\VERSE car Votre miséricorde s'est élevée plus haut que les cieux, et Votre vérité jusqu'aux nues. \EVERSE
\VERSE Soyez exalté, ô Dieu, au-dessus des cieux, et que Votre gloire brille sur toute la terre;  \EVERSE
\VERSE pour que Vos bien-aimés soient délivrés, sauvez-moi par Votre droite et exaucez-moi.  \EVERSE
\VERSE Dieu a parlé dans Son sanctuaire: Je Me réjouirai, et Je partagerai Sichem, et Je mesurerai la vallée des Tentes. \EVERSE
\VERSE Galaad est à Moi, et à Moi Manassé, et Ephraïm est le soutien de Ma tête. Juda est Mon roi;  \EVERSE
\VERSE Moab est comme le vase de Mon espérance. J'étendrai Ma chaussure sur l'Idumée; les étrangers sont devenus Mes amis. \EVERSE
\VERSE Qui me conduira à la ville fortifiée? qui me conduira jusqu'en Idumée? \EVERSE
\VERSE N'est-ce pas Vous, ô Dieu, qui nous avez repoussés? et ne sortirez-Vous pas, ô Dieu, à la tête de nos armées? \EVERSE
\VERSE Donnez-nous du secours contre la tribulation, car la protection de l'homme est vaine. \EVERSE
\VERSE Avec Dieu nous ferons des actes de courage, et Lui-même réduira à néant nos ennemis.

}
%%%%%%% PSAUME 108 %%%%%%%
\newcommand{\psalmcviiifr}{
\VERSE Pour la fin, psaume de David. \EVERSE
\VERSE O Dieu, ne Vous taisez pas sur ma louange, car la bouche du pécheur et la bouche de l'homme fourbe sont ouvertes contre moi. \EVERSE
\VERSE Ils ont parlé contre moi avec une langue perfide, ils m'ont comme assiégé par leurs discours haineux, et ils m'ont fait la guerre sans sujet. \EVERSE
\VERSE Au lieu de m'aimer, ils me calomniaient; mais moi, je demeurais en prière. \EVERSE
\VERSE Ils m'ont rendu le mal pour le bien, et la haine pour mon amour. \EVERSE
\VERSE Livrez-le au pouvoir du pécheur, et que le démon se tienne à sa droite. \EVERSE
\VERSE Lorsqu'on le jugera, qu'il sorte condamné, et que sa prière même soit imputée à péché. \EVERSE
\VERSE Que ses jours soient abrégés, et qu'un autre reçoive sa charge. \EVERSE
\VERSE Que ses enfants deviennent orphelins, et que sa femme devienne veuve. \EVERSE
\VERSE Que ses enfants errent vagabonds et qu'ils mendient, et qu'ils soient chassés de leurs demeures. \EVERSE
\VERSE Que l'usurier recherche et enlève tout son bien, et que les étrangers ravissent le fruit de ses travaux. \EVERSE
\VERSE Que personne ne l'assiste, et que nul n'ait compassion de ses orphelins. \EVERSE
\VERSE Que tous ses enfants périssent, et que son nom soit effacé au cours d'une seule génération. \EVERSE
\VERSE Que l'iniquité de ses pères revive dans le souvenir du Seigneur, et que le péché de sa mère ne soit point effacé. \EVERSE
\VERSE Qu'ils soient toujours présents devant le Seigneur, et que leur mémoire disparaisse de dessus la terre;  \EVERSE
\VERSE parce qu'il ne s'est point souvenu de faire miséricorde, \EVERSE
\VERSE qu'il a poursuivi l'homme pauvre et indigent, et l'homme au coeur brisé, pour le faire mourir. \EVERSE
\VERSE Il a aimé la malédiction, et elle tombera sur lui; il n'a pas voulu de la bénédiction, et elle sera éloignée de lui. Et il s'est revêtu de la malédiction comme d'un vêtement; elle a pénétré comme l'eau au dedans de lui, et comme l'huile dans ses os. \EVERSE
\VERSE Qu'elle lui soit comme le vêtement qui le couvre, et comme la ceinture dont il est toujours ceint. \EVERSE
\VERSE C'est ainsi que le Seigneur punira ceux qui me calomnient, et qui profèrent le mal contre mon âme. \EVERSE
\VERSE Et Vous, Seigneur, Seigneur, prenez ma défense à cause de Votre Nom, parce que Votre miséricorde est pleine de douceur. Délivrez-moi, \EVERSE
\VERSE car je suis pauvre et indigent, et mon coeur est tout troublé au dedans de moi. \EVERSE
\VERSE Je disparais comme l'ombre à son déclin, et je suis secoué comme les sauterelles. \EVERSE
\VERSE Mes genoux se sont affaiblis par le jeûne, et ma chair est toute changée, parce qu'elle est privée d'huile. \EVERSE
\VERSE Je suis devenu pour eux un sujet d'opprobre; ils m'ont vu, et ils ont branlé la tête. \EVERSE
\VERSE Secourez-moi, Seigneur mon Dieu; sauvez-moi selon Votre miséricorde. \EVERSE
\VERSE Et qu'ils sachent que c'est Votre main, et que c'est Vous, Seigneur, qui faites ces choses. \EVERSE
\VERSE Ils maudiront, mais Vous, Vous bénirez. Que ceux qui se lèvent contre moi soient confondus, tandis que Votre serviteur se réjouira. \EVERSE
\VERSE Que ceux qui me calomnient soient couverts de honte, et qu'ils soient revêtus de leur confusion comme d'un manteau double. \EVERSE
\VERSE Ma bouche célébrera le Seigneur de toute sa force, et je Le louerai au milieu d'une grande assemblée, \EVERSE
\VERSE parce qu'Il S'est tenu à la droite du pauvre, pour sauver mon âme de ceux qui la persécutent.

}
%%%%%%% PSAUME 109 %%%%%%%
\newcommand{\psalmcixfr}{
\VERSE Psaume de David. Le Seigneur a dit à mon Seigneur: Asseyez-Vous à ma droite, jusqu'à ce que Je fasse de Vos ennemis l'escabeau de Vos pieds. \EVERSE
\VERSE Le Seigneur fera sortir de Sion le sceptre de Votre puissance; dominez au milieu de Vos ennemis. \EVERSE
\VERSE Avec Vous sera l'empire souverain au jour de Votre puissance, parmi les splendeurs des saints. Je Vous ai engendré de Mon sein avant l'aurore. \EVERSE
\VERSE Le Seigneur a juré, et Il ne S'en repentira point: Vous êtes prêtre à jamais selon l'ordre de Melchisedech. \EVERSE
\VERSE Le Seigneur est à Votre droite; Il a brisé les rois au jour de Sa colère. \EVERSE
\VERSE Il jugera les nations; Il remplira tout de ruines; Il écrasera sur la terre les têtes d'un grand nombre. \EVERSE
\VERSE Il boira de l'eau du torrent dans le chemin; c'est pourquoi Il relèvera la tête.

}
%%%%%%% PSAUME 110 %%%%%%%
\newcommand{\psalmcxfr}{
\VERSE Alleluia: Seigneur, je Vous célébrerai de tout mon coeur dans la réunion et dans l'assemblée des justes. \EVERSE
\VERSE Les oeuvres du Seigneur sont grandes, proportionnées à toutes Ses volontés. \EVERSE
\VERSE Son oeuvre est splendeur et magnificence, et Sa justice demeure dans tous les siècles. \EVERSE
\VERSE Le Seigneur a institué un mémorial de Ses merveilles, Lui qui est miséricordieux et compatissant;  \EVERSE
\VERSE Il a donné une nourriture à ceux qui Le craignent. Il Se souviendra éternellement de Son alliance.  \EVERSE
\VERSE Il fera connaître à Son peuple la puissance de Ses oeuvres, \EVERSE
\VERSE en leur donnant l'héritage des nations. Les oeuvres de Ses mains sont vérité et justice. \EVERSE
\VERSE Tous Ses préceptes sont immuables, affermis pour les siècles des siècles,fondés sur la vérité et l'équité. \EVERSE
\VERSE Il a envoyé la délivrance à Son peuple; Il a établi pour toujours Son alliance. Son Nom est saint et terrible. \EVERSE
\VERSE La crainte du Seigneur est le commencement de la sagesse. La vraie intelligence est en tous ceux qui agissent selon cette crainte. Sa louange subsiste dans les siècles des siècles.

}
%%%%%%% PSAUME 111 %%%%%%%
\newcommand{\psalmcxifr}{
\VERSE Alleluia, au retour d'Aggée et de Zacharie: Heureux l'homme qui craint le Seigneur, et qui met ses délices dans Ses commandements. \EVERSE
\VERSE Sa race sera puissante sur la terre; la postérité des justes sera bénie. \EVERSE
\VERSE La gloire et les richesses sont dans sa maison, et sa justice demeure dans tous les siècles. \EVERSE
\VERSE Une lumière s'est levée dans les ténèbres pour les hommes droits; il est miséricordieux, et compatissant, et juste. \EVERSE
\VERSE Heureux l'homme qui compatit et qui prête, qui règle ses discours avec jugement,  \EVERSE
\VERSE car il ne sera jamais ébranlé. \EVERSE
\VERSE Le souvenir du juste sera éternel; il ne craindra pas d'entendre rien d'affligeant. Son coeur est disposé à espérer au Seigneur. \EVERSE
\VERSE Son coeur est affermi; il ne sera point ébranlé, jusqu'à ce qu'il contemple ses ennemis avec mépris. \EVERSE
\VERSE Il répand ses largesses, il donne aux pauvres. Sa justice demeure dans tous les siècles. Sa puissance sera élevée dans la gloire. \EVERSE
\VERSE Le pécheur le verra et s'irritera; il grincera des dents et séchera de dépit; le désir des pécheurs périra.

}
%%%%%%% PSAUME 112 %%%%%%%
\newcommand{\psalmcxiifr}{
\VERSE Alleluia: Louez le Seigneur, vous Ses serviteurs, louez le Nom du Seigneur. \EVERSE
\VERSE Que le Nom du Seigneur soit béni, dès maintenant et dans tous les siècles. \EVERSE
\VERSE Du lever du soleil à son couchant, le Nom du Seigneur est digne de louange. \EVERSE
\VERSE Le Seigneur est élevé au-dessus de toutes les nations, et Sa gloire est au-dessus des cieux. \EVERSE
\VERSE Qui est semblable au Seigneur notre Dieu, qui habite dans les hauteurs,  \EVERSE
\VERSE et qui regarde ce qui est humble au Ciel et sur la terre? \EVERSE
\VERSE Il tire l'indigent de la poussière, et relève le pauvre du fumier, \EVERSE
\VERSE pour le placer avec les princes, avec les princes de son peuple. \EVERSE
\VERSE Il fait habiter celle qui était stérile dans la maison, comme une mère joyeuse au milieu de ses enfants.

}
%%%%%%% PSAUME 113 %%%%%%%
\newcommand{\psalmcxiiifr}{
\VERSE Alleluia: Lorsque Israël sortit d'Egypte, et la maison de Jacob du milieu d'un peuple barbare, \EVERSE
\VERSE Dieu consacra Juda à Son service, et établit Son empire dans Israël. \EVERSE
\VERSE La mer le vit et s'enfuit; le Jourdain retourna en arrière. \EVERSE
\VERSE Les montagnes bondirent comme des béliers, et les collines comme des agneaux. \EVERSE
\VERSE Qu'as-tu, ô mer, pour t'enfuir? Et toi, Jourdain, pour retourner en arrière? \EVERSE
\VERSE Pourquoi, montagnes, avez-vous bondi comme des béliers? et vous, collines, comme des agneaux? \EVERSE
\VERSE La terre a été ébranlée devant la face du Seigneur, devant la face du Dieu de Jacob, \EVERSE
\VERSE qui a changé la pierre en des torrents d'eaux, et la roche en fontaines abondantes. \EVERSE
\VERSE Que ce ne soit pas à nous, Seigneur, que ce ne soit pas à nous; que ce soit à Votre Nom que Vous donniez la gloire, \EVERSE
\VERSE pour faire éclater Votre miséricorde et Votre vérité; de peur que les nations ne disent: Où est leur Dieu? \EVERSE
\VERSE Notre Dieu est dans le Ciel; tout ce qu'Il a voulu, Il l'a fait. \EVERSE
\VERSE Les idoles des nations sont de l'argent et de l'or, et l'ouvrage des mains des hommes. \EVERSE
\VERSE Elles ont une bouche, et ne parlent point; elles ont des yeux, et ne voient point. \EVERSE
\VERSE Elles ont des oreilles, et n'entendent pas; elles ont des narines, et ne sentent pas. \EVERSE
\VERSE Elles ont des mains, et ne touchent pas; elles ont des pieds, et ne marchent pas; avec leur gorge, elles ne peuvent crier. \EVERSE
\VERSE Que ceux qui les font leur deviennent semblables, avec tous ceux qui mettent en elles leur confiance. \EVERSE
\VERSE La maison d'Israël a espéré au Seigneur; Il est leur secours et leur protecteur. \EVERSE
\VERSE La maison d'Aaron a espéré au Seigneur; Il est leur secours et leur protecteur. \EVERSE
\VERSE Ceux qui craignent le Seigneur ont mis en Lui leur espérance; Il est leur secours et leur protecteur. \EVERSE
\VERSE Le Seigneur S'est souvenu de nous, et Il nous a bénis. Il a béni la maison d'Israël; Il a béni la maison d'Aaron. \EVERSE
\VERSE Il a béni tous ceux qui craignent le Seigneur, les petits et les grands. \EVERSE
\VERSE Que le Seigneur vous comble de nouveaux biens, vous et vos enfants. \EVERSE
\VERSE Soyez béni du Seigneur, qui a fait le ciel et la terre. \EVERSE
\VERSE Le Ciel des cieux est au Seigneur, mais Il a donné la terre aux enfants des hommes. \EVERSE
\VERSE Les morts ne Vous loueront point, Seigneur, ni tous ceux qui descendent dans l'enfer. \EVERSE
\VERSE Mais nous qui vivons, nous bénissons Le Seigneur, dès maintenant et dans tous les siècles.

}
%%%%%%% PSAUME 114 %%%%%%%
\newcommand{\psalmcxivfr}{
\VERSE Alleluia: J'aime le Seigneur, parce qu'il exaucera la voix de ma prière. \EVERSE
\VERSE Parce qu'il a incliné vers moi son oreille, je l'invoquerai tous les jours de ma vie. \EVERSE
\VERSE Les douleurs de la mort m'ont environné, et les périls de l'enfer n'ont surpris. J'ai trouvé l'affliction et la douleur,  \EVERSE
\VERSE et j'ai invoqué le Nom du Seigneur: O Seigneur, délivrez mon âme.  \EVERSE
\VERSE Le Seigneur est miséricordieux et juste, et notre Dieu est compatissant \EVERSE
\VERSE Le Seigneur garde les petits; j'ai été humilié et il m'a déllivré. \EVERSE
\VERSE Rentre, ô mon âme, dans ton repos, car le Seigneur t'a comblée de biens. \EVERSE
\VERSE Car Il a délivré mon âme de la mort, mes yeux des larmes, mes pieds de la chute. \EVERSE
\VERSE Je plairai au Seigneur dans la terre des vivants.

}
%%%%%%% PSAUME 115 %%%%%%%
\newcommand{\psalmcxvfr}{

\VERSE Alleluia. J'ai cru, c'est pourquoi j'ai parlé; mais j'ai été dans une profonde humiliation. \EVERSE
\VERSE J'ai dit dans mon abattement extrême: Tout homme est menteur. \EVERSE
\VERSE Que rendrai-je au Seigneur pour tous les biens qu'Il m'a faits? \EVERSE
\VERSE Je prendrai le calice du salut, et j'invoquerai le Nom du Seigneur. \EVERSE
\VERSE Je rendrai mes voeux au Seigneur devant tout Son peuple.  \EVERSE
\VERSE La mort de Ses saints est précieuse aux yeux du Seigneur. \EVERSE
\VERSE O Seigneur, je suis Votre serviteur; je suis Votre serviteur, et le fils de Votre servante. Vous avez rompu mes liens;  \EVERSE
\VERSE je Vous sacrifierai une hostie de louanges, et j'invoquerai le Nom du Seigneur. \EVERSE
\VERSE Je rendrai mes voeux au Seigneur en présence de tout Son peuple,  \EVERSE
\VERSE dans les parvis de la maison du Seigneur, au milieu de toi, Jérusalem.

}
%%%%%%% PSAUME 116 %%%%%%%
\newcommand{\psalmcxvifr}{
\VERSE Alleluia. Nations, louez toutes le Seigneur; peuples, louez-Le tous. \EVERSE
\VERSE Car Sa miséricorde a été affermie sur nous, et la vérité du Seigneur demeure éternellement.

}
%%%%%%% PSAUME 117 %%%%%%%
\newcommand{\psalmcxviifr}{
\VERSE Alleluia. Célébrez le Seigneur, parce qu'Il est bon, parce que Sa miséricorde est éternelle. \EVERSE
\VERSE Qu'Israël dise maintenant qu'Il est bon, et que Sa miséricorde est éternelle. \EVERSE
\VERSE Que la maison d'Aaron dise maintenant que Sa miséricorde est éternelle. \EVERSE
\VERSE Que ceux qui craignent le Seigneur disent maintenant que Sa miséricorde est éternelle. \EVERSE
\VERSE Du sein de la tribulation j'ai invoqué le Seigneur, et le Seigneur m'a exaucé et mis au large. \EVERSE
\VERSE Le Seigneur est mon secours; je ne craindrai pas ce que l'homme pourra me faire. \EVERSE
\VERSE Le Seigneur est mon secours, et je mépriserai mes ennemis. \EVERSE
\VERSE Il vaut mieux se confier au Seigneur que de se confier dans l'homme. \EVERSE
\VERSE Il vaut mieux espérer au Seigneur, plutôt que d'espérer dans les princes. \EVERSE
\VERSE Toutes les nations m'ont entouré, et au Nom du Seigneur je me suis vengé d'elles. \EVERSE
\VERSE Elles m'ont environné et assiégé, et au Nom du Seigneur je me suis vengé d'elles. \EVERSE
\VERSE Elles m'ont environné comme des abeilles, et elles se sont embrasées comme un feu d'épines; et au Nom du Seigneur je me suis vengé d'elles. \EVERSE
\VERSE J'ai été poussé, heurté et prêt à tomber, et le Seigneur m'a soutenu. \EVERSE
\VERSE Le Seigneur est ma force et ma gloire, et Il S'est fait mon salut. \EVERSE
\VERSE Le cri de l'allégresse et de la délivrance retentit dans les tentes des justes. \EVERSE
\VERSE La droite du Seigneur a fait éclater Sa puissance, la droite du Seigneur m'a exalté; la droite du Seigneur a fait éclater Sa puissance. \EVERSE
\VERSE Je ne mourrai point, mais je vivrai, et je raconterai les oeuvres du Seigneur. \EVERSE
\VERSE Le Seigneur m'a rudement châtié, mais Il ne m'a pas livré à la mort. \EVERSE
\VERSE Ouvrez-moi les portes de la justice, afin que j'y entre et que je célèbre le Seigneur.  \EVERSE
\VERSE C'est là la porte du Seigneur, et les justes entreront par elle. \EVERSE
\VERSE Je Vous rendrai grâces de ce que Vous m'avez exaucé, et que Vous Vous êtes fait mon salut. \EVERSE
\VERSE La pierre rejetée par ceux qui bâtissaient est devenue la pierre angulaire. \EVERSE
\VERSE C'est le Seigneur qui a fait cela, et c'est une chose merveilleuse à nos yeux. \EVERSE
\VERSE Voici le jour que le Seigneur a fait; passons-le dans l'allégresse et dans la joie. \EVERSE
\VERSE O Seigneur, sauvez-moi; ô Seigneur, faites-nous prospérer.  \EVERSE
\VERSE Béni soit celui qui vient au Nom du Seigneur. Nous Vous bénissons de la maison du Seigneur.  \EVERSE
\VERSE Le Seigneur est Dieu, et Il a fait briller sur nous Sa lumière. Rendez ce jour solennel en couvrant tout de feuillage, jusqu'à la corne du l'autel. \EVERSE
\VERSE Vous êtes mon Dieu, et je Vous célébrerai; Vous êtes mon Dieu, et je Vous exalterai. Je Vous célébrerai parce que Vous m'avez exaucé, et que Vous Vous êtes fait mon salut. \EVERSE
\VERSE Louez le Seigneur, parce qu'Il est bon, parce que Sa miséricorde est éternelle.

}
%%%%%%% PSAUME 118 %%%%%%%
\newcommand{\psalmcxviiiafr}{
\VERSE Alleluia. ALEPH Heureux ceux qui sont immaculés dans la voie, qui marchent dans la loi du Seigneur. \EVERSE
\VERSE Heureux ceux qui étudient Ses ordonnances, et qui Le cherchent de tout leur coeur. \EVERSE
\VERSE Car ceux qui commettent l'iniquité ne marchent pas dans Ses voies. \EVERSE
\VERSE Vous avez ordonné que Vos commandements soient très exactement gardés. \EVERSE
\VERSE Puissent mes voies être dirigées de telle sorte, que je garde Vos ordonnances! \EVERSE
\VERSE Je ne serai point confondu, lorsque j'aurai sous les yeux tous Vos préceptes. \EVERSE
\VERSE Je Vous louerai dans la droiture de mon coeur, de ce que j'ai appris les préceptes de Votre justice. \EVERSE
\VERSE Je garderai Vos ordonnances; ne m'abandonnez pas entièrement. \EVERSE
\VERSE BETH Comment le jeune homme corrigera-t-il sa voie? En accomplissant Vos paroles. \EVERSE
\VERSE Je Vous ai cherché de tout mon coeur; ne me rejetez pas de la voie de Vos préceptes. \EVERSE
\VERSE J'ai caché Vos paroles dans mon coeur, pour ne pas pécher contre Vous. \EVERSE
\VERSE Vous êtes béni, Seigneur; enseignez-moi Vos commandements. \EVERSE
\VERSE J'ai prononcé de mes lèvres tous les préceptes de Votre bouche. \EVERSE
\VERSE Je me suis complu dans la voie de Vos ordres, autant que dans toutes les richesses. \EVERSE
\VERSE Je m'exercerai dans Vos commandements, et je considérerai Vos voies. \EVERSE
\VERSE Je méditerai sur Vos ordonnances; je n'oublierai point Vos paroles. \EVERSE
\VERSE GHIMEL Bénissez Votre serviteur; faites-moi vivre, et je garderai Vos paroles. \EVERSE
\VERSE Dévoilez mes yeux, et je considérerai les merveilles de Votre loi. \EVERSE
\VERSE Je suis étranger sur la terre; ne me cachez pas Vos commandements. \EVERSE
\VERSE Mon âme a désiré en tout temps Vos ordonnances avec une grande ardeur. \EVERSE
\VERSE Vous avez menacé les superbes; ceux qui se détournent de Vos préceptes sont maudits. \EVERSE
\VERSE Eloignez de moi l'opprobre et le mépris, car j'ai recherché Vos commandements. \EVERSE
\VERSE Car les princes se sont assis et ont parlé contre moi; mais Votre serviteur méditait sur Vos lois. \EVERSE
\VERSE Car Vos préceptes sont le sujet de ma méditation, et Vos ordonnances me servent de conseil. \EVERSE
\VERSE DALETH Mon âme est prosternée contre terre; rendez-moi la vie selon Votre parole. \EVERSE
\VERSE Je Vous ai exposé mes voies, et Vous m'avez exaucé; enseignez-moi Vos préceptes. \EVERSE
\VERSE Instruisez-moi de la voie de Vos ordonnances, et je m'exercerai dans Vos merveilles. \EVERSE
\VERSE Mon âme s'est assoupie d'ennui; fortifiez-moi par Vos paroles. \EVERSE
\VERSE Eloignez de moi la voie de l'iniquité, et faites-moi miséricorde selon Votre loi. \EVERSE
\VERSE J'ai choisi la voie de la vérité; je n'ai point oublié Vos jugements. \EVERSE
\VERSE Seigneur, je me suis attaché à Vos préceptes; ne permettez pas que je sois confondu. \EVERSE
\VERSE J'ai couru dans la voie de Vos commandements, lorsque Vous avez dilaté mon coeur.
}
\newcommand{\psalmcxviiibfr}{
\VERSE Imposez-moi HE pour loi, Seigneur, la voie de Vos ordonnances, et je la rechercherai sans cesse. \EVERSE
\VERSE Donnez-moi l'intelligence, et j'étudierai Votre loi, et je la garderai de tout mon coeur. \EVERSE
\VERSE Conduisez-moi dans le sentier de Vos commandements, car j'y ai mis mon affection. \EVERSE
\VERSE Faites pencher mon coeur vers Vos préceptes, et non vers l'avarice. \EVERSE
\VERSE Détournez mes yeux, pour qu'ils ne voient pas la vanité; faites-moi vivre dans Votre voie. \EVERSE
\VERSE Etablissez fortement Votre parole dans Votre serviteur par Votre crainte. \EVERSE
\VERSE Eloignez de moi l'opprobre que j'appréhende, car Vos jugements sont pleins de douceur. \EVERSE
\VERSE J'ai beaucoup désiré Vos commandements, faites-moi vivre dans Votre justice. \EVERSE
\VERSE VAU Que Votre miséricorde vienne sur moi, Seigneur, et Votre assistance salutaire, selon Votre parole. \EVERSE
\VERSE Et je pourrai répondre à ceux qui m'insultent que j'espère en Vos promesses. \EVERSE
\VERSE Et n'ôtez pas entièrement de ma bouche la parole de la vérité, car j'espère en Vos jugements. \EVERSE
\VERSE Et je garderai toujours Votre loi, dans les siècles et dans les siècles des siècles. \EVERSE
\VERSE Je marchais au large, car j'ai cherché Vos commandements. \EVERSE
\VERSE Je parlais de Vos préceptes devant les rois, et je n'en avais pas de confusion. \EVERSE
\VERSE Et je méditais sur Vos commandements, car je les aime. \EVERSE
\VERSE J'ai levé mes mains vers Vos commandements que j'aime, et je m'exerçais dans Vos ordonnances. \EVERSE
\VERSE ZAIN Souvenez-Vous de la parole que Vous avez dite à Votre serviteur; par elle Vous m'avez donné de l'espérance. \EVERSE
\VERSE C'est ce qui m'a consolé dans mon humiliation, parce que Votre parole m'a donné la vie. \EVERSE
\VERSE Les superbes agissaient constamment avec injustice; mais je ne me suis point détourné de Votre loi. \EVERSE
\VERSE Seigneur, je me suis souvenu de Vos jugements antiques, et j'ai été consolé. \EVERSE
\VERSE Je suis tombé en défaillance, à cause des pécheurs qui abandonnent Votre loi. \EVERSE
\VERSE Vos préceptes sont le sujet de mes cantiques dans le lieu de mon exil. \EVERSE
\VERSE La nuit je me suis souvenu de Votre Nom, Seigneur, et j'ai gardé Votre loi. \EVERSE
\VERSE Cela m'est arrivé, parce que j'ai recherché Vos préceptes.
}
\newcommand{\psalmcxviiicfr}{
\VERSE Vous êtes mon partage HETH, Seigneur; j'ai résolu de garder Votre loi. \EVERSE
\VERSE J'ai imploré Votre face de tout mon coeur; ayez pitié de moi selon Votre parole. \EVERSE
\VERSE J'ai réfléchi à mes voies, et j'ai tourné mes pas vers Vos préceptes. \EVERSE
\VERSE Je suis prêt, sans que rien puisse me troubler, à garder Vos commandements. \EVERSE
\VERSE Les filets des pécheurs m'ont enveloppé, mais je n'ai pas oublié Votre loi. \EVERSE
\VERSE Au milieu de la nuit je me levais pour Vous louer sur les jugements de Votre justice. \EVERSE
\VERSE Je suis l'associé de tous ceux qui Vous craignent, et qui gardent Vos commandements. \EVERSE
\VERSE La terre, Seigneur, est pleine de Votre miséricorde; enseignez-moi Vos ordonnances. \EVERSE
\VERSE TETH Seigneur, Vous avez usé de bonté envers Votre serviteur, selon Votre parole. \EVERSE
\VERSE Enseignez-moi la bonté, la discipline et la science, parce que j'ai cru à Vos commandements. \EVERSE
\VERSE Avant d'être humilié, j'ai péché; c'est pour cela que j'ai gardé Votre parole. \EVERSE
\VERSE Vous êtes bon, et dans Votre bonté enseignez-moi Vos préceptes. \EVERSE
\VERSE L'iniquité des superbes s'est multipliée contre moi; et moi, j'étudie de tout mon coeur Vos commandements. \EVERSE
\VERSE Leur coeur s'est épaissi comme le lait; mais moi, je me suis appliqué à méditer Votre loi. \EVERSE
\VERSE Il m'est bon que Vous m'ayez humilié, afin que j'apprenne Vos préceptes. \EVERSE
\VERSE Mieux vaut pour moi la loi sortie de Votre bouche, que des millions d'or et d'argent. \EVERSE
\VERSE IOD Vos mains m'ont fait et m'ont formé; donnez-moi l'intelligence, afin que j'apprenne Vos commandements. \EVERSE
\VERSE Ceux qui Vous craignent me verront et se réjouiront, parce que j'ai mis mon espérance dans Vos paroles. \EVERSE
\VERSE J'ai reconnu, Seigneur, que Vos jugements sont équitables, et que Vous m'avez humilié selon Votre justice. \EVERSE
\VERSE Que Votre miséricorde soit ma consolation, selon la parole que Vous avez donnée à Votre serviteur. \EVERSE
\VERSE Que Vos compassions viennent sur moi, afin que je vive; car Votre loi est ma méditation. \EVERSE
\VERSE Que les superbes soient confondus, pour m'avoir maltraité injustement; mais moi, je m'exercerai dans Vos commandements. \EVERSE
\VERSE Que ceux qui Vous craignent se tournent vers moi, et ceux qui connaissent Vos préceptes. \EVERSE
\VERSE Que mon coeur soit pur envers Vos lois, afin que je ne sois pas confondu.
}
\newcommand{\psalmcxviiidfr}{
\VERSE Mon âme languit dans l'attente de Votre salut, CAPH et j'espère fermement en Votre parole. \EVERSE
\VERSE Mes yeux languissent après Votre parole, Vous disant: Quand me consolerez-Vous? \EVERSE
\VERSE Car je suis devenu comme une outre exposée à la gelée; je n'ai point oublié Vos ordonnances. \EVERSE
\VERSE Quel est le nombre des jours de Votre serviteur? Quand ferez-Vous justice de ceux qui me persécutent? \EVERSE
\VERSE Les méchants m'ont entretenu de choses vaines; mais ce n'était pas comme Votre loi. \EVERSE
\VERSE Tous Vos commandements sont la vérité même. Ils m'ont persécuté injustement; secourez-moi. \EVERSE
\VERSE Peu s'en est fallu qu'ils ne m'anéantissent dans le pays; mais je n'ai pas abandonné Vos commandements. \EVERSE
\VERSE Faites-moi vivre selon Votre miséricorde, et je garderai les témoignages de Votre bouche. \EVERSE
\VERSE LAMED Votre parole, Seigneur, subsiste éternellement dans le Ciel. \EVERSE
\VERSE Votre vérité se transmet de génération en génération; Vous avez affermi la terre, et elle demeure. \EVERSE
\VERSE Le jour subsiste par Votre ordre, car toutes choses Vous obéissent. \EVERSE
\VERSE Si je n'avais fait ma méditation de Votre loi, j'aurais peut-être péri dans mon humiliation. \EVERSE
\VERSE Je n'oublierai jamais Vos préceptes, car c'est par eux que Vous m'avez donné la vie. \EVERSE
\VERSE Je suis à Vous; sauvez-moi, parce que j'ai recherché Vos préceptes. \EVERSE
\VERSE Les pécheurs m'ont attendu pour me perdre; mais j'ai compris Vos enseignements. \EVERSE
\VERSE J'ai vu la fin de toute perfection; Votre loi a une étendue infinie. \EVERSE
\VERSE MEM Que j'aime Votre loi, Seigneur! Elle est tout le jour le sujet de ma méditation. \EVERSE
\VERSE Vous m'avez rendu plus sage que mes ennemis par Vos commandements, car ils sont perpétuellement avec moi. \EVERSE
\VERSE J'ai eu plus d'intelligence que tous ceux qui m'instruisaient, car Vos témoignages sont ma méditation. \EVERSE
\VERSE J'ai été plus intelligent que les vieillards, parce que j'ai recherché Vos commandements. \EVERSE
\VERSE J'ai détourné mes pieds de toute voie mauvaise, afin de garder Vos paroles. \EVERSE
\VERSE Je ne me suis point écarté de Vos jugements, parce que c'est Vous qui m'avez prescrit une loi. \EVERSE
\VERSE Que Vos paroles sont douces à mon palais! Elles le sont plus que le miel ne l'est à ma bouche. \EVERSE
\VERSE Vos préceptes m'ont donné l'intelligence; c'est pourquoi je hais toute voie d'iniquité.
}
\newcommand{\psalmcxviiiefr}{
\VERSE Votre parole est une lampe devant mes pas, NUN et une lumière sur mon sentier. \EVERSE
\VERSE J'ai juré et résolu de garder les jugements de Votre justice. \EVERSE
\VERSE J'ai été profondément humilié, Seigneur; faites-moi vivre selon Votre parole. \EVERSE
\VERSE Agréez, Seigneur, l'offrande volontaire de ma bouche, et enseignez-moi Vos jugements. \EVERSE
\VERSE Mon âme est toujours entre mes mains, et je n'ai pas oublié Votre loi. \EVERSE
\VERSE Les pécheurs m'ont tendu un piège, et je ne me suis point écarté de Vos commandements. \EVERSE
\VERSE J'ai acquis Vos enseignements comme un éternel héritage, car ils sont l'allégresse de mon coeur. \EVERSE
\VERSE J'ai porté mon coeur à pratiquer toujours Vos lois, à cause de la récompense. \EVERSE
\VERSE SAMECH J'ai haï les hommes injustes, et j'ai aimé Votre loi. \EVERSE
\VERSE Vous êtes mon défenseur et mon soutien, et j'ai mis toute mon espérance en Votre parole. \EVERSE
\VERSE Eloignez-Vous de moi, méchants, et j'étudierai les commandements de mon Dieu. \EVERSE
\VERSE Soutenez-moi selon Votre parole, et je vivrai; ne permettez pas que je sois confondu dans mon attente. \EVERSE
\VERSE Aidez-moi, et je serai sauvé, et je méditerai sans cesse Vos lois. \EVERSE
\VERSE Vous méprisez tous ceux qui s'éloignent de Vos jugements, car leur pensée est injuste. \EVERSE
\VERSE J'ai regardé comme des prévaricateurs tous les pécheurs de la terre; c'est pourquoi j'ai aimé Vos témoignages. \EVERSE
\VERSE Transpercez ma chair par Votre crainte; je redoute Vos jugements. \EVERSE
\VERSE AIN J'ai accompli le droit et la justice; ne me livrez pas à ceux qui me calomnient. \EVERSE
\VERSE Prenez Votre serviteur sous Votre garde pour son bien; que les superbes cessent de me calomnier. \EVERSE
\VERSE Mes yeux languissent dans l'attente de Votre salut, et après les promesses de Votre justice. \EVERSE
\VERSE Traitez Votre serviteur selon Votre miséricorde, et enseignez-moi Vos préceptes. \EVERSE
\VERSE Je suis Votre serviteur; donnez-moi l'intelligence, afin que je connaisse Vos témoignages. \EVERSE
\VERSE Il est temps que Vous agissiez, Seigneur; ils ont renversé Votre loi. \EVERSE
\VERSE C'est pourquoi j'ai aimé Vos commandements plus que l'or et la topaze. \EVERSE
\VERSE C'est pourquoi je me suis conformé à tous Vos commandements; j'ai haï toute voie injuste.
}
\newcommand{\psalmcxviiiffr}{
\VERSE Vos témoignages sont admirables; PHE aussi mon âme les étudie avec soin. \EVERSE
\VERSE L'explication de Vos paroles éclaire et donne l'intelligence aux petits. \EVERSE
\VERSE J'ai ouvert le bouche, et j'ai attiré l'air, parce que je désirais Vos commandements. \EVERSE
\VERSE Regardez-moi, et ayez pitié de moi; c'est justice envers ceux qui aiment Votre Nom. \EVERSE
\VERSE Conduisez mes pas selon Votre parole, et que nulle injustice ne domine sur moi. \EVERSE
\VERSE Délivrez-moi des calomnies des hommes, afin que je garde Vos commandements. \EVERSE
\VERSE Faites luire Votre visage sur Votre serviteur, et enseignez-moi Vos préceptes. \EVERSE
\VERSE Mes yeux ont répandu des ruisseaux de larmes, parce qu'on n'observe pas Votre loi. \EVERSE
\VERSE TSADE Vous êtes juste, Seigneur, et Votre jugement est droit. \EVERSE
\VERSE Les lois que Vous avez prescrites sont remplies de justice et de Votre vérité. \EVERSE
\VERSE Mon zèle m'a fait sécher de douleur, parce que mes ennemis ont oublié Vos paroles. \EVERSE
\VERSE Votre parole est tout enflammée, et Votre serviteur l'aime uniquement. \EVERSE
\VERSE Je suis jeune et méprisé, mais je n'ai point oublié Vos ordonnances. \EVERSE
\VERSE Votre justice est la justice éternelle, et Votre loi est la vérité même. \EVERSE
\VERSE La tribulation et l'angoisse m'ont saisi; Vos commandements sont ma méditation. \EVERSE
\VERSE Vos préceptes sont éternellement justes; donnez-moi l'intelligence, et je vivrai. \EVERSE
\VERSE COPH J'ai crié de tout mon coeur; exaucez-moi, Seigneur; je rechercherai Vos ordres. \EVERSE
\VERSE J'ai crié vers Vous; sauvez-moi, afin que je garde Vos commandements. \EVERSE
\VERSE J'ai devancé l'aurore, et j'ai crié vers Vous, parce que j'ai beaucoup espéré en Vos promesses. \EVERSE
\VERSE Mes yeux ont devancé l'aurore, se tournant vers Vous,afin de méditer Vos paroles. \EVERSE
\VERSE Ecoutez ma voix, Seigneur, selon Votre miséricorde, et faites-moi vivre selon Votre justice. \EVERSE
\VERSE Mes persécuteurs se sont approchés de l'iniquité, et ils se sont éloignés de Votre loi. \EVERSE
\VERSE Vous êtes proche, Seigneur, et toutes Vos voies sont la vérité même. \EVERSE
\VERSE J'ai reconnu dès le commencement que Vous avez établi à jamais Vos témoignages.
}
\newcommand{\psalmcxviiigfr}{
\VERSE Voyez mon humiliation, RESH et délivrez-moi, car je n'ai point oublié Votre loi. \EVERSE
\VERSE Jugez ma cause, et rachetez-moi; rendez-moi la vie selon Votre parole. \EVERSE
\VERSE Le salut est loin des pécheurs, parce qu'ils n'ont pas recherché Vos lois. \EVERSE
\VERSE Vos miséricordes sont nombreuses, Seigneur; rendez-moi la vie selon Votre jugement. \EVERSE
\VERSE Ceux qui me persécutent et qui m'affligent sont nombreux; mais je ne me suis pas détourné de Vos témoignages. \EVERSE
\VERSE J'ai vu les prévaricateurs, et je séchais de douleur, parce qu'ils n'ont point gardé Vos paroles. \EVERSE
\VERSE Voyez, Seigneur, combien j'ai aimé Vos préceptes; rendez-moi la vie par Votre miséricorde. \EVERSE
\VERSE La vérité est le principe de Vos paroles; tous les jugements de Votre justice sont éternels. \EVERSE
\VERSE Les princes m'ont persécuté sans raison, SCHIN et mon coeur n'a été effrayé que de Vos paroles. \EVERSE
\VERSE Je mets ma joie dans Vos ordres, comme celui qui a trouvé de riches dépouilles. \EVERSE
\VERSE J'ai haï l'iniquité, et je l'ai eue en horreur; mais j'ai aimé Votre loi. \EVERSE
\VERSE Sept fois le jour j'ai dit Votre louange, au sujet des jugements de Votre justice. \EVERSE
\VERSE Il y a une grande paix pour ceux qui aiment Votre loi, et rien n'est pour eux une occasion de chute. \EVERSE
\VERSE J'attendais Votre salut, Seigneur, et j'ai aimé Vos commandements. \EVERSE
\VERSE Mon âme a gardé Vos témoignages, et les a aimés ardemment. \EVERSE
\VERSE J'ai observé Vos commandements et Vos témoignages, car toutes mes voies sont devant Vous. \EVERSE
\VERSE Que ma prière s'approche jusqu'à Vous, THAV Seigneur; donnez-moi l'intelligence selon Votre parole. \EVERSE
\VERSE Que ma demande pénètre en Votre présence; délivrez-moi selon Votre promesse. \EVERSE
\VERSE Mes lèvres feront retentir un hymne à Votre gloire, lorsque Vous m'aurez enseigné Vos préceptes. \EVERSE
\VERSE Ma langue publiera Votre parole, car tous Vos commandements sont équitables. \EVERSE
\VERSE Que Votre main s'étende pour me sauver, car j'ai choisi Vos commandements. \EVERSE
\VERSE J'ai désiré Votre salut, Seigneur, et Votre loi est ma méditation. \EVERSE
\VERSE Mon âme vivra et Vous louera, et Vos jugements seront mon secours. \EVERSE
\VERSE J'ai erré comme une brebis qui s'est perdue; cherchez Votre serviteur, car je n'ai point oublié Vos commandements.
}
%%%%%%% PSAUME 119 %%%%%%%
\newcommand{\psalmcxixfr}{
\VERSE Cantique des degrés. Dans ma tribulation j'ai crié vers le Seigneur, et Il m'a exaucé. \EVERSE
\VERSE Seigneur, délivrez mon âme des lèvres injustes et de la langue trompeuse. \EVERSE
\VERSE Que te sera-t-il donné, et quel fruit te reviendra-t-il pour ta langue trompeuse? \EVERSE
\VERSE Les flèches aiguës du puissant, avec des charbons dévorants. \EVERSE
\VERSE Hélas! mon exil s'est prolongé. J'ai demeuré avec les habitants de Cédar;  \EVERSE
\VERSE mon âme a été longtemps exilée. \EVERSE
\VERSE Avec ceux qui haïssaient la paix, j'étais pacifique; quand je leur parlais, ils m'attaquaient sans sujet.
}
%%%%%%% PSAUME 120 %%%%%%%
\newcommand{\psalmcxxfr}{
\VERSE Cantique des degrés. J'ai élevé mes yeux vers les montagnes, d'où me viendra le secours. \EVERSE
\VERSE Mon secours vient du Seigneur, qui a fait le ciel et la terre. \EVERSE
\VERSE Qu'Il ne permette pas que ton pied chancelle, et que Celui qui te garde ne S'endorme point. \EVERSE
\VERSE Non, Il ne sommeille ni ne dort, Celui qui garde Israël. \EVERSE
\VERSE Le Seigneur te garde, le Seigneur te protège, Se tenant à ta droite. \EVERSE
\VERSE Pendant le jour le soleil ne te brûlera pas, ni la lune pendant la nuit. \EVERSE
\VERSE Le Seigneur te garde de tout mal; que le Seigneur garde ton âme. \EVERSE
\VERSE Que le Seigneur garde ton entrée et ta sortie, dès maintenant et à jamais.

}
%%%%%%% PSAUME 121 %%%%%%%
\newcommand{\psalmcxxifr}{
\VERSE Cantique des degrés. Je me suis réjoui de ce qui m'a été dit: Nous irons dans la maison du Seigneur. \EVERSE
\VERSE Nos pieds se sont arrêtés à tes portes, ô Jérusalem. \EVERSE
\VERSE Jérusalem, qui est bâtie comme une ville, dont toutes les parties se tiennent ensemble. \EVERSE
\VERSE Car c'est là que montaient les tribus, les tribus du Seigneur, selon le précepte donné à Israël, pour célébrer le Nom du Seigneur. \EVERSE
\VERSE Là ont été établis les trônes de la justice, les trônes de la maison de David. \EVERSE
\VERSE Demandez des grâces de paix pour Jérusalem, et que ceux qui t'aiment, ô cité sainte, soient dans l'abondance. \EVERSE
\VERSE Que la paix soit dans tes forteresses, et l'abondance dans tes tours. \EVERSE
\VERSE A cause de mes frères et de mes proches, j'ai demandé pour toi la paix. \EVERSE
\VERSE A cause de la maison du Seigneur notre Dieu, j'ai cherché pour toi le bonheur.

}
%%%%%%% PSAUME 122 %%%%%%%
\newcommand{\psalmcxxiifr}{
\VERSE Cantique des degrés. J'ai élevé mes yeux vers Vous, ô Dieu, qui habitez dans les cieux. \EVERSE
\VERSE Comme les yeux des serviteurs sont fixés sur les mains de leurs maîtres, et comme les yeux de la servante sont fixés sur les mains de sa maîtresse,ainsi nos yeux sont tournés vers le Seigneur notre Dieu, jusqu'à ce qu'Il ait pitié de nous. \EVERSE
\VERSE Ayez pitié de nous, Seigneur, ayez pitié de nous, car nous sommes rassasiés de mépris; \EVERSE
\VERSE car notre âme n'est que trop rassasiée d'être un sujet d'opprobre pour les riches, et de mépris pour les superbes.

}
%%%%%%% PSAUME 123 %%%%%%%
\newcommand{\psalmcxxiiifr}{
\VERSE Cantique des degrés. Si le Seigneur n'avait été avec nous, qu'Israël maintenant le dise,  \EVERSE
\VERSE si le Seigneur n'avait été avec nous, lorsque les hommes s'élevaient contre nous,  \EVERSE
\VERSE ils auraient pu nous dévorer tout vivants; lorsque leur fureur s'est irritée contre nous,  \EVERSE
\VERSE les eaux auraient pu nous engloutir. \EVERSE
\VERSE Notre âme a traversé le torrent; mais notre âme aurait pu pénétrer dans une eau infranchissable. \EVERSE
\VERSE Béni soit le Seigneur, qui ne nous a point donnés en proie à leurs dents. \EVERSE
\VERSE Notre âme s'est échappée, comme un passereau, du filet des chasseurs; le filet a été brisé, et nous avons été délivrés. \EVERSE
\VERSE Notre secours est dans le Nom du Seigneur, qui a fait le ciel et la terre.

}
%%%%%%% PSAUME 124 %%%%%%%
\newcommand{\psalmcxxivfr}{
\VERSE Cantique des degrés. Ceux qui se confient dans le Seigneur sont comme la montagne de Sion. Il ne sera jamais ébranlé, celui qui habite \EVERSE
\VERSE dans Jérusalem. Des montagnes sont autour d'elle; et le Seigneur est autour de Son peuple, dès maintenant et à jamais. \EVERSE
\VERSE Car le Seigneur ne laissera pas toujours la verge des pécheurs sur l'héritage des justes, de peur que les justes n'étendent leurs mains vers l'iniquité. \EVERSE
\VERSE Faites du bien aux bons, Seigneur, et à ceux dont le coeur est droit. \EVERSE
\VERSE Quant à ceux qui se détournent en des voies tortueuses, le Seigneur les emmènera avec ceux qui commettent l'iniquité. Que la paix soit sur Israël!

}
%%%%%%% PSAUME 125 %%%%%%%
\newcommand{\psalmcxxvfr}{
\VERSE Cantique des degrés. Quand le Seigneur ramena les captifs de Sion, nous fûmes tout à fait consolés. \EVERSE
\VERSE Alors notre bouche fut remplie de chants de joie, et notre langue de cris d'allégresse. Alors on disait parmi les nations: Le Seigneur a fait de grandes choses pour eux. \EVERSE
\VERSE Le Seigneur a fait pour nous de grandes choses; nous en avons été remplis de joie. \EVERSE
\VERSE Ramenez, Seigneur, nos captifs, comme un torrent dans le pays du midi. \EVERSE
\VERSE Ceux qui sèment dans les larmes moissonneront dans l'allégresse. \EVERSE
\VERSE Ils allaient et venaient en pleurant, tandis qu'ils jetaient leurs semences. Mais ils reviendront avec allégresse, chargés de leurs gerbes.

}
%%%%%%% PSAUME 126 %%%%%%%
\newcommand{\psalmcxxvifr}{
\VERSE Cantique des degrés, de Salomon. Si le Seigneur ne bâtit la maison, c'est en vain que travaillent ceux qui la bâtissent. Si le Seigneur ne garde la cité, c'est en vain que veille celui qui la garde. \EVERSE
\VERSE C'est en vain que vous vous levez avant le jour. Levez-vous après vous être reposés, vous qui mangez le pain de la douleur, car c'est Dieu qui donne le sommeil à Ses bien-aimés.  \EVERSE
\VERSE C'est un héritage du Seigneur que des enfants; le fruit des entrailles est une récompense. \EVERSE
\VERSE Comme les flèches dans la main d'un homme vaillant, ainsi sont les fils des hommes opprimés. \EVERSE
\VERSE Heureux l'homme qui en a rempli son désir. Il ne sera point confondu lorsqu'il parlera à ses ennemis à la porte de la ville.

}
%%%%%%% PSAUME 127 %%%%%%%
\newcommand{\psalmcxxviifr}{
\VERSE Cantique des degrés. Heureux tous ceux qui craignent le Seigneur, et qui marchent dans Ses voies. \EVERSE
\VERSE Parce que tu te nourriras des travaux de tes mains, tu es heureux et tu prospéreras. \EVERSE
\VERSE Ta femme sera comme une vigne féconde dans l'intérieur de ta maison. Tes enfants seront autour de ta table comme de jeunes plants d'olivier. \EVERSE
\VERSE C'est ainsi que sera béni l'homme qui craint le Seigneur. \EVERSE
\VERSE Que le Seigneur te bénisse de Sion, et puisses-tu voir la prospérité de Jérusalem tous les jours de ta vie! \EVERSE
\VERSE Et puisses-tu voir les enfants de tes enfants, et la paix en Israël!

}
%%%%%%% PSAUME 128 %%%%%%%
\newcommand{\psalmcxxviiifr}{
\VERSE Cantique des degrés. Ils m'ont souvent attaqué depuis ma jeunesse, qu'Israël le dise maintenant; \EVERSE
\VERSE ils m'ont souvent attaqué depuis ma jeunesse, mais ils n'ont pas prévalu contre moi. \EVERSE
\VERSE Les pécheurs ont travaillé sur mon dos; ils m'ont fait sentir longtemps leur injustice. \EVERSE
\VERSE Le Seigneur est juste, Il tranchera la tête des pécheurs.  \EVERSE
\VERSE Qu'ils soient confondus et qu'ils reculent en arrière, tous ceux qui haïssent Sion. \EVERSE
\VERSE Qu'ils deviennent comme l'herbe des toits, qui se sèche avant qu'on l'arrache; \EVERSE
\VERSE le moissonneur n'en remplit pas sa main, et celui qui ramasse les gerbes n'en remplit pas son sein. \EVERSE
\VERSE Et les passants n'ont point dit: Que la bénédiction du Seigneur soit sur nous. Nous vous bénissons au Nom du Seigneur.

}
%%%%%%% PSAUME 129 %%%%%%%
\newcommand{\psalmcxxixfr}{
\VERSE Cantique des degrés. Du fond des abîmes je crie vers Vous, Seigneur;  \EVERSE
\VERSE Seigneur, exaucez ma voix. Que Vos oreilles soient attentives à la voix de ma supplication. \EVERSE
\VERSE Si Vous examinez nos iniquités, Seigneur, Seigneur, qui subsistera devant Vous? \EVERSE
\VERSE Mais auprès de Vous est la miséricorde, et à cause de Votre loi j'ai espéré en Vous. Mon âme s'est soutenue par Sa parole;  \EVERSE
\VERSE Mon âme a espéré au Seigneur. \EVERSE
\VERSE Depuis la veille du matin jusqu'à la nuit, qu'Israël espère au Seigneur; \EVERSE
\VERSE car auprès du Seigneur est la miséricorde, et on trouve en Lui une rédemption abondante. \EVERSE
\VERSE Il rachètera lui-même Israël de toutes ses iniquités.

}
%%%%%%% PSAUME 130 %%%%%%%
\newcommand{\psalmcxxxfr}{
\VERSE Cantique des degrés, de David. Seigneur, mon coeur ne s'est pas enorgueilli, et mes yeux ne se sont point élevés. Je n'ai pas non plus recherché de grandes choses, ni ce qui est placé au-dessus de moi. \EVERSE
\VERSE Si je n'avais pas h'humbles sentiments, et si au contraire j'ai élevé mon âme, que mon âme soit traitée comme l'enfant que sa mère a sevré. \EVERSE
\VERSE Qu'Israël espère au Seigneur, dès maintenant et dans tous les siècles.

}
%%%%%%% PSAUME 131 %%%%%%%
\newcommand{\psalmcxxxifr}{
\VERSE Cantique des degrés. Souvenez-Vous, Seigneur de David et de toute sa douceur. \EVERSE
\VERSE Souvenez-Vous qu'il fait ce serment au Seigneur, ce voeu au Dieu de Jacob: \EVERSE
\VERSE Je n'entrerai pas dans ma maison, je ne monterai pas sur ma couche, \EVERSE
\VERSE je n'accorderai pas de sommeil à mes yeux, ni d'assoupissement à mes paupières, \EVERSE
\VERSE ni de repos à mes tempes, jusqu'à ce que je trouve un lieu pour le Seigneur, un tabernacle pour le Dieu de Jacob. \EVERSE
\VERSE Nous avons entendu dire que l'arche était à Ephrata; nous l'avons trouvée dans les champs de la forêt. \EVERSE
\VERSE Nous entrerons dans Son tabernacle; nous L'adorerons au lieu où Il a posé Ses pieds. \EVERSE
\VERSE Levez-Vous, Seigneur, pour entrer dans Votre repos, Vous et l'arche de Votre sainteté. \EVERSE
\VERSE Que Vos prêtres soient revêtus de justice, et que Vos saints tressaillent de joie. \EVERSE
\VERSE En considération de David Votre serviteur, ne repoussez pas la face de Votre Christ. \EVERSE
\VERSE Le Seigneur a fait à David un serment véridique, et Il ne le trompera point: J'établirai sur ton trône le fruit de ton sein. \EVERSE
\VERSE Si tes fils gardent Mon alliance et les préceptes que Je leur enseignerai, à tout jamais aussi leurs enfants seront assis sur ton trône. \EVERSE
\VERSE Car le Seigneur a choisi Sion; Il l'a choisie pour Sa demeure. \EVERSE
\VERSE C'est là pour toujours le lieu de Mon repos; J'y habiterai, car Je l'ai choisie. \EVERSE
\VERSE Je donnerai à sa veuve une bénédiction abondante; Je rassasierai de pain ses pauvres. \EVERSE
\VERSE Je revêtirai ses prêtres de salut, et ses saints seront ravis de joie. \EVERSE
\VERSE Là Je ferai paraître la puissance de David; J'ai préparé une lampe pour Mon Christ. \EVERSE
\VERSE Je couvrirai Ses ennemis de confusion; mais Ma sainteté fleurira sur Lui.

}
%%%%%%% PSAUME 132 %%%%%%%
\newcommand{\psalmcxxxiifr}{
\VERSE Cantique des degrés, de David. Ah! qu'il est bon et agréable pour des frères d'habiter ensemble! \EVERSE
\VERSE C'est comme le parfum répandu sur la tête, qui descend sur la barbe, la barbe d'Aaron; qui descend sur le bord de son vêtement.  \EVERSE
\VERSE C'est comme la rosée de l'Hermon, qui descend sur la montagne de Sion. Car c'est là que le Seigneur a envoyé Sa bénédiction et la vie à jamais.

}
%%%%%%% PSAUME 133 %%%%%%%
\newcommand{\psalmcxxxiiifr}{
\VERSE Cantique des degrés. Maintenant donc bénissez le Seigneur, vous tous, les serviteurs du Seigneur, qui demeurez dans la maison du Seigneur, dans les parvis de la maison de notre Dieu. \EVERSE
\VERSE Pendant les nuits levez vos mains vers le sanctuaire, et bénissez le Seigneur. \EVERSE
\VERSE Que le Seigneur te bénisse de Sion, Lui qui a fait le ciel et la terre.

}
%%%%%%% PSAUME 134 %%%%%%%
\newcommand{\psalmcxxxivfr}{
\VERSE Alleluia. Louez le Nom du Seigneur; louez le Seigneur, vous Ses serviteurs, \EVERSE
\VERSE qui demeurez dans la maison du Seigneur, dans les parvis de la maison de notre Dieu. \EVERSE
\VERSE Louez le Seigneur, car le Seigneur est bon; chantez à la gloire de Son Nom, car Il est doux. \EVERSE
\VERSE Car le Seigneur S'est choisi Jacob, et Israël pour Sa possession. \EVERSE
\VERSE Pour moi, j'ai reconnu que le Seigneur est grand, et que notre Dieu est au-dessus de tous les dieux. \EVERSE
\VERSE Tout ce qu'Il a voulu, le Seigneur l'a fait, au ciel et sur la terre, dans la mer et dans tous les abîmes. \EVERSE
\VERSE Il fait venir les nuées de l'extrémité de la terre; Il change les foudres en pluie. Il fait sortir les vents de Ses trésors.  \EVERSE
\VERSE Il a frappé les premiers-nés de l'Egypte, depuis l'homme jusqu'à la bête. \EVERSE
\VERSE Et Il a envoyé Ses signes et Ses prodiges au milieu de toi, ô Egypte, contre le Pharaon et contre tous ses serviteurs. \EVERSE
\VERSE Il a frappé des nations nombreuses, et Il a tué des rois puissants: \EVERSE
\VERSE Séhon, roi des Amorrhéens, et Og, roi de Basan, et tous les royaumes de Chanaan. \EVERSE
\VERSE Et Il a donné leur terre en héritage, en héritage à Israël Son peuple. \EVERSE
\VERSE Seigneur, Votre Nom subsistera éternellement; Seigneur, Votre souvenir s'étendra de génération en génération. \EVERSE
\VERSE Car le Seigneur jugera Son peuple, et Il aura pitié de Ses serviteurs. \EVERSE
\VERSE Les idoles des nations sont de l'argent et de l'or, et l'ouvrage des mains des hommes. \EVERSE
\VERSE Elles ont une bouche, et ne parlent pas; elles ont des yeux, et elles ne voient point. \EVERSE
\VERSE Elles ont des oreilles, et elles n'entendent pas; car il n'y a point de souffle dans leur bouche. \EVERSE
\VERSE Que ceux qui les font leur deviennent semblables, et tous ceux aussi qui se confient en elles. \EVERSE
\VERSE Maison d'Israël, bénissez le Seigneur; maison d'Aaron, bénissez le Seigneur. \EVERSE
\VERSE Maison de Lévi, bénissez le Seigneur; vous qui craignez le Seigneur, bénissez le Seigneur. \EVERSE
\VERSE Que le Seigneur soit béni de Sion, Lui qui habite à Jérusalem.

}
%%%%%%% PSAUME 135 %%%%%%%
\newcommand{\psalmcxxxvfr}{
\VERSE Alleluia. Célébrez le Seigneur, car Il est bon, car Sa miséricorde est éternelle. \EVERSE
\VERSE Célébrez le Dieu des dieux, car Sa miséricorde est éternelle. \EVERSE
\VERSE Célébrez le Seigneur des seigneurs, car Sa miséricorde est éternelle. \EVERSE
\VERSE C'est Lui qui fait seul de grands prodiges, car Sa miséricorde est éternelle. \EVERSE
\VERSE Il a fait les cieux avec intelligence, car Sa miséricorde est éternelle. \EVERSE
\VERSE Il a affermi la terre sur les eaux, car Sa miséricorde est éternelle. \EVERSE
\VERSE Il a fait les grands luminaires, car Sa miséricorde est éternelle: \EVERSE
\VERSE le soleil pour présider au jour, car Sa miséricorde est éternelle; \EVERSE
\VERSE la lune et les étoiles, pour présider à la nuit, car Sa miséricorde est éternelle. \EVERSE
\VERSE Il a frappé l'Egypte avec ses premiers-nés, car Sa miséricorde est éternelle. \EVERSE
\VERSE Il a fait sortir Israël du milieu d'eux, car Sa miséricorde est éternelle; \EVERSE
\VERSE avec une main puissante et un bras élevé, car Sa miséricorde est éternelle. \EVERSE
\VERSE Il a divisé en deux la mer Rouge, car Sa miséricorde est éternelle. \EVERSE
\VERSE Il a fait passer Israël par le milieu, car Sa miséricorde est éternelle. \EVERSE
\VERSE Il a renversé le Pharaon et son armée dans la mer Rouge, car Sa miséricorde est éternelle. \EVERSE
\VERSE Il a conduit Son peuple à travers le désert, car Sa miséricorde est éternelle. \EVERSE
\VERSE Il a frappé les grands rois, car Sa miséricorde est éternelle. \EVERSE
\VERSE Il a fait mourir les rois puissants, car Sa miséricorde est éternelle: \EVERSE
\VERSE Séhon, roi des Amorrhéens, car Sa miséricorde est éternelle; \EVERSE
\VERSE et Og, roi de Basan, car Sa miséricorde est éternelle. \EVERSE
\VERSE Et Il a donné leur terre en héritage, car Sa miséricorde est éternelle, \EVERSE
\VERSE en héritage à Israël Son serviteur, car Sa miséricorde est éternelle. \EVERSE
\VERSE Il S'est souvenu de nous dans notre humiliation, car Sa miséricorde est éternelle; \EVERSE
\VERSE et Il nous a délivrés de nos ennemis, car Sa miséricorde est éternelle. \EVERSE
\VERSE Il donne la nourriture à toute chair, car Sa miséricorde est éternelle. \EVERSE
\VERSE Célébrez le Dieu du Ciel, car Sa miséricorde est éternelle. Célébrez le Seigneur des seigneurs, car Sa miséricorde est éternelle.

}
%%%%%%% PSAUME 136 %%%%%%%
\newcommand{\psalmcxxxvifr}{
\VERSE Psaume de David, par Jérémie. Au bord des fleuves de Babylone nous nous sommes assis, et nous avons pleuré, en nous souvenant de Sion. \EVERSE
\VERSE Aux saules qui étaient là nous avons suspendu nos instruments. \EVERSE
\VERSE Car ceux qui nous avaient emmenés captifs nous demandaient de chanter des cantiques; ceux qui nous avaient enlevés disaient: Chantez-nous quelqu'un des hymnes de Sion. \EVERSE
\VERSE Comment chanterons-nous le cantique du Seigneur dans une terre étrangère? \EVERSE
\VERSE Si je t'oublie, ô Jérusalem, que ma main droite soit mise en oubli. \EVERSE
\VERSE Que ma langue s'attache à mon palais, si je ne me souviens point de toi, si je ne place pas Jérusalem au premier rang de mes joies. \EVERSE
\VERSE Souvenez-Vous, Seigneur, des enfants d'Edom, qui, au jour de la ruine de Jérusalem, disaient: Exterminez, exterminez jusqu'à ses fondements. \EVERSE
\VERSE Malheur à toi, fille de Babylone! Heureux celui qui te rendra le mal que tu nous as fait. \EVERSE
\VERSE Heureux celui qui saisira tes petits enfants, et les brisera contre la pierre.

}
%%%%%%% PSAUME 137 %%%%%%%
\newcommand{\psalmcxxxviifr}{
\VERSE De David. Je Vous célébrerai, Seigneur, de tout mon coeur, parce que Vous avez écouté les paroles de ma bouche. Je Vous chanterai des hymnes en présence des Anges;  \EVERSE
\VERSE j'adorerai dans Votre saint temple, et je célébrerai Votre Nom, à cause de Votre miséricorde et de Votre vérité, car Vous avez glorifié Votre saint Nom au-dessus de tout. \EVERSE
\VERSE En quelque jour que je Vous invoque, exaucez-moi; Vous augmenterez la force de mon âme. \EVERSE
\VERSE Que tous les rois de la terre Vous célébrent, Seigneur, parce qu'ils ont entendu toutes les paroles de Votre bouche. \EVERSE
\VERSE Et qu'ils chantent les voies du Seigneur, car la gloire du Seigneur est grande. \EVERSE
\VERSE Car le Seigneur est très élevé, et Il regarde les choses basses, et de loin Il connaît les choses hautes. \EVERSE
\VERSE Si je marche au milieu de la tribulation, Vous me rendrez la vie; Vous avez étendu Votre main contre la fureur de mes ennemis, et Votre droite m'a sauvé. \EVERSE
\VERSE Le Seigneur me vengera. Seigneur, Votre miséricorde est éternelle; ne méprisez pas les oeuvres de Vos mains.

}
%%%%%%% PSAUME 138 %%%%%%%
\newcommand{\psalmcxxxviiifr}{
\VERSE Pour la fin, psaume de David. Seigneur, Vous m'avez sondé et Vous me connaissez;  \EVERSE
\VERSE Vous savez quand je m'assieds et quand je me lève. \EVERSE
\VERSE Vous avez discerné de loin mes pensées; Vous avez remarqué mon sentier et mes démarches, \EVERSE
\VERSE et Vous avez prévu toutes mes voies; et avant même qu'une parole soit sur ma langue, Vous la savez. \EVERSE
\VERSE Voici, Seigneur, que Vous connaissez toutes choses, les nouvelles et les anciennes. C'est Vous qui m'avez formé, et Vous avez mis Votre main sur moi. \EVERSE
\VERSE Votre science merveilleuse est au-dessus de moi; elle me surpasse, et je ne saurais l'atteindre. \EVERSE
\VERSE Où irai-je pour me dérober à Votre esprit, et où m'enfuirai-je de devant Votre face? \EVERSE
\VERSE Si je monte au Ciel, Vous y êtes; si je descends dans l'enfer, Vous y êtes présent. \EVERSE
\VERSE Si je prends des ailes dès l'aurore, et que j'aille habiter aux extrémités de la mer, \EVERSE
\VERSE c'est Votre main qui m'y conduira, et Votre droite me saisira. \EVERSE
\VERSE Et j'ai dit: Peut-être que les ténèbres me couvriront; mais la nuit même devient ma lumière dans mes délices. \EVERSE
\VERSE Car les ténèbres n'ont pas d'obscurité pour Vous; la nuit brille comme le jour, et ses ténèbres sont comme la lumière du jour. \EVERSE
\VERSE Car Vous avez formé mes reins; Vous m'avez reçu dès le sein de ma mère. \EVERSE
\VERSE Je Vous louerai de ce que Votre grandeur a éclaté d'une manière étonnante; Vos oeuvres sont admirables, et mon âme en est toute pénétrée. \EVERSE
\VERSE Mes os ne Vous sont point cachés, à Vous qui les avez faits dans le secret; non plus que ma substance, formée comme au fond de la terre. \EVERSE
\VERSE Vos yeux m'ont vu lorsque j'étais encore informe, et tous les hommes sont écrits dans Votre livre. Vous déterminez leurs jours avant qu'aucun d'eux n'existe. \EVERSE
\VERSE O Dieu, que Vos amis sont singulièrement honorés à mes yeux! Leur empire s'est extraordinairement affermi. \EVERSE
\VERSE Si j'entreprends de les compter, leur nombre surpasse celui du sable de la mer. Et quand je m'éveille, je suis encore avec Vous. \EVERSE
\VERSE O Dieu, si Vous tuez les pécheurs, hommes de sang, éloignez-vous de moi; \EVERSE
\VERSE vous qui dites dans votre pensée: C'est en vain, Seigneur, que les justes posséderont Vos villes. \EVERSE
\VERSE Seigneur, n'ai-je pas haï ceux qui Vous haïssaient? et n'ai-je pas séché d'horreur à cause de Vos ennemis? \EVERSE
\VERSE Je les haïssais d'une haine parfaite, et ils sont devenus mes ennemis. \EVERSE
\VERSE O Dieu, éprouvez-moi, et connaissez mon coeur; interrogez-moi, et connaissez mes sentiers. \EVERSE
\VERSE Voyez si la voie de l'iniquité se trouve en moi, et conduisez-moi dans la voie éternelle.

}
%%%%%%% PSAUME 139 %%%%%%%
\newcommand{\psalmcxxxixfr}{
\VERSE Pour la fin, psaume de David. \EVERSE
\VERSE Délivrez-moi Seigneur, de l'homme méchant; délivrez-moi de l'homme injuste. \EVERSE
\VERSE Ils méditent l'iniquité dans leur coeur; tous les jours ils entreprennent des combats. \EVERSE
\VERSE Ils ont aiguisé leur langue comme celle du serpent; le venin des aspics est sous leurs lèvres. \EVERSE
\VERSE Seigneur, préservez-moi de la main du pécheur, et délivrez-moi des hommes injustes, qui ne pensent qu'à me renverser.  \EVERSE
\VERSE Les superbes m'ont dressé des pièges en secret, et ils ont tendu des filets pour me prendre; près du chemin ils ont mis de quoi me faire tomber. \EVERSE
\VERSE J'ai dit au Seigneur: Vous êtes mon Dieu; exaucez, Seigneur, la voix de ma supplication. \EVERSE
\VERSE Seigneur, Seigneur, qui êtes la force de mon salut, Vous avez mis ma tête à couvert au jour du combat. \EVERSE
\VERSE Seigneur, ne me livrez pas au pécheur contre mon désir; ils ont formé des desseins contre moi; ne m'abandonnez pas, de peur qu'ils ne s'en glorifient. \EVERSE
\VERSE Sur la tête de ceux qui m'environnent retombera l'iniquité de leurs lèvres. \EVERSE
\VERSE Des charbons ardents tomberont sur eux; Vous les précipiterez dans le feu; ils ne pouront subsister dans leurs misères. \EVERSE
\VERSE L'homme qui se laisse emporter par sa langue ne prospérera point sur la terre; les maux accableront l'homme injuste, de manière à le perdre. \EVERSE
\VERSE Je sais que le Seigneur fera justice à l'indigent, et qu'Il vengera les pauvres. \EVERSE
\VERSE Mais les justes célébreront Votre Nom, et les hommes droits habiteront devant Votre visage.

}
%%%%%%% PSAUME 140 %%%%%%%
\newcommand{\psalmcxlfr}{
\VERSE Psaume de David. Seigneur, j'ai crié vers Vous, exaucez-moi; écoutez ma prière, lorsque je crierai vers Vous. \EVERSE
\VERSE Que ma prière s'élève devant Vous comme l'encens; que l'élévation de mes mains Vous soit comme le sacrifice du soir. \EVERSE
\VERSE Mettez, Seigneur, une garde à ma bouche, et une porte de défense à mes lèvres. \EVERSE
\VERSE Ne laissez pas mon coeur se livrer à des paroles de malice, pour chercher des excuses au péché, comme les hommes qui commettent l'iniquité; et je n'aurai aucune part à leurs délices. \EVERSE
\VERSE Que le juste me reprenne et me corrige avec charité; mais l'huile du pécheur ne parfumera point ma tête, car j'opposerai encore ma prière à tout ce qui flatte leur cupidité.  \EVERSE
\VERSE Leurs juges ont été précipités le long du rocher. Ils écouteront enfin mes paroles, car elles sont puissantes.  \EVERSE
\VERSE De même que la motte de terre est renversée sur le sol, nos os ont été dispersés auprès du sépulcre.  \EVERSE
\VERSE Mais, Seigneur, Seigneur, mes yeux s'élèvent vers Vous; j'ai espéré en Vous, ne m'ôtez pas la vie. \EVERSE
\VERSE Gardez-moi du piège qu'ils m'ont dressé, et des embûches de ceux qui commettent l'iniquité. \EVERSE
\VERSE Les pécheurs tomberont dans le filet; pour moi, je suis seul, jusqu'à ce que je passe.

}
%%%%%%% PSAUME 141 %%%%%%%
\newcommand{\psalmcxlifr}{
\VERSE Instruction de David, lorsqu'il était dans la caverne, prière. \EVERSE
\VERSE De ma voix j'ai crié vers le Seigneur; de ma voix j'ai supplié le Seigneur. \EVERSE
\VERSE Je répands ma prière en Sa présence, et j'expose devant Lui ma tribulation. \EVERSE
\VERSE Quand mon espoir est défaillant en moi, Vous connaissez mes voies, Seigneur. Dans la voie où je marchais ils m'ont tendu un piège en secret. \EVERSE
\VERSE Je considérais à ma droite, et je regardais, et il n'y avait personne qui me connût. Tout moyen de m'enfuir m'est ôté, et nul ne cherche à sauver ma vie. \EVERSE
\VERSE J'ai crié vers Vous, Seigneur; j'ai dit: Vous êtes mon espérance, et mon partage dans la terre des vivants. \EVERSE
\VERSE Soyez attentif à ma prière, car je suis extrêmement humilié. Délivrez-moi de ceux qui me persécutent, parce qu'ils sont devenus plus forts que moi. \EVERSE
\VERSE Tirez mon âme de cette prison, afin que je célèbre Votre Nom. Les justes sont dans l'attente, jusqu'à ce que Vous me rendiez justice.

}
%%%%%%% PSAUME 142 %%%%%%%
\newcommand{\psalmcxliifr}{
\VERSE Psaume de David, lorsque son fils Absalon le poursuivait. Seigneur, exaucez ma prière; prêtez l'oreille à ma supplication selon Votre vérité; exaucez-moi selon Votre justice. \EVERSE
\VERSE Et n'entrez pas en jugement avec Votre serviteur, parce que nul homme vivant ne sera trouvé juste devant Vous. \EVERSE
\VERSE Car l'ennemi a poursuivi mon âme; il a humilié ma vie jusqu'à terre. Il m'a placé dans les lieux obscurs, comme ceux qui sont morts depuis longtemps.  \EVERSE
\VERSE Mon esprit s'est replié sur moi dans son angoisse; mon coeur a été troublé au dedans de moi. \EVERSE
\VERSE Je me suis souvenu des jours anciens; j'ai médité sur toutes Vos oeuvres; j'ai médité sur les ouvrages de Vos mains. \EVERSE
\VERSE J'ai étendu mes mains vers Vous; mon âme est devant Vous comme une terre sans eau. \EVERSE
\VERSE Hâtez-Vous, Seigneur, de m'exaucer; mon esprit est tombé en défaillance. Ne détournez pas de moi Votre visage, de peur que je ne sois semblable à ceux qui descendent dans la fosse. \EVERSE
\VERSE Faites-moi sentir dès le matin Votre miséricorde, parce que j'ai espéré en Vous. Faites-moi connaître la voie où je dois marcher, parce que j'ai élevé mon âme vers Vous. \EVERSE
\VERSE Délivrez-moi de mes ennemis, Seigneur, je me réfugie auprès de Vous.  \EVERSE
\VERSE Enseignez-moi à faire Votre volonté, parce que Vous êtes mon Dieu. Votre bon esprit me conduira dans une terre droite et unie.  \EVERSE
\VERSE Seigneur, à cause de Votre Nom Vous me ferez vivre dans Votre justice. Vous ferez sortir mon âme de la tribulation,  \EVERSE
\VERSE et, dans Votre miséricorde, Vous détruirez mes ennemis. et Vous perdrez tous ceux qui persécutent mon âme, car je suis Votre serviteur.

}
%%%%%%% PSAUME 143 %%%%%%%
\newcommand{\psalmcxliiifr}{
\VERSE Psaume de David, contre Goliath. Béni soit le Seigneur mon Dieu, qui enseigne à mes mains le combat, et à mes doigts la guerre. \EVERSE
\VERSE Il est ma miséricorde et mon refuge, mon défenseur et mon libérateur. Il est mon protecteur, et c'est en Lui que j'espère; c'est Lui qui assujettit mon peuple sous moi. \EVERSE
\VERSE Seigneur, qu'est-ce que l'homme, pour que Vous Vous soyez fait connaître à Lui? ou le fils de l'homme, pour que Vous preniez garde à lui? \EVERSE
\VERSE L'homme est devenu semblable au néant; ses jours passent comme l'ombre. \EVERSE
\VERSE Seigneur, abaissez Vos cieux et descendez; touchez les montagnes, et elles seront fumantes. \EVERSE
\VERSE Faites briller Vos éclairs, et Vous les disperserez; lancez Vos flèches, et Vous les mettrez en déroute. \EVERSE
\VERSE Etendez Votre main d'en haut, délivrez-moi, et sauvez-moi des grandes eaux, de la main des fils des étrangers, \EVERSE
\VERSE dont la bouche a proféré la vanité, et dont la droite est une droite d'iniquité. \EVERSE
\VERSE O Dieu, je Vous chanterai un cantique nouveau; je Vous célébrerai sur la lyre à dix cordes. \EVERSE
\VERSE O Vous qui procurez le salut aux rois, qui avez sauvé David, Votre serviteur, du glaive meurtrier. \EVERSE
\VERSE Délivrez-moi et retirez-moi d'entre les mains des fils des étrangers, dont la bouche a proféré la vanité, et dont la droite est une droite d'iniquité. \EVERSE
\VERSE Leurs fils sont comme de nouvelles plantes dans leur jeunesse. Leurs filles sont parées et ornées à la manière d'un temple. \EVERSE
\VERSE Leurs greniers sont remplis, et débordent de l'un dans l'autre. Leurs brebis sont fécondes et innombrables quand elles vont aux pâturages. \EVERSE
\VERSE Leurs génisses sont grasses. Il n'y a pas de brèche ni d'ouverture dans leurs murailles, et jamais un cri sur leurs places publiques. \EVERSE
\VERSE Ils ont proclamé heureux le peuple qui jouit de ces biens; heureux le peuple qui a le Seigneur pour son Dieu.

}
%%%%%%% PSAUME 144 %%%%%%%
\newcommand{\psalmcxlivafr}{
\VERSE Louange de David. Je Vous exalterai, ô Dieu mon roi, et je bénirai Votre Nom à jamais et dans les siècles des siècles. \EVERSE
\VERSE Chaque jour je Vous bénirai, et je louerai Votre Nom à jamais, et dans les siècles des siècles. \EVERSE
\VERSE Le Seigneur est grand et très digne de louange, et Sa grandeur n'a pas de bornes. \EVERSE
\VERSE Chaque génération louera Vos oeuvres et publiera Votre puissance. \EVERSE
\VERSE On parlera de la magnificence glorieuse de Votre sainteté, et on racontera Vos merveilles. \EVERSE
\VERSE On dira quelle est la puissance de Vos oeuvres terribles, et on racontera Votre grandeur. \EVERSE
\VERSE On proclamera le souvenir de Votre immense bonté, et on se réjouira de Votre justice. \EVERSE
\VERSE Le Seigneur est clément et miséricordieux, patient et tout à fait miséricordieux. \EVERSE
\VERSE Le Seigneur est bon envers tous, et Ses miséricordes s'étendent sur toutes Ses oeuvres.
}
\newcommand{\psalmcxlivbfr}{
\VERSE Que toutes Vos oeuvres Vous célèbrent, Seigneur, et que Vos saints Vous bénissent. \EVERSE
\VERSE Ils diront la gloire de Votre règne, et ils parleront de Votre puissance; \EVERSE
\VERSE afin de faire connaître aux enfants des hommes Votre puissance, et la glorieuse magnificence de Votre règne. \EVERSE
\VERSE Votre règne est un règne de tous les siècles, et Votre empire s'étend de génération en génération. Le Seigneur est fidèle dans toutes Ses paroles, et saint dans toutes Ses oeuvres. \EVERSE
\VERSE Le Seigneur soutient tous ceux qui tombent, et Il relève tous ceux qui sont brisés. \EVERSE
\VERSE Les yeux de tous, Seigneur, attendent tournés vers Vous, et Vous leur donnez leur nourriture en son temps. \EVERSE
\VERSE Vous ouvrez Votre main, et Vous comblez de bénédictions tout ce qui a vie. \EVERSE
\VERSE Le Seigneur est juste dans toutes Ses voies, et saint dans toutes Ses oeuvres. \EVERSE
\VERSE Le Seigneur est près de tous ceux qui L'invoquent, de tous ceux qui L'invoquent avec sincérité. \EVERSE
\VERSE Il fera la volonté de ceux qui Le craignent; Il exaucera leurs prières, et Il les sauvera. \EVERSE
\VERSE Le Seigneur garde tous ceux qui L'aiment, et Il perdra tous les pécheurs. \EVERSE
\VERSE Ma bouche publiera la louange du Seigneur. Et que toute chair bénisse Son saint Nom à jamais, et dans les siècles des siècles.

}
%%%%%%% PSAUME 145 %%%%%%%
\newcommand{\psalmcxlvfr}{
\VERSE Alleluia, d'Aggée et de Zacharie. \EVERSE
\VERSE O mon âme, loue le Seigneur. Je louerai le Seigneur pendant ma vie; je chanterai mon Dieu tant que je serai. Ne mettez pas votre confiance dans les princes,  \EVERSE
\VERSE ni dans les enfants des hommes, qui ne peuvent sauver. \EVERSE
\VERSE Leur âme se retirera, et ils retourneront à leur poussière; en ce jour toutes leurs pensées périront. \EVERSE
\VERSE Heureux celui dont le Dieu de Jacob est le protecteur, et dont l'espérance est dans le Seigneur son Dieu,  \EVERSE
\VERSE qui a fait le Ciel et la terre, la mer et tout ce qu'ils contiennent. \EVERSE
\VERSE Il garde à jamais la vérité de Ses promesses, Il fait justice aux opprimés, Il donne la nourriture à ceux qui ont faim. Le Seigneur délivre les captifs;  \EVERSE
\VERSE le Seigneur éclaire les aveugles. Le Seigneur relève ceux qui sont brisés; le Seigneur aime les justes. \EVERSE
\VERSE Le Seigneur protège les étrangers; Il soutient l'orphelin et la veuve, et Il détruira les voies des pécheurs. \EVERSE
\VERSE Le Seigneur règnera à jamais; ton Dieu, Sion, régnera de génération en génération.

}
%%%%%%% PSAUME 146 %%%%%%%
\newcommand{\psalmcxlvifr}{
\VERSE Alleluia. Louez le Seigneur, car Il est bon de Le chanter; que la louange soit agréable à notre Dieu et digne de Lui. \EVERSE
\VERSE C'est le Seigneur qui bâtit Jérusalem, et qui doit rassembler les dispersés d'Israël. \EVERSE
\VERSE Il guérit ceux dont le coeur est brisé, et Il bande leurs plaies; \EVERSE
\VERSE Il compte la multitude des étoiles, et Il leur donne des noms à toutes. \EVERSE
\VERSE Notre Seigneur est grand, et Sa puissance est grande, et Sa sagesse n'a point de bornes. \EVERSE
\VERSE Le Seigneur protège ceux qui sont doux; mais Il abaisse les pécheurs jusqu'à terre. \EVERSE
\VERSE Chantez au Seigneur une action de grâces; célébrez notre Dieu sur la harpe. \EVERSE
\VERSE C'est Lui qui couvre le ciel de nuages, et qui prépare la pluie pour la terre; qui fait croître l'herbe sur les montagnes, et les plantes pour l'usage des hommes; \EVERSE
\VERSE qui donne leur nourriture aux bêtes, et aux petits des corbeaux qui crient vers Lui. \EVERSE
\VERSE Ce n'est pas dans la force du cheval qu'Il Se complaît, et \EVERSE
\VERSE Il ne met pas Son plaisir dans les jambes de l'homme. \EVERSE
\VERSE Le Seigneur met Son plaisir en ceux qui Le craignent, et en ceux qui espèrent en Sa miséricorde.

}
%%%%%%% PSAUME 147 %%%%%%%
\newcommand{\psalmcxlviifr}{
\VERSE Alleluia, Jérusalem, loue le Seigneur; loue ton Dieu, ô Sion. \EVERSE
\VERSE Car Il a consolidé les verrous de tes portes; Il a béni tes fils au milieu de toi. \EVERSE
\VERSE Il a établi la paix sur tes frontières, et Il te rassasie de la fleur du froment. \EVERSE
\VERSE Il envoie Ses ordres à la terre, et Sa parole court avec vitesse. \EVERSE
\VERSE Il fait tomber la neige comme de la laine; Il répand la gelée blanche comme de la cendre. \EVERSE
\VERSE Il lance Sa glace par morceaux; qui peut résister devant Son froid? \EVERSE
\VERSE Il envoie Sa parole et Il fond ces glaces; Son vent souffle, et les eaux coulent. \EVERSE
\VERSE Il annonce Sa parole à Jacob, Ses jugements et Ses préceptes à Israël. \EVERSE
\VERSE Il n'a pas agi de même pour toutes les nations, et Il ne leur a pas manifesté Ses préceptes.
}
%%%%%%% PSAUME 148 %%%%%%%
\newcommand{\psalmcxlviiifr}{
\VERSE Alleluia. Louez le Seigneur du haut des cieux; louez-Le dans les hauteurs. \EVERSE
\VERSE Louez-Le tous, Vous Ses Anges; louez-Le, toutes Ses puissances. \EVERSE
\VERSE Louez-Le, soleil et lune; louez-Le toutes, étoiles et lumière. \EVERSE
\VERSE Louez-Le, cieux des cieux, et que toutes les eaux qui sont au-dessus des cieux \EVERSE
\VERSE louent le Nom du Seigneur. Car Il a parlé, et ces choses ont été faites; Il a commandé, et elles ont été créées. \EVERSE
\VERSE Il les a établies à jamais dans les siècles des siècles; Il leur a prescrit une loi qui ne sera pas violée. \EVERSE
\VERSE Louez le Seigneur de dessus la terre: dragons, et vous tous, abîmes, \EVERSE
\VERSE feu, grêle, neige, glace, vents des tempêtes, qui exécutez Sa parole; \EVERSE
\VERSE montagnes avec toutes les collines, arbres à fruit et tous les cèdres, \EVERSE
\VERSE bêtes sauvages et tous les troupeaux, serpents et oiseaux ailés. \EVERSE
\VERSE Que les rois de la terre et tous les peuples, que les princes et tous les juges de la terre, \EVERSE
\VERSE que les jeunes gens et les jeunes filles, les vieillards et les enfants louent le Nom du Seigneur,  \EVERSE
\VERSE parce qu'il n'y a que Lui dont le Nom est élevé. \EVERSE
\VERSE Sa louange est au-dessus du ciel et de la terre; Il a élevé la puissance de Son peuple. Qu'Il soit loué par tous Ses saints, par les enfants d'Israël, le peuple qui s'approche de Lui. Alleluia.

}
%%%%%%% PSAUME 149 %%%%%%%
\newcommand{\psalmcxlixfr}{
\VERSE Alleluia. Chantez au Seigneur un cantique nouveau; que Sa louange retentisse dans l'assemblée des saints. \EVERSE
\VERSE Qu'Israël se réjouisse en Celui qui l'a créé, et que les enfants de Sion tressaillent de joie en leur Roi. \EVERSE
\VERSE Qu'ils louent Son Nom avec des danses; qu'ils le célèbrent avec le tambour et la harpe. \EVERSE
\VERSE Car le Seigneur Se complaît dans Son peuple, et Il exaltera ceux qui sont doux et les sauvera. \EVERSE
\VERSE Les saints tressailliront dans la gloire; ils se réjouiront sur leurs couches. \EVERSE
\VERSE Les louanges de Dieu seront dans leur bouche, et des glaives à deux tranchants dans leurs mains, \EVERSE
\VERSE pour exercer la vengeance parmi les nations, le châtiment parmi les peuples; \EVERSE
\VERSE pour lier leurs rois avec des entraves, et leurs princes avec des chaînes de fer, \EVERSE
\VERSE et pour exécuter contre eux l'arrêt qui est écrit. Telle est la gloire réservée à tous Ses saints. Alleluia.

}
%%%%%%% PSAUME 150 %%%%%%%
\newcommand{\psalmclfr}{
\VERSE Alleluia, louez le Seigneur dans Son sanctuaire; louez-Le dans le firmament de Sa puissance. \EVERSE
\VERSE Louez-Le pour Ses actes éclatants; louez-Le selon l'immensité de Sa grandeur. \EVERSE
\VERSE Louez-Le au son de la trompette; louez-Le sur le luth et la harpe. \EVERSE
\VERSE Louez-Le avec le tambourin et en choeur; louez-Le avec les instruments à cordes et avec l'orgue. \EVERSE
\VERSE Louez-Le avec des cymbales retentissantes; louez-Le avec des cymables d'allégresse.  \EVERSE
\VERSE Que tout ce qui respire loue le Seigneur. Alleluia.

}

%%% Ad Matutinum
 
% R
\newcommand{\regemventurumdominum}{\emph{Ant}. Regem venturum Dominum * Venite adoremus}
\newcommand{\regemventurumdominumfr}{\emph{Ant}. Le Rois, le Seigneur qui vient * Venez, adorons-le !}
%%% Ad Matutinum

% V

\newcommand{\verbumsupernumprodiens}{
Verbum supernum prodiens,\\
E Patris aeterni sinu\\
Qui natus orbi subvenis,\\
Labente cursu temporis ;\\
\\
Illumina nunc pectora,\\
Tuoque amore concrema,\\
Ut cor caduca deserens\\
Caeli voluptas impleat.\\
\\
Ut, cum tribunal Judicis\\
Damnabit igni noxios,\\
Et vox amica debitum\\
Vocabit ad caelum pios.\\
\\
Non esca flammarum nigros\\
Volvamur inter turbines,\\
Vultu Dei sed compotes\\
Caeli fruamur gaudiis.\\
\\
Patri simulque Filio,\\
Tibique Sancte Spiritus,\\
Sicut fuit, sit jugiter\\
Saeclum per omne gloria.\\
Amen.
}

\newcommand{\verbumsupernumprodiensfr}{
Verbe Suprême qui sortez, / Du sein éternel du Père, / Et qui naissez au déclin des temps, / Venez au secours de l'univers \\
\\
Illuminez en ce moment nos cœurs / Embrasez-les de votre amour / Pour que, détachés des biens périssables / Ils soient remplis d'une joie céleste.\\
\\
Afin qu'au jour où le Juge, de son Tribunal / Condamnera les coupables aux flammes / Et, d'une voix amie // Conviera les justes au Ciel.\\
\\
Nous ne soyons pas livrés en proie aux flammes, / Nous ne soyons pas lancés dans un tourbillon ; / Mais que, favorisés de la vue de Dieu, / Nous jouissions des délices du Paradis.\\
\\
Au Père ainsi qu'au Fils, / Et à Vous, Esprit-Saint, / Soient à jamais dans tous les siècles, / Comme il fut toujours, gloire et honneur,\\
Amen.
}
\newcommand{\IAVi}{Dominus ac Redemptor noster paratos nos invenire desiderans, senescentem mundum quae mala sequantur denuntiat, ut nos ab ejus amore compescat. Appropinquantem ejus terminum quantae percussiones praeveniant, innotescit: ut si Deum metuere in tranquillitate nolumus, saltem vicinum ejus judicium vel percussionibus attriti timeamus.}
\newcommand{\excitaquaesumus}{Excita, quaesumus Domine\index{orationes}{Excita quaesumus Domine}, potentiam tuam, et veni: ut ab imminentibus peccatorum nostrorum periculis, te mereamur protegente eripi, te liberante salvari: \emph{Qui vivis ...}}
\newcommand{\excitaquaesumusfr}{Faites paraître votre puissance, Seigneur, et venez ; afin que nous méritions d'être attachés, par votre secours, aux imminents périls où nos péchés nous engagent, et d'en être sauvés par votre vertu libératrice ; vous qui, étant Dieu, vivez.}
\input{sources/responsoria}
\newcommand{\deusinadiutorium}{\Versicle. Deus, in adjutorium meum intende.\\ \Respons. Domine, ad adjuvandum me festina.}
\newcommand{\deusinadiutoriumfr}{\Versicle. Dieu, venez à mon aide.\\ \Respons. Seigneur, hâtez-vous de me secourir.}


%%% COURANT %%%

\newcommand{\dominelabiamea}{
\BEGINPARA \BLEFT { \fontlat{
\Versicle. Dómine lábia  mea apéries.\\ 
\Respons. Et os meum annuntiábit laudem tuam.\\
\Versicle. Deus  in adiutórium meum inténde.\\
\Respons. Dómine, ad adiuvándum me festína.\\
\Versicle. Glória Patri, et Fílio, * et Spirítui Sancto.\\
\Respons. Sicut erat in princípio, et nunc, et semper, * et in
sæcula sæculórum. Amen.\\
 Alleluia\\}
\ELEFT } \BRIGHT { \fontfr{
\Versicle. Seigneur, ouvrez mes lèvres.\\
\Respons. Et ma bouche annoncera votre louange.\\
\Versicle. Dieu, venez à mon secours.\\
\Respons. Seigneur, hâtez-vous de me secourir.\\
\Versicle. Gloire au Père, au Fils, * et au Saint Esprit.\\
\Respons. Comme Il était au commencement, maintenant, et toujours, * et pout les siècles des siècles. Amen.\\
Alleluia.\\}
\ERIGHT } \ENDPARA}

\newcommand{\paternoster}{
\BEGINPARA \BLEFT { \fontlat{
Pater noster, qui es in cælis, sanctificétur nomen tuum:
advéniat regnum tuum: fiat volúntas tua, sicut in cælo
et in terra. Panem nostrum quotidiánum da nobis hódie:
et dimítte nobis débita nostra, sicut et nos dimíttimus
debitóribus nostris:\\
\Versicle. Et ne nos indúcas in tentatiónem:\\
\Respons. Sed líbera nos a malo.\\
\emph{Absolutio}. Exaudi, Domine Iesu Christe, preces servorum
tuorum, et miserere nobis: Qui cum Patre et Spiritu
Sancto vivis et regnas in saecula saeculorum. \\ Amen.}
\ELEFT } \BRIGHT { \fontfr{
 Notre Père, qui êtes aux cieux ; Que votre nom soit sanctifié ;
 Que votre règne arrive ; Que votre volonté soit faite
 Sur la terre comme au ciel.
 Donnez-nous aujourd'hui notre pain quotidien
 Pardonnez-nous nos offenses,
 Comme nous pardonnons à ceux qui nous ont offensés.
 \Versicle. Et ne nous laissez pas succomber à la tentation.\\
 \Respons. Mais délivrez-nous du mal.\\
 \emph{Absolution} Veuillez exaucer, Seigneur Jésus-Christ, les prières de vos serviteurs,
 et ayez pitié de nous, vous qui vivez et régnez avec le Père 
 et le Saint-Esprit, dans les siècles des siècles. \\ Amen.}
\ERIGHT } \ENDPARA}

\newcommand{\conclusio}{
\BEGINPARA \BLEFT { \fontlat{
\Versicle. Dómine exáudi oratiónem meam. \\
\Respons. Et clamor meus ad te véniat. \\
\Versicle Benedicámus Dómino. \\
\Respons Deo grátias. \\
\Versicle. Fidélium ánimæ per misericórdiam Dei requiéscant in
pace.  \\
\Respons. Amen.}
 \ELEFT } \BRIGHT { \fontfr{
\Versicle. Seigneur, veuillez exaucer ma prière. \\
\Respons. Et que mon cri parvienne jusqu'à Vous. \\
\Versicle. Bénissons le Seigneur. \\
\Respons. Rendons grâces à Dieu. \\
\Versicle. Que les âmes des fidèles trépassés, par la miséricorde de Dieu, reposent en paix. \\
\Respons. Amen.}
\ERIGHT } \ENDPARA}
 
\newcommand{\gloriapatri}{
\BEGINPARA \BLEFT { \fontlat{
\Versicle. Glória Patri, et Fílio, * et Spirítui Sancto.\\
\Respons. Sicut erat in princípio, et nunc, et semper, * et in sǽcula sæculórum. Amen.}
\ELEFT } \BRIGHT { \fontfr
\Versicle. Gloire au Père, au Fils,* et au Saint-Esprit.
\Respons. Comme Il était au commencement, maintenant, et toujours * et pour les siècles des siècles. Amen.
\ERIGHT } \ENDPARA}
 

\newcommand{\tuautemdomine}{\Versicle. Tu autem Dómine miserére nobis.\\ \Respons. Deo grátias. }
\newcommand{\tuautemdominefr}{\Versicle. Vous aussi, Seigneur, prenez pitié de nous.\\ \Respons. Nous rendons grâce à Dieu. }
%%%%%%%%%%%%%%%

\newcommand{\aspiciensalonge}{\Respons. Aspiciens a longe, ecce video Dei potentiam venientem, et nebulam totam terram tegentem. * Ite obviam ei, et dicite: * Nuntia nobis si tu es ipse, * Qui regnaturus es in populo Israël. \\ \Versicle. Quique terrigenae et filii hominum simul in unum dives et pauper. \\ \Respons. Ite obviam ei, et dicite. \\ \Versicle. Qui regis Israël. intende, qui deducis velut ovem Joseph. \\ \Respons. Nuntia nobis, si tu es ipse. \\ \Versicle. Tollite portas principes vestras et elevamini portae aeternales, et introibit Rex gloriae. \\ \Respons. Qui regnaturus es in populo Israël. \\ \Versicle. Glória Patri, et Fílio, * et Spirítui Sancto. \\ \Respons. \emph{Aspiciens a longe ...}}

\newcommand{\eccediesveniunt}{\Respons. Ecce dies veniunt, dicit Dominus, et suscitabo David germen justum et regnabit rex, et sapiens erit, et faciet judicium et justitiam in terra  * Et hoc est nomen quod vocabunt eum: * Dominus justus noster. \\ \Versicle. In diebus illis salvabitur Juda, et Israël habitabit confidenter. \\ \Respons. Et hoc est nomen quod vocabunt eum \\ \Versicle. Glória Patri, et Fílio, * et Spirítui Sancto. \\ \Respons. Dominus justus noster.}

\newcommand{\exsionspecies}{\Versicle. Ex Sion species decoris eius. \\  \Respons. Deus noster manifeste veniet.}

\newcommand{\iubeillenos}{\Versicle. Iube Domine, benedicere \\ \emph{Benedictio}. Ille nos benedicat, qui sine fine vivit et regnat. Amen.}
\newcommand{\iubedivinum}{\Versicle. Iube Domine, benedicere \\ \emph{Benedictio}. Divinum auxilium maneat semper nobiscum. Amen.}
\newcommand{\iubeevangelica}{\Versicle. Iube Domine, benedicere \\ \emph{Benedictio}. Per evangelica dicta, deleantur nostra delicta. Amen. }

\newcommand{\missusestgabriel}{
\Respons. Missus est Gabriel Angelus ad Mariam Virginem desponsatam Joseph, nuntians ei verbum; et expavescit Virgo de lumine: ne timeas, Maria, invenisti gratiam apud Dominum:   * Ecce concipies et paries, et vocabitur Altissimi Filius. \\ \Versicle. Dabit ei Dominus Deus sedem David, patris ejus, et regnabit in domo Jacob in aeternum. \\ \Respons. Ecce concipies et paries, et vocabitur Altissimi Filius. \\ \Versicle. Glória Patri, et Fílio, * et Spirítui Sancto. \\ \Respons. Ecce concipies et paries, et vocabitur Altissimi Filius.}


\newcommand{\isIvI}{\VERSE  Visio Isaiæ, filii Amos, quam vidit super Judam et Jerusalem, in diebus Oziæ, Joathan, Achaz, et Ezechiæ, regum Juda. \EVERSE}
\newcommand{\isIvII}{\VERSE  Audite, cæli, et auribus percipe, terra, quoniam Dominus locutus est. Filios enutrivi, et exaltavi ; ipsi autem spreverunt me. \EVERSE}
\newcommand{\isIvIII}{\VERSE  Cognovit bos possessorem suum, et asinus præsepe domini sui ; Israël autem me non cognovit, et populus meus non intellexit. \EVERSE}
\newcommand{\isIvIV}{\VERSE  Væ genti peccatrici, populo gravi iniquitate, semini nequam, filiis sceleratis ! dereliquerunt Dominum ; blasphemaverunt Sanctum Israël ; abalienati sunt retrorsum. \EVERSE}
\newcommand{\isIvV}{\VERSE  Super quo percutiam vos ultra, addentes prævaricationem ? omne caput languidum, et omne cor mœrens. \EVERSE}
\newcommand{\isIvVI}{\VERSE  A planta pedis usque ad verticem, non est in eo sanitas ; vulnus, et livor, et plaga tumens, non est circumligata, nec curata medicamine, neque fota oleo. \EVERSE}
\newcommand{\isIvVII}{\VERSE  Terra vestra deserta ; civitates vestræ succensæ igni : regionem vestram coram vobis alieni devorant, et desolabitur sicut in vastitate hostili. \EVERSE}
\newcommand{\isIvVIII}{\VERSE  Et derelinquetur filia Sion ut umbraculum in vinea, et sicut tugurium in cucumerario, et sicut civitas quæ vastatur. \EVERSE}
\newcommand{\isIvIX}{\VERSE  Nisi Dominus exercituum reliquisset nobis semen, quasi Sodoma fuissemus, et quasi Gomorrha similes essemus. \EVERSE}
\newcommand{\isIvX}{\VERSE  Audite verbum Domini, principes Sodomorum ; percipite auribus legem Dei nostri, populus Gomorrhæ. \EVERSE}
\newcommand{\isIvXI}{\VERSE  Quo mihi multitudinem victimarum vestrarum ? dicit Dominus. Plenus sum : holocausta arietum, et adipem pinguium, et sanguinem vitulorum et agnorum et hircorum, nolui. \EVERSE}
\newcommand{\isIvXII}{\VERSE  Cum veniretis ante conspectum meum, quis quæsivit hæc de manibus vestris, ut ambularetis in atriis meis ? \EVERSE}
\newcommand{\isIvXIII}{\VERSE  Ne offeratis ultra sacrificium frustra : incensum abominatio est mihi. Neomeniam et sabbatum, et festivitates alias, non feram ; iniqui sunt cœtus vestri. \EVERSE}
\newcommand{\isIvXIV}{\VERSE  Calendas vestras, et solemnitates vestras odivit anima mea : facta sunt mihi molesta ; laboravi sustinens. \EVERSE}
\newcommand{\isIvXV}{\VERSE  Et cum extenderitis manus vestras, avertam oculos meos a vobis, et cum multiplicaveritis orationem, non exaudiam : manus enim vestræ sanguine plenæ sunt. \EVERSE}
\newcommand{\isIvXVI}{\VERSE  Lavamini, mundi estote ; auferte malum cogitationum vestrarum ab oculis meis : quiescite agere perverse, \EVERSE}
\newcommand{\isIvXVII}{\VERSE  discite benefacere ; quærite judicium, subvenite oppresso, judicate pupillo, defendite viduam. \EVERSE}
\newcommand{\isIvXVIII}{\VERSE  Et venite, et arguite me, dicit Dominus. Si fuerint peccata vestra ut coccinum, quasi nix dealbabuntur ; et si fuerint rubra quasi vermiculus, velut lana alba erunt. \EVERSE}
\newcommand{\isIvXIX}{\VERSE  Si volueritis, et audieritis me, bona terræ comeditis. \EVERSE}
\newcommand{\isIvXX}{\VERSE  Quod si nolueritis, et me ad iracundiam provocaveritis, gladius devorabit vos, quia os Domini locutum est. \EVERSE}
\newcommand{\isIvXXI}{\VERSE  Quomodo facta est meretrix civitas fidelis, plena judicii ? justitia habitavit in ea, nunc autem homicidæ. \EVERSE}
\newcommand{\isIvXXII}{\VERSE  Argentum tuum versum est in scoriam ; vinum tuum mistum est aqua. \EVERSE}
\newcommand{\isIvXXIII}{\VERSE  Principes tui infideles, socii furum. Omnes diligunt munera, sequuntur retributiones. Pupillo non judicant, et causa viduæ non ingreditur ad illos. \EVERSE}
\newcommand{\isIvXXIV}{\VERSE  Propter hoc ait Dominus, Deus exercituum, Fortis Israël : Heu ! consolabor super hostibus meis, et vindicabor de inimicis meis. \EVERSE}
\newcommand{\isIvXXV}{\VERSE  Et convertam manum meam ad te, et excoquam ad puram scoriam tuam, et auferam omne stannum tuum. \EVERSE}
\newcommand{\isIvXXVI}{\VERSE  Et restituam judices tuos ut fuerunt prius, et consiliarios tuos sicut antiquitus ; post hæc vocaberis civitas justi, urbs fidelis. \EVERSE}
\newcommand{\isIvXXVII}{\VERSE  Sion in judicio redimetur, et reducent eam in justitia. \EVERSE}
\newcommand{\isIvXXVIII}{\VERSE  Et conteret scelestos, et peccatores simul ; et qui dereliquerunt Dominum consumentur. \EVERSE}
\newcommand{\isIvXXIX}{\VERSE  Confundentur enim ab idolis quibus sacrificaverunt, et erubescetis super hortis quos elegeratis, \EVERSE}
\newcommand{\isIvXXX}{\VERSE  cum fueritis velut quercus defluentibus foliis, et velut hortus absque aqua. \EVERSE}
\newcommand{\isIvXXXI}{\VERSE  Et erit fortitudo vestra ut favilla stuppæ, et opus vestrum quasi scintilla, et succendetur utrumque simul, et non erit qui extinguat. \EVERSE}
\newcommand{\isIIvI}{\VERSE  Verbum quod vidit Isaias, filius Amos, super Juda et Jerusalem. \EVERSE}
\newcommand{\isIIvII}{\VERSE  Et erit in novissimis diebus : præparatus mons domus Domini in vertice montium, et elevabitur super colles ; et fluent ad eum omnes gentes, \EVERSE}
\newcommand{\isIIvIII}{\VERSE  et ibunt populi multi, et dicent : Venite, et ascendamus ad montem Domini, et ad domum Dei Jacob ; et docebit nos vias suas, et ambulabimus in semitis ejus, quia de Sion exibit lex, et verbum Domini de Jerusalem. \EVERSE}
\newcommand{\isIIvIV}{\VERSE  Et judicabit gentes, et arguet populos multos ; et conflabunt gladios suos in vomeres, et lanceas suas in falces. Non levabit gens contra gentem gladium, nec exercebuntur ultra ad prælium. \EVERSE}
\newcommand{\isIIvV}{\VERSE  Domus Jacob, venite, et ambulemus in lumine Domini. \EVERSE}
\newcommand{\isIIvVI}{\VERSE  Projecisti enim populum tuum, domum Jacob, quia repleti sunt ut olim, et augeres habuerunt ut Philisthiim, et pueris alienis adhæserunt. \EVERSE}
\newcommand{\isIIvVII}{\VERSE  Repleta est terra argento et auro, et non est finis thesaurorum ejus. \EVERSE}
\newcommand{\isIIvVIII}{\VERSE  Et repleta est terra ejus equis, et innumerabiles quadrigæ ejus. Et repleta est terra ejus idolis ; opus manuum suarum adoraverunt, quod fecerunt digiti eorum. \EVERSE}
\newcommand{\isIIvIX}{\VERSE  Et incurvavit se homo, et humiliatus est vir ; ne ergo dimittas eis. \EVERSE}
\newcommand{\isIIvX}{\VERSE  Ingredere in petram, et abscondere in fossa humo a facie timoris Domini, et a gloria majestatis ejus. \EVERSE}
\newcommand{\isIIvXI}{\VERSE  Oculi sublimes hominis humiliati sunt, et incurvabitur altitudo virorum ; exaltabitur autem Dominus solus in die illa. \EVERSE}
\newcommand{\isIIvXII}{\VERSE  Quia dies Domini exercituum super omnem superbum, et excelsum, et super omnem arrogantem, et humiliabitur ; \EVERSE}
\newcommand{\isIIvXIII}{\VERSE  et super omnes cedros Libani sublimes et erectas, et super omnes quercus Basan, \EVERSE}
\newcommand{\isIIvXIV}{\VERSE  et super omnes montes excelsos, et super omnes colles elevatos, \EVERSE}
\newcommand{\isIIvXV}{\VERSE  et super omnem turrim excelsam, et super omnem murum munitum, \EVERSE}
\newcommand{\isIIvXVI}{\VERSE  et super omnes naves Tharsis, et super omne quod visu pulchrum est, \EVERSE}
\newcommand{\isIIvXVII}{\VERSE  et incurvabitur sublimitas hominum, et humiliabitur altitudo virorum, et elevabitur Dominus solus in die illa ; \EVERSE}
\newcommand{\isIIvXVIII}{\VERSE  et idola penitus conterentur ; \EVERSE}
\newcommand{\isIIvXIX}{\VERSE  et introibunt in speluncas petrarum, et in voragines terræ, a facie formidinis Domini et a gloria majestatis ejus, cum surrexerit percutere terram. \EVERSE}
\newcommand{\isIIvXX}{\VERSE  In die illa projiciet homo idola argenti sui, et simulacra auri sui, quæ fecerat sibi ut adoraret, talpas et vespertiliones. \EVERSE}
\newcommand{\isIIvXXI}{\VERSE  Et ingreditur scissuras petrarum et in cavernas saxorum, a facie formidinis Domini, et a gloria majestatis ejus, cum surrexerit percutere terram. \EVERSE}
\newcommand{\isIIvXXII}{\VERSE  Quiescite ergo ab homine, cujus spiritus in naribus ejus est, quia excelsus reputatus est ipse. \EVERSE}
\newcommand{\isIIIvI}{\VERSE  Ecce enim Dominator, Dominus exercituum, auferet a Jerusalem et a Juda validum et fortem, omne robur panis, et omne robor aquæ ; \EVERSE}
\newcommand{\isIIIvII}{\VERSE  fortem, et virum bellatorem, judicem, et prophetam, et ariolum, et senem ; \EVERSE}
\newcommand{\isIIIvIII}{\VERSE  principem super quinquaginta, et honorabilem vultu et consiliarium, et sapientem de architectis, et prudentem eloquii mystici. \EVERSE}
\newcommand{\isIIIvIV}{\VERSE  Et dabo pueros principes eorum, et effeminati dominabuntur eis ; \EVERSE}
\newcommand{\isIIIvV}{\VERSE  et irruet populus, vir ad virum, et unusquisque ad proximum suum ; tumultuabitur puer contra senem, et ignobilis contra nobilem. \EVERSE}
\newcommand{\isIIIvVI}{\VERSE  Apprehendet enim vir fratrem suum, domesticum patris sui : Vestimentum tibi est, princeps esto noster, ruina autem hæc sub manu tua. \EVERSE}
\newcommand{\isIIIvVII}{\VERSE  Respondebit in die illa, dicens : Non sum medicus, et in domo mea non est panis neque vestimentum : nolite constituere me principem populi. \EVERSE}
\newcommand{\isIIIvVIII}{\VERSE  Ruit enim Jerusalem, et Judas concidit, quia lingua eorum et adinventiones eorum contra Dominum, ut provocarent oculos majestatis ejus. \EVERSE}
\newcommand{\isIIIvIX}{\VERSE  Agnitio vultus eorum respondit eis ; et peccatum suum quasi Sodoma prædicaverunt, nec absconderunt. Væ animæ eorum, quoniam reddita sunt eis mala ! \EVERSE}
\newcommand{\isIIIvX}{\VERSE  Dicite justo quoniam bene, quoniam fructum adinventionum suarum comedet. \EVERSE}
\newcommand{\isIIIvXI}{\VERSE  Væ impio in malum ! retributio enim manuum ejus fiet ei. \EVERSE}
\newcommand{\isIIIvXII}{\VERSE  Populum meum exactores sui spoliaverunt, et mulieres dominatæ sunt eis. Popule meus, qui te beatum dicunt, ipsi te decipiunt, et viam gressuum tuorum dissipant. \EVERSE}
\newcommand{\isIIIvXIII}{\VERSE  Stat ad judicandum Dominus, et stat ad judicandos populos. \EVERSE}
\newcommand{\isIIIvXIV}{\VERSE  Dominus ad judicium veniet cum senibus populi sui, et principibus ejus ; vos enim depasti estis vineam, et rapina pauperis in domo vestra. \EVERSE}
\newcommand{\isIIIvXV}{\VERSE  Quare atteritis populum meum, et facies pauperum commolitis ? dicit Dominus Deus exercituum. \EVERSE}
\newcommand{\isIIIvXVI}{\VERSE  Et dixit Dominus : Pro eo quod elevatæ sunt filiæ Sion, et ambulaverunt extento collo, et nutibus oculorum ibant, et plaudebant, ambulabant pedibus suis, et composito gradu incedebant ; \EVERSE}
\newcommand{\isIIIvXVII}{\VERSE  decalvabit Dominus verticem filiarum Sion, et Dominus crinem earum nudabit. \EVERSE}
\newcommand{\isIIIvXVIII}{\VERSE  In die illa auferet Dominus ornamentum calceamentum, \EVERSE}
\newcommand{\isIIIvXIX}{\VERSE  et lunulas, et torques, et monilia, et armillas, et mitras, \EVERSE}
\newcommand{\isIIIvXX}{\VERSE  et discriminalia, et periscelidas, et murenulas, et olfactoriola, et inaures, \EVERSE}
\newcommand{\isIIIvXXI}{\VERSE  et annulos, et gemmas in fronte pendentes, \EVERSE}
\newcommand{\isIIIvXXII}{\VERSE  et mutatoria, et palliola, et linteamina, et acus, \EVERSE}
\newcommand{\isIIIvXXIII}{\VERSE  et specula, et sindones, et vittas, et theristra. \EVERSE}
\newcommand{\isIIIvXXIV}{\VERSE  Et erit pro suavi odore fœtor, et pro zona funiculus, et pro crispanti crine calvitium, et pro fascia pectorali cilicium. \EVERSE}
\newcommand{\isIIIvXXV}{\VERSE  Pulcherrimi quoque viri tui gladio cadent, et fortes tui in prælio. \EVERSE}
\newcommand{\isIIIvXXVI}{\VERSE  Et mœrebunt atque lugebunt portæ ejus, et desolata in terra sedebit. \EVERSE}
\newcommand{\isIVvI}{\VERSE  Et apprehendent septem mulieres virum unum in die illa, dicentes : Panem nostrum comedemus, et vestimentis nostris operiemur : tantummodo invocetur nomen tuum super nos ; aufer opprobrium nostrum. \EVERSE}
\newcommand{\isIVvII}{\VERSE  In die illa, erit germen Domini in magnificentia et gloria, et fructus terræ sublimis, et exsultatio his qui salvati fuerint de Israël. \EVERSE}
\newcommand{\isIVvIII}{\VERSE  Et erit : omnis qui relictus fuerit in Sion, et residuus in Jerusalem, Sanctus vocabitur, omnis qui scriptus est in vita in Jerusalem. \EVERSE}
\newcommand{\isIVvIV}{\VERSE  Si abluerit Dominus sordes filiarum Sion, et sanguinem Jerusalem laverit de medio ejus, in spiritu judicii, et spiritu ardoris. \EVERSE}
\newcommand{\isIVvV}{\VERSE  Et creabit Dominus super omnem locum montis Sion, et ubi invocatus est, nubem per diem et fumum, et splendorem ignis flammantis in nocte : super omnem enim gloriam protectio. \EVERSE}
\newcommand{\isIVvVI}{\VERSE  Et tabernaculum erit in umbraculum, diei ab æstu, et in securitatem et absconsionem a turbine et a pluvia. \EVERSE}
\newcommand{\isVvI}{\VERSE  Cantabo dilecto meo canticum patruelis mei vineæ suæ. Vinea facta est dilecto meo in cornu filio olei. \EVERSE}
\newcommand{\isVvII}{\VERSE  Et sepivit eam, et lapides elegit ex illa, et plantavit eam electam ; et ædificavit turrim in medio ejus, et torcular exstruxit in ea ; et exspectavit ut faceret uvas, et fecit labruscas. \EVERSE}
\newcommand{\isVvIII}{\VERSE  Nunc ergo, habitatores Jerusalem et viri Juda, judicate inter me et vineam meam. \EVERSE}
\newcommand{\isVvIV}{\VERSE  Quid est quod debui ultra facere vineæ meæ, et non feci ei ? an quod exspectavi ut faceret uvas, et fecit labruscas ? \EVERSE}
\newcommand{\isVvV}{\VERSE  Et nunc ostendam vobis quid ego faciam vineæ meæ : auferam sepem ejus, et erit in direptionem ; diruam maceriam ejus, et erit in conculcationem. \EVERSE}
\newcommand{\isVvVI}{\VERSE  Et ponam eam desertam ; non putabitur et non fodietur : et ascendent vepres et spinæ, et nubibus mandabo ne pluant super eam imbrem. \EVERSE}
\newcommand{\isVvVII}{\VERSE  Vinea enim Domini exercituum domus Israël est ; et vir Juda germen ejus delectabile : et exspectavi ut faceret judicium, et ecce iniquitas ; et justitiam, et ecce clamor. \EVERSE}
\newcommand{\isVvVIII}{\VERSE  Væ qui conjungitis domum ad domum, et agrum agro copulatis usque ad terminum loci ! Numquid habitabitis vos soli in medio terræ ? \EVERSE}
\newcommand{\isVvIX}{\VERSE  In auribus meis sunt hæc, dicit Dominus exercituum ; nisi domus multæ desertæ fuerint, grandes et pulchræ, absque habitatore. \EVERSE}
\newcommand{\isVvX}{\VERSE  Decem enim jugera vinearum facient lagunculam unam, et triginta modii sementis facient modios tres. \EVERSE}
\newcommand{\isVvXI}{\VERSE  Væ qui consurgitis mane ad ebrietatem sectandam, et potandum usque ad vesperam, ut vino æstuetis ! \EVERSE}
\newcommand{\isVvXII}{\VERSE  Cithara, et lyra, et tympanum, et tibia, et vinum in conviviis vestris ; et opus Domini non respicitis, nec opera manuum ejus consideratis. \EVERSE}
\newcommand{\isVvXIII}{\VERSE  Propterea captivus ductus est populus meus, quia non habuit scientiam, et nobiles ejus interierunt fame, et multitudo ejus siti exaruit. \EVERSE}
\newcommand{\isVvXIV}{\VERSE  Propterea dilatavit infernus animam suam, et aperuit os suum absque ullo termino ; et descendent fortes ejus, et populus ejus, et sublimes gloriosique ejus, ad eum. \EVERSE}
\newcommand{\isVvXV}{\VERSE  Et incurvabitur homo, et humiliabitur vir, et oculi sublimium deprimentur. \EVERSE}
\newcommand{\isVvXVI}{\VERSE  Et exaltabitur Dominus exercituum in judicio ; et Deus sanctus sanctificabitur in justitia. \EVERSE}
\newcommand{\isVvXVII}{\VERSE  Et pascentur agni juxta ordinem suum, et deserta in ubertatem versa advenæ comedent. \EVERSE}
\newcommand{\isVvXVIII}{\VERSE  Væ qui trahitis iniquitatem in funiculis vanitatis, et quasi vinculum plaustri peccatum ! \EVERSE}
\newcommand{\isVvXIX}{\VERSE  qui dicitis : Festinet, et cito veniat opus ejus, ut videamus ; et appropiet, et veniat consilium sancti Israël, et sciemus illud ! \EVERSE}
\newcommand{\isVvXX}{\VERSE  Væ qui dicitis malum bonum, et bonum malum ; ponentes tenebras lucem, et lucem tenebras ; ponentes amarum in dulce, et dulce in amarum ! \EVERSE}
\newcommand{\isVvXXI}{\VERSE  Væ qui sapientes estis in oculis vestris, et coram vobismetipsis prudentes. \EVERSE}
\newcommand{\isVvXXII}{\VERSE  Væ qui potentes estis ad bibendum vinum, et viri fortes ad miscendam ebrietatem ! \EVERSE}
\newcommand{\isVvXXIII}{\VERSE  qui justificatis impium pro muneribus, et justitiam justi aufertis ab eo ! \EVERSE}
\newcommand{\isVvXXIV}{\VERSE  Propter hoc, sicut devorat stipulam lingua ignis, et calor flammæ exurit, sic radix eorum quasi favilla erit, et germen eorum ut pulvis ascendet ; abjecerunt enim legem Domini exercituum, et eloquium sancti Israël blasphemaverunt. \EVERSE}
\newcommand{\isVvXXV}{\VERSE  Ideo iratus est furor Domini in populum suum, et extendit manum suam super eum, et percussit eum : et conturbati sunt montes, et facta sunt morticina eorum quasi stercus in medio platearum. In his omnibus non est adversus furor ejus, sed adhuc manus ejus extenta. \EVERSE}
\newcommand{\isVvXXVI}{\VERSE  Et elevabit signum in nationibus procul, et sibilabit ad eum de finibus terræ : et ecce festinus velociter veniet. \EVERSE}
\newcommand{\isVvXXVII}{\VERSE  Non est deficiens neque laborans in eo ; non dormitabit, neque dormiet ; neque solvetur cingulum renum ejus, nec rumpetur corrigia calceamenti ejus. \EVERSE}
\newcommand{\isVvXXVIII}{\VERSE  Sagittæ ejus acutæ, et omnes arcus ejus extenti. Ungulæ equorum ejus ut silex, et rotæ ejus quasi impetus tempestatis. \EVERSE}
\newcommand{\isVvXXIX}{\VERSE  Rugitus ejus ut leonis ; rugiet ut catuli leonum : et frendet, et tenebit prædam, et amplexabitur, et non erit qui eruat. \EVERSE}
\newcommand{\isVvXXX}{\VERSE  Et sonabit super eum in die illa sicut sonitus maris : aspiciemus in terram, et ecce tenebræ tribulationis, et lux obtenebrata est in caligine ejus. \EVERSE}
\newcommand{\isVIvI}{\VERSE  In anno quo mortuus est rex Ozias, vidi Dominum sedentem super solium excelsum et elevatum ; et ea quæ sub ipso erant replebant templum. \EVERSE}
\newcommand{\isVIvII}{\VERSE  Seraphim stabant super illud : sex alæ uni, et sex alæ alteri ; duabus velabant faciem ejus, et duabus velabant pedes ejus, et duabus volabant. \EVERSE}
\newcommand{\isVIvIII}{\VERSE  Et clamabant alter ad alterum, et dicebant : Sanctus, sanctus, sanctus Dominus, Deus exercituum ; plena est omnis terra gloria ejus. \EVERSE}
\newcommand{\isVIvIV}{\VERSE  Et commota sunt superliminaria cardinum a voce clamantis, et domus repleta est fumo. \EVERSE}
\newcommand{\isVIvV}{\VERSE  Et dixi : Væ mihi, quia tacui, quia vir pollutus labiis ego sum, et in medio populi polluta labia habentis ego habito, et regem Dominum exercituum vidi oculis meis. \EVERSE}
\newcommand{\isVIvVI}{\VERSE  Et volavit ad me unus de seraphim, et in manu ejus calculus, quem forcipe tulerat de altari, \EVERSE}
\newcommand{\isVIvVII}{\VERSE  et tetigit os meum, et dixit : Ecce tetigit hoc labia tua, et auferetur iniquitas tua, et peccatum tuum mundabitur. \EVERSE}
\newcommand{\isVIvVIII}{\VERSE  Et audivi vocem Domini dicentis : Quem mittam ? et quis ibit nobis ? Et dixi : Ecce ego, mitte me. \EVERSE}
\newcommand{\isVIvIX}{\VERSE  Et dixit : Vade, et dices populo huic : Audite audientes, et nolite intelligere ; et videte visionem, et nolite cognoscere. \EVERSE}
\newcommand{\isVIvX}{\VERSE  Excæca cor populi hujus, et aures ejus aggrava, et oculos ejus claude : ne forte videat oculis suis, et auribus suis audiat, et corde suo intelligat, et convertatur, et sanem eum. \EVERSE}
\newcommand{\isVIvXI}{\VERSE  Et dixi : Usquequo, Domine ? Et dixit : Donec desolentur civitates absque habitatore, et domus sine homine, et terra relinquetur deserta. \EVERSE}
\newcommand{\isVIvXII}{\VERSE  Et longe faciet Dominus homines, et multiplicabitur quæ derelicta fuerat in medio terræ. \EVERSE}
\newcommand{\isVIvXIII}{\VERSE  Et adhuc in ea decimatio, et convertetur, et erit in ostensionem sicut terebinthus, et sicut quercus quæ expandit ramos suos ; semen sanctum erit id quod steterit in ea. \EVERSE}
\newcommand{\isVIIvI}{\VERSE  Et factum est in diebus Achaz, filii Joathan, filii Oziæ, regis Juda, ascendit Rasin, rex Syriæ, et Phacee, filius Romeliæ, rex Israël, in Jerusalem, ad præliandum contra eam : et non potuerunt \EVERSE}
\newcommand{\isVIIvII}{\VERSE  debellare eam. Et nuntiaverunt domui David, dicentes : Requievit Syria super Ephraim. Et commotum est cor ejus, et cor populi ejus, sicut moventur ligna silvarum a facie venti. \EVERSE}
\newcommand{\isVIIvIII}{\VERSE  Et dixit Dominus ad Isaiam : Egredere in occursum Achaz, tu et qui derelictus est Jasub, filius tuus, ad extremum aquæductus piscinæ superioris in via agri Fullonis ; \EVERSE}
\newcommand{\isVIIvIV}{\VERSE  et dices ad eum : Vide ut sileas ; noli timere, et cor tuum ne formidet a duabus caudis titionum fumigantium istorum, in ira furoris Rasin, regis Syriæ, et filii Romeliæ ; \EVERSE}
\newcommand{\isVIIvV}{\VERSE  eo quod consilium inierit contra te Syria in malum, Ephraim, et filius Romeliæ, dicentes : \EVERSE}
\newcommand{\isVIIvVI}{\VERSE  Ascendamus ad Judam, et suscitemus eum, et avellamus eum ad nos, et ponamus regem in medio ejus filium Tabeel. \EVERSE}
\newcommand{\isVIIvVII}{\VERSE  Hæc dicit Dominus Deus : Non stabit, et non erit istud ; \EVERSE}
\newcommand{\isVIIvVIII}{\VERSE  sed caput Syriæ Damascus, et caput Damasci Rasin ; et adhuc sexaginta et quinque anni, et desinet Ephraim esse populus ; \EVERSE}
\newcommand{\isVIIvIX}{\VERSE  et caput Ephraim Samaria, et caput Samariæ filius Romeliæ. Si non credideritis, non permanebitis. \EVERSE}
\newcommand{\isVIIvX}{\VERSE  Et adjecit Dominus loqui ad Achaz, dicens : \EVERSE}
\newcommand{\isVIIvXI}{\VERSE  Pete tibi signum a Domino Deo tuo, in profundum inferni, sive in excelsum supra. \EVERSE}
\newcommand{\isVIIvXII}{\VERSE  Et dixit Achaz : Non petam, et non tentabo Dominum. \EVERSE}
\newcommand{\isVIIvXIII}{\VERSE  Et dixit : Audite ergo, domus David. Numquid parum vobis est molestos esse hominibus, quia molesti estis et Deo meo ? \EVERSE}
\newcommand{\isVIIvXIV}{\VERSE  Propter hoc dabit Dominus ipse vobis signum : ecce virgo concipiet, et pariet filium, et vocabitur nomen ejus Emmanuel. \EVERSE}
\newcommand{\isVIIvXV}{\VERSE  Butyrum et mel comedet, ut sciat reprobare malum, et eligere bonum. \EVERSE}
\newcommand{\isVIIvXVI}{\VERSE  Quia antequam sciat puer reprobare malum et eligere bonum, derelinquetur terra quam tu detestaris a facie duorum regum suorum. \EVERSE}
\newcommand{\isVIIvXVII}{\VERSE  Adducet Dominus super te, et super populum tuum, et super domum patris tui, dies qui non venerunt a diebus separationis Ephraim a Juda, cum rege Assyriorum. \EVERSE}
\newcommand{\isVIIvXVIII}{\VERSE  Et erit in die illa : sibilabit Dominus muscæ quæ est in extremo fluminum Ægypti, et api quæ est in terra Assur ; \EVERSE}
\newcommand{\isVIIvXIX}{\VERSE  et venient, et requiescent omnes in torrentibus vallium, et in cavernis petrarum, et in omnibus frutetis, et in universis foraminibus. \EVERSE}
\newcommand{\isVIIvXX}{\VERSE  In die illa radet Dominus in novacula conducta in his qui trans flumen sunt, in rege Assyriorum, caput et pilos pedum, et barbam universam. \EVERSE}
\newcommand{\isVIIvXXI}{\VERSE  Et erit in die illa : nutriet homo vaccam boum, et duas oves, \EVERSE}
\newcommand{\isVIIvXXII}{\VERSE  et præ ubertate lactis comedet butyrum ; butyrum enim et mel manducabit omnis qui relictus fuerit in medio terræ. \EVERSE}
\newcommand{\isVIIvXXIII}{\VERSE  Et erit in die illa : omnis locus ubi fuerint mille vites, mille argenteis, in spinas et in vepres erunt. \EVERSE}
\newcommand{\isVIIvXXIV}{\VERSE  Cum sagittis et arcu ingredientur illuc : vepres enim et spinæ erunt in universa terra. \EVERSE}
\newcommand{\isVIIvXXV}{\VERSE  Et omnes montes qui in sarculo sarrientur, non veniet illuc terror spinarum et veprium : et erit in pascua bovis, et in conculcationem pecoris. \EVERSE}
\newcommand{\isVIIIvI}{\VERSE  Et dixit Dominus ad me : Sume tibi librum grandem, et scribe in eo stylo hominis : Velociter spolia detrahe, cito prædare. \EVERSE}
\newcommand{\isVIIIvII}{\VERSE  Et adhibui mihi testes fideles, Uriam sacerdotem, et Zachariam, filium Barachiæ : \EVERSE}
\newcommand{\isVIIIvIII}{\VERSE  et accessi ad prophetissam, et concepit, et peperit filium. Et dixit Dominus ad me : Voca nomen ejus : Accelera spolia detrahere ; Festina prædari : \EVERSE}
\newcommand{\isVIIIvIV}{\VERSE  quia antequam sciat puer vocare patrem suum et matrem suam, auferetur fortitudo Damasci, et spolia Samariæ, coram rege Assyriorum. \EVERSE}
\newcommand{\isVIIIvV}{\VERSE  Et adjecit Dominus loqui ad me adhuc, dicens : \EVERSE}
\newcommand{\isVIIIvVI}{\VERSE  Pro eo quod abjecit populus iste aquas Siloë, quæ vadunt cum silentio, et assumpsit magis Rasin, et filium Romeliæ : \EVERSE}
\newcommand{\isVIIIvVII}{\VERSE  propter hoc ecce Dominus adducet super eos aquas fluminis fortes et multas, regem Assyriorum, et omnem gloriam ejus, et ascendet super omnes rivos ejus, et fluet super universas ripas ejus ; \EVERSE}
\newcommand{\isVIIIvVIII}{\VERSE  et ibit per Judam, inundans, et transiens : usque ad collum veniet. Et erit extensio alarum ejus implens latitudinem terræ tuæ, o Emmanuel ! \EVERSE}
\newcommand{\isVIIIvIX}{\VERSE  Congregamini, populi, et vincimini ; et audite, universæ procul terræ : confortamini, et vincimini ; accingite vos, et vincimini. \EVERSE}
\newcommand{\isVIIIvX}{\VERSE  Inite consilium, et dissipabitur ; loquimini verbum, et non fiet : quia nobiscum Deus. \EVERSE}
\newcommand{\isVIIIvXI}{\VERSE  Hæc enim ait Dominus ad me : Sicut in manu forti erudivit me, ne irem in via populi hujus, dicens : \EVERSE}
\newcommand{\isVIIIvXII}{\VERSE  Non dicatis : Conjuratio ; omnia enim quæ loquitur populus iste, conjuratio est : et timorem ejus ne timeatis, neque paveatis. \EVERSE}
\newcommand{\isVIIIvXIII}{\VERSE  Dominum exercituum ipsum sanctificate ; ipse pavor vester, et ipse terror vester : \EVERSE}
\newcommand{\isVIIIvXIV}{\VERSE  et erit vobis in sanctificationem ; in lapidem autem offensionis, et in petram scandali, duabus domibus Israël ; in laqueum et in ruinam habitantibus Jerusalem. \EVERSE}
\newcommand{\isVIIIvXV}{\VERSE  Et offendent ex eis plurimi, et cadent, et conterentur, et irretientur, et capientur. \EVERSE}
\newcommand{\isVIIIvXVI}{\VERSE  Liga testimonium, signa legem in discipulis meis. \EVERSE}
\newcommand{\isVIIIvXVII}{\VERSE  Et exspectabo Dominum qui abscondit faciem suam a domo Jacob, et præstolabor eum. \EVERSE}
\newcommand{\isVIIIvXVIII}{\VERSE  Ecce ego et pueri mei quos dedit mihi Dominus in signum, et in portentum Israël a Domino exercituum, qui habitat in monte Sion : \EVERSE}
\newcommand{\isVIIIvXIX}{\VERSE  et cum dixerint ad vos : Quærite a pythonibus et a divinis qui strident in incantationibus suis : numquid non populus a Deo suo requiret, pro vivis a mortuis ? \EVERSE}
\newcommand{\isVIIIvXX}{\VERSE  ad legem magis et ad testimonium. Quod si non dixerint juxta verbum hoc, non erit eis matutina lux. \EVERSE}
\newcommand{\isVIIIvXXI}{\VERSE  Et transibit per eam, corruet, et esuriet ; et cum esurierit, irascetur. Et maledicet regi suo, et Deo suo, et suscipiet sursum, \EVERSE}
\newcommand{\isVIIIvXXII}{\VERSE  et ad terram intuebitur ; et ecce tribulatio et tenebræ, dissolutio et angustia, et caligo persequens, et non poterit avolare de angustia sua. \EVERSE}
\newcommand{\isIXvI}{\VERSE  Primo tempore alleviata est terra Zabulon et terra Nephthali : et novissimo aggravata est via maris trans Jordanem Galilææ gentium. \EVERSE}
\newcommand{\isIXvII}{\VERSE  Populus qui ambulabat in tenebris, vidit lucem magnam ; habitantibus in regione umbræ mortis, lux orta est eis. \EVERSE}
\newcommand{\isIXvIII}{\VERSE  Multiplicasti gentem, et non magnificasti lætitiam. Lætabuntur coram te, sicut qui lætantur in messe ; sicut exsultant victores capta præda, quando dividunt spolia. \EVERSE}
\newcommand{\isIXvIV}{\VERSE  Jugum enim oneris ejus, et virgam humeri ejus, et sceptrum exactoris ejus superasti, sicut in die Madian. \EVERSE}
\newcommand{\isIXvV}{\VERSE  Quia omnis violentia prædatio cum tumultu, et vestimentum mistum sanguine, erit in combustionem, et cibus ignis. \EVERSE}
\newcommand{\isIXvVI}{\VERSE  Parvulus enim natus est nobis, et filius datus est nobis, et factus est principatus super humerum ejus : et vocabitur nomen ejus, Admirabilis, Consiliarius, Deus, Fortis, Pater futuri sæculi, Princeps pacis. \EVERSE}
\newcommand{\isIXvVII}{\VERSE  Multiplicabitur ejus imperium, et pacis non erit finis ; super solium David, et super regnum ejus sedebit, ut confirmet illud et corroboret in judicio et justitia, amodo et usque in sempiternum : zelus Domini exercituum faciet hoc. \EVERSE}
\newcommand{\isIXvVIII}{\VERSE  Verbum misit Dominus in Jacob, et cecidit in Israël. \EVERSE}
\newcommand{\isIXvIX}{\VERSE  Et sciet omnis populus Ephraim, et habitantes Samariam, in superbia et magnitudine cordis dicentes : \EVERSE}
\newcommand{\isIXvX}{\VERSE  Lateres ceciderunt, sed quadris lapidibus ædificabimus ; sycomoros succiderunt, sed cedros immutabimus. \EVERSE}
\newcommand{\isIXvXI}{\VERSE  Et elevabit Dominus hostes Rasin super eum, et inimicos ejus in tumultum vertet. \EVERSE}
\newcommand{\isIXvXII}{\VERSE  Syriam ab oriente, et Philisthiim ab occidente ; et devorabunt Israël toto ore. In omnibus his non est aversus furor ejus, sed adhuc manus ejus extenta. \EVERSE}
\newcommand{\isIXvXIII}{\VERSE  Et populus non est reversus ad percutientem se, et Dominum exercituum non inquisierunt. \EVERSE}
\newcommand{\isIXvXIV}{\VERSE  Et disperdet Dominus ab Israël caput et caudam, incurvantem et refrenantem, die una. \EVERSE}
\newcommand{\isIXvXV}{\VERSE  Longævus et honorabilis, ipse est caput ; et propheta docens mendacium, ipse est cauda. \EVERSE}
\newcommand{\isIXvXVI}{\VERSE  Et erunt qui beatificant populum istum, seducentes ; et qui beatificantur, præcipitati. \EVERSE}
\newcommand{\isIXvXVII}{\VERSE  Propter hoc super adolescentulis ejus non lætabitur Dominus, et pupillorum ejus et viduarum non miserebitur : quia omnis hypocrita est et nequam, et universum os locutum est stultitiam ; in omnibus his non est aversus furor ejus, sed adhuc manus ejus extenta. \EVERSE}
\newcommand{\isIXvXVIII}{\VERSE  Succensa est enim quasi ignis impietas : veprem et spinam vorabit, et succendetur in densitate saltus, et convolvetur superbia fumi. \EVERSE}
\newcommand{\isIXvXIX}{\VERSE  In ira Domini exercituum conturbata est terra, et erit populus quasi esca ignis ; vir fratri suo non parcet. \EVERSE}
\newcommand{\isIXvXX}{\VERSE  Et declinabit ad dexteram, et esuriet ; et comedet ad sinistram, et non saturabitur ; unusquisque carnem brachii sui vorabit : Manasses Ephraim, et Ephraim Manassen ; simul ipsi contra Judam. \EVERSE}
\newcommand{\isIXvXXI}{\VERSE  In omnibus his non est aversus furor ejus, sed adhuc manus ejus extenta. \EVERSE}
\newcommand{\isXvI}{\VERSE  Væ qui condunt leges iniquas, et scribentes injustitiam scripserunt, \EVERSE}
\newcommand{\isXvII}{\VERSE  ut opprimerent in judicio pauperes, et vim facerent causæ humilium populi mei ; ut essent viduæ præda eorum, et pupillos diriperent. \EVERSE}
\newcommand{\isXvIII}{\VERSE  Quid facietis in die visitationis, et calamitatis de longe venientis ? ad cujus confugietis auxilium ? et ubi derelinquetis gloriam vestram, \EVERSE}
\newcommand{\isXvIV}{\VERSE  ne incurvemini sub vinculo, et cum interfectis cadatis ? Super omnibus his non est aversus furor ejus, sed adhuc manus ejus extenta. \EVERSE}
\newcommand{\isXvV}{\VERSE  Væ Assur ! virga furoris mei et baculus ipse est ; in manu eorum indignatio mea. \EVERSE}
\newcommand{\isXvVI}{\VERSE  Ad gentem fallacem mittam eum, et contra populum furoris mei mandabo illi, ut auferat spolia, et diripiat prædam, et ponat illum in conculcationem quasi lutum platearum. \EVERSE}
\newcommand{\isXvVII}{\VERSE  Ipse autem non sic arbitrabitur, et cor ejus non ita existimabit ; sed ad conterendum erit cor ejus, et ad internecionem gentium non paucarum. \EVERSE}
\newcommand{\isXvVIII}{\VERSE  Dicet enim : \EVERSE}
\newcommand{\isXvIX}{\VERSE  Numquid non principes mei simul reges sunt ? numquid non ut Charcamis, sic Calano ? et ut Arphad, sic Emath ? numquid non ut Damascus, sic Samaria ? \EVERSE}
\newcommand{\isXvX}{\VERSE  Quomodo invenit manus mea regna idoli, sic et simulacra eorum de Jerusalem et de Samaria. \EVERSE}
\newcommand{\isXvXI}{\VERSE  Numquid non sicut feci Samariæ et idolis ejus, sic faciam Jerusalem et simulacris ejus ? \EVERSE}
\newcommand{\isXvXII}{\VERSE  Et erit, cum impleverit Dominus cuncta opera sua in monte Sion et in Jerusalem, visitabo super fructum magnifici cordis regis Assur, et super gloriam altitudinis oculorum ejus. \EVERSE}
\newcommand{\isXvXIII}{\VERSE  Dixit enim : In fortitudine manus meæ feci, et in sapientia mea intellexi ; et abstuli terminos populorum, et principes eorum deprædatus sum, et detraxi quasi potens in sublimi residentes. \EVERSE}
\newcommand{\isXvXIV}{\VERSE  Et invenit quasi nidum manus mea fortitudinem populorum ; et sicut colliguntur ova quæ derelicta sunt, sic universam terram ego congregavi ; et non fuit qui moveret pennam, et aperiret os, et ganniret. \EVERSE}
\newcommand{\isXvXV}{\VERSE  Numquid gloriabitur securis contra eum qui secat in ea ? aut exaltabitur serra contra eum a quo trahitur ? Quomodo si elevetur virga contra elevantem se, et exaltetur baculus, qui utique lignum est. \EVERSE}
\newcommand{\isXvXVI}{\VERSE  Propter hoc mittet Dominator, Dominus exercituum, in pinguibus ejus tenuitatem ; et subtus gloriam ejus succensa ardebit quasi combustio ignis. \EVERSE}
\newcommand{\isXvXVII}{\VERSE  Et erit lumen Israël in igne, et Sanctus ejus in flamma ; et succendetur, et devorabitur spina ejus et vepres in die una. \EVERSE}
\newcommand{\isXvXVIII}{\VERSE  Et gloria saltus ejus, et carmeli ejus, ab anima usque ad carnem consumetur ; et erit terrore profugus. \EVERSE}
\newcommand{\isXvXIX}{\VERSE  Et reliquiæ ligni saltus ejus præ paucitate numerabuntur, et puer scribet eos. \EVERSE}
\newcommand{\isXvXX}{\VERSE  Et erit in die illa : non adjiciet residuum Israël, et hi qui fugerint de domo Jacob, inniti super eo qui percutit eos ; sed innitetur super Dominum, Sanctum Israël, in veritate. \EVERSE}
\newcommand{\isXvXXI}{\VERSE  Reliquiæ convertentur ; reliquiæ, inquam, Jacob ad Deum fortem. \EVERSE}
\newcommand{\isXvXXII}{\VERSE  Si enim fuerit populus tuus, Israël, quasi arena maris, reliquiæ convertentur ex eo ; consummatio abbreviata inundabit justitiam. \EVERSE}
\newcommand{\isXvXXIII}{\VERSE  Consummationem enim et abbreviationem Dominus Deus exercituum faciet in medio omnis terræ. \EVERSE}
\newcommand{\isXvXXIV}{\VERSE  Propter hoc, hæc dicit Dominus Deus exercituum : Noli timere, populus meus, habitator Sion, ab Assur : in virga percutiet te, et baculum suum levabit super te, in via Ægypti. \EVERSE}
\newcommand{\isXvXXV}{\VERSE  Adhuc enim paululum modicumque, et consummabitur indignatio et furor meus super scelus eorum. \EVERSE}
\newcommand{\isXvXXVI}{\VERSE  Et suscitabit super eum Dominus exercituum flagellum, juxta plagam Madian in petra Oreb : et virgam suam super mare, et levabit eam in via Ægypti. \EVERSE}
\newcommand{\isXvXXVII}{\VERSE  Et erit in die illa : auferetur onus ejus de humero tuo et jugum ejus de collo tuo, et computrescet jugum a facie olei. \EVERSE}
\newcommand{\isXvXXVIII}{\VERSE  Veniet in Ajath, transibit in Magron, apud Machmas commendabit vasa sua. \EVERSE}
\newcommand{\isXvXXIX}{\VERSE  Transierunt cursim, Gaba sedes nostra ; obstupuit Rama, Gabaath Saulis fugit. \EVERSE}
\newcommand{\isXvXXX}{\VERSE  Hinni voce tua, filia Gallim, attende Laisa, paupercula Anathoth. \EVERSE}
\newcommand{\isXvXXXI}{\VERSE  Migravit Medemena ; habitatores Gabim, confortamini. \EVERSE}
\newcommand{\isXvXXXII}{\VERSE  Adhuc dies est ut in Nobe stetur ; agitabit manum suam super montem filiæ Sion, collem Jerusalem. \EVERSE}
\newcommand{\isXvXXXIII}{\VERSE  Ecce Dominator, Dominus exercituum, confringet lagunculam in terrore ; et excelsi statura succidentur, et sublimes humiliabuntur. \EVERSE}
\newcommand{\isXvXXXIV}{\VERSE  Et subvertentur condensa saltus ferro ; et Libanus cum excelsis cadet. \EVERSE}
\newcommand{\isXIvI}{\VERSE  Et egredietur virga de radice Jesse, et flos de radice ejus ascendet. \EVERSE}
\newcommand{\isXIvII}{\VERSE  Et requiescet super eum spiritus Domini : spiritus sapientiæ et intellectus, spiritus consilii et fortitudinis, spiritus scientiæ et pietatis ; \EVERSE}
\newcommand{\isXIvIII}{\VERSE  et replebit eum spiritus timoris Domini. Non secundum visionem oculorum judicabit, neque secundum auditum aurium arguet ; \EVERSE}
\newcommand{\isXIvIV}{\VERSE  sed judicabit in justitia pauperes, et arguet in æquitate pro mansuetis terræ ; et percutiet terram virga oris sui, et spiritu labiorum suorum interficiet impium. \EVERSE}
\newcommand{\isXIvV}{\VERSE  Et erit justitia cingulum lumborum ejus, et fides cinctorium renum ejus. \EVERSE}
\newcommand{\isXIvVI}{\VERSE  Habitabit lupus cum agno, et pardus cum hædo accubabit ; vitulus, et leo, et ovis, simul morabuntur, et puer parvulus minabit eos. \EVERSE}
\newcommand{\isXIvVII}{\VERSE  Vitulus et ursus pascentur, simul requiescent catuli eorum ; et leo quasi bos comedet paleas. \EVERSE}
\newcommand{\isXIvVIII}{\VERSE  Et delectabitur infans ab ubere super foramine aspidis ; et in caverna reguli qui ablactatus fuerit manum suam mittet. \EVERSE}
\newcommand{\isXIvIX}{\VERSE  Non nocebunt, et non occident in universo monte sancto meo, quia repleta est terra scientia Domini, sicut aquæ maris operientes. \EVERSE}
\newcommand{\isXIvX}{\VERSE  In die illa radix Jesse, qui stat in signum populorum, ipsum gentes deprecabuntur, et erit sepulchrum ejus gloriosum. \EVERSE}
\newcommand{\isXIvXI}{\VERSE  Et erit in die illa : adjiciet Dominus secundo manum suam ad possidendum residuum populi sui, quod relinquetur ab Assyriis, et ab Ægypto, et a Phetros, et ab Æthiopia, et ab Ælam, et a Sennaar, et ab Emath, et ab insulis maris. \EVERSE}
\newcommand{\isXIvXII}{\VERSE  Et levabit signum in nationes, et congregabit profugos Israël, et dispersos Juda colliget a quatuor plagis terræ. \EVERSE}
\newcommand{\isXIvXIII}{\VERSE  Et auferetur zelus Ephraim, et hostes Juda peribunt ; Ephraim non æmulabitur Judam, et Judas non pugnabit contra Ephraim. \EVERSE}
\newcommand{\isXIvXIV}{\VERSE  Et volabunt in humeros Philisthiim per mare, simul prædabuntur filios orientis ; Idumæa et Moab præceptum manus eorum, et filii Ammon obedientes erunt. \EVERSE}
\newcommand{\isXIvXV}{\VERSE  Et desolabit Dominus linguam maris Ægypti, et levabit manum suam super flumen in fortitudine spiritus sui ; et percutiet eum in septem rivis, ita ut transeant per eum calceati. \EVERSE}
\newcommand{\isXIvXVI}{\VERSE  Et erit via residuo populo meo qui relinquetur ab Assyriis, sicut fuit Israëli in die illa qua ascendit de terra Ægypti. \EVERSE}
\newcommand{\isXIIvI}{\VERSE  Et dices in die illa : Confitebor tibi, Domine, quoniam iratus es mihi ; conversus est furor tuus, et consolatus es me. \EVERSE}
\newcommand{\isXIIvII}{\VERSE  Ecce Deus salvator meus ; fiducialiter agam, et non timebo : quia fortitudo mea et laus mea Dominus, et factus est mihi in salutem. \EVERSE}
\newcommand{\isXIIvIII}{\VERSE  Haurietis aquas in gaudio de fontibus salvatoris. \EVERSE}
\newcommand{\isXIIvIV}{\VERSE  Et dicetis in die illa : Confitemini Domino et invocate nomen ejus ; notas facite in populis adinventiones ejus ; mementote quoniam excelsum est nomen ejus. \EVERSE}
\newcommand{\isXIIvV}{\VERSE  Cantate Domino, quoniam magnifice fecit ; annuntiate hoc in universa terra. \EVERSE}
\newcommand{\isXIIvVI}{\VERSE  Exsulta et lauda, habitatio Sion, quia magnus in medio tui Sanctus Israël. \EVERSE}
\newcommand{\isXIIIvI}{\VERSE  Onus Babylonis, quod vidit Isaias, filius Amos. \EVERSE}
\newcommand{\isXIIIvII}{\VERSE  Super montem caliginosum levate signum : exaltate vocem, levate manum, et ingrediantur portas duces. \EVERSE}
\newcommand{\isXIIIvIII}{\VERSE  Ego mandavi sanctificatis meis, et vocavi fortes meos in ira mea, exsultantes in gloria mea. \EVERSE}
\newcommand{\isXIIIvIV}{\VERSE  Vox multitudinis in montibus, quasi populorum frequentium ; vox sonitus regum, gentium congregatarum. Dominus exercituum præcepit militiæ belli, \EVERSE}
\newcommand{\isXIIIvV}{\VERSE  venientibus de terra procul, a summitate cæli ; Dominus, et vasa furoris ejus, ut disperdat omnem terram. \EVERSE}
\newcommand{\isXIIIvVI}{\VERSE  Ululate, quia prope est dies Domini ; quasi vastitas a Domino veniet. \EVERSE}
\newcommand{\isXIIIvVII}{\VERSE  Propter hoc omnes manus dissolventur, et omne cor hominis contabescet, \EVERSE}
\newcommand{\isXIIIvVIII}{\VERSE  et conteretur. Torsiones et dolores tenebunt ; quasi parturiens dolebunt : unusquisque ad proximum suum stupebit, facies combustæ vultus eorum. \EVERSE}
\newcommand{\isXIIIvIX}{\VERSE  Ecce dies Domini veniet, crudelis, et indignationis plenus, et iræ, furorisque, ad ponendam terram in solitudinem, et peccatores ejus conterendos de ea. \EVERSE}
\newcommand{\isXIIIvX}{\VERSE  Quoniam stellæ cæli, et splendor earum, non expandent lumen suum ; obtenebratus est sol in ortu suo, et luna non splendebit in lumine suo. \EVERSE}
\newcommand{\isXIIIvXI}{\VERSE  Et visitabo super orbis mala, et contra impios iniquitatem eorum ; et quiescere faciam superbiam infidelium, et arrogantiam fortium humiliabo. \EVERSE}
\newcommand{\isXIIIvXII}{\VERSE  Pretiosior erit vir auro, et homo mundo obrizo. \EVERSE}
\newcommand{\isXIIIvXIII}{\VERSE  Super hoc cælum turbabo ; et movebitur terra de loco suo, propter indignationem Domini exercituum, et propter diem iræ furoris ejus. \EVERSE}
\newcommand{\isXIIIvXIV}{\VERSE  Et erit quasi damula fugiens, et quasi ovis, et non erit qui congreget. Unusquisque ad populum suum convertetur, et singuli ad terram suam fugient. \EVERSE}
\newcommand{\isXIIIvXV}{\VERSE  Omnis qui inventus fuerit occidetur, et omnis qui supervenerit cadet in gladio ; \EVERSE}
\newcommand{\isXIIIvXVI}{\VERSE  infantes eorum allidentur in oculis eorum, diripientur domus eorum, et uxores eorum violabuntur. \EVERSE}
\newcommand{\isXIIIvXVII}{\VERSE  Ecce ego suscitabo super eos Medos, qui argentum non quærant, nec aurum velint ; \EVERSE}
\newcommand{\isXIIIvXVIII}{\VERSE  sed sagittis parvulos interficient, et lactantibus uteris non miserebuntur, et super filios non parcet oculus eorum. \EVERSE}
\newcommand{\isXIIIvXIX}{\VERSE  Et erit Babylon illa gloriosa in regnis, inclyta superbia Chaldæorum, sicut subvertit Dominus Sodomam et Gomorrham. \EVERSE}
\newcommand{\isXIIIvXX}{\VERSE  Non habitabitur usque in finem, et non fundabitur usque ad generationem et generationem ; nec ponet ibi tentoria Arabs, nec pastores requiescent ibi. \EVERSE}
\newcommand{\isXIIIvXXI}{\VERSE  Sed requiescent ibi bestiæ, et replebuntur domus eorum draconibus, et habitabunt ibi struthiones, et pilosi saltabunt ibi ; \EVERSE}
\newcommand{\isXIIIvXXII}{\VERSE  et respondebunt ibi ululæ in ædibus ejus, et sirenes in delubris voluptatis. \EVERSE}
\newcommand{\isXIVvI}{\VERSE  Prope est ut veniat tempus ejus, et dies ejus non elongabuntur. Miserebitur enim Dominus Jacob, et eliget adhuc de Israël, et requiescere eos faciet super humum suam ; adjungetur advena ad eos, et adhærebit domui Jacob. \EVERSE}
\newcommand{\isXIVvII}{\VERSE  Et tenebunt eos populi, et adducent eos in locum suum ; et possidebit eos domus Israël super terram Domini in servos et ancillas : et erunt capientes eos qui se ceperant, et subjicient exactores suos. \EVERSE}
\newcommand{\isXIVvIII}{\VERSE  Et erit in die illa : cum requiem dederit tibi Deus a labore tuo, et a concussione tua, et a servitute dura qua ante servisti, \EVERSE}
\newcommand{\isXIVvIV}{\VERSE  sumes parabolam istam contra regem Babylonis, et dices : Quomodo cessavit exactor ; quievit tributum ? \EVERSE}
\newcommand{\isXIVvV}{\VERSE  Contrivit Dominus baculum impiorum, virgam dominantium, \EVERSE}
\newcommand{\isXIVvVI}{\VERSE  cædentem populos in indignatione plaga insanabili, subjicientem in furore gentes, persequentem crudeliter. \EVERSE}
\newcommand{\isXIVvVII}{\VERSE  Conquievit et siluit omnis terra, gavisa est et exsultavit ; \EVERSE}
\newcommand{\isXIVvVIII}{\VERSE  abietes quoque lætatæ sunt super te, et cedri Libani : ex quo dormisti, non ascendet qui succidat nos. \EVERSE}
\newcommand{\isXIVvIX}{\VERSE  Infernus subter conturbatus est in occursum adventus tui ; suscitavit tibi gigantes. Omnes principes terræ surrexerunt de soliis suis, omnes principes nationum. \EVERSE}
\newcommand{\isXIVvX}{\VERSE  Universi respondebunt, et dicent tibi : Et tu vulneratus es sicut et nos ; nostri similis effectus es. \EVERSE}
\newcommand{\isXIVvXI}{\VERSE  Detracta est ad inferos superbia tua, concidit cadaver tuum ; subter te sternetur tinea, et operimentum tuum erunt vermes. \EVERSE}
\newcommand{\isXIVvXII}{\VERSE  Quomodo cecidisti de cælo, Lucifer, qui mane oriebaris ? corruisti in terram, qui vulnerabas gentes ? \EVERSE}
\newcommand{\isXIVvXIII}{\VERSE  Qui dicebas in corde tuo : In cælum conscendam, super astra Dei exaltabo solium meum ; sedebo in monte testamenti, in lateribus aquilonis ; \EVERSE}
\newcommand{\isXIVvXIV}{\VERSE  ascendam super altitudinem nubium, similis ero Altissimo ? \EVERSE}
\newcommand{\isXIVvXV}{\VERSE  Verumtamen ad infernum detraheris, in profundum laci. \EVERSE}
\newcommand{\isXIVvXVI}{\VERSE  Qui te viderint, ad te inclinabuntur, teque prospicient : Numquid iste est vir qui conturbavit terram, qui concussit regna, \EVERSE}
\newcommand{\isXIVvXVII}{\VERSE  qui posuit orbem desertum, et urbes ejus destruxit, vinctis ejus non aperuit carcerem ? \EVERSE}
\newcommand{\isXIVvXVIII}{\VERSE  Omnes reges gentium universi dormierunt in gloria, vir in domo sua ; \EVERSE}
\newcommand{\isXIVvXIX}{\VERSE  tu autem projectus es de sepulchro tuo, quasi stirps inutilis pollutus, et obvolutus cum his qui interfecti sunt gladio, et descenderunt ad fundamenta laci, quasi cadaver putridum. \EVERSE}
\newcommand{\isXIVvXX}{\VERSE  Non habebis consortium, neque cum eis in sepultura ; tu enim terram tuam disperdidisti, tu populum tuum occidisti : non vocabitur in æternum semen pessimorum. \EVERSE}
\newcommand{\isXIVvXXI}{\VERSE  Præparate filios ejus occisioni, in iniquitate patrum suorum : non consurgent, nec hæreditabunt terram, neque implebunt faciem orbis civitatum. \EVERSE}
\newcommand{\isXIVvXXII}{\VERSE  Et consurgam super eos, dicit Dominus exercituum ; et perdam Babylonis nomen, et reliquias, et germen, et progeniem, dicit Dominus ; \EVERSE}
\newcommand{\isXIVvXXIII}{\VERSE  et ponam eam in possessionem ericii, et in paludes aquarum, et scopabo eam in scopa terens, dicit Dominus exercituum. \EVERSE}
\newcommand{\isXIVvXXIV}{\VERSE  Juravit Dominus exercituum, dicens : Si non, ut putavi, ita erit ; et quomodo mente tractavi, \EVERSE}
\newcommand{\isXIVvXXV}{\VERSE  sic eveniet : ut conteram Assyrium in terra mea, et in montibus meis conculcem eum ; et auferetur ab eis jugum ejus, et onus illius ab humero eorum tolletur. \EVERSE}
\newcommand{\isXIVvXXVI}{\VERSE  Hoc consilium quod cogitavi super omnem terram ; et hæc est manus extenta super universas gentes. \EVERSE}
\newcommand{\isXIVvXXVII}{\VERSE  Dominus enim exercituum decrevit ; et quis poterit infirmare ? et manus ejus extenta ; et quis avertet eam ? \EVERSE}
\newcommand{\isXIVvXXVIII}{\VERSE  In anno quo mortuus est rex Achaz, factum est onus istud : \EVERSE}
\newcommand{\isXIVvXXIX}{\VERSE  Ne lætaris, Philisthæa omnis tu, quoniam comminuta est virga percussoris tui ; de radice enim colubri egredietur regulus, et semen ejus absorbens volucrem. \EVERSE}
\newcommand{\isXIVvXXX}{\VERSE  Et pascentur primogeniti pauperum, et pauperes fiducialiter requiescent ; et interire faciam in fame radicem tuam, et reliquias tuas interficiam. \EVERSE}
\newcommand{\isXIVvXXXI}{\VERSE  Ulula, porta ; clama civitas ; prostrata est Philisthæa omnis ; ab aquilone enim fumus veniet, et non est qui effugiet agmen ejus. \EVERSE}
\newcommand{\isXIVvXXXII}{\VERSE  Et quid respondebitur nuntiis gentis ? Quia Dominus fundavit Sion, et in ipso sperabunt pauperes populi ejus. \EVERSE}
\newcommand{\isXVvI}{\VERSE  Onus Moab. Quia nocte vastata est Ar Moab, conticuit ; quia nocte vastatus est murus Moab, conticuit. \EVERSE}
\newcommand{\isXVvII}{\VERSE  Ascendit domus, et Dibon ad excelsa, in planctum super Nabo ; et super Medaba, Moab ululavit ; in cunctis capitibus ejus calvitium, et omnis barba radetur. \EVERSE}
\newcommand{\isXVvIII}{\VERSE  In triviis ejus accincti sunt sacco ; super tecta ejus et in plateis ejus omnis ululatus descendit in fletum. \EVERSE}
\newcommand{\isXVvIV}{\VERSE  Clamabit Hesebon et Eleale, usque Jasa audita est vox eorum ; super hoc expediti Moab ululabunt, anima ejus ululabit sibi. \EVERSE}
\newcommand{\isXVvV}{\VERSE  Cor meum ad Moab clamabit ; vectes ejus usque ad Segor, vitulam conternantem ; per ascensum enim Luith flens ascendet, et in via Oronaim clamorem contritionis levabunt. \EVERSE}
\newcommand{\isXVvVI}{\VERSE  Aquæ enim Nemrim desertæ erunt, quia aruit herba, defecit germen, viror omnis interiit. \EVERSE}
\newcommand{\isXVvVII}{\VERSE  Secundum magnitudinem operis, et visitatio eorum : ad torrentem Salicum ducent eos. \EVERSE}
\newcommand{\isXVvVIII}{\VERSE  Quoniam circuivit clamor terminum Moab ; usque ad Gallim ululatus ejus, et usque ad puteum Elim clamor ejus. \EVERSE}
\newcommand{\isXVvIX}{\VERSE  Quia aquæ Dibon repletæ sunt sanguine ; ponam enim super Dibon additamenta ; his qui fugerint de Moab leonem, et reliquiis terræ. \EVERSE}
\newcommand{\isXVIvI}{\VERSE  Emitte agnum, Domine, dominatorem terræ, de petra deserti ad montem filiæ Sion. \EVERSE}
\newcommand{\isXVIvII}{\VERSE  Et erit : sicut avis fugiens, et pulli de nido avolantes, sic erunt filiæ Moab in transcensu Arnon. \EVERSE}
\newcommand{\isXVIvIII}{\VERSE  Ini consilium, coge concilium ; pone quasi noctem umbram tuam in meridie ; absconde fugientes, et vagos ne prodas. \EVERSE}
\newcommand{\isXVIvIV}{\VERSE  Habitabunt apud te profugi mei ; Moab, esto latibulum eorum a facie vastatoris : finitus est enim pulvis, consummatus est miser, defecit qui conculcabat terram. \EVERSE}
\newcommand{\isXVIvV}{\VERSE  Et præparabitur in misericordia solium, et sedebit super illud in veritate in tabernaculo David, judicans et quærens judicium, et velociter reddens quod justum est. \EVERSE}
\newcommand{\isXVIvVI}{\VERSE  Audivimus superbiam Moab : superbus est valde ; superbia ejus, et arrogantia ejus, et indignatio ejus plus quam fortitudo ejus. \EVERSE}
\newcommand{\isXVIvVII}{\VERSE  Idcirco ululabit Moab ad Moab ; universus ululabit : his qui lætantur super muros cocti lateris, loquimini plagas suas. \EVERSE}
\newcommand{\isXVIvVIII}{\VERSE  Quoniam suburbana Hesebon deserta sunt, et vineam Sabama domini gentium exciderunt : flagella ejus usque ad Jazer pervenerunt, erraverunt in deserto ; propagines ejus relictæ sunt, transierunt mare. \EVERSE}
\newcommand{\isXVIvIX}{\VERSE  Super hoc plorabo in fletu Jazer vineam Sabama ; inebriabo de lacrima mea, Hesebon et Eleale, quoniam super vindemiam tuam et super messem tuam vox calcantium irruit. \EVERSE}
\newcommand{\isXVIvX}{\VERSE  Et auferetur lætitia et exsultatio de Carmelo, et in vineis non exsultabit neque jubilabit. Vinum in torculari non calcabit qui calcare consueverat ; vocem calcantium abstuli. \EVERSE}
\newcommand{\isXVIvXI}{\VERSE  Super hoc venter meus ad Moab quasi cithara sonabit, et viscera mea ad murum cocti lateris. \EVERSE}
\newcommand{\isXVIvXII}{\VERSE  Et erit : cum apparuerit quod laboravit Moab super excelsis suis, ingredietur ad sancta sua ut obsecret, et non valebit. \EVERSE}
\newcommand{\isXVIvXIII}{\VERSE  Hoc verbum quod locutus est Dominus ad Moab ex tunc. \EVERSE}
\newcommand{\isXVIvXIV}{\VERSE  Et nunc locutus est Dominus, dicens : In tribus annis, quasi anni mercenarii, auferetur gloria Moab super omni populo multo, et relinquetur parvus et modicus, nequaquam multus. \EVERSE}
\newcommand{\isXVIIvI}{\VERSE  Onus Damasci. Ecce Damascus desinet esse civitas, et erit sicut acervus lapidum in ruina. \EVERSE}
\newcommand{\isXVIIvII}{\VERSE  Derelictæ civitates Aroër gregibus erunt, et requiescent ibi, et non erit qui exterreat. \EVERSE}
\newcommand{\isXVIIvIII}{\VERSE  Et cessabit adjutorium ab Ephraim, et regnum a Damasco ; et reliquiæ Syriæ sicut gloria filiorum Israël erunt, dicit Dominus exercituum. \EVERSE}
\newcommand{\isXVIIvIV}{\VERSE  Et erit in die illa : attenuabitur gloria Jacob, et pinguedo carnis ejus marcescet. \EVERSE}
\newcommand{\isXVIIvV}{\VERSE  Et erit sicut congregans in messe quod restiterit, et brachium ejus spicas leget ; et erit sicut quærens spicas in valle Raphaim. \EVERSE}
\newcommand{\isXVIIvVI}{\VERSE  Et relinquetur in eo sicut racemus et sicut excussio oleæ duarum vel trium olivarum in summitate rami, sive quatuor aut quinque in cacuminibus ejus fructus ejus, dicit Dominus Deus Israël. \EVERSE}
\newcommand{\isXVIIvVII}{\VERSE  In die illa inclinabitur homo ad factorem suum, et oculi ejus ad Sanctum Israël respicient ; \EVERSE}
\newcommand{\isXVIIvVIII}{\VERSE  et non inclinabitur ad altaria quæ fecerunt manus ejus ; et quæ operati sunt digiti ejus non respiciet lucos et delubra. \EVERSE}
\newcommand{\isXVIIvIX}{\VERSE  In die illa erunt civitates fortitudinis ejus derelictæ sicut aratra, et segetes quæ derelictæ sunt a facie filiorum Israël ; et eris deserta. \EVERSE}
\newcommand{\isXVIIvX}{\VERSE  Quia oblitus es Dei salvatoris tui, et fortis adjutoris tui non es recordata : propterea plantabis plantationem fidelem, et germen alienum seminabis ; \EVERSE}
\newcommand{\isXVIIvXI}{\VERSE  in die plantationis tuæ labrusca, et mane semen tuum florebit ; ablata est messis in die hæreditatis, et dolebit graviter. \EVERSE}
\newcommand{\isXVIIvXII}{\VERSE  Væ multitudini populorum multorum, ut multitudo maris sonantis ; et tumultus turbarum, sicut sonitus aquarum multarum. \EVERSE}
\newcommand{\isXVIIvXIII}{\VERSE  Sonabunt populi sicut sonitus aquarum inundantium, et increpabit eum, et fugiet procul ; et rapietur sicut pulvis montium a facie venti, et sicut turbo coram tempestate. \EVERSE}
\newcommand{\isXVIIvXIV}{\VERSE  In tempore vespere, et ecce turbatio ; in matutino, et non subsistet. Hæc est pars eorum qui vastaverunt nos, et sors diripientium nos. \EVERSE}
\newcommand{\isXVIIIvI}{\VERSE  Væ terræ cymbalo alarum, quæ est trans flumina Æthiopiæ, \EVERSE}
\newcommand{\isXVIIIvII}{\VERSE  qui mittit in mare legatos, et in vasis papyri super aquas. Ite, angeli veloces, ad gentem convulsam et dilaceratam ; ad populum terribilem, post quem non est alius ; ad gentem exspectantem et conculcatam, cujus diripuerunt flumina terram ejus. \EVERSE}
\newcommand{\isXVIIIvIII}{\VERSE  Omnes habitatores orbis, qui moramini in terra, cum elevatum fuerit signum in montibus, videbitis, et clangorem tubæ audietis. \EVERSE}
\newcommand{\isXVIIIvIV}{\VERSE  Quia hæc dicit Dominus ad me : Quiescam et considerabo in loco meo, sicut meridiana lux clara est, et sicut nubes roris in die messis. \EVERSE}
\newcommand{\isXVIIIvV}{\VERSE  Ante messem enim totus effloruit, et immatura perfectio germinabit ; et præcidentur ramusculi ejus falcibus, et quæ derelicta fuerint abscindentur et excutientur. \EVERSE}
\newcommand{\isXVIIIvVI}{\VERSE  Et relinquentur simul avibus montium et bestiis terræ ; et æstate perpetua erunt super eum volucres, et omnes bestiæ terræ super illum hiemabunt. \EVERSE}
\newcommand{\isXVIIIvVII}{\VERSE  In tempore illo deferetur munus Domino exercituum a populo divulso et dilacerato, a populo terribili, post quem non fuit alius ; a gente exspectante, exspectante et conculcata, cujus diripuerunt flumina terram ejus ; ad locum nominis Domini exercituum, montem Sion. \EVERSE}
\newcommand{\isXIXvI}{\VERSE  Onus Ægypti. Ecce Dominus ascendet super nubem levem, et ingredietur Ægyptum, et commovebuntur simulacra Ægypti a facie ejus, et cor Ægypti tabescet in medio ejus, \EVERSE}
\newcommand{\isXIXvII}{\VERSE  et concurrere faciam Ægyptios adversus Ægyptios ; et pugnabit vir contra fratrem suum, et vir contra amicum suum, civitas adversus civitatem, regnum adversus regnum. \EVERSE}
\newcommand{\isXIXvIII}{\VERSE  Et dirumpetur spiritus Ægypti in visceribus ejus, et consilium ejus præcipitabo ; et interrogabunt simulacra sua, et divinos suos, et pythones, et ariolos. \EVERSE}
\newcommand{\isXIXvIV}{\VERSE  Et tradam Ægyptum in manu dominorum crudelium, et rex fortis dominabitur eorum, ait Dominus Deus exercituum. \EVERSE}
\newcommand{\isXIXvV}{\VERSE  Et arescet aqua de mari, et fluvius desolabitur atque siccabitur. \EVERSE}
\newcommand{\isXIXvVI}{\VERSE  Et deficient flumina, attenuabuntur et siccabuntur rivi aggerum, calamus et juncus marcescet. \EVERSE}
\newcommand{\isXIXvVII}{\VERSE  Nudabitur alveus rivi a fonte suo, et omnis sementis irrigua siccabitur, arescet, et non erit. \EVERSE}
\newcommand{\isXIXvVIII}{\VERSE  Et mœrebunt piscatores, et lugebunt omnes mittentes in flumen hamum ; et expandentes rete super faciem aquarum emarcescent. \EVERSE}
\newcommand{\isXIXvIX}{\VERSE  Confundentur qui operabantur linum, pectentes et texentes subtilia. \EVERSE}
\newcommand{\isXIXvX}{\VERSE  Et erunt irrigua ejus flaccentia : omnes qui faciebant lacunas ad capiendos pisces. \EVERSE}
\newcommand{\isXIXvXI}{\VERSE  Stulti principes Taneos, sapientes consiliarii Pharaonis dederunt consilium insipiens. Quomodo dicetis Pharaoni : Filius sapientium ego, filius regum antiquorum ? \EVERSE}
\newcommand{\isXIXvXII}{\VERSE  Ubi nunc sunt sapientes tui ? annuntient tibi, et indicent quid cogitaverit Dominus exercituum super Ægyptum. \EVERSE}
\newcommand{\isXIXvXIII}{\VERSE  Stulti facti sunt principes Taneos, emarcuerunt principes Mempheos ; deceperunt Ægyptum, angulum populorum ejus. \EVERSE}
\newcommand{\isXIXvXIV}{\VERSE  Dominus miscuit in medio ejus spiritum vertiginis ; et errare fecerunt Ægyptum in omni opere suo, sicut errat ebrius et vomens. \EVERSE}
\newcommand{\isXIXvXV}{\VERSE  Et non erit Ægypto opus quod faciat caput et caudam, incurvantem et refrenantem. \EVERSE}
\newcommand{\isXIXvXVI}{\VERSE  In die illa erit Ægyptus quasi mulieres ; et stupebunt, et timebunt a facie commotionis manus Domini exercituum, quam ipse movebit super eam. \EVERSE}
\newcommand{\isXIXvXVII}{\VERSE  Et erit terra Juda Ægypto in pavorem ; omnis qui illius fuerit recordatus pavebit a facie consilii Domini exercituum, quod ipse cogitavit super eam. \EVERSE}
\newcommand{\isXIXvXVIII}{\VERSE  In die illa erunt quinque civitates in terra Ægypti loquentes lingua Chanaan, et jurantes per Dominum exercituum : Civitas solis vocabitur una. \EVERSE}
\newcommand{\isXIXvXIX}{\VERSE  In die illa erit altare Domini in medio terræ Ægypti, et titulus Domini juxta terminum ejus. \EVERSE}
\newcommand{\isXIXvXX}{\VERSE  Erit in signum et in testimonium Domino exercituum in terra Ægypti ; clamabunt enim ad Dominum a facie tribulationis, et mittet eis salvatorem et propugnatorem qui liberet eos. \EVERSE}
\newcommand{\isXIXvXXI}{\VERSE  Et cognoscetur Dominus ab Ægypto, et cognoscent Ægyptii Dominum in die illa ; et colent eum in hostiis et in muneribus ; et vota vovebunt Domino, et solvent. \EVERSE}
\newcommand{\isXIXvXXII}{\VERSE  Et percutiet Dominus Ægyptum plaga, et sanabit eam ; et revertentur ad Dominum, et placabitur eis, et sanabit eos. \EVERSE}
\newcommand{\isXIXvXXIII}{\VERSE  In die illa erit via de Ægypto in Assyrios ; et intrabit Assyrius Ægyptum, et Ægyptius in Assyrios, et servient Ægyptii Assur. \EVERSE}
\newcommand{\isXIXvXXIV}{\VERSE  In die illa erit Israël tertius Ægyptio et Assyrio ; benedictio in medio terræ \EVERSE}
\newcommand{\isXIXvXXV}{\VERSE  cui benedixit Dominus exercituum, dicens : Benedictus populus meus Ægypti, et opus manuum mearum Assyrio ; hæreditas autem mea Israël. \EVERSE}
\newcommand{\isXXvI}{\VERSE  In anno quo ingressus est Thathan in Azotum, cum misisset eum Sargon, rex Assyriorum, et pugnasset contra Azotum, et cepisset eam : \EVERSE}
\newcommand{\isXXvII}{\VERSE  in tempore illo locutus est Dominus in manu Isaiæ, filii Amos, dicens : Vade, et solve saccum de lumbis tuis, et calceamenta tua tolle de pedibus tuis. Et fecit sic, vadens nudus et discalceatus. \EVERSE}
\newcommand{\isXXvIII}{\VERSE  Et dixit Dominus : Sicut ambulavit servus meus Isaias nudus et discalceatus, trium annorum signum et portentum erit super Ægyptum et super Æthiopiam ; \EVERSE}
\newcommand{\isXXvIV}{\VERSE  sic minabit rex Assyriorum captivitatem Ægypti, et transmigrationem Æthiopiæ, juvenum et senum, nudam et discalceatam, discoopertis natibus, ad ignominiam Ægypti. \EVERSE}
\newcommand{\isXXvV}{\VERSE  Et timebunt, et confundentur ab Æthiopia spe sua, et ab Ægypto gloria sua. \EVERSE}
\newcommand{\isXXvVI}{\VERSE  Et dicet habitator insulæ hujus in die illa : Ecce hæc erat spes nostra, ad quos confugimus in auxilium, ut liberarent nos a facie regis Assyriorum : et quomodo effugere poterimus nos ? \EVERSE}
\newcommand{\isXXIvI}{\VERSE  Onus deserti maris. Sicut turbines ab africo veniunt, de deserto venit, de terra horribili. \EVERSE}
\newcommand{\isXXIvII}{\VERSE  Visio dura nuntiata est mihi : qui incredulus est infideliter agit ; et qui depopulator est vastat. Ascende, Ælam ; obside, Mede ; omnem gemitum ejus cessare feci. \EVERSE}
\newcommand{\isXXIvIII}{\VERSE  Propterea repleti sunt lumbi mei dolore ; angustia possedit me sicut angustia parturientis ; corrui cum audirem, conturbatus sum cum viderem. \EVERSE}
\newcommand{\isXXIvIV}{\VERSE  Emarcuit cor meum ; tenebræ stupefecerunt me : Babylon dilecta mea posita est mihi in miraculum. \EVERSE}
\newcommand{\isXXIvV}{\VERSE  Pone mensam, contemplare in specula comedentes et bibentes : surgite, principes, arripite clypeum. \EVERSE}
\newcommand{\isXXIvVI}{\VERSE  Hæc enim dixit mihi Dominus : Vade, et pone speculatorem, et quodcumque viderit annuntiet. \EVERSE}
\newcommand{\isXXIvVII}{\VERSE  Et vidit currum duorum equitum, ascensorem asini, et ascensorem cameli ; et contemplatus est diligenter multo intuitu. \EVERSE}
\newcommand{\isXXIvVIII}{\VERSE  Et clamavit leo : Super speculam Domini ego sum, stans jugiter per diem ; et super custodiam meam ego sum, stans totis noctibus. \EVERSE}
\newcommand{\isXXIvIX}{\VERSE  Ecce iste venit ascensor vir bigæ equitum ; et respondit, et dixit : Cecidit, cecidit Babylon, et omnia sculptilia deorum ejus contrita sunt in terram. \EVERSE}
\newcommand{\isXXIvX}{\VERSE  Tritura mea et filii areæ meæ, quæ audivi a Domino exercituum, Deo Israël, annuntiavi vobis. \EVERSE}
\newcommand{\isXXIvXI}{\VERSE  Onus Duma. Ad me clamat ex Seir : Custos, quid de nocte ? custos, quid de nocte ? \EVERSE}
\newcommand{\isXXIvXII}{\VERSE  Dixit custos : Venit mane et nox ; si quæritis, quærite ; convertimini, venite. \EVERSE}
\newcommand{\isXXIvXIII}{\VERSE  Onus in Arabia. In saltu ad vesperam dormietis, in semitis Dedanim. \EVERSE}
\newcommand{\isXXIvXIV}{\VERSE  Occurrentes sitienti ferte aquam, qui habitatis terram austri ; cum panibus occurrite fugienti. \EVERSE}
\newcommand{\isXXIvXV}{\VERSE  A facie enim gladiorum fugerunt, a facie gladii imminentis, a facie arcus extenti, a facie gravis prælii. \EVERSE}
\newcommand{\isXXIvXVI}{\VERSE  Quoniam hæc dicit Dominus ad me : Adhuc in uno anno, quasi in anno mercenarii, et auferetur omnis gloria Cedar. \EVERSE}
\newcommand{\isXXIvXVII}{\VERSE  Et reliquiæ numeri sagittariorum fortium de filiis Cedar imminuentur ; Dominus enim Deus Israël locutus est. \EVERSE}
\newcommand{\isXXIIvI}{\VERSE  Onus vallis Visionis. Quidnam quoque tibi est, quia ascendisti et tu omnis in tecta ? \EVERSE}
\newcommand{\isXXIIvII}{\VERSE  Clamoris plena, urbs frequens, civitas exsultans ; interfecti tui, non interfecti gladio, nec mortui in bello. \EVERSE}
\newcommand{\isXXIIvIII}{\VERSE  Cuncti principes tui fugerunt simul dureque ligati sunt ; omnes qui inventi sunt vincti sunt pariter ; procul fugerunt. \EVERSE}
\newcommand{\isXXIIvIV}{\VERSE  Propterea dixi : Recedite a me : amare flebo ; nolite incumbere ut consolemini me super vastitate filiæ populi mei ; \EVERSE}
\newcommand{\isXXIIvV}{\VERSE  dies enim interfectionis, et conculcationis, et fletuum, Domino Deo exercituum, in valle Visionis, scrutans murum, et magnificus super montem. \EVERSE}
\newcommand{\isXXIIvVI}{\VERSE  Et Ælam sumpsit pharetram, currum hominis equitis, et parietem nudavit clypeus. \EVERSE}
\newcommand{\isXXIIvVII}{\VERSE  Et erunt electæ valles tuæ plenæ quadrigarum, et equites ponent sedes suas in porta. \EVERSE}
\newcommand{\isXXIIvVIII}{\VERSE  Et revelabitur operimentum Judæ, et videbis in die illa armamentarium domus saltus. \EVERSE}
\newcommand{\isXXIIvIX}{\VERSE  Et scissuras civitatis David videbitis, quia multiplicatæ sunt ; et congregastis aquas piscinæ inferioris, \EVERSE}
\newcommand{\isXXIIvX}{\VERSE  et domos Jerusalem numerastis, et destruxistis domos ad muniendum murum. \EVERSE}
\newcommand{\isXXIIvXI}{\VERSE  Et lacum fecistis inter duos muros ad aquam piscinæ veteris ; et non suspexistis ad eum qui fecerat eam, et operatorem ejus de longe non vidistis. \EVERSE}
\newcommand{\isXXIIvXII}{\VERSE  Et vocabit Dominus Deus exercituum in die illa ad fletum, et ad planctum, ad calvitium, et ad cingulum sacci ; \EVERSE}
\newcommand{\isXXIIvXIII}{\VERSE  et ecce gaudium et lætitia, occidere vitulos et jugulare arietes, comedere carnes, et bibere vinum : comedamus et bibamus, cras enim moriemur. \EVERSE}
\newcommand{\isXXIIvXIV}{\VERSE  Et revelata est in auribus meis vox Domini exercituum : Si dimittetur iniquitas hæc vobis donec moriamini, dicit Dominus Deus exercituum. \EVERSE}
\newcommand{\isXXIIvXV}{\VERSE  Hæc dicit Dominus Deus exercituum : Vade, ingredere ad eum qui habitat in tabernaculo, ad Sobnam, præpositum templi, et dices ad eum : \EVERSE}
\newcommand{\isXXIIvXVI}{\VERSE  Quid tu hic, aut quasi quis hic ? quia excidisti tibi hic sepulchrum, excidisti in excelso memoriale diligenter, in petra tabernaculum tibi. \EVERSE}
\newcommand{\isXXIIvXVII}{\VERSE  Ecce Dominus asportari te faciet, sicut asportatur gallus gallinaceus ; et quasi amictum, sic sublevabit te. \EVERSE}
\newcommand{\isXXIIvXVIII}{\VERSE  Coronas coronabit te tribulatione ; quasi pilam mittet te in terram latam et spatiosam ; ibi morieris, et ibi erit currus gloriæ tuæ, ignominia domus domini tui. \EVERSE}
\newcommand{\isXXIIvXIX}{\VERSE  Et expellam te de statione tua, et de ministerio tuo deponam te. \EVERSE}
\newcommand{\isXXIIvXX}{\VERSE  Et erit in die illa : vocabo servum meum Eliacim, filium Helciæ, \EVERSE}
\newcommand{\isXXIIvXXI}{\VERSE  et induam illum tunica tua, et cingulo tuo confortabo eum, et potestatem tuam dabo in manu ejus ; et erit quasi pater habitantibus Jerusalem et domui Juda. \EVERSE}
\newcommand{\isXXIIvXXII}{\VERSE  Et dabo clavem domus David super humerum ejus ; et aperiet, et non erit qui claudat ; et claudet, et non erit qui aperiat. \EVERSE}
\newcommand{\isXXIIvXXIII}{\VERSE  Et figam illum paxillum in loco fideli, et erit in solium gloriæ domui patris ejus. \EVERSE}
\newcommand{\isXXIIvXXIV}{\VERSE  Et suspendent super eum omnem gloriam domus patris ejus ; vasorum diversa genera, omne vas parvulum, a vasis craterarum usque ad omne vas musicorum. \EVERSE}
\newcommand{\isXXIIvXXV}{\VERSE  In die illa, dicit Dominus exercituum, auferetur paxillus qui fixus fuerat in loco fideli, et frangetur, et cadet, et peribit quod pependerat in eo, quia Dominus locutus est. \EVERSE}
\newcommand{\isXXIIIvI}{\VERSE  Onus Tyri. Ululate, naves maris, quia vastata est domus unde venire consueverant : de terra Cethim revelatum est eis. \EVERSE}
\newcommand{\isXXIIIvII}{\VERSE  Tacete, qui habitatis in insula ; negotiatores Sidonis, transfretantes mare, repleverunt te. \EVERSE}
\newcommand{\isXXIIIvIII}{\VERSE  In aquis multis semen Nili ; messis fluminis fruges ejus : et facta est negotiatio gentium. \EVERSE}
\newcommand{\isXXIIIvIV}{\VERSE  Erubesce, Sidon ; ait enim mare, fortitudo maris, dicens : Non parturivi, et non peperi, et non enutrivi juvenes, nec ad incrementum perduxi virgines. \EVERSE}
\newcommand{\isXXIIIvV}{\VERSE  Cum auditum fuerit in Ægypto, dolebunt cum audierint de Tiro. \EVERSE}
\newcommand{\isXXIIIvVI}{\VERSE  Transite maria, ululate, qui habitatis in insula ! \EVERSE}
\newcommand{\isXXIIIvVII}{\VERSE  Numquid non vestra hæc est, quæ gloriabatur a diebus pristinis in antiquitate sua ? Ducent eam pedes sui longe ad peregrinandum. \EVERSE}
\newcommand{\isXXIIIvVIII}{\VERSE  Quis cogitavit hoc super Tyrum quondam coronatam, cujus negotiatores principes, institores ejus inclyti terræ ? \EVERSE}
\newcommand{\isXXIIIvIX}{\VERSE  Dominus exercituum cogitavit hoc, ut detraheret superbiam omnis gloriæ, et ad ignominiam deduceret universos inclytos terræ. \EVERSE}
\newcommand{\isXXIIIvX}{\VERSE  Transi terram tuam quasi flumen, filia maris ! non est cingulum ultra tibi. \EVERSE}
\newcommand{\isXXIIIvXI}{\VERSE  Manum suam extendit super mare ; conturbavit regna. Dominus mandavit adversus Chanaan, ut contereret fortes ejus ; \EVERSE}
\newcommand{\isXXIIIvXII}{\VERSE  et dixit : Non adjicies ultra ut glorieris, calumniam sustinens virgo filia Sidonis : in Cethim consurgens transfreta : ibi quoque non erit requies tibi. \EVERSE}
\newcommand{\isXXIIIvXIII}{\VERSE  Ecce terra Chaldæorum, talis populus non fuit : Assur fundavit eam ; in captivitatem traduxerunt robustos ejus, suffoderunt domos ejus, posuerunt eam in ruinam. \EVERSE}
\newcommand{\isXXIIIvXIV}{\VERSE  Ululate, naves maris, quia devastata est fortitudo vestra. \EVERSE}
\newcommand{\isXXIIIvXV}{\VERSE  Et erit in die illa : in oblivione eris, o Tyre ! septuaginta annis, sicut dies regis unius ; post septuaginta autem annos erit Tyro quasi canticum meretricis : \EVERSE}
\newcommand{\isXXIIIvXVI}{\VERSE  Sume citharam, circui civitatem, meretrix oblivioni tradita : bene cane, frequenta canticum, ut memoria tui sit. \EVERSE}
\newcommand{\isXXIIIvXVII}{\VERSE  Et erit post septuaginta annos : visitabit Dominus Tyrum, et reducet eam ad mercedes suas, et rursum fornicabitur cum universis regnis terræ super faciem terræ ; \EVERSE}
\newcommand{\isXXIIIvXVIII}{\VERSE  et erunt negotiationes ejus et mercedes ejus sanctificatæ Domino : non condentur neque reponentur, quia his qui habitaverint coram Domino erit negotiatio ejus, ut manducent in saturitatem, et vestiantur usque ad vetustatem. \EVERSE}
\newcommand{\isXXIVvI}{\VERSE  Ecce Dominus dissipabit terram : et nudabit eam, et affliget faciem ejus, et disperget habitatores ejus. \EVERSE}
\newcommand{\isXXIVvII}{\VERSE  Et erit sicut populus, sic sacerdos ; et sicut servus, sic dominus ejus ; sicut ancilla, sic domina ejus ; sicut emens, sic ille qui vendit ; sicut fœnerator, sic is qui mutuum accipit ; sicut qui repetit, sic qui debet. \EVERSE}
\newcommand{\isXXIVvIII}{\VERSE  Dissipatione dissipabitur terra, et direptione prædabitur ; Dominus enim locutus est verbum hoc. \EVERSE}
\newcommand{\isXXIVvIV}{\VERSE  Luxit, et defluxit terra, et infirmata est ; defluxit orbis, infirmata est altitudo populi terræ. \EVERSE}
\newcommand{\isXXIVvV}{\VERSE  Et terra infecta est ab habitatoribus suis, quia transgressi sunt leges, mutaverunt jus, dissipaverunt fœdus sempiternum. \EVERSE}
\newcommand{\isXXIVvVI}{\VERSE  Propter hoc maledictio vorabit terram, et peccabunt habitatores ejus ; ideoque insanient cultores ejus, et relinquentur homines pauci. \EVERSE}
\newcommand{\isXXIVvVII}{\VERSE  Luxit vindemia, infirmata est vitis, ingemuerunt omnes qui lætabantur corde ; \EVERSE}
\newcommand{\isXXIVvVIII}{\VERSE  cessavit gaudium tympanorum, quievit sonitus lætantium, conticuit dulcedo citharæ. \EVERSE}
\newcommand{\isXXIVvIX}{\VERSE  Cum cantico non bibent vinum ; amara erit potio bibentibus illam. \EVERSE}
\newcommand{\isXXIVvX}{\VERSE  Attrita est civitas vanitatis, clausa est omnis domus, nullo introëunte. \EVERSE}
\newcommand{\isXXIVvXI}{\VERSE  Clamor erit super vino in plateis, deserta est omnia lætitia, translatum est gaudium terræ. \EVERSE}
\newcommand{\isXXIVvXII}{\VERSE  Relicta est in urbe solitudo, et calamitas opprimet portas. \EVERSE}
\newcommand{\isXXIVvXIII}{\VERSE  Quia hæc erunt in medio terræ in medio populorum, quomodo si paucæ olivæ quæ remanserunt excutiantur ex olea et racemi, cum fuerit finita vindemia. \EVERSE}
\newcommand{\isXXIVvXIV}{\VERSE  Hi levabunt vocem suam, atque laudabunt : cum glorificatus fuerit Dominus, hinnient de mari. \EVERSE}
\newcommand{\isXXIVvXV}{\VERSE  Propter hoc in doctrinis glorificate Dominum ; in insulis maris nomen Domini Dei Israël. \EVERSE}
\newcommand{\isXXIVvXVI}{\VERSE  A finibus terræ laudes audivimus, gloriam Justi. Et dixi : Secretum meum mihi, secretum meum mihi. Væ mihi ! prævaricantes prævaricati sunt, et prævaricatione transgressorum prævaricati sunt. \EVERSE}
\newcommand{\isXXIVvXVII}{\VERSE  Formido, et fovea, et laqueus super te, qui habitator es terræ. \EVERSE}
\newcommand{\isXXIVvXVIII}{\VERSE  Et erit : qui fugerit a voce formidinis cadet in foveam ; et qui se explicaverit de fovea tenebitur laqueo ; quia cataractæ de excelsis apertæ sunt et concutientur fundamenta terræ. \EVERSE}
\newcommand{\isXXIVvXIX}{\VERSE  Confractione confringetur terra, contritione conteretur terra, commotione commovebitur terra ; \EVERSE}
\newcommand{\isXXIVvXX}{\VERSE  agitatione agitabitur terra sicut ebrius, et auferetur quasi tabernaculum unius noctis ; et gravabit eam iniquitas sua, et corruet, et non adjiciet ut resurgat. \EVERSE}
\newcommand{\isXXIVvXXI}{\VERSE  Et erit : in die illa visitabit Dominus super militiam cæli in excelso, et super reges terræ qui sunt super terram ; \EVERSE}
\newcommand{\isXXIVvXXII}{\VERSE  et congregabuntur in congregatione unius fascis in lacum, et claudentur ibi in carcere, et post multos dies visitabuntur. \EVERSE}
\newcommand{\isXXIVvXXIII}{\VERSE  Et erubescet luna, et confundetur sol, cum regnaverit Dominus exercituum in monte Sion et in Jerusalem et in conspectu senum suorum fuerit glorificatus. \EVERSE}
\newcommand{\isXXVvI}{\VERSE  Domine, Deus meus es tu ; exaltabo te, et confitebor nomini tuo : quoniam fecisti mirabilia, cogitationes antiquas fideles. Amen. \EVERSE}
\newcommand{\isXXVvII}{\VERSE  Quia posuisti civitatem in tumulum, urbem fortem in ruinam, domum alienorum : ut non sit civitas, et in sempiternum non ædificetur. \EVERSE}
\newcommand{\isXXVvIII}{\VERSE  Super hoc laudabit te populus fortis ; civitas gentium robustarum timebit te : \EVERSE}
\newcommand{\isXXVvIV}{\VERSE  quia factus es fortitudo pauperi, fortitudo egeno in tribulatione sua, spes a turbine, umbraculum ab æstu ; spiritus enim robustorum quasi turbo impellens parietem. \EVERSE}
\newcommand{\isXXVvV}{\VERSE  Sicut æstus in siti, tumultum alienorum humiliabis ; et quasi calore sub nube torrente, propaginem fortium marcescere facies. \EVERSE}
\newcommand{\isXXVvVI}{\VERSE  Et faciet Dominus exercituum omnibus populis in monte hoc convivium pinguium, convivium vindemiæ, pinguium medullatorum, vindemiæ defæcatæ. \EVERSE}
\newcommand{\isXXVvVII}{\VERSE  Et præcipitabit in monte isto faciem vinculi colligati super omnes populos, et telam quam orditus est super omnes nationes. \EVERSE}
\newcommand{\isXXVvVIII}{\VERSE  Præcipitabit mortem in sempiternum ; et auferet Dominus Deus lacrimam ab omni facie, et opprobrium populi sui auferet de universa terra : quia Dominus locutus est. \EVERSE}
\newcommand{\isXXVvIX}{\VERSE  Et dicet in die illa : Ecce Deus noster iste ; exspectavimus eum, et salvabit nos ; iste Dominus, sustinuimus eum : exsultabimus, et lætabimur in salutari ejus. \EVERSE}
\newcommand{\isXXVvX}{\VERSE  Quia requiescet manus Domini in monte isto ; et triturabitur Moab sub eo, sicuti teruntur paleæ in plaustro. \EVERSE}
\newcommand{\isXXVvXI}{\VERSE  Et extendet manus suas sub eo sicut extendit natans ad natandum ; et humiliabit gloriam ejus cum allisione manuum ejus. \EVERSE}
\newcommand{\isXXVvXII}{\VERSE  Et munimenta sublimium murorum tuorum concident, et humiliabuntur, et detrahentur in terram usque ad pulverem. \EVERSE}
\newcommand{\isXXVIvI}{\VERSE  In die illa cantabitur canticum istud in terra Juda : Urbs fortitudinis nostræ Sion ; salvator ponetur in ea murus et antemurale. \EVERSE}
\newcommand{\isXXVIvII}{\VERSE  Aperite portas, et ingrediatur gens justa, custodiens veritatem. \EVERSE}
\newcommand{\isXXVIvIII}{\VERSE  Vetus error abiit : servabis pacem ; pacem, quia in te speravimus. \EVERSE}
\newcommand{\isXXVIvIV}{\VERSE  Sperastis in Domino in sæculis æternis ; in Domino Deo forti in perpetuum. \EVERSE}
\newcommand{\isXXVIvV}{\VERSE  Quia incurvabit habitantes in excelso ; civitatem sublimem humiliabit : humiliabit eam usque ad terram, detrahet eam usque ad pulverem. \EVERSE}
\newcommand{\isXXVIvVI}{\VERSE  Conculcabit eam pes, pedes pauperis, gressus egenorum. \EVERSE}
\newcommand{\isXXVIvVII}{\VERSE  Semita justi recta est, rectus callis justi ad ambulandum. \EVERSE}
\newcommand{\isXXVIvVIII}{\VERSE  Et in semita judiciorum tuorum, Domine, sustinuimus te : nomen tuum et memoriale tuum in desiderio animæ. \EVERSE}
\newcommand{\isXXVIvIX}{\VERSE  Anima mea desideravit te in nocte, sed et spiritu meo in præcordiis meis de mane vigilabo ad te. Cum feceris judicia tua in terra, justitiam discent habitatores orbis. \EVERSE}
\newcommand{\isXXVIvX}{\VERSE  Misereamur impio, et non discet justitiam ; in terra sanctorum iniqua gessit, et non videbit gloriam Domini. \EVERSE}
\newcommand{\isXXVIvXI}{\VERSE  Domine, exaltetur manus tua, et non videant ; videant, et confundantur zelantes populi ; et ignis hostes tuos devoret. \EVERSE}
\newcommand{\isXXVIvXII}{\VERSE  Domine, dabis pacem nobis : omnia enim opera nostra operatus es nobis. \EVERSE}
\newcommand{\isXXVIvXIII}{\VERSE  Domine Deus noster, possederunt nos domini absque te ; tantum in te recordemur nominis tui. \EVERSE}
\newcommand{\isXXVIvXIV}{\VERSE  Morientes non vivant, gigantes non resurgant : propterea visitasti et contrivisti eos, et perdidisti omnem memoriam eorum. \EVERSE}
\newcommand{\isXXVIvXV}{\VERSE  Indulsisti genti, Domine, indulsisti genti, numquid glorificatus es ? elongasti omnes terminos terræ. \EVERSE}
\newcommand{\isXXVIvXVI}{\VERSE  Domine, in angustia requisierunt te, in tribulatione murmuris doctrina tua eis. \EVERSE}
\newcommand{\isXXVIvXVII}{\VERSE  Sicut quæ concipit, cum appropinquaverit ad partum, dolens clamat in doloribus suis, sic facti sumus a facie tua, Domine. \EVERSE}
\newcommand{\isXXVIvXVIII}{\VERSE  Concepimus, et quasi parturivimus, et peperimus spiritum. Salutes non fecimus in terra ; ideo non ceciderunt habitatores terræ. \EVERSE}
\newcommand{\isXXVIvXIX}{\VERSE  Vivent mortui tui, interfecti mei resurgent. Expergiscimini, et laudate, qui habitatis in pulvere, quia ros lucis ros tuus, et terram gigantum detrahes in ruinam. \EVERSE}
\newcommand{\isXXVIvXX}{\VERSE  Vade, populus meus, intra in cubicula tua ; claude ostia tua super te, abscondere modicum ad momentum, donec pertranseat indignatio. \EVERSE}
\newcommand{\isXXVIvXXI}{\VERSE  Ecce enim Dominus egredietur de loco suo, ut visitet iniquitatem habitatoris terræ contra eum ; et revelabit terra sanguinem suum, et non operiet ultra interfectos suos. \EVERSE}
\newcommand{\isXXVIIvI}{\VERSE  In die illa visitabit Dominus in gladio suo duro, et grandi, et forti, super Leviathan, serpentem vectem, et super Leviathan, serpentem tortuosum, et occidet cetum qui in mari est. \EVERSE}
\newcommand{\isXXVIIvII}{\VERSE  In die illa vinea meri cantabit ei. \EVERSE}
\newcommand{\isXXVIIvIII}{\VERSE  Ego Dominus qui servo eam ; repente propinabo ei. Ne forte visitetur contra eam, nocte et die servo eam. \EVERSE}
\newcommand{\isXXVIIvIV}{\VERSE  Indignatio non est mihi. Quis dabit me spinam et veprem in prælio ? gradiar super eam, succendam eam pariter. \EVERSE}
\newcommand{\isXXVIIvV}{\VERSE  An potius tenebit fortitudinem meam ? faciet pacem mihi, pacem faciet mihi. \EVERSE}
\newcommand{\isXXVIIvVI}{\VERSE  Qui ingrediuntur impetu ad Jacob, florebit et germinabit Israël, et implebunt faciem orbis semine. \EVERSE}
\newcommand{\isXXVIIvVII}{\VERSE  Numquid juxta plagam percutientis se percussit eum ? aut sicut occidit interfectos ejus, sic occisus est ? \EVERSE}
\newcommand{\isXXVIIvVIII}{\VERSE  In mensura contra mensuram, cum abjecta fuerit, judicabis eam ; meditatus est in spiritu suo duro per diem æstus. \EVERSE}
\newcommand{\isXXVIIvIX}{\VERSE  Idcirco super hoc dimittetur iniquitas domui Jacob ; et iste omnis fructus : ut auferatur peccatum ejus, cum posuerit omnes lapides altaris sicut lapides cineris allisos : non stabunt luci et delubra. \EVERSE}
\newcommand{\isXXVIIvX}{\VERSE  Civitas enim munita desolata erit ; speciosa relinquetur, et dimittetur quasi desertum ; ibi pascetur vitulus, et ibi accubabit, et consumet summitates ejus. \EVERSE}
\newcommand{\isXXVIIvXI}{\VERSE  In siccitate messes illius conterentur. Mulieres venientes, et docentes eam ; non est enim populus sapiens : propterea non miserebitur ejus qui fecit eum, et qui formavit eum non parcet ei. \EVERSE}
\newcommand{\isXXVIIvXII}{\VERSE  Et erit : in die illa percutiet Dominus ab alveo fluminis usque ad torrentem Ægypti ; et vos congregabimini unus et unus, filii Israël. \EVERSE}
\newcommand{\isXXVIIvXIII}{\VERSE  Et erit : in die illa clangetur in tuba magna ; et venient qui perditi fuerant de terra Assyriorum, et qui ejecti erant in terra Ægypti, et adorabunt Dominum in monte sancto in Jerusalem. \EVERSE}
\newcommand{\isXXVIIIvI}{\VERSE  Væ coronæ superbiæ, ebriis Ephraim, et flori decidenti, gloriæ exsultationis ejus, qui erant in vertice vallis pinguissimæ, errantes a vino. \EVERSE}
\newcommand{\isXXVIIIvII}{\VERSE  Ecce validus et fortis Dominus sicut impetus grandinis ; turbo confringens, sicut impetus aquarum multarum inundantium et emissarum super terram spatiosam. \EVERSE}
\newcommand{\isXXVIIIvIII}{\VERSE  Pedibus conculcabitur corona superbiæ ebriorum Ephraim. \EVERSE}
\newcommand{\isXXVIIIvIV}{\VERSE  Et erit flos decidens gloriæ exsultationis ejus, qui est super verticem vallis pinguium, quasi temporaneum ante maturitatem autumni, quod, cum aspexerit videns, statim ut manu tenuerit, devorabit illud. \EVERSE}
\newcommand{\isXXVIIIvV}{\VERSE  In die illa erit Dominus exercituum corona gloriæ, et sertum exsultationis residuo populi sui ; \EVERSE}
\newcommand{\isXXVIIIvVI}{\VERSE  et spiritus judicii sedenti super judicium, et fortitudo revertentibus de bello ad portam. \EVERSE}
\newcommand{\isXXVIIIvVII}{\VERSE  Verum hi quoque præ vino nescierunt, et præ ebrietate erraverunt ; sacerdos et propheta nescierunt præ ebrietate ; absorpti sunt a vino, erraverunt in ebrietate, nescierunt videntem, ignoraverunt judicium. \EVERSE}
\newcommand{\isXXVIIIvVIII}{\VERSE  Omnes enim mensæ repletæ sunt vomitu sordiumque, ita ut non esset ultra locus. \EVERSE}
\newcommand{\isXXVIIIvIX}{\VERSE  Quem docebit scientiam ? et quem intelligere faciet auditum ? Ablactatos a lacte, avulsos ab uberibus. \EVERSE}
\newcommand{\isXXVIIIvX}{\VERSE  Quia manda, remanda ; manda, remanda ; exspecta, reexspecta ; exspecta, reexspecta ; modicum ibi, modicum ibi. \EVERSE}
\newcommand{\isXXVIIIvXI}{\VERSE  In loquela enim labii, et lingua altera loquetur ad populum istum. \EVERSE}
\newcommand{\isXXVIIIvXII}{\VERSE  Cui dixit : Hæc est requies mea, reficite lassum ; et hoc est meum refrigerium : et noluerunt audire. \EVERSE}
\newcommand{\isXXVIIIvXIII}{\VERSE  Et erit eis verbum Domini : Manda, remanda ; manda, remanda ; exspecta, reexspecta ; exspecta, reexspecta ; modicum ibi, modicum ibi ; ut vadant, et cadant retrorsum, et conterantur, et illaqueentur, et capiantur. \EVERSE}
\newcommand{\isXXVIIIvXIV}{\VERSE  Propter hoc audite verbum Domini, viri illusores, qui dominamini super populum meum, qui est in Jerusalem. \EVERSE}
\newcommand{\isXXVIIIvXV}{\VERSE  Dixistis enim : Percussimus fœdus cum morte, et cum inferno fecimus pactum : flagellum inundans cum transierit, non veniet super nos quia posuimus mendacium spem nostram, et mendacio protecti sumus. \EVERSE}
\newcommand{\isXXVIIIvXVI}{\VERSE  Idcirco hæc dicit Dominus Deus : Ecce ego mittam in fundamentis Sion lapidem, lapidem probatum, angularem, pretiosum, in fundamento fundatum ; qui crediderit, non festinet. \EVERSE}
\newcommand{\isXXVIIIvXVII}{\VERSE  Et ponam in pondere judicium, et justitiam in mensura ; et subvertet grando spem mendacii, et protectionem aquæ inundabunt. \EVERSE}
\newcommand{\isXXVIIIvXVIII}{\VERSE  Et delebitur fœdus vestrum cum morte, et pactum vestrum cum inferno non stabit : flagellum inundans cum transierit, eritis ei in conculcationem. \EVERSE}
\newcommand{\isXXVIIIvXIX}{\VERSE  Quandocumque pertransierit, tollet vos, quoniam in mane diluculo pertransibit in die et in nocte ; et tantummodo sola vexatio intellectum dabit auditui. \EVERSE}
\newcommand{\isXXVIIIvXX}{\VERSE  Coangustatum est enim stratum, ita ut alter decidat ; et pallium breve utrumque operire non potest. \EVERSE}
\newcommand{\isXXVIIIvXXI}{\VERSE  Sicut enim in monte divisionum stabit Dominus ; sicut in valle quæ est in Gabaon irascetur, ut faciat opus suum, alienum opus ejus : ut operetur opus suum, peregrinum est opus ejus ab eo. \EVERSE}
\newcommand{\isXXVIIIvXXII}{\VERSE  Et nunc nolite illudere, ne forte constringantur vincula vestra ; consummationem enim et abbreviationem audivi a Domino Deo exercituum, super universam terram. \EVERSE}
\newcommand{\isXXVIIIvXXIII}{\VERSE  Auribus percipite, et audite vocem meam : attendite, et audite eloquium meum. \EVERSE}
\newcommand{\isXXVIIIvXXIV}{\VERSE  Numquid tota die arabit arans ut serat ? proscindet et sarriet humum suam ? \EVERSE}
\newcommand{\isXXVIIIvXXV}{\VERSE  Nonne cum adæquaverit faciem ejus, seret gith et cyminum sparget ? et ponet triticum per ordinem, et hordeum, et milium, et viciam in finibus suis ? \EVERSE}
\newcommand{\isXXVIIIvXXVI}{\VERSE  Et erudiet illum in judicio ; Deus suus docebit illum. \EVERSE}
\newcommand{\isXXVIIIvXXVII}{\VERSE  Non enim in serris triturabitur gith, nec rota plaustri super cyminum circuibit ; sed in virga excutietur gith, et cyminum in baculo. \EVERSE}
\newcommand{\isXXVIIIvXXVIII}{\VERSE  Panis autem comminuetur ; verum non in perpetuum triturans triturabit illum, neque vexabit eum rota plaustri, neque ungulis suis comminuet eum. \EVERSE}
\newcommand{\isXXVIIIvXXIX}{\VERSE  Et hoc a Domino Deo exercituum exivit, ut mirabile faceret consilium, et magnificaret justitiam. \EVERSE}
\newcommand{\isXXIXvI}{\VERSE  Væ Ariel, Ariel civitas, quam expugnavit David ! additus est annus ad annum : solemnitates evolutæ sunt. \EVERSE}
\newcommand{\isXXIXvII}{\VERSE  Et circumvallabo Ariel, et erit tristis et mœrens, et erit mihi quasi Ariel. \EVERSE}
\newcommand{\isXXIXvIII}{\VERSE  Et circumdabo quasi sphæram in circuitu tuo, et jaciam contra te aggerem, et munimenta ponam in obsidionem tuam. \EVERSE}
\newcommand{\isXXIXvIV}{\VERSE  Humiliaberis, de terra loqueris, et de humo audietur eloquium tuum ; et erit quasi pythonis de terra vox tua, et de humo eloquium tuum mussitabit. \EVERSE}
\newcommand{\isXXIXvV}{\VERSE  Et erit sicut pulvis tenuis multitudo ventilantium te, et sicut favilla pertransiens multitudo eorum qui contra te prævaluerunt ; \EVERSE}
\newcommand{\isXXIXvVI}{\VERSE  eritque repente confestim. A Domino exercituum visitabitur in tonitruo, et commotione terræ, et voce magna turbinis et tempestatis, et flammæ ignis devorantis. \EVERSE}
\newcommand{\isXXIXvVII}{\VERSE  Et erit sicut somnium visionis nocturnæ multitudo omnium gentium quæ dimicaverunt contra Ariel, et omnes qui militaverunt, et obsederunt, et prævaluerunt adversus eam. \EVERSE}
\newcommand{\isXXIXvVIII}{\VERSE  Et sicut somniat esuriens, et comedit, cum autem fuerit expergefactus, vacua est anima ejus ; et sicut somniat sitiens et bibit, et postquam fuerit expergefactus, lassus adhuc sitit, et anima ejus vacua est : sic erit multitudo omnium gentium quæ dimicaverunt contra montem Sion. \EVERSE}
\newcommand{\isXXIXvIX}{\VERSE  Obstupescite et admiramini ; fluctuate et vacillate ; inebriamini, et non a vino ; movemini, et non ab ebrietate. \EVERSE}
\newcommand{\isXXIXvX}{\VERSE  Quoniam miscuit vobis Dominus spiritum soporis ; claudet oculos vestros : prophetas et principes vestros, qui vident visiones, operiet. \EVERSE}
\newcommand{\isXXIXvXI}{\VERSE  Et erit vobis visio omnium sicut verba libri signati, quem cum dederint scienti litteras, dicent : Lege istum : et respondebit : Non possum, signatus est enim. \EVERSE}
\newcommand{\isXXIXvXII}{\VERSE  Et dabitur liber nescienti litteras, diceturque ei : Lege ; et respondebit : Nescio litteras. \EVERSE}
\newcommand{\isXXIXvXIII}{\VERSE  Et dixit Dominus : Eo quod appropinquat populus iste ore suo, et labiis suis glorificat me, cor autem ejus longe est a me, et timuerunt me mandato hominum et doctrinis, \EVERSE}
\newcommand{\isXXIXvXIV}{\VERSE  ideo ecce ego addam ut admirationem faciam populo huic miraculo grandi et stupendo ; peribit enim sapientia a sapientibus ejus, et intellectus prudentium ejus abscondetur. \EVERSE}
\newcommand{\isXXIXvXV}{\VERSE  Væ qui profundi estis corde, ut a Domino abscondatis consilium ; quorum sunt in tenebris opera, et dicunt : Quis videt nos ? et quis novit nos ? \EVERSE}
\newcommand{\isXXIXvXVI}{\VERSE  Perversa est hæc vestra cogitatio ; quasi si lutum contra figulum cogitet, et dicat opus factori suo : Non fecisti me ; et figmentum dicat fictori suo : Non intelligis. \EVERSE}
\newcommand{\isXXIXvXVII}{\VERSE  Nonne adhuc in modico et in brevi convertetur Libanus in carmel, et carmel in saltum reputabitur ? \EVERSE}
\newcommand{\isXXIXvXVIII}{\VERSE  Et audient in die illa surdi verba libri, et de tenebris et caligine oculi cæcorum videbunt. \EVERSE}
\newcommand{\isXXIXvXIX}{\VERSE  Et addent mites in Domino lætitiam, et pauperes homines in Sancto Israël exsultabunt ; \EVERSE}
\newcommand{\isXXIXvXX}{\VERSE  quoniam defecit qui prævalebat, consummatus est illusor, et succisi sunt omnes qui vigilabant super iniquitatem, \EVERSE}
\newcommand{\isXXIXvXXI}{\VERSE  qui peccare faciebant homines in verbo, et arguentem in porta supplantabant, et declinaverunt frustra a justo. \EVERSE}
\newcommand{\isXXIXvXXII}{\VERSE  Propter hoc, hæc dicit Dominus ad domum Jacob, qui redemit Abraham : Non modo confundetur Jacob, nec modo vultus ejus erubescet ; \EVERSE}
\newcommand{\isXXIXvXXIII}{\VERSE  sed cum viderit filios suos, opera manuum mearum in medio sui sanctificantes nomen meum, et sanctificabunt Sanctum Jacob, et Deum Israël prædicabunt ; \EVERSE}
\newcommand{\isXXIXvXXIV}{\VERSE  et scient errantes spiritu intellectum, et mussitatores discent legem. \EVERSE}
\newcommand{\isXXXvI}{\VERSE  Væ filii desertores, dicit Dominus, ut faceretis consilium, et non ex me, et ordiremini telam, et non per spiritum meum, ut adderetis peccatum super peccatum ; \EVERSE}
\newcommand{\isXXXvII}{\VERSE  qui ambulatis ut descendatis in Ægyptum, et os meum non interrogastis, sperantes auxilium in fortitudine Pharaonis, et habentes fiduciam in umbra Ægypti ! \EVERSE}
\newcommand{\isXXXvIII}{\VERSE  Et erit vobis fortitudo Pharaonis in confusionem, et fiducia umbræ Ægypti in ignominiam. \EVERSE}
\newcommand{\isXXXvIV}{\VERSE  Erant enim in Tani principes tui, et nuntii tui usque ad Hanes pervenerunt. \EVERSE}
\newcommand{\isXXXvV}{\VERSE  Omnes confusi sunt super populo qui eis prodesse non potuit : non fuerunt in auxilium et in aliquam utilitatem, sed in confusionem et in opprobrium. \EVERSE}
\newcommand{\isXXXvVI}{\VERSE  Onus jumentorum austri. In terra tribulationis et angustiæ leæna, et leo ex eis, vipera et regulus volans ; portantes super humeros jumentorum divitias suas, et super gibbum camelorum thesauros suos, ad populum qui eis prodesse non poterit. \EVERSE}
\newcommand{\isXXXvVII}{\VERSE  Ægyptus enim frustra et vane auxiliabitur. Ideo clamavi super hoc : Superbia tantum est, quiesce. \EVERSE}
\newcommand{\isXXXvVIII}{\VERSE  Nunc ergo ingressus, scribe ei super buxum, et in libro diligenter exara illud, et erit in die novissimo in testimonium usque in æternum. \EVERSE}
\newcommand{\isXXXvIX}{\VERSE  Populus enim ad iracundiam provocans est : et filii mendaces, filii nolentes audire legem Dei ; \EVERSE}
\newcommand{\isXXXvX}{\VERSE  qui dicunt videntibus : Nolite videre, et aspicientibus : Nolite aspicere nobis ea quæ recta sunt ; loquimini nobis placentia : videte nobis errores. \EVERSE}
\newcommand{\isXXXvXI}{\VERSE  Auferte a me viam ; declinate a me semitam ; cesset a facie nostra Sanctus Israël. \EVERSE}
\newcommand{\isXXXvXII}{\VERSE  Propterea hæc dicit Sanctus Israël : Pro eo quod reprobastis verbum hoc, et sperastis in calumnia et in tumultu, et innixi estis super eo ; \EVERSE}
\newcommand{\isXXXvXIII}{\VERSE  propterea erit vobis iniquitas hæc sicut interruptio cadens, et requisita in muro excelso, quoniam subito, dum non speratur, veniet contritio ejus. \EVERSE}
\newcommand{\isXXXvXIV}{\VERSE  Et comminuetur sicut conteritur lagena figuli contritione pervalida, et non invenietur de fragmentis ejus testa in qua portetur igniculus de incendio, aut hauriatur parum aquæ de fovea. \EVERSE}
\newcommand{\isXXXvXV}{\VERSE  Quia hæc dicit Dominus Deus, Sanctus Israël : Si revertamini et quiescatis, salvi eritis ; in silentio et in spe erit fortitudo vestra. Et noluistis, \EVERSE}
\newcommand{\isXXXvXVI}{\VERSE  et dixistis : Nequaquam, sed ad equos fugiemus : ideo fugietis ; et : Super veloces ascendemus : ideo velociores erunt qui persequentur vos. \EVERSE}
\newcommand{\isXXXvXVII}{\VERSE  Mille homines a facie terroris unius ; et a facie terroris quinque fugietis, donec relinquamini quasi malus navis in vertice montis, et quasi signum super collem. \EVERSE}
\newcommand{\isXXXvXVIII}{\VERSE  Propterea exspectat Dominus ut misereatur vestri ; et ideo exaltabitur parcens vobis, quia Deus judicii Dominus : beati omnes qui exspectant eum ! \EVERSE}
\newcommand{\isXXXvXIX}{\VERSE  Populus enim Sion habitabit in Jerusalem : plorans nequaquam plorabis : miserans miserebitur tui, ad vocem clamoris tui : statim ut audierit, respondebit tibi. \EVERSE}
\newcommand{\isXXXvXX}{\VERSE  Et dabit vobis Dominus panem arctum, et aquam brevem ; et non faciet avolare a te ultra doctorem tuum ; et erunt oculi tui videntes præceptorem tuum. \EVERSE}
\newcommand{\isXXXvXXI}{\VERSE  Et aures tuæ audient verbum post tergum monentis : Hæc est via ; ambulate in ea, et non declinetis neque ad dexteram, neque ad sinistram. \EVERSE}
\newcommand{\isXXXvXXII}{\VERSE  Et contaminabis laminas sculptilium argenti tui, et vestimentum conflatilis auri tui, et disperges ea sicut immunditiam menstruatæ. Egredere, dices ei. \EVERSE}
\newcommand{\isXXXvXXIII}{\VERSE  Et dabitur pluvia semini tuo, ubicumque seminaveris in terra, et panis frugum terræ erit uberrimus et pinguis ; pascetur in possessione tua in die illo agnus spatiose, \EVERSE}
\newcommand{\isXXXvXXIV}{\VERSE  et tauri tui, et pulli asinorum, qui operantur terram, commistum migma comedent sicut in area ventilatum est. \EVERSE}
\newcommand{\isXXXvXXV}{\VERSE  Et erunt super omnem montem excelsum, et super omnem collem elevatum, rivi currentium aquarum, in die interfectionis multorum, cum ceciderint turres : \EVERSE}
\newcommand{\isXXXvXXVI}{\VERSE  et erit lux lunæ sicut lux solis, et lux solis erit septempliciter sicut lux septem dierum, in die qua alligaverit Dominus vulnus populi sui, et percussuram plagæ ejus sanaverit. \EVERSE}
\newcommand{\isXXXvXXVII}{\VERSE  Ecce nomen Domini venit de longinquo, ardens furor ejus, et gravis ad portandum ; labia ejus repleta sunt indignatione, et lingua ejus quasi ignis devorans. \EVERSE}
\newcommand{\isXXXvXXVIII}{\VERSE  Spiritus ejus velut torrens inundans usque ad medium colli, ad perdendas gentes in nihilum, et frenum erroris quod erat in maxillis populorum. \EVERSE}
\newcommand{\isXXXvXXIX}{\VERSE  Canticum erit vobis sicut nox sanctificatæ solemnitatis, et lætitia cordis sicut qui pergit cum tibia, ut intret in montem Domini ad Fortem Israël. \EVERSE}
\newcommand{\isXXXvXXX}{\VERSE  Et auditam faciet Dominus gloriam vocis suæ, et terrorem brachii sui ostendet in comminatione furoris, et flamma ignis devorantis : allidet in turbine, et in lapide grandinis. \EVERSE}
\newcommand{\isXXXvXXXI}{\VERSE  A voce enim Domini pavebit Assur virga percussus. \EVERSE}
\newcommand{\isXXXvXXXII}{\VERSE  Et erit transitus virgæ fundatus, quam requiescere faciet Dominus super eum in tympanis et citharis ; et in bellis præcipuis expugnabit eos. \EVERSE}
\newcommand{\isXXXvXXXIII}{\VERSE  Præparata est enim ab heri Topheth, a rege præparata, profunda, et dilatata. Nutrimenta ejus, ignis et ligna multa ; flatus Domini sicut torrens sulphuris succendens eam. \EVERSE}
\newcommand{\isXXXIvI}{\VERSE  Væ qui descendunt in Ægyptum ad auxilium, in equis sperantes, et habentes fiduciam super quadrigis, quia multæ sunt ; et super equitibus, quia prævalidi nimis ; et non sunt confisi super Sanctum Israël, et Dominum non requisierunt ! \EVERSE}
\newcommand{\isXXXIvII}{\VERSE  Ipse autem sapiens adduxit malum, et verba sua non abstulit ; et consurget contra domum pessimorum, et contra auxilium operantium iniquitatem. \EVERSE}
\newcommand{\isXXXIvIII}{\VERSE  Ægyptus homo, et non deus ; et equi eorum caro, et non spiritus ; et Dominus inclinabit manum suam, et corruet auxiliator, et cadet cui præstatur auxilium, simulque omnes consumentur. \EVERSE}
\newcommand{\isXXXIvIV}{\VERSE  Quia hæc dicit Dominus ad me : Quomodo si rugiat leo et catulus leonis super prædam suam ; et cum occurrerit ei multitudo pastorum, a voce eorum non formidabit, et a multitudine eorum non pavebit : sic descendet Dominus exercituum ut prælietur super montem Sion et super collem ejus. \EVERSE}
\newcommand{\isXXXIvV}{\VERSE  Sicut aves volantes, sic proteget Dominus exercituum Jerusalem, protegens et liberans, transiens et salvans. \EVERSE}
\newcommand{\isXXXIvVI}{\VERSE  Convertimini, sicut in profundum recesseratis, filii Israël. \EVERSE}
\newcommand{\isXXXIvVII}{\VERSE  In die enim illa abjiciet vir idola argenti sui, et idola auri sui, quæ fecerunt vobis manus vestræ in peccatum. \EVERSE}
\newcommand{\isXXXIvVIII}{\VERSE  Et cadet Assur in gladio non viri, et gladius non hominis vorabit eum : et fugiet non a facie gladii, et juvenes ejus vectigales erunt. \EVERSE}
\newcommand{\isXXXIvIX}{\VERSE  Et fortitudo ejus a terrore transibit, et pavebunt fugientes principes ejus, dixit Dominus : cujus ignis est in Sion et caminus ejus in Jerusalem. \EVERSE}
\newcommand{\isXXXIIvI}{\VERSE  Ecce in justitia regnabit rex, et principes in judicio præerunt. \EVERSE}
\newcommand{\isXXXIIvII}{\VERSE  Et erit vir sicut qui absconditur a vento, et celat se a tempestate ; sicut rivi aquarum in siti, et umbra petræ prominentis in terra deserta. \EVERSE}
\newcommand{\isXXXIIvIII}{\VERSE  Non caligabunt oculi videntium, et aures audientium diligenter auscultabunt. \EVERSE}
\newcommand{\isXXXIIvIV}{\VERSE  Et cor stultorum intelliget scientiam, et lingua balborum velociter loquetur et plane. \EVERSE}
\newcommand{\isXXXIIvV}{\VERSE  Non vocabitur ultra is qui insipiens est, princeps, neque fraudulentus appellabitur major ; \EVERSE}
\newcommand{\isXXXIIvVI}{\VERSE  stultus enim fatua loquetur, et cor ejus faciet iniquitatem, ut perficiat simulationem, et loquatur ad Dominum fraudulenter, et vacuam faciat animam esurientis, et potum sitienti auferat. \EVERSE}
\newcommand{\isXXXIIvVII}{\VERSE  Fraudulenti vasa pessima sunt ; ipse enim cogitationes concinnavit ad perdendos mites in sermone mendacii, cum loqueretur pauper judicium. \EVERSE}
\newcommand{\isXXXIIvVIII}{\VERSE  Princeps vero ea quæ digna sunt principe cogitabit, et ipse super duces stabit. \EVERSE}
\newcommand{\isXXXIIvIX}{\VERSE  Mulieres opulentæ, surgite, et audite vocem meam ; filiæ confidentes, percipite auribus eloquium meum. \EVERSE}
\newcommand{\isXXXIIvX}{\VERSE  Post dies enim et annum, vos conturbabimini confidentes ; consummata est enim vindemia, collectio ultra non veniet. \EVERSE}
\newcommand{\isXXXIIvXI}{\VERSE  Obstupescite, opulentæ ; conturbamini, confidentes : exuite vos et confundimini ; accingite lumbos vestros. \EVERSE}
\newcommand{\isXXXIIvXII}{\VERSE  Super ubera plangite, super regione desiderabili, super vinea fertili. \EVERSE}
\newcommand{\isXXXIIvXIII}{\VERSE  Super humum populi mei spinæ et vepres ascendent : quanto magis super omnes domos gaudii civitatis exultantis ! \EVERSE}
\newcommand{\isXXXIIvXIV}{\VERSE  Domus enim dimissa est, multitudo urbis relicta est, tenebræ et palpatio factæ sunt super speluncas usque in æternum ; gaudium onagrorum, pascua gregum. \EVERSE}
\newcommand{\isXXXIIvXV}{\VERSE  Donec effundatur super nos spiritus de excelso, et erit desertum in carmel, et carmel in saltum reputabitur. \EVERSE}
\newcommand{\isXXXIIvXVI}{\VERSE  Et habitabit in solitudine judicium, et justitia in carmel sedebit. \EVERSE}
\newcommand{\isXXXIIvXVII}{\VERSE  Et erit opus justitiæ pax, et cultus justitiæ silentium, et securitas usque in sempiternum. \EVERSE}
\newcommand{\isXXXIIvXVIII}{\VERSE  Et sedebit populus meus in pulchritudine pacis, et in tabernaculis fiduciæ, et in requie opulenta. \EVERSE}
\newcommand{\isXXXIIvXIX}{\VERSE  Grando autem in descensione saltus, et humilitate humiliabitur civitas. \EVERSE}
\newcommand{\isXXXIIvXX}{\VERSE  Beati qui seminatis super omnes aquas, immittentes pedem bovis et asini. \EVERSE}
\newcommand{\isXXXIIIvI}{\VERSE  Væ qui prædaris ! nonne et ipse prædaberis ? et qui spernis, nonne et ipse sperneris ? Cum consummaveris deprædationem, deprædaberis ; cum fatigatus desieris contemnere, contemneris. \EVERSE}
\newcommand{\isXXXIIIvII}{\VERSE  Domine, miserere nostri, te enim exspectavimus ; esto brachium nostrum in mane, et salus nostra in tempore tribulationis. \EVERSE}
\newcommand{\isXXXIIIvIII}{\VERSE  A voce angeli fugerunt populi, et ab exaltatione tua dispersæ sunt gentes. \EVERSE}
\newcommand{\isXXXIIIvIV}{\VERSE  Et congregabuntur spolia vestra sicut colligitur bruchus, velut cum fossæ plenæ fuerint de eo. \EVERSE}
\newcommand{\isXXXIIIvV}{\VERSE  Magnificatus est Dominus, quoniam habitavit in excelso ; implevit Sion judicio et justitia. \EVERSE}
\newcommand{\isXXXIIIvVI}{\VERSE  Et erit fides in temporibus tuis : divitiæ salutis sapientia et scientia ; timor Domini ipse est thesaurus ejus. \EVERSE}
\newcommand{\isXXXIIIvVII}{\VERSE  Ecce videntes clamabunt foris ; angeli pacis amare flebunt. \EVERSE}
\newcommand{\isXXXIIIvVIII}{\VERSE  Dissipatæ sunt viæ, cessavit transiens per semitam : irritum factum est pactum, projecit civitates, non reputavit homines. \EVERSE}
\newcommand{\isXXXIIIvIX}{\VERSE  Luxit et elanguit terra ; confusus est Libanus, et obsorduit : et factus est Saron sicut desertum, et concussa est Basan, et Carmelus. \EVERSE}
\newcommand{\isXXXIIIvX}{\VERSE  Nunc consurgam, dicit Dominus ; nunc exaltabor, nunc sublevabor. \EVERSE}
\newcommand{\isXXXIIIvXI}{\VERSE  Concipietis ardorem, parietis stipulam ; spiritus vester ut ignis vorabit vos. \EVERSE}
\newcommand{\isXXXIIIvXII}{\VERSE  Et erunt populi quasi de incendio cinis ; spinæ congregatæ igni comburentur. \EVERSE}
\newcommand{\isXXXIIIvXIII}{\VERSE  Audite, qui longe estis, quæ fecerim ; et cognoscite, vicini, fortitudinem meam. \EVERSE}
\newcommand{\isXXXIIIvXIV}{\VERSE  Conterriti sunt in Sion peccatores ; possedit tremor hypocritas. Quis poterit habitare de vobis cum igne devorante ? quis habitabit ex vobis cum ardoribus sempiternis ? \EVERSE}
\newcommand{\isXXXIIIvXV}{\VERSE  Qui ambulat in justitiis et loquitur veritatem, qui projicit avaritiam ex calumnia, et excutit manus suas ab omni munere, qui obturat aures suas ne audiat sanguinem, et claudit oculos suos ne videat malum. \EVERSE}
\newcommand{\isXXXIIIvXVI}{\VERSE  Iste in excelsis habitabit ; munimenta saxorum sublimitas ejus : panis ei datus est, aquæ ejus fideles sunt. \EVERSE}
\newcommand{\isXXXIIIvXVII}{\VERSE  Regem in decore suo videbunt oculi ejus, cernent terram de longe. \EVERSE}
\newcommand{\isXXXIIIvXVIII}{\VERSE  Cor tuum meditabitur timorem : ubi est litteratus ? ubi legis verba ponderans ? ubi doctor parvulorum ? \EVERSE}
\newcommand{\isXXXIIIvXIX}{\VERSE  Populum impudentem non videbis, populum alti sermonis, ita ut non possis intelligere disertitudinem linguæ ejus, in quo nulla est sapientia. \EVERSE}
\newcommand{\isXXXIIIvXX}{\VERSE  Respice, Sion, civitatem solemnitatis nostræ : oculi tui videbunt Jerusalem, habitationem opulentam, tabernaculum quod nequaquam transferri poterit ; nec auferentur clavi ejus in sempiternum, et omnes funiculi ejus non rumpentur : \EVERSE}
\newcommand{\isXXXIIIvXXI}{\VERSE  quia solummodo ibi magnificus est Dominus noster : locus fluviorum rivi latissimi et patentes : non transibit per eum navis remigum, neque trieris magna transgredietur eum. \EVERSE}
\newcommand{\isXXXIIIvXXII}{\VERSE  Dominus enim judex noster, Dominus legifer noster, Dominus rex noster, ipse salvabit nos. \EVERSE}
\newcommand{\isXXXIIIvXXIII}{\VERSE  Laxati sunt funiculi tui, et non prævalebunt ; sic erit malus tuus ut dilatare signum non queas. Tunc dividentur spolia prædarum multarum ; claudi diripient rapinam. \EVERSE}
\newcommand{\isXXXIIIvXXIV}{\VERSE  Nec dicet vicinus : Elangui ; populus qui habitat in ea, auferetur ab eo iniquitas. \EVERSE}
\newcommand{\isXXXIVvI}{\VERSE  Accedite, gentes, et audite ; et populi, attendite : audiat terra, et plenitudo ejus ; orbis, et omne germen ejus. \EVERSE}
\newcommand{\isXXXIVvII}{\VERSE  Quia indignatio Domini super omnes gentes, et furor super universam militiam eorum : interfecit eos, et dedit eos in occisionem. \EVERSE}
\newcommand{\isXXXIVvIII}{\VERSE  Interfecti eorum projicientur, et de cadaveribus eorum ascendet fœtor ; tabescent montes a sanguine eorum. \EVERSE}
\newcommand{\isXXXIVvIV}{\VERSE  Et tabescet omnis militia cælorum, et complicabuntur sicut liber cæli : et omnis militia eorum defluet, sicut defluit folium de vinea et de ficu. \EVERSE}
\newcommand{\isXXXIVvV}{\VERSE  Quoniam inebriatus est in cælo gladius meus ; ecce super Idumæam descendet, et super populum interfectionis meæ, ad judicium. \EVERSE}
\newcommand{\isXXXIVvVI}{\VERSE  Gladius Domini repletus est sanguine, incrassatus est adipe, de sanguine agnorum et hircorum, de sanguine medullatorum arietum : victima enim Domini in Bosra, et interfectio magna in terra Edom. \EVERSE}
\newcommand{\isXXXIVvVII}{\VERSE  Et descendent unicornes cum eis, et tauri cum potentibus ; inebriabitur terra eorum sanguine, et humus eorum adipe pinguium. \EVERSE}
\newcommand{\isXXXIVvVIII}{\VERSE  Quia dies ultionis Domini, annus retributionum judicii Sion. \EVERSE}
\newcommand{\isXXXIVvIX}{\VERSE  Et convertentur torrentes ejus in picem, et humus ejus in sulphur ; et erit terra ejus in picem ardentem. \EVERSE}
\newcommand{\isXXXIVvX}{\VERSE  Nocte et die non extinguetur, in sempiternum ascendet fumus ejus, a generatione in generationem desolabitur, in sæcula sæculorum non erit transiens per eam. \EVERSE}
\newcommand{\isXXXIVvXI}{\VERSE  Et possidebunt illam onocrotalus et ericius ; ibis et corvus habitabunt in ea : et extendetur super eam mensura, ut redigatur ad nihilum, et perpendiculum in desolationem. \EVERSE}
\newcommand{\isXXXIVvXII}{\VERSE  Nobiles ejus non erunt ibi ; regem potius invocabunt, et omnes principes ejus erunt in nihilum. \EVERSE}
\newcommand{\isXXXIVvXIII}{\VERSE  Et orientur in domibus ejus spinæ et urticæ, et paliurus in munitionibus ejus ; et erit cubile draconum, et pascua struthionum. \EVERSE}
\newcommand{\isXXXIVvXIV}{\VERSE  Et occurrent dæmonia onocentauris, et pilosus clamabit alter ad alterum ; ibi cubavit lamia, et invenit sibi requiem. \EVERSE}
\newcommand{\isXXXIVvXV}{\VERSE  Ibi habuit foveam ericius, et enutrivit catulos, et circumfodit, et fovit in umbra ejus ; illuc congregati sunt milvi, alter ad alterum. \EVERSE}
\newcommand{\isXXXIVvXVI}{\VERSE  Requirite diligenter in libro Domini, et legite : Unum ex eis non defuit, alter alterum non quæsivit ; quia quod ex ore meo procedit, ille mandavit, et spiritus ejus ipse congregavit ea. \EVERSE}
\newcommand{\isXXXIVvXVII}{\VERSE  Et ipse misit eis sortem, et manus ejus divisit eam illis in mensuram : usque in æternum possidebunt eam ; in generationem et generationem habitabunt in ea. \EVERSE}
\newcommand{\isXXXVvI}{\VERSE  Lætabitur deserta et invia, et exsultabit solitudo, et florebit quasi lilium. \EVERSE}
\newcommand{\isXXXVvII}{\VERSE  Germinans germinabit, et exsultabit lætabunda et laudans : gloria Libani data est ei, decor Carmeli et Saron ; ipsi videbunt gloriam Domini, et decorem Dei nostri. \EVERSE}
\newcommand{\isXXXVvIII}{\VERSE  Confortate manus dissolutas, et genua debilia roborate. \EVERSE}
\newcommand{\isXXXVvIV}{\VERSE  Dicite pusillanimis : Confortamini, et nolite timere : ecce Deus vester ultionem adducet retributionis ; Deus ipse veniet, et salvabit vos. \EVERSE}
\newcommand{\isXXXVvV}{\VERSE  Tunc aperientur oculi cæcorum, et aures surdorum patebunt ; \EVERSE}
\newcommand{\isXXXVvVI}{\VERSE  tunc saliet sicut cervus claudus, et aperta erit lingua mutorum : quia scissæ sunt in deserto aquæ, et torrentes in solitudine ; \EVERSE}
\newcommand{\isXXXVvVII}{\VERSE  et quæ erat arida, erit in stagnum, et sitiens in fontes aquarum. In cubilibus, in quibus prius dracones habitabant, orietur viror calami et junci. \EVERSE}
\newcommand{\isXXXVvVIII}{\VERSE  Et erit ibi semita et via, et via sancta vocabitur : non transibit per eam pollutus, et hæc erit vobis directa via, ita ut stulti non errent per eam. \EVERSE}
\newcommand{\isXXXVvIX}{\VERSE  Non erit ibi leo, et mala bestia non ascendet per eam, nec invenietur ibi ; et ambulabunt qui liberati fuerint. \EVERSE}
\newcommand{\isXXXVvX}{\VERSE  Et redempti a Domino convertentur, et venient in Sion cum laude, et lætitia sempiterna super caput eorum : gaudium et lætitiam obtinebunt, et fugiet dolor et gemitus. \EVERSE}
\newcommand{\isXXXVIvI}{\VERSE  Et factum est in quartodecimo anno regis Ezechiæ, ascendit Sennacherib, rex Assyriorum, super omnes civitates Juda munitas, et cepit eas. \EVERSE}
\newcommand{\isXXXVIvII}{\VERSE  Et misit rex Assyriorum Rabsacen de Lachis in Jerusalem, ad regem Ezechiam in manu gravi : et stetit in aquæductu piscinæ superioris in via Agri fullonis. \EVERSE}
\newcommand{\isXXXVIvIII}{\VERSE  Et egressus est ad eum Eliacim, filius Helciæ, qui erat super domum, et Sobna scriba, et Joahe filius Asaph, a commentariis. \EVERSE}
\newcommand{\isXXXVIvIV}{\VERSE  Et dixit ad eos Rabsaces : Dicite Ezechiæ : Hæc dicit rex magnus, rex Assyriorum : Quæ est ista fiducia qua confidis ? \EVERSE}
\newcommand{\isXXXVIvV}{\VERSE  aut quo consilio vel fortitudine rebellare disponis ? super quem habes fiduciam, quia recessisti a me ? \EVERSE}
\newcommand{\isXXXVIvVI}{\VERSE  Ecce confidis super baculum arundineum confractum istum, super Ægyptum ; cui si innixus fuerit homo, intrabit in manum ejus, et perforabit eam : sic Pharao, rex Ægypti, omnibus qui confidunt in eo. \EVERSE}
\newcommand{\isXXXVIvVII}{\VERSE  Quod si responderis mihi : In Domino Deo nostro confidimus ; nonne ipse est cujus abstulit Ezechias excelsa et altaria, et dixit Judæ et Jerusalem : Coram altari isto adorabitis ? \EVERSE}
\newcommand{\isXXXVIvVIII}{\VERSE  Et nunc trade te domino meo, regi Assyriorum, et dabo tibi duo millia equorum, nec poteris ex te præbere ascensores eorum : \EVERSE}
\newcommand{\isXXXVIvIX}{\VERSE  et quomodo sustinebis faciem judicis unius loci ex servis domini mei minoribus ? Quod si confidis in Ægypto, in quadrigis et in equitibus, \EVERSE}
\newcommand{\isXXXVIvX}{\VERSE  et nunc numquid sine Domino ascendi ad terram istam, ut disperderem eam ? Dominus dixit ad me : Ascende super terram istam, et disperde eam. \EVERSE}
\newcommand{\isXXXVIvXI}{\VERSE  Et dixit Eliacim, et Sobna, et Joahe, ad Rabsacen : Loquere ad servos tuos syra lingua ; intelligimus enim ; ne loquaris ad nos judaice in auribus populi qui est super murum. \EVERSE}
\newcommand{\isXXXVIvXII}{\VERSE  Et dixit ad eos Rabsaces : Numquid ad dominum tuum et ad te misit me dominus meus, ut loquerer omnia verba ista ? et non potius ad viros qui sedent in muro, ut comedant stercora sua, et bibant urinam pedum suorum vobiscum ? \EVERSE}
\newcommand{\isXXXVIvXIII}{\VERSE  Et stetit Rabsaces, et clamavit voce magna judaice, et dixit : Audite verba regis magni, regis Assyriorum ! \EVERSE}
\newcommand{\isXXXVIvXIV}{\VERSE  Hæc dicit rex : Non seducat vos Ezechias, quia non poterit eruere vos. \EVERSE}
\newcommand{\isXXXVIvXV}{\VERSE  Et non vobis tribuat fiduciam Ezechias super Domino, dicens : Eruens liberabit nos Dominus : non dabitur civitas ista in manu regis Assyriorum. \EVERSE}
\newcommand{\isXXXVIvXVI}{\VERSE  Nolite audire Ezechiam ; hæc enim dicit rex Assyriorum : Facite mecum benedictionem, et egredimini ad me, et comedite unusquisque vineam suam, et unusquisque ficum suam, et bibite unusquisque aquam cisternæ suæ, \EVERSE}
\newcommand{\isXXXVIvXVII}{\VERSE  donec veniam, et tollam vos ad terram quæ est ut terra vestra, terram frumenti et vini, terram panum et vinearum. \EVERSE}
\newcommand{\isXXXVIvXVIII}{\VERSE  Nec conturbet vos Ezechias, dicens : Dominus liberabit nos. Numquid liberaverunt dii gentium unusquisque terram suam de manu regis Assyriorum ? \EVERSE}
\newcommand{\isXXXVIvXIX}{\VERSE  Ubi est deus Emath et Arphad ? ubi est deus Sepharvaim ? numquid liberaverunt Samariam de manu mea ? \EVERSE}
\newcommand{\isXXXVIvXX}{\VERSE  Quis est ex omnibus diis terrarum istarum qui eruerit terram suam de manu mea, ut eruat Dominus Jerusalem de manu mea ? \EVERSE}
\newcommand{\isXXXVIvXXI}{\VERSE  Et siluerunt, et non responderunt ei verbum. Mandaverat enim rex, dicens : Ne respondeatis ei. \EVERSE}
\newcommand{\isXXXVIvXXII}{\VERSE  Et ingressus est Eliacim, filius Helciæ, qui erat super domum, et Sobna scriba, et Joahe filius Asaph, a commentariis, ad Ezechiam, scissis vestibus, et nuntiaverunt ei verba Rabsacis. \EVERSE}
\newcommand{\isXXXVIIvI}{\VERSE  Et factum est, cum audisset rex Ezechias, scidit vestimenta sua, et obvolutus est sacco, et intravit in domum Domini. \EVERSE}
\newcommand{\isXXXVIIvII}{\VERSE  Et misit Eliacim, qui erat super domum, et Sobnam scribam, et seniores de sacerdotibus, opertos saccis, ad Isaiam, filium Amos, prophetam, \EVERSE}
\newcommand{\isXXXVIIvIII}{\VERSE  et dixerunt ad eum : Hæc dicit Ezechias : Dies tribulationis, et correptionis, et blasphemiæ, dies hæc ; quia venerunt filii usque ad partum, et virtus non est pariendi. \EVERSE}
\newcommand{\isXXXVIIvIV}{\VERSE  Si quomodo audiat Dominus Deus tuus verba Rabsacis, quem misit rex Assyriorum dominus suus ad blasphemandum Deum viventem et exprobrandum sermonibus quos audivit Dominus Deus tuus : leva ergo orationem pro reliquiis quæ repertæ sunt. \EVERSE}
\newcommand{\isXXXVIIvV}{\VERSE  Et venerunt servi regis Ezechiæ ad Isaiam. \EVERSE}
\newcommand{\isXXXVIIvVI}{\VERSE  Et dixit ad eos Isaias : Hæc dicetis domino vestro : Hæc dicit Dominus : Ne timeas a facie verborum quæ audisti, quibus blasphemaverunt pueri regis Assyriorum me. \EVERSE}
\newcommand{\isXXXVIIvVII}{\VERSE  Ecce ego dabo ei spiritum, et audiet nuntium, et revertetur ad terram suam, et corruere eum faciam gladio in terra sua. \EVERSE}
\newcommand{\isXXXVIIvVIII}{\VERSE  Reversus est autem Rabsaces, et invenit regem Assyriorum præliantem adversus Lobnam : audierat enim quia profectus esset de Lachis. \EVERSE}
\newcommand{\isXXXVIIvIX}{\VERSE  Et audivit de Tharaca rege Æthiopiæ, dicentes : Egressus est ut pugnet contra te. Quod cum audisset, misit nuntios ad Ezechiam, dicens : \EVERSE}
\newcommand{\isXXXVIIvX}{\VERSE  Hæc dicetis Ezechiæ regi Judæ, loquentes : Non te decipiat Deus tuus in quo tu confidis, dicens : Non dabitur Jerusalem in manu regis Assyriorum. \EVERSE}
\newcommand{\isXXXVIIvXI}{\VERSE  Ecce tu audisti omnia quæ fecerunt reges Assyriorum omnibus terris, quas subverterunt : et tu poteris liberari ? \EVERSE}
\newcommand{\isXXXVIIvXII}{\VERSE  Numquid eruerunt eos dii gentium quos subverterunt patres mei, Gozam, et Haram, et Reseph, et filios Eden qui erant in Thalassar ? \EVERSE}
\newcommand{\isXXXVIIvXIII}{\VERSE  Ubi est rex Emath, et rex Arphad, et rex urbis Sepharvaim, Ana, et Ava ? \EVERSE}
\newcommand{\isXXXVIIvXIV}{\VERSE  Et tulit Ezechias libros de manu nuntiorum, et legit eos, et ascendit in domum Domini, et expandit eos Ezechias coram Domino : \EVERSE}
\newcommand{\isXXXVIIvXV}{\VERSE  et oravit Ezechias ad Dominum, dicens : \EVERSE}
\newcommand{\isXXXVIIvXVI}{\VERSE  Domine exercituum, Deus Israël, qui sedes super cherubim, tu es Deus solus omnium regnorum terræ : tu fecisti cælum et terram. \EVERSE}
\newcommand{\isXXXVIIvXVII}{\VERSE  Inclina, Domine, aurem tuam, et audi ; aperi, Domine, oculos tuos, et vide : et audi omnia verba Sennacherib, quæ misit ad blasphemandum Deum viventem. \EVERSE}
\newcommand{\isXXXVIIvXVIII}{\VERSE  Vere enim, Domine, desertas fecerunt reges Assyriorum terras, et regiones earum, \EVERSE}
\newcommand{\isXXXVIIvXIX}{\VERSE  et dederunt deos earum igni : non enim erant dii, sed opera manuum hominum, lignum et lapis, et comminuerunt eos. \EVERSE}
\newcommand{\isXXXVIIvXX}{\VERSE  Et nunc, Domine Deus noster, salva nos de manu ejus, et cognoscant omnia regna terræ quia tu es Dominus solus. \EVERSE}
\newcommand{\isXXXVIIvXXI}{\VERSE  Et misit Isaias, filius Amos, ad Ezechiam, dicens : Hæc dicit Dominus Deus Israël : Pro quibus rogasti me de Sennacherib, rege Assyriorum, \EVERSE}
\newcommand{\isXXXVIIvXXII}{\VERSE  hoc est verbum quod locutus est Dominus super eum : Despexit te et subsannavit te, virgo filia Sion ; post te caput movit, filia Jerusalem. \EVERSE}
\newcommand{\isXXXVIIvXXIII}{\VERSE  Cui exprobrasti ? et quem blasphemasti ? et super quem exaltasti vocem, et levasti altitudinem oculorum tuorum ? ad Sanctum Israël. \EVERSE}
\newcommand{\isXXXVIIvXXIV}{\VERSE  In manu servorum tuorum exprobrasti Domino, et dixisti : In multitudine quadrigarum mearum ego ascendi altitudinem montium juga Libani ; et succidam excelsa cedrorum ejus, et electas abietes illius, et introibo altitudinem summitatis ejus, saltum Carmeli ejus. \EVERSE}
\newcommand{\isXXXVIIvXXV}{\VERSE  Ego fodi, et bibi aquam, et exsiccavi vestigio pedis mei omnes rivos aggerum. \EVERSE}
\newcommand{\isXXXVIIvXXVI}{\VERSE  Numquid non audisti quæ olim fecerim ei ? Ex diebus antiquis ego plasmavi illud ; et nunc adduxi, et factum est in eradicationem collium compugnantium, et civitatum munitarum. \EVERSE}
\newcommand{\isXXXVIIvXXVII}{\VERSE  Habitatores earum breviata manu contremuerunt, et confusi sunt. Facti sunt sicut fœnum agri, et gramen pascuæ, et herba tectorum, quæ exaruit antequam maturesceret. \EVERSE}
\newcommand{\isXXXVIIvXXVIII}{\VERSE  Habitationem tuam, et egressum tuum, et introitum tuum cognovi, et insaniam tuam contra me. \EVERSE}
\newcommand{\isXXXVIIvXXIX}{\VERSE  Cum fureres adversum me, superbia tua ascendit in aures meas. Ponam ergo circulum in naribus tuis, et frenum in labiis tuis, et reducam te in viam per quem venisti. \EVERSE}
\newcommand{\isXXXVIIvXXX}{\VERSE  Tibi autem hoc erit signum : comede hoc anno quæ sponte nascuntur, et in anno secundo pomis vescere ; in anno autem tertio seminate et metite, et plantate vineas, et comedite fructum earum. \EVERSE}
\newcommand{\isXXXVIIvXXXI}{\VERSE  Et mittet id quod salvatum fuerit de domo Juda, et quod reliquum est, radicem deorsum, et faciet fructum sursum : \EVERSE}
\newcommand{\isXXXVIIvXXXII}{\VERSE  quia de Jerusalem exibunt reliquiæ, et salvatio de monte Sion : zelus Domini exercituum faciet istud. \EVERSE}
\newcommand{\isXXXVIIvXXXIII}{\VERSE  Propterea hæc dicit Dominus de rege Assyriorum : Non intrabit civitatem hanc, et non jaciet ibi sagittam, et non occupabit eam clypeus, et non mittet in circuitu ejus aggerem. \EVERSE}
\newcommand{\isXXXVIIvXXXIV}{\VERSE  In via qua venit, per eam revertetur, et civitatem hanc non ingredietur, dicit Dominus. \EVERSE}
\newcommand{\isXXXVIIvXXXV}{\VERSE  Et protegam civitatem istam, ut salvem eam propter me, et propter David, servum meum. \EVERSE}
\newcommand{\isXXXVIIvXXXVI}{\VERSE  Egressus est autem angelus Domini, et percussit in castris Assyriorum centum octoginta quinque millia. Et surrexerunt mane, et ecce omnes cadavera mortuorum. \EVERSE}
\newcommand{\isXXXVIIvXXXVII}{\VERSE  Et egressus est, et abiit, et reversus est Sennacherib, rex Assyriorum, et habitavit in Ninive. \EVERSE}
\newcommand{\isXXXVIIvXXXVIII}{\VERSE  Et factum est, cum adoraret in templo Nesroch deum suum, Adramelech et Sarasar, filii ejus, percusserunt eum gladio, fugeruntque in terram Ararat ; et regnavit Asarhaddon, filius ejus, pro eo. \EVERSE}
\newcommand{\isXXXVIIIvI}{\VERSE  In diebus illis ægrotavit Ezechias usque ad mortem ; et introivit ad eum Isaias, filius Amos, propheta, et dixit ei : Hæc dicit Dominus : Dispone domui tuæ, quia morieris tu, et non vives. \EVERSE}
\newcommand{\isXXXVIIIvII}{\VERSE  Et convertit Ezechias faciem suam ad parietem, et oravit ad Dominum, \EVERSE}
\newcommand{\isXXXVIIIvIII}{\VERSE  et dixit : Obsecro, Domine, memento, quæso, quomodo ambulaverim coram te in veritate et in corde perfecto, et quod bonum est in oculis tuis fecerim. Et flevit Ezechias fletu magno. \EVERSE}
\newcommand{\isXXXVIIIvIV}{\VERSE  Et factum est verbum Domini ad Isaiam, dicens : \EVERSE}
\newcommand{\isXXXVIIIvV}{\VERSE  Vade, et dic Ezechiæ : Hæc dicit Dominus Deus David patris tui : Audivi orationem tuam, et vidi lacrimas tuas ; ecce ego adjiciam super dies tuos quindecim annos, \EVERSE}
\newcommand{\isXXXVIIIvVI}{\VERSE  et de manu regis Assyriorum eruam te, et civitatem istam, et protegam eam. \EVERSE}
\newcommand{\isXXXVIIIvVII}{\VERSE  Hoc autem tibi erit signum a Domino, quia faciet Dominus verbum hoc quod locutus est : \EVERSE}
\newcommand{\isXXXVIIIvVIII}{\VERSE  ecce ego reverti faciam umbram linearum per quas descenderat in horologio Achaz in sole, retrorsum decem lineis. Et reversus est sol decem lineis per gradus quos descenderat. \EVERSE}
\newcommand{\isXXXVIIIvIX}{\VERSE  Scriptura Ezechiæ, regis Juda, cum ægrotasset et convaluisset de infirmitate sua. \EVERSE}
\newcommand{\isXXXVIIIvX}{\VERSE  Ego dixi in dimidio dierum meorum : Vadam ad portas inferi ; quæsivi residuum annorum meorum. \EVERSE}
\newcommand{\isXXXVIIIvXI}{\VERSE  Dixi : Non videbo Dominum Deum in terra viventium ; non aspiciam hominem ultra, et habitatorem quietis. \EVERSE}
\newcommand{\isXXXVIIIvXII}{\VERSE  Generatio mea ablata est, et convoluta est a me, quasi tabernaculum pastorum. Præcisa est velut a texente vita mea ; dum adhuc ordirer, succidit me : de mane usque ad vesperam finies me. \EVERSE}
\newcommand{\isXXXVIIIvXIII}{\VERSE  Sperabam usque ad mane ; quasi leo, sic contrivit omnia ossa mea : de mane usque ad vesperam finies me. \EVERSE}
\newcommand{\isXXXVIIIvXIV}{\VERSE  Sicut pullus hirundinis, sic clamabo ; meditabor ut columba. Attenuati sunt oculi mei, suspicientes in excelsum. Domine, vim patior : responde pro me. \EVERSE}
\newcommand{\isXXXVIIIvXV}{\VERSE  Quid dicam, aut quid respondebit mihi, cum ipse fecerit ? Recogitabo tibi omnes annos meos in amaritudine animæ meæ. \EVERSE}
\newcommand{\isXXXVIIIvXVI}{\VERSE  Domine, si sic vivitur, et in talibus vita spiritus mei, corripies me, et vivificabis me. \EVERSE}
\newcommand{\isXXXVIIIvXVII}{\VERSE  Ecce in pace amaritudo mea amarissima. Tu autem eruisti animam meam ut non periret ; projecisti post tergum tuum omnia peccata mea. \EVERSE}
\newcommand{\isXXXVIIIvXVIII}{\VERSE  Quia non infernus confitebitur tibi, neque mors laudabit te : non exspectabunt qui descendunt in lacum veritatem tuam. \EVERSE}
\newcommand{\isXXXVIIIvXIX}{\VERSE  Vivens, vivens ipse confitebitur tibi, sicut et ego hodie ; pater filiis notam faciet veritatem tuam. \EVERSE}
\newcommand{\isXXXVIIIvXX}{\VERSE  Domine, salvum me fac ! et psalmos nostros cantabimus cunctis diebus vitæ nostræ in domo Domini. \EVERSE}
\newcommand{\isXXXVIIIvXXI}{\VERSE  Et jussit Isaias ut tollerent massam de ficis, et cataplasmarent super vulnus, et sanaretur. \EVERSE}
\newcommand{\isXXXVIIIvXXII}{\VERSE  Et dixit Ezechias : Quod erit signum quia ascendam in domum Domini ? \EVERSE}
\newcommand{\isXXXIXvI}{\VERSE  In tempore illo misit Merodach Baladan, filius Baladan, rex Babylonis, libros et munera ad Ezechiam : audierat enim quod ægrotasset et convaluisset. \EVERSE}
\newcommand{\isXXXIXvII}{\VERSE  Lætatus est autem super eis Ezechias, et ostendit eis cellam aromatum, et argenti, et auri, et odoramentorum, et unguenti optimi, et omnes apothecas supellectilis suæ, et universa quæ inventa sunt in thesauris ejus. Non fuit verbum quod non ostenderet eis Ezechias in domo sua, et in omni potestate sua. \EVERSE}
\newcommand{\isXXXIXvIII}{\VERSE  Introivit autem Isaias propheta ad Ezechiam regem, et dixit ei : Quid dixerunt viri isti, et unde venerunt ad te ? Et dixit Ezechias : De terra longinqua venerunt ad me, de Babylone. \EVERSE}
\newcommand{\isXXXIXvIV}{\VERSE  Et dixit : Quid viderunt in domo tua ? Et dixit Ezechias : Omnia quæ in domo mea sunt viderunt ; non fuit res quam non ostenderim eis in thesauris meis. \EVERSE}
\newcommand{\isXXXIXvV}{\VERSE  Et dixit Isaias ad Ezechiam : Audi verbum Domini exercituum. \EVERSE}
\newcommand{\isXXXIXvVI}{\VERSE  Ecce dies venient, et auferentur omnia quæ in domo tua sunt, et quæ thesaurizaverunt patres tui usque ad diem hanc, in Babylonem ; non relinquetur quidquam, dicit Dominus. \EVERSE}
\newcommand{\isXXXIXvVII}{\VERSE  Et de filiis tuis, qui exibunt de te, quos genueris, tollent, et erunt eunuchi in palatio regis Babylonis. \EVERSE}
\newcommand{\isXXXIXvVIII}{\VERSE  Et dixit Ezechias ad Isaiam : Bonum verbum Domini, quod locutus est. Et dixit : Fiat tantum pax et veritas in diebus meis ! \EVERSE}
\newcommand{\isXLvI}{\VERSE  Consolamini, consolamini, popule meus, dicit Deus vester. \EVERSE}
\newcommand{\isXLvII}{\VERSE  Loquimini ad cor Jerusalem, et advocate eam, quoniam completa est malitia ejus, dimissa est iniquitas illius : suscepit de manu Domini duplicia pro omnibus peccatis suis. \EVERSE}
\newcommand{\isXLvIII}{\VERSE  Vox clamantis in deserto : Parate viam Domini, rectas facite in solitudine semitas Dei nostri. \EVERSE}
\newcommand{\isXLvIV}{\VERSE  Omnis vallis exaltabitur, et omnis mons et collis humiliabitur, et erunt prava in directa, et aspera in vias planas : \EVERSE}
\newcommand{\isXLvV}{\VERSE  et revelabitur gloria Domini, et videbit omnis caro pariter quod os Domini locutum est. \EVERSE}
\newcommand{\isXLvVI}{\VERSE  Vox dicentis : Clama. Et dixi : Quid clamabo ? Omnis caro fœnum, et omnis gloria ejus quasi flos agri. \EVERSE}
\newcommand{\isXLvVII}{\VERSE  Exsiccatum est fœnum, et cecidit flos, quia spiritus Domini sufflavit in eo. Vere fœnum est populus : \EVERSE}
\newcommand{\isXLvVIII}{\VERSE  exsiccatum est fœnum, et cecidit flos ; verbum autem Domini nostri manet in æternum. \EVERSE}
\newcommand{\isXLvIX}{\VERSE  Super montem excelsum ascende, tu qui evangelizas Sion ; exalta in fortitudine vocem tuam, qui evangelizas Jerusalem : exalta, noli timere. Dic civitatibus Juda : Ecce Deus vester : \EVERSE}
\newcommand{\isXLvX}{\VERSE  ecce Dominus Deus in fortitudine veniet, et brachium ejus dominabitur : ecce merces ejus cum eo, et opus illius coram illo. \EVERSE}
\newcommand{\isXLvXI}{\VERSE  Sicut pastor gregem suum pascet, in brachio suo congregabit agnos, et in sinu suo levabit ; fœtas ipse portabit. \EVERSE}
\newcommand{\isXLvXII}{\VERSE  Quis mensus est pugillo aquas, et cælos palmo ponderavit ? quis appendit tribus digitis molem terræ, et liberavit in pondere montes, et colles in statera ? \EVERSE}
\newcommand{\isXLvXIII}{\VERSE  Quis adjuvit spiritum Domini ? aut quis consiliarius ejus fuit, et ostendit illi ? \EVERSE}
\newcommand{\isXLvXIV}{\VERSE  cum quo iniit consilium, et instruxit eum, et docuit eum semitam justitiæ, et erudivit eum scientiam, et viam prudentiæ ostendit illi ? \EVERSE}
\newcommand{\isXLvXV}{\VERSE  Ecce gentes quasi stilla situlæ, et quasi momentum stateræ reputatæ sunt ; ecce insulæ quasi pulvis exiguus. \EVERSE}
\newcommand{\isXLvXVI}{\VERSE  Et Libanus non sufficiet ad succendendum, et animalia ejus non sufficient ad holocaustum. \EVERSE}
\newcommand{\isXLvXVII}{\VERSE  Omnes gentes quasi non sint, sic sunt coram eo, et quasi nihilum et inane reputatæ sunt ei. \EVERSE}
\newcommand{\isXLvXVIII}{\VERSE  Cui ergo similem fecisti Deum ? aut quam imaginem ponetis ei ? \EVERSE}
\newcommand{\isXLvXIX}{\VERSE  Numquid sculptile conflavit faber ? aut aurifex auro figuravit illud, et laminis argenteis argentarius ? \EVERSE}
\newcommand{\isXLvXX}{\VERSE  Forte lignum et imputribile elegit ; artifex sapiens quærit quomodo statuat simulacrum, quod non moveatur. \EVERSE}
\newcommand{\isXLvXXI}{\VERSE  Numquid non scitis ? numquid non audistis ? numquid non annuntiatum est vobis ab initio ? numquid non intellexistis fundamenta terræ ? \EVERSE}
\newcommand{\isXLvXXII}{\VERSE  Qui sedet super gyrum terræ, et habitatores ejus sunt quasi locustæ ; qui extendit velut nihilum cælos, et expandit eos sicut tabernaculum ad inhabitandum ; \EVERSE}
\newcommand{\isXLvXXIII}{\VERSE  qui dat secretorum scrutatores quasi non sint, judices terræ velut inane fecit. \EVERSE}
\newcommand{\isXLvXXIV}{\VERSE  Et quidem neque plantatus, neque satus, neque radicatus in terra truncus eorum ; repente flavit in eos, et aruerunt, et turbo quasi stipulam auferet eos. \EVERSE}
\newcommand{\isXLvXXV}{\VERSE  Et cui assimilastis me, et adæquastis ? dicit Sanctus. \EVERSE}
\newcommand{\isXLvXXVI}{\VERSE  Levate in excelsum oculos vestros, et videte quis creavit hæc : qui educit in numero militiam eorum, et omnes ex nomine vocat ; præ multitudine fortitudinis et roboris, virtutisque ejus, neque unum reliquum fuit. \EVERSE}
\newcommand{\isXLvXXVII}{\VERSE  Quare dicis, Jacob, et loqueris, Israël : Abscondita est via mea a Domino, et a Deo meo judicium meum transivit ? \EVERSE}
\newcommand{\isXLvXXVIII}{\VERSE  Numquid nescis, aut non audisti ? Deus sempiternus Dominus, qui creavit terminos terræ : non deficiet, neque laborabit, nec est investigatio sapientiæ ejus. \EVERSE}
\newcommand{\isXLvXXIX}{\VERSE  Qui dat lasso virtutem, et his qui non sunt, fortitudinem et robur multiplicat. \EVERSE}
\newcommand{\isXLvXXX}{\VERSE  Deficient pueri, et laborabunt, et juvenes in infirmitate cadent ; \EVERSE}
\newcommand{\isXLvXXXI}{\VERSE  qui autem sperant in Domino mutabunt fortitudinem, assument pennas sicut aquilæ : current et non laborabunt, ambulabunt et non deficient. \EVERSE}
\newcommand{\isXLIvI}{\VERSE  Taceant ad me insulæ, et gentes mutent fortitudinem : accedant, et tunc loquantur ; simul ad judicium propinquemus. \EVERSE}
\newcommand{\isXLIvII}{\VERSE  Quis suscitavit ab oriente Justum, vocavit eum ut sequeretur se ? Dabit in conspectu ejus gentes, et reges obtinebit : dabit quasi pulverem gladio ejus, sicut stipulam vento raptam arcui ejus. \EVERSE}
\newcommand{\isXLIvIII}{\VERSE  Persequetur eos, transibit in pace : semita in pedibus ejus non apparebit. \EVERSE}
\newcommand{\isXLIvIV}{\VERSE  Quis hæc operatus est, et fecit, vocans generationes ab exordio ? Ego Dominus : primus et novissimus ego sum. \EVERSE}
\newcommand{\isXLIvV}{\VERSE  Viderunt insulæ, et timuerunt ; extrema terræ obstupuerunt : appropinquaverunt, et accesserunt. \EVERSE}
\newcommand{\isXLIvVI}{\VERSE  Unusquisque proximo suo auxiliabitur, et fratri suo dicet : Confortare. \EVERSE}
\newcommand{\isXLIvVII}{\VERSE  Confortavit faber ærarius percutiens malleo eum, qui cudebat tunc temporis, dicens : Glutino bonum est ; et confortavit eum clavis, ut non moveretur. \EVERSE}
\newcommand{\isXLIvVIII}{\VERSE  Et tu, Israël, serve meus, Jacob quem elegi, semen Abraham amici mei : \EVERSE}
\newcommand{\isXLIvIX}{\VERSE  in quo apprehendi te ab extremis terræ, et a longinquis ejus vocavi te, et dixi tibi : Servus meus es tu : elegi te, et non abjeci te. \EVERSE}
\newcommand{\isXLIvX}{\VERSE  Ne timeas, quia ego tecum sum ; ne declines, quia ego Deus tuus : confortavi te, et auxiliatus sum tibi, et suscepit te dextera Justi mei. \EVERSE}
\newcommand{\isXLIvXI}{\VERSE  Ecce confundentur et erubescent omnes qui pugnant adversum te ; erunt quasi non sint, et peribunt viri qui contradicunt tibi. \EVERSE}
\newcommand{\isXLIvXII}{\VERSE  Quæres eos, et non invenies, viros rebelles tuos ; erunt quasi non sint, et veluti consumptio homines bellantes adversum te. \EVERSE}
\newcommand{\isXLIvXIII}{\VERSE  Quia ego Dominus Deus tuus, apprehendens manum tuam, dicensque tibi : Ne timeas : ego adjuvi te. \EVERSE}
\newcommand{\isXLIvXIV}{\VERSE  Noli timere, vermis Jacob, qui mortui estis ex Israël : ego auxiliatus sum tibi, dicit Dominus, et redemptor tuus Sanctus Israël. \EVERSE}
\newcommand{\isXLIvXV}{\VERSE  Ego posui te quasi plaustrum triturans novum, habens rostra serrantia ; triturabis montes, et comminues, et colles quasi pulverem pones. \EVERSE}
\newcommand{\isXLIvXVI}{\VERSE  Ventilabis eos, et ventus tollet, et turbo disperget eos ; et tu exsultabis in Domino, in Sancto Israël lætaberis. \EVERSE}
\newcommand{\isXLIvXVII}{\VERSE  Egeni et pauperes quærunt aquas, et non sunt ; lingua eorum siti aruit. Ego Dominus exaudiam eos, Deus Israël, non derelinquam eos. \EVERSE}
\newcommand{\isXLIvXVIII}{\VERSE  Aperiam in supinis collibus flumina, et in medio camporum fontes : ponam desertum in stagna aquarum, et terram inviam in rivos aquarum. \EVERSE}
\newcommand{\isXLIvXIX}{\VERSE  Dabo in solitudinem cedrum, et spinam, et myrtum, et lignum olivæ ; ponam in deserto abietem, ulmum, et buxum simul : \EVERSE}
\newcommand{\isXLIvXX}{\VERSE  ut videant, et sciant, et recogitent, et intelligant pariter, quia manus Domini fecit hoc, et Sanctus Israël creavit illud. \EVERSE}
\newcommand{\isXLIvXXI}{\VERSE  Prope facite judicium vestrum, dicit Dominus ; Afferte, si quid forte habetis, dicit rex Jacob. \EVERSE}
\newcommand{\isXLIvXXII}{\VERSE  Accedant, et nuntient nobis quæcumque ventura sunt ; priora quæ fuerunt, nuntiate, et ponemus cor nostrum, et sciemus novissima eorum ; et quæ ventura sunt, indicate nobis. \EVERSE}
\newcommand{\isXLIvXXIII}{\VERSE  Annuntiate quæ ventura sunt in futurum, et sciemus quia dii estis vos ; bene quoque aut male, si potestis, facite, et loquamur et videamus simul. \EVERSE}
\newcommand{\isXLIvXXIV}{\VERSE  Ecce vos estis ex nihilo, et opus vestrum ex eo quod non est : abominatio est qui elegit vos. \EVERSE}
\newcommand{\isXLIvXXV}{\VERSE  Suscitavi ab aquilone, et veniet ab ortu solis : vocabit nomen meum, et adducet magistratus quasi lutum, et velut plastes conculcans humum. \EVERSE}
\newcommand{\isXLIvXXVI}{\VERSE  Quis annuntiavit ab exordio ut sciamus, et a principio ut dicamus : Justus es ? Non est neque annuntians, neque prædicens, neque audiens sermones vestros. \EVERSE}
\newcommand{\isXLIvXXVII}{\VERSE  Primus ad Sion dicet : Ecce adsunt, et Jerusalem evangelistam dabo. \EVERSE}
\newcommand{\isXLIvXXVIII}{\VERSE  Et vidi, et non erat neque ex istis quisquam qui iniret consilium, et interrogatus responderet verbum. \EVERSE}
\newcommand{\isXLIvXXIX}{\VERSE  Ecce omnes injusti, et vana opera eorum ; ventus et inane simulacra eorum. \EVERSE}
\newcommand{\isXLIIvI}{\VERSE  Ecce servus meus, suscipiam eum ; electus meus, complacuit sibi in illo anima mea : dedi spiritum meum super eum : judicium gentibus proferet. \EVERSE}
\newcommand{\isXLIIvII}{\VERSE  Non clamabit, neque accipiet personam, nec audietur vox ejus foris. \EVERSE}
\newcommand{\isXLIIvIII}{\VERSE  Calamum quassatum non conteret, et linum fumigans non extinguet : in veritate educet judicium. \EVERSE}
\newcommand{\isXLIIvIV}{\VERSE  Non erit tristis, neque turbulentus, donec ponat in terra judicium ; et legem ejus insulæ exspectabunt. \EVERSE}
\newcommand{\isXLIIvV}{\VERSE  Hæc dicit Dominus Deus, creans cælos, et extendens eos ; formans terram, et quæ germinant ex ea ; dans flatum populo qui est super eam, et spiritum calcantibus eam : \EVERSE}
\newcommand{\isXLIIvVI}{\VERSE  Ego Dominus vocavi te in justitia, et apprehendi manum tuam, et servavi te : et dedi te in fœdus populi, in lucem gentium, \EVERSE}
\newcommand{\isXLIIvVII}{\VERSE  ut aperires oculos cæcorum, et educeres de conclusione vinctum, de domo carceris sedentes in tenebris. \EVERSE}
\newcommand{\isXLIIvVIII}{\VERSE  Ego Dominus, hoc est nomen meum ; gloriam meam alteri non dabo, et laudem meam sculptilibus. \EVERSE}
\newcommand{\isXLIIvIX}{\VERSE  Quæ prima fuerunt, ecce venerunt ; nova quoque ego annuntio : antequam oriantur, audita vobis faciam. \EVERSE}
\newcommand{\isXLIIvX}{\VERSE  Cantate Domino canticum novum, laus ejus ab extremis terræ, qui descenditis in mare, et plenitudo ejus ; insulæ, et habitatores earum. \EVERSE}
\newcommand{\isXLIIvXI}{\VERSE  Sublevetur desertum et civitates ejus. In domibus habitabit Cedar : laudate, habitatores petræ ; de vertice montium clamabunt. \EVERSE}
\newcommand{\isXLIIvXII}{\VERSE  Ponent Domino gloriam, et laudem ejus in insulis nuntiabunt. \EVERSE}
\newcommand{\isXLIIvXIII}{\VERSE  Dominus sicut fortis egredietur, sicut vir præliator suscitabit zelum ; vociferabitur, et clamabit : super inimicos suos confortabitur. \EVERSE}
\newcommand{\isXLIIvXIV}{\VERSE  Tacui semper, silui, patiens fui : sicut parturiens loquar ; dissipabo, et absorbebo simul. \EVERSE}
\newcommand{\isXLIIvXV}{\VERSE  Desertos faciam montes et colles, et omne gramen eorum exsiccabo ; et ponam flumina in insulas, et stagna arefaciam. \EVERSE}
\newcommand{\isXLIIvXVI}{\VERSE  Et ducam cæcos in viam quam nesciunt, et in semitis quas ignoraverunt ambulare eos faciam ; ponam tenebras coram eis in lucem, et prava in recta ; hæc verba feci eis, et non dereliqui eos. \EVERSE}
\newcommand{\isXLIIvXVII}{\VERSE  Conversi sunt retrorsum, confundantur confusione, qui confidunt in sculptili ; qui dicunt conflatili : Vos dii nostri. \EVERSE}
\newcommand{\isXLIIvXVIII}{\VERSE  Surdi, audite, et cæci, intuemini ad videndum. \EVERSE}
\newcommand{\isXLIIvXIX}{\VERSE  Quis cæcus, nisi servus meus ; et surdus, nisi ad quem nuntios meos misi ? quis cæcus, nisi qui venundatus est ? et quis cæcus, nisi servus Domini ? \EVERSE}
\newcommand{\isXLIIvXX}{\VERSE  Qui vides multa, nonne custodies ? qui apertas habes aures, nonne audies ? \EVERSE}
\newcommand{\isXLIIvXXI}{\VERSE  Et Dominus voluit ut sanctificaret eum, et magnificaret legem, et extolleret. \EVERSE}
\newcommand{\isXLIIvXXII}{\VERSE  Ipse autem populus direptus, et vastatus ; laqueus juvenum omnes, et in domibus carcerum absconditi sunt ; facti sunt in rapinam, nec est qui eruat ; in direptionem, nec est qui dicat : Redde. \EVERSE}
\newcommand{\isXLIIvXXIII}{\VERSE  Quis est in vobis qui audiat hoc, attendat, et auscultet futura ? \EVERSE}
\newcommand{\isXLIIvXXIV}{\VERSE  Quis dedit in direptionem Jacob, et Israël vastantibus ? nonne Dominus ipse, cui peccavimus ? Et noluerunt in viis ejus ambulare, et non audierunt legem ejus. \EVERSE}
\newcommand{\isXLIIvXXV}{\VERSE  Et effudit super eum indignationem furoris sui, et forte bellum ; et combussit eum in circuitu, et non cognovit ; et succendit eum, et non intellexit. \EVERSE}
\newcommand{\isXLIIIvI}{\VERSE  Et nunc hæc dicit Dominus creans te, Jacob, et formans te, Israël : Noli timere, quia redemi te, et vocavi te nomine tuo : meus es tu. \EVERSE}
\newcommand{\isXLIIIvII}{\VERSE  Cum transieris per aquas, tecum ero, et flumina non operient te ; cum ambulaveris in igne, non combureris, et flamma non ardebit in te. \EVERSE}
\newcommand{\isXLIIIvIII}{\VERSE  Quia ego Dominus Deus tuus, Sanctus Israël, salvator tuus, dedi propitiationem tuam Ægyptum, Æthopiam, et Saba, pro te. \EVERSE}
\newcommand{\isXLIIIvIV}{\VERSE  Ex quo honorabilis factus es in oculis meis, et gloriosus, ego dilexi te, et dabo homines pro te, et populos pro anima tua. \EVERSE}
\newcommand{\isXLIIIvV}{\VERSE  Noli timere, quia ego tecum sum ; ab oriente adducam semen tuum, et ab occidente congregabo te. \EVERSE}
\newcommand{\isXLIIIvVI}{\VERSE  Dicam aquiloni : Da ; et austro : Noli prohibere : affer filios meos de longinquo, et filias meas ab extremis terræ. \EVERSE}
\newcommand{\isXLIIIvVII}{\VERSE  Et omnem qui invocat nomen meum, in gloriam meam creavi eum, formavi eum, et feci eum. \EVERSE}
\newcommand{\isXLIIIvVIII}{\VERSE  Educ foras populum cæcum, et oculos habentem ; surdum, et aures ei sunt. \EVERSE}
\newcommand{\isXLIIIvIX}{\VERSE  Omnes gentes congregatæ sunt simul, et collectæ sunt tribus. Quis in vobis annuntiet istud, et quæ prima sunt audire nos faciet ? Dent testes eorum, justificentur, et audiant, et dicant : Vere. \EVERSE}
\newcommand{\isXLIIIvX}{\VERSE  Vos testes mei, dicit Dominus, et servus meus quem elegi : ut sciatis, et credatis mihi, et intelligatis quia ego ipse sum ; ante me non est formatus Deus, et post me non erit. \EVERSE}
\newcommand{\isXLIIIvXI}{\VERSE  Ego sum, ego sum Dominus, et non est absque me salvator. \EVERSE}
\newcommand{\isXLIIIvXII}{\VERSE  Ego annuntiavi, et salvavi ; auditum feci, et non fuit in vobis alienus : vos testes mei, dicit Dominus, et ego Deus. \EVERSE}
\newcommand{\isXLIIIvXIII}{\VERSE  Et ab initio ego ipse, et non est qui de manu mea eruat. Operabor, et quis avertet illud ? \EVERSE}
\newcommand{\isXLIIIvXIV}{\VERSE  Hæc dicit Dominus, redemptor vester, Sanctus Israël : Propter vos misi in Babylonem, et detraxi vectes universos, et Chaldæos in navibus suis gloriantes. \EVERSE}
\newcommand{\isXLIIIvXV}{\VERSE  Ego Dominus, Sanctus vester, creans Israël, rex vester. \EVERSE}
\newcommand{\isXLIIIvXVI}{\VERSE  Hæc dicit Dominus, qui dedit in mari viam, et in aquis torrentibus semitam ; \EVERSE}
\newcommand{\isXLIIIvXVII}{\VERSE  qui eduxit quadrigam et equum, agmen et robustum : simul obdormierunt, nec resurgent ; contriti sunt quasi linum, et extincti sunt. \EVERSE}
\newcommand{\isXLIIIvXVIII}{\VERSE  Ne memineritis priorum, et antiqua ne intueamini. \EVERSE}
\newcommand{\isXLIIIvXIX}{\VERSE  Ecce ego facio nova, et nunc orientur, utique cognoscetis ea : ponam in deserto viam, et in invio flumina. \EVERSE}
\newcommand{\isXLIIIvXX}{\VERSE  Glorificabit me bestia agri, dracones, et struthiones : quia dedi in deserto aquas, flumina in invio, ut darem potum populo meo, electo meo. \EVERSE}
\newcommand{\isXLIIIvXXI}{\VERSE  Populum istum formavi mihi : laudem meam narrabit. \EVERSE}
\newcommand{\isXLIIIvXXII}{\VERSE  Non me invocasti, Jacob, nec laborasti in me, Israël. \EVERSE}
\newcommand{\isXLIIIvXXIII}{\VERSE  Non obtulisti mihi arietem holocausti tui, et victimis tuis non glorificasti me ; non te servire feci in oblatione, nec laborem tibi præbui in thure. \EVERSE}
\newcommand{\isXLIIIvXXIV}{\VERSE  Non emisti mihi argento calamum, et adipe victimarum tuarum non inebriasti me : verumtamen servire me fecisti in peccatis tuis ; præbuisti mihi laborem in iniquitatibus tuis. \EVERSE}
\newcommand{\isXLIIIvXXV}{\VERSE  Ego sum, ego sum ipse qui deleo iniquitates tuas propter me, et peccatorum tuorum non recordabor. \EVERSE}
\newcommand{\isXLIIIvXXVI}{\VERSE  Reduc me in memoriam, et judicemur simul : narra si quid habes ut justificeris. \EVERSE}
\newcommand{\isXLIIIvXXVII}{\VERSE  Pater tuus primus peccavit, et interpretes tui prævaricati sunt in me : \EVERSE}
\newcommand{\isXLIIIvXXVIII}{\VERSE  et contaminavi principes sanctos ; dedi ad internecionem Jacob, et Israël in blasphemiam. \EVERSE}
\newcommand{\isXLIVvI}{\VERSE  Et nunc audi, Jacob, serve meus, et Israël, quem elegi. \EVERSE}
\newcommand{\isXLIVvII}{\VERSE  Hæc dicit Dominus faciens et formans te, ab utero auxiliator tuus : Noli timere, serve meus Jacob, et rectissime, quem elegi. \EVERSE}
\newcommand{\isXLIVvIII}{\VERSE  Effundam enim aquas super sitientem, et fluenta super aridam ; effundam spiritum meum super semen tuum, et benedictionem meam super stirpem tuam : \EVERSE}
\newcommand{\isXLIVvIV}{\VERSE  et germinabunt inter herbas, quasi salices juxta præterfluentes aquas. \EVERSE}
\newcommand{\isXLIVvV}{\VERSE  Iste dicet : Domini ego sum ; et ille vocabit in nomine Jacob ; et hic scribet manu sua : Domino, et in nomine Israël assimilabitur. \EVERSE}
\newcommand{\isXLIVvVI}{\VERSE  Hæc dicit Dominus, rex Israël, et redemptor ejus, Dominus exercituum : Ego primus, et ego novissimus, et absque me non est deus. \EVERSE}
\newcommand{\isXLIVvVII}{\VERSE  Quis similis mei ? vocet, et annuntiet : et ordinem exponat mihi, ex quo constitui populum antiquum ; ventura et quæ futura sunt annuntient eis. \EVERSE}
\newcommand{\isXLIVvVIII}{\VERSE  Nolite timere, neque conturbemini : ex tunc audire te feci, et annuntiavi ; vos estis testes mei. Numquid est Deus absque me, et formator quem ego non noverim ? \EVERSE}
\newcommand{\isXLIVvIX}{\VERSE  Plastæ idoli omnes nihil sunt, et amantissima eorum non proderunt eis. Ipsi sunt testes eorum, quia non vident, neque intelligunt, ut confundantur. \EVERSE}
\newcommand{\isXLIVvX}{\VERSE  Quis formavit deum, et sculptile conflavit ad nihil utile ? \EVERSE}
\newcommand{\isXLIVvXI}{\VERSE  Ecce omnes participes ejus confundentur, fabri enim sunt ex hominibus ; convenient omnes, stabunt et pavebunt, et confundentur simul. \EVERSE}
\newcommand{\isXLIVvXII}{\VERSE  Faber ferrarius lima operatus est, in prunis et in malleis formavit illud, et operatus est in brachio fortitudinis suæ ; esuriet et deficiet, non bibet aquam et lassescet. \EVERSE}
\newcommand{\isXLIVvXIII}{\VERSE  Artifex lignarius extendit normam, formavit illud in runcina, fecit illud in angularibus, et in circino tornavit illud, et fecit imaginem viri quasi speciosum hominem habitantem in domo ; \EVERSE}
\newcommand{\isXLIVvXIV}{\VERSE  succidit cedros, tulit ilicem, et quercum, quæ steterat inter ligna saltus ; plantavit pinum, quam pluvia nutrivit : \EVERSE}
\newcommand{\isXLIVvXV}{\VERSE  et facta est hominibus in focum ; sumpsit ex eis, et calefactus est ; et succendit et coxit panes ; de reliquo autem operatus est deum et adoravit ; fecit sculptile, et curvatus est ante illud. \EVERSE}
\newcommand{\isXLIVvXVI}{\VERSE  Medium ejus combussit igni, et de medio ejus carnes comedit ; coxit pulmentum, et saturatus est, et calefactus est, et dixit : Vah ! calefactus sum, vidi focum ; \EVERSE}
\newcommand{\isXLIVvXVII}{\VERSE  reliquum autem ejus deum fecit et sculptile sibi ; curvatur ante illud, et adorat illud, et obsecrat, dicens : Libera me, quia deus meus es tu ! \EVERSE}
\newcommand{\isXLIVvXVIII}{\VERSE  Nescierunt, neque intellexerunt ; obliti enim sunt ne videant oculi eorum, et ne intelligant corde suo. \EVERSE}
\newcommand{\isXLIVvXIX}{\VERSE  Non recogitant in mente sua, neque cognoscunt, neque sentiunt, ut dicant : Medietatem ejus combussi igni, et coxi super carbones ejus panes ; coxi carnes et comedi, et de reliquo ejus idolum faciam ? ante truncum ligni procidam ? \EVERSE}
\newcommand{\isXLIVvXX}{\VERSE  Pars ejus cinis est ; cor insipiens adoravit illud, et non liberabit animam suam : neque dicet : Forte mendacium est in dextera mea. \EVERSE}
\newcommand{\isXLIVvXXI}{\VERSE  Memento horum Jacob, et Israël, quoniam servus meus es tu. Formavi te ; servus meus es tu, Israël, ne obliviscaris mei. \EVERSE}
\newcommand{\isXLIVvXXII}{\VERSE  Delevi ut nubem iniquitates tuas, et quasi nebulam peccata tua : revertere ad me, quoniam redemi te. \EVERSE}
\newcommand{\isXLIVvXXIII}{\VERSE  Laudate, cæli, quoniam misericordiam fecit Dominus ; jubilate, extrema terræ ; resonate, montes, laudationem, saltus et omne lignum ejus, quoniam redemit Dominus Jacob, et Israël gloriabitur. \EVERSE}
\newcommand{\isXLIVvXXIV}{\VERSE  Hæc dicit Dominus, redemptor tuus, et formator tuus ex utero : Ego sum Dominus, faciens omnia, extendens cælos solus, stabiliens terram, et nullus mecum ; \EVERSE}
\newcommand{\isXLIVvXXV}{\VERSE  irrita faciens signa divinorum, et ariolos in furorem vertens ; convertens sapientes retrorsum, et scientiam eorum stultam faciens ; \EVERSE}
\newcommand{\isXLIVvXXVI}{\VERSE  suscitans verbum servi sui, et consilium nuntiorum suorum complens ; qui dico Jerusalem : Habitaberis, et civitatibus Juda : Ædificabimini, et deserta ejus suscitabo ; \EVERSE}
\newcommand{\isXLIVvXXVII}{\VERSE  qui dico profundo : Desolare, et flumina tua arefaciam ; \EVERSE}
\newcommand{\isXLIVvXXVIII}{\VERSE  qui dico Cyro : Pastor meus es, et omnem voluntatem meam complebis ; qui dico Jerusalem : Ædificaberis, et templo : Fundaberis. \EVERSE}
\newcommand{\isXLVvI}{\VERSE  Hæc dicit Dominus christo meo Cyro, cujus apprehendi dexteram, ut subjiciam ante faciem ejus gentes, et dorsa regum vertam, et aperiam coram eo januas, et portæ non claudentur : \EVERSE}
\newcommand{\isXLVvII}{\VERSE  Ego ante te ibo, et gloriosos terræ humiliabo ; portas æreas conteram, et vectes ferreos confringam : \EVERSE}
\newcommand{\isXLVvIII}{\VERSE  et dabo tibi thesauros absconditos, et arcana secretorum, ut scias quia ego Dominus, qui voco nomen tuum, Deus Israël, \EVERSE}
\newcommand{\isXLVvIV}{\VERSE  propter servum meum Jacob, et Israël, electum meum ; et vocavi te nomine tuo : assimilavi te, et non cognovisti me. \EVERSE}
\newcommand{\isXLVvV}{\VERSE  Ego Dominus, et non est amplius ; extra me non est deus ; accinxi te, et non cognovisti me : \EVERSE}
\newcommand{\isXLVvVI}{\VERSE  ut sciant hi qui ab ortu solis et qui ab occidente, quoniam absque me non est : ego Dominus, et non est alter : \EVERSE}
\newcommand{\isXLVvVII}{\VERSE  formans lucem et creans tenebras, faciens pacem et creans malum : ego Dominus faciens omnia hæc. \EVERSE}
\newcommand{\isXLVvVIII}{\VERSE  Rorate, cæli, desuper, et nubes pluant justum ; aperiatur terra, et germinet Salvatorem, et justitia oriatur simul : ego Dominus creavi eum. \EVERSE}
\newcommand{\isXLVvIX}{\VERSE  Væ qui contradicit fictori suo, testa de samiis terræ ! Numquid dicet lutum figulo suo : Quid facis, et opus tuum absque manibus est ? \EVERSE}
\newcommand{\isXLVvX}{\VERSE  Væ qui dicit patri : Quid generas ? et mulieri : Quid parturis ? \EVERSE}
\newcommand{\isXLVvXI}{\VERSE  Hæc dicit Dominus, Sanctus Israël, plastes ejus : Ventura interrogate me ; super filios meos et super opus manuum mearum mandate mihi. \EVERSE}
\newcommand{\isXLVvXII}{\VERSE  Ego feci terram, et hominem super eam creavi ego : manus meæ tetenderunt cælos, et omni militiæ eorum mandavi. \EVERSE}
\newcommand{\isXLVvXIII}{\VERSE  Ego suscitavi eum ad justitiam, et omnes vias ejus dirigam ; ipse ædificabit civitatem meam, et captivitatem meam dimittet, non in pretio neque in muneribus, dicit Dominus Deus exercituum. \EVERSE}
\newcommand{\isXLVvXIV}{\VERSE  Hæc dicit Dominus : Labor Ægypti, et negotiatio Æthiopiæ, et Sabaim viri sublimes ad te transibunt, et tui erunt ; post te ambulabunt, vincti manicis pergent, et te adorabunt, teque deprecabuntur. Tantum in te est Deus, et non est absque te deus. \EVERSE}
\newcommand{\isXLVvXV}{\VERSE  Vere tu es Deus absconditus, Deus Israël, salvator. \EVERSE}
\newcommand{\isXLVvXVI}{\VERSE  Confusi sunt, et erubuerunt omnes : simul abierunt in confusionem fabricatores errorum. \EVERSE}
\newcommand{\isXLVvXVII}{\VERSE  Israël salvatus est in Domino salute æterna ; non confundemini, et non erubescetis usque in sæculum sæculi. \EVERSE}
\newcommand{\isXLVvXVIII}{\VERSE  Quia hæc dicit Dominus creans cælos, ipse Deus formans terram et faciens eam, ipse plastes ejus ; non in vanum creavit eam : ut habitaretur formavit eam : Ego Dominus, et non est alius. \EVERSE}
\newcommand{\isXLVvXIX}{\VERSE  Non in abscondito locutus sum, in loco terræ tenebroso ; non dixi semini Jacob frustra : Quærite me : ego Dominus loquens justitiam, annuntians recta. \EVERSE}
\newcommand{\isXLVvXX}{\VERSE  Congregamini, et venite, et accedite simul qui salvati estis ex gentibus : nescierunt qui levant lignum sculpturæ suæ, et rogant deum non salvantem. \EVERSE}
\newcommand{\isXLVvXXI}{\VERSE  Annuntiate, et venite, et consiliamini simul. Quis auditum fecit hoc ab initio, ex tunc prædixit illud ? numquid non ego Dominus, et non est ultra deus absque me ? Deus justus, et salvans non est præter me. \EVERSE}
\newcommand{\isXLVvXXII}{\VERSE  Convertimini ad me, et salvi eritis, omnes fines terræ, quia ego Deus, et non est alius. \EVERSE}
\newcommand{\isXLVvXXIII}{\VERSE  In memetipso juravi ; egredietur de ore meo justitiæ verbum, et non revertetur : quia mihi curvabitur omne genu, et jurabit omnis lingua. \EVERSE}
\newcommand{\isXLVvXXIV}{\VERSE  Ergo in Domino, dicet, meæ sunt justitiæ et imperium ; ad eum venient, et confundentur omnes qui repugnant ei. \EVERSE}
\newcommand{\isXLVvXXV}{\VERSE  In Domino justificabitur, et laudabitur omne semen Israël. \EVERSE}
\newcommand{\isXLVIvI}{\VERSE  Confractus est Bel, contritus est Nabo ; facta sunt simulacra eorum bestiis et jumentis, onera vestra gravi pondere usque ad lassitudinem. \EVERSE}
\newcommand{\isXLVIvII}{\VERSE  Contabuerunt, et contrita sunt simul ; non potuerunt salvare portantem, et anima eorum in captivitatem ibit. \EVERSE}
\newcommand{\isXLVIvIII}{\VERSE  Audite me, domus Jacob, et omne residuum domus Israël ; qui portamini a meo utero, qui gestamini a mea vulva. \EVERSE}
\newcommand{\isXLVIvIV}{\VERSE  Usque ad senectam ego ipse, et usque ad canos ego portabo ; ego feci, et ego feram ; ego portabo, et salvabo. \EVERSE}
\newcommand{\isXLVIvV}{\VERSE  Cui assimilastis me, et adæquastis, et comparastis me, et fecistis similem ? \EVERSE}
\newcommand{\isXLVIvVI}{\VERSE  Qui confertis aurum de sacculo, et argentum statera ponderatis, conducentes aurificem ut faciat deum, et procidunt, et adorant. \EVERSE}
\newcommand{\isXLVIvVII}{\VERSE  Portant illum in humeris gestantes, et ponentes in loco suo, et stabit, ac de loco suo non movebitur : sed et cum clamaverint ad eum, non audiet ; de tribulatione non salvabit eos. \EVERSE}
\newcommand{\isXLVIvVIII}{\VERSE  Mementote istud, et confundamini ; redite, prævaricatores, ad cor. \EVERSE}
\newcommand{\isXLVIvIX}{\VERSE  Recordamini prioris sæculi, quoniam ego sum Deus, et non est ultra deus, nec est similis mei. \EVERSE}
\newcommand{\isXLVIvX}{\VERSE  Annuntians ab exordio novissimum, et ab initio quæ necdum facta sunt, dicens : Consilium meum stabit, et omnis voluntas mea fiet. \EVERSE}
\newcommand{\isXLVIvXI}{\VERSE  Vocans ab oriente avem, et de terra longinqua virum voluntatis meæ : et locutus sum, et adducam illud ; creavi et faciam illud. \EVERSE}
\newcommand{\isXLVIvXII}{\VERSE  Audite me, duro corde, qui longe estis a justitia. \EVERSE}
\newcommand{\isXLVIvXIII}{\VERSE  Prope feci justitiam meam, non elongabitur, et salus mea non morabitur. Dabo in Sion salutem, et in Israël gloriam meam. \EVERSE}
\newcommand{\isXLVIIvI}{\VERSE  Descende, sede in pulvere, virgo filia Babylon : sede in terra ; non est solium filiæ Chaldæorum, quia ultra non vocaberis mollis et tenera. \EVERSE}
\newcommand{\isXLVIIvII}{\VERSE  Tolle molam, et mole farinam ; denuda turpitudinem tuam ; discooperi humerum, revela crura, transi flumina. \EVERSE}
\newcommand{\isXLVIIvIII}{\VERSE  Revelabitur ignominia tua, et videbitur opprobrium tuum ; ultionem capiam, et non resistet mihi homo. \EVERSE}
\newcommand{\isXLVIIvIV}{\VERSE  Redemptor noster, Dominus exercituum nomen illius, Sanctus Israël. \EVERSE}
\newcommand{\isXLVIIvV}{\VERSE  Sede tacens, et intra in tenebras, filia Chaldæorum, quia non vocaberis ultra domina regnorum. \EVERSE}
\newcommand{\isXLVIIvVI}{\VERSE  Iratus sum super populum meum : contaminavi hæreditatem meam, et dedi eos in manu tua : non posuisti eis misericordias ; super senem aggravasti jugum tuum valde. \EVERSE}
\newcommand{\isXLVIIvVII}{\VERSE  Et dixisti : In sempiternum ero domina. Non posuisti hæc super cor tuum, neque recordata es novissimi tui. \EVERSE}
\newcommand{\isXLVIIvVIII}{\VERSE  Et nunc audi hæc delicata, et habitans confidenter, quæ dicis in corde tuo : Ego sum, et non est præter me amplius ; non sedebo vidua, et ignorabo sterilitatem. \EVERSE}
\newcommand{\isXLVIIvIX}{\VERSE  Venient tibi duo hæc subito in die una, sterilitas et viduitas : universa venerunt super te, propter multitudinem maleficiorum tuorum, et propter duritiam incantatorum tuorum vehementem. \EVERSE}
\newcommand{\isXLVIIvX}{\VERSE  Et fiduciam habuisti in malitia tua, et dixisti : Non est qui videat me. Sapientia tua et scientia tua, hæc decepit te. Et dixisti in corde tuo : Ego sum, et præter me non est altera. \EVERSE}
\newcommand{\isXLVIIvXI}{\VERSE  Veniet super te malum, et nescies ortum ejus ; et irruet super te calamitas quam non poteris expiare ; veniet super te repente miseria quam nescies. \EVERSE}
\newcommand{\isXLVIIvXII}{\VERSE  Sta cum incantatoribus tuis et cum multitudine maleficiorum tuorum, in quibus laborasti ab adolescentia tua, si forte quod prosit tibi, aut si possis fieri fortior. \EVERSE}
\newcommand{\isXLVIIvXIII}{\VERSE  Defecisti in multitudine consiliorum tuorum. Stent, et salvent te augures cæli, qui contemplabantur sidera, et supputabant menses, ut ex eis annuntiarent ventura tibi. \EVERSE}
\newcommand{\isXLVIIvXIV}{\VERSE  Ecce facti sunt quasi stipula, ignis combussit eos ; non liberabunt animam suam de manu flammæ ; non sunt prunæ quibus calefiant, nec focus ut sedeant ad eum. \EVERSE}
\newcommand{\isXLVIIvXV}{\VERSE  Sic facta sunt tibi in quibuscumque laboraveras : negotiatores tui ab adolescentia tua, unusquisque in via sua erraverunt ; non est qui salvet te. \EVERSE}
\newcommand{\isXLVIIIvI}{\VERSE  Audite hæc, domus Jacob, qui vocamini nomine Israël, et de aquis Juda existis ; qui juratis in nomine Domini, et Dei Israël recordamini non in veritate neque in justitia. \EVERSE}
\newcommand{\isXLVIIIvII}{\VERSE  De civitate enim sancta vocati sunt, et super Deum Israël constabiliti sunt : Dominus exercituum nomen ejus. \EVERSE}
\newcommand{\isXLVIIIvIII}{\VERSE  Priora ex tunc annuntiavi, et ex ore meo exierunt, et audita feci ea : repente operatus sum, et venerunt. \EVERSE}
\newcommand{\isXLVIIIvIV}{\VERSE  Scivi enim quia durus es tu, et nervus ferreus cervix tua, et frons tua ærea. \EVERSE}
\newcommand{\isXLVIIIvV}{\VERSE  Prædixi tibi ex tunc ; antequam venirent, indicavi tibi, ne forte diceres : Idola mea fecerunt hæc, et sculptilia mea et conflatilia mandaverunt ista. \EVERSE}
\newcommand{\isXLVIIIvVI}{\VERSE  Quæ audisti, vide omnia ; vos autem, num annuntiastis ? Audita feci tibi nova ex tunc, et conservata sunt quæ nescis. \EVERSE}
\newcommand{\isXLVIIIvVII}{\VERSE  Nunc creata sunt et non ex tunc ; et ante diem, et non audisti ea, ne forte dicas : Ecce ego cognovi ea. \EVERSE}
\newcommand{\isXLVIIIvVIII}{\VERSE  Neque audisti, neque cognovisti, neque ex tunc aperta est auris tua : scio enim quia prævaricans prævaricaberis, et transgressorem ex utero vocavi te. \EVERSE}
\newcommand{\isXLVIIIvIX}{\VERSE  Propter nomen meum longe faciam furorem meum, et laude mea infrenabo te, ne intereas. \EVERSE}
\newcommand{\isXLVIIIvX}{\VERSE  Ecce excoxi te, sed non quasi argentum ; elegi te in camino paupertatis. \EVERSE}
\newcommand{\isXLVIIIvXI}{\VERSE  Propter me, propter me faciam, ut non blasphemer ; et gloriam meam alteri non dabo. \EVERSE}
\newcommand{\isXLVIIIvXII}{\VERSE  Audi me, Jacob, et Israël, quem ego voco : ego ipse, ego primus, et ego novissimus. \EVERSE}
\newcommand{\isXLVIIIvXIII}{\VERSE  Manus quoque mea fundavit terram, et dextera mea mensa est cælos ; ego vocabo eos, et stabunt simul. \EVERSE}
\newcommand{\isXLVIIIvXIV}{\VERSE  Congregamini, omnes vos, et audite : quis de eis annuntiavit hæc ? Dominus dilexit eum, faciet voluntatem suam in Babylone, et brachium suum in Chaldæis. \EVERSE}
\newcommand{\isXLVIIIvXV}{\VERSE  Ego, ego locutus sum, et vocavi eum ; adduxi eum, et directa est via ejus. \EVERSE}
\newcommand{\isXLVIIIvXVI}{\VERSE  Accedite ad me et audite hoc : non a principio in abscondito locutus sum : ex tempore antequam fieret, ibi eram : et nunc Dominus Deus misit me, et spiritus ejus. \EVERSE}
\newcommand{\isXLVIIIvXVII}{\VERSE  Hæc dicit Dominus, redemptor tuus, Sanctus Israël : Ego Dominus Deus tuus, docens te utilia, gubernans te in via qua ambulas. \EVERSE}
\newcommand{\isXLVIIIvXVIII}{\VERSE  Utinam attendisses mandata mea : facta fuisset sicut flumen pax tua, et justitia tua sicut gurgites maris : \EVERSE}
\newcommand{\isXLVIIIvXIX}{\VERSE  et fuisset quasi arena semen tuum, et stirps uteri tui ut lapilli ejus ; non interisset et non fuisset attritum nomen ejus a facie mea. \EVERSE}
\newcommand{\isXLVIIIvXX}{\VERSE  Egredimini de Babylone, fugite a Chaldæis, in voce exsultationis annuntiate : auditum facite hoc, et efferte illud usque ad extrema terræ. Dicite : Redemit Dominus servum suum Jacob. \EVERSE}
\newcommand{\isXLVIIIvXXI}{\VERSE  Non sitierunt in deserto, cum educeret eos : aquam de petra produxit eis, et scidit petram, et fluxerunt aquæ. \EVERSE}
\newcommand{\isXLVIIIvXXII}{\VERSE  Non est pax impiis, dicit Dominus. \EVERSE}
\newcommand{\isXLIXvI}{\VERSE  Audite, insulæ, et attendite, populi de longe : Dominus ab utero vocavit me ; de ventre matris meæ recordatus est nominis mei. \EVERSE}
\newcommand{\isXLIXvII}{\VERSE  Et posuit os meum quasi gladium acutum, in umbra manus suæ protexit me, et posuit me sicut sagittam electam : in pharetra sua abscondit me. \EVERSE}
\newcommand{\isXLIXvIII}{\VERSE  Et dixit mihi : Servus meus es tu Israël, quia in te gloriabor. \EVERSE}
\newcommand{\isXLIXvIV}{\VERSE  Et ego dixi : In vacuum laboravi ; sine causa et vane fortitudinem meam consumpsi : ergo judicium meum cum Domino, et opus meum cum Deo meo. \EVERSE}
\newcommand{\isXLIXvV}{\VERSE  Et nunc dicit Dominus, formans me ex utero servum sibi, ut reducam Jacob ad eum, et Israël non congregabitur ; et glorificatus sum in oculis Domini, et Deus meus factus est fortitudo mea. \EVERSE}
\newcommand{\isXLIXvVI}{\VERSE  Et dixit : Parum est ut sis mihi servus ad suscitandas tribus Jacob, et fæces Israël convertendas : ecce dedi te in lucem gentium, ut sis salus mea usque ad extremum terræ. \EVERSE}
\newcommand{\isXLIXvVII}{\VERSE  Hæc dicit Dominus, redemptor Israël, Sanctus ejus, ad contemptibilem animam, ad abominatam gentem, ad servum dominorum : Reges videbunt, et consurgent principes, et adorabunt propter Dominum, quia fidelis est, et Sanctum Israël qui elegit te. \EVERSE}
\newcommand{\isXLIXvVIII}{\VERSE  Hæc dicit Dominus : In tempore placito exaudivi te, et in die salutis auxiliatus sum tui : et servavi te, et dedi te in fœdus populi, ut suscitares terram, et possideres hæreditates dissipatas ; \EVERSE}
\newcommand{\isXLIXvIX}{\VERSE  ut diceres his qui vincti sunt : Exite, et his qui in tenebris : Revelamini. Super vias pascentur, et in omnibus planis pascua eorum. \EVERSE}
\newcommand{\isXLIXvX}{\VERSE  Non esurient neque sitient, et non percutiet eos æstus et sol, quia miserator eorum reget eos, et ad fontes aquarum potabit eos. \EVERSE}
\newcommand{\isXLIXvXI}{\VERSE  Et ponam omnes montes meos in viam, et semitæ meæ exaltabuntur. \EVERSE}
\newcommand{\isXLIXvXII}{\VERSE  Ecce isti de longe venient, et ecce illi ab aquilone et mari, et isti de terra australi. \EVERSE}
\newcommand{\isXLIXvXIII}{\VERSE  Laudate, cæli, et exsulta, terra ; jubilate, montes, laudem, quia consolatus est Dominus populum suum, et pauperum suorum miserebitur. \EVERSE}
\newcommand{\isXLIXvXIV}{\VERSE  Et dixit Sion : Dereliquit me Dominus, et Dominus oblitus est mei. \EVERSE}
\newcommand{\isXLIXvXV}{\VERSE  Numquid oblivisci potest mulier infantem suum, ut non misereatur filio uteri sui ? Etsi illa oblita fuerit, ego tamen non obliviscar tui. \EVERSE}
\newcommand{\isXLIXvXVI}{\VERSE  Ecce in manibus meis descripsi te ; muri tui coram oculis meis semper. \EVERSE}
\newcommand{\isXLIXvXVII}{\VERSE  Venerunt structores tui ; destruentes te et dissipantes a te exibunt. \EVERSE}
\newcommand{\isXLIXvXVIII}{\VERSE  Leva in circuitu oculos tuos, et vide : omnes isti congregati sunt, venerunt tibi. Vivo ego, dicit Dominus, quia omnibus his velut ornamento vestieris, et circumdabis tibi eos quasi sponsa ; \EVERSE}
\newcommand{\isXLIXvXIX}{\VERSE  quia deserta tua, et solitudines tuæ, et terra ruinæ tuæ, nunc angusta erunt præ habitatoribus ; et longe fugabuntur qui absorbebant te. \EVERSE}
\newcommand{\isXLIXvXX}{\VERSE  Adhuc dicent in auribus tuis filii sterilitatis tuæ : Angustus est mihi locus ; fac spatium mihi ut habitem. \EVERSE}
\newcommand{\isXLIXvXXI}{\VERSE  Et dices in corde tuo : Quis genuit mihi istos ? ego sterilis et non pariens, transmigrata, et captiva ; et istos quis enutrivit ? ego destituta et sola ; et isti ubi erant ? \EVERSE}
\newcommand{\isXLIXvXXII}{\VERSE  Hæc dicit Dominus Deus : Ecce levabo ad gentes manum meam, et ad populos exaltabo signum meum : et afferent filios tuos in ulnis, et filias tuas super humeros portabunt. \EVERSE}
\newcommand{\isXLIXvXXIII}{\VERSE  Et erunt reges nutritii tui, et reginæ nutrices tuæ ; vultu in terram demisso adorabunt te, et pulverem pedum tuorum lingent. Et scies quia ego Dominus, super quo non confundentur qui exspectant eum. \EVERSE}
\newcommand{\isXLIXvXXIV}{\VERSE  Numquid tolletur a forti præda ? aut quod captum fuerit a robusto, salvum esse poterit ? \EVERSE}
\newcommand{\isXLIXvXXV}{\VERSE  Quia hæc dicit Dominus : Equidem, et captivitas a forti tolletur, et quod ablatum fuerit a robusto, salvabitur. Eos vero qui judicaverunt te, ego judicabo, et filios tuos ego salvabo. \EVERSE}
\newcommand{\isXLIXvXXVI}{\VERSE  Et cibabo hostes tuos carnibus suis, et quasi musto, sanguine suo inebriabuntur, et sciet omnis caro quia ego Dominus salvans te, et redemptor tuus fortis Jacob. \EVERSE}
\newcommand{\isLvI}{\VERSE  Hæc dicit Dominus : Quis est hic liber repudii matris vestræ, quo dimisi eam ? aut quis est creditor meus, cui vendidi vos ? Ecce in iniquitatibus vestris venditi estis, et in sceleribus vestris dimisi matrem vestram. \EVERSE}
\newcommand{\isLvII}{\VERSE  Quia veni, et non erat vir ; vocavi, et non erat qui audiret. Numquid abbreviata et parvula facta est manus mea, ut non possim redimere ? aut non est in me virtus ad liberandum ? Ecce in increpatione mea desertum faciam mare, ponam flumina in siccum ; computrescent pisces sine aqua, et morientur in siti. \EVERSE}
\newcommand{\isLvIII}{\VERSE  Induam cælos tenebris, et saccum ponam operimentum eorum. \EVERSE}
\newcommand{\isLvIV}{\VERSE  Dominus dedit mihi linguam eruditam, ut sciam sustentare eum qui lassus est verbo. Erigit mane, mane erigit mihi aurem, ut audiam quasi magistrum. \EVERSE}
\newcommand{\isLvV}{\VERSE  Dominus Deus aperuit mihi aurem, ego autem non contradico : retrorsum non abii. \EVERSE}
\newcommand{\isLvVI}{\VERSE  Corpus meum dedi percutientibus, et genas meas vellentibus ; faciem meam non averti ab increpantibus et conspuentibus in me. \EVERSE}
\newcommand{\isLvVII}{\VERSE  Dominus Deus auxiliator meus, ideo non sum confusus ; ideo posui faciem meam ut petram durissimam, et scio quoniam non confundar. \EVERSE}
\newcommand{\isLvVIII}{\VERSE  Juxta est qui justificat me ; quis contradicet mihi ? Stemus simul ; quis est adversarius meus ? accedat ad me. \EVERSE}
\newcommand{\isLvIX}{\VERSE  Ecce Dominus Deus auxiliator meus ; quis est qui condemnet me ? Ecce omnes quasi vestimentum conterentur ; tinea comedet eos. \EVERSE}
\newcommand{\isLvX}{\VERSE  Quis ex vobis timens Dominum, audiens vocem servi sui ? Qui ambulavit in tenebris, et non est lumen ei, speret in nomine Domini, et innitatur super Deum suum. \EVERSE}
\newcommand{\isLvXI}{\VERSE  Ecce vos omnes accendentes ignem, accincti flammis : ambulate in lumine ignis vestri, et in flammis quas succendistis ; de manu mea factum est hoc vobis : in doloribus dormietis. \EVERSE}
\newcommand{\isLIvI}{\VERSE  Audite me, qui sequimini quod justum est, et quæritis Dominum ; attendite ad petram unde excisi estis, et ad cavernam laci de qua præcisi estis. \EVERSE}
\newcommand{\isLIvII}{\VERSE  Attendite ad Abraham, patrem vestrum, et ad Saram, quæ peperit vos : quia unum vocavi eum, et benedixi ei, et multiplicavi eum. \EVERSE}
\newcommand{\isLIvIII}{\VERSE  Consolabitur ergo Dominus Sion, et consolabitur omnes ruinas ejus : et ponet desertum ejus quasi delicias, et solitudinem ejus quasi hortum Domini. Gaudium et lætitia invenietur in ea, gratiarum actio et vox laudis. \EVERSE}
\newcommand{\isLIvIV}{\VERSE  Attendite ad me, popule meus, et tribus mea, me audite : quia lex a me exiet, et judicium meum in lucem populorum requiescet. \EVERSE}
\newcommand{\isLIvV}{\VERSE  Prope est justus meus, egressus est salvator meus, et brachia mea populos judicabunt ; me insulæ exspectabunt, et brachium meum sustinebunt. \EVERSE}
\newcommand{\isLIvVI}{\VERSE  Levate in cælum oculos vestros, et videte sub terra deorsum : quia cæli sicut fumus liquescent, et terra sicut vestimentum atteretur, et habitatores ejus sicut hæc interibunt : salus autem mea in sempiternum erit, et justitia mea non deficiet. \EVERSE}
\newcommand{\isLIvVII}{\VERSE  Audite me, qui scitis justum, populus meus, lex mea in corde eorum : nolite timere opprobrium hominum, et blasphemias eorum ne metuatis : \EVERSE}
\newcommand{\isLIvVIII}{\VERSE  sicut enim vestimentum, sic comedet eos vermis, et sicut lanam, sic devorabit eos tinea : salus autem mea in sempiternum erit, et justitia mea in generationes generationum. \EVERSE}
\newcommand{\isLIvIX}{\VERSE  Consurge, consurge, induere fortitudinem, brachium Domini ! consurge sicut in diebus antiquis, in generationibus sæculorum. Numquid non tu percussisti superbum, vulnerasti draconem ? \EVERSE}
\newcommand{\isLIvX}{\VERSE  numquid non tu siccasti mare, aquam abyssi vehementis ; qui posuisti profundum maris viam, ut transirent liberati ? \EVERSE}
\newcommand{\isLIvXI}{\VERSE  Et nunc qui redempti sunt a Domino, revertentur, et venient in Sion laudantes, et lætitia sempiterna super capita eorum : gaudium et lætitiam tenebunt ; fugiet dolor et gemitus. \EVERSE}
\newcommand{\isLIvXII}{\VERSE  Ego, ego ipse consolabor vos. Quis tu, ut timeres ab homine mortali, et a filio hominis qui quasi fœnum ita arescet ? \EVERSE}
\newcommand{\isLIvXIII}{\VERSE  Et oblitus es Domini, factoris tui, qui tetendit cælos et fundavit terram ; et formidasti jugiter tota die a facie furoris ejus qui te tribulabat, et paraverat ad perdendum. Ubi nunc est furor tribulantis ? \EVERSE}
\newcommand{\isLIvXIV}{\VERSE  Cito veniet gradiens ad aperiendum ; et non interficiet usque ad internecionem, nec deficiet panis ejus. \EVERSE}
\newcommand{\isLIvXV}{\VERSE  Ego autem sum Dominus Deus tuus, qui conturbo mare, et intumescunt fluctus ejus : Dominus exercituum nomen meum. \EVERSE}
\newcommand{\isLIvXVI}{\VERSE  Posui verba mea in ore tuo, et in umbra manus meæ protexi te, ut plantes cælos, et fundes terram, et dicas ad Sion : Populus meus es tu. \EVERSE}
\newcommand{\isLIvXVII}{\VERSE  Elevare, elevare, consurge, Jerusalem, quæ bibisti de manu Domini calicem iræ ejus ; usque ad fundum calicis soporis bibisti, et potasti usque ad fæces. \EVERSE}
\newcommand{\isLIvXVIII}{\VERSE  Non est qui sustentet eam, ex omnibus filiis quos genuit ; et non est qui apprehendat manum ejus, ex omnibus filiis quos enutrivit. \EVERSE}
\newcommand{\isLIvXIX}{\VERSE  Duo sunt quæ occurrerunt tibi ; quis contristabitur super te ? Vastitas, et contritio, et fames, et gladius ; quis consolabitur te ? \EVERSE}
\newcommand{\isLIvXX}{\VERSE  Filii tui projecti sunt, dormierunt in capite omnium viarum sicut oryx illaqueatus, pleni indignatione Domini, increpatione Dei tui. \EVERSE}
\newcommand{\isLIvXXI}{\VERSE  Idcirco audi hoc, paupercula, et ebria non a vino. \EVERSE}
\newcommand{\isLIvXXII}{\VERSE  Hæc dicit dominator tuus Dominus, et Deus tuus, qui pugnabit pro populo suo : Ecce tuli de manu tua calicem soporis, fundum calicis indignationis meæ : non adjicies ut bibas illum ultra. \EVERSE}
\newcommand{\isLIvXXIII}{\VERSE  Et ponam illum in manu eorum qui te humiliaverunt, et dixerunt animæ tuæ : Incurvare, ut transeamus ; et posuisti ut terram corpus tuum, et quasi viam transeuntibus. \EVERSE}
\newcommand{\isLIIvI}{\VERSE  Consurge, consurge, induere fortitudine tua, Sion ! induere vestimentis gloriæ tuæ, Jerusalem, civitas Sancti, quia non adjiciet ultra ut pertranseat per te incircumcisus et immundus. \EVERSE}
\newcommand{\isLIIvII}{\VERSE  Excutere de pulvere, consurge ; sede, Jerusalem ! solve vincula colli tui, captiva filia Sion. \EVERSE}
\newcommand{\isLIIvIII}{\VERSE  Quia hæc dicit Dominus : Gratis venundati estis, et sine argento redimemini. \EVERSE}
\newcommand{\isLIIvIV}{\VERSE  Quia hæc dicit Dominus Deus : In Ægyptum descendit populus meus in principio, ut colonus esset ibi, et Assur absque ulla causa calumniatus est eum. \EVERSE}
\newcommand{\isLIIvV}{\VERSE  Et nunc quid mihi est hic, dicit Dominus, quoniam ablatus est populus meus gratis ? Dominatores ejus inique agunt, dicit Dominus, et jugiter tota die nomen meum blasphematur. \EVERSE}
\newcommand{\isLIIvVI}{\VERSE  Propter hoc sciet populus meus nomen meum in die illa : quia ego ipse qui loquebar, ecce adsum. \EVERSE}
\newcommand{\isLIIvVII}{\VERSE  Quam pulchri super montes pedes annuntiantis et prædicantis pacem ; annuntiantis bonum, prædicantis salutem, dicentis Sion : Regnabit Deus tuus ! \EVERSE}
\newcommand{\isLIIvVIII}{\VERSE  Vox speculatorum tuorum : levaverunt vocem, simul laudabunt, quia oculo ad oculum videbunt cum converterit Dominus Sion. \EVERSE}
\newcommand{\isLIIvIX}{\VERSE  Gaudete, et laudate simul, deserta Jerusalem, quia consolatus est Dominus populum suum ; redemit Jerusalem. \EVERSE}
\newcommand{\isLIIvX}{\VERSE  Paravit Dominus brachium sanctum suum in oculis omnium gentium ; et videbunt omnes fines terræ salutare Dei nostri. \EVERSE}
\newcommand{\isLIIvXI}{\VERSE  Recedite, recedite ; exite inde, pollutum nolite tangere ; exite de medio ejus ; mundamini, qui fertis vasa Domini. \EVERSE}
\newcommand{\isLIIvXII}{\VERSE  Quoniam non in tumultu exibitis, nec in fuga properabitis ; præcedet enim vos Dominus, et congregabit vos Deus Israël. \EVERSE}
\newcommand{\isLIIvXIII}{\VERSE  Ecce intelliget servus meus, exaltabitur et elevabitur, et sublimis erit valde. \EVERSE}
\newcommand{\isLIIvXIV}{\VERSE  Sicut obstupuerunt super te multi, sic inglorius erit inter viros aspectus ejus, et forma ejus inter filios hominum. \EVERSE}
\newcommand{\isLIIvXV}{\VERSE  Iste asperget gentes multas ; super ipsum continebunt reges os suum : quia quibus non est narratum de eo viderunt, et qui non audierunt contemplati sunt. \EVERSE}
\newcommand{\isLIIIvI}{\VERSE  Quis credidit auditui nostro ? et brachium Domini cui revelatum est ? \EVERSE}
\newcommand{\isLIIIvII}{\VERSE  Et ascendet sicut virgultum coram eo, et sicut radix de terra sitienti. Non est species ei, neque decor, et vidimus eum, et non erat aspectus, et desideravimus eum : \EVERSE}
\newcommand{\isLIIIvIII}{\VERSE  despectum, et novissimum virorum, virum dolorum, et scientem infirmitatem, et quasi absconditus vultus ejus et despectus, unde nec reputavimus eum. \EVERSE}
\newcommand{\isLIIIvIV}{\VERSE  Vere languores nostros ipse tulit, et dolores nostros ipse portavit ; et nos putavimus eum quasi leprosum, et percussum a Deo, et humiliatum. \EVERSE}
\newcommand{\isLIIIvV}{\VERSE  Ipse autem vulneratus est propter iniquitates nostras ; attritus est propter scelera nostra : disciplina pacis nostræ super eum, et livore ejus sanati sumus. \EVERSE}
\newcommand{\isLIIIvVI}{\VERSE  Omnes nos quasi oves erravimus, unusquisque in viam suam declinavit : et posuit Dominus in eo iniquitatem omnium nostrum. \EVERSE}
\newcommand{\isLIIIvVII}{\VERSE  Oblatus est quia ipse voluit, et non aperuit os suum ; sicut ovis ad occisionem ducetur, et quasi agnus coram tondente se obmutescet, et non aperiet os suum. \EVERSE}
\newcommand{\isLIIIvVIII}{\VERSE  De angustia, et de judicio sublatus est. Generationem ejus quis enarrabit ? quia abscissus est de terra viventium : propter scelus populi mei percussi eum. \EVERSE}
\newcommand{\isLIIIvIX}{\VERSE  Et dabit impios pro sepultura, et divitem pro morte sua, eo quod iniquitatem non fecerit, neque dolus fuerit in ore ejus. \EVERSE}
\newcommand{\isLIIIvX}{\VERSE  Et Dominus voluit conterere eum in infirmitate. Si posuerit pro peccato animam suam, videbit semen longævum, et voluntas Domini in manu ejus dirigetur. \EVERSE}
\newcommand{\isLIIIvXI}{\VERSE  Pro eo quod laboravit anima ejus, videbit et saturabitur. In scientia sua justificabit ipse justus servus meus multos, et iniquitates eorum ipse portabit. \EVERSE}
\newcommand{\isLIIIvXII}{\VERSE  Ideo dispertiam ei plurimos, et fortium dividet spolia, pro eo quod tradidit in mortem animam suam, et cum sceleratis reputatus est, et ipse peccata multorum tulit, et pro transgressoribus rogavit. \EVERSE}
\newcommand{\isLIVvI}{\VERSE  Lauda, sterilis, quæ non paris ; decanta laudem, et hinni, quæ non pariebas : quoniam multi filii desertæ magis quam ejus quæ habet virum, dicit Dominus. \EVERSE}
\newcommand{\isLIVvII}{\VERSE  Dilata locum tentorii tui, et pelles tabernaculorum tuorum extende : ne parcas : longos fac funiculos tuos, et clavos tuos consolida. \EVERSE}
\newcommand{\isLIVvIII}{\VERSE  Ad dexteram enim et ad lævam penetrabis, et semen tuum gentes hæreditabit, et civitates desertas inhabitabit. \EVERSE}
\newcommand{\isLIVvIV}{\VERSE  Noli timere, quia non confunderis, neque erubesces ; non enim te pudebit, quia confusionis adolescentiæ tuæ oblivisceris, et opprobrii viduitatis tuæ non recordaberis amplius. \EVERSE}
\newcommand{\isLIVvV}{\VERSE  Quia dominabitur tui qui fecit te, Dominus exercituum nomen ejus, et redemptor tuus, Sanctus Israël : Deus omnis terræ vocabitur. \EVERSE}
\newcommand{\isLIVvVI}{\VERSE  Quia et mulierem derelictam et mœrentem spiritu vocavit te Dominus, et uxorem ab adolescentia abjectam, dixit Deus tuus. \EVERSE}
\newcommand{\isLIVvVII}{\VERSE  Ad punctum in modico dereliqui te, et in miserationibus magnis congregabo te. \EVERSE}
\newcommand{\isLIVvVIII}{\VERSE  In momento indignationis abscondi faciem meam parumper a te ; et in misericordia sempiterna misertus sum tui, dixit redemptor tuus, Dominus. \EVERSE}
\newcommand{\isLIVvIX}{\VERSE  Sicut in diebus Noë istud mihi est, cui juravi ne inducerem aquas Noë ultra super terram ; sic juravi ut non irascar tibi, et non increpem te. \EVERSE}
\newcommand{\isLIVvX}{\VERSE  Montes enim commovebuntur, et colles contremiscent ; misericordia autem mea non recedet a te, et fœdus pacis meæ non movebitur, dixit miserator tuus Dominus. \EVERSE}
\newcommand{\isLIVvXI}{\VERSE  Paupercula, tempestate convulsa absque ulla consolatione, ecce ego sternam per ordinem lapides tuos, et fundabo te in sapphiris : \EVERSE}
\newcommand{\isLIVvXII}{\VERSE  et ponam jaspidem propugnacula tua, et portas tuas in lapides sculptos, et omnes terminos tuos in lapides desiderabiles ; \EVERSE}
\newcommand{\isLIVvXIII}{\VERSE  universos filios tuos doctos a Domino, et multitudinem pacis filiis tuis. \EVERSE}
\newcommand{\isLIVvXIV}{\VERSE  Et in justitia fundaberis : recede procul a calumnia, quia non timebis, et a pavore, quia non appropinquabit tibi. \EVERSE}
\newcommand{\isLIVvXV}{\VERSE  Ecce accola veniet qui non erat mecum, advena quondam tuus adjungetur tibi. \EVERSE}
\newcommand{\isLIVvXVI}{\VERSE  Ecce ego creavi fabrum sufflantem in igne prunas, et proferentem vas in opus suum ; et ego creavi interfectorem ad disperdendum. \EVERSE}
\newcommand{\isLIVvXVII}{\VERSE  Omne vas quod fictum est contra te, non dirigetur, et omnem linguam resistentem tibi in judicio, judicabis. Hæc est hæreditas servorum Domini, et justitia eorum apud me, dicit Dominus. \EVERSE}
\newcommand{\isLVvI}{\VERSE  Omnes sitientes, venite ad aquas, et qui non habetis argentum, properate, emite, et comedite : venite, emite absque argento et absque ulla commutatione vinum et lac. \EVERSE}
\newcommand{\isLVvII}{\VERSE  Quare appenditis argentum non in panibus, et laborem vestrum non in saturitate ? Audite, audientes me, et comedite bonum, et delectabitur in crassitudine anima vestra. \EVERSE}
\newcommand{\isLVvIII}{\VERSE  Inclinate aurem vestram, et venite ad me ; audite, et vivet anima vestra, et feriam vobiscum pactum sempiternum, misericordias David fideles. \EVERSE}
\newcommand{\isLVvIV}{\VERSE  Ecce testem populis dedi eum, ducem ac præceptorem gentibus. \EVERSE}
\newcommand{\isLVvV}{\VERSE  Ecce gentem quam nesciebas vocabis, et gentes quæ te non cognoverunt ad te current, propter Dominum Deum tuum, et Sanctum Israël, quia glorificavit te. \EVERSE}
\newcommand{\isLVvVI}{\VERSE  Quærite Dominum dum inveniri potest ; invocate eum dum prope est. \EVERSE}
\newcommand{\isLVvVII}{\VERSE  Derelinquat impius viam suam, et vir iniquus cogitationes suas, et revertatur ad Dominum, et miserebitur ejus ; et ad Deum nostrum, quoniam multus est ad ignoscendum. \EVERSE}
\newcommand{\isLVvVIII}{\VERSE  Non enim cogitationes meæ cogitationes vestræ, neque viæ vestræ viæ meæ, dicit Dominus. \EVERSE}
\newcommand{\isLVvIX}{\VERSE  Quia sicut exaltantur cæli a terra, sic exaltatæ sunt viæ meæ a viis vestris, et cogitationes meæ a cogitationibus vestris. \EVERSE}
\newcommand{\isLVvX}{\VERSE  Et quomodo descendit imber et nix de cælo, et illuc ultra non revertitur, sed inebriat terram, et infundit eam, et germinare eam facit, et dat semen serenti, et panem comedenti : \EVERSE}
\newcommand{\isLVvXI}{\VERSE  sic erit verbum meum quod egredietur de ore meo ; non revertetur ad me vacuum, sed faciet quæcumque volui, et prosperabitur in his ad quæ misi illud. \EVERSE}
\newcommand{\isLVvXII}{\VERSE  Quia in lætitia egrediemini, et in pace deducemini ; montes et colles cantabunt coram vobis laudem, et omnia ligna regionis plaudent manu. \EVERSE}
\newcommand{\isLVvXIII}{\VERSE  Pro saliunca ascendet abies, et pro urtica crescet myrtus ; et erit Dominus nominatus in signum æternum quod non auferetur. \EVERSE}
\newcommand{\isLVIvI}{\VERSE  Hæc dicit Dominus : Custodite judicium, et facite justitiam, quia juxta est salus mea ut veniat, et justitia mea ut reveletur. \EVERSE}
\newcommand{\isLVIvII}{\VERSE  Beatus vir qui facit hoc, et filius hominis qui apprehendet istud, custodiens sabbatum ne polluat illud, custodiens manus suas ne faciat omne malum. \EVERSE}
\newcommand{\isLVIvIII}{\VERSE  Et non dicat filius advenæ qui adhæret Domino, dicens : Separatione dividet me Dominus a populo suo ; et non dicat eunuchus : Ecce ego lignum aridum. \EVERSE}
\newcommand{\isLVIvIV}{\VERSE  Quia hæc dicit Dominus eunuchis : Qui custodierint sabbata mea, et elegerint quæ ego volui, et tenuerint fœdus meum, \EVERSE}
\newcommand{\isLVIvV}{\VERSE  dabo eis in domo mea et in muris meis locum, et nomen melius a filiis et filiabus : nomen sempiternum dabo eis, quod non peribit. \EVERSE}
\newcommand{\isLVIvVI}{\VERSE  Et filios advenæ, qui adhærent Domino, ut colant eum, et diligant nomen ejus, ut sint ei in servos ; omnem custodientem sabbatum ne polluat illud, et tenentem fœdus meum ; \EVERSE}
\newcommand{\isLVIvVII}{\VERSE  adducam eos in montem sanctum meum, et lætificabo eos in domo orationis meæ ; holocausta eorum et victimæ eorum placebunt mihi super altari meo, quia domus mea domus orationis vocabitur cunctis populis. \EVERSE}
\newcommand{\isLVIvVIII}{\VERSE  Ait Dominus Deus, qui congregat dispersos Israël : Adhuc congregabo ad eum congregatos ejus. \EVERSE}
\newcommand{\isLVIvIX}{\VERSE  Omnes bestiæ agri, venite ad devorandum, universæ bestiæ saltus. \EVERSE}
\newcommand{\isLVIvX}{\VERSE  Speculatores ejus cæci omnes ; nescierunt universi : canes muti non valentes latrare, videntes vana, dormientes, et amantes somnia. \EVERSE}
\newcommand{\isLVIvXI}{\VERSE  Et canes imprudentissimi nescierunt saturitatem ; ipsi pastores ignoraverunt intelligentiam : omnes in viam suam declinaverunt ; unusquisque ad avaritiam suam, a summo usque ad novissimum. \EVERSE}
\newcommand{\isLVIvXII}{\VERSE  Venite, sumamus vinum, et impleamur ebrietate ; et erit sicut hodie, sic et cras, et multo amplius. \EVERSE}
\newcommand{\isLVIIvI}{\VERSE  Justus perit, et non est qui recogitet in corde suo ; et viri misericordiæ colliguntur, quia non est qui intelligat : a facie enim malitiæ collectus est justus. \EVERSE}
\newcommand{\isLVIIvII}{\VERSE  Veniat pax ; requiescat in cubili suo qui ambulavit in directione sua. \EVERSE}
\newcommand{\isLVIIvIII}{\VERSE  Vos autem accedite huc, filii auguratricis, semen adulteri et fornicariæ. \EVERSE}
\newcommand{\isLVIIvIV}{\VERSE  Super quem lusistis ? super quem dilatastis os, et ejecistis linguam ? Numquid non vos filii scelesti, semen mendax, \EVERSE}
\newcommand{\isLVIIvV}{\VERSE  qui consolamini in diis subter omne lignum frondosum, immolantes parvulos in torrentibus, subter eminentes petras ? \EVERSE}
\newcommand{\isLVIIvVI}{\VERSE  In partibus torrentis pars tua ; hæc est sors tua : et ipsis effudisti libamen, obtulisti sacrificium. Numquid super his non indignabor ? \EVERSE}
\newcommand{\isLVIIvVII}{\VERSE  Super montem excelsum et sublimem posuisti cubile tuum, et illuc ascendisti ut immolares hostias. \EVERSE}
\newcommand{\isLVIIvVIII}{\VERSE  Et post ostium, et retro postem, posuisti memoriale tuum. Quia juxta me discooperuisti, et suscepisti adulterum, dilatasti cubile tuum, et pepigisti cum eis fœdus ; dilexisti stratum eorum manu aperta. \EVERSE}
\newcommand{\isLVIIvIX}{\VERSE  Et ornasti te regi unguento, et multiplicasti pigmenta tua. Misisti legatos tuos procul, et humiliata es usque ad inferos. \EVERSE}
\newcommand{\isLVIIvX}{\VERSE  In multitudine viæ tuæ laborasti ; non dixisti : Quiescam. Vitam manus tuæ invenisti ; propterea non rogasti. \EVERSE}
\newcommand{\isLVIIvXI}{\VERSE  Pro quo sollicita timuisti, quia mentita es, et mei non es recordata, neque cogitasti in corde tuo ? Quia ego tacens et quasi non videns, et mei oblita es. \EVERSE}
\newcommand{\isLVIIvXII}{\VERSE  Ego annuntiabo justitiam tuam, et opera tua non proderunt tibi. \EVERSE}
\newcommand{\isLVIIvXIII}{\VERSE  Cum clamaveris, liberent te congregati tui, et omnes eos auferet ventus, tollet aura. Qui autem fiduciam habet mei, hæreditabit terram, et possidebit montem sanctum meum. \EVERSE}
\newcommand{\isLVIIvXIV}{\VERSE  Et dicam : Viam facite, præbete iter ; declinate de semita, auferte offendicula de via populi mei. \EVERSE}
\newcommand{\isLVIIvXV}{\VERSE  Quia hæc dicit Excelsus, et Sublimis, habitans æternitatem, et sanctum nomen ejus : in excelso et in sancto habitans, et cum contrito et humili spiritu : ut vivificet spiritum humilium, et vivificet cor contritorum. \EVERSE}
\newcommand{\isLVIIvXVI}{\VERSE  Non enim in sempiternum litigabo, neque usque ad finem irascar, quia spiritus a facie mea egredietur, et flatus ego faciam. \EVERSE}
\newcommand{\isLVIIvXVII}{\VERSE  Propter iniquitatem avaritiæ ejus iratus sum, et percussi eum. Abscondi a te faciem meam, et indignatus sum ; et abiit vagus in via cordis sui. \EVERSE}
\newcommand{\isLVIIvXVIII}{\VERSE  Vias ejus vidi, et sanavi eum ; et reduxi eum, et reddidi consolationes ipsi, et lugentibus ejus. \EVERSE}
\newcommand{\isLVIIvXIX}{\VERSE  Creavi fructum labiorum pacem ; pacem ei qui longe est et qui prope, dixit Dominus, et sanavi eum. \EVERSE}
\newcommand{\isLVIIvXX}{\VERSE  Impii autem quasi mare fervens, quod quiescere non potest, et redundant fluctus ejus in conculcationem et lutum. \EVERSE}
\newcommand{\isLVIIvXXI}{\VERSE  Non est pax impiis, dicit Dominus Deus. \EVERSE}
\newcommand{\isLVIIIvI}{\VERSE  Clama, ne cesses, quasi tuba exalta vocem tuam, et annuntia populo meo scelera eorum, et domui Jacob peccata eorum. \EVERSE}
\newcommand{\isLVIIIvII}{\VERSE  Me etenim de die in diem quærunt, et scire vias meas volunt, quasi gens quæ justitiam fecerit, et judicium Dei sui non dereliquerit. Rogant me judicia justitiæ ; appropinquare Deo volunt. \EVERSE}
\newcommand{\isLVIIIvIII}{\VERSE  Quare jejunavimus, et non aspexisti ; humiliavimus animas nostras, et nescisti ? Ecce in die jejunii vestri invenitur voluntas vestra, et omnes debitores vestros repetitis. \EVERSE}
\newcommand{\isLVIIIvIV}{\VERSE  Ecce ad lites et contentiones jejunatis, et percutitis pugno impie. Nolite jejunare sicut usque ad hanc diem, ut audiatur in excelso clamor vester. \EVERSE}
\newcommand{\isLVIIIvV}{\VERSE  Numquid tale est jejunium quod elegi, per diem affligere hominem animam suam ? numquid contorquere quasi circulum caput suum, et saccum et cinerem sternere ? numquid istud vocabis jejunium, et diem acceptabilem Domino ? \EVERSE}
\newcommand{\isLVIIIvVI}{\VERSE  Nonne hoc est magis jejunium quod elegi ? Dissolve colligationes impietatis, solve fasciculos deprimentes, dimitte eos qui confracti sunt liberos, et omne onus dirumpe ; \EVERSE}
\newcommand{\isLVIIIvVII}{\VERSE  frange esurienti panem tuum, et egenos vagosque induc in domum tuam ; cum videris nudum, operi eum, et carnem tuam ne despexeris. \EVERSE}
\newcommand{\isLVIIIvVIII}{\VERSE  Tunc erumpet quasi mane lumen tuum ; et sanitas tua citius orietur, et anteibit faciem tuam justitia tua, et gloria Domini colliget te. \EVERSE}
\newcommand{\isLVIIIvIX}{\VERSE  Tunc invocabis, et Dominus exaudiet ; clamabis, et dicet : Ecce adsum. Si abstuleris de medio tui catenam, et desieris extendere digitum et loqui quod non prodest ; \EVERSE}
\newcommand{\isLVIIIvX}{\VERSE  cum effuderis esurienti animam tuam, et animam afflictam repleveris, orietur in tenebris lux tua, et tenebræ tuæ erunt sicut meridies. \EVERSE}
\newcommand{\isLVIIIvXI}{\VERSE  Et requiem tibi dabit Dominus semper, et implebit splendoribus animam tuam, et ossa tua liberabit ; et eris quasi hortus irriguus, et sicut fons aquarum cujus non deficient aquæ. \EVERSE}
\newcommand{\isLVIIIvXII}{\VERSE  Et ædificabuntur in te deserta sæculorum, fundamenta generationis et generationis suscitabis ; et vocaberis ædificator sepium, avertens semitas in quietem. \EVERSE}
\newcommand{\isLVIIIvXIII}{\VERSE  Si averteris a sabbato pedem tuum facere voluntatem tuam in die sancto meo, et vocaveris sabbatum delicatum, et sanctum Domini gloriosum, et glorificaveris eum dum non facis vias tuas, et non invenitur voluntas tua, ut loquaris sermonem : \EVERSE}
\newcommand{\isLVIIIvXIV}{\VERSE  tunc delectaberis super Domino, et sustollam te super altitudines terræ, et cibabo te hæreditate Jacob patris tui : os enim Domini locutum est. \EVERSE}
\newcommand{\isLIXvI}{\VERSE  Ecce non est abbreviata manus Domini, ut salvare nequeat, neque aggravata est auris ejus, ut non exaudiat. \EVERSE}
\newcommand{\isLIXvII}{\VERSE  Sed iniquitates vestræ diviserunt inter vos et Deum vestrum ; et peccata vestra absconderunt faciem ejus a vobis, ne exaudiret. \EVERSE}
\newcommand{\isLIXvIII}{\VERSE  Manus enim vestræ pollutæ sunt sanguine, et digiti vestri iniquitate ; labia vestra locuta sunt mendacium, et lingua vestra iniquitatem fatur. \EVERSE}
\newcommand{\isLIXvIV}{\VERSE  Non est qui invocet justitiam, neque est qui judicet vere : sed confidunt in nihilo, et loquuntur vanitates ; conceperunt laborem, et pepererunt iniquitatem. \EVERSE}
\newcommand{\isLIXvV}{\VERSE  Ova aspidum ruperunt, et telas araneæ texuerunt. Qui comederit de ovis eorum, morietur ; et quod confotum est, erumpet in regulum. \EVERSE}
\newcommand{\isLIXvVI}{\VERSE  Telæ eorum non erunt in vestimentum, neque operientur operibus suis ; opera eorum opera inutilia, et opus iniquitatis in manibus eorum. \EVERSE}
\newcommand{\isLIXvVII}{\VERSE  Pedes eorum ad malum currunt, et festinant ut effundant sanguinem innocentem ; cogitationes eorum cogitationes inutiles : vastitas et contritio in viis eorum. \EVERSE}
\newcommand{\isLIXvVIII}{\VERSE  Viam pacis nescierunt, et non est judicium in gressibus eorum ; semitæ eorum incurvatæ sunt eis : omnis qui calcat in eis, ignorat pacem. \EVERSE}
\newcommand{\isLIXvIX}{\VERSE  Propter hoc elongatum est judicium a nobis, et non apprehendet nos justitia. Exspectavimus lucem, et ecce tenebræ ; splendorem, et in tenebris ambulavimus. \EVERSE}
\newcommand{\isLIXvX}{\VERSE  Palpavimus sicut cæci parietem, et quasi absque oculis attrectavimus : impegimus meridie quasi in tenebris ; in caliginosis quasi mortui. \EVERSE}
\newcommand{\isLIXvXI}{\VERSE  Rugiemus quasi ursi omnes, et quasi columbæ meditantes gememus : exspectavimus judicium, et non est ; salutem, et elongata est a nobis. \EVERSE}
\newcommand{\isLIXvXII}{\VERSE  Multiplicatæ sunt enim iniquitates nostræ coram te, et peccata nostra responderunt nobis, quia scelera nostra nobiscum et iniquitates nostras cognovimus. \EVERSE}
\newcommand{\isLIXvXIII}{\VERSE  Peccare et mentiri contra Dominum, et aversi sumus ne iremus post tergum Dei nostri, ut loqueremur calumniam et transgressionem ; concepimus et locuti sumus de corde verba mendacii. \EVERSE}
\newcommand{\isLIXvXIV}{\VERSE  Et conversum est retrorsum judicium, et justitia longe stetit, quia corruit in platea veritas, et æquitas non potuit ingredi. \EVERSE}
\newcommand{\isLIXvXV}{\VERSE  Et facta est veritas in oblivionem, et qui recessit a malo, prædæ patuit. Et vidit Dominus, et malum apparuit in oculis ejus, quia non est judicium. \EVERSE}
\newcommand{\isLIXvXVI}{\VERSE  Et vidit quia non est vir, et aporiatus est, quia non est qui occurrat ; et salvavit sibi brachium suum, et justitia ejus ipsa confirmavit eum. \EVERSE}
\newcommand{\isLIXvXVII}{\VERSE  Indutus est justitia ut lorica, et galea salutis in capite ejus ; indutus est vestimentis ultionis, et opertus est quasi pallio zeli : \EVERSE}
\newcommand{\isLIXvXVIII}{\VERSE  sicut ad vindictam quasi ad retributionem indignationis hostibus suis, et vicissitudinem inimicis suis ; insulis vicem reddet. \EVERSE}
\newcommand{\isLIXvXIX}{\VERSE  Et timebunt qui ab occidente nomen Domini, et qui ab ortu solis gloriam ejus, cum venerit quasi fluvius violentus quem spiritus Domini cogit ; \EVERSE}
\newcommand{\isLIXvXX}{\VERSE  et venerit Sion redemptor, et eis qui redeunt ab iniquitate in Jacob, dicit Dominus. \EVERSE}
\newcommand{\isLIXvXXI}{\VERSE  Hoc fœdus meum cum eis, dicit Dominus : spiritus meus qui est in te, et verba mea quæ posui in ore tuo, non recedent de ore tuo, et de ore seminis tui, et de ore seminis seminis tui, dicit Dominus, amodo et usque in sempiternum. \EVERSE}
\newcommand{\isLXvI}{\VERSE  Surge, illuminare, Jerusalem, quia venit lumen tuum, et gloria Domini super te orta est. \EVERSE}
\newcommand{\isLXvII}{\VERSE  Quia ecce tenebræ operient terram, et caligo populos ; super te autem orietur Dominus, et gloria ejus in te videbitur. \EVERSE}
\newcommand{\isLXvIII}{\VERSE  Et ambulabunt gentes in lumine tuo, et reges in splendore ortus tui. \EVERSE}
\newcommand{\isLXvIV}{\VERSE  Leva in circuitu oculos tuos, et vide : omnes isti congregati sunt, venerunt tibi ; filii tui de longe venient et filiæ tuæ de latere surgent. \EVERSE}
\newcommand{\isLXvV}{\VERSE  Tunc videbis, et afflues ; mirabitur et dilatabitur cor tuum : quando conversa fuerit ad te multitudo maris ; fortitudo gentium venerit tibi. \EVERSE}
\newcommand{\isLXvVI}{\VERSE  Inundatio camelorum operiet te, dromedarii Madian et Epha ; omnes de Saba venient, aurum et thus deferentes, et laudem Domino annuntiantes. \EVERSE}
\newcommand{\isLXvVII}{\VERSE  Omne pecus Cedar congregabitur tibi ; arietes Nabaioth ministrabunt tibi : offerentur super placabili altari meo, et domum majestatis meæ glorificabo. \EVERSE}
\newcommand{\isLXvVIII}{\VERSE  Qui sunt isti qui ut nubes volant, et quasi columbæ ad fenestras suas ? \EVERSE}
\newcommand{\isLXvIX}{\VERSE  Me enim insulæ exspectant, et naves maris in principio, ut adducam filios tuos de longe ; argentum eorum, et aurum eorum cum eis, nomini Domini Dei tui, et Sancto Israël, quia glorificavit te. \EVERSE}
\newcommand{\isLXvX}{\VERSE  Et ædificabunt filii peregrinorum muros tuos, et reges eorum ministrabunt tibi ; in indignatione enim mea percussi te, et in reconciliatione mea misertus sum tui. \EVERSE}
\newcommand{\isLXvXI}{\VERSE  Et aperientur portæ tuæ jugiter ; die ac nocte non claudentur, ut afferatur ad te fortitudo gentium, et reges earum adducantur. \EVERSE}
\newcommand{\isLXvXII}{\VERSE  Gens enim et regnum quod non servierit tibi peribit, et gentes solitudine vastabuntur. \EVERSE}
\newcommand{\isLXvXIII}{\VERSE  Gloria Libani ad te veniet, abies, et buxus, et pinus simul ad ornandum locum sanctificationis meæ ; et locum pedum meorum glorificabo. \EVERSE}
\newcommand{\isLXvXIV}{\VERSE  Et venient ad te curvi filii eorum qui humiliaverunt te, et adorabunt vestigia pedum tuorum omnes qui detrahebant tibi : et vocabunt te civitatem Domini, Sion Sancti Israël. \EVERSE}
\newcommand{\isLXvXV}{\VERSE  Pro eo quod fuisti derelicta et odio habita, et non erat qui per te transiret : ponam te in superbiam sæculorum, gaudium in generationem et generationem : \EVERSE}
\newcommand{\isLXvXVI}{\VERSE  et suges lac gentium, et mamilla regum lactaberis ; et scies quia ego Dominus salvans te, et redemptor tuus, Fortis Jacob. \EVERSE}
\newcommand{\isLXvXVII}{\VERSE  Pro ære afferam aurum, et pro ferro afferam argentum, et pro lignis æs, et pro lapidibus ferrum : et ponam visitationem tuam pacem, et præpositos tuos justitiam. \EVERSE}
\newcommand{\isLXvXVIII}{\VERSE  Non audietur ultra iniquitas in terra tua ; vastitas et contritio in terminis tuis : et occupabit salus muros tuos, et portas tuas laudatio. \EVERSE}
\newcommand{\isLXvXIX}{\VERSE  Non erit tibi amplius sol ad lucendum per diem, nec splendor lunæ illuminabit te : sed erit tibi Dominus in lucem sempiternam, et Deus tuus in gloriam tuam. \EVERSE}
\newcommand{\isLXvXX}{\VERSE  Non occidet ultra sol tuus, et luna tua non minuetur, quia erit tibi Dominus in lucem sempiternam, et complebuntur dies luctus tui. \EVERSE}
\newcommand{\isLXvXXI}{\VERSE  Populus autem tuus omnes justi ; in perpetuum hæreditabunt terram : germen plantationis meæ, opus manus meæ ad glorificandum. \EVERSE}
\newcommand{\isLXvXXII}{\VERSE  Minimus erit in mille, et parvulus in gentem fortissimam. Ego Dominus in tempore ejus subito faciam istud. \EVERSE}
\newcommand{\isLXIvI}{\VERSE  Spiritus Domini super me, eo quod unxerit Dominus me ; ad annuntiandum mansuetis misit me, ut mederer contritis corde, et prædicarem captivis indulgentiam, et clausis apertionem ; \EVERSE}
\newcommand{\isLXIvII}{\VERSE  ut prædicarem annum placabilem Domino, et diem ultionis Deo nostro ; ut consolarer omnes lugentes, \EVERSE}
\newcommand{\isLXIvIII}{\VERSE  ut ponerem lugentibus Sion, et darem eis coronam pro cinere, oleum gaudii pro luctu, pallium laudis pro spiritu mœroris ; et vocabuntur in ea fortes justitiæ, plantatio Domini ad glorificandum. \EVERSE}
\newcommand{\isLXIvIV}{\VERSE  Et ædificabunt deserta a sæculo, et ruinas antiquas erigent, et instaurabunt civitates desertas, dissipatas in generationem et generationem. \EVERSE}
\newcommand{\isLXIvV}{\VERSE  Et stabunt alieni, et pascent pecora vestra, et filii peregrinorum agricolæ et vinitores vestri erunt. \EVERSE}
\newcommand{\isLXIvVI}{\VERSE  Vos autem sacerdotes Domini vocabimini : Ministri Dei nostri, dicetur vobis, fortitudinem gentium comedetis, et in gloria earum superbietis. \EVERSE}
\newcommand{\isLXIvVII}{\VERSE  Pro confusione vestra duplici et rubore, laudabunt partem suam ; propter hoc in terra sua duplicia possidebunt, lætitia sempiterna erit eis. \EVERSE}
\newcommand{\isLXIvVIII}{\VERSE  Quia ego Dominus diligens judicium, et odio habens rapinam in holocausto ; et dabo opus eorum in veritate, et fœdus perpetuum feriam eis. \EVERSE}
\newcommand{\isLXIvIX}{\VERSE  Et scient in gentibus semen eorum, et germen eorum in medio populorum ; omnes qui viderint eos cognoscent illos, quia isti sunt semen cui benedixit Dominus. \EVERSE}
\newcommand{\isLXIvX}{\VERSE  Gaudens gaudebo in Domino, et exsultabit anima mea in Deo meo, quia induit me vestimentis salutis, et indumento justitiæ circumdedit me, quasi sponsum decoratum corona, et quasi sponsam ornatam monilibus suis. \EVERSE}
\newcommand{\isLXIvXI}{\VERSE  Sicut enim terra profert germen suum, et sicut hortus semen suum germinat, sic Dominus Deus germinabit justitiam et laudem coram universis gentibus. \EVERSE}
\newcommand{\isLXIIvI}{\VERSE  Propter Sion non tacebo, et propter Jerusalem non quiescam, donec egrediatur ut splendor justus ejus, et salvator ejus ut lampas accendatur. \EVERSE}
\newcommand{\isLXIIvII}{\VERSE  Et videbunt gentes justum tuum, et cuncti reges inclytum tuum ; et vocabitur tibi nomen novum, quod os Domini nominabit. \EVERSE}
\newcommand{\isLXIIvIII}{\VERSE  Et eris corona gloriæ in manu Domini, et diadema regni in manu Dei tui. \EVERSE}
\newcommand{\isLXIIvIV}{\VERSE  Non vocaberis ultra Derelicta, et terra tua non vocabitur amplius Desolata ; sed vocaberis, Voluntas mea in ea, et terra tua Inhabitata, quia complacuit Domino in te, et terra tua inhabitabitur. \EVERSE}
\newcommand{\isLXIIvV}{\VERSE  Habitabit enim juvenis cum virgine, et habitabunt in te filii tui ; et gaudebit sponsus super sponsam, et gaudebit super te Deus tuus. \EVERSE}
\newcommand{\isLXIIvVI}{\VERSE  Super muros tuos, Jerusalem, constitui custodes ; tota die et tota nocte in perpetuum non tacebunt. Qui reminiscimini Domini, ne taceatis, \EVERSE}
\newcommand{\isLXIIvVII}{\VERSE  et ne detis silentium ei, donec stabiliat et donec ponat Jerusalem laudem in terra. \EVERSE}
\newcommand{\isLXIIvVIII}{\VERSE  Juravit Dominus in dextera sua, et in brachio fortitudinis suæ : Si dedero triticum tuum ultra cibum inimicis tuis ; et si biberint filii alieni vinum tuum in quo laborasti. \EVERSE}
\newcommand{\isLXIIvIX}{\VERSE  Quia qui congregant illud, comedent, et laudabunt Dominum ; et qui comportant illud, bibent in atriis sanctis meis. \EVERSE}
\newcommand{\isLXIIvX}{\VERSE  Transite, transite per portas, præparate viam populo : planum facite iter, eligite lapides, et elevate signum ad populos. \EVERSE}
\newcommand{\isLXIIvXI}{\VERSE  Ecce Dominus auditum fecit in extremis terræ : Dicite filiæ Sion : Ecce Salvator tuus venit ; ecce merces ejus cum eo, et opus ejus coram illo. \EVERSE}
\newcommand{\isLXIIvXII}{\VERSE  Et vocabunt eos, Populus sanctus, redempti a Domino ; tu autem vocaberis, Quæsita civitas, et non Derelicta. \EVERSE}
\newcommand{\isLXIIIvI}{\VERSE  Quis est iste, qui venit de Edom, tinctis vestibus de Bosra ? iste formosus in stola sua, gradiens in multitudine fortitudinis suæ ? Ego qui loquor justitiam, et propugnator sum ad salvandum. \EVERSE}
\newcommand{\isLXIIIvII}{\VERSE  Quare ergo rubrum est indumentum tuum, et vestimenta tua sicut calcantium in torculari ? \EVERSE}
\newcommand{\isLXIIIvIII}{\VERSE  Torcular calcavi solus, et de gentibus non est vir mecum ; calcavi eos in furore meo, et conculcavi eos in ira mea : et aspersus est sanguis eorum super vestimenta mea, et omnia indumenta mea inquinavi. \EVERSE}
\newcommand{\isLXIIIvIV}{\VERSE  Dies enim ultionis in corde meo ; annus redemptionis meæ venit. \EVERSE}
\newcommand{\isLXIIIvV}{\VERSE  Circumspexi, et non erat auxiliator ; quæsivi, et non fuit qui adjuvaret : et salvavit mihi brachium meum, et indignatio mea ipsa auxiliata est mihi. \EVERSE}
\newcommand{\isLXIIIvVI}{\VERSE  Et conculcavi populos in furore meo, et inebriavi eos in indignatione mea, et detraxi in terram virtutem eorum. \EVERSE}
\newcommand{\isLXIIIvVII}{\VERSE  Miserationum Domini recordabor ; laudem Domini super omnibus quæ reddidit nobis Dominus, et super multitudinem bonorum domui Israël, quæ largitus est eis secundum indulgentiam suam, et secundum multitudinem misericordiarum suarum. \EVERSE}
\newcommand{\isLXIIIvVIII}{\VERSE  Et dixit : Verumtamen populus meus est, filii non negantes ; et factus est eis salvator. \EVERSE}
\newcommand{\isLXIIIvIX}{\VERSE  In omni tribulatione eorum non est tribulatus, et angelus faciei ejus salvavit eos : in dilectione sua et in indulgentia sua ipse redemit eos, et portavit eos, et elevavit eos cunctis diebus sæculi. \EVERSE}
\newcommand{\isLXIIIvX}{\VERSE  Ipsi autem ad iracundiam provocaverunt, et afflixerunt spiritum Sancti ejus : et conversus est eis in inimicum, et ipse debellavit eos. \EVERSE}
\newcommand{\isLXIIIvXI}{\VERSE  Et recordatus est dierum sæculi Moysi, et populi sui. Ubi est qui eduxit eos de mari cum pastoribus gregis sui ? Ubi est qui posuit in medio ejus spiritum Sancti sui ; \EVERSE}
\newcommand{\isLXIIIvXII}{\VERSE  qui eduxit ad dexteram Moysen, brachio majestatis suæ ; qui scidit aquas ante eos, ut faceret sibi nomen sempiternum ; \EVERSE}
\newcommand{\isLXIIIvXIII}{\VERSE  qui eduxit eos per abyssos, quasi equum in deserto non impingentem ? \EVERSE}
\newcommand{\isLXIIIvXIV}{\VERSE  Quasi animal in campo descendens, spiritus Domini ductor ejus fuit. Sic adduxisti populum tuum, ut faceres tibi nomen gloriæ. \EVERSE}
\newcommand{\isLXIIIvXV}{\VERSE  Attende de cælo, et vide de habitaculo sancto tuo, et gloriæ tuæ. Ubi est zelus tuus, et fortitudo tua, multitudo viscerum tuorum et miserationum tuarum ? Super me continuerunt se. \EVERSE}
\newcommand{\isLXIIIvXVI}{\VERSE  Tu enim pater noster : et Abraham nescivit nos, et Israël ignoravit nos : tu, Domine, pater noster, redemptor noster, a sæculo nomen tuum. \EVERSE}
\newcommand{\isLXIIIvXVII}{\VERSE  Quare errare nos fecisti, Domine, de viis tuis ; indurasti cor nostrum ne timeremus te ? Convertere propter servos tuos, tribus hæreditatis tuæ. \EVERSE}
\newcommand{\isLXIIIvXVIII}{\VERSE  Quasi nihilum possederunt populum sanctum tuum : hostes nostri conculcaverunt sanctificationem tuam. \EVERSE}
\newcommand{\isLXIIIvXIX}{\VERSE  Facti sumus quasi in principio, cum non dominareris nostri, neque invocaretur nomen tuum super nos. \EVERSE}
\newcommand{\isLXIVvI}{\VERSE  Utinam dirumperes cælos, et descenderes ; a facie tua montes defluerent ; \EVERSE}
\newcommand{\isLXIVvII}{\VERSE  sicut exustio ignis tabescerent, aquæ arderent igni : ut notum fieret nomen tuum inimicis tuis ; a facie tua gentes turbarentur. \EVERSE}
\newcommand{\isLXIVvIII}{\VERSE  Cum feceris mirabilia, non sustinebimus ; descendisti, et a facie tua montes defluxerunt. \EVERSE}
\newcommand{\isLXIVvIV}{\VERSE  A sæculo non audierunt, neque auribus perceperunt ; oculus non vidit, Deus, absque te, quæ præparasti exspectantibus te. \EVERSE}
\newcommand{\isLXIVvV}{\VERSE  Occurristi lætanti, et facienti justitiam ; in viis tuis recordabuntur tui. Ecce tu iratus es, et peccavimus ; in ipsis fuimus semper, et salvabimur. \EVERSE}
\newcommand{\isLXIVvVI}{\VERSE  Et facti sumus ut immundus omnes nos, et quasi pannus menstruatæ universæ justitiæ nostræ ; et cecidimus quasi folium universi, et iniquitates nostræ quasi ventus abstulerunt nos. \EVERSE}
\newcommand{\isLXIVvVII}{\VERSE  Non est qui invocet nomen tuum ; qui consurgat, et teneat te. Abscondisti faciem tuam a nobis, et allisisti nos in manu iniquitatis nostræ. \EVERSE}
\newcommand{\isLXIVvVIII}{\VERSE  Et nunc, Domine, pater noster es tu, nos vero lutum ; et fictor noster tu, et opera manuum tuarum omnes nos. \EVERSE}
\newcommand{\isLXIVvIX}{\VERSE  Ne irascaris, Domine, satis, et ne ultra memineris iniquitatis nostræ ; ecce, respice, populus tuus omnes nos. \EVERSE}
\newcommand{\isLXIVvX}{\VERSE  Civitas Sancti tui facta est deserta, Sion deserta facta est, Jerusalem desolata est. \EVERSE}
\newcommand{\isLXIVvXI}{\VERSE  Domus sanctificationis nostræ et gloriæ nostræ, ubi laudaverunt te patres nostri, facta est in exustionem ignis, et omnia desiderabilia nostra versa sunt in ruinas. \EVERSE}
\newcommand{\isLXIVvXII}{\VERSE  Numquid super his continebis te, Domine ; tacebis, et affliges nos vehementer ? \EVERSE}
\newcommand{\isLXVvI}{\VERSE  Quæsierunt me qui ante non interrogabant ; invenerunt qui non quæsierunt me. Dixi : Ecce ego, ecce ego, ad gentem quæ non invocabat nomen meum. \EVERSE}
\newcommand{\isLXVvII}{\VERSE  Expandi manus meas tota die ad populum incredulum, qui graditur in via non bona post cogitationes suas. \EVERSE}
\newcommand{\isLXVvIII}{\VERSE  Populus qui ad iracundiam provocat me ante faciem meam semper ; qui immolant in hortis, et sacrificant super lateres ; \EVERSE}
\newcommand{\isLXVvIV}{\VERSE  qui habitant in sepulchris, et in delubris idolorum dormiunt ; qui comedunt carnem suillam, et jus profanum in vasis eorum ; \EVERSE}
\newcommand{\isLXVvV}{\VERSE  qui dicunt : Recede a me, non appropinques mihi, quia immundus es. Isti fumus erunt in furore meo, ignis ardens tota die. \EVERSE}
\newcommand{\isLXVvVI}{\VERSE  Ecce scriptum est coram me : Non tacebo, sed reddam, et retribuam in sinum eorum. \EVERSE}
\newcommand{\isLXVvVII}{\VERSE  Iniquitates vestras, et iniquitates patrum vestrorum simul, dicit Dominus ; qui sacrificaverunt super montes, et super colles exprobraverunt mihi ; et remetiar opus eorum primum in sinu eorum. \EVERSE}
\newcommand{\isLXVvVIII}{\VERSE  Hæc dicit Dominus : Quomodo si inveniatur granum in botro, et dicatur : Ne dissipes illud, quoniam benedictio est : sic faciam propter servos meos, ut non disperdam totum. \EVERSE}
\newcommand{\isLXVvIX}{\VERSE  Et educam de Jacob semen, et de Juda possidentem montes meos ; et hæreditabunt eam electi mei, et servi mei habitabunt ibi. \EVERSE}
\newcommand{\isLXVvX}{\VERSE  Et erunt campestria in caulas gregum, et vallis Achor in cubile armentorum, populo meo qui requisierunt me. \EVERSE}
\newcommand{\isLXVvXI}{\VERSE  Et vos qui dereliquistis Dominum, qui obliti estis montem sanctum meum, qui ponitis fortunæ mensam, et libatis super eam : \EVERSE}
\newcommand{\isLXVvXII}{\VERSE  numerabo vos in gladio, et omnes in cæde corruetis : pro eo quod vocavi, et non respondistis ; locutus sum, et non audistis ; et faciebatis malum in oculis meis, et quæ nolui elegistis. \EVERSE}
\newcommand{\isLXVvXIII}{\VERSE  Propter hoc hæc dicit Dominus Deus : Ecce servi mei comedent, et vos esurietis ; ecce servi mei bibent, et vos sitietis ; \EVERSE}
\newcommand{\isLXVvXIV}{\VERSE  ecce servi mei lætabuntur, et vos confundemini ; ecce servi mei laudabunt præ exsultatione cordis, et vos clamabitis præ dolore cordis, et præ contritione spiritus ululabitis, \EVERSE}
\newcommand{\isLXVvXV}{\VERSE  et dimittetis nomen vestrum in juramentum electis meis ; et interficiet te Dominus Deus, et servos suos vocabit nomine alio : \EVERSE}
\newcommand{\isLXVvXVI}{\VERSE  in quo qui benedictus est super terram benedicetur in Deo, amen, et qui jurat in terra jurabit in Deo, amen : quia oblivioni traditæ sunt angustiæ priores, et quia absconditæ sunt ab oculis meis. \EVERSE}
\newcommand{\isLXVvXVII}{\VERSE  Ecce enim ego creo cælos novos, et terram novam ; et non erunt in memoria priora, et non ascendent super cor. \EVERSE}
\newcommand{\isLXVvXVIII}{\VERSE  Sed gaudebitis et exsultabitis usque in sempiternum in his quæ ego creo : quia ecce ego creo Jerusalem exsultationem, et populum ejus gaudium. \EVERSE}
\newcommand{\isLXVvXIX}{\VERSE  Et exsultabo in Jerusalem, et gaudebo in populo meo, et non audietur in eo ultra vox fletus et vox clamoris. \EVERSE}
\newcommand{\isLXVvXX}{\VERSE  Non erit ibi amplius infans dierum, et senex qui non impleat dies suos, quoniam puer centum annorum morietur, et peccator centum annorum maledictus erit. \EVERSE}
\newcommand{\isLXVvXXI}{\VERSE  Et ædificabunt domos, et habitabunt ; et plantabunt vineas, et comedent fructus earum. \EVERSE}
\newcommand{\isLXVvXXII}{\VERSE  Non ædificabunt, et alius habitabit ; non plantabunt, et alius comedet : secundum enim dies ligni erunt dies populi mei, et opera manuum eorum inveterabunt. \EVERSE}
\newcommand{\isLXVvXXIII}{\VERSE  Electi mei non laborabunt frustra, neque generabunt in conturbatione, quia semen benedictorum Domini est, et nepotes eorum cum eis. \EVERSE}
\newcommand{\isLXVvXXIV}{\VERSE  Eritque antequam clament, ego exaudiam ; adhuc illis loquentibus, ego audiam. \EVERSE}
\newcommand{\isLXVvXXV}{\VERSE  Lupus et agnus pascentur simul, leo et bos comedent paleas, et serpenti pulvis panis ejus. Non nocebunt, neque occident in omni monte sancto meo, dicit Dominus. \EVERSE}
\newcommand{\isLXVIvI}{\VERSE  Hæc dicit Dominus : Cælum sedes mea, terra autem scabellum pedum meorum. Quæ est ista domus quam ædificabitis mihi ? et quis est iste locus quietis meæ ? \EVERSE}
\newcommand{\isLXVIvII}{\VERSE  Omnia hæc manus mea fecit, et facta sunt universa ista, dicit Dominus ; ad quem autem respiciam, nisi ad pauperculum, et contritum spiritu, et trementem sermones meos ? \EVERSE}
\newcommand{\isLXVIvIII}{\VERSE  Qui immolat bovem, quasi qui interficiat virum ; qui mactat pecus, quasi qui excerebret canem ; qui offert oblationem, quasi qui sanguinem suillum offerat ; qui recordatur thuris, quasi qui benedicat idolo. Hæc omnia elegerunt in viis suis, et in abominationibus suis anima eorum delectata est. \EVERSE}
\newcommand{\isLXVIvIV}{\VERSE  Unde et ego eligam illusiones eorum, et quæ timebant adducam eis ; quia vocavi, et non erat qui responderet ; locutus sum, et non audierunt ; feceruntque malum in oculis meis, et quæ nolui elegerunt. \EVERSE}
\newcommand{\isLXVIvV}{\VERSE  Audite verbum Domini, qui tremitis ad verbum ejus. Dixerunt fratres vestri odientes vos, et abjicientes propter nomen meum : Glorificetur Dominus, et videbimus in lætitia vestra ; ipsi autem confundentur. \EVERSE}
\newcommand{\isLXVIvVI}{\VERSE  Vox populi de civitate, vox de templo, vox Domini reddentis retributionem inimicis suis. \EVERSE}
\newcommand{\isLXVIvVII}{\VERSE  Antequam parturiret, peperit ; antequam veniret partus ejus, peperit masculum. \EVERSE}
\newcommand{\isLXVIvVIII}{\VERSE  Quis audivit umquam tale ? et quis vidit huic simile ? numquid parturiet terra in die una, aut parietur gens simul, quia parturivit et peperit Sion filios suos ? \EVERSE}
\newcommand{\isLXVIvIX}{\VERSE  Numquid ego qui alios parere facio, ipse non pariam ? dicit Dominus. Si ego, qui generationem ceteris tribuo, sterilis ero ? ait Dominus Deus tuus. \EVERSE}
\newcommand{\isLXVIvX}{\VERSE  Lætamini cum Jerusalem et exsultate in ea, omnes qui diligitis eam ; gaudete cum ea gaudio, universi qui lugetis super eam : \EVERSE}
\newcommand{\isLXVIvXI}{\VERSE  ut sugatis et repleamini ab ubere consolationis ejus ; ut mulgeatis et deliciis affluatis ab omnimoda gloria ejus. \EVERSE}
\newcommand{\isLXVIvXII}{\VERSE  Quia hæc dicit Dominus : Ecce ego declinabo super eam quasi fluvium pacis, et quasi torrentem inundantem gloriam gentium, quam sugetis : ad ubera portabimini, et super genua blandientur vobis. \EVERSE}
\newcommand{\isLXVIvXIII}{\VERSE  Quomodo si cui mater blandiatur, ita ego consolabor vos, et in Jerusalem consolabimini. \EVERSE}
\newcommand{\isLXVIvXIV}{\VERSE  Videbitis, et gaudebit cor vestrum, et ossa vestra quasi herba germinabunt : et cognoscetur manus Domini servis ejus, et indignabitur inimicis suis. \EVERSE}
\newcommand{\isLXVIvXV}{\VERSE  Quia ecce Dominus in igne veniet, et quasi turbo quadrigæ ejus, reddere in indignatione furorem suum et increpationem suam in flamma ignis : \EVERSE}
\newcommand{\isLXVIvXVI}{\VERSE  quia in igne Dominus dijudicabit, et in gladio suo ad omnem carnem ; et multiplicabuntur interfecti a Domino, \EVERSE}
\newcommand{\isLXVIvXVII}{\VERSE  qui sanctificabantur et mundos se putabant in hortis post januam intrinsecus, qui comedebant carnem suillam, et abominationem et murem : simul consumentur, dicit Dominus. \EVERSE}
\newcommand{\isLXVIvXVIII}{\VERSE  Ego autem opera eorum et cogitationes eorum venio ut congregem, cum omnibus gentibus et linguis : et venient, et videbunt gloriam meam. \EVERSE}
\newcommand{\isLXVIvXIX}{\VERSE  Et ponam in eis signum, et mittam ex eis qui salvati fuerint, ad gentes in mare, in Africam, et Lydiam, tendentes sagittam ; in Italiam et Græciam, ad insulas longe, ad eos qui non audierunt de me, et non viderunt gloriam meam. Et annuntiabunt gloriam meam gentibus ; \EVERSE}
\newcommand{\isLXVIvXX}{\VERSE  et adducent omnes fratres vestros de cunctis gentibus donum Domino, in equis, et in quadrigis, et in lecticis, et in mulis, et in carrucis, ad montem sanctum meum Jerusalem, dicit Dominus : quomodo si inferant filii Israël munus in vase mundo in domum Domini. \EVERSE}
\newcommand{\isLXVIvXXI}{\VERSE  Et assumam ex eis in sacerdotes et Levitas, dicit Dominus. \EVERSE}
\newcommand{\isLXVIvXXII}{\VERSE  Quia sicut cæli novi et terra nova, quæ ego facio stare coram me, dicit Dominus, sic stabit semen vestrum et nomen vestrum. \EVERSE}
\newcommand{\isLXVIvXXIII}{\VERSE  Et erit mensis ex mense, et sabbatum ex sabbato : veniet omnis caro ut adoret coram facie mea, dicit Dominus. \EVERSE}
\newcommand{\isLXVIvXXIV}{\VERSE  Et egredientur, et videbunt cadavera virorum qui prævaricati sunt in me ; vermis eorum non morietur, et ignis eorum non extinguetur : et erunt usque ad satietatem visionis omni carni. \EVERSE}

\newcommand{\isIvIfr}{\VERSE  Vision d'Isaïe, fils d'Amos, qu'il a vue sur Juda et Jérusalem, aux jours d'Ozias, de Joathan, d'Achaz et d'Ezéchias, rois de Juda. \EVERSE}
\newcommand{\isIvIIfr}{\VERSE  Cieux, écoutez, et terre, prête l'oreille, car le Seigneur a parlé. J'ai nourri des enfants, et Je les ai élevés; mais ils M'ont méprisé. \EVERSE}
\newcommand{\isIvIIIfr}{\VERSE  Le boeuf connaît son possesseur, et l'âne l'étable de son maître; mais Israël ne M'a point connu, et Mon peuple n'a pas eu d'intelligence. \EVERSE}
\newcommand{\isIvIVfr}{\VERSE  Malheur à la nation pécheresse, au peuple chargé d'iniquité, à la race corrompue, aux enfants scélérats. Ils ont abandonné le Seigneur, ils ont blasphémé le Saint d'Israël, ils se sont tournés en arrière. \EVERSE}
\newcommand{\isIvVfr}{\VERSE  Où vous frapperai-Je encore, vous qui multipliez les prévarications? Toute tête est languissante, et tout coeur est abattu. \EVERSE}
\newcommand{\isIvVIfr}{\VERSE  Depuis la plante des pieds jusqu'au sommet de la tête il n'y a rien de sain en lui; ce n'est que blessure, et contusion, et plaie enflammée, qui n'a pas été bandée, à qui l'on n'a pas appliqué de remède, et qu'on n'a point adoucie avec l'huile. \EVERSE}
\newcommand{\isIvVIIfr}{\VERSE  Votre terre est déserte, vos villes sont brûlées par le feu, les étrangers dévorent votre pays devant vous, et il sera désolé comme une terre ravagée par l'ennemi. \EVERSE}
\newcommand{\isIvVIIIfr}{\VERSE  Et la fille de Sion demeurera comme une cabane dans une vigne, et comme une hutte dans un champ de concombres, et comme une ville livrée au pillage. \EVERSE}
\newcommand{\isIvIXfr}{\VERSE  Si le Seigneur des armées ne nous avait laissé un reste, nous aurions été comme Sodome, et nous serions semblables à Gomorrhe. \EVERSE}
\newcommand{\isIvXfr}{\VERSE  Écoutez la parole du Seigneur, princes de Sodome; prêtez l'oreille à la loi de notre Dieu, peuple de Gomorrhe. \EVERSE}
\newcommand{\isIvXIfr}{\VERSE  Qu'ai-Je affaire de la multitude de vos victimes? dit le Seigneur. J'en suis rassasié. Je ne veux ni des holocaustes de béliers, ni de la graisse des troupeaux, ni du sang des veaux, des agneaux et des boucs. \EVERSE}
\newcommand{\isIvXIIfr}{\VERSE  Lorsque vous veniez devant Moi pour vous promener dans Mes parvis, qui a demandé ces offrandes à vos mains? \EVERSE}
\newcommand{\isIvXIIIfr}{\VERSE  Ne M'offrez plus de vain sacrifice; l'encens M'est en abomination. Je ne puis souffrir les néoménies, les sabbats et les autres fêtes; l'iniquité règne dans vos assemblées. \EVERSE}
\newcommand{\isIvXIVfr}{\VERSE  Mon âme hait vos nouvelles lunes et vos fêtes; elles Me sont devenues à charge, Je suis las de les supporter. \EVERSE}
\newcommand{\isIvXVfr}{\VERSE  Lorsque vous étendrez vos mains, Je détournerai Mes yeux de vous; et lorsque vous multiplierez les prières, Je n'écouterai point, parce que vos mains sont pleines de sang. \EVERSE}
\newcommand{\isIvXVIfr}{\VERSE  Lavez-vous, purifiez-vous, ôtez de devant Mes yeux la malice de vos pensées, cessez de faire le mal, \EVERSE}
\newcommand{\isIvXVIIfr}{\VERSE  apprenez à faire le bien, recherchez la justice, assistez l'opprimé, faites droit à l'orphelin, défendez la veuve. \EVERSE}
\newcommand{\isIvXVIIIfr}{\VERSE  Et venez et attaquez-Moi, dit le Seigneur; et si vos péchés sont comme l'écarlate, ils deviendront blancs comme la neige; et s'ils sont rouges comme le vermillon, ils seront blancs comme la laine. \EVERSE}
\newcommand{\isIvXIXfr}{\VERSE  Si vous voulez et si vous M'écoutez, vous mangerez les biens de la terre. \EVERSE}
\newcommand{\isIvXXfr}{\VERSE  Que si vous ne voulez pas, et si vous provoquez Ma colère, l'épée vous dévorera, car c'est la bouche du Seigneur qui a parlé. \EVERSE}
\newcommand{\isIvXXIfr}{\VERSE  Comment la cité fidèle, pleine d'équité, est-elle devenue une prostituée? La justice habitait en elle, et maintenant il y a des meurtriers. \EVERSE}
\newcommand{\isIvXXIIfr}{\VERSE  Ton argent s'est changé en scories, ton vin a été mêlé d'eau. \EVERSE}
\newcommand{\isIvXXIIIfr}{\VERSE  Tes princes sont infidèles, complices des voleurs. Tous ils aiment les présents, ils recherchent les récompenses. Ils ne font pas droit à l'orphelin, et la cause de la veuve n'a pas d'accès auprès d'eux. \EVERSE}
\newcommand{\isIvXXIVfr}{\VERSE  C'est pourquoi voici ce que dit le Seigneur, le Dieu des armées, le fort d'Israël: Ah! Je Me consolerai par la perte de Mes adversaires, et Je Me vengerai de Mes ennemis. \EVERSE}
\newcommand{\isIvXXVfr}{\VERSE  Et J'étendrai Ma main sur toi, et Je te purifierai par le feu de tes scories, et J'enlèverai tout l'étain qui est en toi. \EVERSE}
\newcommand{\isIvXXVIfr}{\VERSE  Et Je rétablirai tes juges comme ils étaient autrefois, et tes conseillers comme ils étaient à l'origine; après cela tu seras appelée cité du juste, ville fidèle. \EVERSE}
\newcommand{\isIvXXVIIfr}{\VERSE  Sion sera rachetée par le jugement, et on la rétablira par la justice. \EVERSE}
\newcommand{\isIvXXVIIIfr}{\VERSE  Mais les scélérats et les pécheurs périront tous ensemble, et ceux qui auront abandonné le Seigneur seront consumés. \EVERSE}
\newcommand{\isIvXXIXfr}{\VERSE  Car ils seront confondus par les idoles auxquelles ils ont sacrifié, et vous rougirez à cause des jardins que vous aviez choisis; \EVERSE}
\newcommand{\isIvXXXfr}{\VERSE  lorsque vous serez devenus comme un chêne dont les feuilles tombent, et comme un jardin sans eau. \EVERSE}
\newcommand{\isIvXXXIfr}{\VERSE  Votre force sera comme de l'étoupe sêche, et votre oeuvre comme une étincelle; et l'une et l'autre s'embrasera, et il n'y aura personne pour l'éteindre. \EVERSE}
\newcommand{\isIIvIfr}{\VERSE  Vision d'Isaïe, fils d'Amos, sur Juda et Jérusalem. \EVERSE}
\newcommand{\isIIvIIfr}{\VERSE  Il arrivera, dans les derniers temps, que la montagne de la maison du Seigneur sera fondée sur le sommet des montagnes, et qu'elle s'élèvera au-dessus des collines; et toutes les nations y afflueront, \EVERSE}
\newcommand{\isIIvIIIfr}{\VERSE  et des peuples nombreux y viendront, et diront: Venez, et montons à la montagne du Seigneur, et à la maison du Dieu de Jacob; et Il nous enseignera Ses voies, et nous marcherons dans Ses sentiers, car de Sion sortira la loi,et la parole du Seigneur de Jérusalem. \EVERSE}
\newcommand{\isIIvIVfr}{\VERSE  Et Il jugera les nations, et Il convaincra d'erreur des peuples nombreux; et ils forgeront de leurs glaives des socs de charrue, et de leurs lances des faux. Un peuple ne tirera plus l'épée contre un autre, et on ne s'exercera plus au combat. \EVERSE}
\newcommand{\isIIvVfr}{\VERSE  Maison de Jacob, venez, et marchons à la lumière du Seigneur. \EVERSE}
\newcommand{\isIIvVIfr}{\VERSE  Car Vous avez rejeté Votre peuple, la maison de Jacob, parce qu'ils ont été remplis de superstitions comme autrefois, qu'ils ont eu des augures comme les Philistins, et qu'ils se sont attachés aux fils des étrangers. \EVERSE}
\newcommand{\isIIvVIIfr}{\VERSE  Leur terre est remplie d'or et d'argent, et il n'y a pas de fin à leurs trésors. \EVERSE}
\newcommand{\isIIvVIIIfr}{\VERSE  Leur pays est plein de chevaux, et leurs chars sont innombrables. Et leur pays est rempli d'idoles; ils ont adoré l'oeuvre de leurs mains, qu'ils avaient formée de leurs doigts. \EVERSE}
\newcommand{\isIIvIXfr}{\VERSE  L'homme du peuple s'est abaissé, et les grands se sont humiliés: ne leur pardonnez donc pas. \EVERSE}
\newcommand{\isIIvXfr}{\VERSE  Entre dans les rochers, et cache-toi dans les creux de la terre, pour éviter la terreur du Seigneur et la gloire de Sa majesté. \EVERSE}
\newcommand{\isIIvXIfr}{\VERSE  Les yeux altiers de l'homme seront humiliés, la hauteur des grands sera abaissée, et le Seigneur seul sera élevé en ce jour-là. \EVERSE}
\newcommand{\isIIvXIIfr}{\VERSE  Car voici le jour du Seigneur des armées contre tous les superbes, sur les hautains, et sur tous les insolents, et ils seront humiliés; \EVERSE}
\newcommand{\isIIvXIIIfr}{\VERSE  contre tous les cèdres du Liban, hauts et élevés, contre tous les chênes de Basan, \EVERSE}
\newcommand{\isIIvXIVfr}{\VERSE  contre toutes les hautes montagnes, et contre toutes les collines élevées, \EVERSE}
\newcommand{\isIIvXVfr}{\VERSE  contre toutes les hautes tours, et contre toute muraille fortifiée, \EVERSE}
\newcommand{\isIIvXVIfr}{\VERSE  contre tous les vaisseaux de Tharsis, et contre tout ce qui est beau et plaît à la vue. \EVERSE}
\newcommand{\isIIvXVIIfr}{\VERSE  Et l'élévation des hommes sera abaissée, la hauteur des grands sera humiliée, et le Seigneur seul sera élevé en ce jour-là; \EVERSE}
\newcommand{\isIIvXVIIIfr}{\VERSE  et les idoles seront entièrement brisées: \EVERSE}
\newcommand{\isIIvXIXfr}{\VERSE  Et on entrera dans les cavernes des rochers, et dans les gouffres de la terre, pour éviter la terreur du Seigneur et la gloire de Sa majesté, lorsqu'Il Se lèvera pour frapper la terre. \EVERSE}
\newcommand{\isIIvXXfr}{\VERSE  En ce jour-là l'homme jettera ses idoles d'argent et ses statues d'or, qu'il s'était faites pour les adorer, les images des taupes et des chauves-souris; \EVERSE}
\newcommand{\isIIvXXIfr}{\VERSE  et il entrera dans les fentes des rochers et dans les creux des pierres, pour éviter la terreur du Seigneur et la gloire de Sa majesté, lorsqu'Il Se lèvera pour frapper la terre. \EVERSE}
\newcommand{\isIIvXXIIfr}{\VERSE  Cessez donc de vous confier en l'homme, dans les narines duquel il n'y a qu'un souffle, car c'est Dieu seul qui est le Très-Haut. \EVERSE}
\newcommand{\isIIIvIfr}{\VERSE  Voici que le dominateur, le Seigneur des armées, va ôter de Jérusalem et de Juda l'homme valide et l'homme fort, toute la force du pain et toute la force de l'eau, \EVERSE}
\newcommand{\isIIIvIIfr}{\VERSE  l'homme fort et l'homme de guerre, le juge et le prophète, le devin et le vieillard, \EVERSE}
\newcommand{\isIIIvIIIfr}{\VERSE  le chef de cinquante et l'homme au visage vénérable, le conseiller, les plus sages d'entre les architectes, et ceux qui ont l'intelligence des paroles mystiques. \EVERSE}
\newcommand{\isIIIvIVfr}{\VERSE  Je leur donnerai des enfants pour princes, et des efféminés domineront sur eux. \EVERSE}
\newcommand{\isIIIvVfr}{\VERSE  Et le peuple se précipitera, homme contre homme, et l'ami contre l'ami; l'enfant se soulèvera contre le vieillard, et l'homme de rien contre le noble. \EVERSE}
\newcommand{\isIIIvVIfr}{\VERSE  Et l'on saisira son frère, né dans la maison paternelle: Tu as un vêtement, sois notre prince, et que cette ruine soit sous ta main. \EVERSE}
\newcommand{\isIIIvVIIfr}{\VERSE  En ce jour il répondra: Je ne suis pas médecin, et dans ma maison il n'y a ni pain ni vêtement; ne m'établissez pas prince du peuple. \EVERSE}
\newcommand{\isIIIvVIIIfr}{\VERSE  Car Jérusalem chancelle et Juda va tomber, parce que leurs paroles et leurs oeuvres sont contre le Seigneur, pour provoquer les yeux de Sa majesté. \EVERSE}
\newcommand{\isIIIvIXfr}{\VERSE  L'aspect de leur visage témoigne contre eux, et ils ont publié hautement leur péché comme Sodome, et ils ne l'ont point caché. Malheur à leur âme, parce que des maux sont tombés sur eux! \EVERSE}
\newcommand{\isIIIvXfr}{\VERSE  Dites au juste qu'il prospérera, qu'il se nourrira du fruit de ses oeuvres. \EVERSE}
\newcommand{\isIIIvXIfr}{\VERSE  Malheur à l'impie, pour sa ruine, car il lui sera fait selon l'oeuvre de ses mains. \EVERSE}
\newcommand{\isIIIvXIIfr}{\VERSE  Mon peuple a été dépouillé par ses oppresseurs, et des femmes ont dominé sur lui. Mon peuple, ceux qui te disent bienheureux te trompent, et ils détruisent le chemin par où tu dois marcher. \EVERSE}
\newcommand{\isIIIvXIIIfr}{\VERSE  Le Seigneur Se tient debout pour juger, Il Se tient debout pour juger les peuples. \EVERSE}
\newcommand{\isIIIvXIVfr}{\VERSE  Le Seigneur entrera en jugement avec les anciens et les princes de Son peuple, car vous avez dévoré la vigne, et la dépouille du pauvre est dans vos maisons. \EVERSE}
\newcommand{\isIIIvXVfr}{\VERSE  Pourquoi foulez-vous aux pieds Mon peuple et broyez-vous le visage des pauvres? dit le Seigneur, le Dieu des armées. \EVERSE}
\newcommand{\isIIIvXVIfr}{\VERSE  Et le Seigneur dit: Parce que les filles de Sion se sont élevées, qu'elles ont marché le cou tendu, en faisant des signes des yeux et en s'applaudissant, et qu'elles ont mesuré leurs pas et étudié leur démarche, \EVERSE}
\newcommand{\isIIIvXVIIfr}{\VERSE  le Seigneur rendra chauve la tête des filles de Sion, et Il découvrira leur nudité. \EVERSE}
\newcommand{\isIIIvXVIIIfr}{\VERSE  En ce jour, le Seigneur ôtera l'ornement de leurs chaussures, et les croissants, \EVERSE}
\newcommand{\isIIIvXIXfr}{\VERSE  et les colliers, et les filets de perles, et les bracelets, et les mitres, \EVERSE}
\newcommand{\isIIIvXXfr}{\VERSE  les rubans de cheveux, et les chaînettes des pieds, et les chaînes d'or, et les boîtes de senteur, et les pendants d'oreilles, \EVERSE}
\newcommand{\isIIIvXXIfr}{\VERSE  et les anneaux, et les pierreries qui leur pendent sur le front, \EVERSE}
\newcommand{\isIIIvXXIIfr}{\VERSE  et les vêtements précieux, et les écharpes, et les voiles, et les riches épingles, \EVERSE}
\newcommand{\isIIIvXXIIIfr}{\VERSE  et les miroirs, et les chemises de prix, et les bandeaux, et les voiles légers. \EVERSE}
\newcommand{\isIIIvXXIVfr}{\VERSE  Et au lieu de parfum il y aura la puanteur; au lieu de ceinture, une corde; au lieu de cheveux frisés, une tête chauve, et au lieu de riches corps de jupes, un cilice. \EVERSE}
\newcommand{\isIIIvXXVfr}{\VERSE  Tes hommes les plus beaux tomberont sous le glaive, et tes héros dans le combat. \EVERSE}
\newcommand{\isIIIvXXVIfr}{\VERSE  Les portes de Sion seront dans le deuil et dans les larmes, et elle s'assiéra à terre désolée. \EVERSE}
\newcommand{\isIVvIfr}{\VERSE  Et sept femmes saisiront un même homme en ce jour-là, et elles lui diront: Nous mangerons notre pain, et nous nous couvrirons de vêtements à nos frais; agrée seulement que nous portions ton nom, enlève notre opprobre. \EVERSE}
\newcommand{\isIVvIIfr}{\VERSE  En ce jour-là, le Germe du Seigneur sera dans la magnificence et dans la gloire, et le fruit de la terre sera élevé en honneur, et une cause d'allégresse pour ceux d'Israël qui auront été sauvés. \EVERSE}
\newcommand{\isIVvIIIfr}{\VERSE  Alors tous ceux qui seront restés dans Sion et qui seront demeurés dans Jérusalem seront appelés saints, tous ceux qui auront été écrits dans Jérusalem au nombre des vivants. \EVERSE}
\newcommand{\isIVvIVfr}{\VERSE  Alors le Seigneur purifiera les souillures des filles de Sion, et Il lavera Jérusalem du sang qui est au milieu d'elle, par un esprit de justice et par un esprit d'ardeur. \EVERSE}
\newcommand{\isIVvVfr}{\VERSE  Et le Seigneur établira sur toute l'étendue de la montagne de Sion, et au lieu où Il aura été invoqué, une nuée obscure pendant le jour, et l'éclat d'une flamme ardente pendant la nuit; car tout ce qui est glorieux sera protégé. \EVERSE}
\newcommand{\isIVvVIfr}{\VERSE  Et il y aura une tente pour donner de l'ombre contre la chaleur pendant le jour, et pour servir de retraite assurée et d'asile contre l'orage et la pluie. \EVERSE}
\newcommand{\isVvIfr}{\VERSE  Je chanterai à mon bien-aimé le cantique de mon proche parent pour sa vigne. Mon bien-aimé avait une vigne sur une colline fertile. \EVERSE}
\newcommand{\isVvIIfr}{\VERSE  Il l'entoura d'une haie, il en ôta les pierres, et y mit un plant excellent; il bâtit une tour au milieu, et il y construisit un pressoir; et il attendit qu'elle produisît de bons raisins, et elle en a produit de sauvages. \EVERSE}
\newcommand{\isVvIIIfr}{\VERSE  Maintenant donc, habitants de Jérusalem et hommes de Juda, soyez juges entre moi et ma vigne. \EVERSE}
\newcommand{\isVvIVfr}{\VERSE  Qu'ai-je dû faire de plus à ma vigne que je n'aie point fait? Ai-je eu tort d'attendre qu'elle portât de bons raisins, tandis qu'elle en a produit de sauvages? \EVERSE}
\newcommand{\isVvVfr}{\VERSE  Et maintenant je vous montrerai ce que je vais faire à ma vigne. J'en arracherai la haie, et elle sera exposée au pillage; je détruirai son mur, et elle sera foulée aux pieds. \EVERSE}
\newcommand{\isVvVIfr}{\VERSE  Je la rendrai déserte; elle ne sera ni taillée ni labourée; les ronces et les épines y grandiront, et je commanderai aux nuées de ne plus pleuvoir sur elle. \EVERSE}
\newcommand{\isVvVIIfr}{\VERSE  La vigne du Seigneur des armées c'est la maison d'Israël, et les hommes de Juda sont le plant auquel Il prenait Ses délices; et j'ai attendu qu'ils pratiquassent la droiture, et je ne vois qu'iniquité; et qu'ils portassent des fruits de justice, et je n'entends que des cris de détresse. \EVERSE}
\newcommand{\isVvVIIIfr}{\VERSE  Malheur à vous qui joignez maison à maison, et qui ajoutez terres à terres, jusqu'à ce que l'espace vous manque! Serez-vous donc les seuls à habiter sur la terre? \EVERSE}
\newcommand{\isVvIXfr}{\VERSE  J'ai appris ce que vous faites, dit le Seigneur des armées; certainement ces maisons nombreuses, grandes et belles, seront désertes, sans habitant. \EVERSE}
\newcommand{\isVvXfr}{\VERSE  Car dix arpents de vignes ne rempliront qu'une petite bouteille, et trente boisseaux de semence n'en rendront que trois. \EVERSE}
\newcommand{\isVvXIfr}{\VERSE  Malheur à vous, qui vous levez dès le matin pour vous livrer à l'orgie et pour boire jusqu'au soir, jusqu'à ce que vous soyez échauffés par le vin. \EVERSE}
\newcommand{\isVvXIIfr}{\VERSE  La harpe et le luth, le tambourin et la flûte, et le vin, se trouvent dans vos festins; et vous ne prenez pas garde à l'oeuvre du Seigneur, et vous ne considérez pas les ouvrages de ses mains. \EVERSE}
\newcommand{\isVvXIIIfr}{\VERSE  C'est pour cela que Mon peuple a été emmené captif, parce qu'il n'a pas eu de science; ses nobles sont morts de faim, et sa multitude a séché de soif. \EVERSE}
\newcommand{\isVvXIVfr}{\VERSE  C'est pour cela que le séjour des morts a dilaté son âme, et qu'il a ouvert sa bouche sans mesure, et les héros d'Israël, et son peuple, et ses hommes illustres et glorieux y descendront. \EVERSE}
\newcommand{\isVvXVfr}{\VERSE  Et l'homme du peuple devra se courber, et les grands seront humiliés, et les yeux des superbes seront abaissés. \EVERSE}
\newcommand{\isVvXVIfr}{\VERSE  Et le Seigneur des armées sera exalté par le jugement, et le Dieu saint sera sanctifié par la justice. \EVERSE}
\newcommand{\isVvXVIIfr}{\VERSE  Alors les agneaux paîtront selon leur coutume, et les étrangers viendront se nourrir dans les déserts devenus fertiles. \EVERSE}
\newcommand{\isVvXVIIIfr}{\VERSE  Malheur à vous, qui traînez l'iniquité avec les cordes du mensonge, et le péché avec les traits d'un chariot; \EVERSE}
\newcommand{\isVvXIXfr}{\VERSE  vous qui dites: Qu'Il Se hâte, et que Son oeuvre arrive bientôt, afin que nous la voyons; que le décret du Saint d'Israël s'approche et s'accomplisse, afin que nous le connaissions. \EVERSE}
\newcommand{\isVvXXfr}{\VERSE  Malheur à vous, qui dites que le mal est bien, et que le bien est mal; qui changez les ténèbres en lumière, et la lumière en ténèbres; qui changez l'amertume en douceur, et la douceur en amertume. \EVERSE}
\newcommand{\isVvXXIfr}{\VERSE  Malheur à vous, qui êtes sages à vos propres yeux, et qui êtes prudents selon vous-mêmes. \EVERSE}
\newcommand{\isVvXXIIfr}{\VERSE  Malheur à vous, qui êtes puissants à boire le vin, et vaillants pour faire des mélanges enivrants; \EVERSE}
\newcommand{\isVvXXIIIfr}{\VERSE  qui justifiez l'impie pour des présents, et qui ravissez au juste sa justice. \EVERSE}
\newcommand{\isVvXXIVfr}{\VERSE  C'est pourquoi, comme la langue du feu dévore la paille, et comme la chaleur de la flamme la consume, ainsi leur racine sera comme de la cendre, et leur tige se dissipera comme de la poussière, car ils ont rejeté la loi du Seigneur des armées, et ils ont blasphémé la parole du Saint d'Israël. \EVERSE}
\newcommand{\isVvXXVfr}{\VERSE  C'est pourquoi la fureur du Seigneur s'est allumée contre Son peuple, et Il a étendu sa main sur lui, et Il l'a frappé; et les montagnes ont été ébranlées, et les cadavres ont été comme de l'ordure au milieu des places publiques. Malgré tout cela, Sa fureur n'est point apaisée, et Sa main est encore étendue. \EVERSE}
\newcommand{\isVvXXVIfr}{\VERSE  Il élèvera un étendard pour les peuples lointains; Il en appellera un d'un coup de sifflet des extrémités de la terre, et il accourra aussitôt avec une vitesse prodigieuse. \EVERSE}
\newcommand{\isVvXXVIIfr}{\VERSE  Nul, dans ses rangs, ne sentira la lassitude et la fatigue; personne ne sommeillera ni ne dormira; aucun n'aura la ceinture de ses reins détachée, ni la courroie de ses chaussures rompue. \EVERSE}
\newcommand{\isVvXXVIIIfr}{\VERSE  Ses flèches sont acérées, et tous ses arcs bandés. Les sabots de ses chevaux sont comme des cailloux, et les roues de ses chars ont la rapidité de la tempête. \EVERSE}
\newcommand{\isVvXXIXfr}{\VERSE  Son rugissement est comme celui d'un lion, il rugira comme des lionceaux; il frémira et saisira sa proie, et il l'emportera, et personne ne viendra la lui enlever. \EVERSE}
\newcommand{\isVvXXXfr}{\VERSE  En ce jour, un bruit semblable à celui de la mer retentira sur lui; nous regarderons sur la terre, et il n'y aura que les ténèbres de la tribulation, et la lumière disparaîtra dans cette profonde obscurité. \EVERSE}
\newcommand{\isVIvIfr}{\VERSE  L'année de la mort du roi Ozias, je vis le Seigneur assis sur un trône sublime et élevé, et le bas de Ses vêtements remplissait le temple. \EVERSE}
\newcommand{\isVIvIIfr}{\VERSE  Les séraphins se tenaient au-dessus du trône; ils avaient chacun six ailes: deux dont ils voilaient leur face, deux dont ils voilaient leurs pieds, et deux dont ils se servaient pour voler. \EVERSE}
\newcommand{\isVIvIIIfr}{\VERSE  Ils criaient l'un à l'autre et disaient: Saint, Saint, Saint est le Seigneur, le Dieu des armées; toute la terre est remplie de Sa gloire. \EVERSE}
\newcommand{\isVIvIVfr}{\VERSE  Les linteaux des portes furent ébranlés par la voix qui retentissait, et la maison fut remplie de fumée. \EVERSE}
\newcommand{\isVIvVfr}{\VERSE  Alors je dis: Malheur à moi de ce que je me suis tu, car je suis un homme aux lèvres impures, et j'habite au milieu d'un peuple dont les lèvres sont impures, et j'ai vu de mes yeux le Roi, le Seigneur des armées. \EVERSE}
\newcommand{\isVIvVIfr}{\VERSE  Mais un des séraphins vola vers moi, tenant dans sa main un charbon ardent qu'il avait pris avec des pincettes de dessus l'autel; \EVERSE}
\newcommand{\isVIvVIIfr}{\VERSE  et il toucha ma bouche, et dit: Ceci a touché tes lèvres; ton iniquité sera enlevée, et ton péché sera purifié. \EVERSE}
\newcommand{\isVIvVIIIfr}{\VERSE  Et j'entendis la voix du Seigneur disant: Qui enverrai-Je? et qui ira pour Nous? Je répondis: Me voici; envoyez-moi. \EVERSE}
\newcommand{\isVIvIXfr}{\VERSE  Et Il dit: Va, et dis à ce peuple: Ecoutez ce que Je vous dis, et ne le comprenez pas; voyez ce que Je vous fais voir, et ne le discernez pas. \EVERSE}
\newcommand{\isVIvXfr}{\VERSE  Aveugle le coeur de ce peuple, et rends ses oreilles dures, et bouche-lui les yeux, de peur qu'il ne voie de ses yeux, et qu'il n'entende de ses oreilles, et qu'il ne comprenne de son coeur, et qu'il ne se convertisse, et que Je ne le guérisse. \EVERSE}
\newcommand{\isVIvXIfr}{\VERSE  Et je dis: Jusques à quand, Seigneur? Et Il dit: Jusqu'à ce que les villes soient désolées et sans citoyens, les maisons sans habitant, et que la terre demeure déserte. \EVERSE}
\newcommand{\isVIvXIIfr}{\VERSE  Le Seigneur éloignera les hommes, et celle qui avait été délaissée au milieu du pays se multipliera. \EVERSE}
\newcommand{\isVIvXIIIfr}{\VERSE  Et elle sera encore décimée, et elle reviendra au Seigneur, et elle paraîtra dans sa grandeur comme un térébinthe, et comme un chêne qui étend ses rameaux; la race qui demeurera en elle sera sainte. \EVERSE}
\newcommand{\isVIIvIfr}{\VERSE  Il arriva au temps d'Achaz, fils de Joathan, fils d'Ozias, roi de Juda, que Rasin, roi de Syrie, et Phacée, fils de Romélie, roi d'Israël, montèrent contre Jérusalem pour l'assiéger; et ils ne purent s'en emparer. \EVERSE}
\newcommand{\isVIIvIIfr}{\VERSE  Et l'on vint dire à la maison de David: La Syrie a fait sa jonction avec Ephraïm. Et le coeur d'Achaz et le coeur de son peuple furent agités, comme les arbres des forêts sont agités par le vent. \EVERSE}
\newcommand{\isVIIvIIIfr}{\VERSE  Alors le Seigneur dit à Isaïe: Va au-devant d'Achaz, toi et Jasub, ton fils qui t'est resté, vers l'extrémité de l'aqueduc de la piscine supérieure, sur le chemin du champ du Foulon; \EVERSE}
\newcommand{\isVIIvIVfr}{\VERSE  et dis-lui: Aie soin de demeurer en paix; ne crains point, et que ton coeur ne se trouble pas devant ces deux bouts de tisons fumants, devant la colère et la fureur de Rasin, roi de Syrie, et du fils de Romélie; \EVERSE}
\newcommand{\isVIIvVfr}{\VERSE  de ce que la Syrie, Ephraïm et le fils de Romélie ont conspiré ensemble contre toi, en disant: \EVERSE}
\newcommand{\isVIIvVIfr}{\VERSE  Montons contre Juda, faisons-lui la guerre, et rendons-nous-en les maîtres, et établissons-y pour roi le fils de Tabéel. \EVERSE}
\newcommand{\isVIIvVIIfr}{\VERSE  Voici ce que dit le Seigneur Dieu: Cela ne subsistera pas, et cela ne sera pas; \EVERSE}
\newcommand{\isVIIvVIIIfr}{\VERSE  mais Damas sera la tête de la Syrie, et Rasin la tête de Damas; et dans soixante-cinq ans Ephraïm cessera d'être un peuple; \EVERSE}
\newcommand{\isVIIvIXfr}{\VERSE  et Samarie sera la tête d'Ephraïm, et le fils de Romélie la tête de Samarie. Si vous ne croyez pas, vous ne subsisterez pas. \EVERSE}
\newcommand{\isVIIvXfr}{\VERSE  Le Seigneur continua de parler à Achaz et lui dit: \EVERSE}
\newcommand{\isVIIvXIfr}{\VERSE  Demande pour toi un signe au Seigneur ton Dieu, soit au fond de la terre, soit au plus haut du ciel. \EVERSE}
\newcommand{\isVIIvXIIfr}{\VERSE  Et Achaz répondit: Je ne demanderai rien, et je ne tenterai pas le Seigneur. \EVERSE}
\newcommand{\isVIIvXIIIfr}{\VERSE  Et Isaïe dit: Ecoutez donc, maison de David. Ne vous suffit-il pas de lasser la patience des hommes, que vous lassiez encore celle de mon Dieu? \EVERSE}
\newcommand{\isVIIvXIVfr}{\VERSE  C'est pourquoi le Seigneur Lui-même vous donnera un signe: Une vierge concevra, et Elle enfantera un fils, auquel on donnera le nom d'Emmanuel. \EVERSE}
\newcommand{\isVIIvXVfr}{\VERSE  Il mangera du beurre et du miel, en sorte qu'il sache rejeter le mal et choisir le bien. \EVERSE}
\newcommand{\isVIIvXVIfr}{\VERSE  Car avant que l'enfant sache rejeter le mal et choisir le bien, le pays que tu détestes à cause de leurs deux rois sera abandonné. \EVERSE}
\newcommand{\isVIIvXVIIfr}{\VERSE  Le Seigneur fera venir sur toi, sur ton peuple et sur la maison de ton père, par le roi des Assyriens, des jours tels qu'il n'y en a pas eu depuis le temps où Ephraïm s'est séparé de Juda. \EVERSE}
\newcommand{\isVIIvXVIIIfr}{\VERSE  En ce jour-là, le Seigneur appellera d'un coup de sifflet la mouche qui est à l'extrémité des fleuves de l'Egypte, et l'abeille qui est au pays d'Assur; \EVERSE}
\newcommand{\isVIIvXIXfr}{\VERSE  et elles viendront, et elles se poseront dans les torrents des vallées, et dans les creux des rochers, sur tous les arbrisseaux, et dans tous les trous. \EVERSE}
\newcommand{\isVIIvXXfr}{\VERSE  En ce jour-là le Seigneur rasera, avec un rasoir pris à louage, avec ceux qui sont au delà du fleuve, avec le roi des Assyriens, la tête, le poil des pieds, et toute la barbe. \EVERSE}
\newcommand{\isVIIvXXIfr}{\VERSE  En ce jour-là, chacun nourrira une vache et deux brebis, \EVERSE}
\newcommand{\isVIIvXXIIfr}{\VERSE  et le lait sera si abondant qu'on mangera du beurre; car quiconque sera demeuré dans le pays se nourrira de beurre et de miel. \EVERSE}
\newcommand{\isVIIvXXIIIfr}{\VERSE  En ce jour-là, tout lieu où il y avait eu mille pieds de vigne, valant mille pièces d'aargent, sera livré aux ronces et aux épines. \EVERSE}
\newcommand{\isVIIvXXIVfr}{\VERSE  On y entrera avec les flèches et l'arc, car les ronces et les épines couvriront tout le pays. \EVERSE}
\newcommand{\isVIIvXXVfr}{\VERSE  Et toutes les montagnes qui étaient sarclées et cultivées n'inspireront plus de crainte par leurs ronces et leurs épines, mais elles serviront de pâturages aux boeufs, et les troupeaux les fouleront. \EVERSE}
\newcommand{\isVIIIvIfr}{\VERSE  Le Seigneur me dit: Prends un grand livre, et écris dedans, en caractères lisibles: Hâtez-vous de saisir les dépouilles, pillez promptement. \EVERSE}
\newcommand{\isVIIIvIIfr}{\VERSE  Et je pris des témoins fidèles, le prêtre Urie, et Zacharie, fils de Barachie; \EVERSE}
\newcommand{\isVIIIvIIIfr}{\VERSE  et je m'approchai de la prophétesse, et elle conçut et enfanta un fils. Alors le Seigneur me dit: Donne-lui pour noms: Hâtez-vous de saisir les dépouilles, pillez promtement; \EVERSE}
\newcommand{\isVIIIvIVfr}{\VERSE  car avant que l'enfant sache nommer son père et sa mère, la puissance de Damas et les dépouilles de Samarie seront emportées devant le roi des Assyriens. \EVERSE}
\newcommand{\isVIIIvVfr}{\VERSE  Le Seigneur me parla encore, et me dit: \EVERSE}
\newcommand{\isVIIIvVIfr}{\VERSE  Parce que ce peuple a rejeté les eaux de Siloé, qui coulent en silence, et qu'il a préféré s'appuyer sur Rasin et sur le fils de Romélie, \EVERSE}
\newcommand{\isVIIIvVIIfr}{\VERSE  le Seigneur amènera sur lui les puissantes et grandes eaux du fleuve, le roi des Assyriens avec toute sa gloire; il montera de tous côtés au-dessus de son lit, et il débordera sur toutes ses rives, \EVERSE}
\newcommand{\isVIIIvVIIIfr}{\VERSE  et il pénétrera dans Juda, inondant et se répandant, jusqu'à ce qu'on ait de l'eau jusqu'au cou. Il déploiera ses ailes et remplira l'étendue de Votre pays, ô Emmanuel. \EVERSE}
\newcommand{\isVIIIvIXfr}{\VERSE  Assemblez-vous, peuples, et vous serez vaincus; écoutez, vous tous, pays éloignés; réunissez vos forces, et vous serez vaincus; prenez vos armes, et vous serez vaincus; \EVERSE}
\newcommand{\isVIIIvXfr}{\VERSE  formez des desseins, et ils seront dissipés; donnez des ordres, et ils ne s'exécuteront pas, car Dieu est avec nous. \EVERSE}
\newcommand{\isVIIIvXIfr}{\VERSE  Car ainsi m'a parlé le Seigneur me tenant de Sa main puissante, et m'avertissant de ne pas marcher dans la voie de ce peuple, en disant: \EVERSE}
\newcommand{\isVIIIvXIIfr}{\VERSE  Ne dites point: Conjuration; car tout ce que dit ce peuple est conjuration; ne craignez pas ce qu'il craint, et ne vous épouvantez pas; \EVERSE}
\newcommand{\isVIIIvXIIIfr}{\VERSE  mais sanctifiez le Seigneur des armées; qu'Il soit Lui-même votre crainte et votre terreur, \EVERSE}
\newcommand{\isVIIIvXIVfr}{\VERSE  et Il deviendra votre sanctification; et Il sera une pierre d'achoppement et une pierre de scandale pour les deux maisons d'Israël, un piège et un sujet de ruine pour les habitants de Jérusalem. \EVERSE}
\newcommand{\isVIIIvXVfr}{\VERSE  Et beaucoup d'entre eux trébucheront; ils tomberont et se briseront, ils s'engageront dans le filet et seront pris. \EVERSE}
\newcommand{\isVIIIvXVIfr}{\VERSE  Lie cet oracle, scelle cette révélation parmi Mes disciples. \EVERSE}
\newcommand{\isVIIIvXVIIfr}{\VERSE  J'attendrai donc le Seigneur qui cache Son visage à la maison de Jacob, et je demeurerai dans l'attente. \EVERSE}
\newcommand{\isVIIIvXVIIIfr}{\VERSE  Me voici, moi et les enfants que le Seigneur m'a donnés; nous sommes un présage pour Israël par l'ordre du Seigneur des armées qui habite sur la montagne de Sion. \EVERSE}
\newcommand{\isVIIIvXIXfr}{\VERSE  Et lorsqu'ils vous diront: Consultez les magiciens et les devins qui parlent tout bas dans leurs enchantements, répondez: Le peuple ne consultera-t-il pas son Dieu? parle-t-on aux morts en faveur des vivants? \EVERSE}
\newcommand{\isVIIIvXXfr}{\VERSE  Allons plutôt à la loi et au témoignage. S'ils ne tiennent point ce langage, la lumière du matin ne luira pas pour eux. \EVERSE}
\newcommand{\isVIIIvXXIfr}{\VERSE  Ils seront errants sur la terre, ils tomberont, ils souffriront la faim, et lorsqu'ils auront faim ils s'irriteront, et ils maudiront leur roi et leur Dieu, ils tourneront les yeux en haut, \EVERSE}
\newcommand{\isVIIIvXXIIfr}{\VERSE  et ils regarderont la terre, et ils ne verront qu'affliction et ténèbres, qu'abattement et angoisse, et que nuée sombre les poursuivant, et ils ne pourront pas échapper à leur angoisse. \EVERSE}
\newcommand{\isIXvIfr}{\VERSE  Au temps passé le pays de Zabulon et le pays de Nephthali ont été humiliés, et au temps à venir, la route de la mer, au delà du Jourdain, la Galilée des nations, seront couvertes de gloire. \EVERSE}
\newcommand{\isIXvIIfr}{\VERSE  Le peuple qui marchait dans les ténèbres a vu une grande lumière; sur ceux qui habitaient dans la région de l'ombre de la mort, une lumière s'est levée. \EVERSE}
\newcommand{\isIXvIIIfr}{\VERSE  Vous avez multiplié le peuple dont Vous n'aviez point augmenté la joie. Ils se réjouiront devant Vous, comme on se réjouit à la moisson, et comme les vainqueurs tressaillent d'allégresse lorsqu'ils ont pillé l'ennemi, et qu'ils partagent le butin. \EVERSE}
\newcommand{\isIXvIVfr}{\VERSE  Car le joug qui pesait sur lui, la verge qui déchirait son dos, et le sceptre de celui qui l'opprimait, Vous les avez brisés, comme à la journée de Madian. \EVERSE}
\newcommand{\isIXvVfr}{\VERSE  Car toutes les dépouilles remportées avec violence et dans le tumulte, et les vêtements souillés de sang seront mis au feu, et deviendront la pâture de la flamme. \EVERSE}
\newcommand{\isIXvVIfr}{\VERSE  Car un petit enfant nous est né, et un fils nous a été donné; Il portera sur Son épaule la marque de Sa principauté; et Il sera appelé Admirable, Conseiller, Dieu, Fort, Père du siècle futur, Prince de la paix. \EVERSE}
\newcommand{\isIXvVIIfr}{\VERSE  Son empire s'étendra de plus en plus, et la paix n'aura pas de fin; Il S'assiéra sur le trône de David, et Il possédera Son royaume pour l'affermir et le fortifier par le droit et par la justice, dès maintenant et à jamais; le zèle du Seigneur des armées fera ces choses. \EVERSE}
\newcommand{\isIXvVIIIfr}{\VERSE  Le Seigneur a envoyé une parole à Jacob, et elle est tombée sur Israël. \EVERSE}
\newcommand{\isIXvIXfr}{\VERSE  Tout le peuple le saura, Ephraïm et les habitants de Samarie, qui disent dans l'orgueil et dans l'arrogance de leur coeur: \EVERSE}
\newcommand{\isIXvXfr}{\VERSE  Les briques sont tombées, mais nous bâtirons en pierres de taille; ils ont coupé les sycomores, mais nous mettrons des cèdres à leur place. \EVERSE}
\newcommand{\isIXvXIfr}{\VERSE  Le Seigneur suscitera contre Israël les ennemis de Rasin, et Il fera venir en foule ses ennemis, \EVERSE}
\newcommand{\isIXvXIIfr}{\VERSE  les Syriens à l'orient, et les Philistins à l'occident, et ils dévoreront Israël à pleine bouche. Malgré tout cela, Sa fureur n'est point apaisée, et Sa main est encore étendue. \EVERSE}
\newcommand{\isIXvXIIIfr}{\VERSE  Le peuple n'est pas revenu vers Celui qui le frappait, et ils n'ont pas recherché le Seigneur des armées. \EVERSE}
\newcommand{\isIXvXIVfr}{\VERSE  Aussi le Seigneur retranchera en un seul jour la tête et la queue, celui qui s'abaisse et celui qui s'élève. \EVERSE}
\newcommand{\isIXvXVfr}{\VERSE  Le vieillard et le personnage vénérable sont la tête, et le prophète qui enseigne le mensonge est la queue. \EVERSE}
\newcommand{\isIXvXVIfr}{\VERSE  Ceux qui appellent ce peuple heureux se trouveront être des séducteurs, et ceux qu'on proclame bienheureux se trouveront précipités dans la ruine. \EVERSE}
\newcommand{\isIXvXVIIfr}{\VERSE  C'est pourquoi le Seigneur ne mettra point Sa joie dans les jeunes gens d'Israël, Il n'aura pas pitié des orphelins et des veuves, car ils sont tous des hypocrites et des méchants, et toutes les bouches profèrent la folie. Malgré tout cela Sa fureur n'est point apaisée, et Sa main est encore étendue. \EVERSE}
\newcommand{\isIXvXVIIIfr}{\VERSE  Car l'impiété s'est allumée comme un feu: elle dévorera les ronces et les épines; elle s'embrasera dans l'épaisseur de la forêt; et des tourbillons de fumée s'élèveront en haut. \EVERSE}
\newcommand{\isIXvXIXfr}{\VERSE  Par la colère du Seigneur des armées le pays a été épouvanté, et le peuple sera comme la pâture du feu; le frère n'épargnera pas son frère. \EVERSE}
\newcommand{\isIXvXXfr}{\VERSE  On ira à droite, et on aura faim; on dévorera à gauche, et on ne sera pas rassasié; chacun dévorera la chair de son bras: Manassé dévorera Ephraïm, et Ephraïm, Manassé; et ensemble ils se soulèveront contre Juda. \EVERSE}
\newcommand{\isIXvXXIfr}{\VERSE  Malgré tout cela Sa fureur n'est point apaisée, et Sa main est encore étendue. \EVERSE}
\newcommand{\isXvIfr}{\VERSE  Malheur à ceux qui décrètent des lois iniques, et qui écrivent des ordonnances injustes, \EVERSE}
\newcommand{\isXvIIfr}{\VERSE  pour opprimer les pauvres dans le jugemeent, et pour violer le droit des faibles de Mon peuple, pour faire des veuves leur proie et pour piller l'orphelin. \EVERSE}
\newcommand{\isXvIIIfr}{\VERSE  Que ferez-vous au jour de la visite divine, au jour de la calamité qui viendra de loin? Vers qui fuirez-vous pour avoir du secours? et où laisserez-vous votre gloire, \EVERSE}
\newcommand{\isXvIVfr}{\VERSE  pour n'être pas courbés sous les chaînes, et pour ne pas tomber parmi les morts? Malgré tout cela Sa fureur n'est point apaisée, et Sa main est encore étendue. \EVERSE}
\newcommand{\isXvVfr}{\VERSE  Malheur à Assur! Il est la verge et le bâton de Ma fureur; Mon indignation est dans sa main. \EVERSE}
\newcommand{\isXvVIfr}{\VERSE  Je l'enverrai contre une nation perfide, et Je lui donnerai des ordres contre le peuple de Ma fureur, afin qu'il en emporte les dépouilles, qu'il le mette au pillage, et qu'il le foule aux pieds comme la boue des rues. \EVERSE}
\newcommand{\isXvVIIfr}{\VERSE  Mais il ne jugera pas ainsi, et son coeur n'aura pas cette pensée; mais il ne songera qu'à détruire, et à exterminer des peuples nombreux. \EVERSE}
\newcommand{\isXvVIIIfr}{\VERSE  Car il dira: Mes princes ne sont-ils pas autant de rois? \EVERSE}
\newcommand{\isXvIXfr}{\VERSE  N'en a-t-il pas été de Calano comme de Charchamis, d'Emath comme d'Arphad, de Samarie comme de Damas? \EVERSE}
\newcommand{\isXvXfr}{\VERSE  De même que Ma main atteint les royaumes des idoles, ainsi j'emporterai leurs statues de Jérusalem et de Samarie. \EVERSE}
\newcommand{\isXvXIfr}{\VERSE  Ce que j'ai fait à Samarie et à ses idoles, ne ferai-je pas aussi à Jérusalem et à ses images? \EVERSE}
\newcommand{\isXvXIIfr}{\VERSE  Mais, lorsque le Seigneur aura accompli toutes Ses oeuvres sur la montagne de Sion et dans Jérusalem: Je visiterai, dit-Il, le fruit du coeur insolent du roi d'Assur, et l'arrogance de ses yeux altiers. \EVERSE}
\newcommand{\isXvXIIIfr}{\VERSE  Car il a dit: C'est par la force de mon bras que j'ai agi, et c'est par ma sagesse que j'ai compris; et j'ai enlevé les limites des peuples, j'ai pillé les trésors de leurs princes, et comme un héros, j'ai arraché les rois de leurs trônes. \EVERSE}
\newcommand{\isXvXIVfr}{\VERSE  Ma main est descendue sur la richesse des peuples comme sur un nid, et comme on ramasse des oeufs abondonnés, ainsi j'ai ramassé toute la terre, et nul n'a remué l'aile, ni ouvert la bouche, ni poussé le moindre cri. \EVERSE}
\newcommand{\isXvXVfr}{\VERSE  La cognée se glorifie-t-elle contre celui qui s'en sert? la scie se soulève-t-elle contre celui qui la tire? Comme si la verge s'élevait contre celui qui la lève, et comme si le bâton se glorifiait, lui qui n'est que du bois! \EVERSE}
\newcommand{\isXvXVIfr}{\VERSE  C'est pour quoi le dominateur, le Seigneur des armées enverra la maigreur parmi les guerriers robustes d'Assur; et sous sa magnificence s'enflammera comme l'embrasement d'un feu. \EVERSE}
\newcommand{\isXvXVIIfr}{\VERSE  La lumière d'Israël sera un feu, et le Saint d'Israël une flamme, qui embrasera et dévorera ses ronces et ses épines en un seul jour. \EVERSE}
\newcommand{\isXvXVIIIfr}{\VERSE  La gloire de sa forêt et de ses champs délicieux sera consumée, depuis l'âme jusqu'au corps; et ils seront fugitifs de terreur. \EVERSE}
\newcommand{\isXvXIXfr}{\VERSE  Et le reste des arbres de sa forêt pourra être compté, tant il sera faible, et un enfant en écrira le nombre. \EVERSE}
\newcommand{\isXvXXfr}{\VERSE  En ce jour-là le reste d'Israël, et ceux de la maison de Jacob qui auront échappé ne s'appuieront plus sur celui qui les frappait; mais ils s'appuieront sur le Seigneur, le Saint d'Israël, avec sincérité. \EVERSE}
\newcommand{\isXvXXIfr}{\VERSE  Les restes reviendront; les restes, dis-je, de Jacob reviendront au Dieu fort. \EVERSE}
\newcommand{\isXvXXIIfr}{\VERSE  Car quand ton peuple, ô Israël, serait comme le sable de la mer, un reste seulement reviendra; la destruction qui est résolue fera déborder la justice. \EVERSE}
\newcommand{\isXvXXIIIfr}{\VERSE  Car cette destruction qui a été résolue, le Seigneur, le Dieu des armées l'accomplira au milieu de tout le pays. \EVERSE}
\newcommand{\isXvXXIVfr}{\VERSE  C'est pourquoi voici ce que dit le Seigneur, le Dieu des armées: Mon peuple, qui habites dans Sion, ne crains point Assur; il te frappera de la verge, et il lèvera son bâton sur toi, à la manière des Egyptiens. \EVERSE}
\newcommand{\isXvXXVfr}{\VERSE  Mais encore un peu, encore un moment, et Je punirai leur crime dans toute l'étendue de Mon indignation et de Ma fureur. \EVERSE}
\newcommand{\isXvXXVIfr}{\VERSE  Et le Seigneur des armées agitera le fouet contre lui, comme II  frappa Madian à la pierre d'Oreb, et comme Il leva Sa verge sur la mer, Il la lèvera encore, comme en Egypte. \EVERSE}
\newcommand{\isXvXXVIIfr}{\VERSE  En ce jour-là son fardeau sera enlevé de dessus ton épaule, et son joug de dessus ton cou, et ce joug pourrira par l'abondance de l'huile. \EVERSE}
\newcommand{\isXvXXVIIIfr}{\VERSE  Il viendra à Aïath, il passera par Magron; il laissera son bagage à Machmas. \EVERSE}
\newcommand{\isXvXXIXfr}{\VERSE  Ils passeront en courant, ils camperont à Gaba; Rama est dans l'épouvante; Gabaath de Saül prend la fuite. \EVERSE}
\newcommand{\isXvXXXfr}{\VERSE  Fais retentir ta voix, fille de Gallim; prends garde, Laïsa; pauvre Anathoth! \EVERSE}
\newcommand{\isXvXXXIfr}{\VERSE  Médeména a émigré: habitants de Gabim, ralliez-vous. \EVERSE}
\newcommand{\isXvXXXIIfr}{\VERSE  Encore un jour, et il sera à Nobé; il menacera de la main la montagne de Sion, la colline de Jérusalem. \EVERSE}
\newcommand{\isXvXXXIIIfr}{\VERSE  Voici que le Dominateur, le Seigneur des armées va briser le vase de terre par Son bras terrible: ceux qui étaient les plus hauts seront coupés, et les grands seront humiliés. \EVERSE}
\newcommand{\isXvXXXIVfr}{\VERSE  Et les taillis de la forêt seront abattus par le fer, et le Liban tombera avec ses hauts cèdres. \EVERSE}
\newcommand{\isXIvIfr}{\VERSE  Il sortira un rejeton de la tige de Jessé, et une fleur naîtra de sa racine. \EVERSE}
\newcommand{\isXIvIIfr}{\VERSE  Et l'Esprit du Seigneur se reposera sur Lui; l'esprit de sagesse et d'intelligence, l'esprit de conseil et de force, l'esprit de science et de piété; \EVERSE}
\newcommand{\isXIvIIIfr}{\VERSE  et Il sera rempli de l'esprit de la crainte du Seigneur. Il ne jugera point sur le rapport des yeux, et Il ne condamnera point par un oui-dire; \EVERSE}
\newcommand{\isXIvIVfr}{\VERSE  mais Il jugera les pauvres avec justice, et Il Se déclarera le juste vengeur des humbles de la terre; Il frappera la terre avec la verge de Sa bouche, et Il tuera l'impie par le souffle de Ses lèvres. \EVERSE}
\newcommand{\isXIvVfr}{\VERSE  La justice sera la ceinture de Ses reins, et la fidélité le baudrier dont Il sera ceint. \EVERSE}
\newcommand{\isXIvVIfr}{\VERSE  Le loup habitera avec l'agneau, et le léopard se couchera auprès du chevreau; le veau, le lion et la brebis demeureront ensemble, et un petit enfant les conduira. \EVERSE}
\newcommand{\isXIvVIIfr}{\VERSE  Le veau et l'ours iront dans les mêmes pâturages, leurs petits se reposeront ensemble, et le lion mangera la paille comme le boeuf. \EVERSE}
\newcommand{\isXIvVIIIfr}{\VERSE  L'enfant encore à la mamelle se jouera sur le trou de l'aspic, et celui qui aura été sevré mettra sa main dans la caverne du basilic. \EVERSE}
\newcommand{\isXIvIXfr}{\VERSE  Ils ne nuiront point, et ils ne tueront point sur toute Ma montagne sainte, parce que la terre est remplie de la connaissance du Seigneur, comme la mer des eaux qui la couvrent. \EVERSE}
\newcommand{\isXIvXfr}{\VERSE  En ce jour, le rejeton de Jessé sera comme un étendard pour les peuples; les nations Lui offriront leurs prières, et Son sépulcre sera glorieux. \EVERSE}
\newcommand{\isXIvXIfr}{\VERSE  En ce jour-là, le Seigneur étendra une seconde fois Sa main pour prendre possession du reste de Son peuple, qui aura échappé aux Assyriens, à l'Egypte, à Phétros, à l'Ethiopie, à Elam, à Sennaar, à Emath et aux îles de la mer. \EVERSE}
\newcommand{\isXIvXIIfr}{\VERSE  Il lèvera Son étendard parmi les nations, et Il réunira les exilés d'Israël, et Il rassemblera des quatre coins de la terre les dispersés de Juda. \EVERSE}
\newcommand{\isXIvXIIIfr}{\VERSE  La jalousie d'Ephraïm sera détruite, et les ennemis de Juda périront; Ephraïm ne sera plus envieux de Juda, et Juda ne combattra plus contre Ephraïm. \EVERSE}
\newcommand{\isXIvXIVfr}{\VERSE  Ils voleront sur l'épaule des Philistins, du côté de la mer; ils pilleront ensemble les fils de l'Orient; l'Idumée et Moab seront dociles à l'ordre de leur main, et les fils d'Ammon leur obéiront. \EVERSE}
\newcommand{\isXIvXVfr}{\VERSE  Le Seigneur rendra déserte la langue de la mer d'Egypte, et Il lèvera Sa main sur le fleuve, qu'Il agitera par Son souffle puissant; Il le frappera et le divisera en sept ruisseaux, de sorte qu'on le traversera avec des chaussures; \EVERSE}
\newcommand{\isXIvXVIfr}{\VERSE  et il y aura une route pour le reste de Mon peuple qui sera échappé des Assyriens, comme il y en eut une pour le jour où Israël sortit de la terre d'Egypte. \EVERSE}
\newcommand{\isXIIvIfr}{\VERSE  Et tu diras en ce jour-là: Je Vous rends grâces, Seigneur, de ce que Vous Vous êtes irrité contre moi; Votre fureur s'est apaisée, et Vous m'avez consolé. \EVERSE}
\newcommand{\isXIIvIIfr}{\VERSE  Voici que mon Dieu est mon Sauveur; j'agirai avec confiance, et je ne craindrai point, car le Seigneur est ma force et ma gloire, et Il est devenu mon salut. \EVERSE}
\newcommand{\isXIIvIIIfr}{\VERSE  Vous puiserez de l'eau avec joie aux fontaines du Sauveur. \EVERSE}
\newcommand{\isXIIvIVfr}{\VERSE  Et vous direz en ce jour-là: Louez le Seigneur, et invoquez Son nom; faites connaître Ses oeuvres parmi les peuples; souvenez-vous que Son nom est grand. \EVERSE}
\newcommand{\isXIIvVfr}{\VERSE  Chantez au Seigneur, car Il a fait des choses magnifiques; annoncez-les dans toute la terre. \EVERSE}
\newcommand{\isXIIvVIfr}{\VERSE  Tressaille de joie et bénis Dieu, maison de Sion, car Il est grand au milieu de toi, le Saint d'Israël. \EVERSE}
\newcommand{\isXIIIvIfr}{\VERSE  Prophétie contre Babylone, révélée à Isaïe, fils d'Amos. \EVERSE}
\newcommand{\isXIIIvIIfr}{\VERSE  Sur une montagne couverte de nuages dressez un étendard, élevez la voix, agitez la main, et que les princes entrent dans ses portes. \EVERSE}
\newcommand{\isXIIIvIIIfr}{\VERSE  J'ai donné des ordres à ceux que J'ai consacrés pour cette oeuvre; J'ai appelé Mes guerriers dans Ma colère, ils travaillent avec joie pour Ma gloire. \EVERSE}
\newcommand{\isXIIIvIVfr}{\VERSE  Bruit d'une multitude dans les montagnes, semblable à celui de peuples nombreux; bruit confus de rois et de nations rassemblées. \EVERSE}
\newcommand{\isXIIIvVfr}{\VERSE  Le Seigneur des armées a donné des ordres à Ss troupes qui viennent d'un pays lointain, de l'extrémité des cieux; le Seigneur et les instruments de S fureur vont détruire toute la terre. \EVERSE}
\newcommand{\isXIIIvVIfr}{\VERSE  Poussez des cris, car le jour du Seigneur est proche; il viendra comme un ravage du Seigneur. \EVERSE}
\newcommand{\isXIIIvVIIfr}{\VERSE  C'est pourquoi toutes les mains seront languissantes, et tout coeur d'homme se fondra et sera brisé. \EVERSE}
\newcommand{\isXIIIvVIIIfr}{\VERSE  Ils seront saisis de convulsions et de douleurs; ils souffriront comme une femme en travail; ils se regarderont l'un l'autre avec stupeur, et leurs visages seront enflammés. \EVERSE}
\newcommand{\isXIIIvIXfr}{\VERSE  Voici que vient le jour du Seigneur, jour cruel, plein d'indignation, de colère et de fureur, pour transformer la terre en désert, et pour en exterminer les pécheurs. \EVERSE}
\newcommand{\isXIIIvXfr}{\VERSE  Car les étoiles du ciel et leur splendeur ne répandront plus leur lumière; le soleil à son lever se couvrira de ténèbres, et la lune ne fera plus luire sa clarté. \EVERSE}
\newcommand{\isXIIIvXIfr}{\VERSE  Je viendrai châtier les crimes du monde et l'iniquité des impies; Je ferai cesser l'orgueil des infidéles, et J'humilierai l'arrogance des forts. \EVERSE}
\newcommand{\isXIIIvXIIfr}{\VERSE  L'homme sera plus rare que l'or, il sera plus précieux que l'or le plus pur. \EVERSE}
\newcommand{\isXIIIvXIIIfr}{\VERSE  C'est pourquoi J'ébranlerai le ciel, et la terre sortira de sa place, à cause de l'indignation du Seigneur des armées, et du jour de Sa colère et de Sa fureur. \EVERSE}
\newcommand{\isXIIIvXIVfr}{\VERSE  Alors Babylone sera comme un daim qui s'enfuit, et comme des brebis que personne ne rassemble. Chacun retournera vers son peuple, et ils fuiront tous dans leur pays. \EVERSE}
\newcommand{\isXIIIvXVfr}{\VERSE  Quconque sera trouvé sera tué, et tous ceux que l'on rencontrera tomberont par le glaive; \EVERSE}
\newcommand{\isXIIIvXVIfr}{\VERSE  leurs enfants seront écrasés sous leurs yeux; leurs maisons seront pillées, et leur femmes violées. \EVERSE}
\newcommand{\isXIIIvXVIIfr}{\VERSE  Je vais susciter contre eux les Mèdes, qui ne chercheront point d'argent, et qui ne voudront pas d'or; \EVERSE}
\newcommand{\isXIIIvXVIIIfr}{\VERSE  mais de leurs flèches ils perceront les petits enfants, ils n'auront pas compassion du fruit des entrailles, et leur oeil n'épargnera pas les enfants. \EVERSE}
\newcommand{\isXIIIvXIXfr}{\VERSE  Et cette Babylone, glorieuse parmi les royaumes, orgueil éclatant des Chaldéens, sera comme Sodome et Gomorrhe, que le Seigneur a renversées. \EVERSE}
\newcommand{\isXIIIvXXfr}{\VERSE  Elle ne sera plus jamais habitée, et elle ne sera pas rebâtie dans la suite des siècles; les Arabes n'y dresseront pas leurs tentes, et les pasteurs ne s'y reposeront pas. \EVERSE}
\newcommand{\isXIIIvXXIfr}{\VERSE  Mais les bêtes sauvages s'y retireront, ses maisons seront remplies de dragons, les autruches y viendront habiter, et les satyres y danseront; \EVERSE}
\newcommand{\isXIIIvXXIIfr}{\VERSE  les hiboux hurleront à l'envi dans ses palais, et les sirènes dans ses maisons de délices. \EVERSE}
\newcommand{\isXIVvIfr}{\VERSE  Son temps est proche, et ses jours ne seront pas prolongés; car le Seigneur aura pitié de Jacob, Il Se choisira encore des amis dans Israël, et Il les fera reposer dans leur pays; les étrangers se joindront à eux, et s'attacheront à la maison de Jacob. \EVERSE}
\newcommand{\isXIVvIIfr}{\VERSE  Les peuples les prendront, et les reconduiront dans leur pays; et la maison d'Israël les possédera dans la terre du Seigneur comme serviteurs et comme servantes; ceux qui les avaient pris seront leurs captifs, et ils s'assujettiront leurs oppresseurs. \EVERSE}
\newcommand{\isXIVvIIIfr}{\VERSE  En ce temps-là, lorsque le Seigneur t'aura donné du repos après ta fatigue et ton agitation, et après la dure servitude qui t'avait été imposée, \EVERSE}
\newcommand{\isXIVvIVfr}{\VERSE  tu prononceras ce discours figuré contre le roi de Babylone, et tu diras : Qu'est devenu le tyran? Comment le tribut a-t-il cessé? \EVERSE}
\newcommand{\isXIVvVfr}{\VERSE  Le Seigneur a brisé le bâton des impies, la verge des dominateurs, \EVERSE}
\newcommand{\isXIVvVIfr}{\VERSE  celui qui dans son indignation frappait les peuples d'une plaie incurable, celui qui s'assujettissait les nations dans sa fureur, et qui les persécutait cruellement. \EVERSE}
\newcommand{\isXIVvVIIfr}{\VERSE  Toute la terre est dans le repos et dans la paix, elle est dans la joie et dans l'allégresse; \EVERSE}
\newcommand{\isXIVvVIIIfr}{\VERSE  les sapins mêmes et les cèdres du Liban se sont réjouis de ta perte : Depuis que tu es mort, disent-ils, il ne monte personne pour nous abattre. \EVERSE}
\newcommand{\isXIVvIXfr}{\VERSE  Le séjour des morts s'est ému pour t'accueillir à ton arrivée; il a fait lever les géants pour toi. Tous les princes de la terre, tous les princes des nations se sont levés de leurs trônes. \EVERSE}
\newcommand{\isXIVvXfr}{\VERSE  Tous prennent la parole; pour te dire : Toi aussi, tu as été blessé comme nous, tu es devenu semblable à nous! \EVERSE}
\newcommand{\isXIVvXIfr}{\VERSE  Ton orgueil a été précipité dans les enfers; ton cadavre est tombé à terre; sous toi est une couche de vers, et les vers seront ton vêtement. \EVERSE}
\newcommand{\isXIVvXIIfr}{\VERSE  Comment es-tu tombé du ciel, Lucifer, toi qui te levais si brillant le matin? comment as-tu été renversé sur la terre, toi qui frappais les nations? \EVERSE}
\newcommand{\isXIVvXIIIfr}{\VERSE  qui disais en ton coeur : Je monterai au ciel, j'établirai mon trône au-dessus des astres de Dieu, je m'assiérai sur la montagne de l'alliance, aux côtés de l'aquilon; \EVERSE}
\newcommand{\isXIVvXIVfr}{\VERSE  je monterai sur le sommet des nues, je serai semblable au Très-Haut? \EVERSE}
\newcommand{\isXIVvXVfr}{\VERSE  Mais tu as été précipité dans l'enfer, jusqu'au plus profond des abîmes. \EVERSE}
\newcommand{\isXIVvXVIfr}{\VERSE  Ceux qui te verront se pencheront vers toi, et t'envisageront : Est-ce là cet homme qui a fait trembler la terre, qui a ébranlé les royaumes, \EVERSE}
\newcommand{\isXIVvXVIIfr}{\VERSE  qui a fait du monde un désert, qui en a détruit les villes, et qui n'a pas ouvert la prison à ceux qu'il avait enchaînés? \EVERSE}
\newcommand{\isXIVvXVIIIfr}{\VERSE  Tous les rois des nations sont morts avec gloire, et chacun d'eux a son tombeau; \EVERSE}
\newcommand{\isXIVvXIXfr}{\VERSE  mais toi, tu as été jeté loin de ton sépulcre comme un tronc inutile et tout souillé tu as été enveloppé dans la foule de ceux qui ont été tués par l'épée et qu'on fait descendre au fond de la fosse, comme un cadavre pourri. \EVERSE}
\newcommand{\isXIVvXXfr}{\VERSE  Tu ne leur seras pas uni dans le sépulcre, car tu as ruiné ton royaume; et tu as fait périr ton peuple. On ne parlera plus jamais de la race des scélérats. \EVERSE}
\newcommand{\isXIVvXXIfr}{\VERSE  Préparez ses fils pour le massacre, à cause de l'iniquité de leurs pères; ils ne s'élèveront point, ils ne posséderont pas la terre, et ils ne rempliront point de villes la face du monde. \EVERSE}
\newcommand{\isXIVvXXIIfr}{\VERSE  Je M'élèverai contre eux, dit le Seigneur des armées; Je perdrai le nom de Babylone, et ses rejetons, et ses descendants, et toute sa race, dit le Seigneur; \EVERSE}
\newcommand{\isXIVvXXIIIfr}{\VERSE  J'en ferai la demeure des hérissons, et des marais pleins d'eau, et Je la balayerai avec le balai de la destruction, dit le Seigneur des armées. \EVERSE}
\newcommand{\isXIVvXXIVfr}{\VERSE  Le Seigneur des armées a juré; en disant : Oui, ce que J'ai pensé arrivera, et ce que J'ai arrêté dans Mon esprit \EVERSE}
\newcommand{\isXIVvXXVfr}{\VERSE  s'exécutera; Je briserai l'Assyrien dans Mon pays, et Je le foulerai aux pieds sur Mes montagnes; et son joug leur sera enlevé, et son fardeau sera enlevé de leurs épaules. \EVERSE}
\newcommand{\isXIVvXXVIfr}{\VERSE  C'est là le dessein que J'ai formé au sujet de toute la terre; et voilà la main qui est étendue sur toutes les nations. \EVERSE}
\newcommand{\isXIVvXXVIIfr}{\VERSE  Car le Seigneur des armées l'a ordonné; qui pourra s'y opposer? Sa main est étendue; qui la détournera? \EVERSE}
\newcommand{\isXIVvXXVIIIfr}{\VERSE  L'année de la mort du roi Achaz, cet oracle fut prononcé. \EVERSE}
\newcommand{\isXIVvXXIXfr}{\VERSE  Ne te réjouis pas, terre des Philistins, de ce que la verge de celui qui te frappait a été brisée; car de la racine du serpent il sortira un basilic, et ce qui en naîtra dévorera les oiseaux. \EVERSE}
\newcommand{\isXIVvXXXfr}{\VERSE  Alors les plus pauvres seront nourris, et les indigents se reposeront en sécurité; et Je ferai mourir ta racine par la faim, et Je perdrai ce qui restera de toi. \EVERSE}
\newcommand{\isXIVvXXXIfr}{\VERSE  Porte, pousse des hurlements; ville, fais retentir des cris : tout le pays des Philistins est renversé; car de l'aquilon vient une fumée, et nul ne pourra échapper à ses bataillons. \EVERSE}
\newcommand{\isXIVvXXXIIfr}{\VERSE  Et que répondra-t-on aux envoyés de la nation? Que le Seigneur a fondé Sion, et que les pauvres de Son peuple espéreront en Lui. \EVERSE}
\newcommand{\isXVvIfr}{\VERSE  Oracle contre Moab. En une nuit, Arde Moab a été saccagée, elle est anéantie; en une nuit la muraille de Moab a été renversée, elle est anéantie. \EVERSE}
\newcommand{\isXVvIIfr}{\VERSE  La maison royale et Dibon sont montées sur les hauts lieux, pour pleurer la perte de Nabo et de Médaba. Moab pousse des cris; toutes les têtes sont rasées et toutes les barbes sont coupées. \EVERSE}
\newcommand{\isXVvIIIfr}{\VERSE  Ils sont revêtus de sacs dans les rues; sur les toits et dans les places tout se lamente et fond en larmes. \EVERSE}
\newcommand{\isXVvIVfr}{\VERSE  Hésébon et Eléalé poussent des cris, leur voix se fait entendre jusqu'à Jasa; les vaillants de Moab se lamentent sur cela; son âme gémit sur elle-même. \EVERSE}
\newcommand{\isXVvVfr}{\VERSE  Mon coeur poussera des cris sur Moab; ses défenseurs vont jusqu'à Segor, génisse de trois ans. On monte en pleurant par la colline de Luith, et on jette des cris de détresse sur le chemin d'Oronaïm. \EVERSE}
\newcommand{\isXVvVIfr}{\VERSE  Les eaux de Nemrim se changeront en un désert; l'herbe est desséchée, le gazon est détruit, et toute verdure a disparu. \EVERSE}
\newcommand{\isXVvVIIfr}{\VERSE  La grandeur de leurs châtiments égale celle de leurs crimes; les ennemis les mèneront au torrent des Saules. \EVERSE}
\newcommand{\isXVvVIIIfr}{\VERSE  Les cris font le tour des confins de Moab; ses plaintes retentissent jusqu'à Gallion, et ses hurlements jusqu'au puits d'Elim. \EVERSE}
\newcommand{\isXVvIXfr}{\VERSE  Car les eaux  de Dibon sont remplies de sang, et J'enverrai à Dibon de nouveaux malheurs; un lion contre les échappés de Moab et contre les restes du pays. \EVERSE}
\newcommand{\isXVIvIfr}{\VERSE  Seigneur, envoyez l'Agneau dominateur de la terre, de la pierre du désert à la montagne de la fille de Sion. \EVERSE}
\newcommand{\isXVIvIIfr}{\VERSE  Et alors, comme un oiseau qui s'en fuit, et comme les petits qui s'envolent de leur nid, telles seront, au passage de l'Arnon, les filles de Moab. \EVERSE}
\newcommand{\isXVIvIIIfr}{\VERSE  Prends conseil, réunis des assemblées; rends ton ombre, en plein midi, aussi sombre que la nuit; cache les fugitifs et ne trahis pas ceux qui sont errants. \EVERSE}
\newcommand{\isXVIvIVfr}{\VERSE  Mes exilés habiteront auprès de toi; pour Moab, sois un refuge contre le dévastateur; car la poussière a trouvé sa fin, ce misérable n'est plus, celui qui foulait le pays sous ses pieds a disparu. \EVERSE}
\newcommand{\isXVIvVfr}{\VERSE  Et le trône s'affermira par la miséricorde, et on y verra siéger avec fidélité, dans la tente de David, un juge qui cherchera le droit, et qui rendra promptement la justice. \EVERSE}
\newcommand{\isXVIvVIfr}{\VERSE  Nous avons appris l'orgueil de Moab, il est étrangement superbe; son orgueil, son arrogance et sa fureur dépassent sa force. \EVERSE}
\newcommand{\isXVIvVIIfr}{\VERSE  C'est pourquoi Moab criera sur Moab, il criera tout entier; à ceux qui se réjouissent sur leurs murailles de briques, annoncez leurs malheurs. \EVERSE}
\newcommand{\isXVIvVIIIfr}{\VERSE  Car les environs d'Hésébon sont déserts; les princes des nations ont coupé la vigne de Sabama; ses branches se sont étendues jusqu'à Jazer, elles ont couru dans le désert, et ce qui est resté de ses rejetons a passé au delà de la mer. \EVERSE}
\newcommand{\isXVIvIXfr}{\VERSE  C'est pourquoi Je pleurerai la vigne de Sabama avec les pleurs de Jazer; Je vous arroserai de mes larmes, Hésébon et Eléalé, parce que l'ennemi s'est jeté avec de grands cris sur vos vignes et sur vos moissons, et les a foulées aux pieds. \EVERSE}
\newcommand{\isXVIvXfr}{\VERSE  La joie et l'allégresse disparaîtront des campagnes, et dans les vignes il n'y aura plus d'allégresse ni de jubilation. Ceux qui avaient coutume de fouler le vin dans le pressoir ne le fouleront plus; J'ai fait taire la voix de ceux qui pressuraient. \EVERSE}
\newcommand{\isXVIvXIfr}{\VERSE  C'est pourquoi Mon coeur frémit sur Moab comme une harpe, et Mmes entrailles gémissent sur les murailles de briques. \EVERSE}
\newcommand{\isXVIvXIIfr}{\VERSE  Et il arriva que Moab, après s'être fatigué sur ses hauts lieux, entrera dans ses sanctuaires pour prier, et il ne pourra rien obtenir. \EVERSE}
\newcommand{\isXVIvXIIIfr}{\VERSE  Telle est la parole que le Seigneur a prononcée sur Moab depuis longtemps. \EVERSE}
\newcommand{\isXVIvXIVfr}{\VERSE  Et maintenant voici ce que dit le Seigneur : Dans trois années comme les années d'un mercenaire, la gloire de Moab sera détruite avec tout son peuple nombreux, et ce qui restera sera faible, peu de chose, nullement considérable. \EVERSE}
\newcommand{\isXVIIvIfr}{\VERSE  Oracle contre Damas. Voici que Damas va cesser d'être une ville, et elle sera comme un monceau de pierres en ruines. \EVERSE}
\newcommand{\isXVIIvIIfr}{\VERSE  Les villes d'Aroër seront abandonnées aux troupeaux, et ils s'y reposeront sans que personne ne les effraye. \EVERSE}
\newcommand{\isXVIIvIIIfr}{\VERSE  Tout appui sera enlevé à Ephraïm, et le royaume à Damas; et les restes de la Syrie seront comme la gloire des fils d'Israël, dit le Seigneur des armées. \EVERSE}
\newcommand{\isXVIIvIVfr}{\VERSE  En ce jour, la gloire de Jacob sera affaiblie, et la graisse de sa chair disparaîtra. \EVERSE}
\newcommand{\isXVIIvVfr}{\VERSE  Il sera comme celui qui recueille dans la moisson ce qui est resté, et dont le bras ramasse les épis, et comme celui qui cherche des épis dans la vallée de Raphaïm. \EVERSE}
\newcommand{\isXVIIvVIfr}{\VERSE  Ce qui restera d'Israël sera comme une grappe de raisin, et comme un olivier qu'on secoue et dont il reste deux ou trois olives au bout d'une branche, ou quatre ou cinq au haut de l'arbre, dit le Seigneur, le Dieu d'Israël. \EVERSE}
\newcommand{\isXVIIvVIIfr}{\VERSE  En ce jour-là l'homme s'abaissera devant son Créateur, et ses yeux regarderont vers le Saint d'Israël; \EVERSE}
\newcommand{\isXVIIvVIIIfr}{\VERSE  et il ne s'abaissera plus devant les autels qu'avaient construits ses mains; il ne regardera plus les bois et les temples des idoles, que ses doigts avaient préparés. \EVERSE}
\newcommand{\isXVIIvIXfr}{\VERSE  En ce jour-là ses villes fortes seront abandonnées comme les charrues et les moissons qui furent laissées à l'approche des fils d'Israël, et tu seras un pays désert. \EVERSE}
\newcommand{\isXVIIvXfr}{\VERSE  Parce que tu as oublié le Dieu de ton salut, et que tu ne t'es pas souvenue de ton puissant protecteur, tu planteras de bon plant, et tu sèmeras des graines étrangères; \EVERSE}
\newcommand{\isXVIIvXIfr}{\VERSE  et ce que tu auras planté ne produira que des fruits sauvages; ta semence fleurira dès le matin, mais la récolte a disparu au moment d'en jouir, et la douleur est grande. \EVERSE}
\newcommand{\isXVIIvXIIfr}{\VERSE  Malheur à cette multitude de peuples nombreux qui retentit comme le bruit de la mer; tumulte de foule, semblable au bruit des eaux puissantes. \EVERSE}
\newcommand{\isXVIIvXIIIfr}{\VERSE  Les peuples retentiront comme retentissent des eaux qui débordent, Dieu les menacera, et ils fuiront au loin; ils seront emportés comme la poussière des montagnes au souffle du vent, et comme un tourbillon enlevé par la tempête. \EVERSE}
\newcommand{\isXVIIvXIVfr}{\VERSE  Le soir c'était l'épouvante, et au point du jour ils ne seront plus. Voilà le partage de ceux qui nous ont dévastés, et le sort de ceux qui nous pillent. \EVERSE}
\newcommand{\isXVIIIvIfr}{\VERSE  Malheur à la terre où retentit le bruit des ailes, qui est au delà des fleuves d'Ethiopie, \EVERSE}
\newcommand{\isXVIIIvIIfr}{\VERSE  qui envoie des messagers sur la mer et dans des barques de jonc sur les eaux. Allez, messagers rapides, vers une nation divisée et déchirée; vers un peuple terrible, le plus terrible de tous; vers une nation qui attend et qui est foulée aux pieds, dont la terre est ravagée par des fleuves. \EVERSE}
\newcommand{\isXVIIIvIIIfr}{\VERSE  Vous tous, habitants du monde, vous qui demeurez sur la terre, lorsque l'étendard sera élevé sur les montagnes, vous le verrez, et vous entendrez le bruit de la trompette. \EVERSE}
\newcommand{\isXVIIIvIVfr}{\VERSE  Car voici ce que me dit le Seigneur : Je Me tiendrai en repos, et Je contemplerai de Ma demeure, comme une lumière aussi brillante que le soleil en plein midi, et comme un nuage de rosée au temps de la moisson. \EVERSE}
\newcommand{\isXVIIIvVfr}{\VERSE  Car la vigne fleurira toute avant le temps; elle germera sans pouvoir mûrir; ses rejetons seront coupés avec la faux, et ce qui en restera sera retranché et rejeté. \EVERSE}
\newcommand{\isXVIIIvVIfr}{\VERSE  Ils seront tous abandonnés aux oiseaux des montagnes et aux bêtes de la terre; les oiseaux y demeureront pendant tout l'été, et toutes les bêtes de la terre y passeront l'hiver. \EVERSE}
\newcommand{\isXVIIIvVIIfr}{\VERSE  En ce temps-là des offrandes seront apportées au Seigneur des armées de la part d'un peuple divisé et déchiré, d'un peuple terrible, le plus terrible de tous, d'une nation qui attend et qui est foulée aux pieds, dont la terre est ravagée par des fleuves; elles seront apportées au lieu où réside le nom du Seigneur des armées, à la montagne de Sion. \EVERSE}
\newcommand{\isXIXvIfr}{\VERSE  Oracle contre l'Egypte. Le Seigneur montera sur un léger nuage, et Il entrera en Egypte, et les idoles de l'Egypte seront ébranlées devant Lui, et le coeur de l'Egypte se fondra au milieu d'elle. \EVERSE}
\newcommand{\isXIXvIIfr}{\VERSE  Je lancerai les Egyptiens contre les Egyptiens; et le frère combattra contre son frère, l'ami contre son ami, la ville contre la ville, et le royaume contre le royaume. \EVERSE}
\newcommand{\isXIXvIIIfr}{\VERSE  L'esprit de l'Egypte se brisera au milieu d'elle, et J'anéantirai son conseil; alors ils consulteront leurs idoles, leurs devins, leurs sorciers et leurs magiciens. \EVERSE}
\newcommand{\isXIXvIVfr}{\VERSE  Et Je livrerai l'Egypte entre les mains de maîtres cruels, et un roi violent dominera sur eux, dit le Seigneur, le Dieu des armées. \EVERSE}
\newcommand{\isXIXvVfr}{\VERSE  La mer se desséchera, et le fleuve deviendra sec et aride. \EVERSE}
\newcommand{\isXIXvVIfr}{\VERSE  Les rivières tariront, les ruisseaux de l'Egypte baisseront et se sécheront, les roseaux et les joncs se faneront. \EVERSE}
\newcommand{\isXIXvVIIfr}{\VERSE  Le lit des ruisseaux sera sec à sa source même, et tous les grains semés le long de ses eaux se sécheront et périront. \EVERSE}
\newcommand{\isXIXvVIIIfr}{\VERSE  Les pêcheurs seront désolés, tous ceux qui jettent l'hameçon dans le fleuve pleureront, et ceux qui étendent le filet sur la surface de ses eaux tomberont en défaillance. \EVERSE}
\newcommand{\isXIXvIXfr}{\VERSE  Ceux qui travaillaient le lin, qui le peignaient, et qui en tissaient des étoffes fines, seront dans la confusion. \EVERSE}
\newcommand{\isXIXvXfr}{\VERSE  Les lieux arrosés d'eaux sécheront, et tous ceux qui faisaient des fosses pour y prendre du poisson seront confondus. \EVERSE}
\newcommand{\isXIXvXIfr}{\VERSE  Les princes de Tanis sont fous, ces sages conseillers du Pharaon ont donné un conseil insensé. Comment dites-vous au Pharaon : Je suis le fils des sages, le fils des anciens rois? \EVERSE}
\newcommand{\isXIXvXIIfr}{\VERSE  Où sont maintenant tes sages? Qu'ils t'annoncent et qu'ils te prédisent ce que le Seigneur des armées a résolu sur l'Egypte. \EVERSE}
\newcommand{\isXIXvXIIIfr}{\VERSE  Les princes de Tanis sont devenus insensés, les princes de Memphis ont perdu leur force; ils ont séduit l'Egypte, l'angle de ses peuples. \EVERSE}
\newcommand{\isXIXvXIVfr}{\VERSE  Le Seigneur a répandu au milieu d'elle un esprit de vertige, et ils ont fait errer l'Egypte dans toutes ses oeuvres, comme erre un homme ivre et qui vomit. \EVERSE}
\newcommand{\isXIXvXVfr}{\VERSE  L'Egypte sera dans l'incertitude de ce qu'elle doit faire: la tête et la queue, celui qui commande et celui qui obéit. \EVERSE}
\newcommand{\isXIXvXVIfr}{\VERSE  En ce jour-là les Egyptiens deviendront comme des femmes; ils s'étonneront, et ils trembleront, en voyant s'agiter la main du Seigneur des armées, qu'Il agitera sur eux. \EVERSE}
\newcommand{\isXIXvXVIIfr}{\VERSE  Alors le pays de Juda deviendra l'effroi de l'Egypte, et quiconque se souviendra de lui tremblera, à la vue des desseins que le Seigneur des armées a formés contre l'Egypte. \EVERSE}
\newcommand{\isXIXvXVIIIfr}{\VERSE  En ce jour-là il y aura cinq villes dans l'Egypte qui parleront la langue de Chanaan, et qui jureront par le Seigneur des armées. L'une d'elles sera appelée la ville du Soleil. \EVERSE}
\newcommand{\isXIXvXIXfr}{\VERSE  En ce jour-là il y aura un autel du Seigneur au milieu de l'Egypte, et un monument au Seigneur à la frontière. \EVERSE}
\newcommand{\isXIXvXXfr}{\VERSE  Ce sera un signe et un témoignage pour le Seigneur des armées dans la terre d'Egypte; car ils crieront au Seigneur en face de l'oppresseur, et Il leur enverra un sauveur et un défenseur qui les délivrera. \EVERSE}
\newcommand{\isXIXvXXIfr}{\VERSE  Alors le Seigneur sera connu de l'Egypte, et les Egyptiens connaîtront le Seigneur en ce jour-là; il L'honoreront par des sacrifices et des oblations; ils feront des voeux au Seigneur, et ils les accompliront. \EVERSE}
\newcommand{\isXIXvXXIIfr}{\VERSE  Le Seigneur frappera l'Egypte d'une plaie, et Il la guérira; et ils reviendront au Seigneur, et Il leur deviendra favorable, et Il les guérira. \EVERSE}
\newcommand{\isXIXvXXIIIfr}{\VERSE  En ce jour-là il y aura une route d'Egypte en Assyrie : les Assyriens entreront en Egypte, et les Egyptiens en Assyrie, et les Egyptiens serviront les Assyriens. \EVERSE}
\newcommand{\isXIXvXXIVfr}{\VERSE  En ce jour-là Israël sera, lui troisième, uni aux Egyptiens et aux Assyriens; la bénédiction sera au milieu de la terre \EVERSE}
\newcommand{\isXIXvXXVfr}{\VERSE  que le Seigneur a bénie, en disant : Mon peuple d'Egypte est béni, et l'Assyrien est l'oeuvre de Mes mains; mais Israël est Mon héritage. \EVERSE}
\newcommand{\isXXvIfr}{\VERSE  L'année où Tharthan, envoyé par Sargon, roi des Assyriens, vint à Azot, l'assiégea et la prit; \EVERSE}
\newcommand{\isXXvIIfr}{\VERSE  cette année-là, le Seigneur parla à Isaïe, fils d'Amos, et lui dit : Va, détache le sac de tes reins, et ôte tes souliers de tes pieds. Il fit ainsi, et il alla nu et déchaussé. \EVERSE}
\newcommand{\isXXvIIIfr}{\VERSE  Alors le Seigneur dit : De même que Mon serviteur Isaïe a marché nu et déchaussé, pour être un signe et un présage de trois ans pour l'Egypte et pour l'Ethiopie, \EVERSE}
\newcommand{\isXXvIVfr}{\VERSE  ainsi le roi des Assyriens emmèrera d'Egypte et d'Ethiopie, captifs et exilés, les jeunes gens et les vieillards, nus et déchaussés, les reins découverts, à la honte de l'Egypte. \EVERSE}
\newcommand{\isXXvVfr}{\VERSE  Alors les Juifs seront dans l'effroi, et ils rougiront d'avoir mis leur espérance dans l'Ethiopie, et leur gloire dans l'Egypte. \EVERSE}
\newcommand{\isXXvVIfr}{\VERSE  Et les habitants de cette île diront en ce jour-là : C'était donc là notre espérance? Voilà ceux dont nous implorions le secours pour être délivrés du roi des Assyriens! Et comment pourrons-nous échapper nous-mêmes? \EVERSE}
\newcommand{\isXXIvIfr}{\VERSE  Oracle contre le désert de la mer. Comme s'avancent les tourbillons du midi, il vient du désert, d'une terre épouvantable. \EVERSE}
\newcommand{\isXXIvIIfr}{\VERSE  Une terrible vision m'a été révélée : Le perfide agit avec perfidie, et le dévastateur dévaste. Monte, Elam; Mède, assiège; Je vais mettre fin à ses gémissements. \EVERSE}
\newcommand{\isXXIvIIIfr}{\VERSE  C'est pourquoi mes reins sont saisis de douleur; l'angoisse me saisit, comme l'angoisse d'une femme en travail; ce que j'entends m'effraye, et ce que je vois m'épouvante. \EVERSE}
\newcommand{\isXXIvIVfr}{\VERSE  Mon coeur a défailli; les ténèbres m'ont stupéfié : Babylone, ma bien-aimée, devient un sujet d'effroi. \EVERSE}
\newcommand{\isXXIvVfr}{\VERSE  Dresse la table, contemple d'un poste élevé ceux qui mangent et qui boivent. Levez-vous, princes, prenez le bouclier. \EVERSE}
\newcommand{\isXXIvVIfr}{\VERSE  Car voici ce que m'a dit le Seigneur : Va, et place une sentinelle qui t'annoncera tout ce qu'elle verra. \EVERSE}
\newcommand{\isXXIvVIIfr}{\VERSE  Et elle vit un char conduit par deux cavaliers, des hommes montés sur des ânes, et des hommes montés sur des chameaux; et elle contempla soigneusement, avec grande attention. \EVERSE}
\newcommand{\isXXIvVIIIfr}{\VERSE  Puis elle cria comme un lion : Je suis au poste où m'a placé le Seigneur, et j'y demeure tout le jour; je monte ma garde, et j'y demeure les nuits. \EVERSE}
\newcommand{\isXXIvIXfr}{\VERSE  Et voici, l'homme qui conduisait le char s'approcha, et il prit la parole, et dit : Elle est tombée, elle est tombée, Babylone, et toutes les images de ses dieux ont été brisées à terre. \EVERSE}
\newcommand{\isXXIvXfr}{\VERSE  O mon grain trituré et les fils de mon aire, ce que j'ai appris du Seigneur des armées, du Dieu d'Israël, je vous l'ai annoncé. \EVERSE}
\newcommand{\isXXIvXIfr}{\VERSE  Oracle sur Duma. On me crie de Séir : Sentinelle, où en est-on de la nuit? Sentinelle, où en est-on de la nuit? \EVERSE}
\newcommand{\isXXIvXIIfr}{\VERSE  La sentinelle répond : Le matin vient, et la nuit aussi; si vous cherchez, cherchez; convertissez-vous, venez. \EVERSE}
\newcommand{\isXXIvXIIIfr}{\VERSE  Oracle sur l'Arabie. Vous dormirez le soir dans la forêt, dans les sentiers de Dédanim. \EVERSE}
\newcommand{\isXXIvXIVfr}{\VERSE  Venez au-devant de ceux qui ont soif, et portez-leur de l'eau, vous qui habitez la terre du midi; venez avec des pains au-devant des fugitifs; \EVERSE}
\newcommand{\isXXIvXVfr}{\VERSE  car ils fuient devant les glaives, devant le glaive menaçant, devant l'arc tendu et devant un rude combat. \EVERSE}
\newcommand{\isXXIvXVIfr}{\VERSE  Car ainsi m'a parlé le Seigneur : Encore une année, comme une année de mercenaire, et toute la gloire de Cédar sera détruite. \EVERSE}
\newcommand{\isXXIvXVIIfr}{\VERSE  Et le nombre des robustes archers de Cédar qui seront restés diminuera, car le Seigneur, le Dieu d'Israël, a parlé. \EVERSE}
\newcommand{\isXXIIvIfr}{\VERSE  Oracle sur la vallée de la vision. Qu'as-tu donc, que tu montes tout entière sur les toits, \EVERSE}
\newcommand{\isXXIIvIIfr}{\VERSE  ville tumultueuse, pleine de peuple, cité joyeuse? Tes morts n'ont pas péri par l'épée, et ils ne sont pas morts à la guerre. \EVERSE}
\newcommand{\isXXIIvIIIfr}{\VERSE  Tous tes princes ont fui ensemble, ils ont été durement enchaînés; tous ceux que l'ennemi a trouvés ont été liés ensemble, quoiqu'ils se fussent enfuis au loin. \EVERSE}
\newcommand{\isXXIIvIVfr}{\VERSE  C'est pourquoi j'ai dit : Eloignez-vous de moi, je pleurerai amèrement; n'insistez point pour me consoler sur la ruine de la fille de mon peuple; \EVERSE}
\newcommand{\isXXIIvVfr}{\VERSE  car c'est un jour de carnage, et d'écrasement, et de pleurs, que le Seigneur, le Dieu des armées, envoie dans la vallée de la vision; il perce la muraille et manifeste sa gloire sur la montagne. \EVERSE}
\newcommand{\isXXIIvVIfr}{\VERSE  Elam a pris son carquois, ses chars pour ses cavaliers, et il a détaché ses boucliers de la muraille. \EVERSE}
\newcommand{\isXXIIvVIIfr}{\VERSE  Tes plus belles vallées seront remplies de chars de guerre, et les cavaliers iront camper à tes portes. \EVERSE}
\newcommand{\isXXIIvVIIIfr}{\VERSE  Le voile de Juda sera enlevé, et tu visiteras en ce jour-là l'arsenal du palais et de la forêt. \EVERSE}
\newcommand{\isXXIIvIXfr}{\VERSE  Vous examinerez les brèches nombreuses de la cité de David, et vous recueillerez les eaux de la piscine inférieure; \EVERSE}
\newcommand{\isXXIIvXfr}{\VERSE  vous compterez les maisons de Jérusalem, et vous détruirez des maisons pour fortifier la muraille. \EVERSE}
\newcommand{\isXXIIvXIfr}{\VERSE  Vous ferez un réservoir entre les deux murs, auprès des eaux de la piscine ancienne; et vous ne lèverez pas les yeux vers celui qui a fait cela, et vous ne regarderez pas celui qui l'a préparé de loin. \EVERSE}
\newcommand{\isXXIIvXIIfr}{\VERSE  Et le Seigneur, le Dieu des armées, vous invitera en ce jour-là aux larmes et aux gémissements, à vous raser la téte et à vous revêtir de sacs; \EVERSE}
\newcommand{\isXXIIvXIIIfr}{\VERSE  et au lieu de cela voici la gaieté et la joie, on tue des veaux et on égorge des moutons. On mange de la viande et on boit du vin : Mangeons et buvons, car demain nous mourrons. \EVERSE}
\newcommand{\isXXIIvXIVfr}{\VERSE  Et la voix du  Seigneur des armées s'est fait entendre à mes oreilles : Non, cette iniquité ne vous sera pas pardonnée jusqu'à la mort, dit le Seigneur, le Dieu des armées. \EVERSE}
\newcommand{\isXXIIvXVfr}{\VERSE  Voici ce que dit le Seigneur, le Dieu des armées : Va trouver celui qui habite dans le tabernacle, Sobna, préfet du temple, et tu lui diras: \EVERSE}
\newcommand{\isXXIIvXVIfr}{\VERSE  Que fais-tu ici, ou qui es-tu ici, toi qui t'es creusé ici un sépulcre, qui t'es creusé un monument avec tant de soin, sur un lieu élevé, et qui t'es taillé une demeure dans la pierre? \EVERSE}
\newcommand{\isXXIIvXVIIfr}{\VERSE  Voici que le Seigneur te fera emporter comme on emporte un coq, et Il t'enlèvera comme un manteau. \EVERSE}
\newcommand{\isXXIIvXVIIIfr}{\VERSE  Il te couronnera d'une couronne de tribulation, Il te jettera comme une balle sur une terre large et spacieuse; tu mourras là, et là sera ton char magnifique, ô honte de la maison de ton maître. \EVERSE}
\newcommand{\isXXIIvXIXfr}{\VERSE  Je te chasserai de ton poste, et Je te déposerai de ton ministère. \EVERSE}
\newcommand{\isXXIIvXXfr}{\VERSE  Et en ce jour-là J'appellerai mon serviteur Eliacim, fils d'Helcias; \EVERSE}
\newcommand{\isXXIIvXXIfr}{\VERSE  Je le revêtirai de ta tunique, Je le ceindrai de ta ceinture, et Je remettrai ta puissance entre ses mains, et il sera comme un père pour les habitants de Jérusalem et pour la maison de Juda. \EVERSE}
\newcommand{\isXXIIvXXIIfr}{\VERSE  Je mettrai sur son épaule la clef de la maison de David; il ouvrira, et personne ne fermera, et il fermera, et personne n'ouvrira. \EVERSE}
\newcommand{\isXXIIvXXIIIfr}{\VERSE  Je l'enfoncerai comme un pieu dans un lieu solide, et il sera comme un trône de gloire pour la maison de son père. \EVERSE}
\newcommand{\isXXIIvXXIVfr}{\VERSE  Et toute la gloire de la maison de son père sera suspendue sur lui : on y mettra des vases de divers genres, toute sorte de petits instruments, depuis les coupes jusqu'aux instruments de musique. \EVERSE}
\newcommand{\isXXIIvXXVfr}{\VERSE  En ce jour-là, dit le Seigneur des armées, le pieu qui avait été enfoncé dans un lieu solide sera arraché; il sera brisé et il tombera, et tout ce qui y était suspendu périra, car le Seigneur a parlé. \EVERSE}
\newcommand{\isXXIIIvIfr}{\VERSE  Oracle sur Tyr. Hurlez, vaisseaux de la mer, car le lieu d'où les navires avaient coutume de venir a été déruit, c'est du pays de Céthim que la nouvelle leur en est venue. \EVERSE}
\newcommand{\isXXIIIvIIfr}{\VERSE  Soyez muets, habitants de l'île; les marchands de Sidon, qui parcourent la mer, te remplissaient. \EVERSE}
\newcommand{\isXXIIIvIIIfr}{\VERSE  Sur les vastes eaux la semence du Nil, les moissons du fleuve étaient sa nourriture; et elle était devenue le marché des nations. \EVERSE}
\newcommand{\isXXIIIvIVfr}{\VERSE  Rougis de honte, Sidon, car ainsi parle la mer, la force de la mer : Je n'ai pas conçu, je n'ai pas enfanté, je n'ai pas nourri de jeunes gens, et je n'ai point élevé de jeunes filles. \EVERSE}
\newcommand{\isXXIIIvVfr}{\VERSE  Lorsque la nouvelle aura passé en Egypte, on sera saisi de douleur en apprenant la ruine de Tyr. \EVERSE}
\newcommand{\isXXIIIvVIfr}{\VERSE  Traversez les mers, poussez des hurlements, habitants de l'île. \EVERSE}
\newcommand{\isXXIIIvVIIfr}{\VERSE  N'est-ce pas là votre ville, qui se glorifiait de son antiquité depuis les anciens jours? Ses pieds la conduisent au loin sur la terre étrangère. \EVERSE}
\newcommand{\isXXIIIvVIIIfr}{\VERSE  Qui a pensé cela contre Tyr, autrefois couronnée, dont les marchands étaient des princes, dont les trafiquants étaient les nobles de la terre? \EVERSE}
\newcommand{\isXXIIIvIXfr}{\VERSE  C'est le Seigneur des armées qui a pensé cela, pour renverser l'orgueil de toute gloire, et pour faire tomber dans l'ignominie tous les nobles de la terre. \EVERSE}
\newcommand{\isXXIIIvXfr}{\VERSE  Parcours ton pays comme un fleuve, fille de la mer; tu n'as plus de ceinture. \EVERSE}
\newcommand{\isXXIIIvXIfr}{\VERSE  Le Seigneur a étendu Sa main sur la mer, Il a ébranlé les royaumes; Il a donné Ses ordres contre Chanaan, pour détruire ses héros; \EVERSE}
\newcommand{\isXXIIIvXIIfr}{\VERSE  et Il a dit : Tu ne te glorifieras plus à l'avenir, vierge déshonorée, fille de Sidon; lève-toi, passe à Céthim; même là tu ne trouveras pas de repos. \EVERSE}
\newcommand{\isXXIIIvXIIIfr}{\VERSE  Vois le pays des Chaldéens; il n'y eut jamais un tel peuple. Les Assyriens l'avaient fondé; on a emmené captifs ses plus robustes, on a renversé ses maisons, et on a fait d'elle une ruine. \EVERSE}
\newcommand{\isXXIIIvXIVfr}{\VERSE  Hurlez, vaisseaux de la mer, parce que votre force a été détruite. \EVERSE}
\newcommand{\isXXIIIvXVfr}{\VERSE  En ce jour-là, ô Tyr, tu seras dans l'oubli pendant soixante-dix ans, comme les jours d'un même roi; et après soixante-dix ans, Tyr sera comme la courtisane dont parle la chanson: \EVERSE}
\newcommand{\isXXIIIvXVIfr}{\VERSE  Prends la harpe, parcours la ville, courtisane qu'on oublie; chante bien, répète tes chants, afin qu'on se souvienne de toi. \EVERSE}
\newcommand{\isXXIIIvXVIIfr}{\VERSE  Après soixante-dix ans, le Seigneur visitera Tyr, et il la ramènera à son trafic, et elle se prostituera de nouveau à tous les royaumes de la terre, sur la face du globe. \EVERSE}
\newcommand{\isXXIIIvXVIIIfr}{\VERSE  Mais son gain et ses bénéfices seront consacrés au Seigneur; ils ne seront pas enfouis ni mis en réserve, mais son gain sera pour ceux qui habitent devant le Seigneur, afin qu'ils en soient nourris jusqu'à satiété, et qu'ils en soient revêtus jusqu'à leur vieillesse. \EVERSE}
\newcommand{\isXXIVvIfr}{\VERSE  Voici que le Seigneur dévastera la terre; Il la dépouillera, Il en affligera la face, et Il en dispersera les habitants. \EVERSE}
\newcommand{\isXXIVvIIfr}{\VERSE  Alors le prêtre sera comme le peuple, le maître comme son esclave, la maîtresse comme sa servante, celui qui vend comme celui qui achète, celui qui emprunte comme celui qui prête, et celui qui doit comme celui qui redemande ce qu'il a prêté. \EVERSE}
\newcommand{\isXXIVvIIIfr}{\VERSE  La terre sera entièrement dévastée et livrée au pillage; car c'est le Seigneur qui l'a décrété. \EVERSE}
\newcommand{\isXXIVvIVfr}{\VERSE  La terre est dans les larmes, elle fond, elle tombe en défaillance; le monde périt, la grandeur du peuple de la terre est abaissée. \EVERSE}
\newcommand{\isXXIVvVfr}{\VERSE  La terre a été infectée par ses habitants, car ils ont violé les lois, ils ont changé le droit, ils ont rompu l'alliance éternelle. \EVERSE}
\newcommand{\isXXIVvVIfr}{\VERSE  C'est pourquoi la malédiction dévorera la terre, ses habitants s'abandonneront au péché, ceux qui la cultivent seront insensés, et il n'y demeurera que très peu d'hommes. \EVERSE}
\newcommand{\isXXIVvVIIfr}{\VERSE  La vendange pleure, la vigne languit, tous ceux qui avaient le coeur joyeux sont dans les larmes. \EVERSE}
\newcommand{\isXXIVvVIIIfr}{\VERSE  La joie des tambourins a cessé, les cris de réjouissance ont pris fin, la harpe a fait taire ses doux accords. \EVERSE}
\newcommand{\isXXIVvIXfr}{\VERSE  On ne boira plus le vin en chantant; les liqueurs seront amères aux buveurs. \EVERSE}
\newcommand{\isXXIVvXfr}{\VERSE  La ville de vanité est détruite, toutes les maisons sont fermées, personne n'y entre plus. \EVERSE}
\newcommand{\isXXIVvXIfr}{\VERSE  On criera dans les rues, parce que le vin manque; toute joie a cessé, l'allégresse de la terre a été bannie. \EVERSE}
\newcommand{\isXXIVvXIIfr}{\VERSE  La solitude est restée dans la ville et la calamité pressera ses portes. \EVERSE}
\newcommand{\isXXIVvXIIIfr}{\VERSE  Et il en sera au milieu de la terre, au milieu des peuples, comme lorsqu'on secoue quelques olives qui sont restées sur un olivier, et comme quelques raisins après qu'on a fini la vendange. \EVERSE}
\newcommand{\isXXIVvXIVfr}{\VERSE  Ceux-là élèveront leur voix, et ils chanteront des cantiques de louange : lorsque le Seigneur aura été glorifié, ils pousseront des cris du côté de la mer. \EVERSE}
\newcommand{\isXXIVvXVfr}{\VERSE  C'est pourquoi glorifiez le Seigneur par vos doctrines; célébrez le nom du Seigneur, du Dieu d'Israël, dans les îles de la mer. \EVERSE}
\newcommand{\isXXIVvXVIfr}{\VERSE  Des extrémités de la terre nous avons entendu des louanges, la gloire du Juste. Et j'ai dit : Mon secret est à moi, mon secret est à moi. Malheur à moi! Les prévaricateurs ont prévariqué, ils ont prévariqué comme des transgresseurs. \EVERSE}
\newcommand{\isXXIVvXVIIfr}{\VERSE  L'effroi, la fosse et le filet sont sur toi, habitant de la terre. \EVERSE}
\newcommand{\isXXIVvXVIIIfr}{\VERSE  Et voici, celui qui fuira devant l'effroi tombera dans la fosse, et celui qui sera sauvé de la fosse sera saisi par le filet; car les cataractes d'en haut s'ouvriront, et les fondements de la terre seront ébranlés. \EVERSE}
\newcommand{\isXXIVvXIXfr}{\VERSE  La terre sera déchirée par des déchirements, des renversements la briseront, des secousses l'ébranleront; \EVERSE}
\newcommand{\isXXIVvXXfr}{\VERSE  elle sera agitée et chancellera comme un homme ivre; elle sera enlevée comme une tente dressée pour une nuit; son iniquité l'écrasera, et elle tombera et ne se relèvera plus. \EVERSE}
\newcommand{\isXXIVvXXIfr}{\VERSE  En ce jour-là le Seigneur visitera l'armée d'en haut qui est dans le ciel, et les rois du monde qui sont sur la terre; \EVERSE}
\newcommand{\isXXIVvXXIIfr}{\VERSE  et ils seront assemblés et liés comme un faisceau, puis jetés dans l'abîme, où Dieu les tiendra en prison, et Il les visitera longtemps après. \EVERSE}
\newcommand{\isXXIVvXXIIIfr}{\VERSE  La lune rougira, et le soleil sera obscurci, lorsque le Seigneur des armées aura établi Son règne sur la montagne de Sion et dans Jérusalem, et qu'Il aura signalé Sa gloire devant ses anciens. \EVERSE}
\newcommand{\isXXVvIfr}{\VERSE  Seigneur, Vous êtes mon Dieu; je Vous exalterai, et je célébrerai Votre nom, parce que Vous avez fait des merveilles, réalisant Vos desseins antiques et fidèles. Amen. \EVERSE}
\newcommand{\isXXVvIIfr}{\VERSE  Car Vous avez réduit la ville en un monceau; la ville forte n'est plus qu'une ruine, la demeure des étrangers, afin qu'elle cesse d'être une ville, et qu'elle ne soit jamais rebâtie. \EVERSE}
\newcommand{\isXXVvIIIfr}{\VERSE  C'est pourquoi un peuple puissant Vous louera, et la cité des nations redoutables Vous révérera; \EVERSE}
\newcommand{\isXXVvIVfr}{\VERSE  parce que Vous êtes devenu la force du pauvre, la force du faible dans sa tribulation, un refuge contre la tempête, un rafraîchissement contre la chaleur; car la colère des puissants est comme un ouragan qui frappe une muraille. \EVERSE}
\newcommand{\isXXVvVfr}{\VERSE  Vous humilierez l'insolence des étrangers, comme l'ardeur du soleil dans un lieu aride; et Vous ferez sécher les rejetons des violents, comme la chaleur brûlante est étouffée par un nuage. \EVERSE}
\newcommand{\isXXVvVIfr}{\VERSE  Et le Seigneur des armées préparera à tous les peuples sur cette montagne un festin de mets délicieux, un festin de vin, un festin de viandes pleines de suc et de moelle, d'un vin clarifié. \EVERSE}
\newcommand{\isXXVvVIIfr}{\VERSE  Et sur cette montagne Il anéantira la chaîne qui tenait liés tous les peuples, et la toile qu'on avait ourdie sur toutes les nations. \EVERSE}
\newcommand{\isXXVvVIIIfr}{\VERSE  Il anéantira la mort à jamais; et le Seigneur Dieu enlèvera les larmes de tous les visages, et Il enlèvera de dessus la terre l'opprobre de Son peuple; car c'est le Seigneur qui a parlé. \EVERSE}
\newcommand{\isXXVvIXfr}{\VERSE  Et l'on dira en ce jour : Voici, c'est notre Dieu; nous L'avons attendu, et Il nous sauvera; c'est Lui qui est le Seigneur, nous L'avons attendu; nous serons dans l'allégresse, nous nous réjouirons dans Son salut. \EVERSE}
\newcommand{\isXXVvXfr}{\VERSE  Car la main du Seigneur reposera sur cette montagne; et Moab sera brisé sous Lui, comme le sont les pailles par la roue d'un char. \EVERSE}
\newcommand{\isXXVvXIfr}{\VERSE  Et il étendra ses mains sous lui comme un nageur les étend pour nager; et Dieu humiliera son orgueil en lui brisant les mains. \EVERSE}
\newcommand{\isXXVvXIIfr}{\VERSE  Les fortifications de tes murailles élevées tomberont, elles seront renversées à terre, et réduites en poussière. \EVERSE}
\newcommand{\isXXVIvIfr}{\VERSE  En ce jour, on chantera ce cantique dans la terre de Juda : Sion est notre ville forte; le Sauveur en sera la muraille et le boulevard. \EVERSE}
\newcommand{\isXXVIvIIfr}{\VERSE  Ouvrez les portes, et qu'un peuple juste y entre, observateur de la vérité. \EVERSE}
\newcommand{\isXXVIvIIIfr}{\VERSE  L'erreur ancienne a disparu; Vous conserverez la paix, la paix, car nous avons espéré en Vous. \EVERSE}
\newcommand{\isXXVIvIVfr}{\VERSE  Vous avez éternellement espéré dans le Seigneur, dans le Seigneur, le Dieu fort, à jamais. \EVERSE}
\newcommand{\isXXVIvVfr}{\VERSE  Car Il abaissera ceux qui habitent dans les hauteurs, Il humiliera la ville superbe; Il l'humiliera jusqu'à terre, Il la fera descendre jusque dans la poussière. \EVERSE}
\newcommand{\isXXVIvVIfr}{\VERSE  Elle sera foulée aux pieds, aux pieds du pauvre, sous les pas des indigents. \EVERSE}
\newcommand{\isXXVIvVIIfr}{\VERSE  Le sentier du juste est droit, le chemin du juste le conduira droit dans sa voie. \EVERSE}
\newcommand{\isXXVIvVIIIfr}{\VERSE  Aussi nous Vous avons attendu, Seigneur, dans le sentier de Vos jugements; Votre nom et Votre souvenir sont le désir de l'âme. \EVERSE}
\newcommand{\isXXVIvIXfr}{\VERSE  Mon âme Vous a désiré pendant la nuit, et je m'éveillerai dès le matin, pour Vous chercher de mon esprit et de mon coeur. Lorsque Vous aurez exercé Vos jugements sur la terre, les habitants du monde apprendront la justice. \EVERSE}
\newcommand{\isXXVIvXfr}{\VERSE  Faisons grâce à l'impie, et il n'apprendra pas la justice; il a commis l'iniquité dans la terre des Saints, et il ne verra pas la gloire du Seigneur. \EVERSE}
\newcommand{\isXXVIvXIfr}{\VERSE  Seigneur, que Votre main s'élève, et qu'ils ne voient point; que les peuples jaloux voient, et qu'ils soient confondus, et que le feu dévore Vos ennemis! \EVERSE}
\newcommand{\isXXVIvXIIfr}{\VERSE  Seigneur, Vous nous donnerez la paix; car c'est Vous qui avez fait pour nous toutes nos oeuvres. \EVERSE}
\newcommand{\isXXVIvXIIIfr}{\VERSE  Seigneur, notre Dieu, d'autres maîtres que Vous nous ont possédés; faites qu'en Vous seul nous nous souvenions de Votre nom. \EVERSE}
\newcommand{\isXXVIvXIVfr}{\VERSE  Que les morts ne revivent point, que les géants ne ressuscitent pas; car c'est pour cela que Vous les avez visités et anéantis, et que Vous avez détruit tout leur souvenir. \EVERSE}
\newcommand{\isXXVIvXVfr}{\VERSE  Vous favorisez cette nation, Seigneur, Vous la favorisez; n'avez-Vous pas été glorifié? Vous avez reculé toutes les limites de la terre. \EVERSE}
\newcommand{\isXXVIvXVIfr}{\VERSE  Seigneur, ils Vous ont cherché dans l'angoisse, Vous les instruisez par l'affliction qui les fait gémir auprès de Vous. \EVERSE}
\newcommand{\isXXVIvXVIIfr}{\VERSE  Comme une femme qui a conçu, et qui, sur le point d'enfanter, pousse de grands cris dans ses douleurs, ainsi avons-nous été loin de Votre face, Seigneur. \EVERSE}
\newcommand{\isXXVIvXVIIIfr}{\VERSE  Nous avons conçu, nous avons été comme en travail, et nous n'avons enfanté que du vent, nous n'avons pas produit le salut sur la terre; c'est pourquoi les habitants de la terre ne sont pas nés. \EVERSE}
\newcommand{\isXXVIvXIXfr}{\VERSE  Vos morts revivront, Mes enfants tués ressusciteront. Réveillez-vous, et louez Dieu, vous qui habitez la poussière, car votre rosée est une rosée de lumière, et vous ruinerez la terre des géants. \EVERSE}
\newcommand{\isXXVIvXXfr}{\VERSE  Va, Mon peuple, entre dans ta chambre; ferme tes portes sur toi, et cache-toi pour un moment, jusqu'à ce que la colère soit passée. \EVERSE}
\newcommand{\isXXVIvXXIfr}{\VERSE  Car voici que le Seigneur sortira de Sa demeure, pour visiter l'iniquité que les habitants de la terre ont commise contre Lui; et la terre révélera son sang; et ne cachera plus ses morts. \EVERSE}
\newcommand{\isXXVIIvIfr}{\VERSE  En ce jour-là le Seigneur visitera, avec Son glaive dur, grand et fort, Léviathan, ce serpent robuste, Léviathan, ce serpent tortueux, et Il tuera le monstre de la mer. \EVERSE}
\newcommand{\isXXVIIvIIfr}{\VERSE  En ce jour-là, la vigne au vin pur chantera pour Lui. \EVERSE}
\newcommand{\isXXVIIvIIIfr}{\VERSE  Je suis le Seigneur qui la garde; Je l'arroserai à tout instant; de peur qu'on ne lui nuise, Je la garde nuit et jour. \EVERSE}
\newcommand{\isXXVIIvIVfr}{\VERSE  Je n'ai pas de colère. Qui Me donnera des ronces et des épines qui M'attaquent? Je marcherai contre elles, Je les consumerai toutes ensemble. \EVERSE}
\newcommand{\isXXVIIvVfr}{\VERSE  Est-ce qu'elles purront retenir Ma puissance? Qu'elles fassent la paix avec Moi; qu'elles fassent la paix avec Moi. \EVERSE}
\newcommand{\isXXVIIvVIfr}{\VERSE  Qui que ce soit qui se précipite sur Jacob, Israël fleurira et germera, et ils rempliront de fruit la face du monde. \EVERSE}
\newcommand{\isXXVIIvVIIfr}{\VERSE  Est-ce que Dieu l'a frappé comme Il a frappé ses tyrans? et le massacre de ceux qu'Il a tués a-t-il égalé celui des persécuteurs? \EVERSE}
\newcommand{\isXXVIIvVIIIfr}{\VERSE  Lors même qu'Israël sera rejeté, Vous le jugerez avec modération et avec mesure; Il méditera, dans Son esprit irrité, au jour de Sa colère brûlante. \EVERSE}
\newcommand{\isXXVIIvIXfr}{\VERSE  C'est pour cela que l'iniquité de la maison de Jacob sera remise, et tout le fruit sera l'expiation de son péché, lorsqu'Israël aura brisé toutes les pierres de l'autel, comme des pierres réduites en cendres, et qu'il n'y aura plus de bois sacrés ni de temples. \EVERSE}
\newcommand{\isXXVIIvXfr}{\VERSE  Car la ville forte sera désolée; la ville si belle sera dépeuplée, et elle sera abandonnée comme un désert; là paîtra le veau et il s'y reposera, et il broutera les herbes. \EVERSE}
\newcommand{\isXXVIIvXIfr}{\VERSE  Leurs moissons desséchées seront foulées aux pieds. Des femmes viendront les instruire, car ce peuple n'a pas de sagesse; c'est pourquoi Celui qui l'a fait n'en aura pas pitié, et Celui qui l'a formé ne l'épargnera pas. \EVERSE}
\newcommand{\isXXVIIvXIIfr}{\VERSE  En ce jour-là, le Seigneur frappera depuis le lit du fleuve jusqu'au torrent d'Egypte; et vous serez rassemblés un à un, fils d'Israël. \EVERSE}
\newcommand{\isXXVIIvXIIIfr}{\VERSE  En ce jour-là, on sonnera de la grande trompette, et alors reviendront ceux qui étaient perdus dans le pays des Assyriens, et ceux qui avaient été bannis dans le pays d'Egypte, et ils adoreront le Seigneur sur la montagne sainte, à Jérusalem. \EVERSE}
\newcommand{\isXXVIIIvIfr}{\VERSE  Malheur à la couronne d'orgueil, aux ivrognes d'Ephraïm, à la fleur passagère qui fait leur faste et leur joie; à ceux qui habitent en haut de la très fertile vallée, et que le vin fait chanceler. \EVERSE}
\newcommand{\isXXVIIIvIIfr}{\VERSE  Voici que le Seigneur fort et puissant sera comme une grêle impétueuse, comme un tourbillon destructeur, comme un déluge d'eaux qui débordent et qui se précipitent sur une terre étendue. \EVERSE}
\newcommand{\isXXVIIIvIIIfr}{\VERSE  Elle sera foulée aux pieds, la couronne d'orgueil des ivrognes d'Ephraïm. \EVERSE}
\newcommand{\isXXVIIIvIVfr}{\VERSE  Et la fleur passagère qui fait le faste et la joie de ceux qui habitent en haut de la très fertile vallée, sera comme un fruit qui mûrit avant les autres fruits de l'automne; dès que quelqu'un l'aperçoit, il le prend de la main, et le mange aussitôt. \EVERSE}
\newcommand{\isXXVIIIvVfr}{\VERSE  En ce jour-là le Seigneur des armées sera une couronne de gloire, et un diadème d'allégresse pour le reste de Son peuple, \EVERSE}
\newcommand{\isXXVIIIvVIfr}{\VERSE  et un esprit de justice pour celui qui est assis sur le tribunal du jugement et la force de ceux qui retourneront du combat à la porte de la ville. \EVERSE}
\newcommand{\isXXVIIIvVIIfr}{\VERSE  Mais ceux-ci également sont si pleins de vin, qu'ils ne savent ce qu'ils font; tellement ivres, qu'ils chancellent; le prêtre et le prophète sont tellement ivres, qu'ils ne savent ce qu'ils font; ils sont absorbés par le vin, ils chancellent dans l'ivresse; ils n'ont pas reconnu le voyant, ils ont ignoré la justice. \EVERSE}
\newcommand{\isXXVIIIvVIIIfr}{\VERSE  Toutes les tables sont pleines de vomissements et d'ordure, il n'y reste plus de place. \EVERSE}
\newcommand{\isXXVIIIvIXfr}{\VERSE  A qui enseignera-t-Il la science? à qui donnera-t-Il l'intelligence de Sa parole? A des enfants qu'on ne fait que sevrer, qu'on vient d'arracher à la mamelle. \EVERSE}
\newcommand{\isXXVIIIvXfr}{\VERSE  Instruis, instruis encore; instruis, instruis encore; attends, atttends encore; attends, attends encore; un peu ici, un peu là. \EVERSE}
\newcommand{\isXXVIIIvXIfr}{\VERSE  Mais le Seigneur parlera d'une autre manière à ce peuple, Il ne lui tiendra plus le même langage. \EVERSE}
\newcommand{\isXXVIIIvXIIfr}{\VERSE  Il lui avait dit : C'est ici Mon repos, soulagez Ma lassitude; voici le lieu de Mon rafraîchissement; et ils n'ont pas voulu L'entendre. \EVERSE}
\newcommand{\isXXVIIIvXIIIfr}{\VERSE  C'est pourquoi le Seigneur leur dira : Instruis, instruis encore; instruis, instruis encore; attends, attends encore; attends, attends encore; un peu ici, un peu là; afin qu'ils aillent, qu'ils tombent à la renverse et qu'ils soient brisés, qu'ils tombent dans le piège et qu'ils soient pris. \EVERSE}
\newcommand{\isXXVIIIvXIVfr}{\VERSE  C'est pourquoi écoutez la parole du Seigneur, hommes moqueurs, qui dominez sur Mon peuple à Jérusalem. \EVERSE}
\newcommand{\isXXVIIIvXVfr}{\VERSE  Car vous avez dit : Nous avons contracté une alliance avec la mort et nous avons fait un pacte avec l'enfer. Lorsque le fléau débordant passera, il ne viendra pas sur nous, car nous avons mis notre confiance dans le mensonge, et le mensonge nous a protégés. \EVERSE}
\newcommand{\isXXVIIIvXVIfr}{\VERSE  C'est pourquoi ainsi parle le Seigneur Dieu : Je mettrai dans les fondements de Sion une pierre, une pierre éprouvée, angulaire, précieuse, qui sera un ferme fondement. Que celui qui croit ne se hâte pas. \EVERSE}
\newcommand{\isXXVIIIvXVIIfr}{\VERSE  J'établirai un poids de justice et une mesure d'équité, et la grêle détruira l'espérance mensongère, et les eaux emporteront la protection. \EVERSE}
\newcommand{\isXXVIIIvXVIIIfr}{\VERSE  Et votre alliance avec la mort sera rompue, et votre pacte avec l'enfer ne tiendra pas; lorsque le fléau débordant passera, vous serez foulés aux pieds par lui. \EVERSE}
\newcommand{\isXXVIIIvXIXfr}{\VERSE  Toutes les fois qu'il passera, il vous emportera, car il passera dès le matin, jour et nuit; et l'affliction seule vous donnera l'intelligence de ce que vous entendrez. \EVERSE}
\newcommand{\isXXVIIIvXXfr}{\VERSE  Car le lit est si étroit, que, si deux personnes s'y mettent, l'une tombera; et la couverture, trop courte, ne pourra pas les couvrir l'un et l'autre. \EVERSE}
\newcommand{\isXXVIIIvXXIfr}{\VERSE  Le Seigneur va Se lever comme sur la montagne des Divisions; Il va S'irriter comme dans la vallée de Gabaon; et Il fera Son oeuvre, Son oeuvre étrange; Il fera Son oeuvre, Son oeuvre étonnante. \EVERSE}
\newcommand{\isXXVIIIvXXIIfr}{\VERSE  Et maintenant, ne vous moquez plus, de peur que vos chaînes ne se resserrent; car le Seigneur, le Dieu des armées, m'a fait entendre qu'Il va opérer une destruction entière et un retranchement sur toute la terre. \EVERSE}
\newcommand{\isXXVIIIvXXIIIfr}{\VERSE  Prêtez l'oreille et écoutez ma voix; soyez attentifs, et écoutez ma parole. \EVERSE}
\newcommand{\isXXVIIIvXXIVfr}{\VERSE  Celui qui laboure pour semer labourera-t-il toujours? Ouvre-t-il et sarcle-t-il toujours la terre? \EVERSE}
\newcommand{\isXXVIIIvXXVfr}{\VERSE  N'est-ce pas après en avoir aplani la surface qu'il sème du gith et du cumin, et qu'il y met du blé par rangées, de l'orge, du millet et de la vesce sur les bords? \EVERSE}
\newcommand{\isXXVIIIvXXVIfr}{\VERSE  Son Dieu lui a donné du sens, et lui a appris ce qu'il doit faire. \EVERSE}
\newcommand{\isXXVIIIvXXVIIfr}{\VERSE  Le gith ne se foule pas avec les pointes de fer, et on ne fait point passer la roue du char sur le cumin; mais le gith se bat avec la verge, et le cumin avec le bâton. \EVERSE}
\newcommand{\isXXVIIIvXXVIIIfr}{\VERSE  On bat le blé dont on fait le pain; mais celui qui le triture ne le triture pas toujours, il ne le presse pas toujours sous la roue du char, et il ne le bat pas toujours sous les sabots de ses chevaux. \EVERSE}
\newcommand{\isXXVIIIvXXIXfr}{\VERSE  Cela aussi vient du Seigneur, du Dieu des armées, qui a voulu faire admirer Ses conseils, et signaler la grandeur de Sa sagesse. \EVERSE}
\newcommand{\isXXIXvIfr}{\VERSE  Malheur à Ariel, à la ville d'Ariel, prise d'assaut par David! L'année s'ajoutera à l'année, les fêtes accompliront leur cycle. \EVERSE}
\newcommand{\isXXIXvIIfr}{\VERSE  Puis J'environnerai Ariel de tranchées, et elle sera triste et désolée, et elle sera pour Moi comme Ariel. \EVERSE}
\newcommand{\isXXIXvIIIfr}{\VERSE  J'établirai autour de toi comme un cercle, J'élèverai des retranchements contre toi, et Je ferai des fortifications pour t'assiéger. \EVERSE}
\newcommand{\isXXIXvIVfr}{\VERSE  Tu seras humiliée, tu parleras comme de dessous la terre, et on entendra tes paroles venir comme du sol; et ta voix soortira de terre comme celle d'une pythonisse, et c'est de la poussière que tu murmureras tes discours. \EVERSE}
\newcommand{\isXXIXvVfr}{\VERSE  La multitude de tes oppresseurs sera comme une fine poussière, et la multitude de ceux qui te tiendront sous leur puissance sera comme la balle qui vole. \EVERSE}
\newcommand{\isXXIXvVIfr}{\VERSE  Et cela arrivera tout à coup, en un moment. C'est du Seigneur des armées que viendra le châtiment, au milieu des tonnerres, des tremblements de terre, de la grande voix de l'ouragan et de la tempête, et parmi les flammes d'un feu dévorant. \EVERSE}
\newcommand{\isXXIXvVIIfr}{\VERSE  Et la multitude des peuples qui auront pris les armes contre Ariel, et tous ceux qui l'auront combattue, qui l'auront assiégée, et qui s'en seront rendus les maîtres, sera comme le songe d'une vision de nuit. \EVERSE}
\newcommand{\isXXIXvVIIIfr}{\VERSE  Et comme celui qui a faim rêve qu'il mange, puis, lorsqu'il est éveillé, a l'estomac vide, et comme celui qui a soif rêve qu'il boit, puis, lorsqu'il est éveillé, se sent encore fatigué et altéré, et a l'estomac vide : ainsi sera la multitude de toutes les nations qui auront combattu contre la montagne de Sion. \EVERSE}
\newcommand{\isXXIXvIXfr}{\VERSE  Soyez dans l'étonnement et dans la surprise; soyez dans l'agitation et le tremblement; soyez ivres, mais pas de vin; soyez chancelants, mais non par suite de l'ivresse. \EVERSE}
\newcommand{\isXXIXvXfr}{\VERSE  Car le Seigneur a répandu sur vous un esprit d'assoupissement, Il fermera vos yeux; Il couvrira d'un voile vos prophètes et vos princes qui voient des visions. \EVERSE}
\newcommand{\isXXIXvXIfr}{\VERSE  Et toutes les visions vous seront comme les paroles d'un livre fermé avec des sceaux, qu'on donnera à un homme qui sait lire, en lui disant : Lis ce livre; et il répondra : Je ne le puis, parce qu'il est scellé. \EVERSE}
\newcommand{\isXXIXvXIIfr}{\VERSE  Et on donnera le livre à un homme qui ne sait pas lire, et on lui dira : Lis, et il répondra : Je ne sais pas lire. \EVERSE}
\newcommand{\isXXIXvXIIIfr}{\VERSE  Et le Seigneur a dit : Parce que ce peuple s'approche de bouche et Me glorifie des lèvres, tandis que son coeur est éloigné de Moi, et que le culte qu'il Me rend vient de préceptes et d'enseignements humains. \EVERSE}
\newcommand{\isXXIXvXIVfr}{\VERSE  Je ferai encore une merveille pour ce peuple un prodige étrange, surprenant; car la sagesse de ses sages périra, et la prudence de ses hommes intelligents disparaîtra. \EVERSE}
\newcommand{\isXXIXvXVfr}{\VERSE  Malheur à vous qui vous faites profonds de coeur, pour cacher au Seigneur vos desseins; qui accomplissez vos oeuvres dans les ténèbres, et qui dites : Qui nous voit, et qui nous connaît? \EVERSE}
\newcommand{\isXXIXvXVIfr}{\VERSE  Cette pensée est perverse; comme si l'argile s'élevait contre le potier, et si le vase disait à celui qui l'a formé : Ce n'est pas toi qui m'as fait; et comme si l'ouvrage disait à l'ouvrier : Tu n'as pas d'intelligence. \EVERSE}
\newcommand{\isXXIXvXVIIfr}{\VERSE  Ne verra-t-on pas, dans très peu de temps, le Liban devenir un carmel, et le carmel se changer en forêt? \EVERSE}
\newcommand{\isXXIXvXVIIIfr}{\VERSE  En ce jour-là les sourds entendront les paroles du livre, et sortant des ténèbres et de l'obscurité, les yeux des aveugles verront. \EVERSE}
\newcommand{\isXXIXvXIXfr}{\VERSE  Ceux qui sont doux se réjouiront de plus en plus dans le Seigneur, et les pauvres feront du Saint d'Israël un sujet d'allégresse; \EVERSE}
\newcommand{\isXXIXvXXfr}{\VERSE  car l'oppresseur a disparu, le moqueur n'est plus, et on a retranché tous ceux qui veillaient pour faire le mal, \EVERSE}
\newcommand{\isXXIXvXXIfr}{\VERSE  ceux qui faisaient pécher les hommes par leurs paroles, qui tendaient des pièges à quiconque défendait sa cause à la porte, et qui s'éloignaient sans motif du juste. \EVERSE}
\newcommand{\isXXIXvXXIIfr}{\VERSE  C'est pourquoi le Seigneur, qui a racheté Abraham, dit à la maison de Jacob : Jacob ne sera plus confondu, et son visage ne rougira plus; \EVERSE}
\newcommand{\isXXIXvXXIIIfr}{\VERSE  mais lorsqu'il verra ses enfants, qui sont l'oeuvre de Mes mains, rendre gloire à Mon nom, ils béniront ensemble le Saint de Jacob, et ils glorifieront le Dieu d'Israël; \EVERSE}
\newcommand{\isXXIXvXXIVfr}{\VERSE  et ceux dont l'esprit s'égarait acquerront de l'intelligence, et les murmurateurs apprendront la loi. \EVERSE}
\newcommand{\isXXXvIfr}{\VERSE  Malheur à vous, enfants rebelles, dit le Seigneur, qui formez des desseins sans Moi, et qui ourdissez des entreprises qui ne viennent pas de Mon esprit, pour accumuler péché sur péché; \EVERSE}
\newcommand{\isXXXvIIfr}{\VERSE  qui marchez pour descendre en Egypte sans Me consulter, espérant trouver du secours dans la force du pharaon, et mettant votre confiance dans l'ombre de l'Egypte. \EVERSE}
\newcommand{\isXXXvIIIfr}{\VERSE  Et cette force du pharaon sera pour vous une honte, et votre confiance dans l'ombre de l'Egypte, une ignominie. \EVERSE}
\newcommand{\isXXXvIVfr}{\VERSE  Tes princes sont allés jusqu'à Tanis, et tes ambassadeurs ont atteint Hanès. \EVERSE}
\newcommand{\isXXXvVfr}{\VERSE  Ils ont tous été confondus en voyant un peuple qui ne pouvait leur être utile; qui loin de les secourir et de leur rendre quelque service, est devenu leur honte et leur opprobre. \EVERSE}
\newcommand{\isXXXvVIfr}{\VERSE  Oracle sur les bêtes de somme du midi. Ils vont dans une terre de tribulation et d'angoisse, d'où sortent la lionne et le lion, la vipère et le basilic volant; ils portent leurs richesses sur les épaules des bêtes de somme, et leurs trésors sur le dos des chameaux, à un peuple qui ne pourra pas leur être utile. \EVERSE}
\newcommand{\isXXXvVIIfr}{\VERSE  Car le secours de l'Egypte n'est que néant et vanité. C'est pourquoi Je crie à ce sujet : Ce n'est que de l'orgueil; demeurez en paix. \EVERSE}
\newcommand{\isXXXvVIIIfr}{\VERSE  Maintenant donc, va graver cela sur du buis en leur présence, et écris-le avec soin dans un livre, afin qu'au dernier jour ce soit un témoignage éternel. \EVERSE}
\newcommand{\isXXXvIXfr}{\VERSE  Car ce peuple provoque Ma colère; ce sont des enfants menteurs, des enfants qui ne veulent point écouter la loi de Dieu; \EVERSE}
\newcommand{\isXXXvXfr}{\VERSE  qui disent aux voyants : Ne voyez pas; et aux prophètes : Ne regardez point pour nous ce qui est droit; dites-nous des choses flatteuses; voyez pour nous des erreurs. \EVERSE}
\newcommand{\isXXXvXIfr}{\VERSE  Eloignez de moi la voie; détournez de moi le sentier; que le Saint d'Israël disparaisse devant nous. \EVERSE}
\newcommand{\isXXXvXIIfr}{\VERSE  C'est pourquoi voici ce que dit le Saint d'Israël : Parce que vous avez rejeté cette parole, et que vous avez mis votre confiance dans la calomnie et le tumulte, et que vous les avez pris pour appuis, \EVERSE}
\newcommand{\isXXXvXIIIfr}{\VERSE  ce crime sera pour vous comme une lézarde menaçant ruine, qui s'avance sur un mur élevé, et qui s'écroule tout à coup, lorsqu'on n'y pense pas. \EVERSE}
\newcommand{\isXXXvXIVfr}{\VERSE  Il sera brisé comme un vase de terre que l'on casse avec efforts, sans qu'on trouve parmi ses fragments un tesson pour porter un charbon pris au feu, ou pour puiser un peu d'eau dans une fosse. \EVERSE}
\newcommand{\isXXXvXVfr}{\VERSE  Car ainsi parle le Seigneur Dieu, le Saint d'Israël : Si vous revenez, et si vous demeurez en paix, vous serez sauvés; votre force sera dans le silence et dans l'espérance. Et vous n'avez pas voulu; \EVERSE}
\newcommand{\isXXXvXVIfr}{\VERSE  et vous avez dit : Non, mais nous nous enfuirons sur des chevaux; c'est pour cela que vous fuirez. Nous monterons sur des coursiers rapides; c'est pour cela que ceux qui vous poursuivront seront plus rapides. \EVERSE}
\newcommand{\isXXXvXVIIfr}{\VERSE  Mille hommes fuiront épouvantés par un seul; épouvantés par cinq ennemis, vous fuirez, jusqu'à ce que vous restiez comme le mât d'un vaisseau au sommet d'une montagne, ou comme un étendard sur une colline. \EVERSE}
\newcommand{\isXXXvXVIIIfr}{\VERSE  C'est pourquoi le Seigneur attend le moment où Il aura pitié de vous, et Il signalera Sa gloire en vous pardonnant, car le Seigneur est un Dieu d'équité; heureux tous ceux qui L'attendent! \EVERSE}
\newcommand{\isXXXvXIXfr}{\VERSE  Car le peuple de Sion habitera dans Jérusalem; tu cesseras de pleurer; Il aura certainement pitié de toi; lors que tu crieras, dès qu'Il aura entendu ta voix, Il te répondra. \EVERSE}
\newcommand{\isXXXvXXfr}{\VERSE  Le Seigneur vous donnera le pain de l'angoisse et l'eau de l'affliction; Il n'éloignera plus de toi ton docteur, mais tes yeux verront Celui qui t'enseigne. \EVERSE}
\newcommand{\isXXXvXXIfr}{\VERSE  Tes oreilles entendront Sa parole lorsqu'Il criera derrière toi : C'est ici la voie, marchez-y sans vous détourner ni à droite ni à gauche. \EVERSE}
\newcommand{\isXXXvXXIIfr}{\VERSE  Tu regarderas comme profanes les lames d'argent de tes idoles et les vêtements de tes statues d'or; et tu les rejetteras comme un linge souillé. Hors d'ici, leur diras-tu. \EVERSE}
\newcommand{\isXXXvXXIIIfr}{\VERSE  La pluie sera donnée à tes grains partout où tu auras semé; et le fruit que la terre produira sera abondant et excellent; en ce jour-là les agneaux paîtront au large dans tes champs, \EVERSE}
\newcommand{\isXXXvXXIVfr}{\VERSE  et tes taureaux et tes ânons, qui labourent la terre, mangeront un mélange de grains tel qu'il aura été vanné dans l'aire. \EVERSE}
\newcommand{\isXXXvXXVfr}{\VERSE  Sur toute haute montagne et sur toute colline élevée il y aura des ruisseaux d'eaux courantes, au jour du grand carnage, lorsque les tours seront tombées. \EVERSE}
\newcommand{\isXXXvXXVIfr}{\VERSE  La lumière de la lune sera comme la lumière du soleil, et la lumière du soleil sera sept fois plus grande, comme la lumière de sept jours, lorsque le Seigneur aura bandé la blessure de Son peuple, et qu'Il aura guéri la plaie de Ses coups. \EVERSE}
\newcommand{\isXXXvXXVIIfr}{\VERSE  Voici que le nom du Seigneur vient de loin; Sa fureur est ardente et lourde à supporter; Ses lèvres sont pleines d'indignation, et Sa langue est comme un feu dévorant. \EVERSE}
\newcommand{\isXXXvXXVIIIfr}{\VERSE  Son souffle est comme un torrent débordé qui atteint jusqu'au cou, pour perdre et anéantir les nations, et briser le frein de l'erreur qui était dans les mâchoires des peuples. \EVERSE}
\newcommand{\isXXXvXXIXfr}{\VERSE  Vous chanterez des cantiques, comme la nuit de la fête solennelle, et votre coeur sera dans la joie, comme celui qui marche au son de la flûte, pour aller à la montagne du Seigneur, du Fort d'Israël. \EVERSE}
\newcommand{\isXXXvXXXfr}{\VERSE  Et le Seigneur fera entendre Sa voix majestueuse; Il montrera Son bras terrible, dans les menaces de Sa fureur et dans la flamme d'un feu dévorant; Il brisera tout dans la tempête et par des pierres de grêle. \EVERSE}
\newcommand{\isXXXvXXXIfr}{\VERSE  A la voix du Seigneur, Assur frappé de la verge tremblera. \EVERSE}
\newcommand{\isXXXvXXXIIfr}{\VERSE  Le passage de cette verge deviendra permanent; le Seigneur la fera reposer sur lui au son des tambourins et de harpes, et il vaincra Ses ennemis dans de grands combats. \EVERSE}
\newcommand{\isXXXvXXXIIIfr}{\VERSE  Car depuis longtemps Topheth a été préparée, préparée par le roi, profonde et vaste. Sa nourriture, c'est le feu et le bois en abondance, et le souffle du Seigneur est comme un torrent de soufre qui l'embrase. \EVERSE}
\newcommand{\isXXXIvIfr}{\VERSE  Malheur à ceux qui descendent en Egypte pour chercher du secours, qui espèrent dans les chevaux, qui mettent leur confiance dans les chars, parce qu'ils sont nombreux, et dans les cavaliers, parce qu'ils sont très forts, et qui ne s'appuient pas sur le Saint d'Israël et ne recherchent pas le Seigneur. \EVERSE}
\newcommand{\isXXXIvIIfr}{\VERSE  Mais Lui, qui est sage, a fait venir le malheur, et Il n'a pas retiré Ses paroles; Il S'élèvera contre la maison des méchants, et contre le secours de ceux qui commettent l'iniquité. \EVERSE}
\newcommand{\isXXXIvIIIfr}{\VERSE  L'Egypte est un homme, et non un dieu; ses chevaux sont chair, et non esprit; le Seigneur étendra Sa main, et celui qui donnait du secours sera renversé, et celui à qui le secours était donné tombera, et tous ensemble ils périront. \EVERSE}
\newcommand{\isXXXIvIVfr}{\VERSE  Car voici ce que ma dit le Seigneur : Comme lorsqu'un lion ou un lionceau rugit sur sa proie, si une troupe de bergers se présente devant lui, leur voix ne l'effraye pas, et leur multitude ne l'épouvante pas, ainsi le Seigneur des armées descendra pour combattre sur la montagne de Sion et sur sa colline. \EVERSE}
\newcommand{\isXXXIvVfr}{\VERSE  Comme les oiseaux qui volent sur leur couvée, ainsi le Seigneur des armées protégera Jérusalem; Il protégera et délivrera, Il passera et sauvera. \EVERSE}
\newcommand{\isXXXIvVIfr}{\VERSE  Revenez, autant que vous vous étiez profondément éloignés, fils d'Israël. \EVERSE}
\newcommand{\isXXXIvVIIfr}{\VERSE  En ce jour-là chacun rejettera ses idoles d'argent et ses idoles d'or, que vous vous étiez faites de vos mains criminelles. \EVERSE}
\newcommand{\isXXXIvVIIIfr}{\VERSE  Et Assur tombera sous un glaive qui n'est pas celui d'un homme, et un glaive qui n'est pas celui d'un homme le dévorera; il fuira, mais non devant le glaive, et ses jeunes hommes seront tributaires. \EVERSE}
\newcommand{\isXXXIvIXfr}{\VERSE  Sa force disparaîtra devant sa frayeur, et ses princes fuiront pleins d'effroi : ainsi dit le Seigneur qui a Son feu dans Sion, et Sa fournaise dans Jérusalem. \EVERSE}
\newcommand{\isXXXIIvIfr}{\VERSE  Voici que le roi régnera selon la justice, et que les princes gouverneront selon le droit. \EVERSE}
\newcommand{\isXXXIIvIIfr}{\VERSE  Et chacun d'eux sera comme un refuge contre le vent, et un abri contre la tempête; comme des eaux courantes dans une terre altérée, et comme l'ombre d'une roche avancée dans une terre aride. \EVERSE}
\newcommand{\isXXXIIvIIIfr}{\VERSE  Les yeux de ceux qui voient ne seront point troublés, et les oreilles de ceux qui entendent écouteront avec soin. \EVERSE}
\newcommand{\isXXXIIvIVfr}{\VERSE  Le coeur des insensés comprendra la science, et la langue de ceux qui balbutient parlera promptement et distinctement. \EVERSE}
\newcommand{\isXXXIIvVfr}{\VERSE  On ne donnera plus à l'insensé le nom de prince, ni au fourbe celui de grand; \EVERSE}
\newcommand{\isXXXIIvVIfr}{\VERSE  car l'insensé dira des folies, et son coeur s'adonnera à l'iniquité, pour compléter sa dissimulation, pour parler à Dieu avec fourberie, pour faire le vide dans l'âme de celui qui a faim, et pour enlever le breuvage à celui qui a soif. \EVERSE}
\newcommand{\isXXXIIvVIIfr}{\VERSE  Les armes du fourbe sont malignes, car il invente des plans pour perdre les petits par un discours mensonger, lorsque le pauvre parle selon la justice. \EVERSE}
\newcommand{\isXXXIIvVIIIfr}{\VERSE  Mais le prince aura des pensées dignes d'un prince, et il s'élèvera au-dessus des chefs. \EVERSE}
\newcommand{\isXXXIIvIXfr}{\VERSE  Femmes opulentes, levez-vous, et écoutez ma voix; filles si confiantes, prêtez l'oreille à Mes paroles. \EVERSE}
\newcommand{\isXXXIIvXfr}{\VERSE  Dans quelques jours et dans un an vous serez troublées, vous si confiantes; car c'en est fait de la vendange, et la récolte ne viendra plus. \EVERSE}
\newcommand{\isXXXIIvXIfr}{\VERSE  Tremblez, opulentes; soyez troublées, vous si confiantes; dépouillez-vous et soyez couvertes de confusion, revêtez-vous de sacs. \EVERSE}
\newcommand{\isXXXIIvXIIfr}{\VERSE  Frappez-vous les seins, au sujet de votre contrée délicieuse, au sujet de vos vignes fertiles. \EVERSE}
\newcommand{\isXXXIIvXIIIfr}{\VERSE  Les ronces et les épines monteront sur la terre de Mon peuple; combien plus sur toutes les maisons de plaisir de la cité joyeuse! \EVERSE}
\newcommand{\isXXXIIvXIVfr}{\VERSE  Car le palais sera abandonné, la ville si peuplée sera délaissée, ses maisons changées en cavernes seront à jamais couvertes d'épaisses ténèbres; les ânes sauvages s'y joueront, les troupeaux y paîtront, \EVERSE}
\newcommand{\isXXXIIvXVfr}{\VERSE  jusqu'à ce que l'esprit soit répandu sur nous d'en haut, et que le désert se change en carmel, et le carmel en forêt. \EVERSE}
\newcommand{\isXXXIIvXVIfr}{\VERSE  L'équité habitera dans le désert, et la justice aura sa demeure dans le carmel. \EVERSE}
\newcommand{\isXXXIIvXVIIfr}{\VERSE  La paix sera l'oeuvre de la justice, et le fruit de la justice sera le repos, et la sécurité à jamais. \EVERSE}
\newcommand{\isXXXIIvXVIIIfr}{\VERSE  Mon peuple se reposera dans la beauté de la paix, dans des tabernacles de confiance et dans un repos opulent. \EVERSE}
\newcommand{\isXXXIIvXIXfr}{\VERSE  Mais la grêle tombera sur la forêt, et la ville sera profondément humiliée. \EVERSE}
\newcommand{\isXXXIIvXXfr}{\VERSE  Vous êtes heureux, vous qui semez sur toutes les eaux, et qui laissez sans entraves le pied du boeuf et de l'âne. \EVERSE}
\newcommand{\isXXXIIIvIfr}{\VERSE  Malheur à toi qui ravages; ne seras-tu pas toi-même ravagé? et toi qui méprises, ne seras-tu pas toi-même méprisé? Lorsque tu auras fini de ravager tu seras ravagé, et lorsque tu seras las de mépriser tu seras méprisé. \EVERSE}
\newcommand{\isXXXIIIvIIfr}{\VERSE  Seigneur, ayez pitié de nous, car nous Vous avons attendu; soyez notre bras dès le matin, et notre salut au temps de la tribulation. \EVERSE}
\newcommand{\isXXXIIIvIIIfr}{\VERSE  A la voix de Votre Ange, les peuples ont fui, et devant Votre grandeur les nations seront dispersées. \EVERSE}
\newcommand{\isXXXIIIvIVfr}{\VERSE  On amassera vos dépouilles comme on amasse les sauterelles, dont on remplit des fosses entières. \EVERSE}
\newcommand{\isXXXIIIvVfr}{\VERSE  Le Seigneur a été exalté, car Il réside en haut; Il a rempli Sion d'équité et de justice. \EVERSE}
\newcommand{\isXXXIIIvVIfr}{\VERSE  La foi régnera dans votre temps; la sagesse et la science seront les richesses du salut; la crainte du Seigneur en sera le trésor. \EVERSE}
\newcommand{\isXXXIIIvVIIfr}{\VERSE  Ceux qui voient crieront au dehors; les messagers de paix pleureront amèrement. \EVERSE}
\newcommand{\isXXXIIIvVIIIfr}{\VERSE  Les chemins sont abandonnés, personne ne passe dans les sentiers; Il a rompu l'alliance, Il a rejeté les villes, Il n'a pas eu d'égard pour les hommes. \EVERSE}
\newcommand{\isXXXIIIvIXfr}{\VERSE  La terre pleure et languit; le Liban est confus et souillé; Saron a été changé en désert; Basan et le Carmel ont été dépouillés. \EVERSE}
\newcommand{\isXXXIIIvXfr}{\VERSE  Maintenant Je Me lèverai, dit le Seigneur; maintenant Je serai exalté, maintenant Je serai élevé. \EVERSE}
\newcommand{\isXXXIIIvXIfr}{\VERSE  Vous concevrez des flammes, vous enfanterez de la paille; votre esprit, comme un feu, vous dévorera. \EVERSE}
\newcommand{\isXXXIIIvXIIfr}{\VERSE  Et les peuples seront comme la cendre qui reste d'un incendie, et comme un fagot d'épines que le feu brûlera. \EVERSE}
\newcommand{\isXXXIIIvXIIIfr}{\VERSE  Ecoutez, vous qui êtes loin, ce que J'ai fait, et vous qui êtes près, connaissez Ma puissance. \EVERSE}
\newcommand{\isXXXIIIvXIVfr}{\VERSE  Les méchants ont été épouvantés à Sion, la frayeur a saisi les hypocrites. Qui de vous pourra demeurer dans le feu dévorant? qui de vous habitera dans les flammes éternelles? \EVERSE}
\newcommand{\isXXXIIIvXVfr}{\VERSE  Celui qui marche dans la justice et qui parle selon la vérité, qui rejette un gain acquis par extorsion et qui secoue ses mains pour ne recevoir aucun présent, qui bouche ses oreilles pour ne pas entendre de propos sanguinaire, et qui ferme ses yeux pour ne pas voir le mal. \EVERSE}
\newcommand{\isXXXIIIvXVIfr}{\VERSE  Celui-là habitera dans des lieux élevés, les hauts rochers fortifiés seront sa retraite; du pain lui sera donné, et ses eaux ne tariront pas. \EVERSE}
\newcommand{\isXXXIIIvXVIIfr}{\VERSE  Ses yeux contempleront le Roi dans Sa beauté, et verront le pays au loin. \EVERSE}
\newcommand{\isXXXIIIvXVIIIfr}{\VERSE  Ton coeur s'occupera de ce qui faisait sa crainte. Où est le savant? Où est celui qui pèse les paroles de la loi? Où est le docteur des petits enfants? \EVERSE}
\newcommand{\isXXXIIIvXIXfr}{\VERSE  Tu ne verras plus le peuple impudent, le peuple aux discours obscurs, dont tu ne pouvais comprendre le langage étudié, et qui n'a aucune sagesse. \EVERSE}
\newcommand{\isXXXIIIvXXfr}{\VERSE  Regarde Sion, la ville de nos fêtes : tes yeux verront Jérusalem, habitation opulente, tente qui ne pourra plus être transportée;  ses pieux ne seront jamais arrachés, et aucun de ses cordages ne se rompra. \EVERSE}
\newcommand{\isXXXIIIvXXIfr}{\VERSE  Car c'est là seulement que notre Seigneur est magnifique; les fleuves y auront un canal très large et spacieux; le vaisseau à rames n'y passera pas, et la grande galère ne le traversera pas, \EVERSE}
\newcommand{\isXXXIIIvXXIIfr}{\VERSE  car le Seigneur est notre juge, le Seigneur est notre législateur, le Seigneur est notre Roi; c'est Lui qui nous sauvera. \EVERSE}
\newcommand{\isXXXIIIvXXIIIfr}{\VERSE  Tes cordages sont relâchés, et ils ne résisteront pas; ton mât sera dans un tel état, que tu ne pourra pas étendre tes voiles. Alors on partagera les dépouilles d'un butin considérable; les boiteux même prendront part au pillage. \EVERSE}
\newcommand{\isXXXIIIvXXIVfr}{\VERSE  Le voisin ne dira pas : Je suis malade; le peuple qui y habitera recevra le pardon de ses péchés. \EVERSE}
\newcommand{\isXXXIVvIfr}{\VERSE  Approchez-vous, nations, et écoutez; peuples, soyez attentifs; que la terre écoute, et ce qui la remplit; le monde et tout ce qu'il produit. \EVERSE}
\newcommand{\isXXXIVvIIfr}{\VERSE  Car l'indignation du Seigneur va fondre sur toutes les nations, et la fureur sur toute leur armée; Il les tuera et Il les livrera au carnage. \EVERSE}
\newcommand{\isXXXIVvIIIfr}{\VERSE  Leurs morts seront jetés, et la puanteur s'élèvera de leurs cadavres; les montagnes dégoutteront de leur sang. \EVERSE}
\newcommand{\isXXXIVvIVfr}{\VERSE  Et toute la milice des cieux se dissoudra, et les cieux s'enrouleront comme un livre; et toute leur milice en tombera, comme les feuilles tombent de la vigne et du figuier. \EVERSE}
\newcommand{\isXXXIVvVfr}{\VERSE  Car Mon glaive s'est enivré dans le ciel; voici qu'il va descendre sur l'Idumée, et sur le peuple que J'ai voué au carnage, pour en faire justice. \EVERSE}
\newcommand{\isXXXIVvVIfr}{\VERSE  Le glaive du Seigneur est plein de sang; il est tout couvert de graisse, du sang des agneaux et des boucs, du sang des béliers engraissés; car il y a des victimes du Seigneur à Bosra, et Il fera un grand carnage dans la terre d'Edom. \EVERSE}
\newcommand{\isXXXIVvVIIfr}{\VERSE  Les licornes descendront avec eux, et les taureaux avec les plus puissants d'entre eux; la terre s'enivrera de leur sang, et le sol sera imprégné de leur graisse. \EVERSE}
\newcommand{\isXXXIVvVIIIfr}{\VERSE  Car c'est le jour de la vengeance du Seigneur, l'année des représailles pour faire justice à Sion. \EVERSE}
\newcommand{\isXXXIVvIXfr}{\VERSE  Les torrents de l'Idumée se changeront en poix, et son sol en soufre, et sa terre deviendra une poix brûlante. \EVERSE}
\newcommand{\isXXXIVvXfr}{\VERSE  Son feu ne s'éteindra ni jour ni nuit, sa fumée montera à jamais; de génération en génération elle sera désolée, et il n'y passera personne dans la suite des siècles. \EVERSE}
\newcommand{\isXXXIVvXIfr}{\VERSE  Le pélican et le hérisson la posséderont, l'ibis et le corbeau y habiteront; Dieu étendra la ligne sur elle pour la réduire à néant, et le niveau pour la détruire entièrement. \EVERSE}
\newcommand{\isXXXIVvXIIfr}{\VERSE  Ses nobles n'y demeureront plus; mais ils invoqueront le roi, et tous ses princes seront anéantis. \EVERSE}
\newcommand{\isXXXIVvXIIIfr}{\VERSE  Les épines et les orties croîtront dans ses maisons, les chardons dans ses forteresses, et elle deviendra le repaire des dragons et le pâturage des autruches. \EVERSE}
\newcommand{\isXXXIVvXIVfr}{\VERSE  Les démons et les onocentaures s'y rencontreront, et les satyres s'y jetteront des cris l'un à l'autre; la sirène s'y retire, et y trouve son repos. \EVERSE}
\newcommand{\isXXXIVvXVfr}{\VERSE  Le hérisson y fait son trou et y nourrit ses petits, il creuse tout autour, il les fait croître à son ombre; les milans s'y assemblent l'un près de l'autre. \EVERSE}
\newcommand{\isXXXIVvXVIfr}{\VERSE  Cherchez avec soin dans le livre du Seigneur, et lisez : rien de tout cela ne manquera, aucune de ces choses ne fera défaut; car ce qui sort de ma bouche Dieu l'a ordonné, et c'est Son Esprit qui les rassemblera. \EVERSE}
\newcommand{\isXXXIVvXVIIfr}{\VERSE  C'est Lui qui leur fera le partage; Sa main la divisera entre eux au cordeau; ils la posséderont éternellement; ils y habiteront de génération en génération. \EVERSE}
\newcommand{\isXXXVvIfr}{\VERSE  Le pays désert et sans chemin se réjouira, la solitude sera dans l'allégresse et fleurira comme un lis. \EVERSE}
\newcommand{\isXXXVvIIfr}{\VERSE  Elle poussera et germera, elle tressaillira de joie et de louanges; la gloire du Liban lui sera donnée, la beauté du Carmel et de Saron; ils verront eux-mêmes la gloire du Seigneur, et la beauté de notre Dieu. \EVERSE}
\newcommand{\isXXXVvIIIfr}{\VERSE  Fortifiez les mains languissantes, et affermissez les genoux qui chancellent. \EVERSE}
\newcommand{\isXXXVvIVfr}{\VERSE  Dites aux pusillanimes : Prenez courage et ne craignez point; voici votre Dieu qui apporte la vengeance et les représailles; Dieu Lui-même viendra, et Il vous sauvera. \EVERSE}
\newcommand{\isXXXVvVfr}{\VERSE  Alors les yeux des aveugles verront, et les oreilles des sourds seront ouvertes. \EVERSE}
\newcommand{\isXXXVvVIfr}{\VERSE  Alors le boiteux bondira comme un cerf, et la langue des muets sera déliée; car des eaux jailliront dans le désert, et des torrents dans la solitude. \EVERSE}
\newcommand{\isXXXVvVIIfr}{\VERSE  La terre aride se changera en étang, et la terre desséchée, en fontaines d'eaux. Dans les tanières où les dragons habitaient auparavant, naîtra la verdure des roseaux et des joncs. \EVERSE}
\newcommand{\isXXXVvVIIIfr}{\VERSE  Il y aura là un sentier et une voie, qui sera appelée la voie sainte; nul impur n'y passera, et ce sera pour vous une voie droite, de sorte que les insensés ne pourront s'y égarer. \EVERSE}
\newcommand{\isXXXVvIXfr}{\VERSE  Il n'y aura pas là de lion, la bête fauve n'y montera pas et ne s'y trouvera point; ceux qui auront été délivrés y marcheront. \EVERSE}
\newcommand{\isXXXVvXfr}{\VERSE  Et les rachetés du Seigneur retourneront, et viendront à Sion en chantant des louanges; une joie éternelle couronnera leur tête; le ravissement de la joie ne les quittera pas, la douleur et les gémissements s'enfuiront. \EVERSE}
\newcommand{\isXXXVIvIfr}{\VERSE  La quatorzième année du règne d'Ezéchias, Sennachérib, roi des Assyriens, vint assiéger toutes les villes fortes de Juda, et il les prit. \EVERSE}
\newcommand{\isXXXVIvIIfr}{\VERSE  Et le roi des Assyriens envoya Rabsacès de Lachis à Jérusalem vers le roi Ezéchias, avec une forte escorte, et il s'arrêta près de l'aqueduc de la piscine supérieure, sur le chemin du champ du Foulon. \EVERSE}
\newcommand{\isXXXVIvIIIfr}{\VERSE  Eliacim, fils d'Helcias, qui était grand maître de la maison du roi, Sobna, secrétaire, et Joahé, fils d'Asaph, chancelier, sortirent auprès de lui. \EVERSE}
\newcommand{\isXXXVIvIVfr}{\VERSE  Et Rabsacès leur dit : Dites à Ezéchias : Voici ce que dit le grand roi, le roi des Assyriens : Quelle est cette confiance dont tu te flattes? \EVERSE}
\newcommand{\isXXXVIvVfr}{\VERSE  Par quel dessein et avec quelle force prétends-tu te révolter? sur qui t'appuies-tu, pour refuser de m'obéir? \EVERSE}
\newcommand{\isXXXVIvVIfr}{\VERSE  Tu t'appuies sur l'Egypte, ce roseau brisé, qui entrera dans la main de celui qui s'appuiera dessus, et qui la transpercera : c'est ce que sera le pharaon, roi d'Egypte, pour tous ceux qui espèrent en lui. \EVERSE}
\newcommand{\isXXXVIvVIIfr}{\VERSE  Que si tu me réponds : Nous avons confiance dans le Seigneur notre Dieu, n'est-ce pas Lui dont Ezéchias a détruit les hauts lieux et les autels, en disant à Juda et à Jérusalem : Vous adorerez devant cet autel? \EVERSE}
\newcommand{\isXXXVIvVIIIfr}{\VERSE  Rends-toi donc maintenant à mon maître, le roi des Assyriens, et je te donnerai deux mille chevaux, et tu ne pourras trouver assez d'hommes pour les monter. \EVERSE}
\newcommand{\isXXXVIvIXfr}{\VERSE  Et comment soutiendras-tu l'aspect d'un seul gouverneur pris parmi les moindres serviteurs de mon maître? Que si tu as confiance dans l'Egypte, dans ses chars et dans ses cavaliers, \EVERSE}
\newcommand{\isXXXVIvXfr}{\VERSE  est-ce que je suis monté sans le Seigneur dans cette terre pour la perdre? C'est le Seigneur qui m'a dit : Monte contre cette terre, et détruis-là. \EVERSE}
\newcommand{\isXXXVIvXIfr}{\VERSE  Alors Eliacim, Sobna et Joahé dirent à Rabsacès : Parle à tes serviteurs en langue syriaque, car nous la comprenons; mais ne nous parle pas en hébreu aux oreilles du peuple qui est sur la muraille. \EVERSE}
\newcommand{\isXXXVIvXIIfr}{\VERSE  Et Rabsacès leur dit : Est-ce à ton maître et à toi que mon maître m'a envoyé dire ces paroles? et n'est-ce pas plutôt à ces hommes assis sur la muraille, pour manger leurs excréments et pour boire leur urine avec vous? \EVERSE}
\newcommand{\isXXXVIvXIIIfr}{\VERSE  Rabsacès se tenant donc debout, et criant de toute sa force, dit en langue judaïque : Ecoutez les paroles du grand roi, du roi des Assyriens. \EVERSE}
\newcommand{\isXXXVIvXIVfr}{\VERSE  Voici ce que dit le roi : Qu'Ezéchias ne vous séduise pas, car il ne pourra pas vous délivrer. \EVERSE}
\newcommand{\isXXXVIvXVfr}{\VERSE  Et qu'Ezéchias ne vous fasse pas mettre votre confiance dans le Seigneur, en disant : Le Seigneur nous délivrera certainement; cette ville ne sera pas livrée entre les mains du roi des Assyriens. \EVERSE}
\newcommand{\isXXXVIvXVIfr}{\VERSE  N'écoutez pas Ezéchias; car voici ce que dit le roi des Assyriens : Faites alliance avec moi, et venez vous rendre à moi, et chacun mangera du fruit de sa vigne et du fruit de son figuier, et boira l'eau de la citerne, \EVERSE}
\newcommand{\isXXXVIvXVIIfr}{\VERSE  jusqu'à ce que je vienne, et que je vous emmène dans une terre semblable à la vôtre, une terre de blé et de vin, une terre abondante en pains et en vignes. \EVERSE}
\newcommand{\isXXXVIvXVIIIfr}{\VERSE  Qu'Ezéchias ne vous trouble pas, en disant : Le Seigneur nous délivrera. Est-ce que les dieux des nations ont délivré chacun sa terre de la puissance du roi des Assyriens? \EVERSE}
\newcommand{\isXXXVIvXIXfr}{\VERSE  Où est le dieu d'Emath et d'Arphad? où est le dieu de Sepharvaïm? Ont-ils délivré Samarie de ma main puissante? \EVERSE}
\newcommand{\isXXXVIvXXfr}{\VERSE  Quel est, entre tous les dieux de ces pays, celui qui ait pu délivrer son pays de ma main, pour que le Seigneur puisse sauver Jérusalem de ma main? \EVERSE}
\newcommand{\isXXXVIvXXIfr}{\VERSE  Ils se turent, et ils ne lui répondirent pas un mot. Car le roi leur avait donné cet ordre : Ne lui répondez pas. \EVERSE}
\newcommand{\isXXXVIvXXIIfr}{\VERSE  Eliacim, fils d'Helcias, grand maître de la maison du roi, Sobna secrétaire, et Joahé, fils d'Asaph, chancelier, vinrent auprès d'Ezéchias, ayant les vêtements déchirés, et ils lui rapportèrent les paroles de Rabsacès. \EVERSE}
\newcommand{\isXXXVIIvIfr}{\VERSE  Et lorsque le roi Ezéchias eut entendu cela, il déchira ses vêtements, se couvrit d'un sac, et entra dans la maison du Seigneur. \EVERSE}
\newcommand{\isXXXVIIvIIfr}{\VERSE  Et il envoya Eliacim, grand maître de sa maison, et Sobna, secrétaire, et les plus anciens d'entre les prêtres, couverts de sacs, vers le prophète Isaïe, fils d'Amos, \EVERSE}
\newcommand{\isXXXVIIvIIIfr}{\VERSE  et ils lui dirent : Voici ce que dit Ezéchias : Ce jour est un jour de tribulation, de reproche et de blasphème; car les enfants sont sur le point de naître, mais la mère n'a pas assez de force pour enfanter. \EVERSE}
\newcommand{\isXXXVIIvIVfr}{\VERSE  Peut-être que le Seigneur ton Dieu aura entendu les paroles de Rabsacès, qui a été envoyé par le roi des Assyriens, son maître, pour blasphémer le Dieu vivant, et pour L'insulter par les paroles que le Seigneur ton Dieu a entendues. Fais donc monter une prière pour les restes qui subsistent encore. \EVERSE}
\newcommand{\isXXXVIIvVfr}{\VERSE  Les serviteurs du roi Ezéchias vinrent donc trouver Isaïe. \EVERSE}
\newcommand{\isXXXVIIvVIfr}{\VERSE  Et Isaïe leur dit : Vous direz à votre maître : Voici ce que dit le Seigneur : Ne crains pas ces paroles que tu as entendues, et par lesquelles les serviteurs du roi des Assyriens M'ont blasphémé. \EVERSE}
\newcommand{\isXXXVIIvVIIfr}{\VERSE  Je lui enverrai un esprit, et il apprendra une nouvelle, et il retournera dans son pays, et Je le ferai mourir par le glaive dans son pays. \EVERSE}
\newcommand{\isXXXVIIvVIIIfr}{\VERSE  Or Rabsacès s'en retourna, et il trouva le roi d'Assyrie qui assiégeait Lobna; car il avait appris qu'il avait quitté Lachis. \EVERSE}
\newcommand{\isXXXVIIvIXfr}{\VERSE  Alors le roi d'Assyrie reçut une nouvelle au sujet de Tharaca, roi d'Ethiopie; on lui dit : Il s'est mis en marche pour vous combattre. Ayant appris cela, il envoya des ambassadeurs à Ezéchias, avec cet ordre: \EVERSE}
\newcommand{\isXXXVIIvXfr}{\VERSE  Vous direz à Ezéchias, roi de Juda : Que ton Dieu auquel tu as confiance ne te séduise pas, en disant : Jérusalem ne sera pas livrée entre les mains du roi des Assyriens. \EVERSE}
\newcommand{\isXXXVIIvXIfr}{\VERSE  Tu as appris tout ce que les rois des Assyriens ont fait à tous les pays qu'ils ont ruinés; et toi, pourrais-tu être délivré? \EVERSE}
\newcommand{\isXXXVIIvXIIfr}{\VERSE  Est-ce que les dieux des nations ont délivré les peuples que mes pères ont détruits, Gozam, Haram, Réseph et les fils d'Eden qui étaient à Thalassar? \EVERSE}
\newcommand{\isXXXVIIvXIIIfr}{\VERSE  Où est le roi d'Emath, et le roi d'Arphad, et le roi de la ville de Sepharvaïm, d'Ana et d'Ava? \EVERSE}
\newcommand{\isXXXVIIvXIVfr}{\VERSE  Ezéchias prit la lettre de la main des ambassadeurs, et l'ayant lue, il monta à la maison du Seigneur, et la déploya devant le Seigneur; \EVERSE}
\newcommand{\isXXXVIIvXVfr}{\VERSE  et Ezéchias pria le Seigneur en disant: \EVERSE}
\newcommand{\isXXXVIIvXVIfr}{\VERSE  Seigneur des armées, Dieu d'Israël, qui êtes assis sur les chérubins, Vous êtes seul Dieu de tous les royaumes de la terre; c'est Vous qui avez fait le ciel et la terre. \EVERSE}
\newcommand{\isXXXVIIvXVIIfr}{\VERSE  Penchez, Seigneur, Votre oreille et écoutez; ouvrez les yeux, Seigneur, et voyez, et écoutez toutes les paroles que Sennachérib a envoyées pour blasphémer le Dieu vivant. \EVERSE}
\newcommand{\isXXXVIIvXVIIIfr}{\VERSE  Il est vrai, Seigneur, que les rois des Assyriens ont ruiné les pays et leurs provinces, \EVERSE}
\newcommand{\isXXXVIIvXIXfr}{\VERSE  et qu'ils ont livré leurs dieux au feu; car ce n'étaient pas des dieux, mais l'oeuvre de la main des hommes, du bois et de la pierre, qu'ils ont détruits. \EVERSE}
\newcommand{\isXXXVIIvXXfr}{\VERSE  Et maintenant, Seigneur notre Dieu, délivrez-nous de sa main, afin que tous les royaumes de la terre sachent que Vous êtes le seul Seigneur. \EVERSE}
\newcommand{\isXXXVIIvXXIfr}{\VERSE  Alors Isaïe, fils d'Amos, envoya dire à Ezéchias : Voici ce que dit le Seigneur, le Dieu d'Israël : Quant à ce que tu m'as demandé au sujet de Sennachérib, roi d'Assyrie, \EVERSE}
\newcommand{\isXXXVIIvXXIIfr}{\VERSE  voici la parole que le Seigneur a prononcée sur lui : Elle t'a méprisé et elle t'a insulté, la vierge fille de Sion : la fille de Jérusalem a secoué la tête derrière toi. \EVERSE}
\newcommand{\isXXXVIIvXXIIIfr}{\VERSE  Qui as-tu insulté? contre qui as-tu haussé la voix et élevé tes yeux insolents? Contre le Saint d'Israël. \EVERSE}
\newcommand{\isXXXVIIvXXIVfr}{\VERSE  Par tes serviteurs tu as outragé le Seigneur, et tu as dit : Avec la multitude de mes chars, j'ai gravi le sommet des montagnes, les cimes du Liban; j'ai coupé ses cèdres élevés et ses sapins de choix; je suis monté jusqu'à la pointe de son sommet, dans la forêt de son carmel. \EVERSE}
\newcommand{\isXXXVIIvXXVfr}{\VERSE  J'ai creusé et j'ai bu les eaux, et j'ai desséché avec la plante de mes pieds toutes les rivières retenues par des digues. \EVERSE}
\newcommand{\isXXXVIIvXXVIfr}{\VERSE  N'as-tu pas appris ce que J'ai fait autrefois? Dès les jours anciens, J'ai formé ce dessein, et maintenant Je l'ai exécuté, et cela a été accompli pour la ruine des collines qui s'entrebattent et des villes fortes. \EVERSE}
\newcommand{\isXXXVIIvXXVIIfr}{\VERSE  Leurs habitants aux mains débiles ont tremblé et ont été confondus; ils sont devenus comme l'herbe des champs, comme le gazon qui sert de pâture, et comme l'herbe des toits, qui sèche avant de venir à maturité. \EVERSE}
\newcommand{\isXXXVIIvXXVIIIfr}{\VERSE  Ta demeure, et ta sortie, et ton entrée, Je les connais, ainsi que ta fureur insensée contre Moi. \EVERSE}
\newcommand{\isXXXVIIvXXIXfr}{\VERSE  Tandis que ta fureur éclatait contre Moi, ton orgueil est monté jusqu'à Mes oreilles. Je te mettrai donc une boucle aux narines et un mors à la bouche, et Je te ramènerai sur le chemin par lequel tu es venu. \EVERSE}
\newcommand{\isXXXVIIvXXXfr}{\VERSE  Mais pour toi, Ezéchias, tu auras ce signe : Mange cette année ce qui naîtra de soi-même, et vis de fruits la seconde année; mais la troisième année semez et moissonnez, plantez des vignes et recueillez-en le fruit. \EVERSE}
\newcommand{\isXXXVIIvXXXIfr}{\VERSE  Et ce qui aura été sauvé de la maison de Juda et ce qui sera resté poussera des racines en bas, et produira des fruits en haut; \EVERSE}
\newcommand{\isXXXVIIvXXXIIfr}{\VERSE  car de Jérusalem il sortira un reste, et des sauvés du mont Sion; le zèle du Seigneur des armées fera cela. \EVERSE}
\newcommand{\isXXXVIIvXXXIIIfr}{\VERSE  C'est pourquoi ainsi parle le Seigneur sur le roi des Assyriens : Il n'entrera pas dans cette ville et il n'y lancera pas de flèches, il ne l'attaquera pas avec le bouclier, et il n'élèvera pas de retranchements autour d'elle. \EVERSE}
\newcommand{\isXXXVIIvXXXIVfr}{\VERSE  Il s'en retournera par le chemin par lequel il est venu, et il n'entrera pas dans cette ville, dit le Seigneur. \EVERSE}
\newcommand{\isXXXVIIvXXXVfr}{\VERSE  Je protégerai cette ville pour la sauver, à cause de Moi, et à cause de David Mon serviteur. \EVERSE}
\newcommand{\isXXXVIIvXXXVIfr}{\VERSE  Or l'Ange du Seigneur sortit et frappa cent quatre-vingt-cinq mille hommes dans le camp des Assyriens. Et quand on se leva le matin, c'étaient tous des cadavres sans vie. \EVERSE}
\newcommand{\isXXXVIIvXXXVIIfr}{\VERSE  Alors Sennachérib, roi des Assyriens, partait et s'en alla, et s'en retourna, et il demeura à Ninive. \EVERSE}
\newcommand{\isXXXVIIvXXXVIIIfr}{\VERSE  Or comme il était prosterné dans le temple de Nesroch son dieu, Adramélech et Sarasar, ses fils, le frappèrent de leur glaive, et s'enfuirent dans le pays d'Ararat; et Asarhaddon son fils régna à sa place. \EVERSE}
\newcommand{\isXXXVIIIvIfr}{\VERSE  En ce temps-là, Ezéchias fut malade jusqu'à la mort, et le prophète Isaïe, fils d'Amos, vint auprès de lui et lui dit : Voici ce que dit le Seigneur : Mets ordre aux affaires de ta maison, car tu mourras, et tu ne vivras plus. \EVERSE}
\newcommand{\isXXXVIIIvIIfr}{\VERSE  Alors Ezéchias tourna son visage contre le mur, et pria le Seigneur \EVERSE}
\newcommand{\isXXXVIIIvIIIfr}{\VERSE  en disant : Souvenez-vous, Seigneur, je Vous prie, que j'ai marché devant Vous dans la vérité et avec un coeur parfait, et que j'ai fait ce qui était bon à vos yeux. Et Ezéchias versa des larmes abondantes. \EVERSE}
\newcommand{\isXXXVIIIvIVfr}{\VERSE  Alors le Seigneur parla à Isaïe, et lui dit: \EVERSE}
\newcommand{\isXXXVIIIvVfr}{\VERSE  Va, et dis à Ezéchias : Ainsi parle le Seigneur, le Dieu de David ton père : J'ai entendu ta prière et J'ai vu tes larmes; J'ajouterai encore quinze années à tes jours, \EVERSE}
\newcommand{\isXXXVIIIvVIfr}{\VERSE  et Je te délivrerai de la main du roi des Assyriens; cette ville aussi, et Je la protégerai. \EVERSE}
\newcommand{\isXXXVIIIvVIIfr}{\VERSE  Et voici le signe que le Seigneur te donnera, pour t'assurer qu'Il accomplira la parole qu'Il a prononcée: \EVERSE}
\newcommand{\isXXXVIIIvVIIIfr}{\VERSE  Je ferai reculer de dix degrés en arrière, avec le soleil, l'ombre des degrés qui est descendue sur le cadran d'Achaz. Et le soleil recula de dix degrés, sur les degrés où il était descendu. \EVERSE}
\newcommand{\isXXXVIIIvIXfr}{\VERSE  Cantique d'Ezéchias, roi de Juda, lorsque après avoir été malade, il fut guéri de sa maladie. \EVERSE}
\newcommand{\isXXXVIIIvXfr}{\VERSE  J'ai dit : Au milieu de mes jours, j'irai aux portes du tombeau. Je cherche en vain le reste de mes années. \EVERSE}
\newcommand{\isXXXVIIIvXIfr}{\VERSE  J'ai dit : Je ne verrai plus le Seigneur Dieu dans la terre des vivants; je ne verrai plus aucun homme, ni d'habitant du repos. \EVERSE}
\newcommand{\isXXXVIIIvXIIfr}{\VERSE  Le temps de ma vie m'est enlevé, et il est roulé loin de moi, comme une tente de berger. Ma vie a été coupée comme par le tisserand; il m'a retranché tandis que j'ourdissais encore. Du matin au soir vous en finirez avec moi. \EVERSE}
\newcommand{\isXXXVIIIvXIIIfr}{\VERSE  J'espérais jusqu'au matin; comme un lion il a brisé tous mes os. Du matin au soir Vous en finirez avec moi. \EVERSE}
\newcommand{\isXXXVIIIvXIVfr}{\VERSE  Je criais comme le petit de l'hirondelle, je gémissais comme la colombe. Mes yeux se sont lassés à force de regarder en haut. Seigneur, je souffre violence, répondez pour moi. \EVERSE}
\newcommand{\isXXXVIIIvXVfr}{\VERSE  Que dirai-je, et que me répondra-t-Il, puisque c'est Lui qui a fait cela? Je repasserai devant Vous toutes mes années, dans l'amertume de mon âme. \EVERSE}
\newcommand{\isXXXVIIIvXVIfr}{\VERSE  Seigneur, si c'est ainsi que l'on vit, si la vie de mon esprit consiste en ces choses, Vous me châtierez, et Vous me rendrez la vie. \EVERSE}
\newcommand{\isXXXVIIIvXVIIfr}{\VERSE  Je trouverai la paix dans mon affliction la plus amère. Mais Vous, Vous avez délivré mon âme, pour l'empêcher de périr; Vous avez jeté derrière Vous tous mes péchés. \EVERSE}
\newcommand{\isXXXVIIIvXVIIIfr}{\VERSE  Car le séjour des morts ne Vous bénira pas, et la mort ne Vous louera point; ceux qui descendent dans la fosse n'espéreront plus en Votre fidélité. \EVERSE}
\newcommand{\isXXXVIIIvXIXfr}{\VERSE  C'est le vivant, c'est le vivant qui Vous louera, comme je le fais aujourd'hui; le père fera connaître à ses fils Votre vérité. \EVERSE}
\newcommand{\isXXXVIIIvXXfr}{\VERSE  Seigneur, sauvez-moi, et nous chanterons nos cantiques tous les jours de notre vie dans la maison du Seigneur. \EVERSE}
\newcommand{\isXXXVIIIvXXIfr}{\VERSE  Et Isaïe ordonna qu'on prît une masse de figues, et qu'on en fît un cataplasme sur la blessure, afin qu'il fût guéri. \EVERSE}
\newcommand{\isXXXVIIIvXXIIfr}{\VERSE  Et Ezéchias dit : A quel signe saurai-je que j'irai à la maison du Seigneur? \EVERSE}
\newcommand{\isXXXIXvIfr}{\VERSE  En ce temps-là, Mérodach Baladan, fils de Baladan, roi de Babylone, envoya des lettres et des présents à Ezéchias, car il avait appris sa maladie et sa guérison. \EVERSE}
\newcommand{\isXXXIXvIIfr}{\VERSE  Ezéchias en éprouva de la joie, et il montra aux envoyés le lieu où étaient les aromates, l'or et l'argent, les parfums et l'huile précieuse, tout ce qu'il avait de meubles, et tout ce qui se trouvait dans ses trésors. Il n'y eut rien dans son palais, ni dans tout son domaine, qu'Ezéchias ne leur montrât. \EVERSE}
\newcommand{\isXXXIXvIIIfr}{\VERSE  Alors le prophète Isaïe vint auprès du roi Ezéchias, et lui dit : Que t'ont dit ces hommes, et d'où sont-ils venus vers toi? Ezéchias répondit : Ils sont venus vers moi d'un pays lointain, de Babylone. \EVERSE}
\newcommand{\isXXXIXvIVfr}{\VERSE  Isaïe dit encore : Qu'ont-ils vu dans ta maison? Ezéchias répondit : Ils ont vu tout ce qui est dans ma maison; il n'y a rien dans mes trésors que je ne leur aie montré. \EVERSE}
\newcommand{\isXXXIXvVfr}{\VERSE  Isaïe dit à Ezéchias : Ecoute la parole du Seigneur des armées. \EVERSE}
\newcommand{\isXXXIXvVIfr}{\VERSE  Voici, il viendra un temps où tout ce qui est dans ta maison, et ce que tes pères ont amassé jusqu'à ce jour sera emporté à Babylone; il n'en restera rien, dit le Seigneur. \EVERSE}
\newcommand{\isXXXIXvVIIfr}{\VERSE  Et ils prendront de tes fils, qui seront sortis de toi, et que tu auras engendrés, et ils seront eunuques dans le palais du roi de Babylone. \EVERSE}
\newcommand{\isXXXIXvVIIIfr}{\VERSE  Ezéchias répondit à Isaïe : La parole que le Seigneur a dite est bonne. Et il ajouta : Que la paix et la vérité seulement durent pendant mes jours. \EVERSE}
\newcommand{\isXLvIfr}{\VERSE  Consolez-vous, consolez-vous, Mon peuple, dit votre Dieu. \EVERSE}
\newcommand{\isXLvIIfr}{\VERSE  Parlez au coeur de Jérusalem, et dites-lui que ses maux sont finis, que son iniquité est pardonnée, et qu'elle a reçu de la main du Seigneur le double pour tous ses péchés. \EVERSE}
\newcommand{\isXLvIIIfr}{\VERSE  Voix de quelqu'un qui crie dans le désert : Préparez le chemin du Seigneur, rendez droits dans la solitude les sentiers de notre Dieu. \EVERSE}
\newcommand{\isXLvIVfr}{\VERSE  Toute vallée sera exhaussée, toute montagne et toute colline sera abaissée; les chemins tortueux seront redressés, et les raboteux aplanis; \EVERSE}
\newcommand{\isXLvVfr}{\VERSE  et la gloire du Seigneur sera révélée, et toute chair verra en même temps que la bouche du Seigneur a parlé. \EVERSE}
\newcommand{\isXLvVIfr}{\VERSE  Une voix dit : Crie. Et j'ai dit : Que crierai-je? Toute chair est de l'herbe, et toute sa gloire est comme la fleur des champs. \EVERSE}
\newcommand{\isXLvVIIfr}{\VERSE  L'herbe s'est desséchée, et la fleur est tombée, parce que le souffle du Seigneur a soufflé dessus. Le peuple est vraiment de l'herbe; \EVERSE}
\newcommand{\isXLvVIIIfr}{\VERSE  l'herbe s'est desséchée, et la fleur est tombée; mais la parole de notre Seigneur demeure éternellement. \EVERSE}
\newcommand{\isXLvIXfr}{\VERSE  Monte sur une haute montagne, toi qui annonces la bonne nouvelle à Sion; êlève ta voix avec force, toi qui annonces la bonne nouvelle à Jérusalem; élève-la, ne crains point. Dis aux villes de Juda : Voici votre Dieu, \EVERSE}
\newcommand{\isXLvXfr}{\VERSE  Voici que le Seigneur Dieu vient avec puissance, et Son bras dominera; Sa récompense est avec Lui, et Son oeuvre est devant Lui. \EVERSE}
\newcommand{\isXLvXIfr}{\VERSE  Comme un pasteur Il fera paître Son troupeau; Il réunira les agneaux dans Ses bras, et Il les prendra dans Son sein; Il portera Lui-même les brebis pleines. \EVERSE}
\newcommand{\isXLvXIIfr}{\VERSE  Qui a mesuré les eaux dans le creux de Sa main, et qui a pesé les cieux dans Sa paume? qui soutient de trois doigts la masse de la terre? qui a pesé les montagnes avec un poids et les collines dans la balance? \EVERSE}
\newcommand{\isXLvXIIIfr}{\VERSE  Qui a aidé l'Esprit du Seigneur? Qui a été Son conseiller et Lui a montré ce qu'Il devait faire? \EVERSE}
\newcommand{\isXLvXIVfr}{\VERSE  Qui a-t-Il consulté pour en recevoir de l'instruction? Qui Lui a appris le sentier de la justice? Qui Lui a enseigné la science? Qui Lui a montré le chemin de la sagesse? \EVERSE}
\newcommand{\isXLvXVfr}{\VERSE  Voici, les nations sont comme la goutte d'un seau, et comme un grain dans la balance; les îles sont comme une fine poussière. \EVERSE}
\newcommand{\isXLvXVIfr}{\VERSE  Le Liban ne suffirait pas pour le bûcher, et ses animaux ne suffiraient pas pour l'holocauste. \EVERSE}
\newcommand{\isXLvXVIIfr}{\VERSE  Tous les peuples sont devant Lui comme s'ils n'étaient pas, et Il les regarde comme un rien et un néant. \EVERSE}
\newcommand{\isXLvXVIIIfr}{\VERSE  A qui donc ferez-vous ressembler Dieu, et quelle image en tracerez-vous? \EVERSE}
\newcommand{\isXLvXIXfr}{\VERSE  L'ouvrier ne coule-il pas une statue en fonte? L'orfèvre ne la couvre-t-il pas d'or, et celui qui travaille l'argent ne la couvre-t-il pas de lames d'argent? \EVERSE}
\newcommand{\isXLvXXfr}{\VERSE  L'ouvrier habile choisit un bois fort, qui ne pourrisse point; il cherche comment il placera la statue, de sorte qu'elle ne branle pas. \EVERSE}
\newcommand{\isXLvXXIfr}{\VERSE  Ne le savez-vous pas? Ne l'avez-vous pas appris? Cela ne vous a-t-il pas été annoncé dès le commencement? n'avez-vous pas compris comment la terre a été fondée? \EVERSE}
\newcommand{\isXLvXXIIfr}{\VERSE  C'est Lui qui est assis au-dessus du cercle de la terre, et ceux qui l'habitent sont comme des sauterelles; Il étend les cieux comme un rideau, et Il les déploie comme une tente dressée pour y habiter. \EVERSE}
\newcommand{\isXLvXXIIIfr}{\VERSE  Il anéantit ceux qui recherchent les secrets, et Il réduit à rien les juges de la terre. \EVERSE}
\newcommand{\isXLvXXIVfr}{\VERSE  Ils n'avaient pas été plantés ni semés, et leur tronc n'avait pas jeté de racines en terre; tout à coup Il a soufflé sur eux, et ils se sont desséchés, le tourbillon les a emporté comme le chaume. \EVERSE}
\newcommand{\isXLvXXVfr}{\VERSE  A qui M'avez-vous assimilé et égalé? dit le Saint. \EVERSE}
\newcommand{\isXLvXXVIfr}{\VERSE  Levez vos yeux en haut, et voyez qui a créé ces choses, qui fait marcher en ordre l'armée des astres, et qui les appelle tous par leur nom; Il excelle tellement en grandeur, en vertu et en puissance, que pas un d'eux ne fait défaut. \EVERSE}
\newcommand{\isXLvXXVIIfr}{\VERSE  Pourquoi dis-tu, Jacob, pourquoi dis-tu, Israël : Ma voie est cachée au Seigneur, et mon droit passe inaperçu devant mon Dieu? \EVERSE}
\newcommand{\isXLvXXVIIIfr}{\VERSE  Ne le sais-tu pas, ou ne l'as-tu pas appris? Dieu est le Seigneur éternel qui a créé les extrémités de la terre; Il ne Se lasse point, Il ne Se fatigue pas, et Sa sagesse est impénétrable. \EVERSE}
\newcommand{\isXLvXXIXfr}{\VERSE  Il donne de la force à celui qui est fatigué, et Il multiplie la force et la vigueur de ceux qui sont en défaillance. \EVERSE}
\newcommand{\isXLvXXXfr}{\VERSE  Les adolescents se lassent et se fatiguent, et les jeunes gens tombent de faiblesse; \EVERSE}
\newcommand{\isXLvXXXIfr}{\VERSE  mais ceux qui espèrent au Seigneur renouvellent leur force; ils prendront des ailes comme l'aigle, ils courront sans se fatiguer, et ils marcheront sans se lasser. \EVERSE}
\newcommand{\isXLIvIfr}{\VERSE  Que les îles se taisent devant Moi, et que les peuples renouvellent leur force; qu'ils s'approchent, et qu'ensuite ils parlent; allons ensemble devant un juge. \EVERSE}
\newcommand{\isXLIvIIfr}{\VERSE  Qui a fait sortir le juste de l'orient, et qui l'a appelé pour Le suivre? Il lui livrera les nations, et Il lui soumettra les rois; il les donnera à son glaive comme de la poussière, et à son arc comme le chaume que le vent emporte. \EVERSE}
\newcommand{\isXLIvIIIfr}{\VERSE  Il les poursuivra, il passera en paix, la trace de ses pas ne paraîtra point. \EVERSE}
\newcommand{\isXLIvIVfr}{\VERSE  Qui a fait et opéré ces choses? qui appelle les générations dès le commencement? Moi, le Seigneur, Moi qui suis le premier et le dernier. \EVERSE}
\newcommand{\isXLIvVfr}{\VERSE  Les îles ont vu, et elles ont eu peur; les extrémités de la terre ont été frappées de stupeur; elles se sont approchées et elles sont venues. \EVERSE}
\newcommand{\isXLIvVIfr}{\VERSE  Ils s'entr'aideront l'un et l'autre, et chacun dira à son frère : Prends courage. \EVERSE}
\newcommand{\isXLIvVIIfr}{\VERSE  L'ouvrier en airain, frappant du marteau, a encouragé celui qui forgeait alors, en disant : Cela est bon pour souder, et il a fixé l'idole avec des clous, pour qu'elle ne branlât pas. \EVERSE}
\newcommand{\isXLIvVIIIfr}{\VERSE  Mais toi, Israël, Mon serviteur, Jacob que J'ai choisi, race de Mon ami Abraham; \EVERSE}
\newcommand{\isXLIvIXfr}{\VERSE  en qui Je t'ai pris aux extrémités de la terre et Je t'ai appelé d'un pays lointain, et Je t'ai dit : Tu es Mon serviteur, Je t'ai choisi, et Je ne t'ai pas rejeté. \EVERSE}
\newcommand{\isXLIvXfr}{\VERSE  Ne crains point, car Je suis avec toi; ne te détourne pas, car Je suis ton Dieu; Je t'ai fortifié, Je t'ai secouru, et la droite de Mon juste t'a soutenu. \EVERSE}
\newcommand{\isXLIvXIfr}{\VERSE  Voici, tous ceux qui te combattent seront confondus et rougiront de honte, et ceux qui te contredisent seront réduits au néant et périront. \EVERSE}
\newcommand{\isXLIvXIIfr}{\VERSE  Tu les chercheras, ces hommes qui s'opposaient à toi, et tu ne les trouveras plus; ceux qui te faisaient la guerre seront comme s'ils n'étaient pas, et disparaîtront. \EVERSE}
\newcommand{\isXLIvXIIIfr}{\VERSE  Car Je suis le Seigneur ton Dieu, qui te prends par la main, et qui te dis : Ne crains pas, c'est Moi qui t'aide. \EVERSE}
\newcommand{\isXLIvXIVfr}{\VERSE  Ne crains pas, vermisseau de Jacob, ni vous qui êtes morts d'Israël; c'est Moi qui viens te secourir, dit le Seigneur, et le Saint d'Israël est ton rédempteur. \EVERSE}
\newcommand{\isXLIvXVfr}{\VERSE  Je ferai de toi un char neuf à triturer le blé, garni de pointes et de scies; tu écraseras et briseras les montagnes, et tu réduiras les collines en poussière. \EVERSE}
\newcommand{\isXLIvXVIfr}{\VERSE  Tu les vanneras, et le vent les emportera, et la tempête les dissipera; mais toi, tu te réjouiras dans le Seigneur, et tu trouveras tes délices dans le Saint d'Israël. \EVERSE}
\newcommand{\isXLIvXVIIfr}{\VERSE  Les pauvres et les indigents cherchent de l'eau, et il n'y en a point; leur langue est desséchée par la soif. Moi, le Seigneur, Je les exaucerai; Moi, le Dieu d'Israël, Je ne les abandonnerai pas. \EVERSE}
\newcommand{\isXLIvXVIIIfr}{\VERSE  Je ferai jaillir des fleuves au sommet des collines, et des fontaines au milieu des champs; Je changerai le désert en étangs, et la terre sans chemin en courants d'eaux. \EVERSE}
\newcommand{\isXLIvXIXfr}{\VERSE  Je mettrai dans le désert le cèdre, l'épine, le myrte et l'olivier; Je ferai croître ensemble dans la solitude le sapin, l'orme et le buis;  \EVERSE}
\newcommand{\isXLIvXXfr}{\VERSE  afin que tous voient, sachent, considèrent et comprennent que c'est la main du Seigneur qui a fait cela, et que le Saint d'Israël l'a créé. \EVERSE}
\newcommand{\isXLIvXXIfr}{\VERSE  Venez plaider votre cause, dit le Seigneur; si vous avez quelque chose à dire, produisez-le, dit le Roi de Jacob. \EVERSE}
\newcommand{\isXLIvXXIIfr}{\VERSE  Qu'ils s'approchent et qu'ils nous annoncent tout ce qui doit arriver; annoncez les choses passées, et nous y mettrons notre coeur, et nous saurons quelle doit être leur fin; indiquez-nous ce qui doit arriver. \EVERSE}
\newcommand{\isXLIvXXIIIfr}{\VERSE  Annoncez ce qui doit arriver à l'avenir, et nous saurons que vous êtes dieux; faites bien ou mal, si vous le pouvez, afin que nous le disions et que nous le voyions ensemble. \EVERSE}
\newcommand{\isXLIvXXIVfr}{\VERSE  Mais vous venez du néant et votre oeuvre vient de ce qui n'est pas; celui qui vous a choisis est une abomination.  \EVERSE}
\newcommand{\isXLIvXXVfr}{\VERSE  Je l'ai suscité du septentrion, et il viendra de l'orient; il invoquera Mon nom; il traitera les grands comme boue, et comme l'argile que foule le potier. \EVERSE}
\newcommand{\isXLIvXXVIfr}{\VERSE  Qui l'a annoncé dès le commencement, pour que nous le sachions, et dès le début, pour que nous disions : Tu es juste? Mais il n'y a personne qui annonce et qui prédise l'avenir, et qui entende vos paroles. \EVERSE}
\newcommand{\isXLIvXXVIIfr}{\VERSE  Le Seigneur dira le premier à Sion : Les voici, et Je donnerai à Jérusalem un messager de la bonne nouvelle. \EVERSE}
\newcommand{\isXLIvXXVIIIfr}{\VERSE  J'ai regardé, et il n'y avait parmi eux personne qui prît une résolution, ni qui répondît un mot si on l'interrogeait. \EVERSE}
\newcommand{\isXLIvXXIXfr}{\VERSE  Ils sont tous injustes et leurs oeuvres sont vaines; leurs idoles sont du vent et un néant. \EVERSE}
\newcommand{\isXLIIvIfr}{\VERSE  Voici Mon serviteur, Je le soutiendrai; Mon élu en qui Mon âme s'est complue : J'ai mis Mon Esprit sur Lui, Il apportera la justice aux nations. \EVERSE}
\newcommand{\isXLIIvIIfr}{\VERSE  Il ne criera point, Il n'aura pas d'égard aux personnes, et on n'entendra pas Sa voix dans les rues. \EVERSE}
\newcommand{\isXLIIvIIIfr}{\VERSE  Il ne brisera pas le roseau cassé, et Il n'éteindra pas la mèche qui fume encore; Il produira la justice selon la vérité. \EVERSE}
\newcommand{\isXLIIvIVfr}{\VERSE  Il ne sera pas triste, ni précipité, jusqu'à ce qu'Il établisse la justice sur la terre; et les îles attendront Sa loi. \EVERSE}
\newcommand{\isXLIIvVfr}{\VERSE  Voici ce que dit le Seigneur Dieu, qui a créé les cieux et qui les a étendus, qui a affermi la terre avec ce qui en germe; qui donne le souffle au peuple qui vit sur elle, et la respiration à ceux qui y marchent. \EVERSE}
\newcommand{\isXLIIvVIfr}{\VERSE  Moi, le Seigneur, Je T'ai appelé dans la justice, et Je T'ai pris par la main, et Je T'ai gardé, et Je T'ai établi pour l'alliance du peuple et la Lumière des nations, \EVERSE}
\newcommand{\isXLIIvVIIfr}{\VERSE  pour ouvrir les yeux des aveugles, pour tirer des fers celui qui est enchaîné, et de la prison ceux qui sont assis dans les ténèbres. \EVERSE}
\newcommand{\isXLIIvVIIIfr}{\VERSE  Je suis le Seigneur, c'est là Mon nom; Je ne donnerai pas Ma gloire à un autre, ni Mes louanges aux idoles. \EVERSE}
\newcommand{\isXLIIvIXfr}{\VERSE  Les premières choses se sont accomplies, J'en annonce encore de nouvelles; avant qu'elles arrivent Je vous les fais entendre. \EVERSE}
\newcommand{\isXLIIvXfr}{\VERSE  Chantez au Seigneur un cantique nouveau, chantez Sa louange aux extrémités de la terre, vous qui descendez sur la mer et tout ce qui la remplit, vous ses îles et ceux qui les habitent. \EVERSE}
\newcommand{\isXLIIvXIfr}{\VERSE  Que le désert et ses villes élèvent la voix. Cédar habitera dans des maisons; habitants des rochers, louez le Seigneur; que l'on crie du sommet des montagnes. \EVERSE}
\newcommand{\isXLIIvXIIfr}{\VERSE  Ils publieront la gloire du Seigneur, ils annonceront Sa louange dans les îles. \EVERSE}
\newcommand{\isXLIIvXIIIfr}{\VERSE  Le Seigneur sortira comme un héros, Il excitera Son ardeur comme un guerrier; Il élèvera la voix et Il poussera des cris; Il triomphera de Ses ennemis. \EVERSE}
\newcommand{\isXLIIvXIVfr}{\VERSE  Longtemps Je Me suis tu, J'ai gardé le silence, Je Me suis contenu; Je Me ferai entendre comme une femme en travail; Je détruirai et J'anéantirai tout. \EVERSE}
\newcommand{\isXLIIvXVfr}{\VERSE  Je rendrai désertes les montagnes et les collines, Je dessécherai toute leur verdure; Je changerai les fleuves en îles, et Je dessécherai tous les étangs. \EVERSE}
\newcommand{\isXLIIvXVIfr}{\VERSE  Je conduirai les aveugles sur un chemin qu'ils ne connaissent pas, et Je les ferai marcher dans des sentiers qu'ils ignorent; Je changerai devant eux les ténèbres en lumière, et les chemins tortueux en voies droites : Je ferai cela pour eux, et Je ne les abandonnerai pas. \EVERSE}
\newcommand{\isXLIIvXVIIfr}{\VERSE  Ils retourneront en arrière, ils seront couverts de confusion ceux qui se confient aux idoles taillées, qui disent à des images de fonte : Vous êtes nos dieux. \EVERSE}
\newcommand{\isXLIIvXVIIIfr}{\VERSE  Sourds, écoutez; aveugles, regardez et voyez. \EVERSE}
\newcommand{\isXLIIvXIXfr}{\VERSE  Qui est aveugle, sinon Mon serviteur? et qui est sourd, sinon celui à qui J'ai envoyé Mes messagers? Qui est aveugle, sinon celui qui a été vendu? et qui est aveugle, sinon le serviteur du Seigneur? \EVERSE}
\newcommand{\isXLIIvXXfr}{\VERSE  Toi qui vois beaucoup de choses, ne les garderas-tu pas? toi qui as les oreilles ouvertes, n'entendras-tu pas? \EVERSE}
\newcommand{\isXLIIvXXIfr}{\VERSE  Le Seigneur avait voulu sanctifier Son peuple, rendre Sa loi célèbre et la glorifier. \EVERSE}
\newcommand{\isXLIIvXXIIfr}{\VERSE  Et pourtant c'est un peuple pillé et dépouillé; ils sont tous tombés dans les filets des soldats, et ils ont été cachés au fond des prisons; ils ont été mis au pillage, et personne ne les délivre; ils ont été dépouillés et personne ne dit : Restitue. \EVERSE}
\newcommand{\isXLIIvXXIIIfr}{\VERSE  Quel est celui d'entre vous qui écoute ces choses, qui s'y rende attentif, et qui écoute à l'avenir? \EVERSE}
\newcommand{\isXLIIvXXIVfr}{\VERSE  Qui a livré Jacob au pillage, et Israël à ceux qui le dévastent? N'est-ce pas le Seigneur Lui-même que nous avons offensé? car ils n'ont pas voulu marcher dans Ses voies, ni obéir à Sa loi. \EVERSE}
\newcommand{\isXLIIvXXVfr}{\VERSE  Aussi a-t-Il répandu sur lui l'indignation de Sa fureur et la violence de la guerre; Il a allumé un feu autour de lui sans qu'il le sût; Il l'a brûlé sans qu'il le comprit. \EVERSE}
\newcommand{\isXLIIIvIfr}{\VERSE  Et maintenant voici ce que dit le Seigneur qui t'a créé, ô Jacob, et qui t'a formé, ô Israël : Ne crains point, car Je t'ai racheté, et Je t'ai appelé par ton nom; tu es à Moi. \EVERSE}
\newcommand{\isXLIIIvIIfr}{\VERSE  Lorsque tu traverseras les eaux, Je serai avec toi, et les fleuves ne te submergeront pas; lorsque tu marcheras dans le feu, tu ne seras pas brûlé, et la flamme ne t'embrasera pas. \EVERSE}
\newcommand{\isXLIIIvIIIfr}{\VERSE  Car Je suis le Seigneur ton Dieu, le Saint d'Israël, ton sauveur; J'ai donné l'Egypte pour ta rançon, l'Ethiopie et Saba à ta place. \EVERSE}
\newcommand{\isXLIIIvIVfr}{\VERSE  Depuis que tu es devenu précieux et glorieux à Mes yeux, Je t'aime, et Je donnerai des hommes à ta place et des peuples pour ta vie. \EVERSE}
\newcommand{\isXLIIIvVfr}{\VERSE  Ne crains point, car Je suis avec toi; Je ramènerai ta race de l'orient, et Je te rassemblerai de l'occident. \EVERSE}
\newcommand{\isXLIIIvVIfr}{\VERSE  Je dirai à l'aquilon : Donne; et au midi : Ne retiens pas; amène Mes fils des pays lointains, et Mes filles des extrémités de la terre. \EVERSE}
\newcommand{\isXLIIIvVIIfr}{\VERSE  Tous ceux qui invoquent Mon nom, Je les ai créés pour Ma gloire, Je les ai formés et Je les ai faits. \EVERSE}
\newcommand{\isXLIIIvVIIIfr}{\VERSE  Fais sortir le peuple aveugle, qui a des yeux; le peuple sourd, qui a des oreilles. \EVERSE}
\newcommand{\isXLIIIvIXfr}{\VERSE  Que toutes les nations se rassemblent, et que tous les peuples se réunissent. Qui de vous annonce ces choses et qui nous racontera ce qui est arrivé autrefois? Qu'ils produisent leurs témoins; qu'ils se justifient, et on les écoutera, et on dira : C'est vrai. \EVERSE}
\newcommand{\isXLIIIvXfr}{\VERSE  Vous êtes Mes témoins, dit le Seigneur, vous et Mon Serviteur que J'ai choisi; afin que vous sachiez, que vous Me croyiez, et que vous compreniez que c'est Moi-même qui suis; avant Moi il n'a pas été formé de Dieu, et après Moi il n'y en aura pas. \EVERSE}
\newcommand{\isXLIIIvXIfr}{\VERSE  C'est Moi, c'est Moi qui suis le Seigneur, et hors de Moi il n'y a pas de sauveur. \EVERSE}
\newcommand{\isXLIIIvXIIfr}{\VERSE  C'est Moi qui ai annoncé et qui ai sauvé, Je vous ai fait entendre l'avenir, et il n'y a pas eu parmi vous de dieu étranger : vous êtes Mes témoins, dit le Seigneur, et c'est Moi qui suis Dieu. \EVERSE}
\newcommand{\isXLIIIvXIIIfr}{\VERSE  C'est Moi qui suis dès le commencement, et nul ne délivre de Ma main. J'agirai, et qui s'y opposera? \EVERSE}
\newcommand{\isXLIIIvXIVfr}{\VERSE  Voici ce que dit le Seigneur qui vous a rachetés, le Saint d'Israël : J'ai envoyé à cause de vous à Babylone, J'ai fait tomber tous ses appuis et renversé les Chaldéens qui se glorifiaient de leurs vaisseaux. \EVERSE}
\newcommand{\isXLIIIvXVfr}{\VERSE  Je suis le Seigneur, votre Saint, le Créateur d'Israël, votre Roi. \EVERSE}
\newcommand{\isXLIIIvXVIfr}{\VERSE  Voici ce que dit le Seigneur qui a ouvert un chemin dans la mer, et un sentier dans les eaux bouillonnantes; \EVERSE}
\newcommand{\isXLIIIvXVIIfr}{\VERSE  qui mit en campagne les chars et les chevaux, l'armée et le héros; ils se sont endormis ensemble, et ils ne se réveilleront pas; ils furent étouffés et éteints comme une mèche de lin. \EVERSE}
\newcommand{\isXLIIIvXVIIIfr}{\VERSE  Ne vous souvenez plus des choses passées, ne considérez plus ce qui est ancien. \EVERSE}
\newcommand{\isXLIIIvXIXfr}{\VERSE  Voici que Je vais faire des choses nouvelles, elles vont paraître, et vous les connaîtrez; Je mettrai un chemin dans le désert, et des fleuves dans une contrée inaccessible. \EVERSE}
\newcommand{\isXLIIIvXXfr}{\VERSE  Les bêtes sauvages, les dragons et les autruches Me glorifieront, parce que J'ai mis des eaux dans le désert, et des fleuves dans une contrée inaccessible, pour donner à boire à Mon peuple, à Mon élu. \EVERSE}
\newcommand{\isXLIIIvXXIfr}{\VERSE  Je Me suis formé ce peuple, et il publiera Ma louange. \EVERSE}
\newcommand{\isXLIIIvXXIIfr}{\VERSE  Tu ne M'as pas invoqué, Jacob; tu ne t'es pas fatigué pour Moi, Israël. \EVERSE}
\newcommand{\isXLIIIvXXIIIfr}{\VERSE  Tu ne M'as pas offert de bélier en holocauste, et tu ne M'as pas glorifié par tes victimes; Je ne t'ai point contraint en esclavage pour les oblations, et Je ne t'ai pas donné de peine pour l'encens. \EVERSE}
\newcommand{\isXLIIIvXXIVfr}{\VERSE  Tu n'as pas acheté pour Moi à prix d'argent des roseaux odorants, et tu ne M'as pas rassasié par la graisse de tes victimes; mais tu M'as rendu comme esclave par tes péchés, et tu M'as donné de la peine par tes iniquités. \EVERSE}
\newcommand{\isXLIIIvXXVfr}{\VERSE  C'est Moi, c'est Moi-même qui efface tes iniquités pour l'amour de Moi, et Je ne me souviendrai plus de tes péchés. \EVERSE}
\newcommand{\isXLIIIvXXVIfr}{\VERSE  Réveille Ma mémoire et plaidons ensemble; si tu as quelque chose pour te justifier, expose-le. \EVERSE}
\newcommand{\isXLIIIvXXVIIfr}{\VERSE  Ton père a péché le premier, et tes interprètes M'ont désobéi; \EVERSE}
\newcommand{\isXLIIIvXXVIIIfr}{\VERSE  c'est pourquoi J'ai traité en profanes les princes du sanctuaire; J'ai livré Jacob à la boucherie, et Israël à l'opprobre. \EVERSE}
\newcommand{\isXLIVvIfr}{\VERSE  Et maintenant écoute, Jacob Mon serviteur, et toi Israël que J'ai choisi. \EVERSE}
\newcommand{\isXLIVvIIfr}{\VERSE  Voici ce que dit le Seigneur qui t'a fait, qui t'a formé, et qui est ton soutien depuis le sein de ta mère : Ne crains pas, Mon serviteur Jacob, Mon juste que J'ai choisi. \EVERSE}
\newcommand{\isXLIVvIIIfr}{\VERSE  Car Je répandrai des eaux sur le sol altéré, et des fleuves sur la terre desséchée; Je répandrai Mon Esprit sur ta race et Ma bénédiction sur ta postérité; \EVERSE}
\newcommand{\isXLIVvIVfr}{\VERSE  et ils germeront parmi les herbes, comme les saules auprès des eaux courantes. \EVERSE}
\newcommand{\isXLIVvVfr}{\VERSE  L'un dira : Je suis au Seigneur; l'autre se réclamera du nom de Jacob; un autre écrira de sa main : Au Seigneur, et il se glorifiera du nom d'Israël. \EVERSE}
\newcommand{\isXLIVvVIfr}{\VERSE  Voici ce que dit le Seigneur, le Roi d'Israël, et son rédempteur, le Seigneur des armées : Je suis le premier, et Je suis le dernier, et il n'y a pas de Dieu hors de Moi. \EVERSE}
\newcommand{\isXLIVvVIIfr}{\VERSE  Qui est semblable à Moi? Qu'il parle et qu'il prophétise, et qu'il M'expose par ordre ce que J'ai fait depuis que J'ai établi ce peuple antique; qu'il prédise l'avenir et ce qui doit arriver. \EVERSE}
\newcommand{\isXLIVvVIIIfr}{\VERSE  Ne craignez point, et ne vous troublez pas : depuis longtemps Je te l'ai fait savoir, et Je te l'ai annoncé; vous êtes Mes témoins. Y a-t-il un autre Dieu que Moi, et un créateur que Je ne connaisse pas? \EVERSE}
\newcommand{\isXLIVvIXfr}{\VERSE  Tous les fabricants d'idoles ne sont rien, et leurs oeuvres si chères ne leur serviront de rien. Ils sont eux-mêmes témoins qu'elles ne voient pas et ne comprennent pas, afin qu'ils soient confondus. \EVERSE}
\newcommand{\isXLIVvXfr}{\VERSE  Qui est-ce qui forme un dieu, et qui fond une statue qui n'est bonne à rien? \EVERSE}
\newcommand{\isXLIVvXIfr}{\VERSE  Tous ceux qui ont part à ce travail seront confondus, car ces artisans ne sont que des hommes; qu'ils s'assemblent tous, et qu'ils se présentent, et tous ensemble ils seront effrayés et seront couverts de honte. \EVERSE}
\newcommand{\isXLIVvXIIfr}{\VERSE  Le forgeron travaille avec sa lime, il façonne le fer avec le charbon et le marteau; il travaille de toute la force de son bras : il aura faim jusqu'à n'en pouvoir plus, il aura soif et il sera épuisé. \EVERSE}
\newcommand{\isXLIVvXIIIfr}{\VERSE  Le charpentier étend sa règle, il façonne le bois avec le rabot, il le dresse à l'équerre, il lui donne ses traits avec le compas, et il fait l'image d'un homme, comme un bel homme qu'il placera dans une maison. \EVERSE}
\newcommand{\isXLIVvXIVfr}{\VERSE  Il abat des cèdres, il prend une yeuse ou un chêne, qui était debout parmi les arbres de la forêt, il plante un pin que la pluie fait croître. \EVERSE}
\newcommand{\isXLIVvXVfr}{\VERSE  Ces arbres servent à l'homme pour brûler; il en prend et il se chauffe, il en met au feu pour cuire du pain; et de ce qui reste il fait un dieu, et l'adore; il en fait une image devant laquelle il se prosterne. \EVERSE}
\newcommand{\isXLIVvXVIfr}{\VERSE  Il brûle au feu la moitié de ce bois, et de l'autre moitié il fait cuire sa viande, il prépare ses aliments, et se rassasie; il se chauffe et dit : Bon, j'ai chaud, je vois la flamme; \EVERSE}
\newcommand{\isXLIVvXVIIfr}{\VERSE  et avec le reste il se fait un dieu et une idole devant laquelle il se prosterne, qu'il adore et qu'il prie, en disant : Délivre-moi, car tu es mon dieu. \EVERSE}
\newcommand{\isXLIVvXVIIIfr}{\VERSE  Ils ne connaissent et ne comprennent rien; leurs yeux sont couverts, de sorte qu'ils ne voient point, et que leur coeur ne comprend pas. \EVERSE}
\newcommand{\isXLIVvXIXfr}{\VERSE  Ils ne rentrent point en eux-mêmes, ils ne réfléchissent pas, et ils n'ont pas le bon sens de dire : J'en ai brûlé la moitié au feu, et j'ai cuit des pains sur ses charbons; j'ai fait cuire de la viande, que j'ai mangée, et avec le reste je ferais un idole! Je me prosternais devant un tronc d'arbre! \EVERSE}
\newcommand{\isXLIVvXXfr}{\VERSE  Une partie est réduite en cendre; son coeur insensé adore l'autre, et il ne sauve pas son âme, en disant : C'est sans doute un mensonge qui est dans ma main. \EVERSE}
\newcommand{\isXLIVvXXIfr}{\VERSE  Souviens-toi de ceci, Jacob et Israël, parce que tu es Mon serviteur, Je t'ai formé; tu es Mon serviteur, Israël, ne M'oublie pas. \EVERSE}
\newcommand{\isXLIVvXXIIfr}{\VERSE  J'ai effacé tes iniquités comme une nuée, et tes péchés comme un nuage : reviens à Moi, car Je t'ai racheté. \EVERSE}
\newcommand{\isXLIVvXXIIIfr}{\VERSE  Cieux, louez le Seigneur, parce qu'Il a fait miséricorde; extrémités de la terre, soyez dans l'allégresse; montagnes, forêts avec tous vos arbres, faites retentir des louanges, parce que le Seigneur a racheté Jacob, et qu'Il a manifesté Sa gloire en Israël. \EVERSE}
\newcommand{\isXLIVvXXIVfr}{\VERSE  Voici ce que dit le Seigneur qui t'a racheté, et qui t'a formé dès le sein de ta mère : Je suis le Seigneur qui fais tout, qui ai étendu seul les cieux, qui ai affermi la terre sans que personne ne M'aidât; \EVERSE}
\newcommand{\isXLIVvXXVfr}{\VERSE  J'annule les prodiges des devins, Je rends les augures insensés, Je renverse l'esprit des sages, et Je change leur science en folie; \EVERSE}
\newcommand{\isXLIVvXXVIfr}{\VERSE  Je confirme la parole de Mon serviteur, et J'accomplis les oracles de Mes envoyés; Je dis à Jérusalem : Tu seras habitée; et aux villes de Juda : Vous serez rebâties, et Je relèverai leurs ruines. \EVERSE}
\newcommand{\isXLIVvXXVIIfr}{\VERSE  Je dis à l'abîme : Dessèche-toi, Je tarirai tes fleuves. \EVERSE}
\newcommand{\isXLIVvXXVIIIfr}{\VERSE  Je dis à Cyrus : Tu es Mon pasteur, et tu accompliras toute Ma volonté. Je dis à Jérusalem : Tu seras rebâtie; et au temple : Tu seras fondé. \EVERSE}
\newcommand{\isXLVvIfr}{\VERSE  Voici ce que dit le Seigneur à Mon christ Cyrus, que J'ai pris par la main pour lui assujettir les nations, pour mettre les rois en fuite, pour ouvrir devant lui les portes sans qu'aucune lui soit fermée: \EVERSE}
\newcommand{\isXLVvIIfr}{\VERSE  J'irai devant toi, et J'humilierai les grands de la terre; Je romprai les portes d'airain, et Je briserai les gonds de fer; \EVERSE}
\newcommand{\isXLVvIIIfr}{\VERSE  et Je te donnerai des trésors cachés et des richesses enfuies dans le secret, afin que tu saches que Je suis le Seigneur, qui t'ai appelé par ton nom, le Dieu d'Israël; \EVERSE}
\newcommand{\isXLVvIVfr}{\VERSE  à cause de Jacob Mon serviteur, et d'Israël Mon élu, Je t'ai appelé par ton nom; J'ai tracé ton portrait, et tu ne M'as pas connu. \EVERSE}
\newcommand{\isXLVvVfr}{\VERSE  Je suis le Seigneur, et il n'y en a pas d'autre; hors de Moi il n'y a pas de Dieu. Je t'ai ceint, et tu ne M'as pas connu; \EVERSE}
\newcommand{\isXLVvVIfr}{\VERSE  afin que l'on sache, du lever du soleil au couchant, qu'il n'y a pas de Dieu hors de Moi. Je suis le Seigneur, et il n'y en a pas d'autre. \EVERSE}
\newcommand{\isXLVvVIIfr}{\VERSE  Je forme la lumière et Je crée les ténèbres, Je fais la paix et Je crée les maux : Je suis le Seigneur qui fais toutes ces choses. \EVERSE}
\newcommand{\isXLVvVIIIfr}{\VERSE  Cieux, répandez d'en haut votre rosée, et que les nuées fassent pleuvoir le Juste; que la terre s'ouvre, et qu'elle germe le Sauveur, et que la justice naisse en même temps. Moi, le Seigneur, Je l'ai créé. \EVERSE}
\newcommand{\isXLVvIXfr}{\VERSE  Malheur à celui qui dispute contre son Créateur, lui qui n'est qu'un tesson d'argile et de terre. L'argile dit-elle au potier : Que fais-tu? Ton ouvrage n'est pas d'une main habile. \EVERSE}
\newcommand{\isXLVvXfr}{\VERSE  Malheur à celui qui dit à son père : Pourquoi engendres-tu? et à sa mère : Pourquoi enfantes-tu? \EVERSE}
\newcommand{\isXLVvXIfr}{\VERSE  Voici ce que dit le Seigneur, le Saint d'Israël, et Celui qui l'a formé : Interrogez-Moi sur l'avenir; donnez-Moi des ordres au sujet de Mes fils et de l'oeuvre de Mes mains. \EVERSE}
\newcommand{\isXLVvXIIfr}{\VERSE  C'est Moi qui ai fait la terre, et qui ai créé l'homme sur elle; Mes mains ont étendu les cieux, et J'ai imposé des lois à toute leur milice. \EVERSE}
\newcommand{\isXLVvXIIIfr}{\VERSE  C'est Moi qui l'ai suscité pour la justice, et qui aplanirai toutes ses voies; il rebâtira Ma ville, et libérera Mes captifs, sans rançon ni présents, dit le Seigneur, le Dieu des armées. \EVERSE}
\newcommand{\isXLVvXIVfr}{\VERSE  Voici ce que dit le Seigneur : Le travail de l'Egypte, le trafic de l'Ethiopie, et les Sabéens à la taille élevée passeront chez toi, et ils seront à toi; ils marcheront à ta suite, ils viendront les fers aux mains, ils se prosterneront devant toi, et ils te supplieront en disant : Il n'y a de Dieu que chez toi, et hors de toi il n'y a pas de Dieu. \EVERSE}
\newcommand{\isXLVvXVfr}{\VERSE  Vous êtes vraiment un Dieu caché, le Dieu d'Israël, le Sauveur. \EVERSE}
\newcommand{\isXLVvXVIfr}{\VERSE  Ils ont été confondus, ils rougissent tous de honte, et ils sont tous couverts de confusion, les fabricants d'erreurs. \EVERSE}
\newcommand{\isXLVvXVIIfr}{\VERSE  Israël a reçu du Seigneur un salut éternel; vous ne serez pas confondus, et vous ne rougirez pas de honte dans les siècles des siècles. \EVERSE}
\newcommand{\isXLVvXVIIIfr}{\VERSE  Car voici ce que dit le Seigneur qui a créé les cieux, le Dieu qui a formé la terre et qui l'a faite, qui l'a façonnée et qui ne l'a pas créée en vain, mais qui l'a formée pour qu'elle fût habitée : Je suis le Seigneur, et il n'y en a pas d'autre. \EVERSE}
\newcommand{\isXLVvXIXfr}{\VERSE  Je n'ai point parlé en cachette, dans un lieu ténébreux de la terre; Je n'ai point dit en vain à la race de Jacob : Recherchez-Moi; Je suis le Seigneur qui profère la justice et qui annonce la droiture. \EVERSE}
\newcommand{\isXLVvXXfr}{\VERSE  Rassemblez-vous et venez; approchez-vous ensemble, vous qui avez été sauvés des nations; ils sont dans l'ignorance ceux qui portent un bois sculpté par eux, et qui prient un dieu qui ne peut sauver. \EVERSE}
\newcommand{\isXLVvXXIfr}{\VERSE  Enseignez-les et venez, et délibérez ensemble. Qui a annoncé ces choses dès le commencement? qui les a prédites depuis longtemps? N'est-ce pas Moi, le Seigneur, et y a-t-il d'autre Dieu que Moi? Je suis le Dieu juste, et personne ne sauve si ce n'est Moi. \EVERSE}
\newcommand{\isXLVvXXIIfr}{\VERSE  Convertissez-vous à Moi, et vous serez sauvés, peuples de toute la terre, car Je suis Dieu, et il n'y en a pas d'autre. \EVERSE}
\newcommand{\isXLVvXXIIIfr}{\VERSE  J'ai juré par Moi-même; une parole de justice est sortie de Ma bouche, et elle ne sera pas révoquée: \EVERSE}
\newcommand{\isXLVvXXIVfr}{\VERSE  Tout genou fléchira devant Moi, et toute langue jurera par Mon nom. \EVERSE}
\newcommand{\isXLVvXXVfr}{\VERSE  Et l'on dira : Ma justice et ma force résident dans le Seigneur; à Lui viendront, pour être confondus, tous ceux qui s'opposaient à Lui. \EVERSE}
\newcommand{\isXLVvXXVIfr}{\VERSE  Par le Seigneur sera justifiée et glorifiée toute la race d'Israël. \EVERSE}
\newcommand{\isXLVIvIfr}{\VERSE  Bel a été brisé, Nabo a été mis en pièces; leurs idoles ont été placées sur des bêtes et sur des animaux; vos fardeaux les fatiguent par leur grand poids. \EVERSE}
\newcommand{\isXLVIvIIfr}{\VERSE  Elle se sont pourries, et elles ont été mises en pièces; elles n'ont pu sauver ceux qui les portaient, et elles s'en iront elles-mêmes en captivité. \EVERSE}
\newcommand{\isXLVIvIIIfr}{\VERSE  Ecoutez-Moi, maison de Jacob, et vous tous, restes de la maison d'Israël; vous que Je porte dans Mon sein, que Je renferme dans Mes entrailles. \EVERSE}
\newcommand{\isXLVIvIVfr}{\VERSE  Jusqu'à la vieillesse et jusqu'aux cheveux blancs Je vous porterai Moi-même; Je vous ai faits, et Je vous soutiendrai; Je vous porterai et Je vous sauverai. \EVERSE}
\newcommand{\isXLVIvVfr}{\VERSE  A qui M'avez-vous assimilé, et égalé, et comparé, et fait semblable, \EVERSE}
\newcommand{\isXLVIvVIfr}{\VERSE  vous qui tirez l'or de votre bourse, et qui pesez l'argent dans la balance, et qui payez un orfèvre pour qu'il fasse un dieu devant lequel on se prosterne et qu'on l'adore? \EVERSE}
\newcommand{\isXLVIvVIIfr}{\VERSE  Ils le portent sur leurs épaules, et ils le mettent à sa place, et il y demeure et il ne bouge pas de sa place; lorsqu'on criera vers lui, il n'entendra pas, et il ne sauvera pas de l'affliction. \EVERSE}
\newcommand{\isXLVIvVIIIfr}{\VERSE  Souvenez-vous de ces choses, et rougissez-en; rentrez en vous-mêmes, prévaricateurs. \EVERSE}
\newcommand{\isXLVIvIXfr}{\VERSE  Souvenez-vous du temps passé, car Je suis Dieu, et il n'y a pas d'autre Dieu, et nul n'est semblable à Moi. \EVERSE}
\newcommand{\isXLVIvXfr}{\VERSE  J'annonce dès le commencement la fin, et dès le principe, ce qui n'existe pas encore, et Je dis : Ma résolution sera immuable, et toute Ma volonté s'exécutera. \EVERSE}
\newcommand{\isXLVIvXIfr}{\VERSE  J'appelle de l'orient un oiseau, et d'une terre éloignée l'homme de Ma volonté. Je l'ai dit, et Je l'accomplirai; Je l'ai décidé et Je le ferai. \EVERSE}
\newcommand{\isXLVIvXIIfr}{\VERSE  Ecoutez-moi, hommes au coeur dur, qui êtes loin de la justice: \EVERSE}
\newcommand{\isXLVIvXIIIfr}{\VERSE  J'ai fait approcher Ma justice, Je ne la différerai pas, et Mon salut ne tardera pas. Je mettrai le salut dans Sion, et Ma gloire dans Israël. \EVERSE}
\newcommand{\isXLVIIvIfr}{\VERSE  Descends, assieds-toi dans la poussière, vierge fille de Babylone; assieds-toi à terre; il n'y a plus de trône pour la fille des Chaldéens,  on ne l'appellera plus molle et délicate. \EVERSE}
\newcommand{\isXLVIIvIIfr}{\VERSE  Prends la meule, et mouds la farine; dévoile ta honte, découvre ton épaule, montre tes jambes, passe les fleuves. \EVERSE}
\newcommand{\isXLVIIvIIIfr}{\VERSE  Ton ignominie sera découverte, et ton opprobre paraîtra; Je Me vengerai et personne ne Me résistera. \EVERSE}
\newcommand{\isXLVIIvIVfr}{\VERSE  Notre rédempteur, c'est Celui qui a pour nom le Seigneur des armées, le Saint d'Israël. \EVERSE}
\newcommand{\isXLVIIvVfr}{\VERSE  Assieds-toi en silence, et entre dans les ténèbres, fille des Chaldéens, car tu ne seras plus appelée la souveraine des royaumes. \EVERSE}
\newcommand{\isXLVIIvVIfr}{\VERSE  J'étais irrité contre Mon peuple, J'avais profané Mon héritage, et Je les avais livrés entre tes mains, et tu n'as pas eu de compassion pour eux, mais tu as appesanti cruellement ton joug sur le vieillard. \EVERSE}
\newcommand{\isXLVIIvVIIfr}{\VERSE  Et tu as dit : Je serai à jamais souveraine. Tu n'as pas mis ceci dans ton coeur, et tu ne t'es pas souvenue de ta fin. \EVERSE}
\newcommand{\isXLVIIvVIIIfr}{\VERSE  Ecoute maintenant ceci, délicate, toi qui demeures dans la sécurité, qui dis dans ton coeur : C'est moi, et il n'y en a pas d'autre que moi; je ne deviendrai pas veuve, et je ne connaîtrai pas la stérilité. \EVERSE}
\newcommand{\isXLVIIvIXfr}{\VERSE  Ces deux choses viendront tout à coup sur toi en un seul jour, la stérilité et le veuvage; tous ces malheurs viendront sur toi, à cause de la multitude de tes maléfices et de l'extrême dureté de tes enchanteurs. \EVERSE}
\newcommand{\isXLVIIvXfr}{\VERSE  Tu avais confiance dans ta méchanceté, et tu as dit : Il n'y a personne qui me vois. Ta sagesse et ta science même t'ont séduite. Et tu as dit dans ton coeur : C'est moi, et il n'y en a pas d'autre que moi. \EVERSE}
\newcommand{\isXLVIIvXIfr}{\VERSE  Le mal viendra sur toi, et tu ne sauras pas d'où il vient; la calamité fondra sur toi, et tu ne pourras t'en défendre; il viendra tout à coup sur toi une misère que tu n'auras pas prévue. \EVERSE}
\newcommand{\isXLVIIvXIIfr}{\VERSE  Reste avec tes enchanteurs, et avec la multitude de tes maléfices auxquels tu t'es appliquée depuis ta jeunesse, et vois si tu en tireras quelque avantage, ou si tu peux devenir plus forte. \EVERSE}
\newcommand{\isXLVIIvXIIIfr}{\VERSE  Tu t'es fatiguée par la multitude de tes conseillers. Qu'ils se lèvent et qu'ils te sauvent, ces augures du ciel qui contemplent les astres, et qui comptent les mois pour t'annoncer d'après cela ce qui doit t'arriver. \EVERSE}
\newcommand{\isXLVIIvXIVfr}{\VERSE  Ils sont devenus comme la paille, le feu les a dévorés; ils ne délivreront pas leur vie de la flamme; ce ne sera pas du charbon dont on se chauffe, ni un feu auprès duquel on s'assied. \EVERSE}
\newcommand{\isXLVIIvXVfr}{\VERSE  Voilà ce que deviendront toutes ces choses auxquelles tu t'étais fatiguée. Ceux avec qui tu as trafiqué depuis ta jeunesse se disperseront chacun de son côté, et il n'y aura personne pour te sauver. \EVERSE}
\newcommand{\isXLVIIIvIfr}{\VERSE  Ecoutez ceci, maison de Jacob, vous qui portez le nom d'Israël, qui êtes sortis des eaux de Juda, qui jurez au nom du Seigneur, qui vous souvenez du Dieu d'Israël, mais sans vérité et sans justice. \EVERSE}
\newcommand{\isXLVIIIvIIfr}{\VERSE  Car ils prennent leur nom de la ville sainte, et ils s'appuient sur le Dieu d'Israël, qui a pour nom le Seigneur des armées. \EVERSE}
\newcommand{\isXLVIIIvIIIfr}{\VERSE  Je vous ai annoncé longtemps d'avance les premiers événements, ils sont sortis de Ma bouche, et Je les ai publiés; soudain J'ai agi, et ils ont eu lieu. \EVERSE}
\newcommand{\isXLVIIIvIVfr}{\VERSE  Car Je savais que tu es endurci, que ton cou est une barre de fer, et que tu as un front d'airain. \EVERSE}
\newcommand{\isXLVIIIvVfr}{\VERSE  Je t'ai prédit ces faits longtemps d'avance; Je te les ai indiqués avant leur accomplissement, de peur que tu ne dises : Ce sont mes idoles qui ont fait cela, ce sont mes images taillées et coulées en fonte qui l'ont ainsi ordonné. \EVERSE}
\newcommand{\isXLVIIIvVIfr}{\VERSE  Tout ce que tu as entendu, vois-le; mais vous l'avez-vous annoncé? Je t'apprends maintenant des choses nouvelles, que J'ai réservées, et qui te sont inconnues. \EVERSE}
\newcommand{\isXLVIIIvVIIfr}{\VERSE  C'est maintenant qu'elles sont créées et non d'autrefois, et avant ce jour tu n'en as pas entendu parler, de peur que tu ne dise : Je les connaissais. \EVERSE}
\newcommand{\isXLVIIIvVIIIfr}{\VERSE  Tu ne les as ni entendues ni connues, et ton oreille n'a pas été ouverte depuis longtemps à leur sujet; car Je sais que tu seras certainement un prévaricateur, et dès le sein de ta mère Je t'ai appellé transgresseur. \EVERSE}
\newcommand{\isXLVIIIvIXfr}{\VERSE  A cause de Mon nom J'éloignerai de toi Ma fureur, et pour Ma gloire Je te réfrénerai, pour que tu ne périsses pas. \EVERSE}
\newcommand{\isXLVIIIvXfr}{\VERSE  Je t'ai purifié par le feu, mais non comme l'argent; Je t'ai choisi dans la fournaise de la pauvreté. \EVERSE}
\newcommand{\isXLVIIIvXIfr}{\VERSE  C'est pour Moi-même, pour Moi-même, que J'agirai, afin que Je ne sois pas blasphémé, et Je ne donnerai pas Ma gloire à un autre. \EVERSE}
\newcommand{\isXLVIIIvXIIfr}{\VERSE  Ecoute-Moi, Jacob, et toi, Israël, que J'appelle; c'est Moi, Moi-même, qui suis le premier et qui suis le dernier. \EVERSE}
\newcommand{\isXLVIIIvXIIIfr}{\VERSE  C'est Ma main qui a fondé la terre, et Ma droite qui a mesuré les cieux; Je les appellerai, et ils se présenteront ensemble. \EVERSE}
\newcommand{\isXLVIIIvXIVfr}{\VERSE  Rassemblez-vous tous, et écoutez : Qui d'entre eux a annoncé ces choses? Le Seigneur l'a aimé, Il exécutera Sa volonté dans Babylone, et Son bras frappera sur les Chaldéens. \EVERSE}
\newcommand{\isXLVIIIvXVfr}{\VERSE  C'est Moi, c'est Moi qui a parlé; Je l'ai appelé, Je l'ai amené, et J'ai aplani sa voie. \EVERSE}
\newcommand{\isXLVIIIvXVIfr}{\VERSE  Approchez-vous de Moi, et écoutez ceci : Dès le commencement Je n'ai point parlé en cachette, dès l'origine, avant que ces choses se fissent, J'étais là; et maintenant le Seigneur Dieu m'a envoyé avec Son Esprit. \EVERSE}
\newcommand{\isXLVIIIvXVIIfr}{\VERSE  Voici ce que dit le Seigneur qui t'a racheté, le Saint d'Israël : Je suis le Seigneur ton Dieu, qui t'enseigne ce qui est utile, et qui te conduit dans la voie par laquelle tu marches. \EVERSE}
\newcommand{\isXLVIIIvXVIIIfr}{\VERSE  Oh! si tu avais été attentif à Mes préceptes, ta paix serait comme un fleuve, et ta justice comme les flots de la mer; \EVERSE}
\newcommand{\isXLVIIIvXIXfr}{\VERSE  ta postérité serait comme le sable, et les fruits de ton sein comme les grains de sable; ton nom n'aurait pas péri, et n'aurait point été effacé de devant Mes yeux. \EVERSE}
\newcommand{\isXLVIIIvXXfr}{\VERSE  Sortez de Babylone, fuyez du milieu des Chaldéens; faites entendre cette nouvelle, et publiez-la jusqu'aux extrémités de la terre. Dites : Le Seigneur a racheté Son serviteur Jacob. \EVERSE}
\newcommand{\isXLVIIIvXXIfr}{\VERSE  Ils n'ont pas souffert la soif dans le désert lorsqu'Il les a conduits; Il leur a tiré l'eau du rocher; Il a ouvert la pierre, et les eaux ont coulé. \EVERSE}
\newcommand{\isXLVIIIvXXIIfr}{\VERSE  Il n'y a pas de paix pour les impies, dit le Seigneur. \EVERSE}
\newcommand{\isXLIXvIfr}{\VERSE  Ecoutez, îles, et vous, peuples lointains, soyez attentifs. Le Seigneur M'a appelé dès le sein de Ma mère; lorsque J'étais encore dans ses entrailles, il s'est souvenu de Mon nom. \EVERSE}
\newcommand{\isXLIXvIIfr}{\VERSE  Il a rendu Ma bouche semblable à un glaive acéré, Il M'a protégé à l'ombre de Sa main; Il a fait de Moi comme une flèche choisie, Il M'a caché dans Son carquois. \EVERSE}
\newcommand{\isXLIXvIIIfr}{\VERSE  Et Il m'a dit : Tu es Mon serviteur, Israël, et Je Me glorifierai en Toi. \EVERSE}
\newcommand{\isXLIXvIVfr}{\VERSE  Et Moi J'ai dit : C'est en vain que J'ai travaillé, c'est inutilement et sans fruit que J'ai consumé Ma force; mais Mon droit est auprès du Seigneur, et Ma récompense auprès de Mon Dieu. \EVERSE}
\newcommand{\isXLIXvVfr}{\VERSE  Et maintenant le Seigneur dit, Lui qui M'a formé dès le sein de Ma mère pour être Son Serviteur, pour ramener à Lui Jacob, et quoique Israël ne se réunisse point à Lui, Je serai glorifié aux yeux du Seigneur, et Mon Dieu deviendra Ma force. \EVERSE}
\newcommand{\isXLIXvVIfr}{\VERSE  Il dit : C'est peu que Tu sois Mon serviteur pour relever les tribus de Jacob, et pour convertir les restes d'Israël; Je T'ai établi pour être la Lumière des nations, et Mon salut jusqu'à l'extrémité de la terre. \EVERSE}
\newcommand{\isXLIXvVIIfr}{\VERSE  Voici ce que dit le Seigneur, le rédempteur, le Saint d'Israël, à l'âme méprisée, à la nation détestée, à l'esclave des puissants : Les rois verront et les princes se lèveront, et ils adoreront, à cause du Seigneur qui a été fidèle, et du Saint d'Israël qui t'a choisi. \EVERSE}
\newcommand{\isXLIXvVIIIfr}{\VERSE  Voici ce que dit le Seigneur : Au temps favorable Je T'ai exaucé, et au jour du salut Je T'ai secouru; Je T'ai conservé, et Je T'ai établi pour l'alliance du peuple, pour relever le pays, pour posséder les héritages dissipés; \EVERSE}
\newcommand{\isXLIXvIXfr}{\VERSE  pour dire à ceux qui sont dans les chaînes : Sortez; et à ceux qui sont dans les ténèbres : paraissez. Ils paîtront sur les chemins, et toutes les plaines leur serviront de pâturages. \EVERSE}
\newcommand{\isXLIXvXfr}{\VERSE  Ils n'auront plus ni faim ni soif; la chaleur et le soleil ne les frapperont plus, car Celui qui a pitié d'eux les conduira et les mènera boire aux sources des eaux. \EVERSE}
\newcommand{\isXLIXvXIfr}{\VERSE  Alors Je changerai toutes Mes montagnes en chemin, et Mes sentiers seront exhaussés. \EVERSE}
\newcommand{\isXLIXvXIIfr}{\VERSE  Voici, ceux-là viennent de loin, et ceux-ci du septentrion et du couchant, et les autres de la terre du midi. \EVERSE}
\newcommand{\isXLIXvXIIIfr}{\VERSE  Cieux, louez-Le; terre, sois dans l'allégresse; montagnes, faites retentir Sa louange, car le Seigneur consolera Son peuple, et Il aura pitié de ses pauvres. \EVERSE}
\newcommand{\isXLIXvXIVfr}{\VERSE  Cependant Sion a dit : Le Seigneur m'a abandonnée, et le Seigneur m'a oubliée. \EVERSE}
\newcommand{\isXLIXvXVfr}{\VERSE  Une femme peut-elle oublier son enfant, et n'avoir pas pitié du fils de ses entrailles? Mais quand elle l'oublierait, Moi Je ne t'oublierai pas. \EVERSE}
\newcommand{\isXLIXvXVIfr}{\VERSE  Voici, Je t'ai gravée sur Mes mains; tes murs sont toujours devant Mes yeux. \EVERSE}
\newcommand{\isXLIXvXVIIfr}{\VERSE  Ceux qui doivent te rebâtir sont venus; ceux qui t'ont détruite et dévastée sortiront du milieu de toi. \EVERSE}
\newcommand{\isXLIXvXVIIIfr}{\VERSE  Lève les yeux tout autour, et vois : tous ceux-ci se sont rassemblés et sont venus pour toi. Par Ma vie, dit le Seigneur, tu te revêtiras d'eux tous comme d'une parure, et tu t'en ceindras comme une épouse; \EVERSE}
\newcommand{\isXLIXvXIXfr}{\VERSE  car tes déserts, tes solitudes et ton pays ruiné seront désormais trop étroits pour leurs habitants, et ceux qui te dévoraient seront chassés loin de toi. \EVERSE}
\newcommand{\isXLIXvXXfr}{\VERSE  Les enfants de ta stérilité diront à tes oreilles : L'espace est trop étroit pour moi; fais-moi de la place pour que j'y habite. \EVERSE}
\newcommand{\isXLIXvXXIfr}{\VERSE  Et tu diras dans ton coeur : Qui me les a engendrés? car j'étais stérile et je n'enfantais point, j'étais exilée et captive. Et qui les a nourris? car j'étais seule et abandonnée; et ceux-ci, où étaient-ils? \EVERSE}
\newcommand{\isXLIXvXXIIfr}{\VERSE  Voici ce que dit le Seigneur Dieu : Je lèverai la main vers les nations, et Je dresserai Mon étendard vers les peuples. Et ils ramèneront tes fils entre leurs bras, et ils porteront tes filles sur leurs épaules. \EVERSE}
\newcommand{\isXLIXvXXIIIfr}{\VERSE  Les rois seront tes nourriciers, et les reines tes nourrices; ils t'adoreront en baisant le visage contre terre, et ils lécheront la poussière de tes pieds. Et tu sauras que Je suis le Seigneur, et que ceux qui M'attendent ne seront pas confondus. \EVERSE}
\newcommand{\isXLIXvXXIVfr}{\VERSE  Peut-on ravir au puissant sa proie, et enlever à un homme robuste ceux qu'il a rendus captifs? \EVERSE}
\newcommand{\isXLIXvXXVfr}{\VERSE  Mais voici ce que dit le Seigneur : Oui, les captifs du puissant lui seront ravis; et ceux que l'homme robuste avait pris lui seront enlevés. Je jugerai ceux qui t'avaient jugée, et Je sauverai tes fils. \EVERSE}
\newcommand{\isXLIXvXXVIfr}{\VERSE  Je ferai manger à tes ennemis leur propre chair; ils seront enivrés de leur sang comme d'un vin nouveau; et toute chair saura que Je suis le Seigneur qui te sauve, et que le Fort de Jacob est ton rédempteur. \EVERSE}
\newcommand{\isLvIfr}{\VERSE  Voici ce que dit le Seigneur : Quel est cet acte de divorce, par lequel J'ai répudié votre mère? où quel est ce créancier auquel Je vous ai vendue? Voici, c'est à cause de vos iniquités que vous avez été vendus, et c'est à cause de vos crimes que j'ai répudié votre mère. \EVERSE}
\newcommand{\isLvIIfr}{\VERSE  Car Je suis venu, et il n'y avait personne; J'ai appelé, et personne n'entendait. Ma main est-elle devenue trop courte et trop petite pour pouvoir racheter? ou n'ai-Je pas assez de force pour vous délivrer? par une seule menace Je tarirai la mer, Je mettrai les fleuves à sec; les poissons, n'ayant plus d'eau, pourriront et mourront de soif. \EVERSE}
\newcommand{\isLvIIIfr}{\VERSE  J'envelopperai les cieux de ténèbres, et Je les couvrirai d'un sac. \EVERSE}
\newcommand{\isLvIVfr}{\VERSE  Le Seigneur M'a donné une langue savante, afin que Je puisse soutenir par la parole celui qui est abattu. Il éveille le matin, le matin il éveille Mon oreille, afin que Je L'écoute comme un maître. \EVERSE}
\newcommand{\isLvVfr}{\VERSE  Le Seigneur Dieu M'a ouvert l'oreille, et Je ne Le contredis pas; Je ne Me suis point retiré en arrière. \EVERSE}
\newcommand{\isLvVIfr}{\VERSE  J'ai abandonné Mon corps a ceux qui Me frappaient, et Mes joues à ceux qui M'arrachaient la barbe; Je n'ai pas détourné Mon visage de ceux qui Me couvraient d'injures et de crachats. \EVERSE}
\newcommand{\isLvVIIfr}{\VERSE  Le Seigneur Dieu est Mon protecteur; c'est pourquoi Je n'ai pas été confondu; c'est pourquoi J'ai rendu Mon visage semblable à une pierre très dure, et Je sais que Je ne serai pas confondu. \EVERSE}
\newcommand{\isLvVIIIfr}{\VERSE  Celui qui Me justifie est proche; qui se déclarera contre Moi? Comparaissons ensemble; qui est Mon adversaire? qu'il s'approche de Moi. \EVERSE}
\newcommand{\isLvIXfr}{\VERSE  Le Seigneur Dieu est Mon protecteur; quel est celui qui Me condamnera? Voici, ils s'useront tous comme un vêtement; ils seront mangés des vers. \EVERSE}
\newcommand{\isLvXfr}{\VERSE  Qui d'entre vous craint le Seigneur, et entend la voix de Son serviteur? que celui qui marche dans les ténèbres, et qui n'a pas de lumière, espère au nom du Seigneur, et qu'il s'appuie sur son Dieu. \EVERSE}
\newcommand{\isLvXIfr}{\VERSE  Mais vous tous qui allumez un feu, et qui êtes environnés de flammes, marchez à la lumière de votre feu, et dans les flammes que vous avez allumées : C'est par Ma main que cela vous est arrivé; vous dormirez dans les douleurs. \EVERSE}
\newcommand{\isLIvIfr}{\VERSE  Ecoutez-moi, vous qui suivez la justice, et qui cherchez le Seigneur; regardez le rocher dont vous avez été taillés, et la carrière profonde dont vous avez été tirés. \EVERSE}
\newcommand{\isLIvIIfr}{\VERSE  Regardez Abraham votre père, et Sara qui vous a enfantés; Je l'ai appelé lorsqu'il était seul, Je l'ai béni et multiplié. \EVERSE}
\newcommand{\isLIvIIIfr}{\VERSE  Le Seigneur consolera donc Sion et Il consolera toutes ses ruines; Il changera son désert en délices, et sa solitude en un jardin du Seigneur. La joie et l'allégresse se trouveront en elle, l'action de grâces et la voix des cantiques. \EVERSE}
\newcommand{\isLIvIVfr}{\VERSE  Regardez-Moi, Mon peuple; Ma nation, écoutez-Moi; car la loi sortira de Moi, et Ma justice se reposera parmi Mon peuple et sera leur lumière. \EVERSE}
\newcommand{\isLIvVfr}{\VERSE  Mon Juste est proche, Mon Sauveur va paraître, et Mes bras jugeront les peuples; les îles M'attendront, elles attendront Mon bras. \EVERSE}
\newcommand{\isLIvVIfr}{\VERSE  Levez les yeux au ciel, et regardez en bas sur la terre; car le ciel se dissoudra comme la fumée, la terre sera usée comme un vêtement, et ceux qui l'habitent périront avec elle; mais Mon salut sera éternel, et Ma justice ne fera pas défaut. \EVERSE}
\newcommand{\isLIvVIIfr}{\VERSE  Ecoutez-Moi, vous qui connaissez le Juste, Mon peuple, qui avez Ma loi dans vos coeurs; ne craignez pas l'opprobre des hommes, et ne redoutez pas leurs blasphèmes; \EVERSE}
\newcommand{\isLIvVIIIfr}{\VERSE  car les vers les dévoreront comme un vêtement, et la teigne les rongera comme la laine; mais Mon salut sera éternel, et Ma justice subsistera de génération en génération. \EVERSE}
\newcommand{\isLIvIXfr}{\VERSE  Elevez-vous, élevez-vous, revêtez-vous de force, bras du Seigneur; élevez-vous comme aux anciens jours, dans les siècles passés. N'est-ce pas vous qui avez frappé le superbe, qui avez blessé le dragon? \EVERSE}
\newcommand{\isLIvXfr}{\VERSE  N'est-ce pas vous qui avez séché la mer, l'eau de l'impétueux abîme; qui avez fait au fond de la mer un chemin pour faire passer vos affranchis? \EVERSE}
\newcommand{\isLIvXIfr}{\VERSE  C'est ainsi que les rachetés du Seigneur reviendront; ils viendront à Sion avec des chants de louange, et une joie éternelle couronnera leurs têtes; ils seront dans la joie et le ravissement; la douleur et les gémissements s'enfuiront. \EVERSE}
\newcommand{\isLIvXIIfr}{\VERSE  C'est Moi, c'est Moi-même qui vous consolerai. Qui es-tu pour avoir peur d'un homme mortel, et du fils de l'homme qui séchera comme l'herbe? \EVERSE}
\newcommand{\isLIvXIIIfr}{\VERSE  Et tu as oublié le Seigneur qui t'a créé, qui a étendu les cieux et fondé la terre, et tu as tremblé sans cesse tout le jour devant la fureur de celui qui t'affligeait, et qui était prêt à te perdre! Où est maintenant la furie de celui qui t'affligeait? \EVERSE}
\newcommand{\isLIvXIVfr}{\VERSE  Bientôt Celui qui doit ouvrir arrivera; Il ne détruira pas jusqu'à l'extermination, et Son pain ne manquera pas. \EVERSE}
\newcommand{\isLIvXVfr}{\VERSE  C'est Moi qui suis le Seigneur ton Dieu, qui trouble la mer et qui fais soulever ses flots; Mon nom est le Seigneur des armées. \EVERSE}
\newcommand{\isLIvXVIfr}{\VERSE  J'ai mis Mes paroles dans ta bouche, et Je t'ai mis à couvert sous l'ombre de Ma main, pour établir les cieux et fonder la terre, et pour dire à Sion : Tu es Mon peuple. \EVERSE}
\newcommand{\isLIvXVIIfr}{\VERSE  Réveille-toi, réveille-toi, lève-toi, Jérusalem, qui as bu de la main du Seigneur la coupe de Sa colère; tu as bu jusqu'au fond la coupe d'assoupissement, et tu l'as vidée jusqu'à la lie. \EVERSE}
\newcommand{\isLIvXVIIIfr}{\VERSE  De tous les fils qu'elle a enfantés il n'en est aucun qui la soutienne, et de tous les fils qu'elle a nourris aucun ne lui prend la main. \EVERSE}
\newcommand{\isLIvXIXfr}{\VERSE  Ces deux choses te sont arrivées; qui s'attristera sur toi? Le ravage et la ruine, la faim et le glaive; qui te consolera? \EVERSE}
\newcommand{\isLIvXXfr}{\VERSE  Tes fils ont été jetés à terre; ils se sont endormis à la tête de toutes les rues comme un oryx pris au filet, pleins de l'indignation du Seigneur, des menaces de ton Dieu. \EVERSE}
\newcommand{\isLIvXXIfr}{\VERSE  C'est pourquoi écoute ceci, pauvre petite, qui es enivrée, mais non de vin. \EVERSE}
\newcommand{\isLIvXXIIfr}{\VERSE  Voici ce que dit ton dominateur, ton Seigneur et ton Dieu, qui combattra pour Son peuple : Voici, J'enlève de ta main la coupe d'assoupissement, le fond de la coupe de Mon indignation; tu n'en boiras plus à l'avenir. \EVERSE}
\newcommand{\isLIvXXIIIfr}{\VERSE  Je la mettrai dans la main de ceux qui t'ont humiliée, et qui ont dit à ton âme : Courbe-toi, afin que nous passions; et tu as fait de ton corps comme une terre, et comme un chemin pour les passants. \EVERSE}
\newcommand{\isLIIvIfr}{\VERSE  Lève-toi, lève-toi, revêts-toi de ta force, Sion; revêts-toi des vêtements de ta gloire, Jérusalem, ville du Saint, car à l'avenir l'incirconcis et l'impur ne te traversera plus. \EVERSE}
\newcommand{\isLIIvIIfr}{\VERSE  Secoue la poussière, lève-toi, assieds-toi, Jérusalem, détache les chaînes de ton cou, captive, fille de Sion, \EVERSE}
\newcommand{\isLIIvIIIfr}{\VERSE  car voici ce que dit le Seigneur : Vous avez été vendus pour rien, et vous serez rachetés sans argent. \EVERSE}
\newcommand{\isLIIvIVfr}{\VERSE  Car voici ce que dit le Seigneur Dieu : Mon peuple descendit autrefois en Egypte pour y habiter, et Assur l'a opprimé sans aucun sujet. \EVERSE}
\newcommand{\isLIIvVfr}{\VERSE  Et maintenant qu'ai-Je à faire ici, dit le Seigneur, puisque Mon peuple a été enlevé sans raison? Ses oppresseurs agissent injustement, et Mon nom est sans cesse blasphémé tout le jour. \EVERSE}
\newcommand{\isLIIvVIfr}{\VERSE  C'est pourquoi Mon peuple connaîtra Mon nom en ce jour-là, car Moi qui parlais, Me voici. \EVERSE}
\newcommand{\isLIIvVIIfr}{\VERSE  Qu'ils sont beaux sur les montagnes les pieds de celui qui annonce et prêche la paix, qui annonce la bonne nouvelle, qui prêche le salut, qui dit à Sion : Ton Dieu va régner! \EVERSE}
\newcommand{\isLIIvVIIIfr}{\VERSE  La voix de tes sentinelles retentit, elles élèvent la voix, elles chantent ensemble des cantiques de louanges, car elles voient de leurs yeux que le Seigneur ramène Sion. \EVERSE}
\newcommand{\isLIIvIXfr}{\VERSE  Réjouissez-vous et louez ensemble le Seigneur, déserts de Jérusalem, parce qu'Il a consolé Son peuple et qu'Il a racheté Jérusalem. \EVERSE}
\newcommand{\isLIIvXfr}{\VERSE  Le Seigneur a fait voir Son bras saint aux yeux de toutes les nations, et toutes les extrémités de la terre verront le salut de notre Dieu. \EVERSE}
\newcommand{\isLIIvXIfr}{\VERSE  Retirez-vous, retirez-vous; sortez de là, ne touchez rien d'impur; sortez du milieu d'elle; purifiez-vous, vous qui portez les vases du Seigneur. \EVERSE}
\newcommand{\isLIIvXIIfr}{\VERSE  Vous ne sortirez pas en tumulte, ni par une fuite précipitée, car le Seigneur marchera devant vous, et le Dieu d'Israël vous rassemblera. \EVERSE}
\newcommand{\isLIIvXIIIfr}{\VERSE  Voici, Mon serviteur agira avec intelligence, Il sera grand et élevé, et au comble de la gloire. \EVERSE}
\newcommand{\isLIIvXIVfr}{\VERSE  De même que beaucoup ont été stupéfaits à ton sujet, ainsi Son aspect sera sans gloire parmi les hommes, et Sa forme méprisable parmi les fils des hommes. \EVERSE}
\newcommand{\isLIIvXVfr}{\VERSE  Il arrosera des nations nombreuses, devant lui les rois fermeront la bouche; car ceux auxquels Il n'avait pas été annoncé le verront, et ceux qui n'avaient pas entendu parler de Lui Le contempleront. \EVERSE}
\newcommand{\isLIIIvIfr}{\VERSE  Qui a cru à ce que nous avons entendu? et à qui le bras du Seigneur a-t-il été révélé? \EVERSE}
\newcommand{\isLIIIvIIfr}{\VERSE  Il s'élèvera devant Lui comme un arbrisseau, et comme un rejeton qui sort d'une terre desséchée, Il n'a ni beauté ni éclat; nous L'avons vu, et Il n'avais pas d'apparence, et nous L'avons méconnu. \EVERSE}
\newcommand{\isLIIIvIIIfr}{\VERSE  Il était méprisé, le dernier des hommes, un homme de douleurs, qui connaît la souffrance; Son visage était caché; Il était méprisé, et nous n'avons fait aucun cas de Lui. \EVERSE}
\newcommand{\isLIIIvIVfr}{\VERSE  Vraiment Il a porté nos langueurs, et Il S'est chargé Lui-même de nos douleurs; et nous L'avons considéré comme un lépreux, comme un homme frappé de Dieu et humilié. \EVERSE}
\newcommand{\isLIIIvVfr}{\VERSE  Et cependant Il a été blessé pour nos iniquités, Il a été brisé pour nos crimes; le châtiment qui nous procure la paix est tombé sur Lui, et nous avons été guéris par Ses meurtrissures. \EVERSE}
\newcommand{\isLIIIvVIfr}{\VERSE  Nous étions tous errants comme des brebis, chacun s'était détourné sur sa propre voie, et le Seigneur a placé sur Lui l'iniquité de nous tous. \EVERSE}
\newcommand{\isLIIIvVIIfr}{\VERSE  Il a été offert parce que Lui-même l'a voulu, et Il n'a pas ouvert la bouche; comme une brebis qu'on mène à la boucherie, comme un agneau devant celui qui le tond, Il a gardé le silence et Il n'a pas ouvert la bouche. \EVERSE}
\newcommand{\isLIIIvVIIIfr}{\VERSE  Il a été enlevé par l'angoisse et le jugement. Qui racontera Sa génération? car Il a été retranché de la terre des vivants. Je L'ai frappé pour les crimes de Mon peuple. \EVERSE}
\newcommand{\isLIIIvIXfr}{\VERSE  Et Il donnera les impies pour prix de Sa sépulture, et les riches pour prix de Sa mort, parce qu'Il n'a pas commis d'iniquité, et que le mensonge n'a pas été dans Sa bouche. \EVERSE}
\newcommand{\isLIIIvXfr}{\VERSE  Mais le Seigneur a voulu le briser par la souffrance; s'Il livre Son âme pour le péché, Il verra une longue postérité, et la volonté du Seigneur sera dirigée heureusement par Sa main. \EVERSE}
\newcommand{\isLIIIvXIfr}{\VERSE  Parce que Son âme aura souffert, Il verra et sera rassasié. Par Sa science, Mon juste serviteur justifiera beaucoup d'hommes, et Il portera sur Lui leurs iniquités. \EVERSE}
\newcommand{\isLIIIvXIIfr}{\VERSE  C'est pourquoi Je Lui donnerai une grande multitude pour partage, et Il distribuera les dépouilles des forts, parce qu'Il a livré Son âme à la mort, et qu'Il a été mis au nombre des scélérats, qu'Il a porté les péchés de beaucoup d'hommes et qu'Il a prié pour les pécheurs. \EVERSE}
\newcommand{\isLIVvIfr}{\VERSE  Réjouis-toi, stérile qui n'enfantes pas; chante des cantiques de louanges, et pousse des cris de joie, toi qui n'avais pas d'enfants, car la délaissée a plus d'enfants que celle qui avait un mari, dit le Seigneur. \EVERSE}
\newcommand{\isLIVvIIfr}{\VERSE  Elargis l'espace de ta tente; étends les peaux de tes tabernacles, ne les épargne pas; allonge tes cordages, et affermis tes pieux. \EVERSE}
\newcommand{\isLIVvIIIfr}{\VERSE  Car tu t'étendras à droite et à gauche; ta postérité aura les nations pour héritage, et elle habitera les villes désertes. \EVERSE}
\newcommand{\isLIVvIVfr}{\VERSE  Ne crains point, car tu ne sera pas confondue, et tu ne rougiras pas; tu n'auras plus de honte, car tu oublieras la confusion de ta jeunesse, et tu ne te souviendras plus de l'opprobre de ton veuvage. \EVERSE}
\newcommand{\isLIVvVfr}{\VERSE  Car Celui qui t'a créée sera ton maître, Son nom est le Seigneur des armées; et ton rédempteur, le Saint d'Israël, S'appellera le Dieu de toute la terre. \EVERSE}
\newcommand{\isLIVvVIfr}{\VERSE  Car le Seigneur t'a appelée comme une femme délaissée et à l'esprit désolé, et comme une femme répudée dès sa jeunesse, dit ton Dieu. \EVERSE}
\newcommand{\isLIVvVIIfr}{\VERSE  Je t'ai abandonnée pour un peu de temps, pour un moment, et Je te rassemblerai avec d'immenses miséricordes. \EVERSE}
\newcommand{\isLIVvVIIIfr}{\VERSE  Dans un moment d'indignation J'ai détourné Mon visage de toi pour un instant, mais J'ai eu pitié de toi par une miséricorde éternelle, dit le Seigneur qui t'a rachetée. \EVERSE}
\newcommand{\isLIVvIXfr}{\VERSE  J'ai fait pour toi comme aux jours de Noé auquel j'avais juré de ne plus répandre sur la terre les eaux du déluge; J'ai juré de même de ne plus M'irriter contre toi, et de ne plus te faire de reproches. \EVERSE}
\newcommand{\isLIVvXfr}{\VERSE  Car les montagnes seront ébranlées, et les collines trembleront; mais Ma misicorde ne se retirera point de toi, et Mon alliance de paix ne sera pas ébranlée, dit le Seigneur, qui a compassion de toi. \EVERSE}
\newcommand{\isLIVvXIfr}{\VERSE  Pauvre petite, qui as été battue de la tempête sans aucune consolation, voici que Je placerai Moi-même tes pierres dans leur rang, et Je te donnerai des fondements de saphirs; \EVERSE}
\newcommand{\isLIVvXIIfr}{\VERSE  Je bâtirai tes remparts de jaspe, et tes portes de pierres sculptées, et toute ton enceinte sera de pierres choisies; \EVERSE}
\newcommand{\isLIVvXIIIfr}{\VERSE  tous tes enfants seront instruits par le Seigneur, et il y aura abondance de paix pour tes fils. \EVERSE}
\newcommand{\isLIVvXIVfr}{\VERSE  Tu seras fondée sur la justice; éloigne-toi de l'oppression, car tu n'auras plus peur, et de la frayeur, car elle ne s'approchera plus de toi. \EVERSE}
\newcommand{\isLIVvXVfr}{\VERSE  Il te viendra des habitants qui n'étaient point avec Moi, et celui qui autrefois t'était étranger se joindra à toi. \EVERSE}
\newcommand{\isLIVvXVIfr}{\VERSE  C'est Moi qui ai créé l'ouvrier qui souffle les charbons au fer et qui forme l'instrument pour son travail; c'est Moi aussi qui ai créé le meurtrier qui ne pense qu'à détruire. \EVERSE}
\newcommand{\isLIVvXVIIfr}{\VERSE  Toute arme préparée contre toi manquera le but; et toute langue qui te résistera devant le tribunal tu la jugeras. Tel est l'héritage des serviteurs du Seigneur, et leur justice est auprès de Moi, dit le Seigneur. \EVERSE}
\newcommand{\isLVvIfr}{\VERSE  Vous tous qui avez soif, venez aux eaux, et vous qui n'avez pas d'argent, hâtez-vous, achetez et mangez; venez, achetez sans argent et sans aucun échange le vin et le lait. \EVERSE}
\newcommand{\isLVvIIfr}{\VERSE  Pourquoi employez-vous votre argent à ce qui ne peut nourrir, et votre travail à ce qui ne peut rassasier? Ecoutez-Moi bien, et mangez ce qui est bon, et votre âme se délectera de mets savoureux. \EVERSE}
\newcommand{\isLVvIIIfr}{\VERSE  Prêtez l'oreille, et venez à Moi; écoutez-Moi, et votre âme vivra; et Je conclurai avec vous une alliance éternelle, pour rendre stable la miséricorde promise à David. \EVERSE}
\newcommand{\isLVvIVfr}{\VERSE  Voici que Je l'ai donné comme témoin aux peuples, comme maître et comme chef aux nations. \EVERSE}
\newcommand{\isLVvVfr}{\VERSE  Tu appelleras une nation que tu ne connaissais pas, et les peuples qui ne te connaissaient pas accourront à toi, à cause du Seigneur ton Dieu et du Saint d'Israël, qui t'a glorifié. \EVERSE}
\newcommand{\isLVvVIfr}{\VERSE  Cherchez le Seigneur pendant qu'on peut Le trouver; invoquez-Le pendant qu'Il est proche. \EVERSE}
\newcommand{\isLVvVIIfr}{\VERSE  Que l'impie abandonne sa voie et l'homme d'iniquité ses pensées, et qu'il revienne au Seigneur, car Il aura pitié de lui; et à notre Dieu, parce qu'Il est large pour pardonner. \EVERSE}
\newcommand{\isLVvVIIIfr}{\VERSE  Car Mes pensées ne sont pas vos pensées, et Mes voies ne sont pas vos voies, dit le Seigneur. \EVERSE}
\newcommand{\isLVvIXfr}{\VERSE  Mais autant les cieux sont élevés au-dessus de la terre, autant Mes voies sont élevées au-dessus de vos voies, et Mes pensées au-dessus de vos pensées. \EVERSE}
\newcommand{\isLVvXfr}{\VERSE  Et comme la pluie et la neige descendent du ciel et n'y retournent plus, mais qu'elles abreuvent la terre, la fécondent et la font germer, et qu'elle donne la semence au semeur, et le pain à celui qui mange; \EVERSE}
\newcommand{\isLVvXIfr}{\VERSE  ainsi Ma parole qui sort de Ma bouche ne retournera pas à Moi sans fruit; mais elle fera tout ce que Je veux, et elle produira les effets pour lesquels Je l'ai envoyée. \EVERSE}
\newcommand{\isLVvXIIfr}{\VERSE  Car vous sortirez avec joie, et vous serez conduits en paix; les montagnes et les collines chanteront devant vous des louanges, et tous les arbres du pays battront des mains. \EVERSE}
\newcommand{\isLVvXIIIfr}{\VERSE  Au lieu des broussailles le sapin s'élèvera, le myrte croîtra au lieu de l'ortie, et le Seigneur sera nommé comme un signe éternel qui ne sera pas enlevé. \EVERSE}
\newcommand{\isLVIvIfr}{\VERSE  Voici ce que dit le Seigneur : Gardez l'équité, et pratiquez la justice, car Mon salut ne tardera pas à venir, et Ma justice à être manifestée. \EVERSE}
\newcommand{\isLVIvIIfr}{\VERSE  Heureux l'homme qui fait cela, et le fils de l'homme qui s'y applique, qui observe le sabbat pour ne pas le violer, qui veille sur ses mains pour ne faire aucun mal. \EVERSE}
\newcommand{\isLVIvIIIfr}{\VERSE  Que le fils de l'étranger, qui s'est attaché au Seigneur, ne dise pas : Le Seigneur me divisera et me séparera de Son peuple; et que l'eunuque ne dise pas : Je suis un arbre desséché. \EVERSE}
\newcommand{\isLVIvIVfr}{\VERSE  Car voici ce que le Seigneur dit aux eunuques : A ceux qui garderont Mes sabbats, qui choisiront ce qui Me plaît, et qui persévéreront dans Mon alliance, \EVERSE}
\newcommand{\isLVIvVfr}{\VERSE  Je donnerai dans Ma maison et dans Mes murs une place, et un nom meilleur que des fils et des filles; Je leur donnerai un nom éternel, qui ne périra pas. \EVERSE}
\newcommand{\isLVIvVIfr}{\VERSE  Et les fils de l'étranger qui s'attachent au Seigneur pour Le servir, pour aimer Son nom, pour être Ses serviteurs, tous ceux qui observeront Mes sabbats pour ne pas les profaner et qui observeront Mon alliance, \EVERSE}
\newcommand{\isLVIvVIIfr}{\VERSE  Je les amènerai sur Ma montagne sainte, et Je les réjouirai dans Ma maison de prière; leurs holocaustes et leurs victimes Me seront agréables sur Mon autel, car Ma maison sera appelée une maison de prière pour tous les peuples. \EVERSE}
\newcommand{\isLVIvVIIIfr}{\VERSE  Voici ce que dit le Seigneur Dieu, qui rassemble les dispersés d'Israël : Je lui réunirai encore ceux qui se joindront à lui. \EVERSE}
\newcommand{\isLVIvIXfr}{\VERSE  Bêtes des champs, bêtes des forêts, venez toutes pour dévorer. \EVERSE}
\newcommand{\isLVIvXfr}{\VERSE  Ses sentinelles sont toutes aveugles, elles sont toutes dans l'ignorance; ce sont des chiens muets, qui ne peuvent aboyer, qui voient des choses vaines, qui dorment et aiment à rêver. \EVERSE}
\newcommand{\isLVIvXIfr}{\VERSE  Et ces chiens impudents ne peuvent se rassasier; les pasteurs eux-mêmes n'ont aucune intelligence; chacun se détourne pour suivre sa voie; chacun va à son avarice, depuis le plus grand jusqu'au plus petit. \EVERSE}
\newcommand{\isLVIvXIIfr}{\VERSE  Venez, prenons du vin, remplissons-nous-en jusqu'à l'ivresse; et ce sera demain comme aujourd'hui, et encore beaucoup plus. \EVERSE}
\newcommand{\isLVIIvIfr}{\VERSE  Le juste périt, et personne n'y fait réflexion dans son coeur; les hommes de miséricorde sont enlevés, parce qu'il n'y a personne qui comprenne; car c'est pour être délivré de la malice que le juste a été enlevé. \EVERSE}
\newcommand{\isLVIIvIIfr}{\VERSE  Que la paix vienne; que celui qui a marché dans la droiture se repose dans son lit. \EVERSE}
\newcommand{\isLVIIvIIIfr}{\VERSE  Mais vous, approchez ici, fils de sorcière, race d'un adultère et d'une prostituée. \EVERSE}
\newcommand{\isLVIIvIVfr}{\VERSE  De qui vous êtes-vous joués? contre qui avez-vous ouvert une large bouche, et tiré la langue? n'êtes-vous pas des scélérats, une race bâtarde, \EVERSE}
\newcommand{\isLVIIvVfr}{\VERSE  vous qui cherchez votre consolation dans vos dieux sous tout arbre touffu, qui sacrifiez vos petits enfants dans les torrents, sous les roches avancées? \EVERSE}
\newcommand{\isLVIIvVIfr}{\VERSE  C'est dans les pierres du torrent qu'est ton partage, voilà ton lot; tu leur as versé des libations, offert des sacrifices. Est-ce que Je ne M'indignerai pas de ces choses? \EVERSE}
\newcommand{\isLVIIvVIIfr}{\VERSE  Tu as mis ton lit sur une montagne haute et élevée, et tu y es montée pour immoler des victimes. \EVERSE}
\newcommand{\isLVIIvVIIIfr}{\VERSE  Tu as placé derrière la porte, derrière les poteaux, ton mémorial. Près de Moi, tu t'es découverte et tu as reçu un adultère, tu as élargi ton lit; tu as conclu une alliance avec eux, et tu as aimé ouvertement leur couche. \EVERSE}
\newcommand{\isLVIIvIXfr}{\VERSE  Tu t'es parfumée pour plaire au roi, et tu as multiplié tes aromates. Tu as envoyé tes ambassadeurs au loin, et tu t'es abaissée jusqu'au séjour des morts. \EVERSE}
\newcommand{\isLVIIvXfr}{\VERSE  Tu t'es fatiguée de la longueur de ta route, et tu n'as pas dit : Je me reposerai. Tu as trouvé de quoi vivre avec tes mains; c'est pourquoi tu n'as pas prié. \EVERSE}
\newcommand{\isLVIIvXIfr}{\VERSE  Qui as-tu redouté, qui as-tu craint pour Me mentir, pour M'effacer de ta mémoire, pour ne pas rentrer dans ton coeur? Parce que Je Me suis tu et que Je semblais ne pas voir, tu M'as oublié. \EVERSE}
\newcommand{\isLVIIvXIIfr}{\VERSE  Je vais proclamer ta justice, et tes oeuvres ne te serviront de rien. \EVERSE}
\newcommand{\isLVIIvXIIIfr}{\VERSE  Quand tu crieras, que tous ceux que tu as assemblés te délivrent! Le vent les emportera tous, un souffle les enlèvera. Mais celui qui a confiance en Moi aura la terre pour héritage et possédera Ma montagne sainte. \EVERSE}
\newcommand{\isLVIIvXIVfr}{\VERSE  Et Je dirai : Faites place, laissez le chemin libre, détournez-vous du sentier, ôtez les obstacles de la voie de Mon peuple. \EVERSE}
\newcommand{\isLVIIvXVfr}{\VERSE  Voici ce que dit le Très-Haut, le Dieu sublime, qui habite l'éternité, dont le nom est saint, qui réside dans le lieu saint et élevé, et avec l'esprit contrit et humble, pour ranimer l'esprit des humbles et pour ranimer les coeurs contrits. \EVERSE}
\newcommand{\isLVIIvXVIfr}{\VERSE  Car Je ne disputerai pas éternellement, et Ma colère ne durera pas toujours, parce que l'esprit est sorti de Moi, et que J'ai créé les âmes. \EVERSE}
\newcommand{\isLVIIvXVIIfr}{\VERSE  A cause de son avarice coupable Je Me suis irrité contre lui, et Je l'ai frappé. Je t'ai caché Ma face et Je Me suis indigné, et il s'en est allé vagabond sur le chemin de son coeur. \EVERSE}
\newcommand{\isLVIIvXVIIIfr}{\VERSE  J'ai vu ses voies, et Je l'ai guéri; Je l'ai ramené et Je l'ai consolé, lui et ceux qui pleuraient avec lui. \EVERSE}
\newcommand{\isLVIIvXIXfr}{\VERSE  J'ai créé la paix, qui est le fruit des lèvres; la paix pour celui qui est loin et pour celui qui est près, dit le Seigneur, et Je l'ai guéri. \EVERSE}
\newcommand{\isLVIIvXXfr}{\VERSE  Mais les impies sont comme une mer agitée qui ne peut se calmer, et dont les flots se soulèvent pour produire la vase et le limon. \EVERSE}
\newcommand{\isLVIIvXXIfr}{\VERSE  Il n'y a point de paix pour les impies, dit le Seigneur Dieu. \EVERSE}
\newcommand{\isLVIIIvIfr}{\VERSE  Crie, ne t'arrête pas, fais retentir ta voix comme une trompette, et annonce à Mon peuple ses crimes, et à la maison de Jacob ses péchés. \EVERSE}
\newcommand{\isLVIIIvIIfr}{\VERSE  Car ils Me cherchent chaque jour, et ils veulent connaître Mes voies, comme un peuple qui aurait pratiqué la justice, et qui n'aurait pas abandonné la loi de son Dieu. Ils Me demandent des arrêts de justice, ils veulent s'approcher de Dieu. \EVERSE}
\newcommand{\isLVIIIvIIIfr}{\VERSE  Pourquoi avons-nous jeûné, et ne l'avez-Vous pas regardé? pourquoi avons-nous humilié nos âmes et ne l'avez-Vous pas su? C'est qu'au jour de votre jeûne on trouve votre volonté propre, et que vous pressez tous vos débiteurs. \EVERSE}
\newcommand{\isLVIIIvIVfr}{\VERSE  Vous jeûnez pour faire des procès et des querelles, et vous frappez du poing sans pitié. Ne jeûnez plus comme vous l'avez fait jusqu'à ce jour, pour faire entendre en haut vos cris. \EVERSE}
\newcommand{\isLVIIIvVfr}{\VERSE  Est-ce là le jeûne que Je demande, qui fait qu'un homme afflige son âme pendant un jour, lui fait tourner la tête comme un cercle, et se coucher sur le sac et la cendre? Est-ce là ce que tu appelles un jeûne, et un jour agréable au Seigneur? \EVERSE}
\newcommand{\isLVIIIvVIfr}{\VERSE  Le jeûne que J'approuve n'est-il pas plutôt celui-ci? Détache les chaînes de l'impiété, décharge les fardeaux accablants, renvoie libres ceux qui sont opprimés, et brise tout fardeau; \EVERSE}
\newcommand{\isLVIIIvVIIfr}{\VERSE  partage ton pain avec celui qui a faim, et fais entrer dans ta maison les pauvres et ceux qui n'ont pas d'asile; lorsque tu verras un homme nu, couvre-le, et ne méprise pas ta propre chair. \EVERSE}
\newcommand{\isLVIIIvVIIIfr}{\VERSE  Alors ta lumière éclatera comme l'aurore, et ta santé reviendra bientôt; ta justice marchera devant toi, et la gloire du Seigneur te protégera. \EVERSE}
\newcommand{\isLVIIIvIXfr}{\VERSE  Alors tu invoqueras, et le Seigneur t'exaucera; tu crieras, et Il dira : Me voici. Si tu éloignes la chaîne du milieu de toi, si tu cesses d'étendre le doigt et de dire ce qui n'est pas utile; \EVERSE}
\newcommand{\isLVIIIvXfr}{\VERSE  si tu répands ton âme sur l'affamé, et si tu rassasies l'âme affligée, ta lumière se lèvera dans les ténèbres, et tes ténèbres seront comme le midi. \EVERSE}
\newcommand{\isLVIIIvXIfr}{\VERSE  Le Seigneur te donnera toujours du repos; Il remplira ton âme de splendeurs, et Il délivrera tes os; et tu deviendras comme un jardin arrosé, et comme une fontaine dont les eaux ne tarissent pas. \EVERSE}
\newcommand{\isLVIIIvXIIfr}{\VERSE  Les déserts séculaires seront rebâtis par toi, tu relèveras les fondements des générations anciennes, et tu seras appelé le réparateur des haies, et celui qui rétablit les chemins et les rend sûrs. \EVERSE}
\newcommand{\isLVIIIvXIIIfr}{\VERSE  Si tu éloignes ton pied du sabbat, pour ne pas faire ta volonté en Mon saint jour; si tu appelles le sabbat tes délices, et le jour saint et glorieux du Seigneur; si tu l'honores en ne suivant pas tes voies, en ne faisant pas ta volonté, et en ne disant pas des paroles vaines: \EVERSE}
\newcommand{\isLVIIIvXIVfr}{\VERSE  alors tu te réjouiras dans le Seigneur, Je t'élèverai au-dessus des hauteurs de la terre, et Je te donnerai pour nourriture l'héritage de Jacob ton père; car la bouche du Seigneur a parlé. \EVERSE}
\newcommand{\isLIXvIfr}{\VERSE  La main du Seigneur n'est pas raccourcie de manière à ne pouvoir plus sauver, et Son oreille n'est pas devenue dure de manière à ne pouvoir plus entendre. \EVERSE}
\newcommand{\isLIXvIIfr}{\VERSE  Mais ce sont vos iniquités qui ont mis une séparation entre vous et votre Dieu, et ce sont vos péchés qui Lui ont fait cacher Sa face pour ne plus vous exaucer. \EVERSE}
\newcommand{\isLIXvIIIfr}{\VERSE  Car vos mains sont souillées de sang, et vos doigts d'iniquité; vos lèvres ont proféré le mensonge, et votre langue dit l'iniquité. \EVERSE}
\newcommand{\isLIXvIVfr}{\VERSE  Personne n'invoque la justice, et personne ne juge selon la vérité; mais ils se confient dans le néant et disent des vanités; ils conçoivent l'affliction, et ils enfantent l'iniquité. \EVERSE}
\newcommand{\isLIXvVfr}{\VERSE  Ils ont fait éclore des oeufs d'aspics, et ils ont tissé des toiles d'araignées. Celui qui mangera de ces oeufs en mourra, et de ceux qu'on fait couver il sortira un basilic. \EVERSE}
\newcommand{\isLIXvVIfr}{\VERSE  Leurs toiles ne serviront pas de vêtement, et ils ne se couvriront pas de leur ouvrage; car leurs oeuvres sont des oeuvres inutiles, et une oeuvre d'iniquité est dans leurs mains. \EVERSE}
\newcommand{\isLIXvVIIfr}{\VERSE  Leurs pieds courent au mal, et ils se hâtent pour répandre le sang innocent; leurs pensées sont des pensées inutiles; le ravage et la ruine sont sur leurs voies. \EVERSE}
\newcommand{\isLIXvVIIIfr}{\VERSE  Ils ne connaissent pas le chemin de la paix, et il n'y a point de justice sur leurs pas; leurs sentiers sont tortueux; quiconque y marche ne connaît point la paix. \EVERSE}
\newcommand{\isLIXvIXfr}{\VERSE  C'est pour cela que l'équité s'est éloignée de nous, et que la justice ne nous atteint pas. Nous attendions la lumière, et voici les ténèbres; la clarté, et nous marchons dans l'obscurité. \EVERSE}
\newcommand{\isLIXvXfr}{\VERSE  Nous tâtonnons comme des aveugles le long des murs, nous marchons à tâtons comme ceux qui n'ont pas d'yeux; nous nous heurtons en plein midi comme dans les ténèbres, nous sommes dans l'obscurité comme les morts. \EVERSE}
\newcommand{\isLIXvXIfr}{\VERSE  Nous rugissons tous comme des ours, nous soupirons et nous gémissons comme des colombes; nous attendions le jugement, et il n'est pas venu; le salut et il est loin de nous. \EVERSE}
\newcommand{\isLIXvXIIfr}{\VERSE  Car nos iniquités se sont multipliées devant Vous, et nos péchés témoignent contre nous, parce que nos crimes nous sont présents, et nous connaissons nos iniquités: \EVERSE}
\newcommand{\isLIXvXIIIfr}{\VERSE  nous avons péché et nous avons menti contre le Seigneur; nous nous sommes détournés pour ne pas marcher à la suite de notre Dieu, pour proférer la calomnie et la violence; nous avons conçu et fait sortir de notre coeur des paroles de mensonge. \EVERSE}
\newcommand{\isLIXvXIVfr}{\VERSE  Et la justice s'est retournée en arrière, et la justice se tient éloignée, parce que la vérité a été renversée sur la place publique, et que l'équité n'y a pu entrer. \EVERSE}
\newcommand{\isLIXvXVfr}{\VERSE  La vérité a été en oubli, et celui qui s'est retiré du mal a été dépouillé. Le Seigneur l'a vu, et Ses yeux ont été blessés de ce qu'il n'y avait plus de justice. \EVERSE}
\newcommand{\isLIXvXVIfr}{\VERSE  Il a vu qu'il n'y a pas d'homme, et Il a été étonné que personne n'intervînt; alors Son bras L'a sauvé, et Sa propre justice L'a soutenu. \EVERSE}
\newcommand{\isLIXvXVIIfr}{\VERSE  Il S'est revêtu de la justice comme d'une cuirasse, et Il a mis sur Sa tête le casque du salut; Il S'est revêtu de la vengeance comme d'un vêtement, et Il S'est couvert de Sa colère comme d'un manteau. \EVERSE}
\newcommand{\isLIXvXVIIIfr}{\VERSE  Il Se vengera, Il punira dans Sa colère Ses ennemis, Il rendra à Ses adversaires ce qu'ils méritent; Il rendra la pareille aux îles. \EVERSE}
\newcommand{\isLIXvXIXfr}{\VERSE  Ceux de l'occident craindront le nom du Seigneur et ceux de l'orient révéreront Sa gloire, lorsqu'Il viendra comme un fleuve impétueux qu'agite le souffle de Dieu; \EVERSE}
\newcommand{\isLIXvXXfr}{\VERSE  lorsqu'un rédempteur sera venu à Sion, et à ceux de Jacob qui abandonneront l'iniquité, dit le Seigneur. \EVERSE}
\newcommand{\isLIXvXXIfr}{\VERSE  Voici l'alliance que Je ferai avec eux, dit le Seigneur : Mon esprit qui est en toi, et Mes paroles que J'ai mises dans ta bouche ne sortiront pas de ta bouche, ni de la bouche de tes enfants, ni de la bouche des enfants de tes enfants, dit le Seigneur, dès maintenant jusque dans l'éternité. \EVERSE}
\newcommand{\isLXvIfr}{\VERSE  Lève-toi, sois éclairée, Jérusalem, car ta lumière est venue, et la gloire du Seigneur s'est levée sur toi. \EVERSE}
\newcommand{\isLXvIIfr}{\VERSE Car les ténèbres couvriront la terre, et l'obscurité les peuples; mais sur toi Se lèvera le Seigneur, et l'on verra Sa gloire en toi. \EVERSE}
\newcommand{\isLXvIIIfr}{\VERSE  Les nations marcheront à ta lumière, et les rois à la splendeur de ton aurore. \EVERSE}
\newcommand{\isLXvIVfr}{\VERSE  Lève les yeux et regarde autour de toi : tous ceux-ci sont assemblés, ils viennent à toi; tes fils viendront de loin, et tes filles surgiront de tous côtés. \EVERSE}
\newcommand{\isLXvVfr}{\VERSE  Alors tu verras et tu seras dans l'abondance, ton coeur s'étonnera et se dilatera, lorsque les richesses de la mer se tourneront vers toi, et que la force des nations viendra à toi. \EVERSE}
\newcommand{\isLXvVIfr}{\VERSE  Tu seras couverte d'une foule de chameaux, des dromadaires de Madian et d'Epha; tous viendront de Saba, apportant de l'or et de l'encens, et publiant les louanges du Seigneur. \EVERSE}
\newcommand{\isLXvVIIfr}{\VERSE  Tous les troupeaux de Cédar se rassembleront pour toi; les béliers de Nabajoth seront à ton service : on les offrira sur Mon autel qui Me sera agréable, et Je remplirai de gloire la maison de Ma majesté. \EVERSE}
\newcommand{\isLXvVIIIfr}{\VERSE  Quels sont ceux-ci qui volent comme des nuées, et comme des colombes vers leurs colombiers? \EVERSE}
\newcommand{\isLXvIXfr}{\VERSE  Car les îles M'attendent, et les vaisseaux de la mer sont prêts depuis longtemps pour ramener tes enfants de loin, avec leur argent et leur or, pour le consacrer au nom du Seigneur ton Dieu, et du Saint d'Israël qui t'a glorifiée. \EVERSE}
\newcommand{\isLXvXfr}{\VERSE  Les fils des étrangers bâtiront tes murs, et leurs rois seront tes serviteurs; car Je t'ai frappée dans Mon indignation, et dans Ma miséricorde Je Me suis réconcilié avec toi. \EVERSE}
\newcommand{\isLXvXIfr}{\VERSE  Tes portes seront toujours ouvertes; elles ne seront fermées ni jour ni nuit, afin qu'on t'apporte la richesse des nations, et qu'on t'amène leurs rois. \EVERSE}
\newcommand{\isLXvXIIfr}{\VERSE  Car le peuple et le royaume qui ne te serviront pas périront, et ses nations seront transformées en désert. \EVERSE}
\newcommand{\isLXvXIIIfr}{\VERSE  La gloire du Liban viendra à toi, le sapin, le buis et le pin tous ensemble, pour orner le lieu de Mon sanctuaire, et Je glorifierai l'endroit où reposent Mes pieds. \EVERSE}
\newcommand{\isLXvXIVfr}{\VERSE  Les fils de ceux qui t'ont humiliée viendront à toi en s'inclinant, et tous ceux qui te décriaient adoreront les traces de tes pas, et ils t'appelleront la cité du Seigneur, la Sion du Saint d'Israël. \EVERSE}
\newcommand{\isLXvXVfr}{\VERSE  Parce que tu as été abandonnée et en butte à la haine, et qu'il n'y avait personne qui passât par toi, Je ferai de toi l'orgueil des siècles, et un sujet de joie de génération en génération; \EVERSE}
\newcommand{\isLXvXVIfr}{\VERSE  et tu suceras le lait des nations, tu seras allaitée à la mamelle des rois: et tu sauras que Je suis le Seigneur qui te sauve, et le Fort de Jacob qui te rachète. \EVERSE}
\newcommand{\isLXvXVIIfr}{\VERSE  Au lieu d'airain Je ferai venir de l'or, et de l'argent au lieu de fer, et de l'airain au lieu de bois, et du fer au lieu de pierres; et Je ferai régner sur toi la paix, et la justice te gouvernera. \EVERSE}
\newcommand{\isLXvXVIIIfr}{\VERSE  On n'entendra plus parler de violence sur ton terrritoire, ni de destruction et de ruine dans tes frontières; le salut environnera tes murailles, et la louange tes portes. \EVERSE}
\newcommand{\isLXvXIXfr}{\VERSE  Tu n'auras plus le soleil pour t'éclairer pendant le jour, et la clarté de la lune ne luira plus sur toi; mais le Seigneur sera pour toi une lumière éternelle, et ton Dieu sera ta gloire. \EVERSE}
\newcommand{\isLXvXXfr}{\VERSE  Ton soleil ne se couchera plus, et ta lune ne sera plus diminuée, car le Seigneur sera ta lumière éternelle, et les jours de ton deuil seront finis. \EVERSE}
\newcommand{\isLXvXXIfr}{\VERSE  Tout ton peuple sera un peuple de justes; ils posséderont le pays pour toujours; c'est le rejeton que J'ai planté, l'oeuvre de Ma main pour Me glorifier. \EVERSE}
\newcommand{\isLXvXXIIfr}{\VERSE  Mille sortiront du moindre d'entre eux, et du plus petit une nation puissante. Moi, le Seigneur, Je ferai tout à coup ces choses en leur temps. \EVERSE}
\newcommand{\isLXIvIfr}{\VERSE  L'esprit du Seigneur est sur Moi, parce que le Seigneur M'a donné Son onction; il M'a envoyé pour annoncer Sa parole aux doux, pour guérir ceux qui ont le coeur brisé, pour prêcher la grâce aux captifs, et la liberté aux prisonniers; \EVERSE}
\newcommand{\isLXIvIIfr}{\VERSE  pour publier l'année de la réconciliation du Seigneur, et le jour de la vengeance de notre Dieu, pour consoler tous ceux qui pleurent, \EVERSE}
\newcommand{\isLXIvIIIfr}{\VERSE  pour accorder et pour donner à ceux de Sion qui pleurent une couronne au lieu de la cendre, l'huile de joie au lieu du deuil, un vêtement de louange au lieu d'un esprit affligé; et il y aura en elle des hommes puissants en justice, une plantation du Seigneur pour Le glorifier. \EVERSE}
\newcommand{\isLXIvIVfr}{\VERSE  Ils rempliront d'édifices les déserts séculaires, ils relèveront les anciennes ruines, et ils rétabliront les villes abandonnées, dévastées pendant plusieurs générations. \EVERSE}
\newcommand{\isLXIvVfr}{\VERSE  Des étrangers seront là et feront paître vos troupeaux, et les fils des étrangers seront vos laboureurs et vos vignerons. \EVERSE}
\newcommand{\isLXIvVIfr}{\VERSE  Mais vous, vous serez appelés les prêtres du Seigneur; on vous nommera les ministres de notre Dieu; vous mangerez la richesse des naations, et vous vous glorifierez de leur gloire. \EVERSE}
\newcommand{\isLXIvVIIfr}{\VERSE  Au lieu de la double confusion dont vous rougissiez, ils loueront leur partage, et ils posséderont ainsi le double dans leur pays, et leur joie sera éternelle. \EVERSE}
\newcommand{\isLXIvVIIIfr}{\VERSE  Car Je suis le Seigneur qui aime la justice, et qui hais la rapine dans l'holocauste; J'établirai leur oeuvre dans la vérité, et Je contracterai avec eux une alliance éternelle. \EVERSE}
\newcommand{\isLXIvIXfr}{\VERSE  Leur postérité sera connue parmi les nations, et leur race au milieu des peuples; tous ceux qui les verront connaîtront qu'ils sont la race que le Seigneur a bénie. \EVERSE}
\newcommand{\isLXIvXfr}{\VERSE  Je me réjouirai avec effusion dans le Seigneur, et mon âme sera ravie d'allégresse en mon Dieu; car il m'a revêtu des vêtements du salut, et il m'a entouré des ornements de la justice, comme un époux orné d'une couronne, et comme une épouse parée de ses bijoux. \EVERSE}
\newcommand{\isLXIvXIfr}{\VERSE  Car comme la terre fait éclore son germe, et comme un jardin fait pousser sa semence, ainsi le Seigneur Dieu fera germer la justice et la louange en présence de toutes les nations. \EVERSE}
\newcommand{\isLXIIvIfr}{\VERSE  A cause de Sion Je ne Me tairai point, et à cause de Jérusalem Je ne prendrai pas de repos, jusqu'à ce que son Juste paraisse comme une vive lumière, et son Sauveur comme une lampe allumée. \EVERSE}
\newcommand{\isLXIIvIIfr}{\VERSE  Les nations verront ton Juste, et tous les rois ton prince illustre, et on t'appellera d'un nom nouveau, que la bouche du Seigneur désignera. \EVERSE}
\newcommand{\isLXIIvIIIfr}{\VERSE  Tu seras une couronne de gloire dans la main du Seigneur, et un diadème royal dans la main de ton Dieu. \EVERSE}
\newcommand{\isLXIIvIVfr}{\VERSE  On ne t'appellera plus Délaissée, et ta terre ne sera plus appelée Désolée; mais tu seras appelée : Ma volonté est en elle, et ta terre : Habitée, car le Seigneur a mis Son plaisir en toi, et ta terre sera habitée. \EVERSE}
\newcommand{\isLXIIvVfr}{\VERSE  Car le jeune homme habitera avec la vierge, et tes enfants habiteront en toi; l'époux trouvera sa joie dans son épouse, et ton Dieu se réjouira en toi. \EVERSE}
\newcommand{\isLXIIvVIfr}{\VERSE  Sur tes murs, Jérusalem, J'ai placé des gardes; ils ne se tairont jamais, ni le jour ni la nuit. Vous qui vous souvenez du Seigneur, ne vous taisez pas, \EVERSE}
\newcommand{\isLXIIvVIIfr}{\VERSE  et ne Lui donnez pas de repos, jusqu'à ce qu'Il affermisse Jérusalem, et qu'Il la rende glorieuse sur la terre. \EVERSE}
\newcommand{\isLXIIvVIIIfr}{\VERSE  Le Seigneur a juré par Sa droite, et par Son bras puissant : Je ne donnerai plus ton blé pour nourriture à tes ennemis, et les fils des étrangers ne boiront plus ton vin, produit de ton travail. \EVERSE}
\newcommand{\isLXIIvIXfr}{\VERSE  Mais ceux qui auront recueilli le blé le mangeront, et loueront le Seigneur, et ceux qui auront récolté le vin le boiront dans Mes saints parvis. \EVERSE}
\newcommand{\isLXIIvXfr}{\VERSE  Franchissez, franchissez les portes, préparez la voie au peuple, aplanissez le chemin, ôtez les pierres, élevez l'étendard pour les peuples. \EVERSE}
\newcommand{\isLXIIvXIfr}{\VERSE  Voici ce que le Seigneur a fait entendre aux extrémités de la terre : Dites à la fille de Sion : Ton Sauveur vient, Il porte avec Lui Sa récompense et Son salaire Le précède. \EVERSE}
\newcommand{\isLXIIvXIIfr}{\VERSE  Et on les appellera peuple saint, rachetés du Seigneur; et toi, on t'appellera Ville recherchée, non délaissée. \EVERSE}
\newcommand{\isLXIIIvIfr}{\VERSE  Quel est celui qui vient d'Edom, de Bosra, avec Ses vêtements teints? Il est beau dans Sa robe, et Il S'avance avec une force toute-puissante. Je suis Celui qui parle la justice, et Je viens pour défendre et pour sauver. \EVERSE}
\newcommand{\isLXIIIvIIfr}{\VERSE  Pourquoi donc Votre robe est-elle rouge, et pourquoi Vos vêtements sont-ils comme les habits de ceux qui foulent dans la cuve? \EVERSE}
\newcommand{\isLXIIIvIIIfr}{\VERSE  J'ai été seul à fouler au pressoir, et nul homme d'entre les peuples n'était avec Moi; Je les ai foulés dans Ma fureur, et Je les ai écrasés dans Ma colère, et leur sang a rejailli sur Ma robe, et J'ai taché tous Mes vêtements. \EVERSE}
\newcommand{\isLXIIIvIVfr}{\VERSE  Car le jour de la vengeance était dans Mon coeur, l'année de Ma rédemption est venue. \EVERSE}
\newcommand{\isLXIIIvVfr}{\VERSE  J'ai regardé autour de Moi, et il n'y avait personne pour M'aider; J'ai cherché, et Je n'ai pas trouvé de secours; alors Mon bras M'a sauvé, et Ma colère même M'est venue en aide. \EVERSE}
\newcommand{\isLXIIIvVIfr}{\VERSE  J'ai foulé les peuples dans Ma fureur; Je les ai enivrés dans Mon indignation, et J'ai renversé leur force à terre. \EVERSE}
\newcommand{\isLXIIIvVIIfr}{\VERSE  Je me souviendrai des miséricordes du Seigneur; Je louerai le Seigneur pour tout ce qu'Il nous a fait, pour tous Ses bienfaits envers la maison d'Israël, bienfaits qu'Il a répandus sur elle selon Sa bonté et selon la multitude de Ses miséricordes. \EVERSE}
\newcommand{\isLXIIIvVIIIfr}{\VERSE  Il avait dit: Ils sont vraiment Mon peuple, des fils qui ne renient point leur père, et Il est devenu leur sauveur. \EVERSE}
\newcommand{\isLXIIIvIXfr}{\VERSE  Dans toutes leurs afflictions Il ne S'est point lassé, et l'Ange de Sa face les a sauvés. Dans Son amour et dans Sa miséricorde, Il les a rachetés Lui-même, Il les a portés et Il les a soutenus tous les jours du temps passé. \EVERSE}
\newcommand{\isLXIIIvXfr}{\VERSE  Mais ils ont provoqué Sa colère, ils ont affligé l'esprit de Son Saint; et Il est devenu leur ennemi, et Il les a Lui-même combattus. \EVERSE}
\newcommand{\isLXIIIvXIfr}{\VERSE  Puis Il S'est souvenu des anciens jours de Moïse et de Son peuple. Où est Celui qui les a tirés de la mer avec les pasteurs de Son troupeau? où est Celui qui a mis au milieu d'eux l'esprit de Son Saint; \EVERSE}
\newcommand{\isLXIIIvXIIfr}{\VERSE  qui a pris Moïse par la droite, et l'a soutenu par le bras de Sa majesté; qui a fendu les eaux devant eux pour S'acquérir un nom éternel; \EVERSE}
\newcommand{\isLXIIIvXIIIfr}{\VERSE  qui les a conduits à travers les abîmes, comme un cheval qu'on mène au désert sans qu'il bronche? \EVERSE}
\newcommand{\isLXIIIvXIVfr}{\VERSE  Comme un animal qui descend dans la vallée, l'esprit du Seigneur les a conduits. C'est ainsi que Vous avez conduit Votre peuple, pour Vous faire un nom glorieux. \EVERSE}
\newcommand{\isLXIIIvXVfr}{\VERSE  Regardez du ciel, et voyez de Votre demeure sainte et du trône de Votre gloire. Où sont maintenant Votre zèle et Votre force? où est la tendresse de Vos entrailles et de Vos miséricordes? Elles se contiennent envers moi. \EVERSE}
\newcommand{\isLXIIIvXVIfr}{\VERSE  Car c'est Vous qui êtes notre père; Abraham ne nous connaît point, et Israël ignore qui nous sommes; mais Vous, Seigneur, Vous êtes notre père, notre libérateur, Vous dont le nom est éternel. \EVERSE}
\newcommand{\isLXIIIvXVIIfr}{\VERSE  Pourquoi, Seigneur, nous avez-Vous fait errer loin de Vos voies? pourquoi avez-Vous endurci notre coeur de sorte qu'il cessât de Vous craindre? Revenez à cause de Vos serviteurs, à cause des tribus de Votre héritage. \EVERSE}
\newcommand{\isLXIIIvXVIIIfr}{\VERSE  Ils se sont rendus maîtres de Votre peuple saint, comme s'il n'était rien; nos ennemis ont foulé aux pieds Votre sanctuaire. \EVERSE}
\newcommand{\isLXIIIvXIXfr}{\VERSE  Nous sommes devenus comme au commencement, lorsque Vous n'étiez pas notre Roi, et que Votre nom n'était pas invoqué sur nous. \EVERSE}
\newcommand{\isLXIVvIfr}{\VERSE  Oh! si Vous déchiriez les cieux, et si Vous descendiez, devant Vous les montagnes s'écouleraient. \EVERSE}
\newcommand{\isLXIVvIIfr}{\VERSE  Elles fondraient comme brûlées par le feu, les eaux deviendraient embrasées, afin que Votre nom fût connu à Vos ennemis, et que les nations tremblassent devant Votre face. \EVERSE}
\newcommand{\isLXIVvIIIfr}{\VERSE  Lorsque Vous ferez éclater Vos merveilles, nous ne pourrons les supporter. Vous êtes descendu, et les montagnes se sont écoulées devant Vous. \EVERSE}
\newcommand{\isLXIVvIVfr}{\VERSE  Jamais on n'a entendu, l'oreille n'a pas ouï, et l'oeil n'a pas vu, hors Vous seul, ô Dieu, ce que Vous avez préparé pour ceux qui Vous attendent. \EVERSE}
\newcommand{\isLXIVvVfr}{\VERSE  Vous êtes allé au-devant de celui qui se réjouit et qui pratique la justice; ils se souviendront de Vous dans Vos voies. Vous Vous êtes irrité, parce que nous avons péché. Nous avons toujours été dans le péché, mais nous serons sauvés. \EVERSE}
\newcommand{\isLXIVvVIfr}{\VERSE  Nous sommes tous devenus comme un homme impur, et toutes les oeuvres de notre justice sont comme un linge souillé; nous sommes tous tombés comme une feuille, et nos iniquités nous ont emportés comme le vent. \EVERSE}
\newcommand{\isLXIVvVIIfr}{\VERSE  Il n'y a personne qui invoque Votre nom, qui se lève et qui s'attache à Vous. Vous avez détourné de nous Votre visage, et Vous nous avez brisés sous le poids de notre iniquité. \EVERSE}
\newcommand{\isLXIVvVIIIfr}{\VERSE  Cependant, Seigneur, Vous êtes notre père, et nous sommes de l'argile; c'est Vous qui nous avez formés, et nous sommes tous l'oeuvre de Vos mains. \EVERSE}
\newcommand{\isLXIVvIXfr}{\VERSE  Ne Vous irritez pas sans mesure, Seigneur, et ne Vous souvenez plus de notre iniquité; regardez, nous sommes tous Votre peuple. \EVERSE}
\newcommand{\isLXIVvXfr}{\VERSE  La ville de Votre Saint a été changée en désert, Sion est devenue déserte, Jérusalem est désolée. \EVERSE}
\newcommand{\isLXIVvXIfr}{\VERSE  Le temple de notre sanctification et de notre gloire, où nos pères Vous ont loué, a été entièrement consumé, et toutes nos splendeurs ne sont plus que des ruines. \EVERSE}
\newcommand{\isLXIVvXIIfr}{\VERSE  Après cela, Seigneur, Vous contiendrez-Vous encore? Vous tairez-Vous, et nous affligerez-Vous à l'excès? \EVERSE}
\newcommand{\isLXVvIfr}{\VERSE  Ceux qui auparavant ne M'interrogeaient pas M'ont recherché, ceux qui ne Me cherchaient pas M'ont trouvé. J'ai dit à une nation qui n'invoquait pas Mon nom : Me voici, Me voici! \EVERSE}
\newcommand{\isLXVvIIfr}{\VERSE  J'ai étendu Mes mains tout le jour vers un peuple incrédule, qui marche dans une voie qui n'est pas bonne, en suivant ses pensées. \EVERSE}
\newcommand{\isLXVvIIIfr}{\VERSE  C'est un peuple qui, en face, provoque constamment Ma colère, qui immole des victimes dans les jardins, et qui sacrifie sur des briques; \EVERSE}
\newcommand{\isLXVvIVfr}{\VERSE  qui habite dans les sépulcres, et qui dort dans les temples des idoles, qui mange de la chair de pourceau, et qui met dans ses vases une liqueur profane; \EVERSE}
\newcommand{\isLXVvVfr}{\VERSE  qui dit : Retire-toi de moi, ne t'approche pas de moi, car tu n'es pas pur. Ils deviendront une fumée dans Ma fureur, un feu qui brûlera toujours. \EVERSE}
\newcommand{\isLXVvVIfr}{\VERSE  Cela est écrit devant Moi; Je ne Me tairai plus, mais Je le leur rendrai, et Je le verserai dans leur sein. \EVERSE}
\newcommand{\isLXVvVIIfr}{\VERSE  Je punirai vos iniquités, dit le Seigneur, et en même temps les iniquités de vos pères, qui ont sacrifié sur les montagnes et qui M'ont outragé sur les collines; Je verserai dans leur sein une peine proportionnée à leurs oeuvres. \EVERSE}
\newcommand{\isLXVvVIIIfr}{\VERSE  Voici ce que dit le Seigneur: Comme lorsqu'on trouve un grain dans une grappe, et que l'on dit : Ne le détruis pas, car c'est une bénédiction, ainsi agirai-Je en faveur de Mes serviteurs, et Je n'exterminerai pas tout. \EVERSE}
\newcommand{\isLXVvIXfr}{\VERSE  Je ferai sortir de Jacob une postérité, et de Juda le possesseur de Mes montagnes; et Mes élus en hériteront, et Mes serviteurs y habiteront. \EVERSE}
\newcommand{\isLXVvXfr}{\VERSE  Les campagnes serviront de parcs aux troupeaux, et la vallée d'Achor servira de gîte aux boeufs, pour ceux de Mon peuple qui M'auront recherché. \EVERSE}
\newcommand{\isLXVvXIIfr}{\VERSE  Je vous compterai avec le glaive, et vous périrez tous dans la carnage; car J'ai appelé, et vous n'avez pas répondu; J'ai parlé, et vous n'avez pas entendu; vous avez fait le mal devant Mes yeux, et vous avez choisi ce que Je ne voulais pas. \EVERSE}
\newcommand{\isLXVvXIIIfr}{\VERSE  C'est pourquoi voici ce que dit le Seigneur Dieu : Mes serviteurs mangeront, et vous aurez faim; Mes serviteurs boiront, et vous aurez soif; \EVERSE}
\newcommand{\isLXVvXIVfr}{\VERSE  Mes serviteurs se réjouiront, et vous serez confondus; Mes serviteurs Me loueront dans l'allégresse de leur coeur, et vous crierez dans la douleur de votre coeur, et vous hurlerez dans le déchirement de votre esprit; \EVERSE}
\newcommand{\isLXVvXVfr}{\VERSE  et vous laisserez votre nom à Mes élus en imprécation; le Seigneur Dieu vous fera périr, et Il donnera à Ses serviteurs un autre nom. \EVERSE}
\newcommand{\isLXVvXVIfr}{\VERSE  Celui qui sera béni en ce nom sur la terre sera béni du Dieu de vérité, et celui qui jurera sur la terre jurera au nom du Dieu de vérité; car les anciennes angoisses seront mises en oubli, et elles seront cachées à Mes yeux. \EVERSE}
\newcommand{\isLXVvXVIIfr}{\VERSE  Car Je vais créer de nouveaux cieux et une terre nouvelle, et les choses anciennes s'effaceront de la mémoire, et elles ne reviendront plus à l'esprit. \EVERSE}
\newcommand{\isLXVvXVIIIfr}{\VERSE  Mais vous vous réjouirez et vous serez éternellement dans l'allégresse à cause des choses que Je vais créer, car Je vais faire de Jérusalem une ville d'allégresse, et de son peuple un peuple de joie. \EVERSE}
\newcommand{\isLXVvXIXfr}{\VERSE  Je mettrai Mon allégresse dans Jérusalem, et Ma joie dans Mon peuple, et on n'y entendra plus le bruit des pleurs ni le bruit des cris. \EVERSE}
\newcommand{\isLXVvXXfr}{\VERSE  On n'y verra plus d'enfant qui ne vive que peu de jours, ni de vieillard qui n'accomplisse le temps de sa vie; car l'enfant mourra à cent ans, et le pécheur de cent ans sera maudit. \EVERSE}
\newcommand{\isLXVvXXIfr}{\VERSE  Ils bâtiront des maisons, et ils les habiteront; ils planteront des vignes, et ils en mangeront le fruit. \EVERSE}
\newcommand{\isLXVvXXIIfr}{\VERSE  Ils ne bâtiront pas des maisons qu'un autre habitera, ils ne planteront pas pour qu'un autre mange le fruit, car les jours de Mon peuple seront comme les jours des arbres, et les oeuvres de leurs mains seront de longue durée. \EVERSE}
\newcommand{\isLXVvXXIIIfr}{\VERSE  Mes élus ne travailleront point en vain, et ils n'engendreront pas pour le trouble; car ils seront une race bénie du Seigneur, et leurs petits enfants seront avec eux. \EVERSE}
\newcommand{\isLXVvXXIVfr}{\VERSE  Avant qu'ils crient, Je les exaucerai, et lorsqu'ils parleront encore, Je les aurai écoutés. \EVERSE}
\newcommand{\isLXVvXXVfr}{\VERSE  Le loup et l'agneau paîtront ensemble, le lion et le boeuf mangeront de la paille, et la poussière sera la nourriture du serpent. Ils ne nuiront pas et ne tueront pas sur toute Ma montagne sainte, dit le Seigneur. \EVERSE}
\newcommand{\isLXVIvIfr}{\VERSE  Voici ce que dit le Seigneur : Le ciel est Mon trône, et la terre l'escabeau de Mes pieds. Quelle est cette maison que vous Me bâtirez, et quel est ce lieu de Mon repos? \EVERSE}
\newcommand{\isLXVIvIIfr}{\VERSE  C'est Ma main qui a fait tout cela, et toutes ces choses ont été créées, dit le Seigneur; mais qui regarderai-Je, sinon le pauvre, et celui qui a le coeur brisé, et qui craint Mes paroles? \EVERSE}
\newcommand{\isLXVIvIIIfr}{\VERSE  Celui qui immole un boeuf est comme celui qui tuerait un homme; celui qui sacrifie un agneau est comme celui qui assommerait un chien; celui qui présente une offrande est comme celui qui offrirait le sang d'un pourceau, et celui qui se souvient de l'encens est comme celui qui révérerait une idole. Ils ont pris plaisir et se sont habitués à toutes ces choses, et leur âme a fait ses délices de ces abominations. \EVERSE}
\newcommand{\isLXVIvIVfr}{\VERSE  Moi aussi, Je prendrai plaisir à Me moquer d'eux, et Je ferai venir sur eux ce qu'ils craignaient; car J'ai appelé, et personne n'a répondu; J'ai parlé, et ils n'ont pas entendu; mais ils ont fait le mal devant Mes yeux, et ils ont choisi ce que Je ne voulais pas. \EVERSE}
\newcommand{\isLXVIvVfr}{\VERSE  Ecoutez la parole du Seigneur, vous qui l'écoutez avec tremblement. Vos frères qui vous haïssent et qui vous rejettent à cause de Mon nom ont dit : Que le Seigneur montre Sa gloire, et nous verrons votre joie; mais ils seront eux-mêmes confondus. \EVERSE}
\newcommand{\isLXVIvVIfr}{\VERSE  Voix du peuple qui retentit de la ville, voix qui vient du temple, voix du Seigneur qui rend à Ses ennemis ce qu'ils méritent. \EVERSE}
\newcommand{\isLXVIvVIIfr}{\VERSE  Avant d'être en travail elle a enfanté; avant le temps de l'enfantement, elle a mis au monde un enfant mâle. \EVERSE}
\newcommand{\isLXVIvVIIIfr}{\VERSE  Qui a jamais entendu pareille chose? qui a jamais rien vu de semblable? La terre produit-elle son fruit en un seul jour, un peuple est-il engendré en même temps? Car Sion, à peine en travail, a enfanté tous ses fils. \EVERSE}
\newcommand{\isLXVIvIXfr}{\VERSE  Moi qui fais enfanter les autres n'enfanterai-Je pas aussi? dit le Seigneur; Moi qui donne aux autres la fécondité, demeurerai-Je stérile? dit le Seigneur ton Dieu. \EVERSE}
\newcommand{\isLXVIvXfr}{\VERSE  Réjouissez-vous avec Jérusalem, et soyez dans l'allégresse avec elle, vous tous qui l'aimez; joignez votre joie à la sienne, vous tous qui pleurez sur elle; \EVERSE}
\newcommand{\isLXVIvXIfr}{\VERSE  afin que vous suciez et que vous tiriez de ses mamelles le lait de ses consolations, et que vous savouriez avec délices la plénitude de sa gloire. \EVERSE}
\newcommand{\isLXVIvXIIfr}{\VERSE  Car voici ce que dit le Seigneur : Je ferai couler sur elle comme un fleuve de paix, et la gloire des nations comme un torrent qui déborde; vous sucerez son lait, on vous portera à la mamelle, et on vous caressera sur les genoux. \EVERSE}
\newcommand{\isLXVIvXIIIfr}{\VERSE  Comme quelqu'un que sa mère caresse, ainsi Je vous consolerai, et vous serez consolés dans Jérusalem. \EVERSE}
\newcommand{\isLXVIvXIVfr}{\VERSE  Vous le verrez, et votre coeur sera dans la joie, et vos os reprendront de la vigueur comme l'herbe; et le Seigneur fera connaître Sa main puissante à Ses serviteurs, et Il S'irritera contre Ses ennemis. \EVERSE}
\newcommand{\isLXVIvXVfr}{\VERSE  Car le Seigneur viendra dans un feu, et Son char sera comme un tourbillon, pour répandre Son indignation, Sa fureur et Ses menaces en flammes de feu; \EVERSE}
\newcommand{\isLXVIvXVIfr}{\VERSE  car c'est par le feu que le Seigneur jugera, et par Son glaive qu'Il châtiera toute chair; et ceux que le Seigneur tuera seront nombreux. \EVERSE}
\newcommand{\isLXVIvXVIIfr}{\VERSE  Ceux qui se sanctifiaient et qui croyaient se purifier dans les jardins, la porte fermée, qui mangeaient de la chair de porc, des choses abominables et des souris, périront tous ensemble, dit le Seigneur. \EVERSE}
\newcommand{\isLXVIvXVIIIfr}{\VERSE  Mais Moi, Je viens recueillir leurs oeuvres et leurs pensées, et les assembler avec toutes les nations et toutes les langues; ils viendront, et ils verront Ma gloire. \EVERSE}
\newcommand{\isLXVIvXIXfr}{\VERSE  Je mettrai un signe parmi eux, et J'enverrai de ceux d'entre eux qui auront été sauvés vers les nations, du côté de la mer, dans l'Afrique et la Lydie, chez ceux qui sont armés de flèches, dans l'Italie et la Grèce, dans les îles lointaines, vers ceux qui n'ont jamais entendu parler de Moi, et qui n'ont pas vu Ma gloire. Ils annonceront Ma gloire aux gentils; \EVERSE}
\newcommand{\isLXVIvXXfr}{\VERSE  et ils amèneront tous vos frères du milieu de toutes les nations, comme un présent pour le Seigneur, sur des chevaux, sur des chars, sur des litières, sur des mulets et sur des chariots, à Ma montagne sainte de Jérusalem, dit le Seigneur; comme lorsque les enfants d'Israël apportent une offrande au temple du Seigneur dans un vase pur. \EVERSE}
\newcommand{\isLXVIvXXIfr}{\VERSE  Et J'en choisirai parmi eux pour prêtres et lévites, dit le Seigneur. \EVERSE}
\newcommand{\isLXVIvXXIIfr}{\VERSE  Car comme les cieux nouveaux et la terre nouvelle que Je vais créer subsisteront toujours devant Moi, dit le Seigneur, ainsi subsisteront votre race et votre nom. \EVERSE}
\newcommand{\isLXVIvXXIIIfr}{\VERSE  Et de mois en mois, et de sabbat en sabbat, toute chair viendra se prosterner devant Moi, dit le Seigneur. \EVERSE}
\newcommand{\isLXVIvXXIVfr}{\VERSE  Et ils sortiront, et ils verront les cadavres de ceux qui se sont révoltés contre Moi : leur ver ne mourra pas, et leur feu ne s'éteindra pas, et leur vue sera un objet de dégoût pour toute chair. \EVERSE}

\newcommand{\lcIvI}{\VERSE  Quoniam quidem multi conati sunt ordinare narrationem, quæ in nobis completæ sunt, rerum : \EVERSE}
\newcommand{\lcIvII}{\VERSE  sicut tradiderunt nobis, qui ab initio ipsi viderunt, et ministri fuerunt sermonis : \EVERSE}
\newcommand{\lcIvIII}{\VERSE  visum est et mihi, assecuto omnia a principio diligenter, ex ordine tibi scribere, optime Theophile, \EVERSE}
\newcommand{\lcIvIV}{\VERSE  ut cognoscas eorum verborum, de quibus eruditus es, veritatem. \EVERSE}
\newcommand{\lcIvV}{\VERSE  Fuit in diebus Herodis, regis Judææ, sacerdos quidam nomine Zacharias de vice Abia, et uxor illius de filiabus Aaron, et nomen ejus Elisabeth. \EVERSE}
\newcommand{\lcIvVI}{\VERSE  Erant autem justi ambo ante Deum, incedentes in omnibus mandatis et justificationibus Domini sine querela. \EVERSE}
\newcommand{\lcIvVII}{\VERSE  Et non erat illis filius, eo quod esset Elisabeth sterilis, et ambo processissent in diebus suis. \EVERSE}
\newcommand{\lcIvVIII}{\VERSE  Factum est autem, cum sacerdotio fungeretur in ordine vicis suæ ante Deum, \EVERSE}
\newcommand{\lcIvIX}{\VERSE  secundum consuetudinem sacerdotii, sorte exiit ut incensum poneret, ingressus in templum Domini : \EVERSE}
\newcommand{\lcIvX}{\VERSE  et omnis multitudo populi erat orans foris hora incensi. \EVERSE}
\newcommand{\lcIvXI}{\VERSE  Apparuit autem illi angelus Domini, stans a dextris altaris incensi. \EVERSE}
\newcommand{\lcIvXII}{\VERSE  Et Zacharias turbatus est videns, et timor irruit super eum. \EVERSE}
\newcommand{\lcIvXIII}{\VERSE  Ait autem ad illum angelus : Ne timeas, Zacharia, quoniam exaudita est deprecatio tua : et uxor tua Elisabeth pariet tibi filium, et vocabis nomen ejus Joannem : \EVERSE}
\newcommand{\lcIvXIV}{\VERSE  et erit gaudium tibi, et exsultatio, et multi in nativitate ejus gaudebunt : \EVERSE}
\newcommand{\lcIvXV}{\VERSE  erit enim magnus coram Domino : et vinum et siceram non bibet, et Spiritu Sancto replebitur adhuc ex utero matris suæ : \EVERSE}
\newcommand{\lcIvXVI}{\VERSE  et multos filiorum Israël convertet ad Dominum Deum ipsorum : \EVERSE}
\newcommand{\lcIvXVII}{\VERSE  et ipse præcedet ante illum in spiritu et virtute Eliæ : ut convertat corda patrum in filios, et incredulos ad prudentiam justorum, parare Domino plebem perfectam. \EVERSE}
\newcommand{\lcIvXVIII}{\VERSE  Et dixit Zacharias ad angelum : Unde hoc sciam ? ego enim sum senex, et uxor mea processit in diebus suis. \EVERSE}
\newcommand{\lcIvXIX}{\VERSE  Et respondens angelus dixit ei : Ego sum Gabriel, qui asto ante Deum : et missus sum loqui ad te, et hæc tibi evangelizare. \EVERSE}
\newcommand{\lcIvXX}{\VERSE  Et ecce eris tacens, et non poteris loqui usque in diem quo hæc fiant, pro eo quod non credidisti verbis meis, quæ implebuntur in tempore suo. \EVERSE}
\newcommand{\lcIvXXI}{\VERSE  Et erat plebs exspectans Zachariam : et mirabantur quod tardaret ipse in templo. \EVERSE}
\newcommand{\lcIvXXII}{\VERSE  Egressus autem non poterat loqui ad illos, et cognoverunt quod visionem vidisset in templo. Et ipse erat innuens illis, et permansit mutus. \EVERSE}
\newcommand{\lcIvXXIII}{\VERSE  Et factum est, ut impleti sunt dies officii ejus, abiit in domum suam : \EVERSE}
\newcommand{\lcIvXXIV}{\VERSE  post hos autem dies concepit Elisabeth uxor ejus, et occultabat se mensibus quinque, dicens : \EVERSE}
\newcommand{\lcIvXXV}{\VERSE  Quia sic fecit mihi Dominus in diebus, quibus respexit auferre opprobrium meum inter homines. \EVERSE}
\newcommand{\lcIvXXVI}{\VERSE  In mense autem sexto, missus est angelus Gabriel a Deo in civitatem Galilææ, cui nomen Nazareth, \EVERSE}
\newcommand{\lcIvXXVII}{\VERSE  ad virginem desponsatam viro, cui nomen erat Joseph, de domo David : et nomen virginis Maria. \EVERSE}
\newcommand{\lcIvXXVIII}{\VERSE  Et ingressus angelus ad eam dixit : Ave gratia plena : Dominus tecum : benedicta tu in mulieribus. \EVERSE}
\newcommand{\lcIvXXIX}{\VERSE  Quæ cum audisset, turbata est in sermone ejus, et cogitabat qualis esset ista salutatio. \EVERSE}
\newcommand{\lcIvXXX}{\VERSE  Et ait angelus ei : Ne timeas, Maria : invenisti enim gratiam apud Deum. \EVERSE}
\newcommand{\lcIvXXXI}{\VERSE  Ecce concipies in utero, et paries filium, et vocabis nomen ejus Jesum : \EVERSE}
\newcommand{\lcIvXXXII}{\VERSE  hic erit magnus, et Filius Altissimi vocabitur, et dabit illi Dominus Deus sedem David patris ejus : et regnabit in domo Jacob in æternum, \EVERSE}
\newcommand{\lcIvXXXIII}{\VERSE  et regni ejus non erit finis. \EVERSE}
\newcommand{\lcIvXXXIV}{\VERSE  Dixit autem Maria ad angelum : Quomodo fiet istud, quoniam virum non cognosco ? \EVERSE}
\newcommand{\lcIvXXXV}{\VERSE  Et respondens angelus dixit ei : Spiritus Sanctus superveniet in te, et virtus Altissimi obumbrabit tibi. Ideoque et quod nascetur ex te sanctum, vocabitur Filius Dei. \EVERSE}
\newcommand{\lcIvXXXVI}{\VERSE  Et ecce Elisabeth cognata tua, et ipsa concepit filium in senectute sua : et hic mensis sextus est illi, quæ vocatur sterilis : \EVERSE}
\newcommand{\lcIvXXXVII}{\VERSE  quia non erit impossibile apud Deum omne verbum. \EVERSE}
\newcommand{\lcIvXXXVIII}{\VERSE  Dixit autem Maria : Ecce ancilla Domini : fiat mihi secundum verbum tuum. Et discessit ab illa angelus. \EVERSE}
\newcommand{\lcIvXXXIX}{\VERSE  Exsurgens autem Maria in diebus illis, abiit in montana cum festinatione, in civitatem Juda : \EVERSE}
\newcommand{\lcIvXL}{\VERSE  et intravit in domum Zachariæ, et salutavit Elisabeth. \EVERSE}
\newcommand{\lcIvXLI}{\VERSE  Et factum est, ut audivit salutationem Mariæ Elisabeth, exsultavit infans in utero ejus : et repleta est Spiritu Sancto Elisabeth : \EVERSE}
\newcommand{\lcIvXLII}{\VERSE  et exclamavit voce magna, et dixit : Benedicta tu inter mulieres, et benedictus fructus ventris tui. \EVERSE}
\newcommand{\lcIvXLIII}{\VERSE  Et unde hoc mihi, ut veniat mater Domini mei ad me ? \EVERSE}
\newcommand{\lcIvXLIV}{\VERSE  Ecce enim ut facta est vox salutationis tuæ in auribus meis, exsultavit in gaudio infans in utero meo. \EVERSE}
\newcommand{\lcIvXLV}{\VERSE  Et beata, quæ credidisti, quoniam perficientur ea, quæ dicta sunt tibi a Domino. \EVERSE}
\newcommand{\lcIvXLVI}{\VERSE  Et ait Maria : Magnificat anima mea Dominum : \EVERSE}
\newcommand{\lcIvXLVII}{\VERSE  et exsultavit spiritus meus in Deo salutari meo. \EVERSE}
\newcommand{\lcIvXLVIII}{\VERSE  Quia respexit humilitatem ancillæ suæ : ecce enim ex hoc beatam me dicent omnes generationes, \EVERSE}
\newcommand{\lcIvXLIX}{\VERSE  quia fecit mihi magna qui potens est : et sanctum nomen ejus, \EVERSE}
\newcommand{\lcIvL}{\VERSE  et misericordia ejus a progenie in progenies timentibus eum. \EVERSE}
\newcommand{\lcIvLI}{\VERSE  Fecit potentiam in brachio suo : dispersit superbos mente cordis sui. \EVERSE}
\newcommand{\lcIvLII}{\VERSE  Deposuit potentes de sede, et exaltavit humiles. \EVERSE}
\newcommand{\lcIvLIII}{\VERSE  Esurientes implevit bonis : et divites dimisit inanes. \EVERSE}
\newcommand{\lcIvLIV}{\VERSE  Suscepit Israël puerum suum, recordatus misericordiæ suæ : \EVERSE}
\newcommand{\lcIvLV}{\VERSE  sicut locutus est ad patres nostros, Abraham et semini ejus in sæcula. \EVERSE}
\newcommand{\lcIvLVI}{\VERSE  Mansit autem Maria cum illa quasi mensibus tribus : et reversa est in domum suam. \EVERSE}
\newcommand{\lcIvLVII}{\VERSE  Elisabeth autem impletum est tempus pariendi, et peperit filium. \EVERSE}
\newcommand{\lcIvLVIII}{\VERSE  Et audierunt vicini et cognati ejus quia magnificavit Dominus misericordiam suam cum illa, et congratulabantur ei. \EVERSE}
\newcommand{\lcIvLIX}{\VERSE  Et factum est in die octavo, venerunt circumcidere puerum, et vocabant eum nomine patris sui Zachariam. \EVERSE}
\newcommand{\lcIvLX}{\VERSE  Et respondens mater ejus, dixit : Nequaquam, sed vocabitur Joannes. \EVERSE}
\newcommand{\lcIvLXI}{\VERSE  Et dixerunt ad illam : Quia nemo est in cognatione tua, qui vocetur hoc nomine. \EVERSE}
\newcommand{\lcIvLXII}{\VERSE  Innuebant autem patri ejus, quem vellet vocari eum. \EVERSE}
\newcommand{\lcIvLXIII}{\VERSE  Et postulans pugillarem scripsit, dicens : Joannes est nomen ejus. Et mirati sunt universi. \EVERSE}
\newcommand{\lcIvLXIV}{\VERSE  Apertum est autem illico os ejus, et lingua ejus, et loquebatur benedicens Deum. \EVERSE}
\newcommand{\lcIvLXV}{\VERSE  Et factus est timor super omnes vicinos eorum : et super omnia montana Judææ divulgabantur omnia verba hæc : \EVERSE}
\newcommand{\lcIvLXVI}{\VERSE  et posuerunt omnes qui audierant in corde suo, dicentes : Quis, putas, puer iste erit ? etenim manus Domini erat cum illo. \EVERSE}
\newcommand{\lcIvLXVII}{\VERSE  Et Zacharias pater ejus repletus est Spiritu Sancto : et prophetavit, dicens : \EVERSE}
\newcommand{\lcIvLXVIII}{\VERSE  Benedictus Dominus Deus Israël, quia visitavit, et fecit redemptionem plebis suæ : \EVERSE}
\newcommand{\lcIvLXIX}{\VERSE  et erexit cornu salutis nobis in domo David pueri sui, \EVERSE}
\newcommand{\lcIvLXX}{\VERSE  sicut locutum est per os sanctorum, qui a sæculo sunt, prophetarum ejus : \EVERSE}
\newcommand{\lcIvLXXI}{\VERSE  salutem ex inimicis nostris, et de manu omnium qui oderunt nos : \EVERSE}
\newcommand{\lcIvLXXII}{\VERSE  ad faciendam misericordiam cum patribus nostris : et memorari testamenti sui sancti : \EVERSE}
\newcommand{\lcIvLXXIII}{\VERSE  jusjurandum, quod juravit ad Abraham patrem nostrum, daturum se nobis \EVERSE}
\newcommand{\lcIvLXXIV}{\VERSE  ut sine timore, de manu inimicorum nostrorum liberati, serviamus illi \EVERSE}
\newcommand{\lcIvLXXV}{\VERSE  in sanctitate et justitia coram ipso, omnibus diebus nostris. \EVERSE}
\newcommand{\lcIvLXXVI}{\VERSE  Et tu puer, propheta Altissimi vocaberis : præibis enim ante faciem Domini parare vias ejus, \EVERSE}
\newcommand{\lcIvLXXVII}{\VERSE  ad dandam scientiam salutis plebi ejus in remissionem peccatorum eorum \EVERSE}
\newcommand{\lcIvLXXVIII}{\VERSE  per viscera misericordiæ Dei nostri, in quibus visitavit nos, oriens ex alto : \EVERSE}
\newcommand{\lcIvLXXIX}{\VERSE  illuminare his qui in tenebris et in umbra mortis sedent : ad dirigendos pedes nostros in viam pacis. \EVERSE}
\newcommand{\lcIvLXXX}{\VERSE  Puer autem crescebat, et confortabatur spiritu : et erat in desertis usque in diem ostensionis suæ ad Israël. \EVERSE}
\newcommand{\lcIIvI}{\VERSE  Factum est autem in diebus illis, exiit edictum a Cæsare Augusto ut describeretur universus orbis. \EVERSE}
\newcommand{\lcIIvII}{\VERSE  Hæc descriptio prima facta est a præside Syriæ Cyrino : \EVERSE}
\newcommand{\lcIIvIII}{\VERSE  et ibant omnes ut profiterentur singuli in suam civitatem. \EVERSE}
\newcommand{\lcIIvIV}{\VERSE  Ascendit autem et Joseph a Galilæa de civitate Nazareth in Judæam, in civitatem David, quæ vocatur Bethlehem : eo quod esset de domo et familia David, \EVERSE}
\newcommand{\lcIIvV}{\VERSE  ut profiteretur cum Maria desponsata sibi uxore prægnante. \EVERSE}
\newcommand{\lcIIvVI}{\VERSE  Factum est autem, cum essent ibi, impleti sunt dies ut pareret. \EVERSE}
\newcommand{\lcIIvVII}{\VERSE  Et peperit filium suum primogenitum, et pannis eum involvit, et reclinavit eum in præsepio : quia non erat eis locus in diversorio. \EVERSE}
\newcommand{\lcIIvVIII}{\VERSE  Et pastores erant in regione eadem vigilantes, et custodientes vigilias noctis super gregem suum. \EVERSE}
\newcommand{\lcIIvIX}{\VERSE  Et ecce angelus Domini stetit juxta illos, et claritas Dei circumfulsit illos, et timuerunt timore magno. \EVERSE}
\newcommand{\lcIIvX}{\VERSE  Et dixit illis angelus : Nolite timere : ecce enim evangelizo vobis gaudium magnum, quod erit omni populo : \EVERSE}
\newcommand{\lcIIvXI}{\VERSE  quia natus est vobis hodie Salvator, qui est Christus Dominus, in civitate David. \EVERSE}
\newcommand{\lcIIvXII}{\VERSE  Et hoc vobis signum : invenietis infantem pannis involutum, et positum in præsepio. \EVERSE}
\newcommand{\lcIIvXIII}{\VERSE  Et subito facta est cum angelo multitudo militiæ cælestis laudantium Deum, et dicentium : \EVERSE}
\newcommand{\lcIIvXIV}{\VERSE  Gloria in altissimis Deo, et in terra pax hominibus bonæ voluntatis. \EVERSE}
\newcommand{\lcIIvXV}{\VERSE  Et factum est, ut discesserunt ab eis angeli in cælum : pastores loquebantur ad invicem : Transeamus usque Bethlehem, et videamus hoc verbum, quod factum est, quod Dominus ostendit nobis. \EVERSE}
\newcommand{\lcIIvXVI}{\VERSE  Et venerunt festinantes : et invenerunt Mariam, et Joseph, et infantem positum in præsepio. \EVERSE}
\newcommand{\lcIIvXVII}{\VERSE  Videntes autem cognoverunt de verbo, quod dictum erat illis de puero hoc. \EVERSE}
\newcommand{\lcIIvXVIII}{\VERSE  Et omnes qui audierunt, mirati sunt : et de his quæ dicta erant a pastoribus ad ipsos. \EVERSE}
\newcommand{\lcIIvXIX}{\VERSE  Maria autem conservabat omnia verba hæc, conferens in corde suo. \EVERSE}
\newcommand{\lcIIvXX}{\VERSE  Et reversi sunt pastores glorificantes et laudantes Deum in omnibus quæ audierant et viderant, sicut dictum est ad illos. \EVERSE}
\newcommand{\lcIIvXXI}{\VERSE  Et postquam consummati sunt dies octo, ut circumcideretur puer, vocatum est nomen ejus Jesus, quod vocatum est ab angelo priusquam in utero conciperetur. \EVERSE}
\newcommand{\lcIIvXXII}{\VERSE  Et postquam impleti sunt dies purgationis ejus secundum legem Moysi, tulerunt illum in Jerusalem, ut sisterent eum Domino, \EVERSE}
\newcommand{\lcIIvXXIII}{\VERSE  sicut scriptum est in lege Domini : Quia omne masculinum adaperiens vulvam, sanctum Domino vocabitur : \EVERSE}
\newcommand{\lcIIvXXIV}{\VERSE  et ut darent hostiam secundum quod dictum est in lege Domini, par turturum, aut duos pullos columbarum. \EVERSE}
\newcommand{\lcIIvXXV}{\VERSE  Et ecce homo erat in Jerusalem, cui nomen Simeon, et homo iste justus, et timoratus, exspectans consolationem Israël : et Spiritus Sanctus erat in eo. \EVERSE}
\newcommand{\lcIIvXXVI}{\VERSE  Et responsum acceperat a Spiritu Sancto, non visurum se mortem, nisi prius videret Christum Domini. \EVERSE}
\newcommand{\lcIIvXXVII}{\VERSE  Et venit in spiritu in templum. Et cum inducerent puerum Jesum parentes ejus, ut facerent secundum consuetudinem legis pro eo, \EVERSE}
\newcommand{\lcIIvXXVIII}{\VERSE  et ipse accepit eum in ulnas suas : et benedixit Deum, et dixit : \EVERSE}
\newcommand{\lcIIvXXIX}{\VERSE  Nunc dimittis servum tuum Domine, secundum verbum tuum in pace : \EVERSE}
\newcommand{\lcIIvXXX}{\VERSE  quia viderunt oculi mei salutare tuum, \EVERSE}
\newcommand{\lcIIvXXXI}{\VERSE  quod parasti ante faciem omnium populorum : \EVERSE}
\newcommand{\lcIIvXXXII}{\VERSE  lumen ad revelationem gentium, et gloriam plebis tuæ Israël. \EVERSE}
\newcommand{\lcIIvXXXIII}{\VERSE  Et erat pater ejus et mater mirantes super his quæ dicebantur de illo. \EVERSE}
\newcommand{\lcIIvXXXIV}{\VERSE  Et benedixit illis Simeon, et dixit ad Mariam matrem ejus : Ecce positus est hic in ruinam et in resurrectionem multorum in Israël, et in signum cui contradicetur : \EVERSE}
\newcommand{\lcIIvXXXV}{\VERSE  et tuam ipsius animam pertransibit gladius ut revelentur ex multis cordibus cogitationes. \EVERSE}
\newcommand{\lcIIvXXXVI}{\VERSE  Et erat Anna prophetissa, filia Phanuel, de tribu Aser : hæc processerat in diebus multis, et vixerat cum viro suo annis septem a virginitate sua. \EVERSE}
\newcommand{\lcIIvXXXVII}{\VERSE  Et hæc vidua usque ad annos octoginta quatuor : quæ non discedebat de templo, jejuniis et obsecrationibus serviens nocte ac die. \EVERSE}
\newcommand{\lcIIvXXXVIII}{\VERSE  Et hæc, ipsa hora superveniens, confitebatur Domino : et loquebatur de illo omnibus, qui exspectabant redemptionem Israël. \EVERSE}
\newcommand{\lcIIvXXXIX}{\VERSE  Et ut perfecerunt omnia secundum legem Domini, reversi sunt in Galilæam in civitatem suam Nazareth. \EVERSE}
\newcommand{\lcIIvXL}{\VERSE  Puer autem crescebat, et confortabatur plenus sapientia : et gratia Dei erat in illo. \EVERSE}
\newcommand{\lcIIvXLI}{\VERSE  Et ibant parentes ejus per omnes annos in Jerusalem, in die solemni Paschæ. \EVERSE}
\newcommand{\lcIIvXLII}{\VERSE  Et cum factus esset annorum duodecim, ascendentibus illis Jerosolymam secundum consuetudinem diei festi, \EVERSE}
\newcommand{\lcIIvXLIII}{\VERSE  consummatisque diebus, cum redirent, remansit puer Jesus in Jerusalem, et non cognoverunt parentes ejus. \EVERSE}
\newcommand{\lcIIvXLIV}{\VERSE  Existimantes autem illum esse in comitatu, venerunt iter diei, et requirebant eum inter cognatos et notos. \EVERSE}
\newcommand{\lcIIvXLV}{\VERSE  Et non invenientes, regressi sunt in Jerusalem, requirentes eum. \EVERSE}
\newcommand{\lcIIvXLVI}{\VERSE  Et factum est, post triduum invenerunt illum in templo sedentem in medio doctorum, audientem illos, et interrogantem eos. \EVERSE}
\newcommand{\lcIIvXLVII}{\VERSE  Stupebant autem omnes qui eum audiebant, super prudentia et responsis ejus. \EVERSE}
\newcommand{\lcIIvXLVIII}{\VERSE  Et videntes admirati sunt. Et dixit mater ejus ad illum : Fili, quid fecisti nobis sic ? ecce pater tuus et ego dolentes quærebamus te. \EVERSE}
\newcommand{\lcIIvXLIX}{\VERSE  Et ait ad illos : Quid est quod me quærebatis ? nesciebatis quia in his quæ Patris mei sunt, oportet me esse ? \EVERSE}
\newcommand{\lcIIvL}{\VERSE  Et ipsi non intellexerunt verbum quod locutus est ad eos. \EVERSE}
\newcommand{\lcIIvLI}{\VERSE  Et descendit cum eis, et venit Nazareth : et erat subditus illis. Et mater ejus conservabat omnia verba hæc in corde suo. \EVERSE}
\newcommand{\lcIIvLII}{\VERSE  Et Jesus proficiebat sapientia, et ætate, et gratia apud Deum et homines. \EVERSE}
\newcommand{\lcIIIvI}{\VERSE  Anno autem quintodecimo imperii Tiberii Cæsaris, procurante Pontio Pilato Judæam, tetrarcha autem Galiææ Herode, Philippo autem fratre ejus tetrarcha Iturææ, et Trachonitidis regionis, et Lysania Abilinæ tetrarcha, \EVERSE}
\newcommand{\lcIIIvII}{\VERSE  sub principibus sacerdotum Anna et Caipha : factum est verbum Domini super Joannem, Zachariæ filium, in deserto. \EVERSE}
\newcommand{\lcIIIvIII}{\VERSE  Et venit in omnem regionem Jordanis, prædicans baptismum pœnitentiæ in remissionem peccatorum, \EVERSE}
\newcommand{\lcIIIvIV}{\VERSE  sicut scriptum est in libro sermonum Isaiæ prophetæ : Vox clamantis in deserto : Parate viam Domini ; rectas facite semitas ejus : \EVERSE}
\newcommand{\lcIIIvV}{\VERSE  omnis vallis implebitur, et omnis mons, et collis humiliabitur : et erunt prava in directa, et aspera in vias planas : \EVERSE}
\newcommand{\lcIIIvVI}{\VERSE  et videbit omnis caro salutare Dei. \EVERSE}
\newcommand{\lcIIIvVII}{\VERSE  Dicebat ergo ad turbas quæ exibant ut baptizarentur ab ipso : Genimina viperarum, quis ostendit vobis fugere a ventura ira ? \EVERSE}
\newcommand{\lcIIIvVIII}{\VERSE  Facite ergo fructus dignos pœnitentiæ, et ne cœperitis dicere : Patrem habemus Abraham. Dico enim vobis quia potens est Deus de lapidibus istis suscitare filios Abrahæ. \EVERSE}
\newcommand{\lcIIIvIX}{\VERSE  Jam enim securis ad radicem arborum posita est. Omnis ergo arbor non faciens fructum bonum, excidetur, et in ignem mittetur. \EVERSE}
\newcommand{\lcIIIvX}{\VERSE  Et interrogabant eum turbæ, dicentes : Quid ergo faciemus ? \EVERSE}
\newcommand{\lcIIIvXI}{\VERSE  Respondens autem dicebat illis : Qui habet duas tunicas, det non habenti : et qui habet escas, similiter faciat. \EVERSE}
\newcommand{\lcIIIvXII}{\VERSE  Venerunt autem et publicani ut baptizarentur, et dixerunt ad illum : Magister, quid faciemus ? \EVERSE}
\newcommand{\lcIIIvXIII}{\VERSE  At ille dixit ad eos : Nihil amplius, quam quod constitutum est vobis, faciatis. \EVERSE}
\newcommand{\lcIIIvXIV}{\VERSE  Interrogabant autem eum et milites, dicentes : Quid faciemus et nos ? Et ait illis : Neminem concutiatis, neque calumniam faciatis : et contenti estote stipendiis vestris. \EVERSE}
\newcommand{\lcIIIvXV}{\VERSE  Existimante autem populo, et cogitantibus omnibus in cordibus suis de Joanne, ne forte ipse esset Christus, \EVERSE}
\newcommand{\lcIIIvXVI}{\VERSE  respondit Joannes, dicens omnibus : Ego quidem aqua baptizo vos : veniet autem fortior me, cujus non sum dignus solvere corrigiam calceamentorum ejus : ipse vos baptizabit in Spiritu Sancto et igni : \EVERSE}
\newcommand{\lcIIIvXVII}{\VERSE  cujus ventilabrum in manu ejus, et purgabit aream suam, et congregabit triticum in horreum suum, paleas autem comburet igni inextinguibili. \EVERSE}
\newcommand{\lcIIIvXVIII}{\VERSE  Multa quidem et alia exhortans evangelizabat populo. \EVERSE}
\newcommand{\lcIIIvXIX}{\VERSE  Herodes autem tetrarcha cum corriperetur ab illo de Herodiade uxore fratris sui, et de omnibus malis quæ fecit Herodes, \EVERSE}
\newcommand{\lcIIIvXX}{\VERSE  adjecit et hoc super omnia, et inclusit Joannem in carcere. \EVERSE}
\newcommand{\lcIIIvXXI}{\VERSE  Factum est autem cum baptizaretur omnis populus, et Jesu baptizato, et orante, apertum est cælum : \EVERSE}
\newcommand{\lcIIIvXXII}{\VERSE  et descendit Spiritus Sanctus corporali specie sicut columba in ipsum : et vox de cælo facta est : Tu es filius meus dilectus, in te complacui mihi. \EVERSE}
\newcommand{\lcIIIvXXIII}{\VERSE  Et ipse Jesus erat incipiens quasi annorum triginta, ut putabatur, filius Joseph, qui fuit Heli, qui fuit Mathat, \EVERSE}
\newcommand{\lcIIIvXXIV}{\VERSE  qui fuit Levi, qui fuit Melchi, qui fuit Janne, qui fuit Joseph, \EVERSE}
\newcommand{\lcIIIvXXV}{\VERSE  qui fuit Mathathiæ, qui fuit Amos, qui fuit Nahum, qui fuit Hesli, qui fuit Nagge, \EVERSE}
\newcommand{\lcIIIvXXVI}{\VERSE  qui fuit Mahath, qui fuit Mathathiæ, qui fuit Semei, qui fuit Joseph, qui fuit Juda, \EVERSE}
\newcommand{\lcIIIvXXVII}{\VERSE  qui fuit Joanna, qui fuit Resa, qui fuit Zorobabel, qui fuit Salatheil, qui fuit Neri, \EVERSE}
\newcommand{\lcIIIvXXVIII}{\VERSE  qui fuit Melchi, qui fuit Addi, qui fuit Cosan, qui fuit Elmadan, qui fuit Her, \EVERSE}
\newcommand{\lcIIIvXXIX}{\VERSE  qui fuit Jesu, qui fuit Eliezer, qui fuit Jorim, qui fuit Mathat, qui fuit Levi, \EVERSE}
\newcommand{\lcIIIvXXX}{\VERSE  qui fuit Simeon, qui fuit Juda, qui fuit Joseph, qui fuit Jona, qui fuit Eliakim, \EVERSE}
\newcommand{\lcIIIvXXXI}{\VERSE  qui fuit Melea, qui fuit Menna, qui fuit Mathatha, qui fuit Natham, qui fuit David, \EVERSE}
\newcommand{\lcIIIvXXXII}{\VERSE  qui fuit Jesse, qui fuit Obed, qui fuit Booz, qui fuit Salmon, qui fuit Naasson, \EVERSE}
\newcommand{\lcIIIvXXXIII}{\VERSE  qui fuit Aminadab, qui fuit Aram, qui fuit Esron, qui fuit Phares, qui fuit Judæ, \EVERSE}
\newcommand{\lcIIIvXXXIV}{\VERSE  qui fuit Jacob, qui fuit Isaac, qui fuit Abrahæ, qui fuit Thare, qui fuit Nachor, \EVERSE}
\newcommand{\lcIIIvXXXV}{\VERSE  qui fuit Sarug, qui fuit Ragau, qui fuit Phaleg, qui fuit Heber, qui fuit Sale, \EVERSE}
\newcommand{\lcIIIvXXXVI}{\VERSE  qui fuit Cainan, qui fuit Arphaxad, qui fuit Sem, qui fuit Noë, qui fuit Lamech, \EVERSE}
\newcommand{\lcIIIvXXXVII}{\VERSE  qui fuit Methusale, qui fuit Henoch, qui fuit Jared, qui fuit Malaleel, qui fuit Cainan, \EVERSE}
\newcommand{\lcIIIvXXXVIII}{\VERSE  qui fuit Henos, qui fuit Seth, qui fuit Adam, qui fuit Dei. \EVERSE}
\newcommand{\lcIVvI}{\VERSE  Jesus autem plenus Spiritu Sancto regressus est a Jordane : et agebatur a Spiritu in desertum \EVERSE}
\newcommand{\lcIVvII}{\VERSE  diebus quadraginta, et tentabatur a diabolo. Et nihil manducavit in diebus illis : et consummatis illis esuriit. \EVERSE}
\newcommand{\lcIVvIII}{\VERSE  Dixit autem illi diabolus : Si Filius Dei es, dic lapidi huic ut panis fiat. \EVERSE}
\newcommand{\lcIVvIV}{\VERSE  Et respondit ad illum Jesus : Scriptum est : Quia non in solo pane vivit homo, sed in omni verbo Dei. \EVERSE}
\newcommand{\lcIVvV}{\VERSE  Et duxit illum diabolus in montem excelsum, et ostendit illi omnia regna orbis terræ in momento temporis, \EVERSE}
\newcommand{\lcIVvVI}{\VERSE  et ait illi : Tibi dabo potestatem hanc universam, et gloriam illorum : quia mihi tradita sunt, et cui volo do illa. \EVERSE}
\newcommand{\lcIVvVII}{\VERSE  Tu ergo si adoraveris coram me, erunt tua omnia. \EVERSE}
\newcommand{\lcIVvVIII}{\VERSE  Et respondens Jesus, dixit illi : Scriptum est : Dominum Deum tuum adorabis, et illi soli servies. \EVERSE}
\newcommand{\lcIVvIX}{\VERSE  Et duxit illum in Jerusalem, et statuit eum super pinnam templi, et dixit illi : Si Filius Dei es, mitte te hinc deorsum. \EVERSE}
\newcommand{\lcIVvX}{\VERSE  Scriptum est enim quod angelis suis mandavit de te, ut conservent te : \EVERSE}
\newcommand{\lcIVvXI}{\VERSE  et quia in manibus tollent te, ne forte offendas ad lapidem pedem tuum. \EVERSE}
\newcommand{\lcIVvXII}{\VERSE  Et respondens Jesus, ait illi : Dictum est : Non tentabis Dominum Deum tuum. \EVERSE}
\newcommand{\lcIVvXIII}{\VERSE  Et consummata omni tentatione, diabolus recessit ab illo, usque ad tempus. \EVERSE}
\newcommand{\lcIVvXIV}{\VERSE  Et regressus est Jesus in virtute Spiritus in Galilæam, et fama exiit per universam regionem de illo. \EVERSE}
\newcommand{\lcIVvXV}{\VERSE  Et ipse docebat in synagogis eorum, et magnificabatur ab omnibus. \EVERSE}
\newcommand{\lcIVvXVI}{\VERSE  Et venit Nazareth, ubi erat nutritus, et intravit secundum consuetudinem suam die sabbati in synagogam, et surrexit legere. \EVERSE}
\newcommand{\lcIVvXVII}{\VERSE  Et traditus est illi liber Isaiæ prophetæ. Et ut revolvit librum, invenit locum ubi scriptum erat : \EVERSE}
\newcommand{\lcIVvXVIII}{\VERSE  Spiritus Domini super me : propter quod unxit me, evangelizare pauperibus misit me, sanare contritos corde, \EVERSE}
\newcommand{\lcIVvXIX}{\VERSE  prædicare captivis remissionem, et cæcis visum, dimittere confractos in remissionem, prædicare annum Domini acceptum et diem retributionis. \EVERSE}
\newcommand{\lcIVvXX}{\VERSE  Et cum plicuisset librum, reddit ministro, et sedit. Et omnium in synagoga oculi erant intendentes in eum. \EVERSE}
\newcommand{\lcIVvXXI}{\VERSE  Cœpit autem dicere ad illos : Quia hodie impleta est hæc scriptura in auribus vestris. \EVERSE}
\newcommand{\lcIVvXXII}{\VERSE  Et omnes testimonium illi dabant : et mirabantur in verbis gratiæ, quæ procedebant de ore ipsius, et dicebant : Nonne hic est filius Joseph ? \EVERSE}
\newcommand{\lcIVvXXIII}{\VERSE  Et ait illis : Utique dicetis mihi hanc similitudinem : Medice cura teipsum : quanta audivimus facta in Capharnaum, fac et hic in patria tua. \EVERSE}
\newcommand{\lcIVvXXIV}{\VERSE  Ait autem : Amen dico vobis, quia nemo propheta acceptus est in patria sua. \EVERSE}
\newcommand{\lcIVvXXV}{\VERSE  In veritate dico vobis, multæ viduæ erant in diebus Eliæ in Israël, quando clausum est cælum annis tribus et mensibus sex, cum facta esset fames magna in omni terra : \EVERSE}
\newcommand{\lcIVvXXVI}{\VERSE  et ad nullam illarum missus est Elias, nisi in Sarepta Sidoniæ, ad mulierem viduam. \EVERSE}
\newcommand{\lcIVvXXVII}{\VERSE  Et multi leprosi erant in Israël sub Eliseo propheta : et nemo eorum mundatus est nisi Naaman Syrus. \EVERSE}
\newcommand{\lcIVvXXVIII}{\VERSE  Et repleti sunt omnes in synagoga ira, hæc audientes. \EVERSE}
\newcommand{\lcIVvXXIX}{\VERSE  Et surrexerunt, et ejecerunt illum extra civitatem : et duxerunt illum usque ad supercilium montis, super quem civitas illorum erat ædificata, ut præcipitarent eum. \EVERSE}
\newcommand{\lcIVvXXX}{\VERSE  Ipse autem transiens per medium illorum, ibat. \EVERSE}
\newcommand{\lcIVvXXXI}{\VERSE  Et descendit in Capharnaum civitatem Galilææ, ibique docebat illos sabbatis. \EVERSE}
\newcommand{\lcIVvXXXII}{\VERSE  Et stupebant in doctrina ejus, quia in potestate erat sermo ipsius. \EVERSE}
\newcommand{\lcIVvXXXIII}{\VERSE  Et in synagoga erat homo habens dæmonium immundum, et exclamavit voce magna, \EVERSE}
\newcommand{\lcIVvXXXIV}{\VERSE  dicens : Sine, quid nobis et tibi, Jesu Nazarene ? venisti perdere nos ? scio te quis sis, Sanctus Dei. \EVERSE}
\newcommand{\lcIVvXXXV}{\VERSE  Et increpavit illum Jesus, dicens : Obmutesce, et exi ab eo. Et cum projecisset illum dæmonium in medium, exiit ab illo, nihilque illum nocuit. \EVERSE}
\newcommand{\lcIVvXXXVI}{\VERSE  Et factus est pavor in omnibus, et colloquebantur ad invicem, dicentes : Quod est hoc verbum, quia in potestate et virtute imperat immundis spiritibus, et exeunt ? \EVERSE}
\newcommand{\lcIVvXXXVII}{\VERSE  Et divulgabatur fama de illo in omnem locum regionis. \EVERSE}
\newcommand{\lcIVvXXXVIII}{\VERSE  Surgens autem Jesus de synagoga, introivit in domum Simonis. Socrus autem Simonis tenebatur magnis febribus : et rogaverunt illum pro ea. \EVERSE}
\newcommand{\lcIVvXXXIX}{\VERSE  Et stans super illam imperavit febri : et dimisit illam. Et continuo surgens, ministrabat illis. \EVERSE}
\newcommand{\lcIVvXL}{\VERSE  Cum autem sol occidisset, omnes qui habebant infirmos variis languoribus, ducebant illos ad eum. At ille singulis manus imponens, curabat eos. \EVERSE}
\newcommand{\lcIVvXLI}{\VERSE  Exibant autem dæmonia a multis clamantia, et dicentia : Quia tu es Filius Dei : et increpans non sinebat ea loqui : quia sciebant ipsum esse Christum. \EVERSE}
\newcommand{\lcIVvXLII}{\VERSE  Facta autem die egressus ibat in desertum locum, et turbæ requirebant eum, et venerunt usque ad ipsum : et detinebant illum ne discederet ab eis. \EVERSE}
\newcommand{\lcIVvXLIII}{\VERSE  Quibus ille ait : Quia et aliis civitatibus oportet me evangelizare regnum Dei : quia ideo missus sum. \EVERSE}
\newcommand{\lcIVvXLIV}{\VERSE  Et erat prædicans in synagogis Galilææ. \EVERSE}
\newcommand{\lcVvI}{\VERSE  Factum est autem, cum turbæ irruerunt in eum ut audirent verbum Dei, et ipse stabat secus stagnum Genesareth. \EVERSE}
\newcommand{\lcVvII}{\VERSE  Et vidit duas naves stantes secus stagnum : piscatores autem descenderant, et lavabant retia. \EVERSE}
\newcommand{\lcVvIII}{\VERSE  Ascendens autem in unam navim, quæ erat Simonis, rogavit eum a terra reducere pusillum. Et sedens docebat de navicula turbas. \EVERSE}
\newcommand{\lcVvIV}{\VERSE  Ut cessavit autem loqui, dixit ad Simonem : Duc in altum, et laxate retia vestra in capturam. \EVERSE}
\newcommand{\lcVvV}{\VERSE  Et respondens Simon, dixit illi : Præceptor, per totam noctem laborantes nihil cepimus : in verbo autem tuo laxabo rete. \EVERSE}
\newcommand{\lcVvVI}{\VERSE  Et cum hoc fecissent, concluserunt piscium multitudinem copiosam : rumpebatur autem rete eorum. \EVERSE}
\newcommand{\lcVvVII}{\VERSE  Et annuerunt sociis, qui erant in alia navi, ut venirent, et adjuvarent eos. Et venerunt, et impleverunt ambas naviculas, ita ut pene mergerentur. \EVERSE}
\newcommand{\lcVvVIII}{\VERSE  Quod cum videret Simon Petrus, procidit ad genua Jesu, dicens : Exi a me, quia homo peccator sum, Domine. \EVERSE}
\newcommand{\lcVvIX}{\VERSE  Stupor enim circumdederat eum, et omnes qui cum illo erant, in captura piscium, quam ceperant : \EVERSE}
\newcommand{\lcVvX}{\VERSE  similiter autem Jacobum et Joannem, filios Zebedæi, qui erant socii Simonis. Et ait ad Simonem Jesus : Noli timere : ex hoc jam homines eris capiens. \EVERSE}
\newcommand{\lcVvXI}{\VERSE  Et subductis ad terram navibus, relictis omnibus, secuti sunt eum. \EVERSE}
\newcommand{\lcVvXII}{\VERSE  Et factum est, cum esset in una civitatum, et ecce vir plenus lepra, et videns Jesum, et procidens in faciem, rogavit eum, dicens : Domine, si vis, potes me mundare. \EVERSE}
\newcommand{\lcVvXIII}{\VERSE  Et extendens manum, tetigit eum dicens : Volo : mundare. Et confestim lepra discessit ab illo. \EVERSE}
\newcommand{\lcVvXIV}{\VERSE  Et ipse præcepit illi ut nemini diceret : sed, Vade, ostende te sacerdoti, et offer pro emundatione tua, sicut præcepit Moyses, in testimonium illis. \EVERSE}
\newcommand{\lcVvXV}{\VERSE  Perambulabat autem magis sermo de illo : et conveniebant turbæ multæ ut audirent, et curarentur ab infirmitatibus suis. \EVERSE}
\newcommand{\lcVvXVI}{\VERSE  Ipse autem secedebat in desertum, et orabat. \EVERSE}
\newcommand{\lcVvXVII}{\VERSE  Et factum est in una dierum, et ipse sedebat docens. Et erant pharisæi sedentes, et legis doctores, qui venerant ex omni castello Galilææ, et Judææ, et Jerusalem : et virtus Domini erat ad sanandum eos. \EVERSE}
\newcommand{\lcVvXVIII}{\VERSE  Et ecce viri portantes in lecto hominem, qui erat paralyticus : et quærebant eum inferre, et ponere ante eum. \EVERSE}
\newcommand{\lcVvXIX}{\VERSE  Et non invenientes qua parte illum inferrent præ turba, ascenderunt supra tectum, et per tegulas summiserunt eum cum lecto in medium ante Jesum. \EVERSE}
\newcommand{\lcVvXX}{\VERSE  Quorum fidem ut vidit, dixit : Homo, remittuntur tibi peccata tua. \EVERSE}
\newcommand{\lcVvXXI}{\VERSE  Et cœperunt cogitare scribæ et pharisæi, dicentes : Quis est hic, qui loquitur blasphemias ? quis potest dimittere peccata, nisi solus Deus ? \EVERSE}
\newcommand{\lcVvXXII}{\VERSE  Ut cognovit autem Jesus cogitationes eorum, respondens, dixit ad illos : Quid cogitatis in cordibus vestris ? \EVERSE}
\newcommand{\lcVvXXIII}{\VERSE  Quid est facilius dicere : Dimittuntur tibi peccata : an dicere : Surge, et ambula ? \EVERSE}
\newcommand{\lcVvXXIV}{\VERSE  Ut autem sciatis quia Filius hominis habet potestatem in terra dimittendi peccata, (ait paralytico) tibi dico, surge, tolle lectum tuum, et vade in domum tuam. \EVERSE}
\newcommand{\lcVvXXV}{\VERSE  Et confestim consurgens coram illis, tulit lectum in quo jacebat : et abiit in domum suam, magnificans Deum. \EVERSE}
\newcommand{\lcVvXXVI}{\VERSE  Et stupor apprehendit omnes, et magnificabant Deum. Et repleti sunt timore, dicentes : Quia vidimus mirabilia hodie. \EVERSE}
\newcommand{\lcVvXXVII}{\VERSE  Et post hæc exiit, et vidit publicanum nomine Levi, sedentem ad telonium, et ait illi : Sequere me. \EVERSE}
\newcommand{\lcVvXXVIII}{\VERSE  Et relictis omnibus, surgens secutus est eum. \EVERSE}
\newcommand{\lcVvXXIX}{\VERSE  Et fecit ei convivium magnum Levi in domo sua : et erat turba multa publicanorum, et aliorum qui cum illis erant discumbentes. \EVERSE}
\newcommand{\lcVvXXX}{\VERSE  Et murmurabant pharisæi et scribæ eorum, dicentes ad discipulos ejus : Quare cum publicanis et peccatoribus manducatis et bibitis ? \EVERSE}
\newcommand{\lcVvXXXI}{\VERSE  Et respondens Jesus, dixit ad illos : Non egent qui sani sunt medico, sed qui male habent. \EVERSE}
\newcommand{\lcVvXXXII}{\VERSE  Non veni vocare justos, sed peccatores ad pœnitentiam. \EVERSE}
\newcommand{\lcVvXXXIII}{\VERSE  At illi dixerunt ad eum : Quare discipuli Joannis jejunant frequenter, et obsecrationes faciunt, similiter et pharisæorum : tui autem edunt et bibunt ? \EVERSE}
\newcommand{\lcVvXXXIV}{\VERSE  Quibus ipse ait : Numquid potestis filios sponsi, dum cum illis est sponsus, facere jejunare ? \EVERSE}
\newcommand{\lcVvXXXV}{\VERSE  Venient autem dies, cum ablatus fuerit ab illis sponsus : tunc jejunabunt in illis diebus. \EVERSE}
\newcommand{\lcVvXXXVI}{\VERSE  Dicebat autem et similitudinem ad illos : Quia nemo commissuram a novo vestimento immittit in vestimentum vetus : alioquin et novum rumpit, et veteri non convenit commissura a novo. \EVERSE}
\newcommand{\lcVvXXXVII}{\VERSE  Et nemo mittit vinum novum in utres veteres : alioquin rumpet vinum novum utres, et ipsum effundetur, et utres peribunt : \EVERSE}
\newcommand{\lcVvXXXVIII}{\VERSE  sed vinum novum in utres novos mittendum est, et utraque conservantur. \EVERSE}
\newcommand{\lcVvXXXIX}{\VERSE  Et nemo bibens vetus, statim vult novum : dicit enim : Vetus melius est. \EVERSE}
\newcommand{\lcVIvI}{\VERSE  Factum est autem in sabbato secundo, primo, cum transiret per sata, vellebant discipuli ejus spicas, et manducabant confricantes manibus. \EVERSE}
\newcommand{\lcVIvII}{\VERSE  Quidam autem pharisæorum, dicebant illis : Quid facitis quod non licet in sabbatis ? \EVERSE}
\newcommand{\lcVIvIII}{\VERSE  Et respondens Jesus ad eos, dixit : Nec hoc legistis quod fecit David, cum esurisset ipse, et qui cum illo erant ? \EVERSE}
\newcommand{\lcVIvIV}{\VERSE  quomodo intravit in domum Dei, et panes propositionis sumpsit, et manducavit, et dedit his qui cum ipso erant : quos non licet manducare nisi tantum sacerdotibus ? \EVERSE}
\newcommand{\lcVIvV}{\VERSE  Et dicebat illis : Quia dominus est Filius hominis etiam sabbati. \EVERSE}
\newcommand{\lcVIvVI}{\VERSE  Factum est autem in alio sabbato, ut intraret in synagogam, et doceret. Et erat ibi homo, et manus ejus dextra erat arida. \EVERSE}
\newcommand{\lcVIvVII}{\VERSE  Observabant autem scribæ et pharisæi si in sabbato curaret, ut invenirent unde accusarent eum. \EVERSE}
\newcommand{\lcVIvVIII}{\VERSE  Ipse vero sciebat cogitationes eorum : et ait homini qui habebat manum aridam : Surge, et sta in medium. Et surgens stetit. \EVERSE}
\newcommand{\lcVIvIX}{\VERSE  Ait autem ad illos Jesus : Interrogo vos si licet sabbatis benefacere, an male : animam salvam facere, an perdere ? \EVERSE}
\newcommand{\lcVIvX}{\VERSE  Et circumspectis omnibus dixit homini : Extende manum tuam. Et extendit : et restituta est manus ejus. \EVERSE}
\newcommand{\lcVIvXI}{\VERSE  Ipsi autem repleti sunt insipientia, et colloquebantur ad invicem, quidnam facerent Jesu. \EVERSE}
\newcommand{\lcVIvXII}{\VERSE  Factum est autem in illis diebus, exiit in montem orare, et erat pernoctans in oratione Dei. \EVERSE}
\newcommand{\lcVIvXIII}{\VERSE  Et cum dies factus esset, vocavit discipulos suos : et elegit duodecim ex ipsis (quos et apostolos nominavit) : \EVERSE}
\newcommand{\lcVIvXIV}{\VERSE  Simonem, quem cognominavit Petrum, et Andream fratrem ejus, Jacobum, et Joannem, Philippum, et Bartholomæum, \EVERSE}
\newcommand{\lcVIvXV}{\VERSE  Matthæum, et Thomam, Jacobum Alphæi, et Simonem, qui vocatur Zelotes, \EVERSE}
\newcommand{\lcVIvXVI}{\VERSE  et Judam Jacobi, et Judam Iscariotem, qui fuit proditor. \EVERSE}
\newcommand{\lcVIvXVII}{\VERSE  Et descendens cum illis, stetit in loco campestri, et turba discipulorum ejus, et multitudo copiosa plebis ab omni Judæa, et Jerusalem, et maritima, et Tyri, et Sidonis, \EVERSE}
\newcommand{\lcVIvXVIII}{\VERSE  qui venerant ut audirent eum, et sanarentur a languoribus suis. Et qui vexabantur a spiritibus immundis, curabantur. \EVERSE}
\newcommand{\lcVIvXIX}{\VERSE  Et omnis turba quærebat eum tangere : quia virtus de illo exibat, et sanabat omnes. \EVERSE}
\newcommand{\lcVIvXX}{\VERSE  Et ipse elevatis oculis in discipulis suis, dicebat : Beati pauperes, quia vestrum est regnum Dei. \EVERSE}
\newcommand{\lcVIvXXI}{\VERSE  Beati qui nunc esuritis, quia saturabimini. Beati qui nunc fletis, quia ridebitis. \EVERSE}
\newcommand{\lcVIvXXII}{\VERSE  Beati eritis cum vos oderint homines, et cum separaverint vos, et exprobraverint, et ejicerint nomen vestrum tamquam malum propter Filium hominis. \EVERSE}
\newcommand{\lcVIvXXIII}{\VERSE  Gaudete in illa die, et exsultate : ecce enim merces vestra multa est in cælo : secundum hæc enim faciebant prophetis patres eorum. \EVERSE}
\newcommand{\lcVIvXXIV}{\VERSE  Verumtamen væ vobis divitibus, quia habetis consolationem vestram. \EVERSE}
\newcommand{\lcVIvXXV}{\VERSE  Væ vobis, qui saturati estis : quia esurietis. Væ vobis, qui ridetis nunc : quia lugebitis et flebitis. \EVERSE}
\newcommand{\lcVIvXXVI}{\VERSE  Væ cum benedixerint vobis homines : secundum hæc enim faciebant pseudoprophetis patres eorum. \EVERSE}
\newcommand{\lcVIvXXVII}{\VERSE  Sed vobis dico, qui auditis : diligite inimicos vestros, benefacite his qui oderunt vos. \EVERSE}
\newcommand{\lcVIvXXVIII}{\VERSE  Benedicite maledicentibus vobis, et orate pro calumniantibus vos. \EVERSE}
\newcommand{\lcVIvXXIX}{\VERSE  Et qui te percutit in maxillam, præbe et alteram. Et ab eo qui aufert tibi vestimentum, etiam tunicam noli prohibere. \EVERSE}
\newcommand{\lcVIvXXX}{\VERSE  Omni autem petenti te, tribue : et qui aufert quæ tua sunt, ne repetas. \EVERSE}
\newcommand{\lcVIvXXXI}{\VERSE  Et prout vultis ut faciant vobis homines, et vos facite illis similiter. \EVERSE}
\newcommand{\lcVIvXXXII}{\VERSE  Et si diligitis eos qui vos diligunt, quæ vobis est gratia ? nam et peccatores diligentes se diligunt. \EVERSE}
\newcommand{\lcVIvXXXIII}{\VERSE  Et si benefeceritis his qui vobis benefaciunt, quæ vobis est gratia ? siquidem et peccatores hoc faciunt. \EVERSE}
\newcommand{\lcVIvXXXIV}{\VERSE  Et si mutuum dederitis his a quibus speratis recipere, quæ gratia est vobis ? nam et peccatores peccatoribus fœnerantur, ut recipiant æqualia. \EVERSE}
\newcommand{\lcVIvXXXV}{\VERSE  Verumtamen diligite inimicos vestros : benefacite, et mutuum date, nihil inde sperantes : et erit merces vestra multa, et eritis filii Altissimi, quia ipse benignus est super ingratos et malos. \EVERSE}
\newcommand{\lcVIvXXXVI}{\VERSE  Estote ergo misericordes sicut et Pater vester misericors est. \EVERSE}
\newcommand{\lcVIvXXXVII}{\VERSE  Nolite judicare, et non judicabimini : nolite condemnare, et non condemnabimini. Dimitte, et dimittemini. \EVERSE}
\newcommand{\lcVIvXXXVIII}{\VERSE  Date, et dabitur vobis : mensuram bonam, et confertam, et coagitatam, et supereffluentem dabunt in sinum vestrum. Eadem quippe mensura, qua mensi fueritis, remetietur vobis. \EVERSE}
\newcommand{\lcVIvXXXIX}{\VERSE  Dicebat autem illis et similitudinem : Numquid potest cæcus cæcum ducere ? nonne ambo in foveam cadunt ? \EVERSE}
\newcommand{\lcVIvXL}{\VERSE  Non est discipulus super magistrum : perfectus autem omnis erit, si sit sicut magister ejus. \EVERSE}
\newcommand{\lcVIvXLI}{\VERSE  Quid autem vides festucam in oculo fratris tui, trabem autem, quæ in oculo tuo est, non consideras ? \EVERSE}
\newcommand{\lcVIvXLII}{\VERSE  aut quomodo potes dicere fratri tuo : Frater, sine ejiciam festucam de oculo tuo : ipse in oculo tuo trabem non videns ? Hypocrita, ejice primum trabem de oculo tuo : et tunc perspicies ut educas festucam de oculo fratris tui. \EVERSE}
\newcommand{\lcVIvXLIII}{\VERSE  Non est enim arbor bona, quæ facit fructus malos : neque arbor mala, faciens fructum bonum. \EVERSE}
\newcommand{\lcVIvXLIV}{\VERSE  Unaquæque enim arbor de fructu suo cognoscitur. Neque enim de spinis colligunt ficus : neque de rubo vindemiant uvam. \EVERSE}
\newcommand{\lcVIvXLV}{\VERSE  Bonus homo de bono thesauro cordis sui profert bonum : et malus homo de malo thesauro profert malum. Ex abundantia enim cordis os loquitur. \EVERSE}
\newcommand{\lcVIvXLVI}{\VERSE  Quid autem vocatis me Domine, Domine : et non facitis quæ dico ? \EVERSE}
\newcommand{\lcVIvXLVII}{\VERSE  Omnis qui venit ad me, et audit sermones meos, et facit eos, ostendam vobis cui similis sit : \EVERSE}
\newcommand{\lcVIvXLVIII}{\VERSE  similis est homini ædificanti domum, qui fodit in altum, et posuit fundamentum super petram : inundatione autem facta, illisum est flumen domui illi, et non potuit eam movere : fundata enim erat super petram. \EVERSE}
\newcommand{\lcVIvXLIX}{\VERSE  Qui autem audit, et non facit, similis est homini ædificanti domum suam super terram sine fundamento : in quam illisus est fluvius, et continuo cecidit : et facta est ruina domus illius magna. \EVERSE}
\newcommand{\lcVIIvI}{\VERSE  Cum autem implesset omnia verba sua in aures plebis, intravit Capharnaum. \EVERSE}
\newcommand{\lcVIIvII}{\VERSE  Centurionis autem cujusdam servus male habens, erat moriturus : qui illi erat pretiosus. \EVERSE}
\newcommand{\lcVIIvIII}{\VERSE  Et cum audisset de Jesu, misit ad eum seniores Judæorum, rogans eum ut veniret et salvaret servum ejus. \EVERSE}
\newcommand{\lcVIIvIV}{\VERSE  At illi cum venissent ad Jesum, rogabant eum sollicite, dicentes ei : Quia dignus est ut hoc illi præstes : \EVERSE}
\newcommand{\lcVIIvV}{\VERSE  diligit enim gentem nostram, et synagogam ipse ædificavit nobis. \EVERSE}
\newcommand{\lcVIIvVI}{\VERSE  Jesus autem ibat cum illis. Et cum jam non longe esset a domo, misit ad eum centurio amicos, dicens : Domine, noli vexari : non enim sum dignus ut sub tectum meum intres : \EVERSE}
\newcommand{\lcVIIvVII}{\VERSE  propter quod et meipsum non sum dignum arbitratus ut venirem ad te : sed dic verbo, et sanabitur puer meus. \EVERSE}
\newcommand{\lcVIIvVIII}{\VERSE  Nam et ego homo sum sub potestate constitutus, habens sub me milites : et dico huic, Vade, et vadit : et alii, Veni, et venit : et servo meo, Fac hoc, et facit. \EVERSE}
\newcommand{\lcVIIvIX}{\VERSE  Quo audito Jesus miratus est : et conversus sequentibus se turbis, dixit : Amen dico vobis, nec in Israël tantam fidem inveni. \EVERSE}
\newcommand{\lcVIIvX}{\VERSE  Et reversi, qui missi fuerant, domum, invenerunt servum, qui languerat, sanum. \EVERSE}
\newcommand{\lcVIIvXI}{\VERSE  Et factum est : deinceps ibat in civitatem quæ vocatur Naim : et ibant cum eo discipuli ejus et turba copiosa. \EVERSE}
\newcommand{\lcVIIvXII}{\VERSE  Cum autem appropinquaret portæ civitatis, ecce defunctus efferebatur filius unicus matris suæ : et hæc vidua erat : et turba civitatis multa cum illa. \EVERSE}
\newcommand{\lcVIIvXIII}{\VERSE  Quam cum vidisset Dominus, misericordia motus super eam, dixit illi : Noli flere. \EVERSE}
\newcommand{\lcVIIvXIV}{\VERSE  Et accessit, et tetigit loculum. (Hi autem qui portabant, steterunt.) Et ait : Adolescens, tibi dico, surge. \EVERSE}
\newcommand{\lcVIIvXV}{\VERSE  Et resedit qui erat mortuus, et cœpit loqui. Et dedit illum matri suæ. \EVERSE}
\newcommand{\lcVIIvXVI}{\VERSE  Accepit autem omnes timor : et magnificabant Deum, dicentes : Quia propheta magnus surrexit in nobis : et quia Deus visitavit plebem suam. \EVERSE}
\newcommand{\lcVIIvXVII}{\VERSE  Et exiit hic sermo in universam Judæam de eo, et in omnem circa regionem. \EVERSE}
\newcommand{\lcVIIvXVIII}{\VERSE  Et nuntiaverunt Joanni discipuli ejus de omnibus his. \EVERSE}
\newcommand{\lcVIIvXIX}{\VERSE  Et convocavit duos de discipulis suis Joannes, et misit ad Jesum, dicens : Tu es qui venturus es, an alium exspectamus ? \EVERSE}
\newcommand{\lcVIIvXX}{\VERSE  Cum autem venissent ad eum viri, dixerunt : Joannes Baptista misit nos ad te dicens : Tu es qui venturus es, an alium exspectamus ? \EVERSE}
\newcommand{\lcVIIvXXI}{\VERSE  (In ipsa autem hora multos curavit a languoribus, et plagis, et spiritibus malis, et cæcis multis donavit visum.) \EVERSE}
\newcommand{\lcVIIvXXII}{\VERSE  Et respondens, dixit illis : Euntes renuntiate Joanni quæ audistis et vidistis : quia cæci vident, claudi ambulant, leprosi mundantur, surdi audiunt, mortui resurgunt, pauperes evangelizantur : \EVERSE}
\newcommand{\lcVIIvXXIII}{\VERSE  et beatus est quicumque non fuerit scandalizatus in me. \EVERSE}
\newcommand{\lcVIIvXXIV}{\VERSE  Et cum discessissent nuntii Joannis, cœpit de Joanne dicere ad turbas : Quid existis in desertum videre ? arundinem vento agitatam ? \EVERSE}
\newcommand{\lcVIIvXXV}{\VERSE  Sed quid existis videre ? hominem mollibus vestibus indutum ? Ecce qui in veste pretiosa sunt et deliciis, in domibus regum sunt. \EVERSE}
\newcommand{\lcVIIvXXVI}{\VERSE  Sed quid existis videre ? prophetam ? Utique dico vobis, et plus quam prophetam : \EVERSE}
\newcommand{\lcVIIvXXVII}{\VERSE  hic est, de quo scriptum est : Ecce mitto angelum meum ante faciem tuam, qui præparabit viam tuam ante te. \EVERSE}
\newcommand{\lcVIIvXXVIII}{\VERSE  Dico enim vobis : major inter natos mulierum propheta Joanne Baptista nemo est : qui autem minor est in regno Dei, major est illo. \EVERSE}
\newcommand{\lcVIIvXXIX}{\VERSE  Et omnis populus audiens et publicani, justificaverunt Deum, baptizati baptismo Joannis. \EVERSE}
\newcommand{\lcVIIvXXX}{\VERSE  Pharisæi autem et legisperiti consilium Dei spreverunt in semetipsos, non baptizati ab eo. \EVERSE}
\newcommand{\lcVIIvXXXI}{\VERSE  Ait autem Dominus : Cui ergo similes dicam homines generationis hujus ? et cui similes sunt ? \EVERSE}
\newcommand{\lcVIIvXXXII}{\VERSE  Similes sunt pueris sedentibus in foro, et loquentibus ad invicem, et dicentibus : Cantavimus vobis tibiis, et non saltastis : lamentavimus, et non plorastis. \EVERSE}
\newcommand{\lcVIIvXXXIII}{\VERSE  Venit enim Joannes Baptista, neque manducans panem, neque bibens vinum, et dicitis : Dæmonium habet. \EVERSE}
\newcommand{\lcVIIvXXXIV}{\VERSE  Venit Filius hominis manducans, et bibens, et dicitis : Ecce homo devorator, et bibens vinum, amicus publicanorum et peccatorum. \EVERSE}
\newcommand{\lcVIIvXXXV}{\VERSE  Et justificata est sapientia ab omnibus filiis suis. \EVERSE}
\newcommand{\lcVIIvXXXVI}{\VERSE  Rogabat autem illum quidam de pharisæis ut manducaret cum illo. Et ingressus domum pharisæi discubuit. \EVERSE}
\newcommand{\lcVIIvXXXVII}{\VERSE  Et ecce mulier, quæ erat in civitate peccatrix, ut cognovit quod accubuisset in domo pharisæi, attulit alabastrum unguenti : \EVERSE}
\newcommand{\lcVIIvXXXVIII}{\VERSE  et stans retro secus pedes ejus, lacrimis cœpit rigare pedes ejus, et capillis capitis sui tergebat, et osculabatur pedes ejus, et unguento ungebat. \EVERSE}
\newcommand{\lcVIIvXXXIX}{\VERSE  Videns autem pharisæus, qui vocaverat eum, ait intra se dicens : Hic si esset propheta, sciret utique quæ et qualis est mulier, quæ tangit eum : quia peccatrix est. \EVERSE}
\newcommand{\lcVIIvXL}{\VERSE  Et respondens Jesus, dixit ad illum : Simon, habeo tibi aliquid dicere. At ille ait : Magister, dic. \EVERSE}
\newcommand{\lcVIIvXLI}{\VERSE  Duo debitores erant cuidam fœneratori : unus debebat denarios quingentos, et alius quinquaginta. \EVERSE}
\newcommand{\lcVIIvXLII}{\VERSE  Non habentibus illis unde redderent, donavit utrisque. Quis ergo eum plus diligit ? \EVERSE}
\newcommand{\lcVIIvXLIII}{\VERSE  Respondens Simon dixit : Æstimo quia is cui plus donavit. At ille dixit ei : Recte judicasti. \EVERSE}
\newcommand{\lcVIIvXLIV}{\VERSE  Et conversus ad mulierem, dixit Simoni : Vides hanc mulierem ? Intravi in domum tuam, aquam pedibus meis non dedisti : hæc autem lacrimis rigavit pedes meos, et capillis suis tersit. \EVERSE}
\newcommand{\lcVIIvXLV}{\VERSE  Osculum mihi non dedisti : hæc autem ex quo intravit, non cessavit osculari pedes meos. \EVERSE}
\newcommand{\lcVIIvXLVI}{\VERSE  Oleo caput meum non unxisti : hæc autem unguento unxit pedes meos. \EVERSE}
\newcommand{\lcVIIvXLVII}{\VERSE  Propter quod dico tibi : remittuntur ei peccata multa, quoniam dilexit multum. Cui autem minus dimittitur, minus diligit. \EVERSE}
\newcommand{\lcVIIvXLVIII}{\VERSE  Dixit autem ad illam : Remittuntur tibi peccata. \EVERSE}
\newcommand{\lcVIIvXLIX}{\VERSE  Et cœperunt qui simul accumbebant, dicere intra se : Quis est hic qui etiam peccata dimittit ? \EVERSE}
\newcommand{\lcVIIvL}{\VERSE  Dixit autem ad mulierem : Fides tua te salvam fecit : vade in pace. \EVERSE}
\newcommand{\lcVIIIvI}{\VERSE  Et factum est deinceps, et ipse iter faciebat per civitates, et castella prædicans, et evangelizans regnum Dei : et duodecim cum illo, \EVERSE}
\newcommand{\lcVIIIvII}{\VERSE  et mulieres aliquæ, quæ erant curatæ a spiritibus malignis et infirmatibus : Maria, quæ vocatur Magdalene, de qua septem dæmonia exierant, \EVERSE}
\newcommand{\lcVIIIvIII}{\VERSE  et Joanna uxor Chusæ procuratoris Herodis, et Susanna, et aliæ multæ, quæ ministrabant ei de facultatibus suis. \EVERSE}
\newcommand{\lcVIIIvIV}{\VERSE  Cum autem turba plurima convenirent, et de civitatibus properarent ad eum, dixit per similitudinem : \EVERSE}
\newcommand{\lcVIIIvV}{\VERSE  Exiit qui seminat, seminare semen suum. Et dum seminat, aliud cecidit secus viam, et conculcatum est, et volucres cæli comederunt illud. \EVERSE}
\newcommand{\lcVIIIvVI}{\VERSE  Et aliud cecidit supra petram : et natum aruit, quia non habebat humorem. \EVERSE}
\newcommand{\lcVIIIvVII}{\VERSE  Et aliud cecidit inter spinas, et simul exortæ spinæ suffocaverunt illud. \EVERSE}
\newcommand{\lcVIIIvVIII}{\VERSE  Et aliud cecidit in terram bonam : et ortum fecit fructum centuplum. Hæc dicens clamabat : Qui habet aures audiendi, audiat. \EVERSE}
\newcommand{\lcVIIIvIX}{\VERSE  Interrogabant autem eum discipuli ejus, quæ esset hæc parabola. \EVERSE}
\newcommand{\lcVIIIvX}{\VERSE  Quibus ipse dixit : Vobis datum est nosse mysterium regni Dei, ceteris autem in parabolis : ut videntes non videant, et audientes non intelligant. \EVERSE}
\newcommand{\lcVIIIvXI}{\VERSE  Est autem hæc parabola : Semen est verbum Dei. \EVERSE}
\newcommand{\lcVIIIvXII}{\VERSE  Qui autem secus viam, hi sunt qui audiunt : deinde venit diabolus, et tollit verbum de corde eorum, ne credentes salvi fiant. \EVERSE}
\newcommand{\lcVIIIvXIII}{\VERSE  Nam qui supra petram, qui cum audierint, cum gaudio suscipiunt verbum : et hi radices non habent : qui ad tempus credunt, et in tempore tentationis recedunt. \EVERSE}
\newcommand{\lcVIIIvXIV}{\VERSE  Quod autem in spinas cecidit : hi sunt qui audierunt, et a sollicitudinibus, et divitiis, et voluptatibus vitæ euntes, suffocantur, et non referunt fructum. \EVERSE}
\newcommand{\lcVIIIvXV}{\VERSE  Quod autem in bonam terram : hi sunt qui in corde bono et optimo audientes verbum retinent, et fructum afferunt in patientia. \EVERSE}
\newcommand{\lcVIIIvXVI}{\VERSE  Nemo autem lucernam accendens, operit eam vase, aut subtus lectum ponit : sed supra candelabrum ponit, ut intrantes videant lumen. \EVERSE}
\newcommand{\lcVIIIvXVII}{\VERSE  Non est enim occultum, quod non manifestetur : nec absconditum, quod non cognoscatur, et in palam veniat. \EVERSE}
\newcommand{\lcVIIIvXVIII}{\VERSE  Videte ergo quomodo audiatis. Qui enim habet, dabitur illi : et quicumque non habet, etiam quod putat se habere, auferetur ab illo. \EVERSE}
\newcommand{\lcVIIIvXIX}{\VERSE  Venerunt autem ad illum mater et fratres ejus, et non poterant adire eum præ turba. \EVERSE}
\newcommand{\lcVIIIvXX}{\VERSE  Et nuntiatum est illi : Mater tua et fratres tui stant foris, volentes te videre. \EVERSE}
\newcommand{\lcVIIIvXXI}{\VERSE  Qui respondens, dixit ad eos : Mater mea et fratres mei hi sunt, qui verbum Dei audiunt et faciunt. \EVERSE}
\newcommand{\lcVIIIvXXII}{\VERSE  Factum est autem in una dierum : et ipse ascendit in naviculam, et discipuli ejus, et ait ad illos : Transfretemus trans stagnum. Et ascenderunt. \EVERSE}
\newcommand{\lcVIIIvXXIII}{\VERSE  Et navigantibus illis, obdormivit, et descendit procella venti in stagnum, et complebantur, et periclitabantur. \EVERSE}
\newcommand{\lcVIIIvXXIV}{\VERSE  Accedentes autem suscitaverunt eum, dicentes : Præceptor, perimus. At ille surgens, increpavit ventum, et tempestatem aquæ, et cessavit : et facta est tranquillitas. \EVERSE}
\newcommand{\lcVIIIvXXV}{\VERSE  Dixit autem illis : Ubi est fides vestra ? Qui timentes, mirati sunt ad invicem, dicentes : Quis putas hic est, quia et ventis, et mari imperat, et obediunt ei ? \EVERSE}
\newcommand{\lcVIIIvXXVI}{\VERSE  Et navigaverunt ad regionem Gerasenorum, quæ est contra Galilæam. \EVERSE}
\newcommand{\lcVIIIvXXVII}{\VERSE  Et cum egressus esset ad terram, occurrit illi vir quidam, qui habebat dæmonium jam temporibus multis, et vestimento non induebatur, neque in domo manebat, sed in monumentis. \EVERSE}
\newcommand{\lcVIIIvXXVIII}{\VERSE  Is, ut vidit Jesum, procidit ante illum : et exclamans voce magna, dixit : Quid mihi et tibi est, Jesu Fili Dei Altissimi ? obsecro te, ne me torqueas. \EVERSE}
\newcommand{\lcVIIIvXXIX}{\VERSE  Præcipiebat enim spiritui immundo ut exiret ab homine. Multis enim temporibus arripiebat illum, et vinciebatur catenis, et compedibus custoditus. Et ruptis vinculis agebatur a dæmonio in deserta. \EVERSE}
\newcommand{\lcVIIIvXXX}{\VERSE  Interrogavit autem illum Jesus, dicens : Quod tibi nomen est ? At ille dixit : Legio : quia intraverant dæmonia multa in eum. \EVERSE}
\newcommand{\lcVIIIvXXXI}{\VERSE  Et rogabant illum ne imperaret illis ut in abyssum irent. \EVERSE}
\newcommand{\lcVIIIvXXXII}{\VERSE  Erat autem ibi grex porcorum multorum pascentium in monte : et rogabant eum, ut permitteret eis in illos ingredi. Et permisit illis. \EVERSE}
\newcommand{\lcVIIIvXXXIII}{\VERSE  Exierunt ergo dæmonia ab homine, et intraverunt in porcos : et impetu abiit grex per præceps in stagnum, et suffocatus est. \EVERSE}
\newcommand{\lcVIIIvXXXIV}{\VERSE  Quod ut viderunt factum qui pascebant, fugerunt, et nuntiaverunt in civitatem et in villas. \EVERSE}
\newcommand{\lcVIIIvXXXV}{\VERSE  Exierunt autem videre quod factum est, et venerunt ad Jesum, et invenerunt hominem sedentem, a quo dæmonia exierant, vestitum ac sana mente, ad pedes ejus, et timuerunt. \EVERSE}
\newcommand{\lcVIIIvXXXVI}{\VERSE  Nuntiaverunt autem illis et qui viderant, quomodo sanus factus esset a legione : \EVERSE}
\newcommand{\lcVIIIvXXXVII}{\VERSE  et rogaverunt illum omnis multitudo regionis Gerasenorum ut discederet ab ipsis : quia magno timore tenebantur. Ipse autem ascendens navim, reversus est. \EVERSE}
\newcommand{\lcVIIIvXXXVIII}{\VERSE  Et rogabat illum vir, a quo dæmonia exierant, ut cum eo esset. Dimisit autem eum Jesus, dicens : \EVERSE}
\newcommand{\lcVIIIvXXXIX}{\VERSE  Redi in domum tuam, et narra quanta tibi fecit Deus. Et abiit per universam civitatem, prædicans quanta illi fecisset Jesus. \EVERSE}
\newcommand{\lcVIIIvXL}{\VERSE  Factum est autem cum rediisset Jesus, excepit illum turba : erunt enim omnes exspectantes eum. \EVERSE}
\newcommand{\lcVIIIvXLI}{\VERSE  Et ecce venit vir, cui nomen Jairus, et ipse princeps synagogæ erat : et cecidit ad pedes Jesu, rogans eum ut intraret in domum ejus, \EVERSE}
\newcommand{\lcVIIIvXLII}{\VERSE  quia unica filia erat ei fere annorum duodecim, et hæc moriebatur. Et contigit, dum iret, a turba comprimebatur. \EVERSE}
\newcommand{\lcVIIIvXLIII}{\VERSE  Et mulier quædam erat in fluxu sanguinis ab annis duodecim, quæ in medicos erogaverat omnem substantiam suam, nec ab ullo potuit curari : \EVERSE}
\newcommand{\lcVIIIvXLIV}{\VERSE  accessit retro, et tetigit fimbriam vestimenti ejus : et confestim stetit fluxus sanguinis ejus. \EVERSE}
\newcommand{\lcVIIIvXLV}{\VERSE  Et ait Jesus : Quis est, qui me tetigit ? Negantibus autem omnibus, dixit Petrus, et qui cum illo erant : Præceptor, turbæ te comprimunt, et affligunt, et dicis : Quis me tetigit ? \EVERSE}
\newcommand{\lcVIIIvXLVI}{\VERSE  Et dicit Jesus : Tetigit me aliquis : nam ego novi virtutem de me exiisse. \EVERSE}
\newcommand{\lcVIIIvXLVII}{\VERSE  Videns autem mulier, quia non latuit, tremens venit, et procidit ante pedes ejus : et ob quam causam tetigerit eum, indicavit coram omni populo : et quemadmodum confestim sanata sit. \EVERSE}
\newcommand{\lcVIIIvXLVIII}{\VERSE  At ipse dixit ei : Filia, fides tua salvam te fecit : vade in pace. \EVERSE}
\newcommand{\lcVIIIvXLIX}{\VERSE  Adhuc illo loquente, venit quidam ad principem synagogæ, dicens ei : Quia mortua est filia tua, noli vexare illum. \EVERSE}
\newcommand{\lcVIIIvL}{\VERSE  Jesus autem, audito hoc verbo, respondit patri puellæ : Noli timere, crede tantum, et salva erit. \EVERSE}
\newcommand{\lcVIIIvLI}{\VERSE  Et cum venisset domum, non permisit intrare secum quemquam, nisi Petrum, et Jacobum, et Joannem, et patrem, et matrem puellæ. \EVERSE}
\newcommand{\lcVIIIvLII}{\VERSE  Flebant autem omnes, et plangebant illam. At ille dixit : Nolite flere : non est mortua puella, sed dormit. \EVERSE}
\newcommand{\lcVIIIvLIII}{\VERSE  Et deridebant eum, scientes quod mortua esset. \EVERSE}
\newcommand{\lcVIIIvLIV}{\VERSE  Ipse autem tenens manum ejus clamavit, dicens : Puella, surge. \EVERSE}
\newcommand{\lcVIIIvLV}{\VERSE  Et reversus est spiritus ejus, et surrexit continuo. Et jussit illi dari manducare. \EVERSE}
\newcommand{\lcVIIIvLVI}{\VERSE  Et stupuerunt parentes ejus, quibus præcepit ne alicui dicerent quod factum erat. \EVERSE}
\newcommand{\lcIXvI}{\VERSE  Convocatis autem duodecim Apostolis, dedit illis virtutem et potestatem super omnia dæmonia, et ut languores curarent. \EVERSE}
\newcommand{\lcIXvII}{\VERSE  Et misit illos prædicare regnum Dei, et sanare infirmos. \EVERSE}
\newcommand{\lcIXvIII}{\VERSE  Et ait ad illos : Nihil tuleritis in via, neque virgam, neque peram, neque panem, neque pecuniam, neque duas tunicas habeatis. \EVERSE}
\newcommand{\lcIXvIV}{\VERSE  Et in quamcumque domum intraveritis, ibi manete, et inde ne exeatis. \EVERSE}
\newcommand{\lcIXvV}{\VERSE  Et quicumque non receperint vos : exeuntes de civitate illa, etiam pulverem pedum vestrorum excutite in testimonium supra illos. \EVERSE}
\newcommand{\lcIXvVI}{\VERSE  Egressi autem circuibant per castella evangelizantes, et curantes ubique. \EVERSE}
\newcommand{\lcIXvVII}{\VERSE  Audivit autem Herodes tetrarcha omnia quæ fiebant ab eo, et hæsitabat eo quod diceretur \EVERSE}
\newcommand{\lcIXvVIII}{\VERSE  a quibusdam : Quia Joannes surrexit a mortuis : a quibusdam vero : Quia Elias apparuit : ab aliis autem : Quia propheta unus de antiquis surrexit. \EVERSE}
\newcommand{\lcIXvIX}{\VERSE  Et ait Herodes : Joannem ego decollavi : quis est autem iste, de quo ego talia audio ? Et quærebat videre eum. \EVERSE}
\newcommand{\lcIXvX}{\VERSE  Et reversi Apostoli, narraverunt illi quæcumque fecerunt : et assumptis illis secessit seorsum in locum desertum, qui est Bethsaidæ. \EVERSE}
\newcommand{\lcIXvXI}{\VERSE  Quod cum cognovissent turbæ, secutæ sunt illum : et excepit eos, et loquebatur illis de regno Dei, et eos, qui cura indigebant, sanabat. \EVERSE}
\newcommand{\lcIXvXII}{\VERSE  Dies autem cœperat declinare, et accedentes duodecim dixerunt illi : Dimitte turbas, ut euntes in castella villasque quæ circa sunt, divertant, et inveniant escas : quia hic in loco deserto sumus. \EVERSE}
\newcommand{\lcIXvXIII}{\VERSE  Ait autem ad illos : Vos date illis manducare. At illi dixerunt : Non sunt nobis plus quam quinque panes et duo pisces : nisi forte nos eamus, et emamus in omnem hanc turbam escas. \EVERSE}
\newcommand{\lcIXvXIV}{\VERSE  Erant autem fere viri quinque millia. Ait autem ad discipulos suos : Facite illos discumbere per convivia quinquagenos. \EVERSE}
\newcommand{\lcIXvXV}{\VERSE  Et ita fecerunt : et discumbere fecerunt omnes. \EVERSE}
\newcommand{\lcIXvXVI}{\VERSE  Acceptis autem quinque panibus et duobus piscibus, respexit in cælum, et benedixit illis : et fregit, et distribuit discipulis suis, ut ponerent ante turbas. \EVERSE}
\newcommand{\lcIXvXVII}{\VERSE  Et manducaverunt omnes, et saturati sunt. Et sublatum est quod superfuit illis, fragmentorum cophini duodecim. \EVERSE}
\newcommand{\lcIXvXVIII}{\VERSE  Et factum est cum solus esset orans, erant cum illo et discipuli : et interrogavit illos, dicens : Quem me dicunt esse turbæ ? \EVERSE}
\newcommand{\lcIXvXIX}{\VERSE  At illi responderunt, et dixerunt : Joannem Baptistam, alii autem Eliam, alii vero quia unus propheta de prioribus surrexit. \EVERSE}
\newcommand{\lcIXvXX}{\VERSE  Dixit autem illis : Vos autem quem me esse dicitis ? Respondens Simon Petrus, dixit : Christum Dei. \EVERSE}
\newcommand{\lcIXvXXI}{\VERSE  At ille increpans illos, præcepit ne cui dicerent hoc, \EVERSE}
\newcommand{\lcIXvXXII}{\VERSE  dicens : Quia oportet Filium hominis multa pati, et reprobari a senioribus, et principibus sacerdotum, et scribis, et occidi, et tertia die resurgere. \EVERSE}
\newcommand{\lcIXvXXIII}{\VERSE  Dicebat autem ad omnes : Si quis vult post me venire, abneget semetipsum, et tollat crucem suam quotidie, et sequatur me. \EVERSE}
\newcommand{\lcIXvXXIV}{\VERSE  Qui enim voluerit animam suam salvam facere, perdet illam : nam qui perdiderit animam suam propter me, salvam faciet illam. \EVERSE}
\newcommand{\lcIXvXXV}{\VERSE  Quid enim proficit homo, si lucretur universum mundum, se autem ipsum perdat, et detrimentum sui faciat ? \EVERSE}
\newcommand{\lcIXvXXVI}{\VERSE  Nam qui me erubuerit, et meos sermones : hunc Filius hominis erubescet cum venerit in majestate sua, et Patris, et sanctorum angelorum. \EVERSE}
\newcommand{\lcIXvXXVII}{\VERSE  Dico autem vobis vere : sunt aliqui hic stantes, qui non gustabunt mortem donec videant regnum Dei. \EVERSE}
\newcommand{\lcIXvXXVIII}{\VERSE  Factum est autem post hæc verba fere dies octo, et assumpsit Petrum, et Jacobum, et Joannem, et ascendit in montem ut oraret. \EVERSE}
\newcommand{\lcIXvXXIX}{\VERSE  Et facta est, dum oraret, species vultus ejus altera : et vestitus ejus albus et refulgens. \EVERSE}
\newcommand{\lcIXvXXX}{\VERSE  Et ecce duo viri loquebantur cum illo. Erant autem Moyses et Elias, \EVERSE}
\newcommand{\lcIXvXXXI}{\VERSE  visi in majestate : et dicebant excessum ejus, quem completurus erat in Jerusalem. \EVERSE}
\newcommand{\lcIXvXXXII}{\VERSE  Petrus vero, et qui cum illo erant, gravati erant somno. Et evigilantes viderunt majestatem ejus, et duos viros qui stabant cum illo. \EVERSE}
\newcommand{\lcIXvXXXIII}{\VERSE  Et factum est cum discederent ab illo, ait Petrus ad Jesum : Præceptor, bonum est nos hic esse : et faciamus tria tabernacula, unum tibi, et unum Moysi, et unum Eliæ : nesciens quid diceret. \EVERSE}
\newcommand{\lcIXvXXXIV}{\VERSE  Hæc autem illo loquente, facta est nubes, et obumbravit eos : et timuerunt, intrantibus illis in nubem. \EVERSE}
\newcommand{\lcIXvXXXV}{\VERSE  Et vox facta est de nube, dicens : Hic est Filius meus dilectus, ipsum audite. \EVERSE}
\newcommand{\lcIXvXXXVI}{\VERSE  Et dum fieret vox, inventus est Jesus solus. Et ipsi tacuerunt, et nemini dixerunt in illis diebus quidquam ex his quæ viderant. \EVERSE}
\newcommand{\lcIXvXXXVII}{\VERSE  Factum est autem in sequenti die, descendentibus illis de monte, occurrit illis turba multa. \EVERSE}
\newcommand{\lcIXvXXXVIII}{\VERSE  Et ecce vir de turba exclamavit, dicens : Magister, obsecro te, respice in filium meum quia unicus est mihi : \EVERSE}
\newcommand{\lcIXvXXXIX}{\VERSE  et ecce spiritus apprehendit eum, et subito clamat, et elidit, et dissipat eum cum spuma, et vix discedit dilanians eum : \EVERSE}
\newcommand{\lcIXvXL}{\VERSE  et rogavi discipulos tuos ut ejicerent illum, et non potuerunt. \EVERSE}
\newcommand{\lcIXvXLI}{\VERSE  Respondens autem Jesus, dixit : O generatio infidelis, et perversa, usquequo ero apud vos, et patiar vos ? adduc huc filium tuum. \EVERSE}
\newcommand{\lcIXvXLII}{\VERSE  Et cum accederet, elisit illum dæmonium, et dissipavit. \EVERSE}
\newcommand{\lcIXvXLIII}{\VERSE  Et increpavit Jesus spiritum immundum, et sanavit puerum, et reddidit illum patri ejus. \EVERSE}
\newcommand{\lcIXvXLIV}{\VERSE  Stupebant autem omnes in magnitudine Dei : omnibusque mirantibus in omnibus quæ faciebat, dixit ad discipulos suos : Ponite vos in cordibus vestris sermones istos : Filius enim hominis futurum est ut tradatur in manus hominum. \EVERSE}
\newcommand{\lcIXvXLV}{\VERSE  At illi ignorabant verbum istud, et erat velatum ante eos ut non sentirent illud : et timebant eum interrogare de hoc verbo. \EVERSE}
\newcommand{\lcIXvXLVI}{\VERSE  Intravit autem cogitatio in eos quis eorum major esset. \EVERSE}
\newcommand{\lcIXvXLVII}{\VERSE  At Jesus videns cogitationes cordis illorum, apprehendit puerum, et statuit illum secus se, \EVERSE}
\newcommand{\lcIXvXLVIII}{\VERSE  et ait illis : Quicumque susceperit puerum istum in nomine meo, me recipit : et quicumque me receperit, recipit eum qui me misit. Nam qui minor est inter vos omnes, hic major est. \EVERSE}
\newcommand{\lcIXvXLIX}{\VERSE  Respondens autem Joannes dixit : Præceptor, vidimus quemdam in nomine tuo ejicientem dæmonia, et prohibuimus eum : quia non sequitur nobiscum. \EVERSE}
\newcommand{\lcIXvL}{\VERSE  Et ait ad illum Jesus : Nolite prohibere : qui enim non est adversum vos, pro vobis est. \EVERSE}
\newcommand{\lcIXvLI}{\VERSE  Factum est autem dum complerentur dies assumptionis ejus, et ipse faciem suam firmavit ut iret in Jerusalem. \EVERSE}
\newcommand{\lcIXvLII}{\VERSE  Et misit nuntios ante conspectum suum : et euntes intraverunt in civitatem Samaritanorum ut parerent illi. \EVERSE}
\newcommand{\lcIXvLIII}{\VERSE  Et non receperunt eum, quia facies ejus erat euntis in Jerusalem. \EVERSE}
\newcommand{\lcIXvLIV}{\VERSE  Cum vidissent autem discipuli ejus Jacobus et Joannes, dixerunt : Domine, vis dicimus ut ignis descendat de cælo, et consumat illos ? \EVERSE}
\newcommand{\lcIXvLV}{\VERSE  Et conversus increpavit illos, dicens : Nescitis cujus spiritus estis. \EVERSE}
\newcommand{\lcIXvLVI}{\VERSE  Filius hominis non venit animas perdere, sed salvare. Et abierunt in aliud castellum. \EVERSE}
\newcommand{\lcIXvLVII}{\VERSE  Factum est autem : ambulantibus illis in via, dixit quidam ad illum : Sequar te quocumque ieris. \EVERSE}
\newcommand{\lcIXvLVIII}{\VERSE  Dixit illi Jesus : Vulpes foveas habent, et volucres cæli nidos : Filius autem hominis non habet ubi caput reclinet. \EVERSE}
\newcommand{\lcIXvLIX}{\VERSE  Ait autem ad alterum : Sequere me : ille autem dixit : Domine, permitte mihi primum ire, et sepelire patrem meum. \EVERSE}
\newcommand{\lcIXvLX}{\VERSE  Dixitque ei Jesus : Sine ut mortui sepeliant mortuos suos : tu autem vade, et annuntia regnum Dei. \EVERSE}
\newcommand{\lcIXvLXI}{\VERSE  Et ait alter : Sequar te Domine, sed permitte mihi primum renuntiare his quæ domi sunt. \EVERSE}
\newcommand{\lcIXvLXII}{\VERSE  Ait ad illum Jesus : Nemo mittens manum suam ad aratrum, et respiciens retro, aptus est regno Dei. \EVERSE}
\newcommand{\lcXvI}{\VERSE  Post hæc autem designavit Dominus et alios septuaginta duos : et misit illos binos ante faciem suam in omnem civitatem et locum, quo erat ipse venturus. \EVERSE}
\newcommand{\lcXvII}{\VERSE  Et dicebat illis : Messis quidem multa, operarii autem pauci. Rogate ergo dominum messis ut mittat operarios in messem suam. \EVERSE}
\newcommand{\lcXvIII}{\VERSE  Ite : ecce ego mitto vos sicut agnos inter lupos. \EVERSE}
\newcommand{\lcXvIV}{\VERSE  Nolite portare sacculum, neque peram, neque calceamenta, et neminem per viam salutaveritis. \EVERSE}
\newcommand{\lcXvV}{\VERSE  In quamcumque domum intraveritis, primum dicite : Pax huic domui : \EVERSE}
\newcommand{\lcXvVI}{\VERSE  et si ibi fuerit filius pacis, requiescet super illum pax vestra : sin autem, ad vos revertetur. \EVERSE}
\newcommand{\lcXvVII}{\VERSE  In eadem autem domo manete, edentes et bibentes quæ apud illos sunt : dignus est enim operarius mercede sua. Nolite transire de domo in domum. \EVERSE}
\newcommand{\lcXvVIII}{\VERSE  Et in quamcumque civitatem intraveritis, et susceperint vos, manducate quæ apponuntur vobis : \EVERSE}
\newcommand{\lcXvIX}{\VERSE  et curate infirmos, qui in illa sunt, et dicite illis : Appropinquavit in vos regnum Dei. \EVERSE}
\newcommand{\lcXvX}{\VERSE  In quamcumque autem civitatem intraveritis, et non susceperint vos, exeuntes in plateas ejus, dicite : \EVERSE}
\newcommand{\lcXvXI}{\VERSE  Etiam pulverem, qui adhæsit nobis de civitate vestra, extergimus in vos : tamen hoc scitote, quia appropinquavit regnum Dei. \EVERSE}
\newcommand{\lcXvXII}{\VERSE  Dico vobis, quia Sodomis in die illa remissius erit, quam illi civitati. \EVERSE}
\newcommand{\lcXvXIII}{\VERSE  Væ tibi Corozain ! væ tibi Bethsaida ! quia si in Tyro et Sidone factæ fuissent virtutes quæ factæ sunt in vobis, olim in cilicio et cinere sedentes pœniterent. \EVERSE}
\newcommand{\lcXvXIV}{\VERSE  Verumtamen Tyro et Sidoni remissius erit in judicio, quam vobis. \EVERSE}
\newcommand{\lcXvXV}{\VERSE  Et tu Capharnaum, usque ad cælum exaltata, usque ad infernum demergeris. \EVERSE}
\newcommand{\lcXvXVI}{\VERSE  Qui vos audit, me audit : et qui vos spernit, me spernit. Qui autem me spernit, spernit eum qui misit me. \EVERSE}
\newcommand{\lcXvXVII}{\VERSE  Reversi sunt autem septuaginta duo cum gaudio, dicentes : Domine, etiam dæmonia subjiciuntur nobis in nomine tuo. \EVERSE}
\newcommand{\lcXvXVIII}{\VERSE  Et ait illis : Videbam Satanam sicut fulgor de cælo cadentem. \EVERSE}
\newcommand{\lcXvXIX}{\VERSE  Ecce dedi vobis potestatem calcandi supra serpentes, et scorpiones, et super omnem virtutem inimici : et nihil vobis nocebit. \EVERSE}
\newcommand{\lcXvXX}{\VERSE  Verumtamen in hoc nolite gaudere quia spiritus vobis subjiciuntur : gaudete autem, quod nomina vestra scripta sunt in cælis. \EVERSE}
\newcommand{\lcXvXXI}{\VERSE  In ipsa hora exsultavit Spiritu Sancto, et dixit : Confiteor tibi Pater, Domine cæli et terræ, quod abscondisti hæc a sapientibus et prudentibus, et revelasti ea parvulis. Etiam Pater : quoniam sic placuit ante te. \EVERSE}
\newcommand{\lcXvXXII}{\VERSE  Omnia mihi tradita sunt a Patre meo. Et nemo scit quis sit Filius, nisi Pater : et quis sit Pater, nisi Filius, et cui voluerit Filius revelare. \EVERSE}
\newcommand{\lcXvXXIII}{\VERSE  Et conversus ad discipulos suos, dixit : Beati oculi qui vident quæ vos videtis. \EVERSE}
\newcommand{\lcXvXXIV}{\VERSE  Dico enim vobis quod multi prophetæ et reges voluerunt videre quæ vos videtis, et non viderunt : et audire quæ auditis, et non audierunt. \EVERSE}
\newcommand{\lcXvXXV}{\VERSE  Et ecce quidam legisperitus surrexit tentans illum, et dicens : Magister, quid faciendo vitam æternam possidebo ? \EVERSE}
\newcommand{\lcXvXXVI}{\VERSE  At ille dixit ad eum : In lege quid scriptum est ? quomodo legis ? \EVERSE}
\newcommand{\lcXvXXVII}{\VERSE  Ille respondens dixit : Diliges Dominum Deum tuum ex toto corde tuo, et ex tota anima tua, et ex omnibus virtutibus tuis, et ex omni mente tua : et proximum tuum sicut teipsum. \EVERSE}
\newcommand{\lcXvXXVIII}{\VERSE  Dixitque illi : Recte respondisti : hoc fac, et vives. \EVERSE}
\newcommand{\lcXvXXIX}{\VERSE  Ille autem volens justificare seipsum, dixit ad Jesum : Et quis est meus proximus ? \EVERSE}
\newcommand{\lcXvXXX}{\VERSE  Suscipiens autem Jesus, dixit : Homo quidam descendebat ab Jerusalem in Jericho, et incidit in latrones, qui etiam despoliaverunt eum : et plagis impositis abierunt semivivo relicto. \EVERSE}
\newcommand{\lcXvXXXI}{\VERSE  Accidit autem ut sacerdos quidam descenderet eadem via : et viso illo præterivit. \EVERSE}
\newcommand{\lcXvXXXII}{\VERSE  Similiter et Levita, cum esset secus locum, et videret eum, pertransiit. \EVERSE}
\newcommand{\lcXvXXXIII}{\VERSE  Samaritanus autem quidam iter faciens, venit secus eum : et videns eum, misericordia motus est. \EVERSE}
\newcommand{\lcXvXXXIV}{\VERSE  Et appropians alligavit vulnera ejus, infundens oleum et vinum : et imponens illum in jumentum suum, duxit in stabulum, et curam ejus egit. \EVERSE}
\newcommand{\lcXvXXXV}{\VERSE  Et altera die protulit duos denarios, et dedit stabulario, et ait : Curam illius habe : et quodcumque supererogaveris, ego cum rediero reddam tibi. \EVERSE}
\newcommand{\lcXvXXXVI}{\VERSE  Quis horum trium videtur tibi proximus fuisse illi, qui incidit in latrones ? \EVERSE}
\newcommand{\lcXvXXXVII}{\VERSE  At ille dixit : Qui fecit misericordiam in illum. Et ait illi Jesus : Vade, et tu fac similiter. \EVERSE}
\newcommand{\lcXvXXXVIII}{\VERSE  Factum est autem, dum irent, et ipse intravit in quoddam castellum : et mulier quædam, Martha nomine, excepit illum in domum suam, \EVERSE}
\newcommand{\lcXvXXXIX}{\VERSE  et huic erat soror nomine Maria, quæ etiam sedens secus pedes Domini, audiebat verbum illius. \EVERSE}
\newcommand{\lcXvXL}{\VERSE  Martha autem satagebat circa frequens ministerium : quæ stetit, et ait : Domine, non est tibi curæ quod soror mea reliquit me solam ministrare ? dic ergo illi ut me adjuvet. \EVERSE}
\newcommand{\lcXvXLI}{\VERSE  Et respondens dixit illi Dominus : Martha, Martha, sollicita es, et turbaris erga plurima, \EVERSE}
\newcommand{\lcXvXLII}{\VERSE  porro unum est necessarium. Maria optimam partem elegit, quæ non auferetur ab ea. \EVERSE}
\newcommand{\lcXIvI}{\VERSE  Et factum est : cum esset in quodam loco orans, ut cessavit, dixit unus ex discipulis ejus ad eum : Domine, doce nos orare, sicut docuit et Joannes discipulos suos. \EVERSE}
\newcommand{\lcXIvII}{\VERSE  Et ait illis : Cum oratis, dicite : Pater, sanctificetur nomen tuum. Adveniat regnum tuum. \EVERSE}
\newcommand{\lcXIvIII}{\VERSE  Panem nostrum quotidianum da nobis hodie. \EVERSE}
\newcommand{\lcXIvIV}{\VERSE  Et dimitte nobis peccata nostra, siquidem et ipsi dimittimus omni debenti nobis. Et ne nos inducas in tentationem. \EVERSE}
\newcommand{\lcXIvV}{\VERSE  Et ait ad illos : Quis vestrum habebit amicum, et ibit ad illum media nocte, et dicet illi : Amice, commoda mihi tres panes, \EVERSE}
\newcommand{\lcXIvVI}{\VERSE  quoniam amicus meus venit de via ad me, et non habeo quod ponam ante illum, \EVERSE}
\newcommand{\lcXIvVII}{\VERSE  et ille de intus respondens dicat : Noli mihi molestus esse, jam ostium clausum est, et pueri mei mecum sunt in cubili : non possum surgere, et dare tibi. \EVERSE}
\newcommand{\lcXIvVIII}{\VERSE  Et si ille perseveraverit pulsans : dico vobis, etsi non dabit illi surgens eo quod amicus ejus sit, propter improbitatem tamen ejus surget, et dabit illi quotquot habet necessarios. \EVERSE}
\newcommand{\lcXIvIX}{\VERSE  Et ego dico vobis : Petite, et dabitur vobis ; quærite, et invenietis ; pulsate, et aperietur vobis. \EVERSE}
\newcommand{\lcXIvX}{\VERSE  Omnis enim qui petit, accipit : et qui quærit, invenit : et pulsanti aperietur. \EVERSE}
\newcommand{\lcXIvXI}{\VERSE  Quis autem ex vobis patrem petit panem, numquid lapidem dabit illi ? aut piscem, numquid pro pisce serpentem dabit illi ? \EVERSE}
\newcommand{\lcXIvXII}{\VERSE  aut si petierit ovum, numquid porriget illi scorpionem ? \EVERSE}
\newcommand{\lcXIvXIII}{\VERSE  Si ergo vos, cum sitis mali, nostis bona data dare filiis vestris : quanto magis Pater vester de cælo dabit spiritum bonum petentibus se ? \EVERSE}
\newcommand{\lcXIvXIV}{\VERSE  Et erat ejiciens dæmonium, et illud erat mutum. Et cum ejecisset dæmonium, locutus est mutus, et admiratæ sunt turbæ. \EVERSE}
\newcommand{\lcXIvXV}{\VERSE  Quidam autem ex eis dixerunt : In Beelzebub principe dæmoniorum ejicit dæmonia. \EVERSE}
\newcommand{\lcXIvXVI}{\VERSE  Et alii tentantes, signum de cælo quærebant ab eo. \EVERSE}
\newcommand{\lcXIvXVII}{\VERSE  Ipse autem ut vidit cogitationes eorum, dixit eis : Omne regnum in seipsum divisum desolabitur, et domus supra domum cadet. \EVERSE}
\newcommand{\lcXIvXVIII}{\VERSE  Si autem et Satanas in seipsum divisus est, quomodo stabit regnum ejus ? quia dicitis in Beelzebub me ejicere dæmonia. \EVERSE}
\newcommand{\lcXIvXIX}{\VERSE  Si autem ego in Beelzebub ejicio dæmonia : filii vestri in quo ejiciunt ? ideo ipsi judices vestri erunt. \EVERSE}
\newcommand{\lcXIvXX}{\VERSE  Porro si in digito Dei ejicio dæmonia : profecto pervenit in vos regnum Dei. \EVERSE}
\newcommand{\lcXIvXXI}{\VERSE  Cum fortis armatus custodit atrium suum, in pace sunt ea quæ possidet. \EVERSE}
\newcommand{\lcXIvXXII}{\VERSE  Si autem fortior eo superveniens vicerit eum, universa arma ejus auferet, in quibus confidebat, et spolia ejus distribuet. \EVERSE}
\newcommand{\lcXIvXXIII}{\VERSE  Qui non est mecum, contra me est : et qui non colligit mecum, dispergit. \EVERSE}
\newcommand{\lcXIvXXIV}{\VERSE  Cum immundus spiritus exierit de homine, ambulat per loca inaquosa, quærens requiem : et non inveniens dicit : Revertar in domum meam unde exivi. \EVERSE}
\newcommand{\lcXIvXXV}{\VERSE  Et cum venerit, invenit eam scopis mundatam, et ornatam. \EVERSE}
\newcommand{\lcXIvXXVI}{\VERSE  Tunc vadit, et assumit septem alios spiritus secum, nequiores se, et ingressi habitant ibi. Et fiunt novissima hominis illius pejora prioribus. \EVERSE}
\newcommand{\lcXIvXXVII}{\VERSE  Factum est autem, cum hæc diceret : extollens vocem quædam mulier de turba dixit illi : Beatus venter qui te portavit, et ubera quæ suxisti. \EVERSE}
\newcommand{\lcXIvXXVIII}{\VERSE  At ille dixit : Quinimmo beati, qui audiunt verbum Dei et custodiunt illud. \EVERSE}
\newcommand{\lcXIvXXIX}{\VERSE  Turbis autem concurrentibus cœpit dicere : Generatio hæc, generatio nequam est : signum quærit, et signum non dabitur ei, nisi signum Jonæ prophetæ. \EVERSE}
\newcommand{\lcXIvXXX}{\VERSE  Nam sicut fuit Jonas signum Ninivitis, ita erit et Filius hominis generationi isti. \EVERSE}
\newcommand{\lcXIvXXXI}{\VERSE  Regina austri surget in judicio cum viris generationis hujus, et condemnabit illos : quia venit a finibus terræ audire sapientiam Salomonis : et ecce plus quam Salomon hic. \EVERSE}
\newcommand{\lcXIvXXXII}{\VERSE  Viri Ninivitæ surgent in judicio cum generatione hac, et condemnabunt illam : quia pœnitentiam egerunt ad prædicationem Jonæ, et ecce plus quam Jonas hic. \EVERSE}
\newcommand{\lcXIvXXXIII}{\VERSE  Nemo lucernam accendit, et in abscondito ponit, neque sub modio : sed supra candelabrum, ut qui ingrediuntur, lumen videant. \EVERSE}
\newcommand{\lcXIvXXXIV}{\VERSE  Lucerna corporis tui est oculus tuus. Si oculus tuus fuerit simplex, totum corpus tuum lucidum erit : si autem nequam fuerit, etiam corpus tuum tenebrosum erit. \EVERSE}
\newcommand{\lcXIvXXXV}{\VERSE  Vide ergo ne lumen quod in te est, tenebræ sint. \EVERSE}
\newcommand{\lcXIvXXXVI}{\VERSE  Si ergo corpus tuum totum lucidum fuerit, non habens aliquam partem tenebrarum, erit lucidum totum, et sicut lucerna fulgoris illuminabit te. \EVERSE}
\newcommand{\lcXIvXXXVII}{\VERSE  Et cum loqueretur, rogavit illum quidam pharisæus ut pranderet apud se. Et ingressus recubuit. \EVERSE}
\newcommand{\lcXIvXXXVIII}{\VERSE  Pharisæus autem cœpit intra se reputans dicere, quare non baptizatus esset ante prandium. \EVERSE}
\newcommand{\lcXIvXXXIX}{\VERSE  Et ait Dominus ad illum : Nunc vos pharisæi, quod deforis est calicis et catini, mundatis : quod autem intus est vestrum, plenum est rapina et iniquitate. \EVERSE}
\newcommand{\lcXIvXL}{\VERSE  Stulti ! nonne qui fecit quod deforis est, etiam id quod deintus est fecit ? \EVERSE}
\newcommand{\lcXIvXLI}{\VERSE  Verumtamen quod superest, date eleemosynam : et ecce omnia munda sunt vobis. \EVERSE}
\newcommand{\lcXIvXLII}{\VERSE  Sed væ vobis, pharisæis, quia decimatis mentham, et rutam, et omne olus, et præteritis judicium et caritatem Dei : hæc autem oportuit facere, et illa non omittere. \EVERSE}
\newcommand{\lcXIvXLIII}{\VERSE  Væ vobis, pharisæis, quia diligitis primas cathedras in synagogis, et salutationes in foro. \EVERSE}
\newcommand{\lcXIvXLIV}{\VERSE  Væ vobis, quia estis ut monumenta, quæ non apparent, et homines ambulantes supra, nesciunt. \EVERSE}
\newcommand{\lcXIvXLV}{\VERSE  Respondens autem quidam ex legisperitis, ait illi : Magister, hæc dicens etiam contumeliam nobis facis. \EVERSE}
\newcommand{\lcXIvXLVI}{\VERSE  At ille ait : Et vobis legisperitis væ : quia oneratis homines oneribus, quæ portare non possunt, et ipsi uno digito vestro non tangitis sarcinas. \EVERSE}
\newcommand{\lcXIvXLVII}{\VERSE  Væ vobis, qui ædificatis monumenta prophetarum : patres autem vestri occiderunt illos. \EVERSE}
\newcommand{\lcXIvXLVIII}{\VERSE  Profecto testificamini quod consentitis operibus patrum vestrorum : quoniam ipsi quidem eos occiderunt, vos autem ædificatis eorum sepulchra. \EVERSE}
\newcommand{\lcXIvXLIX}{\VERSE  Propterea et sapientia Dei dixit : Mittam ad illos prophetas, et apostolos, et ex illis occident, et persequentur : \EVERSE}
\newcommand{\lcXIvL}{\VERSE  ut inquiratur sanguis omnium prophetarum, qui effusus est a constitutione mundi a generatione ista, \EVERSE}
\newcommand{\lcXIvLI}{\VERSE  a sanguine Abel, usque ad sanguinem Zachariæ, qui periit inter altare et ædem. Ita dico vobis, requiretur ab hac generatione. \EVERSE}
\newcommand{\lcXIvLII}{\VERSE  Væ vobis, legisperitis, quia tulistis clavem scientiæ : ipsi non introistis, et eos qui introibant, prohibuistis. \EVERSE}
\newcommand{\lcXIvLIII}{\VERSE  Cum autem hæc ad illos diceret, cœperunt pharisæi et legisperiti graviter insistere, et os ejus opprimere de multis, \EVERSE}
\newcommand{\lcXIvLIV}{\VERSE  insidiantes ei, et quærentes aliquid capere de ore ejus, ut accusarent eum. \EVERSE}
\newcommand{\lcXIIvI}{\VERSE  Multis autem turbis circumstantibus, ita ut se invicem conculcarent, cœpit dicere ad discipulos suos : Attendite a fermento pharisæorum, quod est hypocrisis. \EVERSE}
\newcommand{\lcXIIvII}{\VERSE  Nihil autem opertum est, quod non reveletur : neque absconditum, quod non sciatur. \EVERSE}
\newcommand{\lcXIIvIII}{\VERSE  Quoniam quæ in tenebris dixistis, in lumine dicentur : et quod in aurem locuti estis in cubiculis, prædicabitur in tectis. \EVERSE}
\newcommand{\lcXIIvIV}{\VERSE  Dico autem vobis amicis meis : Ne terreamini ab his qui occidunt corpus, et post hæc non habent amplius quid faciant. \EVERSE}
\newcommand{\lcXIIvV}{\VERSE  Ostendam autem vobis quem timeatis : timete eum qui, postquam occiderit, habet potestatem mittere in gehennam : ita dico vobis, hunc timete. \EVERSE}
\newcommand{\lcXIIvVI}{\VERSE  Nonne quinque passeres veneunt dipondio, et unus ex illis non est in oblivione coram Deo ? \EVERSE}
\newcommand{\lcXIIvVII}{\VERSE  sed et capilli capitis vestri omnes numerati sunt. Nolite ergo timere : multis passeribus pluris estis vos. \EVERSE}
\newcommand{\lcXIIvVIII}{\VERSE  Dico autem vobis : Omnis quicumque confessus fuerit me coram hominibus, et Filius hominis confitebitur illum coram angelis Dei : \EVERSE}
\newcommand{\lcXIIvIX}{\VERSE  qui autem negaverit me coram hominibus, negabitur coram angelis Dei. \EVERSE}
\newcommand{\lcXIIvX}{\VERSE  Et omnis qui dicit verbum in Filium hominis, remittetur illi : ei autem qui in Spiritum Sanctum blasphemaverit, non remittetur. \EVERSE}
\newcommand{\lcXIIvXI}{\VERSE  Cum autem inducent vos in synagogas, et ad magistratus, et potestates, nolite solliciti esse qualiter, aut quid respondeatis, aut quid dicatis. \EVERSE}
\newcommand{\lcXIIvXII}{\VERSE  Spiritus enim Sanctus docebit vos in ipsa hora quid oporteat vos dicere. \EVERSE}
\newcommand{\lcXIIvXIII}{\VERSE  Ait autem ei quidam de turba : Magister, dic fratri meo ut dividat mecum hæreditatem. \EVERSE}
\newcommand{\lcXIIvXIV}{\VERSE  At ille dixit illi : Homo, quis me constituit judicem, aut divisorem super vos ? \EVERSE}
\newcommand{\lcXIIvXV}{\VERSE  Dixitque ad illos : Videte, et cavete ab omni avaritia : quia non in abundantia cujusquam vita ejus est ex his quæ possidet. \EVERSE}
\newcommand{\lcXIIvXVI}{\VERSE  Dixit autem similitudinem ad illos, dicens : Hominis cujusdam divitis uberes fructus ager attulit : \EVERSE}
\newcommand{\lcXIIvXVII}{\VERSE  et cogitabat intra se dicens : Quid faciam, quia non habeo quo congregem fructus meos ? \EVERSE}
\newcommand{\lcXIIvXVIII}{\VERSE  Et dixit : Hoc faciam : destruam horrea mea, et majora faciam : et illuc congregabo omnia quæ nata sunt mihi, et bona mea, \EVERSE}
\newcommand{\lcXIIvXIX}{\VERSE  et dicam animæ meæ : Anima, habes multa bona posita in annos plurimos : requiesce, comede, bibe, epulare. \EVERSE}
\newcommand{\lcXIIvXX}{\VERSE  Dixit autem illi Deus : Stulte, hac nocte animam tuam repetunt a te : quæ autem parasti, cujus erunt ? \EVERSE}
\newcommand{\lcXIIvXXI}{\VERSE  Sic est qui sibi thesaurizat, et non est in Deum dives. \EVERSE}
\newcommand{\lcXIIvXXII}{\VERSE  Dixitque ad discipulos suos : Ideo dico vobis, nolite solliciti esse animæ vestræ quid manducetis, neque corpori quid induamini. \EVERSE}
\newcommand{\lcXIIvXXIII}{\VERSE  Anima plus est quam esca, et corpus plus quam vestimentum. \EVERSE}
\newcommand{\lcXIIvXXIV}{\VERSE  Considerate corvos, quia non seminant, neque metunt, quibus non est cellarium, neque horreum, et Deus pascit illos. Quanto magis vos pluris estis illis ? \EVERSE}
\newcommand{\lcXIIvXXV}{\VERSE  Quis autem vestrum cogitando potest adjicere ad staturam suam cubitum unum ? \EVERSE}
\newcommand{\lcXIIvXXVI}{\VERSE  Si ergo neque quod minimum est potestis, quid de ceteris solliciti estis ? \EVERSE}
\newcommand{\lcXIIvXXVII}{\VERSE  Considerate lilia quomodo crescunt : non laborant, neque nent : dico autem vobis, nec Salomon in omni gloria sua vestiebatur sicut unum ex istis. \EVERSE}
\newcommand{\lcXIIvXXVIII}{\VERSE  Si autem fœnum, quod hodie est in agro, et cras in clibanum mittitur, Deus sic vestit : quanto magis vos pusillæ fidei ? \EVERSE}
\newcommand{\lcXIIvXXIX}{\VERSE  Et vos nolite quærere quid manducetis, aut quid bibatis : et nolite in sublime tolli : \EVERSE}
\newcommand{\lcXIIvXXX}{\VERSE  hæc enim omnia gentes mundi quærunt. Pater autem vester scit quoniam his indigetis. \EVERSE}
\newcommand{\lcXIIvXXXI}{\VERSE  Verumtamen quærite primum regnum Dei, et justitiam ejus : et hæc omnia adjicientur vobis. \EVERSE}
\newcommand{\lcXIIvXXXII}{\VERSE  Nolite timere pusillus grex, quia complacuit Patri vestro dare vobis regnum. \EVERSE}
\newcommand{\lcXIIvXXXIII}{\VERSE  Vendite quæ possidetis, et date eleemosynam. Facite vobis sacculos, qui non veterascunt, thesaurum non deficientem in cælis : quo fur non appropriat, neque tinea corrumpit. \EVERSE}
\newcommand{\lcXIIvXXXIV}{\VERSE  Ubi enim thesaurus vester est, ibi et cor vestrum erit. \EVERSE}
\newcommand{\lcXIIvXXXV}{\VERSE  Sint lumbi vestri præcincti, et lucernæ ardentes in manibus vestris, \EVERSE}
\newcommand{\lcXIIvXXXVI}{\VERSE  et vos similes hominibus exspectantibus dominum suum quando revertatur a nuptiis : ut, cum venerit et pulsaverit, confestim aperiant ei. \EVERSE}
\newcommand{\lcXIIvXXXVII}{\VERSE  Beati servi illi quos, cum venerit dominus, invenerit vigilantes : amen dico vobis, quod præcinget se, et faciet illos discumbere, et transiens ministrabit illis. \EVERSE}
\newcommand{\lcXIIvXXXVIII}{\VERSE  Et si venerit in secunda vigilia, et si in tertia vigilia venerit, et ita invenerit, beati sunt servi illi. \EVERSE}
\newcommand{\lcXIIvXXXIX}{\VERSE  Hoc autem scitote, quoniam si sciret paterfamilias, qua hora fur veniret, vigilaret utique, et non sineret perfodi domum suam. \EVERSE}
\newcommand{\lcXIIvXL}{\VERSE  Et vos estote parati : quia qua hora non putatis, Filius hominis veniet. \EVERSE}
\newcommand{\lcXIIvXLI}{\VERSE  Ait autem ei Petrus : Domine, ad nos dicis hanc parabolam, an et ad omnes ? \EVERSE}
\newcommand{\lcXIIvXLII}{\VERSE  Dixit autem Dominus : Quis, putas, est fidelis dispensator, et prudens, quem constituit dominus supra familiam suam, ut det illis in tempore tritici mensuram ? \EVERSE}
\newcommand{\lcXIIvXLIII}{\VERSE  Beatus ille servus quem, cum venerit dominus, invenerit ita facientem. \EVERSE}
\newcommand{\lcXIIvXLIV}{\VERSE  Vere dico vobis, quoniam supra omnia quæ possidet, constituet illum. \EVERSE}
\newcommand{\lcXIIvXLV}{\VERSE  Quod si dixerit servus ille in corde suo : Moram facit dominus meus venire : et cœperit percutere servos, et ancillas, et edere, et bibere, et inebriari : \EVERSE}
\newcommand{\lcXIIvXLVI}{\VERSE  veniet dominus servi illius in die qua non sperat, et hora qua nescit, et dividet eum, partemque ejus cum infidelibus ponet. \EVERSE}
\newcommand{\lcXIIvXLVII}{\VERSE  Ille autem servus qui cognovit voluntatem domini sui, et non præparavit, et non facit secundum voluntatem ejus, vapulabit multis : \EVERSE}
\newcommand{\lcXIIvXLVIII}{\VERSE  qui autem non cognovit, et fecit digna plagis, vapulabit paucis. Omni autem cui multum datum est, multum quæretur ab eo : et cui commendaverunt multum, plus petent ab eo. \EVERSE}
\newcommand{\lcXIIvXLIX}{\VERSE  Ignem veni mittere in terram, et quid volo nisi ut accendatur ? \EVERSE}
\newcommand{\lcXIIvL}{\VERSE  Baptismo autem habeo baptizari : et quomodo coarctor usque dum perficiatur ? \EVERSE}
\newcommand{\lcXIIvLI}{\VERSE  Putatis quia pacem veni dare in terram ? non, dico vobis, sed separationem : \EVERSE}
\newcommand{\lcXIIvLII}{\VERSE  erunt enim ex hoc quinque in domo una divisi, tres in duos, et duo in tres \EVERSE}
\newcommand{\lcXIIvLIII}{\VERSE  dividentur : pater in filium, et filius in patrem suum, mater in filiam, et filia in matrem, socrus in nurum suam, et nurus in socrum suam. \EVERSE}
\newcommand{\lcXIIvLIV}{\VERSE  Dicebat autem et ad turbas : Cum videritis nubem orientem ab occasu, statim dicitis : Nimbus venit : et ita fit. \EVERSE}
\newcommand{\lcXIIvLV}{\VERSE  Et cum austrum flantem, dicitis : Quia æstus erit : et fit. \EVERSE}
\newcommand{\lcXIIvLVI}{\VERSE  Hypocritæ ! faciem cæli et terræ nostis probare : hoc autem tempus quomodo non probatis ? \EVERSE}
\newcommand{\lcXIIvLVII}{\VERSE  quid autem et a vobis ipsis non judicatis quod justum est ? \EVERSE}
\newcommand{\lcXIIvLVIII}{\VERSE  Cum autem vadis cum adversario tuo ad principem, in via da operam liberari ab illo, ne forte trahat te ad judicem, et judex tradat te exactori, et exactor mittat te in carcerem. \EVERSE}
\newcommand{\lcXIIvLIX}{\VERSE  Dico tibi, non exies inde, donec etiam novissimum minutum reddas. \EVERSE}
\newcommand{\lcXIIIvI}{\VERSE  Aderant autem quidam ipso in tempore, nuntiantes illi de Galilæis, quorum sanguinem Pilatus miscuit cum sacrificiis eorum. \EVERSE}
\newcommand{\lcXIIIvII}{\VERSE  Et respondens dixit illis : Putatis quod hi Galilæi præ omnibus Galilæis peccatores fuerint, quia talia passi sunt ? \EVERSE}
\newcommand{\lcXIIIvIII}{\VERSE  Non, dico vobis : sed nisi pœnitentiam habueritis, omnes similiter peribitis. \EVERSE}
\newcommand{\lcXIIIvIV}{\VERSE  Sicut illi decem et octo, supra quos cecidit turris in Siloë, et occidit eos : putatis quia et ipsi debitores fuerint præter omnes homines habitantes in Jerusalem ? \EVERSE}
\newcommand{\lcXIIIvV}{\VERSE  Non, dico vobis : sed si pœnitentiam non egeritis, omnes similiter peribitis. \EVERSE}
\newcommand{\lcXIIIvVI}{\VERSE  Dicebat autem et hanc similitudinem : Arborem fici habebat quidam plantatam in vinea sua, et venit quærens fructum in illa, et non invenit. \EVERSE}
\newcommand{\lcXIIIvVII}{\VERSE  Dixit autem ad cultorem vineæ : Ecce anni tres sunt ex quo venio quærens fructum in ficulnea hac, et non invenio : succide ergo illam : ut quid etiam terram occupat ? \EVERSE}
\newcommand{\lcXIIIvVIII}{\VERSE  At ille respondens, dicit illi : Domine dimitte illam et hoc anno, usque dum fodiam circa illam, et mittam stercora, \EVERSE}
\newcommand{\lcXIIIvIX}{\VERSE  et siquidem fecerit fructum : sin autem, in futurum succides eam. \EVERSE}
\newcommand{\lcXIIIvX}{\VERSE  Erat autem docens in synagoga eorum sabbatis. \EVERSE}
\newcommand{\lcXIIIvXI}{\VERSE  Et ecce mulier, quæ habebat spiritum infirmitatis annis decem et octo : et erat inclinata, nec omnino poterat sursum respicere. \EVERSE}
\newcommand{\lcXIIIvXII}{\VERSE  Quam cum videret Jesus, vocavit eam ad se, et ait illi : Mulier, dimissa es ab infirmitate tua. \EVERSE}
\newcommand{\lcXIIIvXIII}{\VERSE  Et imposuit illi manus, et confestim erecta est, et glorificabat Deum. \EVERSE}
\newcommand{\lcXIIIvXIV}{\VERSE  Respondens autem archisynagogus, indignans quia sabbato curasset Jesus, dicebat turbæ : Sex dies sunt in quibus oportet operari : in his ergo venite, et curamini, et non in die sabbati. \EVERSE}
\newcommand{\lcXIIIvXV}{\VERSE  Respondens autem ad illum Dominus, dixit : Hypocritæ, unusquisque vestrum sabbato non solvit bovem suum, aut asinum a præsepio, et ducit adaquare ? \EVERSE}
\newcommand{\lcXIIIvXVI}{\VERSE  Hanc autem filiam Abrahæ, quam alligavit Satanas, ecce decem et octo annis, non oportuit solvi a vinculo isto die sabbati ? \EVERSE}
\newcommand{\lcXIIIvXVII}{\VERSE  Et cum hæc diceret, erubescebant omnes adversarii ejus : et omnis populus gaudebat in universis, quæ gloriose fiebant ab eo. \EVERSE}
\newcommand{\lcXIIIvXVIII}{\VERSE  Dicebat ergo : Cui simile est regnum Dei, et cui simile æstimabo illud ? \EVERSE}
\newcommand{\lcXIIIvXIX}{\VERSE  Simile est grano sinapis, quod acceptum homo misit in hortum suum, et crevit, et factum est in arborem magnam : et volucres cæli requieverunt in ramis ejus. \EVERSE}
\newcommand{\lcXIIIvXX}{\VERSE  Et iterum dixit : Cui simile æstimabo regnum Dei ? \EVERSE}
\newcommand{\lcXIIIvXXI}{\VERSE  Simile est fermento, quod acceptum mulier abscondit in farinæ sata tria, donec fermentaretur totum. \EVERSE}
\newcommand{\lcXIIIvXXII}{\VERSE  Et ibat per civitates et castella, docens, et iter faciens in Jerusalem. \EVERSE}
\newcommand{\lcXIIIvXXIII}{\VERSE  Ait autem illi quidam : Domine, si pauci sunt, qui salvantur ? Ipse autem dixit ad illos : \EVERSE}
\newcommand{\lcXIIIvXXIV}{\VERSE  Contendite intrare per angustam portam : quia multi, dico vobis, quærent intrare, et non poterunt. \EVERSE}
\newcommand{\lcXIIIvXXV}{\VERSE  Cum autem intraverit paterfamilias, et clauserit ostium, incipietis foris stare, et pulsare ostium, dicentes : Domine, aperi nobis : et respondens dicet vobis : Nescio vos unde sitis : \EVERSE}
\newcommand{\lcXIIIvXXVI}{\VERSE  tunc incipietis dicere : Manducavimus coram te, et bibimus, et in plateis nostris docuisti. \EVERSE}
\newcommand{\lcXIIIvXXVII}{\VERSE  Et dicet vobis : Nescio vos unde sitis : discedite a me omnes operarii iniquitatis. \EVERSE}
\newcommand{\lcXIIIvXXVIII}{\VERSE  Ibi erit fletus et stridor dentium : cum videritis Abraham, et Isaac, et Jacob, et omnes prophetas in regno Dei, vos autem expelli foras. \EVERSE}
\newcommand{\lcXIIIvXXIX}{\VERSE  Et venient ab oriente, et occidente, et aquilone, et austro, et accumbent in regno Dei. \EVERSE}
\newcommand{\lcXIIIvXXX}{\VERSE  Et ecce sunt novissimi qui erunt primi, et sunt primi qui erunt novissimi. \EVERSE}
\newcommand{\lcXIIIvXXXI}{\VERSE  In ipsa die accesserunt quidam pharisæorum, dicentes illi : Exi, et vade hinc : quia Herodes vult te occidere. \EVERSE}
\newcommand{\lcXIIIvXXXII}{\VERSE  Et ait illis : Ite, et dicite vulpi illi : Ecce ejicio dæmonia, et sanitates perficio hodie, et cras, et tertia die consummor. \EVERSE}
\newcommand{\lcXIIIvXXXIII}{\VERSE  Verumtamen oportet me hodie et cras et sequenti die ambulare : quia non capit prophetam perire extra Jerusalem. \EVERSE}
\newcommand{\lcXIIIvXXXIV}{\VERSE  Jerusalem, Jerusalem, quæ occidis prophetas, et lapidas eos qui mittuntur ad te, quoties volui congregare filios tuos quemadmodum avis nidum suum sub pennis, et noluisti ? \EVERSE}
\newcommand{\lcXIIIvXXXV}{\VERSE  Ecce relinquetur vobis domus vestra deserta. Dico autem vobis, quia non videbitis me donec veniat cum dicetis : Benedictus qui venit in nomine Domini. \EVERSE}
\newcommand{\lcXIVvI}{\VERSE  Et factum est cum intraret Jesus in domum cujusdam principis pharisæorum sabbato manducare panem, et ipsi observabant eum. \EVERSE}
\newcommand{\lcXIVvII}{\VERSE  Et ecce homo quidam hydropicus erat ante illum. \EVERSE}
\newcommand{\lcXIVvIII}{\VERSE  Et respondens Jesus dixit ad legisperitos et pharisæos, dicens : Si licet sabbato curare ? \EVERSE}
\newcommand{\lcXIVvIV}{\VERSE  At illi tacuerunt. Ipse vero apprehensum sanavit eum, ac dimisit. \EVERSE}
\newcommand{\lcXIVvV}{\VERSE  Et respondens ad illos dixit : Cujus vestrum asinus, aut bos in puteum cadet, et non continuo extrahet illum die sabbati ? \EVERSE}
\newcommand{\lcXIVvVI}{\VERSE  Et non poterant ad hæc respondere illi. \EVERSE}
\newcommand{\lcXIVvVII}{\VERSE  Dicebat autem et ad invitatos parabolam, intendens quomodo primos accubitus eligerent, dicens ad illos : \EVERSE}
\newcommand{\lcXIVvVIII}{\VERSE  Cum invitatus fueris ad nuptias, non discumbas in primo loco, ne forte honoratior te sit invitatus ab illo. \EVERSE}
\newcommand{\lcXIVvIX}{\VERSE  Et veniens is, qui te et illum vocavit, dicat tibi : Da huic locum : et tunc incipias cum rubore novissimum locum tenere. \EVERSE}
\newcommand{\lcXIVvX}{\VERSE  Sed cum vocatus fueris, vade, recumbe in novissimo loco : ut, cum venerit qui te invitavit, dicat tibi : Amice, ascende superius. Tunc erit tibi gloria coram simul discumbentibus : \EVERSE}
\newcommand{\lcXIVvXI}{\VERSE  quia omnis, qui se exaltat, humiliabitur : et qui se humiliat, exaltabitur. \EVERSE}
\newcommand{\lcXIVvXII}{\VERSE  Dicebat autem et ei, qui invitaverat : Cum facis prandium, aut cœnam, noli vocare amicos tuos, neque fratres tuos, neque cognatos, neque vicinos divites : ne forte te et ipsi reinvitent, et fiat tibi retributio ; \EVERSE}
\newcommand{\lcXIVvXIII}{\VERSE  sed cum facis convivium, voca pauperes, debiles, claudos, et cæcos : \EVERSE}
\newcommand{\lcXIVvXIV}{\VERSE  et beatus eris, quia non habent retribuere tibi : retribuetur enim tibi in resurrectione justorum. \EVERSE}
\newcommand{\lcXIVvXV}{\VERSE  Hæc cum audisset quidam de simul discumbentibus, dixit illi : Beatus qui manducabit panem in regno Dei. \EVERSE}
\newcommand{\lcXIVvXVI}{\VERSE  At ipse dixit ei : Homo quidam fecit cœnam magnam, et vocavit multos. \EVERSE}
\newcommand{\lcXIVvXVII}{\VERSE  Et misit servum suum hora cœnæ dicere invitatis ut venirent, quia jam parata sunt omnia. \EVERSE}
\newcommand{\lcXIVvXVIII}{\VERSE  Et cœperunt simul omnes excusare. Primus dixit ei : Villam emi, et necesse habeo exire, et videre illam : rogo te, habe me excusatum. \EVERSE}
\newcommand{\lcXIVvXIX}{\VERSE  Et alter dixit : Juga boum emi quinque, et eo probare illa : rogo te, habe me excusatum. \EVERSE}
\newcommand{\lcXIVvXX}{\VERSE  Et alius dixit : Uxorem duxi, et ideo non possum venire. \EVERSE}
\newcommand{\lcXIVvXXI}{\VERSE  Et reversus servus nuntiavit hæc domino suo. Tunc iratus paterfamilias, dixit servo suo : Exi cito in plateas et vicos civitatis : et pauperes, ac debiles, et cæcos, et claudos introduc huc. \EVERSE}
\newcommand{\lcXIVvXXII}{\VERSE  Et ait servus : Domine, factum est ut imperasti, et adhuc locus est. \EVERSE}
\newcommand{\lcXIVvXXIII}{\VERSE  Et ait dominus servo : Exi in vias, et sæpes : et compelle intrare, ut impleatur domus mea. \EVERSE}
\newcommand{\lcXIVvXXIV}{\VERSE  Dico autem vobis quod nemo virorum illorum qui vocati sunt, gustabit cœnam meam. \EVERSE}
\newcommand{\lcXIVvXXV}{\VERSE  Ibant autem turbæ multæ cum eo : et conversus dixit ad illos : \EVERSE}
\newcommand{\lcXIVvXXVI}{\VERSE  Si quis venit ad me, et non odit patrem suum, et matrem, et uxorem, et filios, et fratres, et sorores, adhuc autem et animam suam, non potest meus esse discipulus. \EVERSE}
\newcommand{\lcXIVvXXVII}{\VERSE  Et qui non bajulat crucem suam, et venit post me, non potest meus esse discipulus. \EVERSE}
\newcommand{\lcXIVvXXVIII}{\VERSE  Quis enim ex vobis volens turrim ædificare, non prius sedens computat sumptus, qui necessarii sunt, si habeat ad perficiendum, \EVERSE}
\newcommand{\lcXIVvXXIX}{\VERSE  ne, posteaquam posuerit fundamentum, et non potuerit perficere, omnes qui vident, incipiant illudere ei, \EVERSE}
\newcommand{\lcXIVvXXX}{\VERSE  dicentes : Quia hic homo cœpit ædificare, et non potuit consummare ? \EVERSE}
\newcommand{\lcXIVvXXXI}{\VERSE  Aut quis rex iturus committere bellum adversus alium regem, non sedens prius cogitat, si possit cum decem millibus occurrere ei, qui cum viginti millibus venit ad se ? \EVERSE}
\newcommand{\lcXIVvXXXII}{\VERSE  Alioquin adhuc illo longe agente, legationem mittens rogat ea quæ pacis sunt. \EVERSE}
\newcommand{\lcXIVvXXXIII}{\VERSE  Sic ergo omnis ex vobis, qui non renuntiat omnibus quæ possidet, non potest meus esse discipulus. \EVERSE}
\newcommand{\lcXIVvXXXIV}{\VERSE  Bonum est sal : si autem sal evanuerit, in quo condietur ? \EVERSE}
\newcommand{\lcXIVvXXXV}{\VERSE  Neque in terram, neque in sterquilinium utile est, sed foras mittetur. Qui habet aures audiendi, audiat. \EVERSE}
\newcommand{\lcXVvI}{\VERSE  Erant autem appropinquantes ei publicani, et peccatores ut audirent illum. \EVERSE}
\newcommand{\lcXVvII}{\VERSE  Et murmurabant pharisæi, et scribæ, dicentes : Quia hic peccatores recipit, et manducat cum illis. \EVERSE}
\newcommand{\lcXVvIII}{\VERSE  Et ait ad illos parabolam istam dicens : \EVERSE}
\newcommand{\lcXVvIV}{\VERSE  Quis ex vobis homo, qui habet centum oves, et si perdiderit unam ex illis, nonne dimittit nonaginta novem in deserto, et vadit ad illam quæ perierat, donec inveniat eam ? \EVERSE}
\newcommand{\lcXVvV}{\VERSE  Et cum invenerit eam, imponit in humeros suos gaudens : \EVERSE}
\newcommand{\lcXVvVI}{\VERSE  et veniens domum convocat amicos et vicinos, dicens illis : Congratulamini mihi, quia inveni ovem meam, quæ perierat. \EVERSE}
\newcommand{\lcXVvVII}{\VERSE  Dico vobis quod ita gaudium erit in cælo super uno peccatore pœnitentiam agente, quam super nonaginta novem justis, qui non indigent pœnitentia. \EVERSE}
\newcommand{\lcXVvVIII}{\VERSE  Aut quæ mulier habens drachmas decem, si perdiderit drachmam unam, nonne accendit lucernam, et everrit domum, et quærit diligenter, donec inveniat ? \EVERSE}
\newcommand{\lcXVvIX}{\VERSE  Et cum invenerit convocat amicas et vicinas, dicens : Congratulamini mihi, quia inveni drachmam quam perdideram. \EVERSE}
\newcommand{\lcXVvX}{\VERSE  Ita, dico vobis, gaudium erit coram angelis Dei super uno peccatore pœnitentiam agente. \EVERSE}
\newcommand{\lcXVvXI}{\VERSE  Ait autem : Homo quidam habuit duos filios : \EVERSE}
\newcommand{\lcXVvXII}{\VERSE  et dixit adolescentior ex illis patri : Pater, da mihi portionem substantiæ, quæ me contingit. Et divisit illis substantiam. \EVERSE}
\newcommand{\lcXVvXIII}{\VERSE  Et non post multos dies, congregatis omnibus, adolescentior filius peregre profectus est in regionem longinquam, et ibi dissipavit substantiam suam vivendo luxuriose. \EVERSE}
\newcommand{\lcXVvXIV}{\VERSE  Et postquam omnia consummasset, facta est fames valida in regione illa, et ipse cœpit egere. \EVERSE}
\newcommand{\lcXVvXV}{\VERSE  Et abiit, et adhæsit uni civium regionis illius : et misit illum in villam suam ut pasceret porcos. \EVERSE}
\newcommand{\lcXVvXVI}{\VERSE  Et cupiebat implere ventrem suum de siliquis, quas porci manducabant : et nemo illi dabat. \EVERSE}
\newcommand{\lcXVvXVII}{\VERSE  In se autem reversus, dixit : Quanti mercenarii in domo patris mei abundant panibus, ego autem hic fame pereo ! \EVERSE}
\newcommand{\lcXVvXVIII}{\VERSE  surgam, et ibo ad patrem meum, et dicam ei : Pater, peccavi in cælum, et coram te : \EVERSE}
\newcommand{\lcXVvXIX}{\VERSE  jam non sum dignus vocari filius tuus : fac me sicut unum de mercenariis tuis. \EVERSE}
\newcommand{\lcXVvXX}{\VERSE  Et surgens venit ad patrem suum. Cum autem adhuc longe esset, vidit illum pater ipsius, et misericordia motus est, et accurrens cecidit super collum ejus, et osculatus est eum. \EVERSE}
\newcommand{\lcXVvXXI}{\VERSE  Dixitque ei filius : Pater, peccavi in cælum, et coram te : jam non sum dignus vocari filius tuus. \EVERSE}
\newcommand{\lcXVvXXII}{\VERSE  Dixit autem pater ad servos suos : Cito proferte stolam primam, et induite illum, et date annulum in manum ejus, et calceamenta in pedes ejus : \EVERSE}
\newcommand{\lcXVvXXIII}{\VERSE  et adducite vitulum saginatum, et occidite, et manducemus, et epulemur : \EVERSE}
\newcommand{\lcXVvXXIV}{\VERSE  quia hic filius meus mortuus erat, et revixit : perierat, et inventus est. Et cœperunt epulari. \EVERSE}
\newcommand{\lcXVvXXV}{\VERSE  Erat autem filius ejus senior in agro : et cum veniret, et appropinquaret domui, audivit symphoniam et chorum : \EVERSE}
\newcommand{\lcXVvXXVI}{\VERSE  et vocavit unum de servis, et interrogavit quid hæc essent. \EVERSE}
\newcommand{\lcXVvXXVII}{\VERSE  Isque dixit illi : Frater tuus venit, et occidit pater tuus vitulum saginatum, quia salvum illum recepit. \EVERSE}
\newcommand{\lcXVvXXVIII}{\VERSE  Indignatus est autem, et nolebat introire. Pater ergo illius egressus, cœpit rogare illum. \EVERSE}
\newcommand{\lcXVvXXIX}{\VERSE  At ille respondens, dixit patri suo : Ecce tot annis servio tibi, et numquam mandatum tuum præterivi : et numquam dedisti mihi hædum ut cum amicis meis epularer. \EVERSE}
\newcommand{\lcXVvXXX}{\VERSE  Sed postquam filius tuus hic, qui devoravit substantiam suam cum meretricibus, venit, occidisti illi vitulum saginatum. \EVERSE}
\newcommand{\lcXVvXXXI}{\VERSE  At ipse dixit illi : Fili, tu semper mecum es, et omnia mea tua sunt : \EVERSE}
\newcommand{\lcXVvXXXII}{\VERSE  epulari autem, et gaudere oportebat, quia frater tuus hic mortuus erat, et revixit ; perierat, et inventus est. \EVERSE}
\newcommand{\lcXVIvI}{\VERSE  Dicebat autem et ad discipulos suos : Homo quidam erat dives, qui habebat villicum : et hic diffamatus est apud illum quasi dissipasset bona ipsius. \EVERSE}
\newcommand{\lcXVIvII}{\VERSE  Et vocavit illum, et ait illi : Quid hoc audio de te ? redde rationem villicationis tuæ : jam enim non poteris villicare. \EVERSE}
\newcommand{\lcXVIvIII}{\VERSE  Ait autem villicus intra se : Quid faciam, quia dominus meus aufert a me villicationem ? Fodere non valeo, mendicare erubesco. \EVERSE}
\newcommand{\lcXVIvIV}{\VERSE  Scio quid faciam, ut, cum amotus fuero a villicatione, recipiant me in domos suas. \EVERSE}
\newcommand{\lcXVIvV}{\VERSE  Convocatis itaque singulis debitoribus domini sui, dicebat primo : Quantum debes domino meo ? \EVERSE}
\newcommand{\lcXVIvVI}{\VERSE  At ille dixit : Centum cados olei. Dixitque illi : Accipe cautionem tuam : et sede cito, scribe quinquaginta. \EVERSE}
\newcommand{\lcXVIvVII}{\VERSE  Deinde alii dixit : Tu vero quantum debes ? Qui ait : Centum coros tritici. Ait illi : Accipe litteras tuas, et scribe octoginta. \EVERSE}
\newcommand{\lcXVIvVIII}{\VERSE  Et laudavit dominus villicum iniquitatis, quia prudenter fecisset : quia filii hujus sæculi prudentiores filiis lucis in generatione sua sunt. \EVERSE}
\newcommand{\lcXVIvIX}{\VERSE  Et ego vobis dico : facite vobis amicos de mammona iniquitatis : ut, cum defeceritis, recipiant vos in æterna tabernacula. \EVERSE}
\newcommand{\lcXVIvX}{\VERSE  Qui fidelis est in minimo, et in majori fidelis est : et qui in modico iniquus est, et in majori iniquus est. \EVERSE}
\newcommand{\lcXVIvXI}{\VERSE  Si ergo in iniquo mammona fideles non fuistis quod verum est, quis credet vobis ? \EVERSE}
\newcommand{\lcXVIvXII}{\VERSE  Et si in alieno fideles non fuistis, quod vestrum est, quis dabit vobis ? \EVERSE}
\newcommand{\lcXVIvXIII}{\VERSE  Nemo servus potest duobus dominis servire : aut enim unum odiet, et alterum diliget : aut uni adhærebit, et alterum contemnet. Non potestis Deo servire et mammonæ. \EVERSE}
\newcommand{\lcXVIvXIV}{\VERSE  Audiebant autem omnia hæc pharisæi, qui erant avari : et deridebant illum. \EVERSE}
\newcommand{\lcXVIvXV}{\VERSE  Et ait illis : Vos estis qui justificatis vos coram hominibus : Deus autem novit corda vestra : quia quod hominibus altum est, abominatio est ante Deum. \EVERSE}
\newcommand{\lcXVIvXVI}{\VERSE  Lex et prophetæ usque ad Joannem : ex eo regnum Dei evangelizatur, et omnis in illud vim facit. \EVERSE}
\newcommand{\lcXVIvXVII}{\VERSE  Facilius est autem cælum et terram præterire, quam de lege unum apicem cadere. \EVERSE}
\newcommand{\lcXVIvXVIII}{\VERSE  Omnis qui dimittit uxorem suam et alteram ducit, mœchatur : et qui dimissam a viro ducit, mœchatur. \EVERSE}
\newcommand{\lcXVIvXIX}{\VERSE  Homo quidam erat dives, qui induebatur purpura et bysso, et epulabatur quotidie splendide. \EVERSE}
\newcommand{\lcXVIvXX}{\VERSE  Et erat quidam mendicus, nomine Lazarus, qui jacebat ad januam ejus, ulceribus plenus, \EVERSE}
\newcommand{\lcXVIvXXI}{\VERSE  cupiens saturari de micis quæ cadebant de mensa divitis, et nemo illi dabat : sed et canes veniebant, et lingebant ulcera ejus. \EVERSE}
\newcommand{\lcXVIvXXII}{\VERSE  Factum est autem ut moreretur mendicus, et portaretur ab angelis in sinum Abrahæ. Mortuus est autem et dives, et sepultus est in inferno. \EVERSE}
\newcommand{\lcXVIvXXIII}{\VERSE  Elevans autem oculos suos, cum esset in tormentis, vidit Abraham a longe, et Lazarum in sinu ejus : \EVERSE}
\newcommand{\lcXVIvXXIV}{\VERSE  et ipse clamans dixit : Pater Abraham, miserere mei, et mitte Lazarum ut intingat extremum digiti sui in aquam, ut refrigeret linguam meam, quia crucior in hac flamma. \EVERSE}
\newcommand{\lcXVIvXXV}{\VERSE  Et dixit illi Abraham : Fili, recordare quia recepisti bona in vita tua, et Lazarus similiter mala : nunc autem hic consolatur, tu vero cruciaris : \EVERSE}
\newcommand{\lcXVIvXXVI}{\VERSE  et in his omnibus inter nos et vos chaos magnum firmatum est : ut hi qui volunt hinc transire ad vos, non possint, neque inde huc transmeare. \EVERSE}
\newcommand{\lcXVIvXXVII}{\VERSE  Et ait : Rogo ergo te, pater, ut mittas eum in domum patris mei : \EVERSE}
\newcommand{\lcXVIvXXVIII}{\VERSE  habeo enim quinque fratres : ut testetur illis, ne et ipsi veniant in hunc locum tormentorum. \EVERSE}
\newcommand{\lcXVIvXXIX}{\VERSE  Et ait illi Abraham : Habent Moysen et prophetas : audiant illos. \EVERSE}
\newcommand{\lcXVIvXXX}{\VERSE  At ille dixit : Non, pater Abraham : sed si quis ex mortuis ierit ad eos, pœnitentiam agent. \EVERSE}
\newcommand{\lcXVIvXXXI}{\VERSE  Ait autem illi : Si Moysen et prophetas non audiunt, neque si quis ex mortuis resurrexerit, credent. \EVERSE}
\newcommand{\lcXVIIvI}{\VERSE  Et ait ad discipulos suos : Impossibile est ut non veniant scandala : væ autem illi per quem veniunt. \EVERSE}
\newcommand{\lcXVIIvII}{\VERSE  Utilius est illi si lapis molaris imponatur circa collum ejus, et projiciatur in mare quam ut scandalizet unum de pusillis istis. \EVERSE}
\newcommand{\lcXVIIvIII}{\VERSE  Attendite vobis : Si peccaverit in te frater tuus, increpa illum : et si pœnitentiam egerit, dimitte illi. \EVERSE}
\newcommand{\lcXVIIvIV}{\VERSE  Et si septies in die peccaverit in te, et septies in die conversus fuerit ad te, dicens : Pœnitet me, dimitte illi. \EVERSE}
\newcommand{\lcXVIIvV}{\VERSE  Et dixerunt apostoli Domino : Adauge nobis fidem. \EVERSE}
\newcommand{\lcXVIIvVI}{\VERSE  Dixit autem Dominus : Si habueritis fidem sicut granum sinapis, dicetis huic arbori moro : Eradicare, et transplantare in mare, et obediet vobis. \EVERSE}
\newcommand{\lcXVIIvVII}{\VERSE  Quis autem vestrum habens servum arantem aut pascentem, qui regresso de agro dicat illi : Statim transi, recumbe : \EVERSE}
\newcommand{\lcXVIIvVIII}{\VERSE  et non dicat ei : Para quod cœnem, et præcinge te, et ministra mihi donec manducem, et bibam, et post hæc tu manducabis, et bibes ? \EVERSE}
\newcommand{\lcXVIIvIX}{\VERSE  Numquid gratiam habet servo illi, quia fecit quæ ei imperaverat ? \EVERSE}
\newcommand{\lcXVIIvX}{\VERSE  non puto. Sic et vos cum feceritis omnia quæ præcepta sunt vobis, dicite : Servi inutiles sumus : quod debuimus facere, fecimus. \EVERSE}
\newcommand{\lcXVIIvXI}{\VERSE  Et factum est, dum iret in Jerusalem, transibat per mediam Samariam et Galilæam. \EVERSE}
\newcommand{\lcXVIIvXII}{\VERSE  Et cum ingrederetur quoddam castellum, occurrerunt ei decem viri leprosi, qui steterunt a longe : \EVERSE}
\newcommand{\lcXVIIvXIII}{\VERSE  et levaverunt vocem, dicentes : Jesu præceptor, miserere nostri. \EVERSE}
\newcommand{\lcXVIIvXIV}{\VERSE  Quos ut vidit, dixit : Ite, ostendite vos sacerdotibus. Et factum est, dum irent, mundati sunt. \EVERSE}
\newcommand{\lcXVIIvXV}{\VERSE  Unus autem ex illis, ut vidit quia mundatus est, regressus est, cum magna voce magnificans Deum, \EVERSE}
\newcommand{\lcXVIIvXVI}{\VERSE  et cecidit in faciem ante pedes ejus, gratias agens : et hic erat Samaritanus. \EVERSE}
\newcommand{\lcXVIIvXVII}{\VERSE  Respondens autem Jesus, dixit : Nonne decem mundati sunt ? et novem ubi sunt ? \EVERSE}
\newcommand{\lcXVIIvXVIII}{\VERSE  Non est inventus qui rediret, et daret gloriam Deo, nisi hic alienigena. \EVERSE}
\newcommand{\lcXVIIvXIX}{\VERSE  Et ait illi : Surge, vade : quia fides tua te salvum fecit. \EVERSE}
\newcommand{\lcXVIIvXX}{\VERSE  Interrogatus autem a pharisæis : Quando venit regnum Dei ? respondens eis, dixit : Non venit regnum Dei cum observatione : \EVERSE}
\newcommand{\lcXVIIvXXI}{\VERSE  neque dicent : Ecce hic, aut ecce illic. Ecce enim regnum Dei intra vos est. \EVERSE}
\newcommand{\lcXVIIvXXII}{\VERSE  Et ait ad discipulos suos : Venient dies quando desideretis videre unum diem Filii hominis, et non videbitis. \EVERSE}
\newcommand{\lcXVIIvXXIII}{\VERSE  Et dicent vobis : Ecce hic, et ecce illic. Nolite ire, neque sectemini : \EVERSE}
\newcommand{\lcXVIIvXXIV}{\VERSE  nam, sicut fulgur coruscans de sub cælo in ea quæ sub cælo sunt, fulget : ita erit Filius hominis in die sua. \EVERSE}
\newcommand{\lcXVIIvXXV}{\VERSE  Primum autem oportet illum multa pati, et reprobari a generatione hac. \EVERSE}
\newcommand{\lcXVIIvXXVI}{\VERSE  Et sicut factum est in diebus Noë, ita erit et in diebus Filii hominis : \EVERSE}
\newcommand{\lcXVIIvXXVII}{\VERSE  edebant et bibebant : uxores ducebant et dabantur ad nuptias, usque in diem, qua intravit Noë in arcam : et venit diluvium, et perdidit omnes. \EVERSE}
\newcommand{\lcXVIIvXXVIII}{\VERSE  Similiter sicut factum est in diebus Lot : edebant et bibebant, emebant et vendebant, plantabant et ædificabant : \EVERSE}
\newcommand{\lcXVIIvXXIX}{\VERSE  qua die autem exiit Lot a Sodomis, pluit ignem et sulphur de cælo, et omnes perdidit : \EVERSE}
\newcommand{\lcXVIIvXXX}{\VERSE  secundum hæc erit qua die Filius hominis revelabitur. \EVERSE}
\newcommand{\lcXVIIvXXXI}{\VERSE  In illa hora, qui fuerit in tecto, et vasa ejus in domo, ne descendat tollere illa : et qui in agro, similiter non redeat retro. \EVERSE}
\newcommand{\lcXVIIvXXXII}{\VERSE  Memores estote uxoris Lot. \EVERSE}
\newcommand{\lcXVIIvXXXIII}{\VERSE  Quicumque quæsierit animam suam salvam facere, perdet illam : et quicumque perdiderit illam, vivificabit eam. \EVERSE}
\newcommand{\lcXVIIvXXXIV}{\VERSE  Dico vobis : In illa nocte erunt duo in lecto uno : unus assumetur, et alter relinquetur : \EVERSE}
\newcommand{\lcXVIIvXXXV}{\VERSE  duæ erunt molentes in unum : una assumetur, et altera relinquetur : duo in agro : unus assumetur, et alter relinquetur. \EVERSE}
\newcommand{\lcXVIIvXXXVI}{\VERSE  Respondentes dicunt illi : Ubi Domine ? \EVERSE}
\newcommand{\lcXVIIvXXXVII}{\VERSE  Qui dixit illis : Ubicumque fuerit corpus, illuc congregabuntur et aquilæ. \EVERSE}
\newcommand{\lcXVIIIvI}{\VERSE  Dicebat autem et parabolam ad illos, quoniam oportet semper orare et non deficere, \EVERSE}
\newcommand{\lcXVIIIvII}{\VERSE  dicens : Judex quidam erat in quadam civitate, qui Deum non timebat, et hominem non reverebatur. \EVERSE}
\newcommand{\lcXVIIIvIII}{\VERSE  Vidua autem quædam erat in civitate illa, et veniebat ad eum, dicens : Vindica me de adversario meo. \EVERSE}
\newcommand{\lcXVIIIvIV}{\VERSE  Et nolebat per multum tempus. Post hæc autem dixit intra se : Etsi Deum non timeo, nec hominem revereor : \EVERSE}
\newcommand{\lcXVIIIvV}{\VERSE  tamen quia molesta est mihi hæc vidua, vindicabo illam, ne in novissimo veniens sugillet me. \EVERSE}
\newcommand{\lcXVIIIvVI}{\VERSE  Ait autem Dominus : Audite quid judex iniquitatis dicit : \EVERSE}
\newcommand{\lcXVIIIvVII}{\VERSE  Deus autem non faciet vindictam electorum suorum clamantium ad se die ac nocte, et patientiam habebit in illis ? \EVERSE}
\newcommand{\lcXVIIIvVIII}{\VERSE  Dico vobis quia cito faciet vindictam illorum. Verumtamen Filius hominis veniens, putas, inveniet fidem in terra ? \EVERSE}
\newcommand{\lcXVIIIvIX}{\VERSE  Dixit autem et ad quosdam qui in se confidebant tamquam justi, et aspernabantur ceteros, parabolam istam : \EVERSE}
\newcommand{\lcXVIIIvX}{\VERSE  Duo homines ascenderunt in templum ut orarent : unus pharisæus et alter publicanus. \EVERSE}
\newcommand{\lcXVIIIvXI}{\VERSE  Pharisæus stans, hæc apud se orabat : Deus, gratias ago tibi, quia non sum sicut ceteri hominum : raptores, injusti, adulteri, velut etiam hic publicanus : \EVERSE}
\newcommand{\lcXVIIIvXII}{\VERSE  jejuno bis in sabbato, decimas do omnium quæ possideo. \EVERSE}
\newcommand{\lcXVIIIvXIII}{\VERSE  Et publicanus a longe stans, nolebat nec oculos ad cælum levare : sed percutiebat pectus suum, dicens : Deus propitius esto mihi peccatori. \EVERSE}
\newcommand{\lcXVIIIvXIV}{\VERSE  Dico vobis, descendit hic justificatus in domum suam ab illo : quia omnis qui se exaltat, humiliabitur, et qui se humiliat, exaltabitur. \EVERSE}
\newcommand{\lcXVIIIvXV}{\VERSE  Afferebant autem ad illum et infantes, ut eos tangeret. Quod cum viderent discipuli, increpabant illos. \EVERSE}
\newcommand{\lcXVIIIvXVI}{\VERSE  Jesus autem convocans illos, dixit : Sinite pueros venire ad me, et nolite vetare eos : talium est enim regnum Dei. \EVERSE}
\newcommand{\lcXVIIIvXVII}{\VERSE  Amen dico vobis, quicumque non acceperit regnum Dei sicut puer, non intrabit in illud. \EVERSE}
\newcommand{\lcXVIIIvXVIII}{\VERSE  Et interrogavit eum quidam princeps, dicens : Magister bone, quid faciens vitam æternam possidebo ? \EVERSE}
\newcommand{\lcXVIIIvXIX}{\VERSE  Dixit autem ei Jesus : Quid me dicis bonum ? nemo bonus nisi solus Deus. \EVERSE}
\newcommand{\lcXVIIIvXX}{\VERSE  Mandata nosti : non occides ; non mœchaberis ; non furtum facies ; non falsum testimonium dices ; honora patrem tuum et matrem. \EVERSE}
\newcommand{\lcXVIIIvXXI}{\VERSE  Qui ait : Hæc omnia custodivi a juventute mea. \EVERSE}
\newcommand{\lcXVIIIvXXII}{\VERSE  Quo audito, Jesus ait ei : Adhuc unum tibi deest : omnia quæcumque habes vende, et da pauperibus, et habebis thesaurum in cælo : et veni, sequere me. \EVERSE}
\newcommand{\lcXVIIIvXXIII}{\VERSE  His ille auditis, contristatus est : quia dives erat valde. \EVERSE}
\newcommand{\lcXVIIIvXXIV}{\VERSE  Videns autem Jesus illum tristem factum, dixit : Quam difficile, qui pecunias habent, in regnum Dei intrabunt ! \EVERSE}
\newcommand{\lcXVIIIvXXV}{\VERSE  facilius est enim camelum per foramen acus transire quam divitem intrare in regnum Dei. \EVERSE}
\newcommand{\lcXVIIIvXXVI}{\VERSE  Et dixerunt qui audiebant : Et quis potest salvus fieri ? \EVERSE}
\newcommand{\lcXVIIIvXXVII}{\VERSE  Ait illis : Quæ impossibilia sunt apud homines, possibilia sunt apud Deum. \EVERSE}
\newcommand{\lcXVIIIvXXVIII}{\VERSE  Ait autem Petrus : Ecce nos dimisimus omnia et secuti sumus te. \EVERSE}
\newcommand{\lcXVIIIvXXIX}{\VERSE  Qui dixit eis : Amen dico vobis, nemo est qui reliquit domum, aut parentes, aut fratres, aut uxorem, aut filios propter regnum Dei, \EVERSE}
\newcommand{\lcXVIIIvXXX}{\VERSE  et non recipiat multo plura in hoc tempore, et in sæculo venturo vitam æternam. \EVERSE}
\newcommand{\lcXVIIIvXXXI}{\VERSE  Assumpsit autem Jesus duodecim, et ait illis : Ecce ascendimus Jerosolymam, et consummabuntur omnia quæ scripta sunt per prophetas de Filio hominis : \EVERSE}
\newcommand{\lcXVIIIvXXXII}{\VERSE  tradetur enim gentibus, et illudetur, et flagellabitur, et conspuetur : \EVERSE}
\newcommand{\lcXVIIIvXXXIII}{\VERSE  et postquam flagellaverint, occident eum, et tertia die resurget. \EVERSE}
\newcommand{\lcXVIIIvXXXIV}{\VERSE  Et ipsi nihil horum intellexerunt, et erat verbum istud absconditum ab eis, et non intelligebant quæ dicebantur. \EVERSE}
\newcommand{\lcXVIIIvXXXV}{\VERSE  Factum est autem, cum appropinquaret Jericho, cæcus quidam sedebat secus viam, mendicans. \EVERSE}
\newcommand{\lcXVIIIvXXXVI}{\VERSE  Et cum audiret turbam prætereuntem, interrogabat quid hoc esset. \EVERSE}
\newcommand{\lcXVIIIvXXXVII}{\VERSE  Dixerunt autem ei quod Jesus Nazarenus transiret. \EVERSE}
\newcommand{\lcXVIIIvXXXVIII}{\VERSE  Et clamavit, dicens : Jesu, fili David, miserere mei. \EVERSE}
\newcommand{\lcXVIIIvXXXIX}{\VERSE  Et qui præibant, increpabant eum ut taceret. Ipse vero multo magis clamabat : Fili David, miserere mei. \EVERSE}
\newcommand{\lcXVIIIvXL}{\VERSE  Stans autem Jesus jussit illum adduci ad se. Et cum appropinquasset, interrogavit illum, \EVERSE}
\newcommand{\lcXVIIIvXLI}{\VERSE  dicens : Quid tibi vis faciam ? At ille dixit : Domine, ut videam. \EVERSE}
\newcommand{\lcXVIIIvXLII}{\VERSE  Et Jesus dixit illi : Respice, fides tua te salvum fecit. \EVERSE}
\newcommand{\lcXVIIIvXLIII}{\VERSE  Et confestim vidit, et sequebatur illum magnificans Deum. Et omnis plebs ut vidit, dedit laudem Deo. \EVERSE}
\newcommand{\lcXIXvI}{\VERSE  Et ingressus perambulabat Jericho. \EVERSE}
\newcommand{\lcXIXvII}{\VERSE  Et ecce vir nomine Zachæus : et hic princeps erat publicanorum, et ipse dives : \EVERSE}
\newcommand{\lcXIXvIII}{\VERSE  et quærebat videre Jesum, quis esset : et non poterat præ turba, quia statura pusillus erat. \EVERSE}
\newcommand{\lcXIXvIV}{\VERSE  Et præcurrens ascendit in arborem sycomorum ut videret eum : quia inde erat transiturus. \EVERSE}
\newcommand{\lcXIXvV}{\VERSE  Et cum venisset ad locum, suspiciens Jesus vidit illum, et dixit ad eum : Zachæe, festinans descende : quia hodie in domo tua oportet me manere. \EVERSE}
\newcommand{\lcXIXvVI}{\VERSE  Et festinans descendit, et excepit illum gaudens. \EVERSE}
\newcommand{\lcXIXvVII}{\VERSE  Et cum viderent omnes, murmurabant, dicentes quod ad hominem peccatorem divertisset. \EVERSE}
\newcommand{\lcXIXvVIII}{\VERSE  Stans autem Zachæus, dixit ad Dominum : Ecce dimidium bonorum meorum, Domine, do pauperibus : et si quid aliquem defraudavi, reddo quadruplum. \EVERSE}
\newcommand{\lcXIXvIX}{\VERSE  Ait Jesus ad eum : Quia hodie salus domui huic facta est : eo quod et ipse filius sit Abrahæ. \EVERSE}
\newcommand{\lcXIXvX}{\VERSE  Venit enim Filius hominis quærere, et salvum facere quod perierat. \EVERSE}
\newcommand{\lcXIXvXI}{\VERSE  Hæc illis audientibus adjiciens, dixit parabolam, eo quod esset prope Jerusalem : et quia existimarent quod confestim regnum Dei manifestaretur. \EVERSE}
\newcommand{\lcXIXvXII}{\VERSE  Dixit ergo : Homo quidam nobilis abiit in regionem longinquam accipere sibi regnum, et reverti. \EVERSE}
\newcommand{\lcXIXvXIII}{\VERSE  Vocatis autem decem servis suis, dedit eis decem mnas, et ait ad illos : Negotiamini dum venio. \EVERSE}
\newcommand{\lcXIXvXIV}{\VERSE  Cives autem ejus oderant eum : et miserunt legationem post illum, dicentes : Nolumus hunc regnare super nos. \EVERSE}
\newcommand{\lcXIXvXV}{\VERSE  Et factum est ut rediret accepto regno : et jussit vocari servos, quibus dedit pecuniam, ut sciret quantum quisque negotiatus esset. \EVERSE}
\newcommand{\lcXIXvXVI}{\VERSE  Venit autem primus dicens : Domine, mna tua decem mnas acquisivit. \EVERSE}
\newcommand{\lcXIXvXVII}{\VERSE  Et ait illi : Euge bone serve, quia in modico fuisti fidelis, eris potestatem habens super decem civitates. \EVERSE}
\newcommand{\lcXIXvXVIII}{\VERSE  Et alter venit, dicens : Domine, mna tua fecit quinque mnas. \EVERSE}
\newcommand{\lcXIXvXIX}{\VERSE  Et huic ait : Et tu esto super quinque civitates. \EVERSE}
\newcommand{\lcXIXvXX}{\VERSE  Et alter venit, dicens : Domine, ecce mna tua, quam habui repositam in sudario : \EVERSE}
\newcommand{\lcXIXvXXI}{\VERSE  timui enim te, quia homo austerus es : tollis quod non posuisti, et metis quod non seminasti. \EVERSE}
\newcommand{\lcXIXvXXII}{\VERSE  Dicit ei : De ore tuo te judico, serve nequam. Sciebas quod ego homo austerus sum, tollens quod non posui, et metens quod non seminavi : \EVERSE}
\newcommand{\lcXIXvXXIII}{\VERSE  et quare non dedisti pecuniam meam ad mensam, ut ego veniens cum usuris utique exegissem illam ? \EVERSE}
\newcommand{\lcXIXvXXIV}{\VERSE  Et astantibus dixit : Auferte ab illo mnam, et date illi qui decem mnas habet. \EVERSE}
\newcommand{\lcXIXvXXV}{\VERSE  Et dixerunt ei : Domine, habet decem mnas. \EVERSE}
\newcommand{\lcXIXvXXVI}{\VERSE  Dico autem vobis, quia omni habenti dabitur, et abundabit : ab eo autem qui non habet, et quod habet auferetur ab eo. \EVERSE}
\newcommand{\lcXIXvXXVII}{\VERSE  Verumtamen inimicos meos illos, qui noluerunt me regnare super se, adducite huc : et interficite ante me. \EVERSE}
\newcommand{\lcXIXvXXVIII}{\VERSE  Et his dictis, præcedebat ascendens Jerosolymam. \EVERSE}
\newcommand{\lcXIXvXXIX}{\VERSE  Et factum est, cum appropinquasset ad Bethphage et Bethaniam, ad montem qui vocatur Oliveti, misit duos discipulos suos, \EVERSE}
\newcommand{\lcXIXvXXX}{\VERSE  dicens : Ite in castellum quod contra est : in quod introëuntes, invenietis pullum asinæ alligatum, cui nemo umquam hominum sedit : solvite illum, et adducite. \EVERSE}
\newcommand{\lcXIXvXXXI}{\VERSE  Et si quis vos interrogaverit : Quare solvitis ? sic dicetis ei : Quia Dominus operam ejus desiderat. \EVERSE}
\newcommand{\lcXIXvXXXII}{\VERSE  Abierunt autem qui missi erant : et invenerunt, sicut dixit illis, stantem pullum. \EVERSE}
\newcommand{\lcXIXvXXXIII}{\VERSE  Solventibus autem illis pullum, dixerunt domini ejus ad illos : Quid solvitis pullum ? \EVERSE}
\newcommand{\lcXIXvXXXIV}{\VERSE  At illi dixerunt : Quia Dominus eum necessarium habet. \EVERSE}
\newcommand{\lcXIXvXXXV}{\VERSE  Et duxerunt illum ad Jesum. Et jactantes vestimenta sua supra pullum, imposuerunt Jesum. \EVERSE}
\newcommand{\lcXIXvXXXVI}{\VERSE  Eunte autem illo, substernebant vestimenta sua in via : \EVERSE}
\newcommand{\lcXIXvXXXVII}{\VERSE  et cum appropinquaret jam ad descensum montis Oliveti, cœperunt omnes turbæ discipulorum gaudentes laudare Deum voce magna super omnibus, quas viderant, virtutibus, \EVERSE}
\newcommand{\lcXIXvXXXVIII}{\VERSE  dicentes : Benedictus, qui venit rex in nomine Domini : pax in cælo, et gloria in excelsis. \EVERSE}
\newcommand{\lcXIXvXXXIX}{\VERSE  Et quidam pharisæorum de turbis dixerunt ad illum : Magister, increpa discipulos tuos. \EVERSE}
\newcommand{\lcXIXvXL}{\VERSE  Quibus ipse ait : Dico vobis, quia si hi tacuerint, lapides clamabunt. \EVERSE}
\newcommand{\lcXIXvXLI}{\VERSE  Et ut appropinquavit, videns civitatem flevit super illam, dicens : \EVERSE}
\newcommand{\lcXIXvXLII}{\VERSE  Quia si cognovisses et tu, et quidem in hac die tua, quæ ad pacem tibi : nunc autem abscondita sunt ab oculis tuis. \EVERSE}
\newcommand{\lcXIXvXLIII}{\VERSE  Quia venient dies in te : et circumdabunt te inimici tui vallo, et circumdabunt te : et coangustabunt te undique : \EVERSE}
\newcommand{\lcXIXvXLIV}{\VERSE  et ad terram prosternent te, et filios tuos, qui in te sunt, et non relinquent in te lapidem super lapidem : eo quod non cognoveris tempus visitationis tuæ. \EVERSE}
\newcommand{\lcXIXvXLV}{\VERSE  Et ingressus in templum, cœpit ejicere vendentes in illo, et ementes, \EVERSE}
\newcommand{\lcXIXvXLVI}{\VERSE  dicens illis : Scriptum est : Quia domus mea domus orationis est : vos autem fecistis illam speluncam latronum. \EVERSE}
\newcommand{\lcXIXvXLVII}{\VERSE  Et erat docens quotidie in templo. Principes autem sacerdotum, et scribæ, et princeps plebis quærebant illum perdere : \EVERSE}
\newcommand{\lcXIXvXLVIII}{\VERSE  et non inveniebant quid facerent illi. Omnis enim populus suspensus erat, audiens illum. \EVERSE}
\newcommand{\lcXXvI}{\VERSE  Et factum est in una dierum, docente illo populum in templo, et evangelizante, convenerunt principes sacerdotum, et scribæ cum senioribus, \EVERSE}
\newcommand{\lcXXvII}{\VERSE  et aiunt dicentes ad illum : Dic nobis in qua potestate hæc facis ? aut quis est qui dedit tibi hanc potestatem ? \EVERSE}
\newcommand{\lcXXvIII}{\VERSE  Respondens autem Jesus, dixit ad illos : Interrogabo vos et ego unum verbum. Respondete mihi : \EVERSE}
\newcommand{\lcXXvIV}{\VERSE  baptismus Joannis de cælo erat, an ex hominibus ? \EVERSE}
\newcommand{\lcXXvV}{\VERSE  At illi cogitabant intra se, dicentes : Quia si dixerimus : De cælo, dicet : Quare ergo non credidistis illi ? \EVERSE}
\newcommand{\lcXXvVI}{\VERSE  Si autem dixerimus : Ex hominibus, plebs universa lapidabit nos : certi sunt enim Joannem prophetam esse. \EVERSE}
\newcommand{\lcXXvVII}{\VERSE  Et responderunt se nescire unde esset. \EVERSE}
\newcommand{\lcXXvVIII}{\VERSE  Et Jesus ait illis : Neque ego dico vobis in qua potestate hæc facio. \EVERSE}
\newcommand{\lcXXvIX}{\VERSE  Cœpit autem dicere ad plebem parabolam hanc : Homo plantavit vineam, et locavit eam colonis : et ipse peregre fuit multis temporibus. \EVERSE}
\newcommand{\lcXXvX}{\VERSE  Et in tempore misit ad cultores servum, ut de fructu vineæ darent illi. Qui cæsum dimiserunt eum inanem. \EVERSE}
\newcommand{\lcXXvXI}{\VERSE  Et addidit alterum servum mittere. Illi autem hunc quoque cædentes, et afficientes contumelia, dimiserunt inanem. \EVERSE}
\newcommand{\lcXXvXII}{\VERSE  Et addidit tertium mittere : qui et illum vulnerantes ejecerunt. \EVERSE}
\newcommand{\lcXXvXIII}{\VERSE  Dixit autem dominus vineæ : Quid faciam ? Mittam filium meum dilectum : forsitan, cum hunc viderint, verebuntur. \EVERSE}
\newcommand{\lcXXvXIV}{\VERSE  Quem cum vidissent coloni, cogitaverunt intra se, dicentes : Hic est hæres, occidamus illum, ut nostra fiat hæreditas. \EVERSE}
\newcommand{\lcXXvXV}{\VERSE  Et ejectum illum extra vineam, occiderunt. Quid ergo faciet illis dominus vineæ ? \EVERSE}
\newcommand{\lcXXvXVI}{\VERSE  veniet, et perdet colonos istos, et dabit vineam aliis. Quo audito, dixerunt illi : Absit. \EVERSE}
\newcommand{\lcXXvXVII}{\VERSE  Ille autem aspiciens eos, ait : Quid est ergo hoc quod scriptum est : Lapidem quem reprobaverunt ædificantes, hic factus est in caput anguli ? \EVERSE}
\newcommand{\lcXXvXVIII}{\VERSE  Omnis qui ceciderit super illum lapidem, conquassabitur : super quem autem ceciderit, comminuet illum. \EVERSE}
\newcommand{\lcXXvXIX}{\VERSE  Et quærebant principes sacerdotum et scribæ mittere in illum manus illa hora, et timuerunt populum : cognoverunt enim quod ad ipsos dixerit similitudinem hanc. \EVERSE}
\newcommand{\lcXXvXX}{\VERSE  Et observantes miserunt insidiatores, qui se justos simularent, ut caperent eum in sermone, ut traderent illum principatui, et potestati præsidis. \EVERSE}
\newcommand{\lcXXvXXI}{\VERSE  Et interrogaverunt eum, dicentes : Magister, scimus quia recte dicis et doces : et non accipis personam, sed viam Dei in veritate doces. \EVERSE}
\newcommand{\lcXXvXXII}{\VERSE  Licet nobis tributum dare Cæsari, an non ? \EVERSE}
\newcommand{\lcXXvXXIII}{\VERSE  Considerans autem dolum illorum, dixit ad eos : Quid me tentatis ? \EVERSE}
\newcommand{\lcXXvXXIV}{\VERSE  ostendite mihi denarium. Cujus habet imaginem et inscriptionem ? Respondentes dixerunt ei : Cæsaris. \EVERSE}
\newcommand{\lcXXvXXV}{\VERSE  Et ait illis : Reddite ergo quæ sunt Cæsaris, Cæsari : et quæ sunt Dei, Deo. \EVERSE}
\newcommand{\lcXXvXXVI}{\VERSE  Et non potuerunt verbum ejus reprehendere coram plebe : et mirati in responso ejus, tacuerunt. \EVERSE}
\newcommand{\lcXXvXXVII}{\VERSE  Accesserunt autem quidam sadducæorum, qui negant esse resurrectionem, et interrogaverunt eum, \EVERSE}
\newcommand{\lcXXvXXVIII}{\VERSE  dicentes : Magister, Moyses scripsit nobis : Si frater alicujus mortuus fuerit habens uxorem, et hic sine liberis fuerit, ut accipiat eam frater ejus uxorem, et suscitet semen fratri suo. \EVERSE}
\newcommand{\lcXXvXXIX}{\VERSE  Septem ergo fratres erant : et primus accepit uxorem, et mortuus est sine filiis. \EVERSE}
\newcommand{\lcXXvXXX}{\VERSE  Et sequens accepit illam, et ipse mortuus est sine filio. \EVERSE}
\newcommand{\lcXXvXXXI}{\VERSE  Et tertius accepit illam. Similiter et omnes septem, et non reliquerunt semen, et mortui sunt. \EVERSE}
\newcommand{\lcXXvXXXII}{\VERSE  Novissime omnium mortua est et mulier. \EVERSE}
\newcommand{\lcXXvXXXIII}{\VERSE  In resurrectione ergo, cujus eorum erit uxor ? siquidem septem habuerunt eam uxorem. \EVERSE}
\newcommand{\lcXXvXXXIV}{\VERSE  Et ait illis Jesus : Filii hujus sæculi nubunt, et traduntur ad nuptias : \EVERSE}
\newcommand{\lcXXvXXXV}{\VERSE  illi vero qui digni habebuntur sæculo illo, et resurrectione ex mortuis, neque nubent, neque ducent uxores : \EVERSE}
\newcommand{\lcXXvXXXVI}{\VERSE  neque enim ultra mori potuerunt : æquales enim angelis sunt, et filii sunt Dei, cum sint filii resurrectionis. \EVERSE}
\newcommand{\lcXXvXXXVII}{\VERSE  Quia vero resurgant mortui, et Moyses ostendit secus rubum, sicut dicit Dominum, Deum Abraham, et Deum Isaac, et Deum Jacob. \EVERSE}
\newcommand{\lcXXvXXXVIII}{\VERSE  Deus autem non est mortuorum, sed vivorum : omnes enim vivunt ei. \EVERSE}
\newcommand{\lcXXvXXXIX}{\VERSE  Respondentes autem quidam scribarum, dixerunt ei : Magister, bene dixisti. \EVERSE}
\newcommand{\lcXXvXL}{\VERSE  Et amplius non audebant eum quidquam interrogare. \EVERSE}
\newcommand{\lcXXvXLI}{\VERSE  Dixit autem ad illos : Quomodo dicunt Christum filium esse David ? \EVERSE}
\newcommand{\lcXXvXLII}{\VERSE  et ipse David dicit in libro Psalmorum : Dixit Dominus Domino meo : sede a dextris meis, \EVERSE}
\newcommand{\lcXXvXLIII}{\VERSE  donec ponam inimicos tuos scabellum pedum tuorum. \EVERSE}
\newcommand{\lcXXvXLIV}{\VERSE  David ergo Dominum illum vocat : et quomodo filius ejus est ? \EVERSE}
\newcommand{\lcXXvXLV}{\VERSE  Audiente autem omni populo, dixit discipulis suis : \EVERSE}
\newcommand{\lcXXvXLVI}{\VERSE  Attendite a scribis, qui volunt ambulare in stolis, et amant salutationes in foro, et primas cathedras in synagogis, et primos discubitus in conviviis, \EVERSE}
\newcommand{\lcXXvXLVII}{\VERSE  qui devorant domos viduarum, simulantes longam orationem : hi accipient damnationem majorem. \EVERSE}
\newcommand{\lcXXIvI}{\VERSE  Respiciens autem, vidit eos qui mittebant munera sua in gazophylacium, divites. \EVERSE}
\newcommand{\lcXXIvII}{\VERSE  Vidit autem et quamdam viduam pauperculam mittentem æra minuta duo. \EVERSE}
\newcommand{\lcXXIvIII}{\VERSE  Et dixit : Vere dico vobis, quia vidua hæc pauper plus quam omnes misit. \EVERSE}
\newcommand{\lcXXIvIV}{\VERSE  Nam omnes hi ex abundanti sibi miserunt in munera Dei : hæc autem ex eo quod deest illi, omnem victum suum quem habuit, misit. \EVERSE}
\newcommand{\lcXXIvV}{\VERSE  Et quibusdam dicentibus de templo quod bonis lapidibus et donis ornatum esset, dixit : \EVERSE}
\newcommand{\lcXXIvVI}{\VERSE  Hæc quæ videtis, venient dies in quibus non relinquetur lapis super lapidem, qui non destruatur. \EVERSE}
\newcommand{\lcXXIvVII}{\VERSE  Interrogaverunt autem illum, dicentes : Præceptor, quando hæc erunt, et quod signum cum fieri incipient ? \EVERSE}
\newcommand{\lcXXIvVIII}{\VERSE  Qui dixit : Videte ne seducamini : multi enim venient in nomine meo, dicentes quia ego sum : et tempus appropinquavit : nolite ergo ire post eos. \EVERSE}
\newcommand{\lcXXIvIX}{\VERSE  Cum autem audieritis prælia et seditiones, nolite terreri : oportet primum hæc fieri, sed nondum statim finis. \EVERSE}
\newcommand{\lcXXIvX}{\VERSE  Tunc dicebat illis : Surget gens contra gentem, et regnum adversus regnum. \EVERSE}
\newcommand{\lcXXIvXI}{\VERSE  Et terræmotus magni erunt per loca, et pestilentiæ, et fames, terroresque de cælo, et signa magna erunt. \EVERSE}
\newcommand{\lcXXIvXII}{\VERSE  Sed ante hæc omnia injicient vobis manus suas, et persequentur tradentes in synagogas et custodias, trahentes ad reges et præsides propter nomen meum : \EVERSE}
\newcommand{\lcXXIvXIII}{\VERSE  continget autem vobis in testimonium. \EVERSE}
\newcommand{\lcXXIvXIV}{\VERSE  Ponite ergo in cordibus vestris non præmeditari quemadmodum respondeatis : \EVERSE}
\newcommand{\lcXXIvXV}{\VERSE  ego enim dabo vobis os et sapientiam, cui non poterunt resistere et contradicere omnes adversarii vestri. \EVERSE}
\newcommand{\lcXXIvXVI}{\VERSE  Trademini autem a parentibus, et fratribus, et cognatis, et amicis, et morte afficient ex vobis : \EVERSE}
\newcommand{\lcXXIvXVII}{\VERSE  et eritis odio omnibus propter nomen meum : \EVERSE}
\newcommand{\lcXXIvXVIII}{\VERSE  et capillus de capite vestro non peribit. \EVERSE}
\newcommand{\lcXXIvXIX}{\VERSE  In patientia vestra possidebitis animas vestras. \EVERSE}
\newcommand{\lcXXIvXX}{\VERSE  Cum autem videritis circumdari ab exercitu Jerusalem, tunc scitote quia appropinquavit desolatio ejus : \EVERSE}
\newcommand{\lcXXIvXXI}{\VERSE  tunc qui in Judæa sunt, fugiant ad montes, et qui in medio ejus, discedant : et qui in regionibus, non intrent in eam, \EVERSE}
\newcommand{\lcXXIvXXII}{\VERSE  quia dies ultionis hi sunt, ut impleantur omnia quæ scripta sunt. \EVERSE}
\newcommand{\lcXXIvXXIII}{\VERSE  Væ autem prægnantibus et nutrientibus in illis diebus ! erit enim pressura magna super terram, et ira populo huic. \EVERSE}
\newcommand{\lcXXIvXXIV}{\VERSE  Et cadent in ore gladii, et captivi ducentur in omnes gentes, et Jerusalem calcabitur a gentibus, donec impleantur tempora nationum. \EVERSE}
\newcommand{\lcXXIvXXV}{\VERSE  Et erunt signa in sole, et luna, et stellis, et in terris pressura gentium præ confusione sonitus maris, et fluctuum : \EVERSE}
\newcommand{\lcXXIvXXVI}{\VERSE  arescentibus hominibus præ timore, et exspectatione, quæ supervenient universo orbi : nam virtutes cælorum movebuntur : \EVERSE}
\newcommand{\lcXXIvXXVII}{\VERSE  et tunc videbunt Filium hominis venientem in nube cum potestate magna et majestate. \EVERSE}
\newcommand{\lcXXIvXXVIII}{\VERSE  His autem fieri incipientibus, respicite, et levate capita vestra : quoniam appropinquat redemptio vestra. \EVERSE}
\newcommand{\lcXXIvXXIX}{\VERSE  Et dixit illis similitudinem : Videte ficulneam, et omnes arbores : \EVERSE}
\newcommand{\lcXXIvXXX}{\VERSE  cum producunt jam ex se fructum, scitis quoniam prope est æstas. \EVERSE}
\newcommand{\lcXXIvXXXI}{\VERSE  Ita et vos cum videritis hæc fieri, scitote quoniam prope est regnum Dei. \EVERSE}
\newcommand{\lcXXIvXXXII}{\VERSE  Amen dico vobis, quia non præteribit generatio hæc, donec omnia fiant. \EVERSE}
\newcommand{\lcXXIvXXXIII}{\VERSE  Cælum et terra transibunt : verba autem mea non transibunt. \EVERSE}
\newcommand{\lcXXIvXXXIV}{\VERSE  Attendite autem vobis, ne forte graventur corda vestra in crapula, et ebrietate, et curis hujus vitæ, et superveniat in vos repentina dies illa : \EVERSE}
\newcommand{\lcXXIvXXXV}{\VERSE  tamquam laqueus enim superveniet in omnes qui sedent super faciem omnis terræ. \EVERSE}
\newcommand{\lcXXIvXXXVI}{\VERSE  Vigilate itaque, omni tempore orantes, ut digni habeamini fugere ista omnia quæ futura sunt, et stare ante Filium hominis. \EVERSE}
\newcommand{\lcXXIvXXXVII}{\VERSE  Erat autem diebus docens in templo : noctibus vero exiens, morabatur in monte qui vocatur Oliveti. \EVERSE}
\newcommand{\lcXXIvXXXVIII}{\VERSE  Et omnis populus manicabat ad eum in templo audire eum. \EVERSE}
\newcommand{\lcXXIIvI}{\VERSE  Appropinquabat autem dies festus azymorum, qui dicitur Pascha : \EVERSE}
\newcommand{\lcXXIIvII}{\VERSE  et quærebant principes sacerdotum, et scribæ, quomodo Jesum interficerent : timebant vero plebem. \EVERSE}
\newcommand{\lcXXIIvIII}{\VERSE  Intravit autem Satanas in Judam, qui cognominabatur Iscariotes, unum de duodecim : \EVERSE}
\newcommand{\lcXXIIvIV}{\VERSE  et abiit, et locutus est cum principibus sacerdotum, et magistratibus, quemadmodum illum traderet eis. \EVERSE}
\newcommand{\lcXXIIvV}{\VERSE  Et gavisi sunt, et pacti sunt pecuniam illi dare. \EVERSE}
\newcommand{\lcXXIIvVI}{\VERSE  Et spopondit, et quærebat opportunitatem ut traderet illum sine turbis. \EVERSE}
\newcommand{\lcXXIIvVII}{\VERSE  Venit autem dies azymorum, in qua necesse erat occidi pascha. \EVERSE}
\newcommand{\lcXXIIvVIII}{\VERSE  Et misit Petrum et Joannem, dicens : Euntes parate nobis pascha, ut manducemus. \EVERSE}
\newcommand{\lcXXIIvIX}{\VERSE  At illi dixerunt : Ubi vis paremus ? \EVERSE}
\newcommand{\lcXXIIvX}{\VERSE  Et dixit ad eos : Ecce introëuntibus vobis in civitatem occurret vobis homo quidam amphoram aquæ portans : sequimini eum in domum, in quam intrat, \EVERSE}
\newcommand{\lcXXIIvXI}{\VERSE  et dicetis patrifamilias domus : Dicit tibi Magister : Ubi est diversorium, ubi pascha cum discipulis meis manducem ? \EVERSE}
\newcommand{\lcXXIIvXII}{\VERSE  Et ipse ostendet vobis cœnaculum magnum stratum, et ibi parate. \EVERSE}
\newcommand{\lcXXIIvXIII}{\VERSE  Euntes autem invenerunt sicut dixit illis, et paraverunt pascha. \EVERSE}
\newcommand{\lcXXIIvXIV}{\VERSE  Et cum facta esset hora, discubuit, et duodecim apostoli cum eo. \EVERSE}
\newcommand{\lcXXIIvXV}{\VERSE  Et ait illis : Desiderio desideravi hoc pascha manducare vobiscum, antequam patiar. \EVERSE}
\newcommand{\lcXXIIvXVI}{\VERSE  Dico enim vobis, quia ex hoc non manducabo illud, donec impleatur in regno Dei. \EVERSE}
\newcommand{\lcXXIIvXVII}{\VERSE  Et accepto calice gratias egit, et dixit : Accipite, et dividite inter vos. \EVERSE}
\newcommand{\lcXXIIvXVIII}{\VERSE  Dico enim vobis quod non bibam de generatione vitis donec regnum Dei veniat. \EVERSE}
\newcommand{\lcXXIIvXIX}{\VERSE  Et accepto pane gratias egit, et fregit, et dedit eis, dicens : Hoc est corpus meum, quod pro vobis datur : hoc facite in meam commemorationem. \EVERSE}
\newcommand{\lcXXIIvXX}{\VERSE  Similiter et calicem, postquam cœnavit, dicens : Hic est calix novum testamentum in sanguine meo, qui pro vobis fundetur. \EVERSE}
\newcommand{\lcXXIIvXXI}{\VERSE  Verumtamen ecce manus tradentis me, mecum est in mensa. \EVERSE}
\newcommand{\lcXXIIvXXII}{\VERSE  Et quidem Filius hominis, secundum quod definitum est, vadit : verumtamen væ homini illi per quem tradetur. \EVERSE}
\newcommand{\lcXXIIvXXIII}{\VERSE  Et ipsi cœperunt quærere inter se quis esset ex eis qui hoc facturus esset. \EVERSE}
\newcommand{\lcXXIIvXXIV}{\VERSE  Facta est autem et contentio inter eos, quis eorum videretur esse major. \EVERSE}
\newcommand{\lcXXIIvXXV}{\VERSE  Dixit autem eis : Reges gentium dominantur eorum : et qui potestatem habent super eos, benefici vocantur. \EVERSE}
\newcommand{\lcXXIIvXXVI}{\VERSE  Vos autem non sic : sed qui major est in vobis, fiat sicut minor : et qui præcessor est, sicut ministrator. \EVERSE}
\newcommand{\lcXXIIvXXVII}{\VERSE  Nam quis major est, qui recumbit, an qui ministrat ? nonne qui recumbit ? Ego autem in medio vestrum sum, sicut qui ministrat : \EVERSE}
\newcommand{\lcXXIIvXXVIII}{\VERSE  vos autem estis, qui permansistis mecum in tentationibus meis. \EVERSE}
\newcommand{\lcXXIIvXXIX}{\VERSE  Et ego dispono vobis sicut disposuit mihi Pater meus regnum, \EVERSE}
\newcommand{\lcXXIIvXXX}{\VERSE  ut edatis et bibatis super mensam meam in regno meo, et sedeatis super thronos judicantes duodecim tribus Israël. \EVERSE}
\newcommand{\lcXXIIvXXXI}{\VERSE  Ait autem Dominus : Simon, Simon, ecce Satanas expetivit vos ut cribraret sicut triticum : \EVERSE}
\newcommand{\lcXXIIvXXXII}{\VERSE  ego autem rogavi pro te ut non deficiat fides tua : et tu aliquando conversus, confirma fratres tuos. \EVERSE}
\newcommand{\lcXXIIvXXXIII}{\VERSE  Qui dixit ei : Domine, tecum paratus sum et in carcerem et in mortem ire. \EVERSE}
\newcommand{\lcXXIIvXXXIV}{\VERSE  At ille dixit : Dico tibi, Petre, non cantabit hodie gallus, donec ter abneges nosse me. Et dixit eis : \EVERSE}
\newcommand{\lcXXIIvXXXV}{\VERSE  Quando misi vos sine sacculo, et pera, et calceamentis, numquid aliquid defuit vobis ? \EVERSE}
\newcommand{\lcXXIIvXXXVI}{\VERSE  At illi dixerunt : Nihil. Dixit ergo eis : Sed nunc qui habet sacculum, tollat ; similiter et peram : et qui non habet, vendat tunicam suam et emat gladium. \EVERSE}
\newcommand{\lcXXIIvXXXVII}{\VERSE  Dico enim vobis, quoniam adhuc hoc quod scriptum est, oportet impleri in me : Et cum iniquis deputatus est. Etenim ea quæ sunt de me finem habent. \EVERSE}
\newcommand{\lcXXIIvXXXVIII}{\VERSE  At illi dixerunt : Domine, ecce duo gladii hic. At ille dixit eis : Satis est. \EVERSE}
\newcommand{\lcXXIIvXXXIX}{\VERSE  Et egressus ibat secundum consuetudinem in monte Olivarum. Secuti sunt autem illum et discipuli. \EVERSE}
\newcommand{\lcXXIIvXL}{\VERSE  Et cum pervenisset ad locum, dixit illis : Orate ne intretis in tentationem. \EVERSE}
\newcommand{\lcXXIIvXLI}{\VERSE  Et ipse avulsus est ab eis quantum jactus est lapidis : et positis genibus orabat, \EVERSE}
\newcommand{\lcXXIIvXLII}{\VERSE  dicens : Pater, si vis, transfer calicem istum a me : verumtamen non mea voluntas, sed tua fiat. \EVERSE}
\newcommand{\lcXXIIvXLIII}{\VERSE  Apparuit autem illi angelus de cælo, confortans eum. Et factus in agonia, prolixius orabat. \EVERSE}
\newcommand{\lcXXIIvXLIV}{\VERSE  Et factus est sudor ejus sicut guttæ sanguinis decurrentis in terram. \EVERSE}
\newcommand{\lcXXIIvXLV}{\VERSE  Et cum surrexisset ab oratione et venisset ad discipulos suos, invenit eos dormientes præ tristitia. \EVERSE}
\newcommand{\lcXXIIvXLVI}{\VERSE  Et ait illis : Quid dormitis ? surgite, orate, ne intretis in tentationem. \EVERSE}
\newcommand{\lcXXIIvXLVII}{\VERSE  Adhuc eo loquente, ecce turba : et qui vocabatur Judas, unus de duodecim, antecedebat eos, et appropinquavit Jesu ut oscularetur eum. \EVERSE}
\newcommand{\lcXXIIvXLVIII}{\VERSE  Jesus autem dixit illi : Juda, osculo Filium hominis tradis ? \EVERSE}
\newcommand{\lcXXIIvXLIX}{\VERSE  Videntes autem hi qui circa ipsum erant, quod futurum erat, dixerunt ei : Domine, si percutimus in gladio ? \EVERSE}
\newcommand{\lcXXIIvL}{\VERSE  Et percussit unus ex illis servum principis sacerdotum, et amputavit auriculam ejus dexteram. \EVERSE}
\newcommand{\lcXXIIvLI}{\VERSE  Respondens autem Jesus, ait : Sinite usque huc. Et cum tetigisset auriculam ejus, sanavit eum. \EVERSE}
\newcommand{\lcXXIIvLII}{\VERSE  Dixit autem Jesus ad eos qui venerant ad se principes sacerdotum, et magistratus templi, et seniores : Quasi ad latronem existis cum gladiis et fustibus ? \EVERSE}
\newcommand{\lcXXIIvLIII}{\VERSE  Cum quotidie vobiscum fuerim in templo, non extendistis manus in me : sed hæc est hora vestra, et potestas tenebrarum. \EVERSE}
\newcommand{\lcXXIIvLIV}{\VERSE  Comprehendentes autem eum, duxerunt ad domum principis sacerdotum : Petrus vero sequebatur a longe. \EVERSE}
\newcommand{\lcXXIIvLV}{\VERSE  Accenso autem igne in medio atrii et circumsedentibus illis, erat Petrus in medio eorum. \EVERSE}
\newcommand{\lcXXIIvLVI}{\VERSE  Quem cum vidisset ancilla quædam sedentem ad lumen, et eum fuisset intuita, dixit : Et hic cum illo erat. \EVERSE}
\newcommand{\lcXXIIvLVII}{\VERSE  At ille negavit eum, dicens : Mulier, non novi illum. \EVERSE}
\newcommand{\lcXXIIvLVIII}{\VERSE  Et post pusillum alius videns eum, dixit : Et tu de illis es. Petrus vero ait : O homo, non sum. \EVERSE}
\newcommand{\lcXXIIvLIX}{\VERSE  Et intervallo facto quasi horæ unius, alius quidam affirmabat, dicens : Vere et hic cum illo erat : nam et Galilæus est. \EVERSE}
\newcommand{\lcXXIIvLX}{\VERSE  Et ait Petrus : Homo, nescio quid dicis. Et continuo, adhuc illo loquente, cantavit gallus. \EVERSE}
\newcommand{\lcXXIIvLXI}{\VERSE  Et conversus Dominus respexit Petrum, et recordatus est Petrus verbi Domini, sicut dixerat : Quia priusquam gallus cantet, ter me negabis. \EVERSE}
\newcommand{\lcXXIIvLXII}{\VERSE  Et egressus foras Petrus flevit amare. \EVERSE}
\newcommand{\lcXXIIvLXIII}{\VERSE  Et viri qui tenebant illum, illudebant ei, cædentes. \EVERSE}
\newcommand{\lcXXIIvLXIV}{\VERSE  Et velaverunt eum, et percutiebant faciem ejus : et interrogabant eum, dicentes : Prophetiza, quis est, qui te percussit ? \EVERSE}
\newcommand{\lcXXIIvLXV}{\VERSE  Et alia multa blasphemantes dicebant in eum. \EVERSE}
\newcommand{\lcXXIIvLXVI}{\VERSE  Et ut factus est dies, convenerunt seniores plebis, et principes sacerdotum, et scribæ, et duxerunt illum in concilium suum, dicentes : Si tu es Christus, dic nobis. \EVERSE}
\newcommand{\lcXXIIvLXVII}{\VERSE  Et ait illis : Si vobis dixero, non credetis mihi : \EVERSE}
\newcommand{\lcXXIIvLXVIII}{\VERSE  si autem et interrogavero, non respondebitis mihi, neque dimittetis. \EVERSE}
\newcommand{\lcXXIIvLXIX}{\VERSE  Ex hoc autem erit Filius hominis sedens a dextris virtutis Dei. \EVERSE}
\newcommand{\lcXXIIvLXX}{\VERSE  Dixerunt autem omnes : Tu ergo es Filius Dei ? Qui ait : Vos dicitis, quia ego sum. \EVERSE}
\newcommand{\lcXXIIvLXXI}{\VERSE  At illi dixerunt : Quid adhuc desideramus testimonium ? ipsi enim audivimus de ore ejus. \EVERSE}
\newcommand{\lcXXIIIvI}{\VERSE  Et surgens omnis multitudo eorum, duxerunt illum ad Pilatum. \EVERSE}
\newcommand{\lcXXIIIvII}{\VERSE  Cœperunt autem illum accusare, dicentes : Hunc invenimus subvertentem gentem nostram, et prohibentem tributa dare Cæsari, et dicentem se Christum regem esse. \EVERSE}
\newcommand{\lcXXIIIvIII}{\VERSE  Pilatus autem interrogavit eum, dicens : Tu es rex Judæorum ? At ille respondens ait : Tu dicis. \EVERSE}
\newcommand{\lcXXIIIvIV}{\VERSE  Ait autem Pilatus ad principes sacerdotum et turbas : Nihil invenio causæ in hoc homine. \EVERSE}
\newcommand{\lcXXIIIvV}{\VERSE  At illi invalescebant, dicentes : Commovet populum docens per universam Judæam, incipiens a Galilæa usque huc. \EVERSE}
\newcommand{\lcXXIIIvVI}{\VERSE  Pilatus autem audiens Galilæam, interrogavit si homo Galilæus esset. \EVERSE}
\newcommand{\lcXXIIIvVII}{\VERSE  Et ut cognovit quod de Herodis potestate esset, remisit eum ad Herodem, qui et ipse Jerosolymis erat illis diebus. \EVERSE}
\newcommand{\lcXXIIIvVIII}{\VERSE  Herodes autem viso Jesu, gavisus est valde. Erat enim cupiens ex multo tempore videre eum, eo quod audierat multa de eo, et sperabat signum aliquod videre ab eo fieri. \EVERSE}
\newcommand{\lcXXIIIvIX}{\VERSE  Interrogabat autem eum multis sermonibus. At ipse nihil illi respondebat. \EVERSE}
\newcommand{\lcXXIIIvX}{\VERSE  Stabant autem principes sacerdotum et scribæ constanter accusantes eum. \EVERSE}
\newcommand{\lcXXIIIvXI}{\VERSE  Sprevit autem illum Herodes cum exercitu suo : et illusit indutum veste alba, et remisit ad Pilatum. \EVERSE}
\newcommand{\lcXXIIIvXII}{\VERSE  Et facti sunt amici Herodes et Pilatus in ipsa die : nam antea inimici erant ad invicem. \EVERSE}
\newcommand{\lcXXIIIvXIII}{\VERSE  Pilatus autem, convocatis principibus sacerdotum, et magistratibus, et plebe, \EVERSE}
\newcommand{\lcXXIIIvXIV}{\VERSE  dixit ad illos : Obtulistis mihi hunc hominem, quasi avertentem populum, et ecce ego coram vobis interrogans, nullam causam inveni in homine isto ex his in quibus eum accusatis. \EVERSE}
\newcommand{\lcXXIIIvXV}{\VERSE  Sed neque Herodes : nam remisi vos ad illum, et ecce nihil dignum morte actum est ei. \EVERSE}
\newcommand{\lcXXIIIvXVI}{\VERSE  Emendatum ergo illum dimittam. \EVERSE}
\newcommand{\lcXXIIIvXVII}{\VERSE  Necesse autem habebat dimittere eis per diem festum unum. \EVERSE}
\newcommand{\lcXXIIIvXVIII}{\VERSE  Exclamavit autem simul universa turba, dicens : Tolle hunc, et dimitte nobis Barabbam : \EVERSE}
\newcommand{\lcXXIIIvXIX}{\VERSE  qui erat propter seditionem quamdam factam in civitate et homicidium missus in carcerem. \EVERSE}
\newcommand{\lcXXIIIvXX}{\VERSE  Iterum autem Pilatus locutus est ad eos, volens dimittere Jesum. \EVERSE}
\newcommand{\lcXXIIIvXXI}{\VERSE  At illi succlamabant, dicentes : Crucifige, crucifige eum. \EVERSE}
\newcommand{\lcXXIIIvXXII}{\VERSE  Ille autem tertio dixit ad illos : Quid enim mali fecit iste ? nullam causam mortis invenio in eo : corripiam ergo illum et dimittam. \EVERSE}
\newcommand{\lcXXIIIvXXIII}{\VERSE  At illi instabant vocibus magnis postulantes ut crucifigeretur : et invalescebant voces eorum. \EVERSE}
\newcommand{\lcXXIIIvXXIV}{\VERSE  Et Pilatus adjudicavit fieri petitionem eorum. \EVERSE}
\newcommand{\lcXXIIIvXXV}{\VERSE  Dimisit autem illis eum qui propter homicidium et seditionem missus fuerat in carcerem, quem petebant : Jesum vero tradidit voluntati eorum. \EVERSE}
\newcommand{\lcXXIIIvXXVI}{\VERSE  Et cum ducerent eum, apprehenderunt Simonem quemdam Cyrenensem venientem de villa : et imposuerunt illi crucem portare post Jesum. \EVERSE}
\newcommand{\lcXXIIIvXXVII}{\VERSE  Sequebatur autem illum multa turba populi et mulierum, quæ plangebant et lamentabantur eum. \EVERSE}
\newcommand{\lcXXIIIvXXVIII}{\VERSE  Conversus autem ad illas Jesus, dixit : Filiæ Jerusalem, nolite flere super me, sed super vos ipsas flete et super filios vestros. \EVERSE}
\newcommand{\lcXXIIIvXXIX}{\VERSE  Quoniam ecce venient dies in quibus dicent : Beatæ steriles, et ventres qui non genuerunt, et ubera quæ non lactaverunt. \EVERSE}
\newcommand{\lcXXIIIvXXX}{\VERSE  Tunc incipient dicere montibus : Cadite super nos ; et collibus : Operite nos. \EVERSE}
\newcommand{\lcXXIIIvXXXI}{\VERSE  Quia si in viridi ligno hæc faciunt, in arido quid fiet ? \EVERSE}
\newcommand{\lcXXIIIvXXXII}{\VERSE  Ducebantur autem et alii duo nequam cum eo, ut interficerentur. \EVERSE}
\newcommand{\lcXXIIIvXXXIII}{\VERSE  Et postquam venerunt in locum qui vocatur Calvariæ, ibi crucifixerunt eum : et latrones, unum a dextris, et alterum a sinistris. \EVERSE}
\newcommand{\lcXXIIIvXXXIV}{\VERSE  Jesus autem dicebat : Pater, dimitte illis : non enim sciunt quid faciunt. Dividentes vero vestimenta ejus, miserunt sortes. \EVERSE}
\newcommand{\lcXXIIIvXXXV}{\VERSE  Et stabat populus spectans, et deridebant eum principes cum eis, dicentes : Alios salvos fecit, se salvum faciat, si hic est Christus Dei electus. \EVERSE}
\newcommand{\lcXXIIIvXXXVI}{\VERSE  Illudebant autem ei et milites accedentes, et acetum offerentes ei, \EVERSE}
\newcommand{\lcXXIIIvXXXVII}{\VERSE  et dicentes : Si tu es rex Judæorum, salvum te fac. \EVERSE}
\newcommand{\lcXXIIIvXXXVIII}{\VERSE  Erat autem et superscriptio scripta super eum litteris græcis, et latinis, et hebraicis : Hic est rex Judæorum. \EVERSE}
\newcommand{\lcXXIIIvXXXIX}{\VERSE  Unus autem de his, qui pendebant, latronibus, blasphemabat eum, dicens : Si tu es Christus, salvum fac temetipsum et nos. \EVERSE}
\newcommand{\lcXXIIIvXL}{\VERSE  Respondens autem alter increpabat eum, dicens : Neque tu times Deum, quod in eadem damnatione es. \EVERSE}
\newcommand{\lcXXIIIvXLI}{\VERSE  Et nos quidem juste, nam digna factis recipimus : hic vero nihil mali gessit. \EVERSE}
\newcommand{\lcXXIIIvXLII}{\VERSE  Et dicebat ad Jesum : Domine, memento mei cum veneris in regnum tuum. \EVERSE}
\newcommand{\lcXXIIIvXLIII}{\VERSE  Et dixit illi Jesus : Amen dico tibi : hodie mecum eris in paradiso. \EVERSE}
\newcommand{\lcXXIIIvXLIV}{\VERSE  Erat autem fere hora sexta, et tenebræ factæ sunt in universam terram usque ad horam nonam. \EVERSE}
\newcommand{\lcXXIIIvXLV}{\VERSE  Et obscuratus est sol, et velum templi scissum est medium. \EVERSE}
\newcommand{\lcXXIIIvXLVI}{\VERSE  Et clamans voce magna Jesus ait : Pater, in manus tuas commendo spiritum meum. Et hæc dicens, expiravit. \EVERSE}
\newcommand{\lcXXIIIvXLVII}{\VERSE  Videns autem centurio quod factum fuerat, glorificavit Deum, dicens : Vere hic homo justus erat. \EVERSE}
\newcommand{\lcXXIIIvXLVIII}{\VERSE  Et omnis turba eorum, qui simul aderant ad spectaculum istud, et videbant quæ fiebant, percutientes pectora sua revertebantur. \EVERSE}
\newcommand{\lcXXIIIvXLIX}{\VERSE  Stabant autem omnes noti ejus a longe, et mulieres, quæ secutæ eum erant a Galilæa, hæc videntes. \EVERSE}
\newcommand{\lcXXIIIvL}{\VERSE  Et ecce vir nomine Joseph, qui erat decurio, vir bonus et justus : \EVERSE}
\newcommand{\lcXXIIIvLI}{\VERSE  hic non consenserat consilio, et actibus eorum : ab Arimathæa civitate Judææ, qui exspectabat et ipse regnum Dei : \EVERSE}
\newcommand{\lcXXIIIvLII}{\VERSE  hic accessit ad Pilatum et petiit corpus Jesu : \EVERSE}
\newcommand{\lcXXIIIvLIII}{\VERSE  et depositum involvit sindone, et posuit eum in monumento exciso, in quo nondum quisquam positus fuerat. \EVERSE}
\newcommand{\lcXXIIIvLIV}{\VERSE  Et dies erat parasceves, et sabbatum illucescebat. \EVERSE}
\newcommand{\lcXXIIIvLV}{\VERSE  Subsecutæ autem mulieres, quæ cum eo venerant de Galilæa, viderunt monumentum, et quemadmodum positum erat corpus ejus. \EVERSE}
\newcommand{\lcXXIIIvLVI}{\VERSE  Et revertentes paraverunt aromata, et unguenta : et sabbato quidem siluerunt secundum mandatum. \EVERSE}
\newcommand{\lcXXIVvI}{\VERSE  Una autem sabbati valde diluculo venerunt ad monumentum, portantes quæ paraverant aromata : \EVERSE}
\newcommand{\lcXXIVvII}{\VERSE  et invenerunt lapidem revolutum a monumento. \EVERSE}
\newcommand{\lcXXIVvIII}{\VERSE  Et ingressæ non invenerunt corpus Domini Jesu. \EVERSE}
\newcommand{\lcXXIVvIV}{\VERSE  Et factum est, dum mente consternatæ essent de isto, ecce duo viri steterunt secus illas in veste fulgenti. \EVERSE}
\newcommand{\lcXXIVvV}{\VERSE  Cum timerent autem, et declinarent vultum in terram, dixerunt ad illas : Quid quæritis viventem cum mortuis ? \EVERSE}
\newcommand{\lcXXIVvVI}{\VERSE  non est hic, sed surrexit : recordamini qualiter locutus est vobis, cum adhuc in Galilæa esset, \EVERSE}
\newcommand{\lcXXIVvVII}{\VERSE  dicens : Quia oportet Filium hominis tradi in manus hominum peccatorum, et crucifigi, et die tertia resurgere. \EVERSE}
\newcommand{\lcXXIVvVIII}{\VERSE  Et recordatæ sunt verborum ejus. \EVERSE}
\newcommand{\lcXXIVvIX}{\VERSE  Et regressæ a monumento nuntiaverunt hæc omnia illis undecim, et ceteris omnibus. \EVERSE}
\newcommand{\lcXXIVvX}{\VERSE  Erat autem Maria Magdalene, et Joanna, et Maria Jacobi, et ceteræ quæ cum eis erant, quæ dicebant ad apostolos hæc. \EVERSE}
\newcommand{\lcXXIVvXI}{\VERSE  Et visa sunt ante illos sicut deliramentum verba ista, et non crediderunt illis. \EVERSE}
\newcommand{\lcXXIVvXII}{\VERSE  Petrus autem surgens cucurrit ad monumentum : et procumbens vidit linteamina sola posita, et abiit secum mirans quod factum fuerat. \EVERSE}
\newcommand{\lcXXIVvXIII}{\VERSE  Et ecce duo ex illis ibant ipsa die in castellum, quod erat in spatio stadiorum sexaginta ab Jerusalem, nomine Emmaus. \EVERSE}
\newcommand{\lcXXIVvXIV}{\VERSE  Et ipsi loquebantur ad invicem de his omnibus quæ acciderant. \EVERSE}
\newcommand{\lcXXIVvXV}{\VERSE  Et factum est, dum fabularentur, et secum quærerent : et ipse Jesus appropinquans ibat cum illis : \EVERSE}
\newcommand{\lcXXIVvXVI}{\VERSE  oculi autem illorum tenebantur ne eum agnoscerent. \EVERSE}
\newcommand{\lcXXIVvXVII}{\VERSE  Et ait ad illos : Qui sunt hi sermones, quos confertis ad invicem ambulantes, et estis tristes ? \EVERSE}
\newcommand{\lcXXIVvXVIII}{\VERSE  Et respondens unus, cui nomen Cleophas, dixit ei : Tu solus peregrinus es in Jerusalem, et non cognovisti quæ facta sunt in illa his diebus ? \EVERSE}
\newcommand{\lcXXIVvXIX}{\VERSE  Quibus ille dixit : Quæ ? Et dixerunt : De Jesu Nazareno, qui fuit vir propheta, potens in opere et sermone coram Deo et omni populo : \EVERSE}
\newcommand{\lcXXIVvXX}{\VERSE  et quomodo eum tradiderunt summi sacerdotes et principes nostri in damnationem mortis, et crucifixerunt eum : \EVERSE}
\newcommand{\lcXXIVvXXI}{\VERSE  nos autem sperabamus quia ipse esset redempturus Israël : et nunc super hæc omnia, tertia dies est hodie quod hæc facta sunt. \EVERSE}
\newcommand{\lcXXIVvXXII}{\VERSE  Sed et mulieres quædam ex nostris terruerunt nos, quæ ante lucem fuerunt ad monumentum, \EVERSE}
\newcommand{\lcXXIVvXXIII}{\VERSE  et non invento corpore ejus, venerunt, dicentes se etiam visionem angelorum vidisse, qui dicunt eum vivere. \EVERSE}
\newcommand{\lcXXIVvXXIV}{\VERSE  Et abierunt quidam ex nostris ad monumentum : et ita invenerunt sicut mulieres dixerunt, ipsum vero non invenerunt. \EVERSE}
\newcommand{\lcXXIVvXXV}{\VERSE  Et ipse dixit ad eos : O stulti, et tardi corde ad credendum in omnibus quæ locuti sunt prophetæ ! \EVERSE}
\newcommand{\lcXXIVvXXVI}{\VERSE  Nonne hæc oportuit pati Christum, et ita intrare in gloriam suam ? \EVERSE}
\newcommand{\lcXXIVvXXVII}{\VERSE  Et incipiens a Moyse, et omnibus prophetis, interpretabatur illis in omnibus scripturis quæ de ipso erant. \EVERSE}
\newcommand{\lcXXIVvXXVIII}{\VERSE  Et appropinquaverunt castello quo ibant : et ipse se finxit longius ire. \EVERSE}
\newcommand{\lcXXIVvXXIX}{\VERSE  Et coëgerunt illum, dicentes : Mane nobiscum, quoniam advesperascit, et inclinata est jam dies. Et intravit cum illis. \EVERSE}
\newcommand{\lcXXIVvXXX}{\VERSE  Et factum est, dum recumberet cum eis, accepit panem, et benedixit, ac fregit, et porrigebat illis. \EVERSE}
\newcommand{\lcXXIVvXXXI}{\VERSE  Et aperti sunt oculi eorum, et cognoverunt eum : et ipse evanuit ex oculis eorum. \EVERSE}
\newcommand{\lcXXIVvXXXII}{\VERSE  Et dixerunt ad invicem : Nonne cor nostrum ardens erat in nobis dum loqueretur in via, et aperiret nobis Scripturas ? \EVERSE}
\newcommand{\lcXXIVvXXXIII}{\VERSE  Et surgentes eadem hora regressi sunt in Jerusalem : et invenerunt congregatos undecim, et eos qui cum illis erant, \EVERSE}
\newcommand{\lcXXIVvXXXIV}{\VERSE  dicentes : Quod surrexit Dominus vere, et apparuit Simoni. \EVERSE}
\newcommand{\lcXXIVvXXXV}{\VERSE  Et ipsi narrabant quæ gesta erant in via, et quomodo cognoverunt eum in fractione panis. \EVERSE}
\newcommand{\lcXXIVvXXXVI}{\VERSE  Dum autem hæc loquuntur, stetit Jesus in medio eorum, et dicit eis : Pax vobis : ego sum, nolite timere. \EVERSE}
\newcommand{\lcXXIVvXXXVII}{\VERSE  Conturbati vero et conterriti, existimabant se spiritum videre. \EVERSE}
\newcommand{\lcXXIVvXXXVIII}{\VERSE  Et dixit eis : Quid turbati estis, et cogitationes ascendunt in corda vestra ? \EVERSE}
\newcommand{\lcXXIVvXXXIX}{\VERSE  videte manus meas, et pedes, quia ego ipse sum ; palpate et videte, quia spiritus carnem et ossa non habet, sicut me videtis habere. \EVERSE}
\newcommand{\lcXXIVvXL}{\VERSE  Et cum hoc dixisset, ostendit eis manus et pedes. \EVERSE}
\newcommand{\lcXXIVvXLI}{\VERSE  Adhuc autem illis non credentibus, et mirantibus præ gaudio, dixit : Habetis hic aliquid quod manducetur ? \EVERSE}
\newcommand{\lcXXIVvXLII}{\VERSE  At illi obtulerunt ei partem piscis assi et favum mellis. \EVERSE}
\newcommand{\lcXXIVvXLIII}{\VERSE  Et cum manducasset coram eis, sumens reliquias dedit eis. \EVERSE}
\newcommand{\lcXXIVvXLIV}{\VERSE  Et dixit ad eos : Hæc sunt verba quæ locutus sum ad vos cum adhuc essem vobiscum, quoniam necesse est impleri omnia quæ scripta sunt in lege Moysi, et prophetis, et Psalmis de me. \EVERSE}
\newcommand{\lcXXIVvXLV}{\VERSE  Tunc aperuit illis sensum ut intelligerent Scripturas, \EVERSE}
\newcommand{\lcXXIVvXLVI}{\VERSE  et dixit eis : Quoniam sic scriptum est, et sic oportebat Christum pati, et resurgere a mortuis tertia die : \EVERSE}
\newcommand{\lcXXIVvXLVII}{\VERSE  et prædicari in nomine ejus pœnitentiam, et remissionem peccatorum in omnes gentes, incipientibus ab Jerosolyma. \EVERSE}
\newcommand{\lcXXIVvXLVIII}{\VERSE  Vos autem testes estis horum. \EVERSE}
\newcommand{\lcXXIVvXLIX}{\VERSE  Et ego mitto promissum Patris mei in vos ; vos autem sedete in civitate, quoadusque induamini virtute ex alto. \EVERSE}
\newcommand{\lcXXIVvL}{\VERSE  Eduxit autem eos foras in Bethaniam, et elevatis manibus suis benedixit eis. \EVERSE}
\newcommand{\lcXXIVvLI}{\VERSE  Et factum est, dum benediceret illis, recessit ab eis, et ferebatur in cælum. \EVERSE}
\newcommand{\lcXXIVvLII}{\VERSE  Et ipsi adorantes regressi sunt in Jerusalem cum gaudio magno : \EVERSE}
\newcommand{\lcXXIVvLIII}{\VERSE  et erant semper in templo, laudantes et benedicentes Deum. Amen. \EVERSE}

\newcommand{\lcIvIfr}{\VERSE  Plusieurs ayant entrepris d'écrire l'histoire des choses qui se sont accomplies parmi nous, \EVERSE}
\newcommand{\lcIvIIfr}{\VERSE  suivant ce que nous ont transmis ceux qui les ont vues eux-mêmes dès le commencement, et qui ont été les ministres de la parole, \EVERSE}
\newcommand{\lcIvIIIfr}{\VERSE  il m'a paru bon, à moi aussi, après m'être soigneusement informé de tout depuis l'origine, de te les exposer par écrit d'une manière suivie, excellent Théophile, \EVERSE}
\newcommand{\lcIvIVfr}{\VERSE  afin que tu reconnaisses la vérité des paroles que l'on t'a enseignées. \EVERSE}
\newcommand{\lcIvVfr}{\VERSE  Il y avait, aux jours d'Hérode, roi de Judée, un prêtre nommé Zacharie, de la classe d'Abia; et sa femme était d'entre les filles d'Aaron, et s'appelait Elisabeth. \EVERSE}
\newcommand{\lcIvVIfr}{\VERSE  Ils étaient tous deux justes devant Dieu, marchant sans reproche dans tous les commandements et tous les préceptes du Seigneur. \EVERSE}
\newcommand{\lcIvVIIfr}{\VERSE  Et ils n'avaient pas d'enfant, parce qu'Elisabeth était stérile, et qu'ils étaient tous deux avancés en âge. \EVERSE}
\newcommand{\lcIvVIIIfr}{\VERSE  Or il arriva, lorsqu'il accomplissait devant Dieu les fonctions du sacerdoce selon le rang de sa classe, \EVERSE}
\newcommand{\lcIvIXfr}{\VERSE  qu'il lui échut par le sort, d'après la coutume établie entre les prêtres, d'entrer dans le temple du Seigneur pour y offrir l'encens. \EVERSE}
\newcommand{\lcIvXfr}{\VERSE  Et toute la multitude du peuple était dehors, en prière, à l'heure de l'encens. \EVERSE}
\newcommand{\lcIvXIfr}{\VERSE  Et un Ange du Seigneur lui apparut, se tenant debout à droit de l'autel de l'encens. \EVERSE}
\newcommand{\lcIvXIIfr}{\VERSE  Zacharie fut troublé en le voyant, et la frayeur le saisit. \EVERSE}
\newcommand{\lcIvXIIIfr}{\VERSE  Mais l'Ange lui dit: Ne crains point, Zacharie, car ta prière a été exaucée, et ta femme Elisabeth t'enfantera un fils, auquel tu donneras le nom de Jean. \EVERSE}
\newcommand{\lcIvXIVfr}{\VERSE  Il sera pour toi un sujet de joie et d'allégresse, et beaucoup se réjouiront de sa naissance, \EVERSE}
\newcommand{\lcIvXVfr}{\VERSE  car il sera grand devant le Seigneur.  Il ne boira pas de vin ni de liqueur enivrante, et il sera rempli du Saint-Esprit dès le sein de sa mère; \EVERSE}
\newcommand{\lcIvXVIfr}{\VERSE  et il convertira un grand nombre des enfants d'Israël au Seigneur leur Dieu. \EVERSE}
\newcommand{\lcIvXVIIfr}{\VERSE  Et il marchera devant Lui dans l'esprit et la vertu d'Elie, pour ramener les coeurs des pères vers les enfants, et les incrédules à la prudence des justes, de manière à préparer au Seigneur un peuple parfait. \EVERSE}
\newcommand{\lcIvXVIIIfr}{\VERSE  Zacharie dit à l'Ange: A quoi connaîtrai-je cela? car je suis vieux, et ma femme est avancée en âge. \EVERSE}
\newcommand{\lcIvXIXfr}{\VERSE  Et l'Ange lui répondit: Je suis Gabriel, qui me tiens devant Dieu; et j'ai été envoyé pour te parler, et pour t'annoncer cette bonne nouvelle. \EVERSE}
\newcommand{\lcIvXXfr}{\VERSE  Et voici que tu seras muet, et que tu ne pourras plus parler, jusqu'au jour où ces choses arriveront, parce que tu n'as pas cru à mes paroles, qui s'accompliront en leur temps. \EVERSE}
\newcommand{\lcIvXXIfr}{\VERSE  Cependant de peuple attendait Zacharie, et on s'étonnait qu'il s'attardât dans le temple. \EVERSE}
\newcommand{\lcIvXXIIfr}{\VERSE  Mais, étant sorti, il ne pouvait leur parler; et ils comprirent qu'il avait eu une vision dans le temple.  Et lui, il leur faisait des signes, et il demeura muet. \EVERSE}
\newcommand{\lcIvXXIIIfr}{\VERSE  Lorsque les jours de son ministère furent écoulés, il s'en alla dans sa maison. \EVERSE}
\newcommand{\lcIvXXIVfr}{\VERSE  Quelque temps après, Elisabeth sa femme conçut; et elle se tenait cachée durant cinq mois, disant: \EVERSE}
\newcommand{\lcIvXXVfr}{\VERSE  Voilà ce que le Seigneur a fait pour moi aux jours où Il m'a regardée, afin de me délivrer de mon opprobre parmi les hommes. \EVERSE}
\newcommand{\lcIvXXVIfr}{\VERSE  Or, au sixième mois, l'Ange Gabriel fut envoyé de Dieu dans une ville de Galilée, appelée Nazareth, \EVERSE}
\newcommand{\lcIvXXVIIfr}{\VERSE  auprès d'une Vierge fiancée à un homme de la maison de David, nommé Joseph; et le nom de la Vierge était Marie. \EVERSE}
\newcommand{\lcIvXXVIIIfr}{\VERSE  L'Ange, étant entré auprès d'Elle, Lui dit: Je Vous salue, pleine de grâce; le Seigneur est avec Vous, Vous êtes bénie entre les femmes. \EVERSE}
\newcommand{\lcIvXXIXfr}{\VERSE  Elle, l'ayant entendu, fut troublée de ses paroles, et Elle se demandait quelle pouvait être cette salutation. \EVERSE}
\newcommand{\lcIvXXXfr}{\VERSE  Et l'Ange Lui dit: Ne craignez point, Marie, car Vous avez trouvé grâce devant Dieu. \EVERSE}
\newcommand{\lcIvXXXIfr}{\VERSE  Voici que Vous concevrez dans Votre sein, et Vous enfanterez un fils, et Vous lui donnerez le nom de Jésus. \EVERSE}
\newcommand{\lcIvXXXIIfr}{\VERSE  Il sera grand, et sera appelé le Fils du Très-Haut; et le Seigneur Dieu Lui donnera le trône de David Son père, et Il régnera èternellement sur la maison de Jacob, \EVERSE}
\newcommand{\lcIvXXXIIIfr}{\VERSE  et Son règne n'aura pas de fin. \EVERSE}
\newcommand{\lcIvXXXIVfr}{\VERSE  Alors Marie dit à l'Ange: Comment cela se fera-t-il? car Je ne connais point d'homme. \EVERSE}
\newcommand{\lcIvXXXVfr}{\VERSE  L'Ange Lui répondit: L'Esprit-Saint surviendra en Vous, et la vertu du Très-Haut Vous couvrira de Son ombre; c'est pourquoi le fruit saint qui naîtra de Vous sera appelé le Fils de Dieu. \EVERSE}
\newcommand{\lcIvXXXVIfr}{\VERSE  Et voici qu'Elisabeth, Votre parente, a conçu, elle aussi, un fils dans sa vieillesse, et ce mois est le sixième de celle qui est appelée stérile; \EVERSE}
\newcommand{\lcIvXXXVIIfr}{\VERSE  car il n'y a rien d'impossible à Dieu. \EVERSE}
\newcommand{\lcIvXXXVIIIfr}{\VERSE  Et Marie dit: Voici la servante du Seigneur; qu'il Me soit fait selon votre parole.  Et l'Ange s'éloigna d'Elle. \EVERSE}
\newcommand{\lcIvXXXIXfr}{\VERSE  En ces jours-là, Marie, Se levant, S'en alla en grande hâte vers les montagnes, dans une ville de Juda; \EVERSE}
\newcommand{\lcIvXLfr}{\VERSE  et Elle entra dans la maison de Zacharie, et salua Elisabeth. \EVERSE}
\newcommand{\lcIvXLIfr}{\VERSE  Et il arriva, aussitôt qu'Elisabeth eut entendu la salutation de Marie, que l'enfant tressaillit dans son sein; et Elisabeth fut remplie du Saint-Esprit. \EVERSE}
\newcommand{\lcIvXLIIfr}{\VERSE  Et elle s'écria d'une voix forte: Vous êtes bénie entre les femmes, et le fruit de Votre sein est béni. \EVERSE}
\newcommand{\lcIvXLIIIfr}{\VERSE  Et d'où m'est-il accordé que la Mère de mon Seigneur vienne à moi? \EVERSE}
\newcommand{\lcIvXLIVfr}{\VERSE  Car voici, dès que Votre voix a frappé mon oreille, quand Vous m'avez saluée, l'enfant a tressailli de joie dans mon sein. \EVERSE}
\newcommand{\lcIvXLVfr}{\VERSE  Et Vous êtes bienheureuse d'avoir cru; car ce qui Vous a été dit de la part du Seigneur s'accomplira. \EVERSE}
\newcommand{\lcIvXLVIfr}{\VERSE  Et Marie dit: Mon âme glorifie le Seigneur, \EVERSE}
\newcommand{\lcIvXLVIIfr}{\VERSE  et Mon esprit a tressailli d'allégresse en Dieu Mon Sauveur, \EVERSE}
\newcommand{\lcIvXLVIIIfr}{\VERSE  parce qu'Il a jeté les yeux sur la bassesse de Sa servante.  Car voici que, désormais, toutes les générations me diront bienheureuse, \EVERSE}
\newcommand{\lcIvXLIXfr}{\VERSE  parce que Celui qui est puissant a fait en Moi de grandes choses, et Son nom est saint; \EVERSE}
\newcommand{\lcIvLfr}{\VERSE  et Sa miséricorde se répand d'âge en âge sur ceux qui Le craignent. \EVERSE}
\newcommand{\lcIvLIfr}{\VERSE  Il a déployé la force de Son bras, Il a dispersé ceux qui s'enorgueillissaient dans les pensées de leur coeur. \EVERSE}
\newcommand{\lcIvLIIfr}{\VERSE  Il a renversé les puissants de leur trône, et Il a élevé les humbles. \EVERSE}
\newcommand{\lcIvLIIIfr}{\VERSE  Il a rempli de biens les affamés, et Il a renvoyé les riches les mains vides. \EVERSE}
\newcommand{\lcIvLIVfr}{\VERSE  Il a relevé Israël, Son serviteur, se souvenant de Sa miséricorde: \EVERSE}
\newcommand{\lcIvLVfr}{\VERSE  selon ce qu'Il avait dit à nos pères, à Abraham et à sa race pour toujours. \EVERSE}
\newcommand{\lcIvLVIfr}{\VERSE  Marie demeura environ trois mois avec Elisabeth; puis Elle S'en retourna dans Sa maison. \EVERSE}
\newcommand{\lcIvLVIIfr}{\VERSE  Cependant, le temps où Elisabeth devait enfanter s'accomplit, et elle mit au monde un fils. \EVERSE}
\newcommand{\lcIvLVIIIfr}{\VERSE  Ses voisins et ses parents apprirent que le Seigneur avait signalé envers elle Sa miséricorde, et ils l'en félicitaient. \EVERSE}
\newcommand{\lcIvLIXfr}{\VERSE  Et il arriva qu'au huitième jour ils vinrent pour circoncire l'enfant, et ils l'appelaient Zacharie, du nom de son père. \EVERSE}
\newcommand{\lcIvLXfr}{\VERSE  Mais sa mère, prenant la parole, dit: Non; mais il sera appelé Jean. \EVERSE}
\newcommand{\lcIvLXIfr}{\VERSE  Ils lui dirent: Il n'y a personne dans ta famille qui soit appelé de ce nom. \EVERSE}
\newcommand{\lcIvLXIIfr}{\VERSE  Et ils faisaient des signes à son père, pour savoir comment il voulait qu'on l'appelât. \EVERSE}
\newcommand{\lcIvLXIIIfr}{\VERSE  Et, demandant des tablettes, il écrivit: Jean est son nom.  Et tous furent dans l'étonnement. \EVERSE}
\newcommand{\lcIvLXIVfr}{\VERSE  Au même instant, sa bouche s'ouvrit, et sa langue se délia, et il parlait en bénissant Dieu. \EVERSE}
\newcommand{\lcIvLXVfr}{\VERSE  Et la crainte s'empara de tous leurs voisins, et, dans toutes les montagnes de la Judée, toutes ces choses étaient divulguées. \EVERSE}
\newcommand{\lcIvLXVIfr}{\VERSE  Et tous ceux qui les entendirent les conservèrent dans leur coeur, en disant: Que pensez-vous que sera cet enfant?  Car la main du Seigneur était avec lui. \EVERSE}
\newcommand{\lcIvLXVIIfr}{\VERSE  Et Zacharie, son père, fut rempli du Saint-Esprit, et il prophétisa, en disant: \EVERSE}
\newcommand{\lcIvLXVIIIfr}{\VERSE  Béni soit le Seigneur, le Dieu d'Israël, de ce qu'Il a visité et racheté Son peuple, \EVERSE}
\newcommand{\lcIvLXIXfr}{\VERSE  et nous a suscité un puissant Sauveur dans la maison de David, Son serviteur, \EVERSE}
\newcommand{\lcIvLXXfr}{\VERSE  ainsi qu'Il a dit par la bouche de Ses saints prophètes des temps anciens, \EVERSE}
\newcommand{\lcIvLXXIfr}{\VERSE  qu'Il nous délivrerait de nos ennemis et de la main de tous ceux qui nous haïssent, \EVERSE}
\newcommand{\lcIvLXXIIfr}{\VERSE  pour exercer Sa miséricorde envers nos pères, et Se souvenir de Son alliance sainte, \EVERSE}
\newcommand{\lcIvLXXIIIfr}{\VERSE  selon le serment qu'Il a juré à Abraham, notre père, de nous accorder cette grâce, \EVERSE}
\newcommand{\lcIvLXXIVfr}{\VERSE  qu'étant délivrés de la main de nos ennemis, nous Le servions sans crainte, \EVERSE}
\newcommand{\lcIvLXXVfr}{\VERSE  marchant devant Lui dans la sainteté et la justice, tous les jours de notre vie. \EVERSE}
\newcommand{\lcIvLXXVIfr}{\VERSE  Et toi, petit enfant, tu seras appelé le prophète du Très-Haut: car tu marcheras devant la face du Seigneur, pour préparer Ses voies, \EVERSE}
\newcommand{\lcIvLXXVIIfr}{\VERSE  afin de donner à Son peuple la connaissance du salut, pour la rémission de leurs péchés, \EVERSE}
\newcommand{\lcIvLXXVIIIfr}{\VERSE  par les entrailles de la miséricorde de notre Dieu, grâce auxquelles le soleil levant nous a visités d'en haut, \EVERSE}
\newcommand{\lcIvLXXIXfr}{\VERSE  pour éclairer ceux qui sont assis dans les ténèbres et à l'ombre de la mort, pour diriger nos pas dans la voie de la paix. \EVERSE}
\newcommand{\lcIvLXXXfr}{\VERSE  Or l'enfant croissait, et se fortifiait en esprit; et il demeurait dans les déserts, jusqu'au jour de sa manifestation à Israël. \EVERSE}
\newcommand{\lcIIvIfr}{\VERSE  Or il arriva qu'en ces jours-là, il parut un édit de César Auguste, ordonnant un recensement de toute la terre. \EVERSE}
\newcommand{\lcIIvIIfr}{\VERSE  Ce premier recensement fut fait par Cyrinus, gouverneur de Syrie. \EVERSE}
\newcommand{\lcIIvIIIfr}{\VERSE  Et tous allaient se faire enregistrer, chacun dans sa ville. \EVERSE}
\newcommand{\lcIIvIVfr}{\VERSE  Joseph aussi monta de Nazareth, ville de Galilée, en Judée, dans la ville de David, appelée Bethléem, parce qu'il était de la maison et de la famille de David, \EVERSE}
\newcommand{\lcIIvVfr}{\VERSE  pour se faire enregistrer avec Marie son Epouse, qui était enceinte. \EVERSE}
\newcommand{\lcIIvVIfr}{\VERSE  Or il arriva, pendant qu'ils étaient là, que les jours où Elle devait enfanter furent accomplis. \EVERSE}
\newcommand{\lcIIvVIIfr}{\VERSE  Et Elle enfanta Son Fils premier-né, et Elle L'enveloppa de langes, et Le coucha dans une crèche, parce qu'il n'y avait pas de place pour eux dans l'hôtellerie. \EVERSE}
\newcommand{\lcIIvVIIIfr}{\VERSE  Et il y avait, dans la même contrée, des bergers qui passaient les veilles de la nuit à la garde de leur troupeau. \EVERSE}
\newcommand{\lcIIvIXfr}{\VERSE  Et voici qu'un Ange du Seigneur leur apparut, et qu'une lumière divine resplendit autour d'eux; et ils furent saisis d'une grande crainte. \EVERSE}
\newcommand{\lcIIvXfr}{\VERSE  Et l'Ange leur dit: Ne craignez point; car voici que je vous annonce une grande joie qui sera pour tout le peuple: \EVERSE}
\newcommand{\lcIIvXIfr}{\VERSE  c'est qu'il vous est né aujourd'hui, dans la ville de David, un Sauveur, qui est le Christ, le Seigneur. \EVERSE}
\newcommand{\lcIIvXIIfr}{\VERSE  Et vous Le reconnaïtrez à ce signe: Vous trouverez un Enfant enveloppé de langes, et couché dans une crèche. \EVERSE}
\newcommand{\lcIIvXIIIfr}{\VERSE  Au même instant, il se joignit à l'Ange une troupe de l'armée céleste, louant Dieu, et disant: \EVERSE}
\newcommand{\lcIIvXIVfr}{\VERSE  Gloire à Dieu au plus haut des Cieux, et, sur la terre, paix aux hommes de bonne volonté. \EVERSE}
\newcommand{\lcIIvXVfr}{\VERSE  Et il arriva que, lorsque les Anges les eurent quittés pour retourner dans le Ciel, les bergers se disaient l'un à l'autre: Passons jusqu'à Bethléem, et voyons ce qui est arrivé, ce que le Seigneur nous a fait connaître. \EVERSE}
\newcommand{\lcIIvXVIfr}{\VERSE  Et ils y allèrent en grande hâte, et ils trouvèrent Marie et Joseph, et l'Enfant couché dans une crèche. \EVERSE}
\newcommand{\lcIIvXVIIfr}{\VERSE  Et en Le voyant, ils reconnurent la vérité de ce qui leur avait été dit au sujet de cet Enfant. \EVERSE}
\newcommand{\lcIIvXVIIIfr}{\VERSE  Et tous ceux qui l'entendirent admirèrent ce qui leur avait été raconté par les bergers. \EVERSE}
\newcommand{\lcIIvXIXfr}{\VERSE  Or Marie conservait toutes ces choses, les repassant dans Son coeur. \EVERSE}
\newcommand{\lcIIvXXfr}{\VERSE  Et les bergers s'en retournèrent, glorifiant et louant Dieu de tout ce qu'ils avaient entendu et vu, selon ce qu'il leur avait été dit. \EVERSE}
\newcommand{\lcIIvXXIfr}{\VERSE  Le huitième pour, auquel l'Enfant devait être circoncis, étant arrivé, on Lui donna le nom de Jésus, que l'Ange avait indiqué avant qu'Il fût conçu dans le sein de Sa Mère. \EVERSE}
\newcommand{\lcIIvXXIIfr}{\VERSE  Quand les jours de la purification de Marie furent accomplis, selon la loi de Moïse, ils Le portèrent à Jérusalem, pour Le présenter au Seigneur, \EVERSE}
\newcommand{\lcIIvXXIIIfr}{\VERSE  selon qu'il est prescrit dans la loi du Seigneur: Tout enfant mâle premier-né sera consacrè au Seigneur; \EVERSE}
\newcommand{\lcIIvXXIVfr}{\VERSE  et pour offrir en sacrifice, selon qu'il est prescrit dans la loi du Seigneur, deux tourterelles, ou deux petits de colombes. \EVERSE}
\newcommand{\lcIIvXXVfr}{\VERSE  Et voici qu'il y avait à Jérusalem un homme appelé Siméon, et cet homme était juste et craignant Dieu, et il attendait la consolation d'Israël, et l'Esprit-Saint était en lui. \EVERSE}
\newcommand{\lcIIvXXVIfr}{\VERSE  Et il lui avait été révélé par l'Esprit-Saint qu'il ne verrait pas la mort avant d'avoir vu le Christ du Seigneur. \EVERSE}
\newcommand{\lcIIvXXVIIfr}{\VERSE  Il vint au temple, poussé par l'Esprit de Dieu.  Et comme les parents de l'Enfant Jésus L'apportaient, afin d'accomplir pour Lui ce que la loi ordonnait, \EVERSE}
\newcommand{\lcIIvXXVIIIfr}{\VERSE  il Le prit entre ses bras, et bénit Dieu, et dit: \EVERSE}
\newcommand{\lcIIvXXIXfr}{\VERSE  Maintenant, Seigneur, vous laisserez Votre serviteur s'en aller en paix, selon Votre parole, \EVERSE}
\newcommand{\lcIIvXXXfr}{\VERSE  puisque mes yeux ont vu le salut qui vient de Vous, \EVERSE}
\newcommand{\lcIIvXXXIfr}{\VERSE  que Vous avez préparé à la face de tous les peuples: \EVERSE}
\newcommand{\lcIIvXXXIIfr}{\VERSE  lumiére pour éclairer les nations, et gloire d'Israël Votre peuple. \EVERSE}
\newcommand{\lcIIvXXXIIIfr}{\VERSE  Son père et Sa Mère étaient dans l'admiration des choses qu'on disait de Lui. \EVERSE}
\newcommand{\lcIIvXXXIVfr}{\VERSE  Et Siméon les bénit, et dit à Marie Sa Mère: Voici que cet Enfant est établi pour la ruine et pour la résurrection d'un grand nombre en Israël, et comme un signe qui excitera la contradiction, \EVERSE}
\newcommand{\lcIIvXXXVfr}{\VERSE  et, à Vous-même, un glaive Vous percera l'âme, afin que les pensées de coeurs nombreux soient dévoilées. \EVERSE}
\newcommand{\lcIIvXXXVIfr}{\VERSE  Il y avait aussi une prophétesse, Anne, fille de Phanuel, de la tribu d'Aser; elle était très avancée en âge, et elle avait vécu sept ans avec son mari depuis sa virginité. \EVERSE}
\newcommand{\lcIIvXXXVIIfr}{\VERSE  Elle était veuve alors, et âgée de quatre-vingt-quatre ans; elle ne s'éloignait pas du temple, servant Dieu jour et nuit dans les jeûnes et les prières. \EVERSE}
\newcommand{\lcIIvXXXVIIIfr}{\VERSE  Elle aussi, étant survenue à cette même heure, elle louait le Seigneur, et parlait de Lui à tous ceux qui attendaient la rédemption d'Israël. \EVERSE}
\newcommand{\lcIIvXXXIXfr}{\VERSE  Après qu'ils eurent tout accompli selon la loi du Seigneur, ils retournèrent en Galilée, à Nazareth, leur ville. \EVERSE}
\newcommand{\lcIIvXLfr}{\VERSE  Cependant l'Enfant croissait et Se fortifiait, rempli de sagesse, et la grâce de Dieu était en Lui. \EVERSE}
\newcommand{\lcIIvXLIfr}{\VERSE  Ses parents allaient tous les ans à Jérusalem, au jour solennel de la Pâque. \EVERSE}
\newcommand{\lcIIvXLIIfr}{\VERSE  Et lorsqu'Il fut âgé de douze ans, ils montèrent à Jérusalem, selon la coutume de la fête; \EVERSE}
\newcommand{\lcIIvXLIIIfr}{\VERSE  puis, les jours de la fête étant passés, lorsqu'ils s'en retournèrent, l'Enfant Jésus resta à Jérusalem, et Ses parents ne s'en aperçurent pas. \EVERSE}
\newcommand{\lcIIvXLIVfr}{\VERSE  Et pensant qu'Il était avec ceux de leur compagnie, ils marchèrent durant un jour, et ils Le cherchaient parmi leurs parents et leurs connaissances. \EVERSE}
\newcommand{\lcIIvXLVfr}{\VERSE  Mais ne Le trouvant pas, ils revinrent à Jérusalem, en Le cherchant. \EVERSE}
\newcommand{\lcIIvXLVIfr}{\VERSE  Et il arriva qu'après trois jours ils Le trouvèrent dans le temple, assis au milieu des docteurs, les écoutant et les interrogeant. \EVERSE}
\newcommand{\lcIIvXLVIIfr}{\VERSE  Et tous ceux qui L'entendaient étaient ravis de Sa sagesse et de Ses réponses. \EVERSE}
\newcommand{\lcIIvXLVIIIfr}{\VERSE  En Le voyant, ils furent étonnés.  Et Sa Mère Lui dit: Mon Fils, pourquoi as-Tu agi ainsi avec nous?  Voici que Ton père et Moi nous Te cherchions, tout affligés. \EVERSE}
\newcommand{\lcIIvXLIXfr}{\VERSE  Il leur dit: Pourquoi Me cherchiez-vous?  Ne saviez-vous pas qu'il faut que Je sois aux affaires de Mon Père? \EVERSE}
\newcommand{\lcIIvLfr}{\VERSE  Mais ils ne comprirent pas ce qu'Il leur disait. \EVERSE}
\newcommand{\lcIIvLIfr}{\VERSE  Et Il descendit avec eux, et vint à Nazareth; et Il leur était soumis.  Sa Mère conservait toutes ces choses dans Son coeur. \EVERSE}
\newcommand{\lcIIvLIIfr}{\VERSE  Et Jésus croissait en sagesse, et en âge, et en grâce, devant Dieu et devant les hommes. \EVERSE}
\newcommand{\lcIIIvIfr}{\VERSE  La quinzième année du règne de Tibère César, Ponce Pilate étant gouverneur de la Judée; Hérode, tétrarque de la Galilée; Philippe, son frère, tétrarque de l'Iturée et de la province de Trachonite, et Lysanias, tétrarque de l'Abilène; \EVERSE}
\newcommand{\lcIIIvIIfr}{\VERSE  sous les grands prêtres Anne et Caïphe, la parole du Seigneur se fit entendre à Jean, fils de Zacharie, dans le désert. \EVERSE}
\newcommand{\lcIIIvIIIfr}{\VERSE  Et il vint dans toute la région du Jourdain, prêchant le baptême de pénitence pour la rémission des péchés, \EVERSE}
\newcommand{\lcIIIvIVfr}{\VERSE  ainsi qu'il est écrit au livre des discours du prophète Isaïe: Voix de celui qui crie dans le désert: Préparez le chemin du Seigneur, rendez droits Ses sentiers; \EVERSE}
\newcommand{\lcIIIvVfr}{\VERSE  toute vallée sera comblée, et toute montagne et toute colline seront abaissées, ce qui est tortueux sera redressé, et ce qui est raboteux sera aplani; \EVERSE}
\newcommand{\lcIIIvVIfr}{\VERSE  et toute chair verra le salut de Dieu. \EVERSE}
\newcommand{\lcIIIvVIIfr}{\VERSE  Il disait donc aux foules qui venaient pour être baptisés par lui: Race de vipères, qui vous a montrés à fuir la colère à venir? \EVERSE}
\newcommand{\lcIIIvVIIIfr}{\VERSE  Faites donc de dignes fruits de pénitence, et ne commencez point par dire: Nous avons Abraham pour père.  Car je vous déclare que, de ces pierres, Dieu peut susciter des enfants à Abraham. \EVERSE}
\newcommand{\lcIIIvIXfr}{\VERSE  Déjà la cognée est mise à la racine des arbres: tout arbre donc qui ne produit pas de bon fruit sera coupé et jeté au feu. \EVERSE}
\newcommand{\lcIIIvXfr}{\VERSE  Et les foules l'interrogeaient, en disant: Que ferons-nous donc? \EVERSE}
\newcommand{\lcIIIvXIfr}{\VERSE  Et il leur répondait en ces termes: Que celui qui a deux tuniques en donne une à celui qui n'en a point, et que celui qui a de quoi manger fasse de même. \EVERSE}
\newcommand{\lcIIIvXIIfr}{\VERSE  Des publicains vinrent aussi pour être baptisés, et ils lui dirent: Maître, que ferons-nous? \EVERSE}
\newcommand{\lcIIIvXIIIfr}{\VERSE  Et il leur dit: N'exigez rien au delà de ce qui vous a été ordonné. \EVERSE}
\newcommand{\lcIIIvXIVfr}{\VERSE  Les soldats l'interrogeaient aussi, disant: Et nous, que ferons-nous?  Et il leur dit: N'usez de violence envers personne, ne calomniez pas, et contentez-vous de votre solde. \EVERSE}
\newcommand{\lcIIIvXVfr}{\VERSE  Cependant, comme le peuple supposait, et que tous pensaient dans leurs coeurs, que Jean était peut-être le Christ, \EVERSE}
\newcommand{\lcIIIvXVIfr}{\VERSE  Jean répondit, en disant à tous: Moi, je vous baptise dans l'eau; mais il viendra Quelqu'un de plus puissant que moi, et je ne suis pas digne de délier la courroie de Ses sandales: C'est Lui qui vous baptisera dans l'Esprit-Saint et dans le feu. \EVERSE}
\newcommand{\lcIIIvXVIIfr}{\VERSE  Le van est dans Sa main, et Il nettoiera Son aire; et Il amassera le blé dans Son grenier, mais Il brûlera la paille dans un feu qui ne s'éteint point. \EVERSE}
\newcommand{\lcIIIvXVIIIfr}{\VERSE  Il évangélisait le peuple, en lui adressant encore beaucoup d'autres exhortations. \EVERSE}
\newcommand{\lcIIIvXIXfr}{\VERSE  Mais, comme il reprenait Hérode le tértrarque, au sujet d'Hérodiade, femme de son frère, et de toutes les mauvaises actions qu'il avait commises, \EVERSE}
\newcommand{\lcIIIvXXfr}{\VERSE  Hérode ajouta encore à tous ses crimes celui d'enfermer Jean en prison. \EVERSE}
\newcommand{\lcIIIvXXIfr}{\VERSE  Or, il arriva que, tout le peuple recevant le baptême, Jésus ayant aussi été baptisé, comme Il priait, le Ciel s'ouvrit, \EVERSE}
\newcommand{\lcIIIvXXIIfr}{\VERSE  et l'Esprit-Saint descendit sur Lui sous une forme corporelle, comme une colombe; et une voix se fit entendre du Ciel: Tu es Mon Fils bien-aimé; en Toi Je Me suis complu. \EVERSE}
\newcommand{\lcIIIvXXIIIfr}{\VERSE  Or Jésus avait environ trente ans lorsqu'Il commença Son ministère, étant, comme on le croyait, fils de Joseph, qui le fut d'Héli, qui le fut de Mathat, \EVERSE}
\newcommand{\lcIIIvXXIVfr}{\VERSE  qui le fut de Lévi, qui le fut de Melchi, qui le fut de Janné, qui le fut de Joseph, \EVERSE}
\newcommand{\lcIIIvXXVfr}{\VERSE  qui le fut de Mathathias, qui le fut d'Amos, qui le fut de Nahum, qui le fut d'Hesli, qui le fut de Naggé, \EVERSE}
\newcommand{\lcIIIvXXVIfr}{\VERSE  qui le fut de Mahath, qui le fut de Mathathias, qui le fut de Séméi, qui le fut de Joseph, qui le fut de Juda; \EVERSE}
\newcommand{\lcIIIvXXVIIfr}{\VERSE  qui le fut de Joanna, qui le fut de Résa, qui le fut de Zorobabel, qui le fut de Salathiel, qui le fut de Néri, \EVERSE}
\newcommand{\lcIIIvXXVIIIfr}{\VERSE  qui le fut de Melchi, qui le fut d'Addi, qui le fut de Cosan, qui le fut d'Elmadan, qui le fut de Her, \EVERSE}
\newcommand{\lcIIIvXXIXfr}{\VERSE  qui le fut de Jésus, qui le fut d'Eliézer, qui le fut de Jorim, qui le fut de Mathat, qui le fut de Lévi, \EVERSE}
\newcommand{\lcIIIvXXXfr}{\VERSE  qui le fut de Siméon, qui le fut de Juda, qui le fut de Joseph, qui le fut de Jona, qui le fut d'Eliakim, \EVERSE}
\newcommand{\lcIIIvXXXIfr}{\VERSE  qui le fut de Méléa, qui le fut de Menna, qui le fut de Mathatha, qui le fut de Nathan, qui le fut de David, \EVERSE}
\newcommand{\lcIIIvXXXIIfr}{\VERSE  qui le fut de Jessé, qui le fut d'Obed, qui le fut de Booz, qui le fut de Salmon, qui le fut de Naasson, \EVERSE}
\newcommand{\lcIIIvXXXIIIfr}{\VERSE  qui le fut d'Aminadab, qui le fut d'Aram, qui le fut d'Esron, qui le fut de Pharès, qui le fut de Juda, \EVERSE}
\newcommand{\lcIIIvXXXIVfr}{\VERSE  qui le fut de Jacob, qui le fut d'Isaac, qui le fut d'Abraham, qui le fut de Tharé, qui le fut de Nachor, \EVERSE}
\newcommand{\lcIIIvXXXVfr}{\VERSE  qui le fut de Sarug, qui le fut de Ragaü, qui le fut de Phaleg, qui le fut d'Héber, qui le fut de Salé, \EVERSE}
\newcommand{\lcIIIvXXXVIfr}{\VERSE  qui le fut de Caïnan, qui le fut d'Arphaxad, qui le fut de Sem, qui le fut de Noé, qui le fut de Lamech, \EVERSE}
\newcommand{\lcIIIvXXXVIIfr}{\VERSE  qui le fut de Mathusalé, qui le fut d'Hénoch, qui le fut de Jared, qui le fut de Malaléel, qui le fut de Caïnan, \EVERSE}
\newcommand{\lcIIIvXXXVIIIfr}{\VERSE  qui le fut d'Hénos, qui le fut de Seth, qui le fut d'Adam, qui le fut de Dieu. \EVERSE}
\newcommand{\lcIVvIfr}{\VERSE  Or Jésus, plein de l'Esprit-Saint, revint du Jourdain, et Il fut poussé par l'Esprit dans le désert \EVERSE}
\newcommand{\lcIVvIIfr}{\VERSE  pendant quarante jours, et Il fut tenté par le diable.  Et Il ne mangea rien durant ces jours-là, et lorsqu'ils furent écoulés, Il eut faim. \EVERSE}
\newcommand{\lcIVvIIIfr}{\VERSE  Alors le diable Lui dit: Si Vous êtes le Fils de Dieu, dites à cette pierre qu'elle devienne du pain. \EVERSE}
\newcommand{\lcIVvIVfr}{\VERSE  Jésus lui répondit:  Il est écrit: L'homme ne vit pas seulement de pain, mais de toute parole de Dieu. \EVERSE}
\newcommand{\lcIVvVfr}{\VERSE  Et le diable Le conduisit sur une haute montagne, et Lui montra en un instant tous les royaumes de la terre; \EVERSE}
\newcommand{\lcIVvVIfr}{\VERSE  puis il Lui dit: Je Vous donnerai toute cette puissance et la gloire de ces royaumes; car ils m'ont été livrés, et je les donne à qui je veux. \EVERSE}
\newcommand{\lcIVvVIIfr}{\VERSE  Si donc Vous Vous prosternez devant moi, toutes ces choses seront à Vous. \EVERSE}
\newcommand{\lcIVvVIIIfr}{\VERSE  Jésus lui répondit:  Il est écrit: Tu adoreras le Seigneur ton Dieu, et tu Le serviras Lui seul. \EVERSE}
\newcommand{\lcIVvIXfr}{\VERSE  Et il Le conduisit à Jérusalem, et Le plaça sur le pinacle du temple; puis il Lui dit: Si Vous êtes le Fils de Dieu, jetez-Vous d'ici en bas. \EVERSE}
\newcommand{\lcIVvXfr}{\VERSE  Car il est écrit: Il a donné des ordres à Ses Anges à Ton sujet, afin qu'ils Te gardent, \EVERSE}
\newcommand{\lcIVvXIfr}{\VERSE  et ils Te porteront dans leurs mains, de peur que Tu ne heurtes Ton pied contre une pierre. \EVERSE}
\newcommand{\lcIVvXIIfr}{\VERSE  Jésus lui répondit:  Il a été dit: Tu ne tenteras pas le Seigneur ton Dieu. \EVERSE}
\newcommand{\lcIVvXIIIfr}{\VERSE  Après avoir achevé toutes ces tentations, le diable s'éloigna de Lui pour un temps. \EVERSE}
\newcommand{\lcIVvXIVfr}{\VERSE  Alors Jésus retourna en Galilée par la vertu de l'Esprit, et Sa renommée se répandit dans tout le pays. \EVERSE}
\newcommand{\lcIVvXVfr}{\VERSE  Et Il enseignait dans leurs synagogues, et Il était glorifié par tous. \EVERSE}
\newcommand{\lcIVvXVIfr}{\VERSE  Il vint à Nazareth, où Il avait été élevé; et Il entra selon Sa coutume, le jour du sabbat, dans la synagogue, et Il Se leva pour lire. \EVERSE}
\newcommand{\lcIVvXVIIfr}{\VERSE  On Lui donna le livre du prophète Isaïe.  Et ayant déroulé le livre, Il trouva l'endroit où il était écrit: \EVERSE}
\newcommand{\lcIVvXVIIIfr}{\VERSE  L'Esprit du Seigneur est sur Moi; c'est pourquoi Il M'a sacré par Son onction; Il M'a envoyé évangéliser les pauvres, guérir ceux qui ont le coeur broyé, \EVERSE}
\newcommand{\lcIVvXIXfr}{\VERSE  annoncer aux captifs la délivrance, et aux aveugles le recouvrement de la vue, mettre en liberté ceux qui sont brisés sous les fers, publier l'année favorable du Seigneur et le jour de la rétribution. \EVERSE}
\newcommand{\lcIVvXXfr}{\VERSE  Ayant replié le livre, Il le rendit au ministre, et S'assit.  Et tous, dans le synagogue, avaient les yeux fixés sur Lui. \EVERSE}
\newcommand{\lcIVvXXIfr}{\VERSE  Et Il commença à leur dire: Aujourd'hui, cette parole de l'Ecriture que vous venez d'entendre est accomplie. \EVERSE}
\newcommand{\lcIVvXXIIfr}{\VERSE  Et tous Lui rendaient témoignage, et ils admiraient les paroles de grâce qui sortaient de Sa bouche, et ils disaient: N'est-ce pas là le fils de Joseph? \EVERSE}
\newcommand{\lcIVvXXIIIfr}{\VERSE  Alors Il leur dit: Sans doute, vous M'appliquerez ce proverbe: Médecin, guéris-toi toi-même; les grandes choses faites à Capharnaüm, dont nous avons entendu parler, faites-les également ici, dans Votre pays. \EVERSE}
\newcommand{\lcIVvXXIVfr}{\VERSE  Et Il ajouta: En vérité, Je vous le dis, aucun prophète n'est bien reçu dans sa patrie. \EVERSE}
\newcommand{\lcIVvXXVfr}{\VERSE  En vérité, Je vous le dis, il y avait beaucoup de veuves en Israël au temps d'Elie, lorsque le ciel fut fermé pendant trois ans et six mois, et qu'il y eut une grande famine dans tout le pays; \EVERSE}
\newcommand{\lcIVvXXVIfr}{\VERSE  et cependant, Elie ne fut envoyé à aucune d'elles, mais à une femme veuve de Sarepta, dans le pays de Sidon. \EVERSE}
\newcommand{\lcIVvXXVIIfr}{\VERSE  Il y avait aussi beaucoup de lépreux en Israël au temps du prophète Elisée; et aucun d'eux ne fut guéri, si ce n'est Naaman, le Syrien. \EVERSE}
\newcommand{\lcIVvXXVIIIfr}{\VERSE  Ils furent tous remplis de colère, dans la synagogue, en entendant ces paroles. \EVERSE}
\newcommand{\lcIVvXXIXfr}{\VERSE  Et se levant, ils Le chassèrent hors de la ville, et ils Le menèrent jusqu'au sommet de la montagne sur laquelle leur ville était bâtie, pour Le précipiter en bas. \EVERSE}
\newcommand{\lcIVvXXXfr}{\VERSE  Mais Lui, passant au milieu d'eux, S'en alla. \EVERSE}
\newcommand{\lcIVvXXXIfr}{\VERSE  Et Il descendit à Capharnaüm, ville de Galilée, et là Il les enseignait les jours de sabbat. \EVERSE}
\newcommand{\lcIVvXXXIIfr}{\VERSE  Et ils étaient frappés de Sa doctrine, car Il parlait avec autorité. \EVERSE}
\newcommand{\lcIVvXXXIIIfr}{\VERSE  Il y avait dans la synagogue un homme possédé d'un démon impur, qui cria d'une voix forte, \EVERSE}
\newcommand{\lcIVvXXXIVfr}{\VERSE  en disant: Laissez-nous; qu'y a-t-il de commun entre nous et Vous, Jésus de Nazareth?  Etes-Vous venu pour nous perdre?  Je sais qui Vous êtes: le Saint de Dieu. \EVERSE}
\newcommand{\lcIVvXXXVfr}{\VERSE  Mais Jésus le menaça, en disant: Tais-toi, et sors de cet homme.  Et le démon, l'ayant jeté à terre au milieu de l'assemblée, sortit de lui, sans lui faire aucun mal. \EVERSE}
\newcommand{\lcIVvXXXVIfr}{\VERSE  Et l'épouvante les saisit tous, et ils se parlaient l'un à l'autre, en disant: Quelle est cette parole?  Il commande avec autorité et avec puissance aux esprits impurs, et ils sortent. \EVERSE}
\newcommand{\lcIVvXXXVIIfr}{\VERSE  Et Sa renommée se répandit de tous côtés dans le pays. \EVERSE}
\newcommand{\lcIVvXXXVIIIfr}{\VERSE  Etant sorti de la synagogue, Jésus entra dans la maison de Simon.  Or la belle-mère de Simon était retenue par une forte fièvre; et ils Le prièrent pour elle. \EVERSE}
\newcommand{\lcIVvXXXIXfr}{\VERSE  Alors, debout auprès d'elle, Il commanda à la fièvre, et la fièvre la quitta.  Et se levant aussitôt, elle les servait. \EVERSE}
\newcommand{\lcIVvXLfr}{\VERSE  Lorsque le soleil fut couché, tous ceux qui avaient des malades atteints de diverses maladies les Lui amenaient.  Et Lui, imposant les mains sur chacun d'eux, les guérissait. \EVERSE}
\newcommand{\lcIVvXLIfr}{\VERSE  Et les démons sortaient d'un grand nombre, criant et disant: Vous êtes le Fils de Dieu.  Mais Il les menaçait, et Il ne leur permettait pas de dire qu'ils savaient qu'Il était le Christ. \EVERSE}
\newcommand{\lcIVvXLIIfr}{\VERSE  Lorsqu'il fut jour, Il sortit et alla dans un lieu désert; et les foules Le cherchaient, et elles vinrent jusqu'à Lui, et elles voulaient Le retenir, de peur qu'Il ne les quittât. \EVERSE}
\newcommand{\lcIVvXLIIIfr}{\VERSE  Il leur dit: Il faut que J'annonce aussi aux autres villes la bonne nouvelle du royaume de Dieu; car c'est pour cela que J'ai été envoyé. \EVERSE}
\newcommand{\lcIVvXLIVfr}{\VERSE  Et Il prêchait dans les synagogues de Galilée. \EVERSE}
\newcommand{\lcVvIfr}{\VERSE  Or il arriva, tandis que les foules se précipitaient sur Lui pour entendre la parole de Dieu, qu'Il était lui-même au bord du lac de Génésareth. \EVERSE}
\newcommand{\lcVvIIfr}{\VERSE  Et Il vit deux barques arrêtées au bord du lac; les pêcheurs étaient descendus, et lavaient leurs filets. \EVERSE}
\newcommand{\lcVvIIIfr}{\VERSE  Et montant dans l'une de ces barques, qui appartenait à Simon, Il le pria de s'éloigner un peu de la terre; et S'étant assis, Il enseignait les foules de dessus la barque. \EVERSE}
\newcommand{\lcVvIVfr}{\VERSE  Lorsqu'Il eut cessé de parler, Il dit à Simon: Pousse au large, et jetez vos filets pour pêcher. \EVERSE}
\newcommand{\lcVvVfr}{\VERSE  Simon, Lui répondant, dit: Maître, nous avons travaillé toute la nuit sans rien prendre; mais, sur Votre parole, je jetterai le filet. \EVERSE}
\newcommand{\lcVvVIfr}{\VERSE  Lorsqu'ils l'eurent fait, ils prirent une si grande quantité de poissons, que leur filet se rompait. \EVERSE}
\newcommand{\lcVvVIIfr}{\VERSE  Et ils firent signe à leurs compagnons, qui étaient dans l'autre barque, de venir les aider.  Ils vinrent, et ils remplirent les deux barques, au point qu'elles étaient presque submergées. \EVERSE}
\newcommand{\lcVvVIIIfr}{\VERSE  Quand Simon Pierre vit cela, il tomba aux pieds de Jésus, en disant: Seigneur, retirez-Vous de moi, car je suis un pécheur. \EVERSE}
\newcommand{\lcVvIXfr}{\VERSE  Car l'épouvante l'avait saisi, et aussi tous ceux qui étaient avec Lui, à cause de la pêche des poissons qu'ils avaient faite; \EVERSE}
\newcommand{\lcVvXfr}{\VERSE  et de même Jacques et Jean, fils de Zébédée, qui étaient compagnons de Simon.  Alors Jésus dit à Simon: Ne crains point; désormais ce sont des hommes que tu prendras. \EVERSE}
\newcommand{\lcVvXIfr}{\VERSE  Et ayant ramené les barques à terre, ils quittèrent tout, et Le suivirent. \EVERSE}
\newcommand{\lcVvXIIfr}{\VERSE  Et comme Il était dans une des villes, voici qu'un homme couvert de lèpre, voyant Jésus, se prosterna la face contre terre, et Le pria, en disant: Seigneur, si Vous voulez, Vous pouvez me guérir. \EVERSE}
\newcommand{\lcVvXIIIfr}{\VERSE  Jésus, êtendant la main, le toucha et dit: Je le veux, sois guéri.  Et, au même instant, la lèpre le quitta. \EVERSE}
\newcommand{\lcVvXIVfr}{\VERSE  Et Il lui ordonna de n'en parler à personne: Mais, dit-Il, va, montre-toi au prêtre, et offre pour ta guérison ce que Moïse a prescrit, afin que cela leur serve de témoignage. \EVERSE}
\newcommand{\lcVvXVfr}{\VERSE  Cependant, Sa renommée se répandait de plus en plus, et des foules nombreuses venaient pour L'entendre, et pour être guéries de leurs maladies. \EVERSE}
\newcommand{\lcVvXVIfr}{\VERSE  Mais Lui, Il Se retirait dans le désert et priait. \EVERSE}
\newcommand{\lcVvXVIIfr}{\VERSE  Il arriva qu'un jour Il était assis et enseignait.  Et des pharisiens et des docteurs de la loi, qui étaient venus de tous les villages de la Galilée, et de la Judée, et de Jérusalem, étaient assis auprès de Lui; et la puissance du Seigneur agissait pour opérer des guérisons. \EVERSE}
\newcommand{\lcVvXVIIIfr}{\VERSE  Et voici que des gens, portant sur un lit un homme qui était paralytique, cherchaient à le faire entrer et à le déposer devant Jésus. \EVERSE}
\newcommand{\lcVvXIXfr}{\VERSE  Mais, ne trouvant point par où le faire entrer, à cause de la foule, ils montèrent sur le toit, et, par les tuiles, ils le descendirent avec le lit au milieu de l'assemblée, devant Jésus. \EVERSE}
\newcommand{\lcVvXXfr}{\VERSE  Dès qu'Il vit leur foi, Il dit: Homme, tes péchés te sont remis. \EVERSE}
\newcommand{\lcVvXXIfr}{\VERSE  Alors, les scribes et les pharisiens se mirent à penser et à dire en eux-mêmes: Quel est Celui-ci, qui profère des blasphèmes?  Qui peut remettre les péchés, si ce n'est Dieu seul? \EVERSE}
\newcommand{\lcVvXXIIfr}{\VERSE  Mais Jésus, connaissant leurs pensées, prit la parole et leur dit: Que pensez-vous dans vos coeurs? \EVERSE}
\newcommand{\lcVvXXIIIfr}{\VERSE  Lequel est le plus facile, de dire: Tes péchés te sont remis; ou de dire: Lève-toi et marche? \EVERSE}
\newcommand{\lcVvXXIVfr}{\VERSE  Or, afin que vous sachiez que le Fils de l'homme a sur la terre le pouvoir de remettre les péchés: Je te l'ordonne, dit-Il au paralytique; lève-toi, prends ton lit et va dans ta maison. \EVERSE}
\newcommand{\lcVvXXVfr}{\VERSE  Et aussitôt, se levant devant eux, il prit le lit sur lequel il était couché, et s'en alla dans sa maison, glorifiant Dieu. \EVERSE}
\newcommand{\lcVvXXVIfr}{\VERSE  Et la stupeur les saisit tous, et ils glorifiaient Dieu.  Et ils furent remplis de crainte, et ils disaient: Nous avons vu aujourd'hui des choses prodigieuses. \EVERSE}
\newcommand{\lcVvXXVIIfr}{\VERSE  Après cela, Jêsus sortit, et vit un publicain, nommé Lévi, assis au bureau des impôts.  Et Il lui dit: Suis-Moi. \EVERSE}
\newcommand{\lcVvXXVIIIfr}{\VERSE  Et laissant tout, il se leva et Le suivit. \EVERSE}
\newcommand{\lcVvXXIXfr}{\VERSE  Lévi Lui fit un grand festin dans sa maison, et il y avait une foule nombreuse de publicains et d'autres personnes qui étaient à table avec eux. \EVERSE}
\newcommand{\lcVvXXXfr}{\VERSE  Mais les pharisiens et leurs scribes murmuraient, et disaient à Ses disciples: Pourquoi mangez-vous et buvez-vous avec les publicains et les pécheurs? \EVERSE}
\newcommand{\lcVvXXXIfr}{\VERSE  Et Jésus, prenant la parole, leur dit: Ce ne sont pas ceux qui se portent bien qui ont besoin du médecin, mais les malades. \EVERSE}
\newcommand{\lcVvXXXIIfr}{\VERSE  Je ne suis pas venu appeler les justes, mais les pécheurs, à la pénitence. \EVERSE}
\newcommand{\lcVvXXXIIIfr}{\VERSE  Alors ils Lui dirent: Pourquoi les disciples de Jean font-ils souvent des jeûnes et des prières, de même ceux des pharisiens, tandis que les Vôtres mangent et boivent? \EVERSE}
\newcommand{\lcVvXXXIVfr}{\VERSE  Il leur répondit: Pouvez-vous faire jeûner les amis de l'Epoux, pendant que l'Epoux est avec eux? \EVERSE}
\newcommand{\lcVvXXXVfr}{\VERSE  Mais viendront des jours où l'Epoux leur sera enlevé, et alors ils jeûneront en ces jours-là. \EVERSE}
\newcommand{\lcVvXXXVIfr}{\VERSE  Il leur proposa aussi cette comparaison: Personne ne met une piéce d'un vêtement neuf à un vieux vêtement; autrement on déchire le neuf, et la pièce du vêtement neuf ne convient point au vieux vêtement. \EVERSE}
\newcommand{\lcVvXXXVIIfr}{\VERSE  Et personne ne met du vin nouveau dans de vieilles outres; autrement le vin nouveau rompra les outres, et il se répandra, et les outres seront perdues. \EVERSE}
\newcommand{\lcVvXXXVIIIfr}{\VERSE  Mais il faut mettre le vin nouveau dans des outres neuves, et ainsi les deux se conservent. \EVERSE}
\newcommand{\lcVvXXXIXfr}{\VERSE  Et personne, buvant du vin vieux, n'en veut aussitôt du nouveau; car il dit: Le vieux est meilleur. \EVERSE}
\newcommand{\lcVIvIfr}{\VERSE  Or, un jour de sabbat appelé second-premier, il arriva que, comme Il passait le long des blés, Ses disciples arrachaient des épis, et les mangeaient, après les avoir froissés dans leurs mains. \EVERSE}
\newcommand{\lcVIvIIfr}{\VERSE  Et quelques-uns des pharisiens leur disaient: Pourquoi faites-vous ce qui n'est pas permis aux jours de sabbat? \EVERSE}
\newcommand{\lcVIvIIIfr}{\VERSE  Et Jésus leur répondit: N'avez-vous pas lu ce que fit David, lorsqu'il eut faim lui et ceux qui l'accompagnaient; \EVERSE}
\newcommand{\lcVIvIVfr}{\VERSE  comment il entra dans la maison de Dieu, et prit les pains de proposition, en mangea, et en donna à ceux qui étaient avec lui, quoiqu'il ne soit permis qu'aux seuls prêtres d'en manger? \EVERSE}
\newcommand{\lcVIvVfr}{\VERSE  Et Il leur disait: Le Fils de l'homme est Maître même du sabbat. \EVERSE}
\newcommand{\lcVIvVIfr}{\VERSE  Il arriva, un autre jour de sabbat, qu'Il entra dans la synagogue et qu'Il enseignait; et il y avait là un homme dont la main droite était desséchée. \EVERSE}
\newcommand{\lcVIvVIIfr}{\VERSE  Or les scribes et les pharisiens L'observaient, pour voir s'Il ferait une guérison le jour du sabbat, afin de trouver de quoi L'accuser. \EVERSE}
\newcommand{\lcVIvVIIIfr}{\VERSE  Mais Lui, Il connaissait leurs pensées, et Il dit à l'homme qui avait la main desséchée: Lève-toi, et tiens-toi là au milieu.  Et se levant, il se tint debout. \EVERSE}
\newcommand{\lcVIvIXfr}{\VERSE  Alors Jésus leur dit: Je vous demande s'il est permis, les jours de sabbat, de faire du bien ou de faire du mal, de sauver la vie ou de l'ôter? \EVERSE}
\newcommand{\lcVIvXfr}{\VERSE  Et ayant promené Ses regards sur eux tous, Il dit à l'homme: Etends ta main.  Il l'étendit, et sa main fut guérie. \EVERSE}
\newcommand{\lcVIvXIfr}{\VERSE  Mais eux, remplis de démence, s'entretenaient ensemble de ce qu'ils feraient à Jésus. \EVERSE}
\newcommand{\lcVIvXIIfr}{\VERSE  Or il arriva qu'en ces jours-là Il S'en alla sur une montagne pour prier, et Il passa toute la nuit à prier Dieu. \EVERSE}
\newcommand{\lcVIvXIIIfr}{\VERSE  Et quand le jour fut venu, Il appela Ses disciples; et Il en choisit douze d'entre eux, qu'Il nomma apôtres: \EVERSE}
\newcommand{\lcVIvXIVfr}{\VERSE  Simon auquel Il donna le surnom de Pierre, et André son frère, Jacques et Jean, Philippe et Barthélemy, \EVERSE}
\newcommand{\lcVIvXVfr}{\VERSE  Matthieu et Thomas, Jacques fils d'Alphée, et Simon appelé le Zélote, \EVERSE}
\newcommand{\lcVIvXVIfr}{\VERSE  Jude frère de Jacques, et Judas Iscariote, qui fut le traître. \EVERSE}
\newcommand{\lcVIvXVIIfr}{\VERSE  Et descendant avec eux, Il S'arrêta dans une plaine, avec la troupe de Ses disciples et une grande multitude de peuple de toute la Judée, et de Jérusalem, et de la contrée maritime, et de Tyr, et de Sidon; \EVERSE}
\newcommand{\lcVIvXVIIIfr}{\VERSE  ils étaient venus pour L'entendre et pour être guéris de leurs maladies.  Et ceux qui étaient tourmentés par des esprits impurs étaient guéris. \EVERSE}
\newcommand{\lcVIvXIXfr}{\VERSE  Et toute la foule cherchait à Le toucher, parce qu'une vertu sortait de Lui et les guérissait tous. \EVERSE}
\newcommand{\lcVIvXXfr}{\VERSE  Et Lui, levant les yeux sur Ses disciples, disait: Bienheureux, vous qui êtes pauvres, parce que le royaume de Dieu est à vous. \EVERSE}
\newcommand{\lcVIvXXIfr}{\VERSE  Bienheureux, vous qui avez faim maintenant, parce que vous serez rassasiés.  Bienheureux, vous qui pleurez maintenant, parce que vous rirez. \EVERSE}
\newcommand{\lcVIvXXIIfr}{\VERSE  Bienheurex serez-vous lorsque les hommes vous haïront, et vous repousseront, et vous outrageront, et lorsqu'ils rejetteront votre nom comme infâme, à cause du Fils de l'homme. \EVERSE}
\newcommand{\lcVIvXXIIIfr}{\VERSE  Réjouissez-vous en ce jour-là et soyez dans l'allégresse, parce que votre récompense est grande dans le Ciel; car c'est ainsi que leurs pères traitaient les prophètes. \EVERSE}
\newcommand{\lcVIvXXIVfr}{\VERSE  Mais malheur à vous, riches, parce que vous avez votre consolation. \EVERSE}
\newcommand{\lcVIvXXVfr}{\VERSE  Malheur à vous qui êtes rassasiés, parce que vous aurez faim.  Malheur à vous qui riez maintenant, parce que vous serez dans le deuil et dans les larmes. \EVERSE}
\newcommand{\lcVIvXXVIfr}{\VERSE  Malheur à vous lorsque les hommes diront du bien de vous, car c'est ainsi que leurs pères traitaient les faux prophètes. \EVERSE}
\newcommand{\lcVIvXXVIIfr}{\VERSE  Mais à vous qui M'écoutez, Je dis: Aimez vos ennemis, faites du bien à ceux qui vous haïssent. \EVERSE}
\newcommand{\lcVIvXXVIIIfr}{\VERSE  Bénissez ceux qui vous maudissent, et priez pour ceux qui vous calomnient. \EVERSE}
\newcommand{\lcVIvXXIXfr}{\VERSE  Et à celui qui te frappe sur une joue, présente encore l'autre; et celui qui te prend ton manteau, ne l'empêche pas de prendre aussi ta tunique. \EVERSE}
\newcommand{\lcVIvXXXfr}{\VERSE  Donne à quiconque te demande, et ne redemande pas ton bien à celui qui s'en empare. \EVERSE}
\newcommand{\lcVIvXXXIfr}{\VERSE  Et ce que vous voulez que les hommes vous fassent, faites-le-leur vous aussi, pareillement. \EVERSE}
\newcommand{\lcVIvXXXIIfr}{\VERSE  Si vous aimez ceux qui vous aiment, quel gré vous en saura-t-on? car les pécheurs aussi aiment ceux qui les aiment. \EVERSE}
\newcommand{\lcVIvXXXIIIfr}{\VERSE  Et si vous faites du bien à ceux qui vous en font, quel gré vous en saura-t-on? car les pécheurs aussi font cela. \EVERSE}
\newcommand{\lcVIvXXXIVfr}{\VERSE  Et si vous prêtez à ceux de qui vous espérez recevoir, quel gré vous en saura-t-on? car les pécheurs aussi prêtent aux pécheurs, afin de recevoir la pareille. \EVERSE}
\newcommand{\lcVIvXXXVfr}{\VERSE  Mais vous, aimez vos ennemis, faites du bien, et donnez beaucoup sans en rien esperer, et votre récompense sera grande, et vous serez les fils du Très-Haut, car Il est bon pour les ingrats et les méchants. \EVERSE}
\newcommand{\lcVIvXXXVIfr}{\VERSE  Soyez donc miséricordieux, comme votre Père est miséricordieux. \EVERSE}
\newcommand{\lcVIvXXXVIIfr}{\VERSE  Ne jugez point, et vous ne serez pas jugés; ne condamnez point, et vous ne serez pas condamnés; pardonnez, et on vous pardonnera. \EVERSE}
\newcommand{\lcVIvXXXVIIIfr}{\VERSE  Donnez, et on vous donnera: on versera dans votre sein une bonne mesure, pressée, et secouée, et qui débordera.  Car la même mesure avec laquelle vous aurez mesuré servira de mesure pour vous. \EVERSE}
\newcommand{\lcVIvXXXIXfr}{\VERSE  Il leur proposait aussi cette comparaison: Est-ce qu'un aveugle peut conduire un aveugle?  Ne tomberont-ils pas tous deux dans la fosse? \EVERSE}
\newcommand{\lcVIvXLfr}{\VERSE  Le disciple n'est pas au-dessus du maître; mais tout disciple sera parfait, s'il est comme son maître. \EVERSE}
\newcommand{\lcVIvXLIfr}{\VERSE  Pourquoi vois-tu le fétu dans l'oeil de ton frère, sans apercevoir la poutre qui est dans ton oeil? \EVERSE}
\newcommand{\lcVIvXLIIfr}{\VERSE  Ou comment peux-tu dire à ton frère: Frère, laisse-moi ôter le fétu qui est dans ton oeil, toi qui ne vois pas la poutre qui est dans le tien?  Hypocrite, ôte d'abord la poutre qui est dans ton oeil, et ensuite tu verras comment tu pourras ôter le fétu de l'oeil de ton frère. \EVERSE}
\newcommand{\lcVIvXLIIIfr}{\VERSE  Car un arbre n'est pas bon, s'il produit de mauvais fruits, et un arbre n'est pas mauvais, s'il produit de bons fruits. \EVERSE}
\newcommand{\lcVIvXLIVfr}{\VERSE  Car chaque arbre se connaît à son fruit.  On ne cueille point de figues sur les épines, et on ne vendange pas le raisin sur des ronces. \EVERSE}
\newcommand{\lcVIvXLVfr}{\VERSE  L'homme bon tire de bonnes choses du bon trésor de son coeur, et l'homme mauvais tire de mauvaises choses de son mauvais trésor; car la bouche parle de l'abondance du coeur. \EVERSE}
\newcommand{\lcVIvXLVIfr}{\VERSE  Pourquoi M'appelez-vous Seigneur! Seigneur! et ne faites-vous pas ce que Je vous dis? \EVERSE}
\newcommand{\lcVIvXLVIIfr}{\VERSE  Quiconque vient à Moi, et écoute Mes paroles, et les met en pratique, Je vous montrerai à qui il ressemble. \EVERSE}
\newcommand{\lcVIvXLVIIIfr}{\VERSE  Il ressemble à un homme qui, bâtissant une maison, a creusé bien avant, et a posé le fondement sur la pierre; l'inondation étant survenue, le torrent s'est précipité sur cette maison et n'a pu l'ébranler, parce qu'elle était fondée sur la pierre. \EVERSE}
\newcommand{\lcVIvXLIXfr}{\VERSE  Mais celui qui écoute et ne met pas en pratique, ressemble à un homme qui a bâti sa maison sur la terre, sans fondement; le torrent s'est précipité sur elle, et aussitôt elle est tombée, et la ruine de cette maison a été grande. \EVERSE}
\newcommand{\lcVIIvIfr}{\VERSE  Lorsqu'Il eut achevé de faire entendre au peuple toutes ces paroles, Il entra dans Capharnaüm. \EVERSE}
\newcommand{\lcVIIvIIfr}{\VERSE  Or un centurion avait un serviteur malade et sur le point de mourir, qui lui était très cher. \EVERSE}
\newcommand{\lcVIIvIIIfr}{\VERSE  Et ayant entendu parler de Jésus, il Lui envoya quelques anciens des Juifs, Le priant de venir et de guérir son serviteur. \EVERSE}
\newcommand{\lcVIIvIVfr}{\VERSE  Ceux-ci, étant venus auprès de Jésus, Le priaient avec instance, en Lui disant: Il mérite que Vous lui accordiez cela; \EVERSE}
\newcommand{\lcVIIvVfr}{\VERSE  car il aime notre nation, et il nous a lui-même bâti une synagogue. \EVERSE}
\newcommand{\lcVIIvVIfr}{\VERSE  Et Jésus allait avec eux.  Et comme Il n'était plus guère éloigné de la maison, le centurion Lui envoya de ses amis, pour Lui dire: Seigneur, ne prenez pas tant de peine, car je ne suis pas digne que Vous entriez sous mon toit. \EVERSE}
\newcommand{\lcVIIvVIIfr}{\VERSE  C'est pour cela que je ne me suis pas cru digne de venir moi-même auprès de Vous; mais dites un mot, et mon serviteur sera guéri. \EVERSE}
\newcommand{\lcVIIvVIIIfr}{\VERSE  Car moi, qui suis un homme soumis à des chefs, j'ai sous moi des soldats; et je dis à l'un: Va, et il va; et à l'autre: Viens, et il vient; et à mon serviteur: Fais ceci, et il le fait. \EVERSE}
\newcommand{\lcVIIvIXfr}{\VERSE  Ayant entendu ces paroles, Jésus fut dans l'admiration; et Se tournant vers les foules qui Le suivaient, Il dit: En vérité, Je vous le dis, même en Israël Je n'ai pas trouvé une aussi grande foi. \EVERSE}
\newcommand{\lcVIIvXfr}{\VERSE  De retour à la maison, ceux que le centurion avait envoyés trouvèrent guéri le serviteur qui avait été malade. \EVERSE}
\newcommand{\lcVIIvXIfr}{\VERSE  Il arriva ensuite que Jésus allait dans une ville appelée Naïm; et Ses disciples allaient avec Lui, ainsi qu'une foule nombreuse. \EVERSE}
\newcommand{\lcVIIvXIIfr}{\VERSE  Et comme Il approchait de la porte de la ville, voici qu'on emportait un mort, fils unique de sa mère, et celle-ci était veuve; et il y avait avec elle beaucoup de personnes de la ville. \EVERSE}
\newcommand{\lcVIIvXIIIfr}{\VERSE  Lorsque le Seigneur l'eut vue, touché de compassion pour elle, Il lui dit: Ne pleure point. \EVERSE}
\newcommand{\lcVIIvXIVfr}{\VERSE  Puis Il S'approcha, et toucha le cercueil.  Ceux qui le portaient s'arrêtèrent.  Et Il dit: Jeune homme, Je te l'ordonne, lève-toi. \EVERSE}
\newcommand{\lcVIIvXVfr}{\VERSE  Et le mort se mit sur son séant, et commença à parler.  Et Jésus le rendit à sa mère. \EVERSE}
\newcommand{\lcVIIvXVIfr}{\VERSE  Tous furent saisis de crainte, et ils glorifiaient Dieu, en disant: Un grand prophète a surgi parmi nous, et Dieu a visité Son peuple. \EVERSE}
\newcommand{\lcVIIvXVIIfr}{\VERSE  Et le bruit de ce miracle se répandit dans toute la Judée, et dans tout le pays d'alentour. \EVERSE}
\newcommand{\lcVIIvXVIIIfr}{\VERSE  Les disciples de Jean lui rapportèrent toutes ces choses. \EVERSE}
\newcommand{\lcVIIvXIXfr}{\VERSE  Et Jean appela deux de ses disciples, et les envoya vers Jésus, pour Lui dire: Etes-Vous Celui qui doit venir, ou devons-nous en attendre un autre? \EVERSE}
\newcommand{\lcVIIvXXfr}{\VERSE  Ces hommes, étant venus auprès de Jésus, Lui dirent:  Jean-Baptiste nous a envoyés vers Vous, pour Vous dire: Etes-Vous Celui qui doit venir, ou devons-nous en attendre un autre? \EVERSE}
\newcommand{\lcVIIvXXIfr}{\VERSE  A cette heure même, Il guérit beaucoup de personnes qui avaient des maladies, et des plaies, et des esprits mauvais, et Il rendit la vue à de nombreux aveugles. \EVERSE}
\newcommand{\lcVIIvXXIIfr}{\VERSE  Puis, leur répondant, Il dit: Allez, et rapportez à Jean ce que vous avez entendu et ce que vous avez vu: les aveugles voient, les boiteux marchent, les lépreux sont guéris, les sourds entendent, les morts ressuscitent, l'Evangile est annoncé aux pauvres; \EVERSE}
\newcommand{\lcVIIvXXIIIfr}{\VERSE  et bienheureux est celui qui ne sera pas scandalisé en Moi. \EVERSE}
\newcommand{\lcVIIvXXIVfr}{\VERSE  Lorsque les envoyés de Jean furent partis, Il Se mit à dire aux foules, au sujet de Jean: Qu'êtes-vous allés voir dans le désert?  Un roseau agité par le vent? \EVERSE}
\newcommand{\lcVIIvXXVfr}{\VERSE  Mais qu'êtes-vous allés voir?  Un homme vêtu avec mollesse?  Ceux qui portent des vêtements précieux et qui vivent dans les délices sont dans les maisons des rois. \EVERSE}
\newcommand{\lcVIIvXXVIfr}{\VERSE  Qu'êtes-vous donc allés voir?  Un prophète?  Oui, vous dis-Je, et plus qu'un prophète. \EVERSE}
\newcommand{\lcVIIvXXVIIfr}{\VERSE  C'est de lui qu'il est écrit: Voici que J'envoie Mon ange devant Ta face, et il préparera Ton chemin devant Toi. \EVERSE}
\newcommand{\lcVIIvXXVIIIfr}{\VERSE  Car, Je vous le dis, parmi ceux qui sont nés des femmes, nul n'est plus grand prophète que Jean-Baptiste.  Mais celui qui est le plus petit dans le royaume de Dieu est plus grand que lui. \EVERSE}
\newcommand{\lcVIIvXXIXfr}{\VERSE  Tout le peuple qui L'a entendu, et les publicains, ont justifié Dieu, en se faisant baptiser du baptême de Jean. \EVERSE}
\newcommand{\lcVIIvXXXfr}{\VERSE  Mais les pharisiens et les docteurs de la loi ont méprisé le dessein de Dieu à leur égard, en ne se faisant pas baptiser par Jean. \EVERSE}
\newcommand{\lcVIIvXXXIfr}{\VERSE  Le Seigneur ajouta: A qui donc comparerai-je les hommes de cette génération, et à qui sont-ils semblables? \EVERSE}
\newcommand{\lcVIIvXXXIIfr}{\VERSE  Ils sont semblables à des enfants assis sur la place publique, et qui, se parlant les uns aux autres, disent: Nous vous avons joué de la flûte, et vous n'avez pas dansé; nous avons chanté des airs lugubres, et vous n'avez pas pleuré. \EVERSE}
\newcommand{\lcVIIvXXXIIIfr}{\VERSE  Car Jean-baptiste est venu, ne mangeant pas de pain, et ne buvant pas de vin; et vous dites: Il est possédé du démon. \EVERSE}
\newcommand{\lcVIIvXXXIVfr}{\VERSE  Le Fils de l'homme est venu, mangeant et buvant; et vous dites: Voici un homme de bonne chère et un buveur de vin, un ami des publicains et des pécheurs. \EVERSE}
\newcommand{\lcVIIvXXXVfr}{\VERSE  Mais la sagesse a été justifiée par tous ses enfants. \EVERSE}
\newcommand{\lcVIIvXXXVIfr}{\VERSE  Or un pharisien pria Jésus de manger avec lui.  Et étant entré dans la maison du pharisien, Il Se mit à table. \EVERSE}
\newcommand{\lcVIIvXXXVIIfr}{\VERSE  Et voici qu'une femme, qui était une pécheresse dans la ville, ayant su qu'Il était à table dans la maison du pharisien, apporta un vase d'albâtre rempli de parfum; \EVERSE}
\newcommand{\lcVIIvXXXVIIIfr}{\VERSE  et se tenant derrière Lui, à Ses pieds, elle se mit à arroser Ses pieds de ses larmes, et elle les essuyait avec les cheveux de Sa tête, et elle baisait Ses pieds et les oignait de parfum. \EVERSE}
\newcommand{\lcVIIvXXXIXfr}{\VERSE  Voyant cela, le pharisien qui L'avait invité dit en lui-même: Si cet homme était prophète, Il saurait certainement qui et de quelle espèce est la femme qui Le touche; car c'est une pécheresse. \EVERSE}
\newcommand{\lcVIIvXLfr}{\VERSE  Et Jésus, prenant la parole, lui dit: Simon, J'ai quelque chose à te dire.  Il répondit: Maître, dites. \EVERSE}
\newcommand{\lcVIIvXLIfr}{\VERSE  Un créancier avait deux débiteurs: l'un devait cinq cents deniers, et l'autre cinquante. \EVERSE}
\newcommand{\lcVIIvXLIIfr}{\VERSE  Comme ils n'avaient pas de quoi les rendre, il leur remit à tous deux leur dette.  Lequel donc l'aimera davantage? \EVERSE}
\newcommand{\lcVIIvXLIIIfr}{\VERSE  Simon répondit: Je pense que c'est celui auquel il a remis davantage.  Jésus lui dit: Tu as bien jugé. \EVERSE}
\newcommand{\lcVIIvXLIVfr}{\VERSE  Et Se tournant vers la femme, il dit à Simon: Tu vois là cette femme?  Je suis entré dans ta maison: tu ne M'as pas donné d'eau pour Mes pieds; mais elle a arrosé Mes pieds des ses larmes, et elle les a essuyé avec ses cheveux. \EVERSE}
\newcommand{\lcVIIvXLVfr}{\VERSE  Tu ne M'as pas donné de baiser; mais elle, depuis qu'elle est entrée, n'a pas cessé de baiser Mes pieds. \EVERSE}
\newcommand{\lcVIIvXLVIfr}{\VERSE  Tu n'as pas oint Ma tête d'huile; mais elle, elle a oint Mes pieds de parfum. \EVERSE}
\newcommand{\lcVIIvXLVIIfr}{\VERSE  C'est pourquoi, Je te le dis, beaucoup de péchés lui sont remis, parce qu'elle a beaucoup aimé.  Mais celui à qui on remet moins, aime moins. \EVERSE}
\newcommand{\lcVIIvXLVIIIfr}{\VERSE  Alors Il dit à cette femme: Tes péchés te sont remis. \EVERSE}
\newcommand{\lcVIIvXLIXfr}{\VERSE  Et ceux qui étaient à table avec Lui commencèrent à dire en eux-mêmes: Quel est Celui-ci, qui remet même les péchés? \EVERSE}
\newcommand{\lcVIIvLfr}{\VERSE  Et Il dit à la femme: Ta foi t'a sauvée; va en paix. \EVERSE}
\newcommand{\lcVIIIvIfr}{\VERSE  Il arriva ensuite que Jésus parcourait les villes et les villages, prêchant et annonçant l'Evangile du royaume de Dieu.  Et les douze étaient avec Lui, \EVERSE}
\newcommand{\lcVIIIvIIfr}{\VERSE  comme aussi quelques femmes, qui avaient été guéries d'esprits malins et de maladies: Marie, appelée Madeleine, de laquelle sept démons étaient sortis; \EVERSE}
\newcommand{\lcVIIIvIIIfr}{\VERSE  Jeanne, femme de Chusa, intendant d'Hérode, et Suzanne, et beaucoup d'autres, qui L'assistaient de leurs biens. \EVERSE}
\newcommand{\lcVIIIvIVfr}{\VERSE  Or, comme une grande foule s'était assemblée, et qu'on accourait des villes auprès de Lui, Il dit en parabole: \EVERSE}
\newcommand{\lcVIIIvVfr}{\VERSE  Celui qui sème alla semer sa semence.  Et tandis qu'il semait, une partie tomba le long du chemin; et elle fut foulée aux pieds, et les oiseaux du ciel la mangèrent. \EVERSE}
\newcommand{\lcVIIIvVIfr}{\VERSE  Une autre partie tomba sur la pierre; et ayant levé, elle sécha, parce qu'elle n'avait pas d'humidité. \EVERSE}
\newcommand{\lcVIIIvVIIfr}{\VERSE  Une autre tomba au milieu des épines; et les épines, croissant avec elle, l'étouffèrent. \EVERSE}
\newcommand{\lcVIIIvVIIIfr}{\VERSE  Une autre partie tomba dans une bonne terre, et, ayant levé, elle porta du fruit au centuple.  En disant cela, Il criait: Que celui-là entende, qui a des oreilles pour entendre. \EVERSE}
\newcommand{\lcVIIIvIXfr}{\VERSE  Ses disciples Lui demandèrent ensuite ce que signifiait cette parabole. \EVERSE}
\newcommand{\lcVIIIvXfr}{\VERSE  Il leur dit: A vous il a été donné de connaître le mystère du royaume de Dieu; mais aux autres il n'est proposé qu'en paraboles, afin que, regardant, ils ne voient point, et qu'entendant, ils ne comprennent point. \EVERSE}
\newcommand{\lcVIIIvXIfr}{\VERSE  Voici le sens de cette parabole.  La semence, c'est la parole de Dieu. \EVERSE}
\newcommand{\lcVIIIvXIIfr}{\VERSE  Ceux qui sont le long du chemin sont ceux qui écoutent; ensuite le diable vient, et enlève de leur coeur la parole, de peur qu'ils ne croient et ne soient sauvées. \EVERSE}
\newcommand{\lcVIIIvXIIIfr}{\VERSE  Ceux qui sont sur la pierre sont ceux qui, entendant la parole, la reçoivent avec joie; mais ils n'ont pas de racines: ils croient pour un temps, et au moment de la tentation ils se retirent. \EVERSE}
\newcommand{\lcVIIIvXIVfr}{\VERSE  Ce qui tombe parmi les épines, ce sont ceux qui ont écouté la parole, et qui s'en vont et sont étouffés par les sollicitudes, les richesses et les plaisirs de la vie, et ils ne portent pas de fruit. \EVERSE}
\newcommand{\lcVIIIvXVfr}{\VERSE  Ce qui tombe dans la bonne terre, ce sont ceux qui, ayant écouté la parole avec un coeur bon et excellent, la retiennent, et portent du fruit par la patience. \EVERSE}
\newcommand{\lcVIIIvXVIfr}{\VERSE  Personne, après avoir allumé une lampe, ne la couvre d'un vase ou ne la met sous un lit; mais il la met sur un candélabre, afin que ceux qui entrent voient la lumière. \EVERSE}
\newcommand{\lcVIIIvXVIIfr}{\VERSE  Car il n'y a rien de caché qui ne soit manifesté, ni rien de secret qui ne soit connu et ne vienne au grand jour. \EVERSE}
\newcommand{\lcVIIIvXVIIIfr}{\VERSE  Prenez donc garde à la manière dont vous écoutez.  Car à celui qui a, on donnera; et à celui qui n'a pas, on ôtera même ce qu'il croit avoir. \EVERSE}
\newcommand{\lcVIIIvXIXfr}{\VERSE  Cependant, Sa Mère et Ses frères vinrent auprès de Lui, et ils ne pouvaient L'aborder, à cause de la foule. \EVERSE}
\newcommand{\lcVIIIvXXfr}{\VERSE  On L'en avertit: Votre Mère et Vos frères sont dehors et veulent Vous voir. \EVERSE}
\newcommand{\lcVIIIvXXIfr}{\VERSE  Et répondant, Il leur dit: Ma mère et Mes frères, ce sont ceux qui écoutent la parole de Dieu, et qui la pratiquent. \EVERSE}
\newcommand{\lcVIIIvXXIIfr}{\VERSE  Or il arriva qu'un de ces jours, Il monta sur une barque avec Ses disciples; et Il leur dit: Passons de l'autre côté du lac.  Et ils partirent. \EVERSE}
\newcommand{\lcVIIIvXXIIIfr}{\VERSE  Pendant qu'ils naviguaient, Il S'endormit; et un tourbillon de vent fondit sur le lac, et la barque se remplissait d'eau, et ils étaient en péril. \EVERSE}
\newcommand{\lcVIIIvXXIVfr}{\VERSE  S'approchant donc, ils L'éveillèrent, en disant: Maître, nous périssons.  Mais Lui, S'étant levé, menaça le vent et les flots agités; et ils s'apaisèrent, et le calme se fit. \EVERSE}
\newcommand{\lcVIIIvXXVfr}{\VERSE  Alors Il leur dit: Où est votre foi?  Mais eux, remplis de crainte et d'admiration, se disaient l'un à l'autre: Quel est donc Celui-ci, qui commande aux vents et à la mer, et ils Lui obéissent? \EVERSE}
\newcommand{\lcVIIIvXXVIfr}{\VERSE  Ils abordèrent dans le pays des Géraséniens, qui est en face de la Galilée. \EVERSE}
\newcommand{\lcVIIIvXXVIIfr}{\VERSE  Et lorsque Jésus fut descendu à terre, il vint au-devant de Lui un homme qui était possédé du démon depuis longtemps déjà, et qui ne portait pas de vêtement, et qui ne demeurait pas dans une maison, mais dans les sépulcres. \EVERSE}
\newcommand{\lcVIIIvXXVIIIfr}{\VERSE  Dès qu'il eut vu Jésus, il se prosterna devant Lui, et poussant un grand cri, il dit: Qu'y a-t-il entre Vous et moi, Jésus, Fils du Dieu très haut?  Je Vous en conjure, ne me tourmentez pas. \EVERSE}
\newcommand{\lcVIIIvXXIXfr}{\VERSE  Car Il commandait à l'esprit impur de sortir de cet homme.  Il s'était, en effet, emparé de lui depuis longtemps, et quoiqu'on le gardât lié de chaînes et les fers aux pieds, il rompait ses liens, et était entraîné par le démon dans les déserts. \EVERSE}
\newcommand{\lcVIIIvXXXfr}{\VERSE  Jésus l'interrogea, en disant: Quel est ton nom?  Il répondit: Légion; car de nombreux démons étaient entrés en lui. \EVERSE}
\newcommand{\lcVIIIvXXXIfr}{\VERSE  Et ils Le suppliaient de ne pas leur commander de s'en aller dans l'abîme. \EVERSE}
\newcommand{\lcVIIIvXXXIIfr}{\VERSE  Or il y avait là un grand troupeau de pourceaux, qui paissaient sur la montagne; et les démons Le suppliaient de leur permettre d'entrer dans ces pourceaux.  Et Il le leur permit. \EVERSE}
\newcommand{\lcVIIIvXXXIIIfr}{\VERSE  Les démons sortirent donc de cet homme, et entrèrent dans les pourceaux; et le troupeau alla se précipiter impétueusement dans le lac, et se noya. \EVERSE}
\newcommand{\lcVIIIvXXXIVfr}{\VERSE  Quand ceux qui les faisaient paître eurent vu ce qui était arrivé, ils s'enfuirent, et ils l'annoncèrent dans la ville et dans les campagnes. \EVERSE}
\newcommand{\lcVIIIvXXXVfr}{\VERSE  Les habitants sortirent pour voir ce qui était arrivé, et ils vinrent auprès de Jésus; et ils trouvèrent l'homme, de qui les démons étaient sortis, assis à Ses pieds, vêtu, et plein de bons sens; et ils furent saisis de crainte. \EVERSE}
\newcommand{\lcVIIIvXXXVIfr}{\VERSE  Ceux qui avaient vu ce qui s'était passé leur racontèrent comment il avait été délivré de la légion. \EVERSE}
\newcommand{\lcVIIIvXXXVIIfr}{\VERSE  Alors tout le peuple du pays des Géraséniens pria Jésus de S'éloigner d'eux, car ils étaient saisis d'une grande crainte.  Et Lui, montant dans la barque, S'en retourna. \EVERSE}
\newcommand{\lcVIIIvXXXVIIIfr}{\VERSE  Et l'homme de qui les démons étaient sortis Lui demandait de rester avec Lui.  Mais Jésus le renvoya, en disant: \EVERSE}
\newcommand{\lcVIIIvXXXIXfr}{\VERSE  Retourne dans ta maison, et raconte les grandes choses que Dieu t'a faites.  Et il s'en alla par toute la ville, publiant les grandes choses que Jésus lui avait faites. \EVERSE}
\newcommand{\lcVIIIvXLfr}{\VERSE  Or il arriva que Jésus, à Son retour, fut reçu par la foule: car tous L'attendaient. \EVERSE}
\newcommand{\lcVIIIvXLIfr}{\VERSE  Et voici qu'un homme, nommé Jaïre, qui était chef de la synagogue, vint et se jeta aux pieds de Jésus, Le suppliant d'entrer dans sa maison, \EVERSE}
\newcommand{\lcVIIIvXLIIfr}{\VERSE  parce qu'il avait une fille unique, âgée d'environ douze ans, qui se mourait.  Et il arriva qu'en y allant Il était pressé par la foule. \EVERSE}
\newcommand{\lcVIIIvXLIIIfr}{\VERSE  Et une femme qui souffrait d'une perte de sang dupuis douze ans, et qui avait dépensé tout son bien en médecins, sans qu'aucun eût pu la guérir, \EVERSE}
\newcommand{\lcVIIIvXLIVfr}{\VERSE  s'approcha par derrière, et toucha la frange de Son vêtement; et aussitôt sa perte de sang s'arrêta. \EVERSE}
\newcommand{\lcVIIIvXLVfr}{\VERSE  Et Jésus dit: Qui est-ce qui M'a touché?  Mais comme tous s'en défendaient, Pierre et ceux qui étaient avec Lui répondirent: Maître, les foules Vous pressent et Vous accablent, et Vous dites: Qui M'a touché? \EVERSE}
\newcommand{\lcVIIIvXLVIfr}{\VERSE  Et Jésus dit: Quelqu'un M'a touché, car J'ai connu qu'une vertu était sortie de Moi. \EVERSE}
\newcommand{\lcVIIIvXLVIIfr}{\VERSE  Alors la femme, voyant qu'elle n'avait pu rester cachée, vint toute tremblante, et se jeta à Ses pieds; et elle déclara devant tout le peuple pour quel motif elle L'avait touché, et comment elle avait été guérie à l'instant. \EVERSE}
\newcommand{\lcVIIIvXLVIIIfr}{\VERSE  Et Jésus lui dit: Ma fille, ta foi t'a sauvée; va en paix. \EVERSE}
\newcommand{\lcVIIIvXLIXfr}{\VERSE  Comme Il parlait encore, quelqu'un vint dire au chef de synagogue: Ta fille est morte; ne l'importune pas. \EVERSE}
\newcommand{\lcVIIIvLfr}{\VERSE  Mais Jésus, ayant entendu cette parole, dit au père de la jeune fille: Ne crains point; crois seulement, et elle vivra. \EVERSE}
\newcommand{\lcVIIIvLIfr}{\VERSE  Et lorsqu'Il fut arrivé à la maison, Il ne permit à personne d'entrer avec Lui, si ce n'est à Pierre, à Jacques et à Jean, et au père et à la mère de la jeune fille. \EVERSE}
\newcommand{\lcVIIIvLIIfr}{\VERSE  Or, tous pleuraient et se lamentaient sur elle.  Mais Il dit: Ne pleurez pas; la jeune fille n'est pas morte, mais elle dort. \EVERSE}
\newcommand{\lcVIIIvLIIIfr}{\VERSE  Et ils se moquaient de Lui, sachant qu'elle était morte. \EVERSE}
\newcommand{\lcVIIIvLIVfr}{\VERSE  Mais Lui, la prenant par la main, S'écria, en disant: Jeune fille, lève-toi. \EVERSE}
\newcommand{\lcVIIIvLVfr}{\VERSE  Et son esprit revint, et elle se leva aussitôt.  Et Il ordonna de lui donner à manger. \EVERSE}
\newcommand{\lcVIIIvLVIfr}{\VERSE  Ses parents furent remplis d'étonnement; et Il leur commanda de ne dire à personne ce qui était arrivé. \EVERSE}
\newcommand{\lcIXvIfr}{\VERSE  Jésus, ayant assemblé les douze Apôtres, leur donna puissance et autorité sur tous les démons, et le pouvoir de guérir les maladies. \EVERSE}
\newcommand{\lcIXvIIfr}{\VERSE  Puis Il les envoya prêcher le royaume de Dieu et guérir les malades. \EVERSE}
\newcommand{\lcIXvIIIfr}{\VERSE  Et Il leur dit: Ne portez rien en route, ni bâton, ni sac, ni pain, ni argent, et n'ayez pas deux tuniques. \EVERSE}
\newcommand{\lcIXvIVfr}{\VERSE  Dans quelque maison que vous soyez entrés, demeurez-y et n'en sortez pas. \EVERSE}
\newcommand{\lcIXvVfr}{\VERSE  Et lorsqu'on ne vous aura pas reçus, sortant de cette ville, secouez la poussière même de vos pieds, en témoignage contre eux. \EVERSE}
\newcommand{\lcIXvVIfr}{\VERSE  Etant donc partis, ils parcouraient les villages, annonçant l'Evangile et guérissant partout. \EVERSE}
\newcommand{\lcIXvVIIfr}{\VERSE  Cependant, Hérode le tétrarque entendit parler de tout ce que faisait Jésus; et il était perplexe, parce que les uns disaient: \EVERSE}
\newcommand{\lcIXvVIIIfr}{\VERSE  Jean est ressuscité d'entre les morts; les autres: Elie est apparu; et d'autres: Un des anciens prophètes est ressuscité. \EVERSE}
\newcommand{\lcIXvIXfr}{\VERSE  Et Hérode dit: J'ai décapité Jean; mais quel est donc Celui-ci, de qui j'entends dire de telles choses?  Et il cherchait à Le voir. \EVERSE}
\newcommand{\lcIXvXfr}{\VERSE  Les Apôtres, étant revenus, racontèrent à Jésus tout ce qu'ils avaient fait; et les prenant avec Lui, Il Se retira à l'écart dans un lieu désert, près de Bethsaïda. \EVERSE}
\newcommand{\lcIXvXIfr}{\VERSE  Quand les foules l'eurent appris, elles Le suivirent; et Il les accueillit, et Il leur parlait du royaume de Dieu, et guérissait ceux qui avaient besoin d'être guéris. \EVERSE}
\newcommand{\lcIXvXIIfr}{\VERSE  Or, le jour commençait à baisser, et les douze, s'approchant, Lui dirent: Renvoyez les foules, afin qu'elles aillent dans les villages et dans les campagnes d'alentour, pour se loger et trouver des vivres; car nous sommes ici dans un lieu désert. \EVERSE}
\newcommand{\lcIXvXIIIfr}{\VERSE  Mais Il leur dit: Donnez-leur vous-mêmes à manger.  Ils Lui dirent: Nous n'avons que cinq pains et deux poissons; à moins que nous n'allions nous-mêmes acheter des vivres pour toute cette foule. \EVERSE}
\newcommand{\lcIXvXIVfr}{\VERSE  Or il y avait là environ cinq mille hommes.  Alors Il dit à Ses disciples: Faites-les asseoir par groupes de cinquante. \EVERSE}
\newcommand{\lcIXvXVfr}{\VERSE  Ils firent ainsi, et les firent tous asseoir. \EVERSE}
\newcommand{\lcIXvXVIfr}{\VERSE  Alors Jésus, ayant pris les cinq pains et les deux poissons, leva les yeux au Ciel, et les bénit, les rompit, et les distribua à Ses disciples, afin qu'ils les présentassent aux foules. \EVERSE}
\newcommand{\lcIXvXVIIfr}{\VERSE  Ils mangèrent tous et furent rassasiés; et on emporta douze corbeilles de morceaux qui étaient restés. \EVERSE}
\newcommand{\lcIXvXVIIIfr}{\VERSE  Il arriva, comme Il priait à l'écart, ayant Ses disciples avec Lui, qu'Il les interrogea, en disant: Les foules, qui disent-elles que Je suis? \EVERSE}
\newcommand{\lcIXvXIXfr}{\VERSE  Ils répondirent, en disant: Jean-Baptiste; les autres, Elie; les autres, qu'un des anciens prophètes est ressuscité. \EVERSE}
\newcommand{\lcIXvXXfr}{\VERSE  Et Il leur dit: Mais vous, qui dites-vous que Je suis?  Simon-Pierre, prenant la parole, dit: Le Christ de Dieu. \EVERSE}
\newcommand{\lcIXvXXIfr}{\VERSE  Alors Il leur défendit, avec de sévères recommandations, de dire cela à personne, \EVERSE}
\newcommand{\lcIXvXXIIfr}{\VERSE  ajoutant: Il faut que le Fils de l'homme souffre beaucoup, qu'Il soit rejeté par les anciens, par les princes des prêtres et par les scribes, qu'Il soit mis à mort, et qu'Il ressuscite le troisième jour. \EVERSE}
\newcommand{\lcIXvXXIIIfr}{\VERSE  Il disait aussi à tous: Si quelqu'un veut venir après Moi, qu'il renonce à lui-même, et qu'il porte sa croix tous les jours, et qu'il Me suive. \EVERSE}
\newcommand{\lcIXvXXIVfr}{\VERSE  Car celui qui voudra sauver sa vie la perdra, et celui qui perdra sa vie à cause de Moi la sauvera. \EVERSE}
\newcommand{\lcIXvXXVfr}{\VERSE  Et quel avantage aurait un homme à gagner le monde entier, s'il se perd lui-même et cause sa ruine? \EVERSE}
\newcommand{\lcIXvXXVIfr}{\VERSE  Car si quelqu'un rougit de Moi et de Mes paroles, le Fils de l'homme rougira de lui lorsqu'Il viendra dans Sa gloire, et dans celle du Père et des saints Anges. \EVERSE}
\newcommand{\lcIXvXXVIIfr}{\VERSE  Je vous le dis, en vérité, il en est quelques-uns, ici présents, qui ne goûteront pas la mort avant d'avoir vu le royaume de Dieu. \EVERSE}
\newcommand{\lcIXvXXVIIIfr}{\VERSE  Or il arriva qu'environ huit jours après ces paroles, Il prit avec Lui Pierre, Jacques et Jean, et Il monta sur une montagne pour prier. \EVERSE}
\newcommand{\lcIXvXXIXfr}{\VERSE  Et pendant qu'Il priait, l'aspect de Son visage devint tout autre, et Ses vêtements devinrent blancs et brillants. \EVERSE}
\newcommand{\lcIXvXXXfr}{\VERSE  Et voici que deux hommes s'entretenaient avec Lui: c'étaient Moïse et Elie, \EVERSE}
\newcommand{\lcIXvXXXIfr}{\VERSE  apparaissant avec gloire; et ils parlaient de Sa sortie du monde, qu'Il devait accomplir à Jérusalem. \EVERSE}
\newcommand{\lcIXvXXXIIfr}{\VERSE  Cependant Pierre et ceux qui étaient avec Lui étaient appesantis par le sommeil; et, s'éveillant, ils virent Sa gloire, et les deux hommes qui étaient avec Lui. \EVERSE}
\newcommand{\lcIXvXXXIIIfr}{\VERSE  Et il arriva qu'au moment où ceux-ci s'éloignaient de Jésus, Pierre Lui dit: Maître, il est bon pour nous d'être ici; faisons trois tentes, une pour vous, une pour Moïse et une pour Elie.  Il ne savait pas ce qu'il disait. \EVERSE}
\newcommand{\lcIXvXXXIVfr}{\VERSE  Comme il parlait ainsi, une nuée apparut et les couvrit; et ils furent effrayés lorsqu'ils entrèrent dans la nuée. \EVERSE}
\newcommand{\lcIXvXXXVfr}{\VERSE  Et une voix sortit de la nuée, disant: Celui-ci est Mon Fils bien-aimé; écoutez-Le. \EVERSE}
\newcommand{\lcIXvXXXVIfr}{\VERSE  Et pendant que la voix retentissait, Jésus Se trouva seul.  Et les disciples se turent, et ne dirent à personne, en ces jours-là, rien de ce qu'ils avaient vu. \EVERSE}
\newcommand{\lcIXvXXXVIIfr}{\VERSE  Or il arriva, le jour suivant, comme ils descendaient de la montagne, qu'une foule nombreuse vint au-devant d'eux. \EVERSE}
\newcommand{\lcIXvXXXVIIIfr}{\VERSE  Et voici qu'un homme s'écria, du sein de la foule, et dit: Maître, je Vous en supplie, jetez un regard sur mon fils, car c'est mon unique enfant. \EVERSE}
\newcommand{\lcIXvXXXIXfr}{\VERSE  Un esprit se saisit de lui, et aussi-tôt il pousse des cris; il le renverse à terre, il l'agite en le faisant écumer, et il ne le quitte qu'à grand'peine, après l'avoir tout déchiré. \EVERSE}
\newcommand{\lcIXvXLfr}{\VERSE  J'ai prié Vos disciples de le chasser, et ils n'ont pas pu. \EVERSE}
\newcommand{\lcIXvXLIfr}{\VERSE  Alors Jésus, prenant la parole, dit: O race incrédule et perverse, jusques à quand serai-Je avec vous et vous souffrirai-Je?  Amène ici ton fils. \EVERSE}
\newcommand{\lcIXvXLIIfr}{\VERSE  Et comme il approchait, le démon le jeta par terre et l'agita violemment. \EVERSE}
\newcommand{\lcIXvXLIIIfr}{\VERSE  Mais Jésus menaça l'esprit impur, et guérit l'enfant, et le rendit à son père. \EVERSE}
\newcommand{\lcIXvXLIVfr}{\VERSE  Et tous étaient frappés de la grandeur de Dieu; et comme tous étaient dans l'admiration de tout ce que faisait Jésus, Il dit à Ses disciples: Vous, mettez bien dans vos coeurs ces paroles: Le Fils de l'homme doit être livré entre les mains des hommes. \EVERSE}
\newcommand{\lcIXvXLVfr}{\VERSE  Mais ils ne comprenaient pas cette parole, et elle était voilée pour eux, de sorte qu'ils n'en avaient pas le sens; et ils craignaient de L'interroger à ce sujet. \EVERSE}
\newcommand{\lcIXvXLVIfr}{\VERSE  Or une pensée leur vint dans l'esprit: lequel d'entre eux était le plus grand. \EVERSE}
\newcommand{\lcIXvXLVIIfr}{\VERSE  Mais Jésus, voyant les pensées de leurs coeurs, prit un enfant et le plaça auprès de Lui. \EVERSE}
\newcommand{\lcIXvXLVIIIfr}{\VERSE  Puis Il leur dit: Quiconque reçoit cet enfant en Mon nom, Me reçoit; et quiconque Me reçoit, reçoit Celui qui M'a envoyé.  Car celui qui est le plus petit parmi vous tous, celui-là est le plus grand. \EVERSE}
\newcommand{\lcIXvXLIXfr}{\VERSE  Alors Jean, prenant la parole, dit: Maître, nous avons vu un homme chasser les démons en Votre nom, et nous l'en avons empêché, parce qu'il ne Vous suit pas avec nous. \EVERSE}
\newcommand{\lcIXvLfr}{\VERSE  Et Jésus lui dit: Ne l'en empêchez point; car celui qui n'est pas contre vous est pour vous. \EVERSE}
\newcommand{\lcIXvLIfr}{\VERSE  Or il arriva, lorsque les jours où Il devait être enlevé du monde approchaient, qu'Il prit un visage assuré, pour aller à Jérusalem. \EVERSE}
\newcommand{\lcIXvLIIfr}{\VERSE  Et Il envoya devant Lui des messagers; ceux-ci, étant partis, entrèrent dans une ville des Samaritains, pour Lui préparer un logement. \EVERSE}
\newcommand{\lcIXvLIIIfr}{\VERSE  Mais ils ne Le reçurent point, parce que Son aspect était celui d'un homme qui va à Jérusalem. \EVERSE}
\newcommand{\lcIXvLIVfr}{\VERSE  Ayant vu cela, Ses disciples Jacques et Jean Lui dirent: Seigneur, voulez-Vous que nous commandions que le feu descende du ciel et les consume? \EVERSE}
\newcommand{\lcIXvLVfr}{\VERSE  Et Se tournant vers eux, Il les réprimanda, en disant: Vous ne savez pas de quel esprit Vous êtes. \EVERSE}
\newcommand{\lcIXvLVIfr}{\VERSE  Le Fils de l'homme n'est pas venu pour perdre les âmes, mais pour les sauver.  Et ils s'en allèrent dans un autre bourg. \EVERSE}
\newcommand{\lcIXvLVIIfr}{\VERSE  Or il arriva, tandis qu'ils étaient en chemin, que quelqu'un Lui dit: Je Vous suivrai partout où Vous irez. \EVERSE}
\newcommand{\lcIXvLVIIIfr}{\VERSE  Jésus lui répondit: Les renards ont des tanières, et les oiseaux du ciel des nids; mais le Fils de l'homme n'a pas où reposer Sa tête. \EVERSE}
\newcommand{\lcIXvLIXfr}{\VERSE  Il dit à un autre: Suis-moi.  Mais celui-ci répondit: Seigneur, permettez-moi d'aller d'abord ensevelir mon père. \EVERSE}
\newcommand{\lcIXvLXfr}{\VERSE  Et Jésus lui dit: Laisse les morts ensevelir leurs morts; pour toi, va et annonce le royaume de Dieu. \EVERSE}
\newcommand{\lcIXvLXIfr}{\VERSE  Un autre dit: Seigneur, je Vous suivrai; mais permettez-moi d'abord de disposer de ce qui est dans ma maison. \EVERSE}
\newcommand{\lcIXvLXIIfr}{\VERSE  Jésus lui dit: Quiconque met la main à la charrue et regarde en arriére, n'est pas propre au royaume de Dieu. \EVERSE}
\newcommand{\lcXvIfr}{\VERSE  Après cela, le Seigneur désigna encore soixante-douze autres disciples, et Il les envoya devant Lui, deux à deux, dans toutes les villes et tous les lieux où Il devait aller Lui-même. \EVERSE}
\newcommand{\lcXvIIfr}{\VERSE  Et Il leur disait: La moisson est grande, mais les ouvriers sont peu nombreux.  Priez donc le Maître de la moisson d'envoyer des ouvriers dans Sa moisson. \EVERSE}
\newcommand{\lcXvIIIfr}{\VERSE  Allez; voici que Je vous envoie comme des agneaux au milieu des loups. \EVERSE}
\newcommand{\lcXvIVfr}{\VERSE  Ne portez ni bourse, ni sac, ni chaussures, et ne saluez personne en chemin. \EVERSE}
\newcommand{\lcXvVfr}{\VERSE  Dans quelque maison que vous entriez, dites d'abord: Paix à cette maison. \EVERSE}
\newcommand{\lcXvVIfr}{\VERSE  Et s'il s'y trouve un enfant de paix, votre paix reposera sur lui; sinon, elle reviendra à vous. \EVERSE}
\newcommand{\lcXvVIIfr}{\VERSE  Demeurez dans la même maison, mangeant et buvant de ce qu'il y aura chez eux; car l'ouvrier est digne de son salaire.  Ne passez pas de maison en maison. \EVERSE}
\newcommand{\lcXvVIIIfr}{\VERSE  Dans quelque ville que vous entriez, et où l'on vous recevra, mangez ce qui vous sera présenté. \EVERSE}
\newcommand{\lcXvIXfr}{\VERSE  Guérissez les malades qui s'y trouvent, et dites-leur: Le royaume de Dieu s'est approché de vous. \EVERSE}
\newcommand{\lcXvXfr}{\VERSE  Et dans quelque ville que vous entriez, et où l'on ne vous recevra pas, sortez sur les places publiques, et dites: \EVERSE}
\newcommand{\lcXvXIfr}{\VERSE  La poussière même de votre ville, qui s'est attachée à nous, nous la secouons contre vous; sachez cependant ceci, que le royaume de Dieu est proche. \EVERSE}
\newcommand{\lcXvXIIfr}{\VERSE  Je vous le dis, en ce jour-là, il y aura moins de rigueur pour Sodome que pour cette ville. \EVERSE}
\newcommand{\lcXvXIIIfr}{\VERSE  Malheur à toi, Corozaïn!  malheur à toi, Bethsaïda!  car si les miracles qui ont été faits au milieu de vous avaient été faits dans Tyr et dans Sidon, depuis longtemps elles auraient fait pénitence, revêtues d'un sac et assises dans la cendre. \EVERSE}
\newcommand{\lcXvXIVfr}{\VERSE  C'est pourquoi, au jugement, il y aura moins de rigueur pour Tyr et pour Sidon que pour vous. \EVERSE}
\newcommand{\lcXvXVfr}{\VERSE  Et toi, Capharnaüm, qui as été élevée jusqu'au Ciel, tu seras plongée jusque dans l'enfer. \EVERSE}
\newcommand{\lcXvXVIfr}{\VERSE  Celui qui vous écoute, M'écoute; celui qui vous méprise, Me méprise.  Et celui qui Me méprise, méprise Celui qui M'a envoyé. \EVERSE}
\newcommand{\lcXvXVIIfr}{\VERSE  Or les soixante-douze revinrent avec joie, disant: Seigneur, les démons même nous sont soumis en Votre nom. \EVERSE}
\newcommand{\lcXvXVIIIfr}{\VERSE  Et Il leur dit: Je voyais Satan tomber du Ciel comme la foudre. \EVERSE}
\newcommand{\lcXvXIXfr}{\VERSE  Voici que Je vous ai donné le pouvoir de fouler aux pieds les serpents, et les scorpions, et toute la puissance de l'ennemi; et rien ne pourra vous nuire. \EVERSE}
\newcommand{\lcXvXXfr}{\VERSE  Cependant, ne vous réjouissez pas de ce que les esprits vous sont soumis; mais réjouissez-vous de ce que vos noms sont écrits dans les Cieux. \EVERSE}
\newcommand{\lcXvXXIfr}{\VERSE  En cette heure même, Il tressaillit de joie dans l'Esprit-Saint, et dit: Je vous rends gloire, Père, Seigneur du ciel et de la terre, de ce que Vous avez caché ces choses aux sages et aux prudents, et de ce que Vous les avez révélées aux petits.  Oui, Père, car il Vous a plu ainsi. \EVERSE}
\newcommand{\lcXvXXIIfr}{\VERSE  Toutes choses M'ont été données par mon Père; et nul ne sait qui est le Fils, si ce n'est le Pére; ni qui est le Pére, si ce n'est le Fils, et celui à qui le Fils aura voulu le révéler. \EVERSE}
\newcommand{\lcXvXXIIIfr}{\VERSE  Et Se tournant vers Ses disciples, Il dit: Heureux les yeux qui voient ce que vous voyez. \EVERSE}
\newcommand{\lcXvXXIVfr}{\VERSE  Car Je vous le dis, beaucoup de prophètes et de rois ont voulu voir ce que vous voyez, et ne l'ont pas vu; et entendre ce que vous entendez, et ne l'ont pas entendu. \EVERSE}
\newcommand{\lcXvXXVfr}{\VERSE  Et voici qu'un docteur de la loi se leva pour Le tenter, et Lui dit: Maître, que dois-je faire pour posséder la vie éternelle? \EVERSE}
\newcommand{\lcXvXXVIfr}{\VERSE  Et Jêsus lui dit: Qu'y a-t-il d'écrit dans la loi?  qu'y lis-tu? \EVERSE}
\newcommand{\lcXvXXVIIfr}{\VERSE  Il répondit: Tu aimeras le Seigneur ton Dieu de tout ton coeur, et de toute ton âme, et de toutes tes forces, et de tout ton esprit; et ton prochain comme toi-même. \EVERSE}
\newcommand{\lcXvXXVIIIfr}{\VERSE  Jésus lui dit: Tu as bien répondu; fais cela, et tu vivras. \EVERSE}
\newcommand{\lcXvXXIXfr}{\VERSE  Mais lui, voulant se justifier, dit à Jésus: Et qui est mon prochain? \EVERSE}
\newcommand{\lcXvXXXfr}{\VERSE  Alors Jésus, prenant la parole, dit: Un homme descendait de Jérusalem à Jéricho, et il tomba au milieu des voleurs, qui le dépouillèrent, et, après l'avoir couvert de blessures, s'en allèrent, le laissant à demi mort. \EVERSE}
\newcommand{\lcXvXXXIfr}{\VERSE  Or il arriva qu'un prêtre descendait par le même chemin; et l'ayant vu, il passa outre. \EVERSE}
\newcommand{\lcXvXXXIIfr}{\VERSE  Pareillement, un lévite, qui se trouvait en cet endroit, le vit et passa outre. \EVERSE}
\newcommand{\lcXvXXXIIIfr}{\VERSE  Mais un Samaritain, qui était en voyage, vint près de lui, et, le voyant, fut touché de compassion. \EVERSE}
\newcommand{\lcXvXXXIVfr}{\VERSE  Et s'étant approché, il banda ses plaies, et y versa de l'huile et du vin; puis, le plaçant sur sa monture, il le conduisit dans une hôtellerie et prit soin de lui. \EVERSE}
\newcommand{\lcXvXXXVfr}{\VERSE  Le lendemain, il tira deux deniers, et les donna à l'hôtelier, et dit: Aie soin de lui; et tout ce que tu dépenseras de plus, je te le rendrai à mon retour. \EVERSE}
\newcommand{\lcXvXXXVIfr}{\VERSE  Lequel de ces trois te semble avoir été le prochain de celui qui était tombé entre les mains des voleurs? \EVERSE}
\newcommand{\lcXvXXXVIIfr}{\VERSE  Le docteur répondit: Celui qui a exercé la miséricorde envers lui.  Et Jésus lui dit: Va, et fais de même. \EVERSE}
\newcommand{\lcXvXXXVIIIfr}{\VERSE  Or il arriva, tandis qu'ils étaient en chemin, qu'Il entra dans un bourg; et une femme, nommée Marthe, Le reçut dans sa maison. \EVERSE}
\newcommand{\lcXvXXXIXfr}{\VERSE  Et elle avait une soeur, nommée Marie, qui, assise aux pieds du Seigneur, écoutait Sa parole; \EVERSE}
\newcommand{\lcXvXLfr}{\VERSE  mais Marthe s'empressait aux soins multiples du service.  Elle s'arrêta, et dit: Seigneur, n'avez-Vous aucun souci de ce que ma soeur me laisse servir seule?  Dites-lui donc de m'aider. \EVERSE}
\newcommand{\lcXvXLIfr}{\VERSE  Le Seigneur, répondant, lui dit: Marthe, Marthe, tu t'inquètes et tu te troubles pour beaucoup de choses. \EVERSE}
\newcommand{\lcXvXLIIfr}{\VERSE  Or une seule chose est nécessaire.  Marie a choisi la meilleure part, qui ne lui sera pas ôtée. \EVERSE}
\newcommand{\lcXIvIfr}{\VERSE  Il arriva, comme Il priait dans un certain lieu, que, lorsqu'Il eut achevé, un de Ses disciples Lui dit: Seigneur, apprenez-nous à prier, comme Jean l'a appris à ses disciples. \EVERSE}
\newcommand{\lcXIvIIfr}{\VERSE  Et Il leur dit: Lorsque vous priez, dites: Père, que Votre nom soit sanctifié; que Votre règne arrive; \EVERSE}
\newcommand{\lcXIvIIIfr}{\VERSE  donnez-nous aujourd'hui notre pain de chaque jour. \EVERSE}
\newcommand{\lcXIvIVfr}{\VERSE  Et remettez-nous nos péchés, puisque nous remettons, nous aussi, à quiconque nous doit; et ne nous induisez pas en tentation. \EVERSE}
\newcommand{\lcXIvVfr}{\VERSE  Il leur dit encore: Si l'un de vous a un ami, et qu'il aille le trouver au milieu de la nuit, pour lui dire: Mon ami, prête-moi trois pains, \EVERSE}
\newcommand{\lcXIvVIfr}{\VERSE  car un de mes amis est arrivé de voyage chez moi, et je n'ai rien à lui offrir, \EVERSE}
\newcommand{\lcXIvVIIfr}{\VERSE  et si, de l'intérieur, l'autre répond: Ne m'importune pas; la porte est déjà fermée, et mes enfants et moi nous sommes au lit; je ne puis me lever pour t'en donner; \EVERSE}
\newcommand{\lcXIvVIIIfr}{\VERSE  si cependant le premier continue de frapper, Je vous le dis, quand même il ne se lèverait pas pour lui en donner parce qu'il est son ami, il se lèvera du moins à cause de son importunité, et il lui en donnera autant qu'il lui en faut. \EVERSE}
\newcommand{\lcXIvIXfr}{\VERSE  Et Moi, Je vous dis: Demandez, et on vous donnera; cherchez, et vous trouverez; frappez à la porte, et on vous ouvrira. \EVERSE}
\newcommand{\lcXIvXfr}{\VERSE  Car quiconque demande, reçoit; et qui cherche, trouve; et à celui qui frappe à la porte, on ouvrira. \EVERSE}
\newcommand{\lcXIvXIfr}{\VERSE  Si l'un de vous demande du pain à son père, celui-ci lui donnera-t-il une pierre?  Ou, s'il demande un poisson, lui donnera-t-il un serpent au lieu du poisson? \EVERSE}
\newcommand{\lcXIvXIIfr}{\VERSE  Ou, s'il demande un oeuf, lui donnera-t-il un scorpion? \EVERSE}
\newcommand{\lcXIvXIIIfr}{\VERSE  Si donc vous, qui êtes méchants, vous savez donner de bonnes choses à vos enfants, à combien plus forte raison votre Père qui est dans le Ciel donnera-t-Il l'Esprit bon à ceux qui le Lui demandent! \EVERSE}
\newcommand{\lcXIvXIVfr}{\VERSE  Jésus chassait un démon, et ce démon était muet.  Et lorsqu'Il eut chassé le démon, le muet parla, et les foules furent dans l'admiration. \EVERSE}
\newcommand{\lcXIvXVfr}{\VERSE  Mais quelques-uns d'entre eux dirent: C'est par Béelzébub, prince des démons, qu'Il chasse les démons. \EVERSE}
\newcommand{\lcXIvXVIfr}{\VERSE  Et d'autres, pour Le tenter, Lui demandaient un signe qui vînt du Ciel. \EVERSE}
\newcommand{\lcXIvXVIIfr}{\VERSE  Mais Lui, ayant vu leurs pensées, leur dit: Tout royaume divisé contre lui-même sera dévasté, et la maison tombera sur la maison. \EVERSE}
\newcommand{\lcXIvXVIIIfr}{\VERSE  Si donc Satan est aussi divisé contre lui-même, comment son règne subsistera-t-il?  Car vous dites que c'est par Béelzébub que Je chasse les démons. \EVERSE}
\newcommand{\lcXIvXIXfr}{\VERSE  Or si c'est par Béelzébub que Je chasse les démons, par qui vos fils les chassent-ils?  C'est pourquoi ils seront eux-mêmes vos juges. \EVERSE}
\newcommand{\lcXIvXXfr}{\VERSE  Mais si c'est par le doigt de Dieu que Je chasse les démons, assurément le royaume de Dieu est arrivé jusqu'à vous. \EVERSE}
\newcommand{\lcXIvXXIfr}{\VERSE  Lorsque l'homme fort, armé, garde sa maison, ce qu'il possède est en paix. \EVERSE}
\newcommand{\lcXIvXXIIfr}{\VERSE  Mais si un plus fort que lui survient et triomphe de lui, il emportera toutes ses armes, dans lesquelles il se confiait, et il distribuera ses dépouilles. \EVERSE}
\newcommand{\lcXIvXXIIIfr}{\VERSE  Celui qui n'est point avec Moi est contre Moi, et celui qui ne recueille pas avec Moi dissipe. \EVERSE}
\newcommand{\lcXIvXXIVfr}{\VERSE  Lorsque l'esprit impur est sorti d'un homme, il va par des lieux arides, cherchant du repos; et n'en trouvant pas, il dit: Je retournerai dans ma maison, d'où je suis sorti. \EVERSE}
\newcommand{\lcXIvXXVfr}{\VERSE  Et quand il arrive, il la trouve balayée et ornée. \EVERSE}
\newcommand{\lcXIvXXVIfr}{\VERSE  Alors il s'en va, et prend avec lui sept autres esprits, plus méchants que lui, et, entrant dans cette maison, ils y habitent.  Et le dernier état de cet homme devient pire que le premier. \EVERSE}
\newcommand{\lcXIvXXVIIfr}{\VERSE  Or il arriva, tandis qu'Il disait ces choses, qu'une femme, élevant la voix du milieu de la foule, Lui dit: Heureux le sein qui Vous a porté, et les mamelles qui Vous ont allaité. \EVERSE}
\newcommand{\lcXIvXXVIIIfr}{\VERSE  Mais Il dit: Heureux plutôt ceux qui écoutent la parole de Dieu et qui la gardent. \EVERSE}
\newcommand{\lcXIvXXIXfr}{\VERSE  Et comme les foules accouraient, Il Se mit à dire: Cette génération est une génération méchante; elle demande un signe, et il ne lui sera pas donné de signe, si ce n'est le signe du prophète Jonas. \EVERSE}
\newcommand{\lcXIvXXXfr}{\VERSE  Car, de même que Jonas fut un signe pour les Ninivites, ainsi en sera-t-il du Fils de l'homme pour cette génération. \EVERSE}
\newcommand{\lcXIvXXXIfr}{\VERSE  La reine du Midi se lèvera, lors du jugement, contre les hommes de cette génération, et les condamnera; car elle est venue des extrémités de la terre pour entendre la sagesse de Salomon, et voici qu'il y a plus que Salomon ici. \EVERSE}
\newcommand{\lcXIvXXXIIfr}{\VERSE  Les Ninivites se lèveront, lors du jugement, contre cette génération, et la condamneront; car ils ont fait pénitence à la prédication de Jonas, et voici qu'il y a plus que Jonas ici. \EVERSE}
\newcommand{\lcXIvXXXIIIfr}{\VERSE  Personne n'allume une lampe pour la mettre dans un lieu caché, ou sous le boisseau; mais on la met sur le candélabre, afin que ceux qui entrent voient la lumière. \EVERSE}
\newcommand{\lcXIvXXXIVfr}{\VERSE  La lampe de ton corps, c'est ton oeil.  Si ton oeil est simple, tout ton corps sera lumineux; mais s'il est mauvais, ton corps aussi sera ténébreux. \EVERSE}
\newcommand{\lcXIvXXXVfr}{\VERSE  Prends donc garde que la lumière qui est en toi ne soit ténèbres. \EVERSE}
\newcommand{\lcXIvXXXVIfr}{\VERSE  Si donc tout ton corps est éclairé, n'ayant aucune partie ténébreuse, il sera tout lumineux, et tu seras éclairé comme par une lampe brillante. \EVERSE}
\newcommand{\lcXIvXXXVIIfr}{\VERSE  Pendant qu'Il parlait, un pharisien Le pria de dîner chez lui; et étant entré, Il Se mit à table. \EVERSE}
\newcommand{\lcXIvXXXVIIIfr}{\VERSE  Or le pharisien, pensant en lui-même, commença à se demander pourquoi Il ne S'était pas lavé avant le repas. \EVERSE}
\newcommand{\lcXIvXXXIXfr}{\VERSE  Mais le Seigneur lui dit: Vous autres, pharisiens, vous nettoyez le dehors de la coupe et du plat; mais ce qui est au dedans de vous est plein de rapine et d'iniquité. \EVERSE}
\newcommand{\lcXIvXLfr}{\VERSE  Insensés, celui qui a fait le dehors n'a-t-il pas fait aussi le dedans? \EVERSE}
\newcommand{\lcXIvXLIfr}{\VERSE  Cependant donnez en aumône votre superflu, et voici que tout sera pur pour vous. \EVERSE}
\newcommand{\lcXIvXLIIfr}{\VERSE  Mais malheur à vous, pharisiens, parce que vous payez la dîme de la menthe, et de la rue, et de tous les légumes, et que vous négligez la justice et l'amour de Dieu; il fallait cependant faire ces choses, sans omettre les autres. \EVERSE}
\newcommand{\lcXIvXLIIIfr}{\VERSE  Malheur à vous, pharisiens, parce que vous aimez les premiers sièges dans les synagogues, et les salutations sur la place publique. \EVERSE}
\newcommand{\lcXIvXLIVfr}{\VERSE  Malheur à vous, parce que vous êtes comme des sépulcres que ne paraissent point, et sur lesquels les hommes marchent sans le savoir. \EVERSE}
\newcommand{\lcXIvXLVfr}{\VERSE  Alors un des docteurs de la loi, prenant la parole, Lui dit: Maître, en parlant de la sorte, Vous nous faites injure à nous aussi. \EVERSE}
\newcommand{\lcXIvXLVIfr}{\VERSE  Mais Jésus dit: Malheur à vous aussi, docteurs de la loi, parce que vous chargez les hommes de fardeaux qu'ils ne peuvent porter, et que vous-mêmes vous ne touchez pas ces fardeaux d'un seul de vos doigts. \EVERSE}
\newcommand{\lcXIvXLVIIfr}{\VERSE  Malheur à vous, qui bâtissez les tombeaux des prophètes; et ce sont vos pères qui les ont tués. \EVERSE}
\newcommand{\lcXIvXLVIIIfr}{\VERSE  Certes, vous témoignez que vous consentez aux oeuvres de vos pères; car eux, ils les ont tués, et vous, vous bâtissez leurs tombeaux. \EVERSE}
\newcommand{\lcXIvXLIXfr}{\VERSE  C'est pourquoi la sagesse de Dieu a dit: Je leur enverrai des prophètes et des apôtres, et ils tueront les uns et persécuteront les autres, \EVERSE}
\newcommand{\lcXIvLfr}{\VERSE  afin qu'il soit demandé compte à cette génération du sang de tous les prophètes qui a été répandu depuis la création du monde, \EVERSE}
\newcommand{\lcXIvLIfr}{\VERSE  depuis le sang d'Abel jusqu'au sang de Zacharie, qui a été tué entre l'autel et le temple.  Oui, Je vous le dis, il en sera demandé compte à cette génération. \EVERSE}
\newcommand{\lcXIvLIIfr}{\VERSE  Malheur à vous, docteurs de la loi, parce que vous avez pris la clef de la science; vous-mêmes, vous n'êtes pas entrés, et vous avez arrêté ceux qui voulaient entrer. \EVERSE}
\newcommand{\lcXIvLIIIfr}{\VERSE  Comme Il leur disait ces choses, les pharisiens et les docteurs de la loi commencèrent à Le presser vivement et à Le harceler par une multitude de questions, \EVERSE}
\newcommand{\lcXIvLIVfr}{\VERSE  Lui tendant des pièges, et cherchant à surprendre quelque parole de Sa bouche, afin de L'accuser. \EVERSE}
\newcommand{\lcXIIvIfr}{\VERSE  Or des foules nombreuses s'étant assemblées autour de Jésus, à ce point qu'on marchait les uns sur les autres, Il commença à dire à Ses disciples: Gardez-vous du levain des pharisiens, qui est l'hypocrisie. \EVERSE}
\newcommand{\lcXIIvIIfr}{\VERSE  Il n'y a rien de secret qui ne doive être découvert, ni rien de caché qui ne doive être connu. \EVERSE}
\newcommand{\lcXIIvIIIfr}{\VERSE  Car, ce que vous avez dit dans les ténèbres, on le dira dans la lumière; et ce que vous avez dit à l'oreille, dans les chambres, sera prêché sur les toits. \EVERSE}
\newcommand{\lcXIIvIVfr}{\VERSE  Je vous dis donc à vous, qui êtes Mes amis: Ne craignez point ceux qui tuent le corps, et qui, après cela, ne peuvent rien faire de plus. \EVERSE}
\newcommand{\lcXIIvVfr}{\VERSE  Mais Je vous montrerai qui vous devez craindre: craignez Celui qui, après avoir tué, a le pouvoir de jeter dans la géhenne.  Oui, Je vous le dis, Celui-là, craignez-Le. \EVERSE}
\newcommand{\lcXIIvVIfr}{\VERSE  Cinq passereaux ne se vendent-ils pas deux as?  Et pas un d'eux n'est en oubli devant Dieu. \EVERSE}
\newcommand{\lcXIIvVIIfr}{\VERSE  Les cheveux même de votre tête sont tous comptés.  Ne craignez donc point; vous valez plus que beaucoup de passereaux. \EVERSE}
\newcommand{\lcXIIvVIIIfr}{\VERSE  Or, Je vous le dis, quiconque Me confessera devant les hommes, le Fils de l'homme le confessera aussi devant les Anges de Dieu. \EVERSE}
\newcommand{\lcXIIvIXfr}{\VERSE  Mais celui que M'aura renié devant les hommes sera renié devant les Anges de Dieu. \EVERSE}
\newcommand{\lcXIIvXfr}{\VERSE  Et à quiconque prononcera une parole contre le Fils de l'homme, il sera pardonné; mais à celui qui aura blasphémé contre le Saint-Esprit, il ne sera point pardonné. \EVERSE}
\newcommand{\lcXIIvXIfr}{\VERSE  Lorsqu'on vous conduira dans les synagogues, et devant les magistrats et les autorités, ne vous inquiétez point de quelle manière ou de ce que vous répondrez, ni de ce que vous direz; \EVERSE}
\newcommand{\lcXIIvXIIfr}{\VERSE  car l'Esprit-Saint vous enseignera, à l'heure même, ce qu'il faudra que vous disiez. \EVERSE}
\newcommand{\lcXIIvXIIIfr}{\VERSE  Alors quelqu'un de la foule Lui dit: Maître, dites à mon frère de partager avec moi notre héritage. \EVERSE}
\newcommand{\lcXIIvXIVfr}{\VERSE  Mais Jésus lui répondit: Homme, qui M'a établi sur vous juge ou faiseur de partages? \EVERSE}
\newcommand{\lcXIIvXVfr}{\VERSE  Puis Il leur dit: Voyez, et gardez-vous de toute avarice; car un homme fût-il dans l'abondance, sa vie ne dépend pas des biens qu'il possède. \EVERSE}
\newcommand{\lcXIIvXVIfr}{\VERSE  Il leur dit ensuite cette parabole: Le champ d'un homme riche lui rapporta des fruits abondants. \EVERSE}
\newcommand{\lcXIIvXVIIfr}{\VERSE  Et il pensait en lui-même, disant: Que ferai-je?  car je n'ai pas où serrer mes fruits. \EVERSE}
\newcommand{\lcXIIvXVIIIfr}{\VERSE  Et il dit: Voici ce que je ferai: j'abattrai mes greniers et j'en bâtirai de plus grands, et j'y amasserai tous mes produits et mes biens. \EVERSE}
\newcommand{\lcXIIvXIXfr}{\VERSE  Et je dirai à mon âme: Mon âme, tu as beaucoup de biens en réserve pour de nombreuses années; repose-toi, mange, bois, fais bonne chère. \EVERSE}
\newcommand{\lcXIIvXXfr}{\VERSE  Mais Dieu lui dit: Insensé, cette nuit même on te redemandera ton âme; et ce que tu as préparé, à qui sera-ce? \EVERSE}
\newcommand{\lcXIIvXXIfr}{\VERSE  Ainsi en est-il de celui qui amasse des trésors pour lui-même, et qui n'est pas riche pour Dieu. \EVERSE}
\newcommand{\lcXIIvXXIIfr}{\VERSE  Il dit ensuite à Ses disciples: C'est pourquoi Je vous le dis, ne soyez point inquiets pour votre vie, de ce que vous mangerez; ni pour votre corps, de quoi vous serez vêtus. \EVERSE}
\newcommand{\lcXIIvXXIIIfr}{\VERSE  La vie est plus que la nourriture, et le corps plus que le vêtement. \EVERSE}
\newcommand{\lcXIIvXXIVfr}{\VERSE  Considérez les corbeaux: ils ne sèment, ni ne moissonnent; ils n'ont ni cellier, ni grenier; cependant Dieu les nourrit.  Combien ne valez-vous pas plus qu'eux! \EVERSE}
\newcommand{\lcXIIvXXVfr}{\VERSE  Mais qui de vous, en réfléchissant, peut ajouter à sa taille une coudée? \EVERSE}
\newcommand{\lcXIIvXXVIfr}{\VERSE  Si donc vous ne pouvez pas même ce qu'il y a de moindre, pourquoi vous inquiétez-vous des autres choses? \EVERSE}
\newcommand{\lcXIIvXXVIIfr}{\VERSE  Considérez les lis, comme ils croissent: ils ne travaillent, ni ne filent; cependant, Je vous le dis, Salomon lui-même, dans toute sa gloire, n'était pas vêtu comme l'un d'eux. \EVERSE}
\newcommand{\lcXIIvXXVIIIfr}{\VERSE  Si donc Dieu revêt ainsi l'herbe qui est aujourd'hui dans les champs, et qui demain sera jetée au four, combien plus vous-mêmes, hommes de peu de foi! \EVERSE}
\newcommand{\lcXIIvXXIXfr}{\VERSE  Et vous, ne vous préoccupez pas de ce que vous mangerez ou de ce que vous boirez, et ne vous élevez pas si haut. \EVERSE}
\newcommand{\lcXIIvXXXfr}{\VERSE  Car ce sont les païens du monde qui recherchent toutes ces choses; mais votre Père sait que vous en avez besoin. \EVERSE}
\newcommand{\lcXIIvXXXIfr}{\VERSE  C'est pourquoi, cherchez premièrement le royaume de Dieu et Sa justice, et toutes ces choses vous seront données par surcroît. \EVERSE}
\newcommand{\lcXIIvXXXIIfr}{\VERSE  Ne craignez point, petit troupeau; car il a plu à votre Père de vous donner le royaume. \EVERSE}
\newcommand{\lcXIIvXXXIIIfr}{\VERSE  Vendez ce que vous possédez et donnez-le en aumônes; faites-vous des bourses qui ne s'usent point, un trésor inépuisable dans les Cieux, dont le voleur n'approche pas et que le ver ne détruit pas. \EVERSE}
\newcommand{\lcXIIvXXXIVfr}{\VERSE  Car où est votre trésor, là sera aussi votre coeur. \EVERSE}
\newcommand{\lcXIIvXXXVfr}{\VERSE  Que vos reins soient ceints, et les lampes allumées dans vos mains. \EVERSE}
\newcommand{\lcXIIvXXXVIfr}{\VERSE  Et vous, soyez semblables à des hommes qui attendent que leur maître revienne des noces, afin que, lorsqu'il arrivera et frappera, ils lui ouvrent aussitôt. \EVERSE}
\newcommand{\lcXIIvXXXVIIfr}{\VERSE  Heureux ces serviteurs que le maître, à son arrivée, trouvera veillant; en vérité, Je vous le dis, il se ceindra, les fera asseoir à table, et passant devant eux, il les servira. \EVERSE}
\newcommand{\lcXIIvXXXVIIIfr}{\VERSE  Et, s'il vient à la seconde veille, s'il vient à la troisième veille, et qu'il les trouve en cet état, heureux sont ces serviteurs! \EVERSE}
\newcommand{\lcXIIvXXXIXfr}{\VERSE  Or sachez que, si le père de famille savait à quelle heure le voleur doit venir, il veillerait certainement, et ne laisserait pas percer sa maison. \EVERSE}
\newcommand{\lcXIIvXLfr}{\VERSE  Vous aussi, soyez prêts; car, à l'heure que vous ne pensez pas, le Fils de l'homme viendra. \EVERSE}
\newcommand{\lcXIIvXLIfr}{\VERSE  Alors Pierre Lui dit: Seigneur, est-ce à nous que Vous adressez cette parabole, ou est-ce à tous? \EVERSE}
\newcommand{\lcXIIvXLIIfr}{\VERSE  Et le Seigneur lui dit: Quel est, penses-tu, le dispensateur fidèle et prudent, que le maître a établi sur ses serviteurs pour leur donner, au temps fixé, leur mesure de blé? \EVERSE}
\newcommand{\lcXIIvXLIIIfr}{\VERSE  Heureux ce serviteur, que le maître, à son arrivée, trouvera agissant ainsi! \EVERSE}
\newcommand{\lcXIIvXLIVfr}{\VERSE  En vérité, Je vous le dis, il l'établira sur tout ce qu'il possède. \EVERSE}
\newcommand{\lcXIIvXLVfr}{\VERSE  Mais si ce serviteur dit en son coeur: Mon maître tarde à venir, et s'il se met à frapper les serviteurs et les servantes, à manger, à boire et à s'enivrer, \EVERSE}
\newcommand{\lcXIIvXLVIfr}{\VERSE  le maître de ce serviteur viendra au jour où il ne s'y attend pas et à l'heure qu'il ne sait pas, et il le retranchera, et lui donnera sa part avec les infidèles. \EVERSE}
\newcommand{\lcXIIvXLVIIfr}{\VERSE  Le serviteur qui a connu la volonté de son maître, et n'a rien préparé, et n'a pas agi selon sa volonté, recevra un grand nombre de coups; \EVERSE}
\newcommand{\lcXIIvXLVIIIfr}{\VERSE  mais celui qui ne l'a pas connue, et qui a fait des choses dignes de châtiment, recevra peu de coups.  A quiconque beaucoup aura été donné, beaucoup sera demandé; et de celui à qui on a confié beaucoup, on exigera davantage. \EVERSE}
\newcommand{\lcXIIvXLIXfr}{\VERSE  Je suis venu jeter le feu sur la terre, et quel est Mon désir, sinon qu'il s'allume? \EVERSE}
\newcommand{\lcXIIvLfr}{\VERSE  J'ai à être baptisé d'un baptême, et comme Je Me sens pressé jusqu'à ce qu'il s'accomplisse! \EVERSE}
\newcommand{\lcXIIvLIfr}{\VERSE  Pensez-vous que Je sois venu apporter la paix sur la terre?  Non, vous dis-Je, mais la division. \EVERSE}
\newcommand{\lcXIIvLIIfr}{\VERSE  Car désormais, dans une même maison, cinq seront divisés: trois contre deux, et deux contre trois. \EVERSE}
\newcommand{\lcXIIvLIIIfr}{\VERSE  Seront divisés: le père contre le fils et le fils contre son père, la mère contre la fille et la fille contre la mère, la belle-mère contre sa belle-fille et la belle-fille contre sa belle-mère. \EVERSE}
\newcommand{\lcXIIvLIVfr}{\VERSE  Il disait aussi aux foules: Lorsque vous voyez un nuage s'élever à l'occident, vous dites aussitôt: La pluie vient; et il arrive ainsi. \EVERSE}
\newcommand{\lcXIIvLVfr}{\VERSE  Et quand vous voyez souffler le vent du midi, vous dites: Il fera chaud; et cela arrive. \EVERSE}
\newcommand{\lcXIIvLVIIfr}{\VERSE  Comment ne discernez-vous pas aussi par vous-mêmes ce que est juste? \EVERSE}
\newcommand{\lcXIIvLVIIIfr}{\VERSE  Lorsque tu vas avec ton adversaire devant le magistrat, tâche de te dégager de lui en chemin, de peur qu'il ne te traîne devant le juge, et que le juge ne te livre à l'exécuteur, et que l'exécuteur ne te mette en prison. \EVERSE}
\newcommand{\lcXIIvLIXfr}{\VERSE  Je te le dis, tu ne sortiras pas de là que tu n'aies payé jusqu'à la dernière obole. \EVERSE}
\newcommand{\lcXIIIvIfr}{\VERSE  En ce même temps, il y avait là quelques hommes, qui Lui annonçaient ce qui était arrivé aux Galiléens dont Pilate avait mêlé le sang avec celui de leurs sacrifices. \EVERSE}
\newcommand{\lcXIIIvIIfr}{\VERSE  Et prenant la parole, Il leur dit: Pensez-vous que ces Galiléens fussent plus pécheurs que tous les autres Galiléens, parce qu'ils ont souffert de telles choses? \EVERSE}
\newcommand{\lcXIIIvIIIfr}{\VERSE  Non, Je vous le dis; mais, si vous ne faites pénitence, vous périrez tous pareillement. \EVERSE}
\newcommand{\lcXIIIvIVfr}{\VERSE  Comme ces dix-huit personnes sur lesquelles est tombée la tour de Siloé, et qu'elle a tuées: pensez-vous que leur dette fût plus grande que celle de tous les habitants de Jérusalem? \EVERSE}
\newcommand{\lcXIIIvVfr}{\VERSE  Non, Je vous le dis; mais, si vous ne faites pénitence, vous périrez tous pareillement. \EVERSE}
\newcommand{\lcXIIIvVIfr}{\VERSE  Il disait aussi cette parabole: Un homme avait un figuier planté dans sa vigne; et il vint y chercher du fruit, et n'en trouva point. \EVERSE}
\newcommand{\lcXIIIvVIIfr}{\VERSE  Alors il dit au vigneron: Voilà trois ans que je viens chercher des fruits sur ce figuier, et je n'en trouve pas; coupe-le donc: pourquoi occupe-t-il encore le sol? \EVERSE}
\newcommand{\lcXIIIvVIIIfr}{\VERSE  Le vigneron, répondant, lui dit: Seigneur, laisse-le encore cette année, jusqu'à ce que je creuse tout autour et que j'y mette du fumier; \EVERSE}
\newcommand{\lcXIIIvIXfr}{\VERSE  peut-être portera-t-il du fruit; sinon, tu le couperas ensuite. \EVERSE}
\newcommand{\lcXIIIvXfr}{\VERSE  Or Jésus enseignait dans leur synagogue les jours de sabbat. \EVERSE}
\newcommand{\lcXIIIvXIfr}{\VERSE  Et voici qu'il y vînt une femme, possédée d'un esprit qui la rendait infirme depuis dix-huit ans; et elle était courbée, et ne pouvait pas du tout regarder en haut. \EVERSE}
\newcommand{\lcXIIIvXIIfr}{\VERSE  Jésus, la voyant, l'appela auprès de Lui et lui dit: Femme, tu es délivrée de ton infirmité. \EVERSE}
\newcommand{\lcXIIIvXIIIfr}{\VERSE  Et Il lui imposa les mains; et aussitôt elle redevint droite, et elle glorifiait Dieu. \EVERSE}
\newcommand{\lcXIIIvXIVfr}{\VERSE  Mais le chef de la synagogue prit la parole, indigné de ce que Jésus avait opéré cette guérison un jour de sabbat; et il disait à la foule: Il y a six jours pendant lesquels on doit travailler; venez donc en ces jours-là, et faites-vous guérir, et non pas le jour du sabbat. \EVERSE}
\newcommand{\lcXIIIvXVfr}{\VERSE  Le Seigneur lui répondit, en disant: Hypocrites, est-ce que chacun de vous, le jour du sabbat, ne délie pas son boeuf ou son âne de la crèche, et ne les mène pas boire? \EVERSE}
\newcommand{\lcXIIIvXVIfr}{\VERSE  Et cette fille d'Abraham, que Satan avait liée voilà dix-huit ans, ne fallait-il pas la délivrer de ce lien le jour du sabbat? \EVERSE}
\newcommand{\lcXIIIvXVIIfr}{\VERSE  Tandis qu'Il parlait ainsi, tous Ses adversaires rougissaient; et tout le peuple se réjouissait de toutes les choses glorieuses qu'Il accomplissait. \EVERSE}
\newcommand{\lcXIIIvXVIIIfr}{\VERSE  Il disait aussi: A quoi est semblable le royaume de Dieu, et à quoi le comparerai-Je? \EVERSE}
\newcommand{\lcXIIIvXIXfr}{\VERSE  Il est semblable à un grain de sénevé, qu'un homme a pris et mis dans son jardin; et il a crû et est devenu un grand arbre, et les oiseaux du ciel se sont reposés sur ses branches. \EVERSE}
\newcommand{\lcXIIIvXXfr}{\VERSE  Il dit encore: A quoi comparerai-Je le royaume de Dieu? \EVERSE}
\newcommand{\lcXIIIvXXIfr}{\VERSE  Il est semblable à du levain, qu'une femme a pris et mêlé dans trois mesures de farine, jusqu'à ce que tout fût fermenté. \EVERSE}
\newcommand{\lcXIIIvXXIIfr}{\VERSE  Et Il allait à travers les villes et les villages, enseignant, et faisant route vers Jérusalem. \EVERSE}
\newcommand{\lcXIIIvXXIIIfr}{\VERSE  Or quelqu'un Lui dit: Seigneur, y en a-t-il peu qui soient sauvés?  Et Il leur dit: \EVERSE}
\newcommand{\lcXIIIvXXIVfr}{\VERSE  Efforcez-vous d'entrer par la porte étroite; car beaucoup, Je vous le dis, chercheront à entrer, et ne le pourront pas. \EVERSE}
\newcommand{\lcXIIIvXXVfr}{\VERSE  Et lorsque le Père de famille sera entré, et aura fermé la porte, vous, étant dehors, vous commencerez à frapper à la porte, en disant: Seigneur, ouvrez-nous.  Et vous répondant, Il dira: Je ne sais d'où vous êtes. \EVERSE}
\newcommand{\lcXIIIvXXVIfr}{\VERSE  Alors vous commencerez à dire: Nous avons mangé et bu devant Vous, et Vous avez enseigné sur nos places publiques. \EVERSE}
\newcommand{\lcXIIIvXXVIIfr}{\VERSE  Et Il vous dira: Je ne sais d'où vous êtes; retirez-vous de Moi, vous tous, ouvriers d'iniquité. \EVERSE}
\newcommand{\lcXIIIvXXVIIIfr}{\VERSE  Là il y aura des pleurs et des grincements de dents, quand vous verrez Abraham, et Isaac, et Jacob, et tous les prophètes dans le royaume de Dieu, et que vous, vous serez chassés dehors. \EVERSE}
\newcommand{\lcXIIIvXXIXfr}{\VERSE  Il en viendra de l'orient et de l'occident, de l'aquilon et du midi, et ils se mettront à table dans le royaume de Dieu. \EVERSE}
\newcommand{\lcXIIIvXXXfr}{\VERSE  Et voici, ce sont les derniers qui seront les premiers, et ce sont les premiers qui seront les derniers. \EVERSE}
\newcommand{\lcXIIIvXXXIfr}{\VERSE  Le même jour, quelques-uns des pharisiens s'approchèrent, et Lui dirent: Allez-Vous-en, et partez d'ici, car Hérode veut Vous tuer. \EVERSE}
\newcommand{\lcXIIIvXXXIIfr}{\VERSE  Il leur dit: Allez, et dites à ce renard: Voici que Je chasse les démons, et que J'opère des guérisons aujourd'hui et demain, et le troisième jour tout sera consommé pour Moi. \EVERSE}
\newcommand{\lcXIIIvXXXIIIfr}{\VERSE  Cependant il faut que Je marche aujourd'hui, et demain, et le jour suivant, car il ne convient pas qu'un prophète périsse hors de Jérusalem. \EVERSE}
\newcommand{\lcXIIIvXXXIVfr}{\VERSE  Jérusalem, Jérusalem, qui tues les prophètes, et qui lapides ceux qui te sont envoyés, combien de fois ai-Je voulu rassembler tes enfants, comme un oiseau rassemble sa couvée sous ses ailes, et tu n'as pas voulu! \EVERSE}
\newcommand{\lcXIIIvXXXVfr}{\VERSE  Voici que votre maison vous sera laissée déserte.  Je vous le dis, vous ne Me verrez plus, jusqu'à ce que vienne le moment où vous direz: Béni soit Celui qui vient au nom du Seigneur! \EVERSE}
\newcommand{\lcXIVvIfr}{\VERSE  Et il arriva que Jésus entra, un jour de sabbat, dans la maison d'un des principaux pharisiens, pour y manger du pain; et ceux-ci L'observaient. \EVERSE}
\newcommand{\lcXIVvIIfr}{\VERSE  Et voici qu'un homme hydropique était devant Lui. \EVERSE}
\newcommand{\lcXIVvIIIfr}{\VERSE  Et Jésus, prenant la parole, dit aux docteurs de la loi et aux pharisiens: Est-il permis de guérir le jour du sabbat? \EVERSE}
\newcommand{\lcXIVvIVfr}{\VERSE  Mais ils gardèrent le silence.  Alors Lui, prenant cet homme par la main, le guérit et le renvoya. \EVERSE}
\newcommand{\lcXIVvVfr}{\VERSE  Puis, S'adressant à eux, Il dit: Qui de vous, si son âne ou son boeuf tombe dans un puits, ne l'en retirera pas aussitôt, le jour du sabbat? \EVERSE}
\newcommand{\lcXIVvVIfr}{\VERSE  Et ils ne pouvaient rien répondre à cela. \EVERSE}
\newcommand{\lcXIVvVIIfr}{\VERSE  Il dit aussi aux invités cette parabole, considérant comment ils choisissaient les premières places.  Il leur dit: \EVERSE}
\newcommand{\lcXIVvVIIIfr}{\VERSE  Quand tu seras invité à des noces, ne te mets pas à la première place, de peur qu'il n'y ait parmi les invités une personne plus considérable que toi, \EVERSE}
\newcommand{\lcXIVvIXfr}{\VERSE  et que celui qui vous a conviés, toi et lui, ne vienne te dire: Cède la place à celui-ci, et qu'alors tu n'ailles, en rougissant, occuper la dernière place. \EVERSE}
\newcommand{\lcXIVvXfr}{\VERSE  Mais, quand tu auras été invité, va, mets-toi à la dernière place, afin que, lorsque celui qui t'a invité sera venu, il te dise: Mon ami, monte plus haut.  Et alors ce sera une gloire pour toi devant ceux qui seront à table avec toi. \EVERSE}
\newcommand{\lcXIVvXIfr}{\VERSE  Car quiconque s'élève sera humilié, et quiconque s'humilie sera élevé. \EVERSE}
\newcommand{\lcXIVvXIIfr}{\VERSE  Il dit aussi à celui qui L'avait invité: Lorsque tu donnes à dîner ou à souper, n'appelle pas tes amis, ni tes frères, ni tes parents, ni tes voisins riches, de peur qu'ils ne t'invitent à leur tour, et ne te rendent ce qu'ils ont reçu de toi. \EVERSE}
\newcommand{\lcXIVvXIIIfr}{\VERSE  Mais lorsque tu fais un festin, appelle les pauvres, les estropiés, les boiteux et les aveugles; \EVERSE}
\newcommand{\lcXIVvXIVfr}{\VERSE  et tu seras heureux de ce qu'ils n'ont pas le moyen de te le rendre, car cela te sera rendu à la résurrection des justes. \EVERSE}
\newcommand{\lcXIVvXVfr}{\VERSE  Un de ceux que étaient à table avec Jésus, ayant entendu ces paroles, Lui dit: Heureux celui qui mangera du pain dans le royaume de Dieu! \EVERSE}
\newcommand{\lcXIVvXVIfr}{\VERSE  Alors Jésus lui dit: Un homme fit un grand souper, et invita de nombreux convives. \EVERSE}
\newcommand{\lcXIVvXVIIfr}{\VERSE  Et à l'heure du souper, il envoya son serviteur dire aux invités de venir, parce que tout était prêt. \EVERSE}
\newcommand{\lcXIVvXVIIIfr}{\VERSE  Mais tous, unanimement, commencèrent à s'excuser.  Le premier lui dit: J'ai acheté une terre, et il est nécessaire que j'aille la voir; je t'en prie, excuse-moi. \EVERSE}
\newcommand{\lcXIVvXIXfr}{\VERSE  Le second dit: J'ai acheté cinq paires de boeufs, et je vais les essayer; je t'en prie, excuse-moi. \EVERSE}
\newcommand{\lcXIVvXXfr}{\VERSE  Et un autre dit: J'ai épousé une femme, et c'est pourquoi je ne puis venir. \EVERSE}
\newcommand{\lcXIVvXXIfr}{\VERSE  A son retour, le serviteur rapporta cela à son maître.  Alors le père de famille, irrité, dit à son serviteur: Va promptement sur les places et dans les rues de la ville, et amène ici les pauvres, les estropiés, les aveugles et les boiteux. \EVERSE}
\newcommand{\lcXIVvXXIIfr}{\VERSE  Le serviteur dit ensuite: Seigneur, ce que vous avez commandé a été fait, et il y a encore de la place. \EVERSE}
\newcommand{\lcXIVvXXIIIfr}{\VERSE  Et le maître dit au serviteur: Va dans les chemins et le long des haies, et contrains les gens d'entrer, afin que ma maison soit remplie. \EVERSE}
\newcommand{\lcXIVvXXIVfr}{\VERSE  Car, je vous le dis, aucun de ces hommes qui avaient été invités ne goûtera de mon souper. \EVERSE}
\newcommand{\lcXIVvXXVfr}{\VERSE  Or de grandes foules marchaient avec Jésus; et Se tournant vers elles, Il leur dit: \EVERSE}
\newcommand{\lcXIVvXXVIfr}{\VERSE  Si quelqu'un vient à Moi, et ne hait pas son père, et sa mère, et sa femme, et ses enfants, et ses frères, et ses soeurs, et même sa propre vie, il ne peut être Mon disciple. \EVERSE}
\newcommand{\lcXIVvXXVIIfr}{\VERSE  Et celui qui ne porte pas sa croix, et ne Me suit pas, ne peut être Mon disciple. \EVERSE}
\newcommand{\lcXIVvXXVIIIfr}{\VERSE  Car quel est celui de vous qui, voulant bâtir une tour, ne s'assied d'abord, et ne suppute les dépenses qui sont nécessaires, afin de voir s'il aura de quoi l'achever; \EVERSE}
\newcommand{\lcXIVvXXIXfr}{\VERSE  de peur qu'après avoir posé les fondements, il ne puisse l'achever, et que tous ceux qui verront cela ne se mettent à se moquer de lui, \EVERSE}
\newcommand{\lcXIVvXXXfr}{\VERSE  en disant: Cet homme a commencé à bâtir, et il n'a pu achever? \EVERSE}
\newcommand{\lcXIVvXXXIfr}{\VERSE  Ou quel roi, sur le point de faire la guerre à un autre roi, ne s'assied d'abord, afin d'examiner s'il pourra, avec dix mille hommes, marcher contre celui qui s'avance sur lui avec vingt mille? \EVERSE}
\newcommand{\lcXIVvXXXIIfr}{\VERSE  Autrement, tandis que l'autre roi est encore loin, il lui envoie une ambassade, et lui fait des propositions de paix. \EVERSE}
\newcommand{\lcXIVvXXXIIIfr}{\VERSE  Ainsi donc, quiconque d'entre vous ne renonce pas à tout ce qu'il possède ne peut être Mon disciple. \EVERSE}
\newcommand{\lcXIVvXXXIVfr}{\VERSE  Le sel est bon; mais, si le sel s'affadit, avec quoi l'assaisonnera-t-on? \EVERSE}
\newcommand{\lcXIVvXXXVfr}{\VERSE  Il n'est plus propre ni pour la terre, ni pour le fumier; mais on le jettera dehors.  Que celui qui a des oreilles pour entendre, entende. \EVERSE}
\newcommand{\lcXVvIfr}{\VERSE  Or les publicains et les pécheurs s'approchaient de Jésus pour L'écouter. \EVERSE}
\newcommand{\lcXVvIIfr}{\VERSE  Et les pharisiens et les scribes murmuraient, en disant: Cet homme accueille les pécheurs, et mange avec eux. \EVERSE}
\newcommand{\lcXVvIIIfr}{\VERSE  Alors Il leur dit cette parabole: \EVERSE}
\newcommand{\lcXVvIVfr}{\VERSE  Quel est l'homme parmi vous qui a cent brebis, et qui, s'il en perd une, ne laisse les quatre-vingt-dix-neuf autres dans le désert, pour s'en aller après celle qui est perdue, jusqu'à ce qu'il la trouve? \EVERSE}
\newcommand{\lcXVvVfr}{\VERSE  Et lorsqu'il l'a trouvée, il la met sur ses épaules avec joie; \EVERSE}
\newcommand{\lcXVvVIfr}{\VERSE  et venant dans sa maison, il appelle ses amis et ses voisins, et leur dit: Réjouissez-vous avec moi, car j'ai trouvé ma brebis qui était perdue. \EVERSE}
\newcommand{\lcXVvVIIfr}{\VERSE  Je vous le dis, il y aura de même plus de joie dans le Ciel pour un seul pécheur qui fait pénitence, que pour quatre-vingt-dix-neuf justes qui n'ont pas besoin de pénitence. \EVERSE}
\newcommand{\lcXVvVIIIfr}{\VERSE  Ou quelle est la femme qui, ayant dix drachmes, si elle en perd une, n'allume la lampe, ne balaye la maison, et ne cherche avec soin jusqu'à ce qu'elle la trouve? \EVERSE}
\newcommand{\lcXVvIXfr}{\VERSE  Et lorsqu'elle l'a trouvée, elle appelle ses amies et ses voisines, et leur dit: Réjouissez-vous avec moi, car j'ai trouvé la drachme que j'avais perdue. \EVERSE}
\newcommand{\lcXVvXfr}{\VERSE  De même, Je vous le dis, il y aura de la joie parmi les Anges de Dieu, pour un seul pécheur qui fait pénitence. \EVERSE}
\newcommand{\lcXVvXIfr}{\VERSE  Il dit encore: Un homme avait deux fils; \EVERSE}
\newcommand{\lcXVvXIIfr}{\VERSE  et le plus jeune des deux dit à son père: Mon père, donne-moi la part de bien qui doit me revenir.  Et le père leur partagea son bien. \EVERSE}
\newcommand{\lcXVvXIIIfr}{\VERSE  Et peu de jours après, le plus jeune fils, ayant rassemblé tout ce qu'il avait, partit pour un pays étranger et lointain, et là il dissipa son bien, en vivant dans la débauche. \EVERSE}
\newcommand{\lcXVvXIVfr}{\VERSE  Et aprés qu'il eut tout dépensé, il survint une grande famine dans ce pays-là, et il commença à être dans le besoin. \EVERSE}
\newcommand{\lcXVvXVfr}{\VERSE  Il alla donc, et s'attacha au service d'un des habitants du pays, qui l'envoya dans sa maison des champs pour garder les pourceaux. \EVERSE}
\newcommand{\lcXVvXVIfr}{\VERSE  Et il désirait remplir son ventre des gousses que les pourceaux mangeaient; mais personne ne lui en donnait. \EVERSE}
\newcommand{\lcXVvXVIIfr}{\VERSE  Et étant rentré en lui-même, il dit: Combien de mercenaires, dans la maison de mon père, ont du pain en abondance, et moi je meurs ici de faim! \EVERSE}
\newcommand{\lcXVvXVIIIfr}{\VERSE  Je me lèverai, et j'irai vers mon père, et je lui dirai: Mon père, j'ai péché contre le Ciel et contre toi; \EVERSE}
\newcommand{\lcXVvXIXfr}{\VERSE  Je ne suis plus digne désormais d'être appelé ton fils, traite-moi comme l'un de tes mercenaires. \EVERSE}
\newcommand{\lcXVvXXfr}{\VERSE  Et se levant, il vint vers son père.  Comme il était encore loin, son père le vit, et fut ému de compassion; et accourant, il se jeta à son cou, et le baisa. \EVERSE}
\newcommand{\lcXVvXXIfr}{\VERSE  Et le fils lui dit: Mon pére, j'ai péché contre le Ciel et contre toi; je ne suis plus digne d'être appelé ton fils. \EVERSE}
\newcommand{\lcXVvXXIIfr}{\VERSE  Alors le père dit à ses serviteurs: Vite, apportez la plus belle robe, et revêtez-l'en; et mettez un anneau à sa main, et des chaussures à ses pieds; \EVERSE}
\newcommand{\lcXVvXXIIIfr}{\VERSE  puis amenez le veau gras, et tuez-le; et mangeons, et faisons bonne chère; \EVERSE}
\newcommand{\lcXVvXXIVfr}{\VERSE  car mons fils que voici était mort, et il est revenu à la vie; il était perdu, et il est retrouvé.  Et ils commencèrent à faire grande chère. \EVERSE}
\newcommand{\lcXVvXXVfr}{\VERSE  Cependant son fils aîné était dans les champs; et comme il revenait et s'approchait de la maison, il entendit la musique et les danses. \EVERSE}
\newcommand{\lcXVvXXVIfr}{\VERSE  Et il appela un des serviteurs, et demanda ce que c'était. \EVERSE}
\newcommand{\lcXVvXXVIIfr}{\VERSE  Celui-ci lui dit: Ton frère est revenu, et ton père a tué le veau gras, parce qu'il l'a recouvré sain et sauf. \EVERSE}
\newcommand{\lcXVvXXVIIIfr}{\VERSE  Il s'indigna, et ne voulait pas entrer.  Son père sortit donc, et se mit à le prier. \EVERSE}
\newcommand{\lcXVvXXIXfr}{\VERSE  Mais, répondant à son père, il dit: Voilà tant d'années que je te sers, et je n'ai jamais transgressé tes ordres, et jamais tu ne m'as donné un chevreau pour faire bonne chère avec mes amis; \EVERSE}
\newcommand{\lcXVvXXXfr}{\VERSE  mais dès que cet autre fils, qui a dévoré son bien avec des femmes perdues, est revenu, tu as tué pour lui le veau gras. \EVERSE}
\newcommand{\lcXVvXXXIfr}{\VERSE  Alors le père lui dit: Mon fils, tu es toujours avec moi, et tout ce que j'ai est à toi; \EVERSE}
\newcommand{\lcXVvXXXIIfr}{\VERSE  mais il fallait faire bonne chère et se réjouir, parce que ton frère que voici était mort, et qu'il est revenu à la vie; parce qu'il était perdu, et qu'il est retrouvé. \EVERSE}
\newcommand{\lcXVIvIfr}{\VERSE  Jésus disait aussi à Ses disciples: Un homme riche avait un économe, et celui-ci fut accusé auprès de lui d'avoir dissipé ses biens. \EVERSE}
\newcommand{\lcXVIvIIfr}{\VERSE  Et il l'appela, et lui dit: Qu'est-ce que j'entends dire de toi?  Rends compte de ta gestion, car tu ne pourras plus désormais gérer mon bien. \EVERSE}
\newcommand{\lcXVIvIIIfr}{\VERSE  Alors l'économe dit en lui-même: Que ferai-je, puisque mon maître m'ôte la gestion de son bien?  Travailler la terre, je ne le puis, et je rougis de mendier. \EVERSE}
\newcommand{\lcXVIvIVfr}{\VERSE  Je sais ce que je ferai, afin que, lorsque j'aurai été destitué de la gestion, il y ait des gens qui me reçoivent dans leurs maisons. \EVERSE}
\newcommand{\lcXVIvVfr}{\VERSE  Ayant donc fait appeler chacun des débiteurs de son maître, il disait au premier: Combien dois-tu à mon maître? \EVERSE}
\newcommand{\lcXVIvVIfr}{\VERSE  Il répondit: Cent mesures d'huile.  Et l'économe lui dit: Prends ton obligation, assieds-toi vite, et écris cinquante. \EVERSE}
\newcommand{\lcXVIvVIIfr}{\VERSE  Il dit ensuite à un autre: Et toi, combien dois-tu?  Il répondit: Cent mesures de froment.  Et il lui dit: Prends ton obligation, et écris quatre-vingts. \EVERSE}
\newcommand{\lcXVIvVIIIfr}{\VERSE  Et le maître loua l'économe infidèle de ce qu'il avait agi habilement; car les enfants de ce siècle sont, dans leur monde, plus habiles que les enfants de lumière. \EVERSE}
\newcommand{\lcXVIvIXfr}{\VERSE  Et Moi Je vous dis: Faites-vous des amis avec les richesses d'iniquité, afin que, lorsque vous viendrez à manquer, ils vous reçoivent dans les tabernacles éternels. \EVERSE}
\newcommand{\lcXVIvXfr}{\VERSE  Celui qui est fidèle dans les moindres choses, est fidèle aussi dans les grandes; et celui qui est injuste dans les moindres choses, est injuste aussi dans les grandes. \EVERSE}
\newcommand{\lcXVIvXIfr}{\VERSE  Si donc vous n'avez pas été fidèles dans les richesses injustes, qui vous confiera les véritables? \EVERSE}
\newcommand{\lcXVIvXIIfr}{\VERSE  Et si vous n'avez pas été fidèles dans ce qui est à autrui, qui vous donnera ce qui est à vous? \EVERSE}
\newcommand{\lcXVIvXIIIfr}{\VERSE  Aucun serviteur ne peut servir deux maîtres; car ou il haïra l'un et aimera l'autre, ou il s'attachera à l'un et méprisera l'autre.  Vous ne pouvez pas servir Dieu et mammon. \EVERSE}
\newcommand{\lcXVIvXIVfr}{\VERSE  Or les pharisiens, qui étaient avares, entendaient toutes ces choses, et ils se moquaient de Lui. \EVERSE}
\newcommand{\lcXVIvXVfr}{\VERSE  Et Il leur dit: Vous, vous cherchez à paraître justes devant les hommes, mais Dieu connaît vos coeurs; car ce qui est grand pour les hommes est une abomination devant Dieu. \EVERSE}
\newcommand{\lcXVIvXVIfr}{\VERSE  La loi et les prophètes ont duré jusqu'à Jean; depuis lors, le royaume de Dieu est annoncé, et chacun fait effort pour y entrer. \EVERSE}
\newcommand{\lcXVIvXVIIfr}{\VERSE  Il est plus facile que le ciel et la terre passent, qu'il ne l'est qu'un seul trait de la loi vienne à tomber. \EVERSE}
\newcommand{\lcXVIvXVIIIfr}{\VERSE  Quiconque renvoie sa femme, et en épouse une autre, commet un adultère; et quiconque épouse celle qui a été renvoyée par son mari, commet un adultère. \EVERSE}
\newcommand{\lcXVIvXIXfr}{\VERSE  Il y avait un homme riche, qui était vêtu de pourpre et de lin, et qui faisait chaque jour une chère splendide. \EVERSE}
\newcommand{\lcXVIvXXfr}{\VERSE  Il y avait aussi un mendiant, nommé Lazare, qui était couché à sa porte, couvert d'ulceres, \EVERSE}
\newcommand{\lcXVIvXXIfr}{\VERSE  désirant se rassasier des miettes qui tombaient de la table du riche, et personne ne lui en donnait; mais les chiens venaient aussi, et léchaient ses plaies. \EVERSE}
\newcommand{\lcXVIvXXIIfr}{\VERSE  Or il arriva que le mendiant mourut, et fut emporté par les Anges dans le sein d'Abraham.  Le riche mourut aussi, et il fut enseveli dans l'enfer. \EVERSE}
\newcommand{\lcXVIvXXIIIfr}{\VERSE  Et levant les yeux, lorsqu'il était dans les tourments, il vit de loin Abraham, et Lazare dans son sein; \EVERSE}
\newcommand{\lcXVIvXXIVfr}{\VERSE  et s'écriant, il dit: Père Abraham, ayez pitié de moi, et envoyez Lazare, afin qu'il trempe l'extrémité de son doigt dans l'eau, pour rafraîchir ma langue, car je suis tourmenté dans cette flamme. \EVERSE}
\newcommand{\lcXVIvXXVfr}{\VERSE  Mais Abraham lui dit: Mons fils, souviens-toi que tu as reçu les biens pendant ta vie, et que Lazare a reçu de même les maux; or maintenant il est consolé, et toi, tu es tourmenté. \EVERSE}
\newcommand{\lcXVIvXXVIfr}{\VERSE  De plus, entre nous et vous un grand abîme a été établi; de sorte que ceux qui voudraient passer d'ici vers vous, ou de là venir ici, ne le peuvent pas. \EVERSE}
\newcommand{\lcXVIvXXVIIfr}{\VERSE  Le riche dit: Je vous supplie donc, père, de l'envoyer dans la maison de mon père; \EVERSE}
\newcommand{\lcXVIvXXVIIIfr}{\VERSE  car j'ai cinq frères, afin qu'il leur atteste ces choses, de peur qu'ils ne viennent, eux aussi, dans ce lieu de tourments. \EVERSE}
\newcommand{\lcXVIvXXIXfr}{\VERSE  Et Abraham lui dit: Ils ont Moïse et les prophètes; qu'ils les écoutent. \EVERSE}
\newcommand{\lcXVIvXXXfr}{\VERSE  Et il reprit: Non, père Abraham; mais si quelqu'un des morts va vers eux, ils feront pénitence. \EVERSE}
\newcommand{\lcXVIvXXXIfr}{\VERSE  Abraham lui dit: S'ils n'écoutent pas Moïse et les prophètes, quand même quelqu'un des morts ressusciterait, ils ne croiront pas. \EVERSE}
\newcommand{\lcXVIIvIfr}{\VERSE  Jésus dit à Ses disciples: Il est impossible qu'il n'arrive des scandales; mais malheur à celui par qui ils arrivent. \EVERSE}
\newcommand{\lcXVIIvIIfr}{\VERSE  Il vaudrait mieux pour lui qu'on lui mît au cou une meule de moulin, et qu'on le jetât dans la mer, que s'il scandalisait un de ces petits. \EVERSE}
\newcommand{\lcXVIIvIIIfr}{\VERSE  Prenez garde à vous.  Si ton frère a péché contre toi, reprends-le; et s'il se repent, pardonne-lui. \EVERSE}
\newcommand{\lcXVIIvIVfr}{\VERSE  S'il pèche contre toi sept fois dans un jour, et que sept fois dans un jour il revienne à toi, en disant: Je me repens, pardonne-lui. \EVERSE}
\newcommand{\lcXVIIvVfr}{\VERSE  Alors les Apôtres dirent au Seigneur: Augmentez-nous la foi. \EVERSE}
\newcommand{\lcXVIIvVIfr}{\VERSE  Et le Seigneur leur dit: Si vous avez la foi comme un grain de sénevé, vous direz à ce mûrier: Déracine-toi, et plante-toi dans la mer; et il vous obéira. \EVERSE}
\newcommand{\lcXVIIvVIIfr}{\VERSE  Qui de vous, ayant un serviteur qui laboure ou fait paître les troupeaux, lui dit, lorsqu'il revient des champs: Approche-toi vite, mets-toi à table? \EVERSE}
\newcommand{\lcXVIIvVIIIfr}{\VERSE  Ne lui dira-t-il pas: Prépare-moi à souper, et ceins-toi, et sers-moi jusqu'à ce que j'aie mangé et bu; après cela, tu mangeras et tu boiras? \EVERSE}
\newcommand{\lcXVIIvIXfr}{\VERSE  A-t-il de la reconnaissance pour ce serviteur, parce qu'il a fait ce qu'il lui avait ordonné? \EVERSE}
\newcommand{\lcXVIIvXfr}{\VERSE  Je ne le pense pas.  Et vous de même, quand vous aurez fait tout ce qui vous est commandé, dites: Nous sommes des serviteurs inutiles; nous avons fait ce que nous devions faire. \EVERSE}
\newcommand{\lcXVIIvXIfr}{\VERSE  Et il arriva, tandis qu'Il allait à Jérusalem, qu'Il passa par les confins de la Samarie et de la Galilée. \EVERSE}
\newcommand{\lcXVIIvXIIfr}{\VERSE  Et comme Il entrait dns un village, dix lépreux vinrent au-devant de Lui; et, se tenant éloignés, \EVERSE}
\newcommand{\lcXVIIvXIIIfr}{\VERSE  ils élevèrent la voix, en disant: Jésus, Maître, ayez pitié de nous. \EVERSE}
\newcommand{\lcXVIIvXIVfr}{\VERSE  Lorsqu'Il les eut vus, Il dit: Allez, montrez-vous aux prêtres.  Et comme ils y allaient, ils furent guéris. \EVERSE}
\newcommand{\lcXVIIvXVfr}{\VERSE  Or l'un d'eux, voyant qu'il était guéri, revint, glorifiant Dieu à haute voix. \EVERSE}
\newcommand{\lcXVIIvXVIfr}{\VERSE  Et il se jeta le visage contre terre aux pieds de Jésus, Lui rendant grâces, et celui-là était Samaritain. \EVERSE}
\newcommand{\lcXVIIvXVIIfr}{\VERSE  Alors Jésus, prenant la parole, dit: Est-ce que les dix n'ont pas été guéris?  Où sont donc les neuf autres? \EVERSE}
\newcommand{\lcXVIIvXVIIIfr}{\VERSE  Il ne s'en est pas trouvé qui soit revenu, et qui ait rendu gloire à Dieu si non cet étranger. \EVERSE}
\newcommand{\lcXVIIvXIXfr}{\VERSE  Et Il lui dit: Lève-toi, va; ta foi t'a sauvé. \EVERSE}
\newcommand{\lcXVIIvXXfr}{\VERSE  Les pharisiens Lui demandèrent: Quand viendra le royaume de Dieu?  Il leur répondit: Le royaume de Dieu ne vient pas d'une manière apparente; \EVERSE}
\newcommand{\lcXVIIvXXIfr}{\VERSE  et on ne dira point: Il est ici, ou: Il est là.  Car voici, le royaume de Dieu est au dedans de vous. \EVERSE}
\newcommand{\lcXVIIvXXIIfr}{\VERSE  Puis Il dit à Ses disciples: Des jours viendront où vous désirerez voir un jour du Fils de l'homme, et vous ne le verrez point. \EVERSE}
\newcommand{\lcXVIIvXXIIIfr}{\VERSE  Et l'on vous dira: Il est ici, Il est là.  Mais n'y allez pas, et ne les suivez pas. \EVERSE}
\newcommand{\lcXVIIvXXIVfr}{\VERSE  Car, comme l'éclair resplendit et brille d'une extrémité du ciel jusqu'à l'autre, ainsi sera le Fils de l'homme en Son jour. \EVERSE}
\newcommand{\lcXVIIvXXVfr}{\VERSE  Mais il faut auparavant qu'Il souffre beaucoup, et qu'Il soit rejeté par cette génération. \EVERSE}
\newcommand{\lcXVIIvXXVIfr}{\VERSE  Et comme il est arrivé aux jours de Noé, ainsi en sera-t-il aux jours du Fils de l'homme. \EVERSE}
\newcommand{\lcXVIIvXXVIIfr}{\VERSE  Les hommes mangeaient et buvaient, se mariaient et donnaient leurs filles en mariage, jusqu'au jour où Noé entra dans l'arche; et alors le déluge vint, et les fit tous périr. \EVERSE}
\newcommand{\lcXVIIvXXVIIIfr}{\VERSE  Et comme il est arrivé encore aux jours de Lot: les hommes mangeaient et buvaient, achetaient et vendaient, plantaient et bâtissaient; \EVERSE}
\newcommand{\lcXVIIvXXIXfr}{\VERSE  mais le jour où Lot sortit de Sodome, il tomba du ciel une pluie de feu et de soufre, qui les fit tous périr. \EVERSE}
\newcommand{\lcXVIIvXXXfr}{\VERSE  Il en sera de même le jour où le Fils de l'homme sera révélé. \EVERSE}
\newcommand{\lcXVIIvXXXIfr}{\VERSE  A cette heure-là, que celui qui sera sur le toit, et qui aura ses effets dans la maison, ne descende pas pour les prendre; et que celui qui sera dans les champs ne retourne pas non plus en arrière. \EVERSE}
\newcommand{\lcXVIIvXXXIIfr}{\VERSE  Souvenez-vous de la femme de Lot. \EVERSE}
\newcommand{\lcXVIIvXXXIIIfr}{\VERSE  Quiconque cherchera à sauver sa vie, la perdra; et quiconque la perdra, la sauvera. \EVERSE}
\newcommand{\lcXVIIvXXXIVfr}{\VERSE  Je vous le dis, en cette nuit-là, deux seront dans le même lit: l'un sera pris, et l'autre laissé. \EVERSE}
\newcommand{\lcXVIIvXXXVfr}{\VERSE  Deux femmes moudront ensemble: l'une sera prise, et l'autre laissée.  Deux hommes seront dans un champ: l'un sera pris, et l'autre laissé. \EVERSE}
\newcommand{\lcXVIIvXXXVIfr}{\VERSE  Prenant la parole, ils Lui dirent: Où sera-ce, Seigneur? \EVERSE}
\newcommand{\lcXVIIvXXXVIIfr}{\VERSE  Il leur répondit: Partout où sera le corps, là aussi se rassembleront les aigles. \EVERSE}
\newcommand{\lcXVIIIvIfr}{\VERSE  Il leur disait aussi une parabole, pour leur montrer qu'il faut toujours prier, et ne pas se lasser. \EVERSE}
\newcommand{\lcXVIIIvIIfr}{\VERSE  Il y avait, dit-il, dans une ville, un juge qui ne craignait pas Dieu et ne se souciait pas des hommes. \EVERSE}
\newcommand{\lcXVIIIvIIIfr}{\VERSE  Il y avait aussi, dans cette ville, une veuve qui venait auprès de lui, disant: Fais-moi justice de mon adversaire. \EVERSE}
\newcommand{\lcXVIIIvIVfr}{\VERSE  Et il refusait pendant longtemps; mais ensuite il dit en lui-même: Quoique je ne craigne pas Dieu, et que je ne me soucie pas des hommes, \EVERSE}
\newcommand{\lcXVIIIvVfr}{\VERSE  néanmoins, parce que cette veuve m'importune, je lui ferai justice, de peur qu'à la fin elle n'en vienne à me frapper. \EVERSE}
\newcommand{\lcXVIIIvVIfr}{\VERSE  Le Seigneur ajouta: Entendez ce que dit ce juge d'iniquité. \EVERSE}
\newcommand{\lcXVIIIvVIIfr}{\VERSE  Et Dieu ne ferait pas justice à Ses élus, qui crient à Lui jour et nuit, et Il tarderait à les secourir? \EVERSE}
\newcommand{\lcXVIIIvVIIIfr}{\VERSE  Je vous le dis, Il leur fera promptement justice.  Mais lorsque le Fils de l'homme viendra, pensez-vous qu'Il trouve la foi sur la terre? \EVERSE}
\newcommand{\lcXVIIIvIXfr}{\VERSE  Il dit aussi cette parabole à quelques-uns qui se confiaient en eux-mêmes, comme étant justes, et qui méprisaient les autres: \EVERSE}
\newcommand{\lcXVIIIvXfr}{\VERSE  Deux hommes montèrent au temple pour prier; l'un était pharisien, et l'autre publicain. \EVERSE}
\newcommand{\lcXVIIIvXIfr}{\VERSE  Le pharisien, se tenant debout, priait ainsi en lui-même: O Dieu, je Vous rends grâces de ce que je ne suis pas comme le reste des hommes, qui sont voleurs, injustes, adultères, ni même comme ce publicain. \EVERSE}
\newcommand{\lcXVIIIvXIIfr}{\VERSE  Je jeûne deux fois la semaine, je donne la dîme de tout ce que je possède. \EVERSE}
\newcommand{\lcXVIIIvXIIIfr}{\VERSE  Et le publicain, se tenant éloigné, n'osait pas même lever les yeux au Ciel; mais il frappait sa poitrine, en disant: O Dieu, ayez pitié de moi, qui suis un pécheur. \EVERSE}
\newcommand{\lcXVIIIvXIVfr}{\VERSE  Je vous le dis, celui-ci descendit dans sa maison justifié, plutôt que l'autre; car quiconque s'élève sera humilié, et quiconque s'humilie sera élevé. \EVERSE}
\newcommand{\lcXVIIIvXVfr}{\VERSE  On Lui amenait aussi de petits enfants, afin qu'Il les touchât; mais les disciples, voyant cela, les repoussaient. \EVERSE}
\newcommand{\lcXVIIIvXVIfr}{\VERSE  Mais Jésus, les appelant, dit: Laissez venir à Moi les petits enfants, et ne les en empêchez pas; car le royaume de Dieu est à ceux qui leur ressemblent. \EVERSE}
\newcommand{\lcXVIIIvXVIIfr}{\VERSE  En vérité, Je vous le dis, quiconque ne recevra pas le royaume de Dieu comme un enfant, n'y entrera point. \EVERSE}
\newcommand{\lcXVIIIvXVIIIfr}{\VERSE  Un chef de synagogue L'interrogea, en disant: Bon Maître, que ferai-je pour posséder la vie éternelle? \EVERSE}
\newcommand{\lcXVIIIvXIXfr}{\VERSE  Jésus lui dit: Pourquoi M'appelles-tu bon?  Nul n'est bon, si ce n'est Dieu seul. \EVERSE}
\newcommand{\lcXVIIIvXXfr}{\VERSE  Tu connais les commandements: Tu ne tueras point; Tu ne commettras pas d'adultère; Tu ne déroberas point; Tu ne porteras pas de faux témoignage; Honore ton père et ta mère. \EVERSE}
\newcommand{\lcXVIIIvXXIfr}{\VERSE  Il répondit: J'ai observé toutes ces choses depuis ma jeunesse. \EVERSE}
\newcommand{\lcXVIIIvXXIIfr}{\VERSE  Ayant entendu cela, Jésus lui dit: Il te manque encore une chose: vends tout ce que tu as, et donne-le aux pauvres, et tu auras un trésor dans le Ciel; puis viens, et suis-Moi. \EVERSE}
\newcommand{\lcXVIIIvXXIIIfr}{\VERSE  Mais lui, ayant entendu ces paroles, fut attristé; car il était très riche. \EVERSE}
\newcommand{\lcXVIIIvXXIVfr}{\VERSE  Et Jésus, voyant qu'il était devenu triste, dit: Qu'il sera difficile à ceux qui ont des richesses d'entrer dans le royaume de Dieu! \EVERSE}
\newcommand{\lcXVIIIvXXVfr}{\VERSE  Il est plus facile à un chameau de passer par le trou d'une aiguille, qu'à un riche d'entrer dans le royaume de Dieu. \EVERSE}
\newcommand{\lcXVIIIvXXVIfr}{\VERSE  Et ceux qui L'écoutaient dirent: Qui peut donc être sauvé? \EVERSE}
\newcommand{\lcXVIIIvXXVIIfr}{\VERSE  Il leur dit: Ce qui est impossible aux hommes est possible à Dieu. \EVERSE}
\newcommand{\lcXVIIIvXXVIIIfr}{\VERSE  Alors Pierre dit: Voici que nous avons tout quitté, et que nous Vous avons suivi. \EVERSE}
\newcommand{\lcXVIIIvXXIXfr}{\VERSE  Il leur dit: En vérité, Je vous le dis, personne ne quittera sa maison, ou ses parents, ou ses frères, ou sa femme, ou ses enfants, pour le royaume de Dieu, \EVERSE}
\newcommand{\lcXVIIIvXXXfr}{\VERSE  qu'il ne reçoive beaucoup plus dans le temps présent, et, dans le siècle à venir, la vie éternelle. \EVERSE}
\newcommand{\lcXVIIIvXXXIfr}{\VERSE  Ensuite, Jésus prit à part les douze, et leur dit: Voici que nous montons à Jérusalem, et tout ce que a été écrit par les prophètes au sujet du Fils de l'homme s'accomplira. \EVERSE}
\newcommand{\lcXVIIIvXXXIIfr}{\VERSE  Car Il sera livré aux gentils, et on se moquera de Lui, et on Le flagellera, et on crachera sur Lui; \EVERSE}
\newcommand{\lcXVIIIvXXXIIIfr}{\VERSE  et après qu'on L'aura flagellé, on Le fera mourir; et le troisième jour Il ressuscitera. \EVERSE}
\newcommand{\lcXVIIIvXXXIVfr}{\VERSE  Mais ils ne comprirent rien à cela; ce langage leur était caché, et ils ne saisissaient point ce que était dit. \EVERSE}
\newcommand{\lcXVIIIvXXXVfr}{\VERSE  Or il arriva, comme Il approchait de Jéricho, qu'un aveugle était assis au bord du chemin, demandant l'aumône. \EVERSE}
\newcommand{\lcXVIIIvXXXVIfr}{\VERSE  Et entendant la foule passer, il demanda ce que c'était. \EVERSE}
\newcommand{\lcXVIIIvXXXVIIfr}{\VERSE  On lui dit que c'était Jésus de Nazareth qui passait. \EVERSE}
\newcommand{\lcXVIIIvXXXVIIIfr}{\VERSE  Et il cria, en disant: Jésus, Fils de David, ayez pitié de moi. \EVERSE}
\newcommand{\lcXVIIIvXXXIXfr}{\VERSE  Et ceux qui marchaient en avant le reprenaient rudement pour qu'il se tût; mais il criait encore plus: Fils de David, ayez pitié de moi. \EVERSE}
\newcommand{\lcXVIIIvXLfr}{\VERSE  Alors Jésus, S'arrêtant, ordonna qu'on le Lui amenât.  Et lorsqu'il se fut approché, Il l'interrogea, \EVERSE}
\newcommand{\lcXVIIIvXLIfr}{\VERSE  en disant: Que veux-tu que Je te fasse?  Il répondit: Seigneur, que je voie. \EVERSE}
\newcommand{\lcXVIIIvXLIIfr}{\VERSE  Et Jésus lui dit: Vois; ta foi t'a sauvé. \EVERSE}
\newcommand{\lcXVIIIvXLIIIfr}{\VERSE  Et aussitôt il vit, et il Le suivait, en glorifiant Dieu.  Et tout le peuple, ayant vu cela, rendit gloire à Dieu. \EVERSE}
\newcommand{\lcXIXvIfr}{\VERSE  Jésus, étant entré dans Jéricho, traversait la ville. \EVERSE}
\newcommand{\lcXIXvIIfr}{\VERSE  Et voici qu'un homme, nommé Zachée, chef des publicains, et fort riche, \EVERSE}
\newcommand{\lcXIXvIIIfr}{\VERSE  cherchait à voir qui était Jésus; et il ne le pouvait, à cause de la foule, parce qu'il était petit de taille. \EVERSE}
\newcommand{\lcXIXvIVfr}{\VERSE  Courant donc en avant, il monta sur un sycomore pour Le voir, parce qu'Il devait passer par là. \EVERSE}
\newcommand{\lcXIXvVfr}{\VERSE  Arrivé en cet endroit, Jésus leva les yeux; et l'ayant vu, Il lui dit: Zachée, hâte-toi de descendre; car, aujourd'hui, il faut que Je demeure dans ta maison. \EVERSE}
\newcommand{\lcXIXvVIfr}{\VERSE  Zachée se hâta de descendre, et Le reçut avec joie. \EVERSE}
\newcommand{\lcXIXvVIIfr}{\VERSE  Voyant cela, tous murmuraient, disant qu'Il était allé loger chez un homme pécheur. \EVERSE}
\newcommand{\lcXIXvVIIIfr}{\VERSE  Cependant Zachée, se tenant devant le Seigneur, Lui dit: Seigneur, voici que je donne la moitié de mes biens aux pauvres; et si j'ai fait tort de quelque chose à quelqu'un, je lui rends le quadruple. \EVERSE}
\newcommand{\lcXIXvIXfr}{\VERSE  Jésus lui dit: Aujourd'hui le salut a été accordé à cette maison, parce que celui-ci est aussi un fils d'Abraham. \EVERSE}
\newcommand{\lcXIXvXfr}{\VERSE  Car le Fils de l'homme est venu chercher et sauver ce qui était perdu. \EVERSE}
\newcommand{\lcXIXvXIfr}{\VERSE  Comme ils écoutaient ces choses, Il ajouta une parabole, parce qu'Il était près de Jérusalem, et qu'ils pensaient que le royaume de Dieu allait être manifesté à l'instant. \EVERSE}
\newcommand{\lcXIXvXIIfr}{\VERSE  Il dit donc: Un homme de haute naissance s'en alla dans un pays lointain, pour prendre possession d'un royaume, et revenir ensuite. \EVERSE}
\newcommand{\lcXIXvXIIIfr}{\VERSE  Ayant appelé dix de ses serviteurs, il leur donna dix mines, et leur dit: Faites-les valoir jusqu'à ce que je revienne. \EVERSE}
\newcommand{\lcXIXvXIVfr}{\VERSE  Mais ses concitoyens le haïssaient, et ils envoyèrent après lui une ambassade, pour dire: Nous ne voulons pas que cet homme règne sur nous. \EVERSE}
\newcommand{\lcXIXvXVfr}{\VERSE  Et il arriva qu'à son retour, après avoir pris possession du royaume, il ordonna qu'on appelât les serviteurs auxquels il avait donné de l'argent, pour savoir comment chacun l'avait fait valoir. \EVERSE}
\newcommand{\lcXIXvXVIfr}{\VERSE  Le premier vint, et dit: Seigneur, ta mine a produit dix mines. \EVERSE}
\newcommand{\lcXIXvXVIIfr}{\VERSE  Et il lui dit: C'est bien, bon serviteur; parce que tu as été fidèle en peu de chose, tu auras puissance sur dix villes. \EVERSE}
\newcommand{\lcXIXvXVIIIfr}{\VERSE  Le second vint, et dit: Seigneur, ta mine a produit cinq mines. \EVERSE}
\newcommand{\lcXIXvXIXfr}{\VERSE  Et il lui dit: Et toi, sois établi sur cinq villes. \EVERSE}
\newcommand{\lcXIXvXXfr}{\VERSE  Un autre vint, et dit: Seigneur, voici ta mine, que j'ai tenue enveloppée dans un mouchoir; \EVERSE}
\newcommand{\lcXIXvXXIfr}{\VERSE  car je t'ai craint, parce que tu es un homme sévère: tu enlèves ce que tu n'as pas déposé, et tu moissonnes ce que tu n'as pas semé. \EVERSE}
\newcommand{\lcXIXvXXIIfr}{\VERSE  Il lui dit: Je te juge par ta propre bouche, méchant serviteur.  Tu savais que je suis un homme sévère, enlevant ce que je n'ai pas déposé, et moissonnant ce que je n'ai pas semé; \EVERSE}
\newcommand{\lcXIXvXXIIIfr}{\VERSE  pourquoi donc n'as-tu pas mis mon argent à la banque, afin qu'à mon retour je le retirasse avec les intérêts? \EVERSE}
\newcommand{\lcXIXvXXIVfr}{\VERSE  Puis il dit à ceux que étaient présents: Otez-lui la mine, et donnez-la à celui qui en a dix. \EVERSE}
\newcommand{\lcXIXvXXVfr}{\VERSE  Et ils lui dirent: Seigneur, il a dix mines. \EVERSE}
\newcommand{\lcXIXvXXVIfr}{\VERSE  Je vous le dis, on donnera à celui qui a déjà, et il sera dans l'abondance; mais à celui qui n'a pas, on ôtera même ce qu'il a. \EVERSE}
\newcommand{\lcXIXvXXVIIfr}{\VERSE  Quant à mes ennemis, qui n'ont pas voulu que je règne sur eux, amenez-les ici, et tuez-les devant moi. \EVERSE}
\newcommand{\lcXIXvXXVIIIfr}{\VERSE  Et après avoir ainsi parlé, Il marchait devant eux, montant à Jérusalem. \EVERSE}
\newcommand{\lcXIXvXXIXfr}{\VERSE  Et il arriva, lorsqu'Il approchait de Bethphagé et de Béthanie, près de la montagne appelée des Oliviers, qu'Il envoya deux de Ses disciples, \EVERSE}
\newcommand{\lcXIXvXXXfr}{\VERSE  en disant: Allez au village qui est en face; en y entrant, vous trouverez un ânon attaché, sur lequel aucun homme ne s'est jamais assis; déliez-le, et amenez-le. \EVERSE}
\newcommand{\lcXIXvXXXIfr}{\VERSE  Et si quelqu'un vous demande: Pourquoi le déliez-vous? vous lui répondrez: Parce que le Seigneur désire S'en servir. \EVERSE}
\newcommand{\lcXIXvXXXIIfr}{\VERSE  Ceux que étaient envoyés partirent donc et trouvèrent l'ânon, comme Il le leur avait dit. \EVERSE}
\newcommand{\lcXIXvXXXIIIfr}{\VERSE  Et comme ils déliaient l'ânon, ses maîtres leur dirent: Pourquoi déliez-vous cet ânon? \EVERSE}
\newcommand{\lcXIXvXXXIVfr}{\VERSE  Ils répondirent: Parce que le Seigneur en a besoin. \EVERSE}
\newcommand{\lcXIXvXXXVfr}{\VERSE  Et ils l'amenèrent à Jésus.  Et jetant leurs vêtements sur l'ânon, ils y placèrent Jésus. \EVERSE}
\newcommand{\lcXIXvXXXVIfr}{\VERSE  Et tandis qu'Il avançait, le peuple étendit ses vêtements sur le chemin. \EVERSE}
\newcommand{\lcXIXvXXXVIIfr}{\VERSE  Et lorsqu'Il approchait déjà de la descente de la montagne des Oliviers, toutes les foules des disciples, transportées de joie, se mirent à louer Dieu à haute voix pour toutes les merveilles qu'ils avaient vues, \EVERSE}
\newcommand{\lcXIXvXXXVIIIfr}{\VERSE  en disant: Béni soit le roi qui vient au nom du Seigneur!  Paix dans le Ciel, et gloire au plus haut des Cieux! \EVERSE}
\newcommand{\lcXIXvXXXIXfr}{\VERSE  Alors quelques-un des pharisiens, du milieu de la foule, Lui dirent: Maître, reprenez Vos disciples. \EVERSE}
\newcommand{\lcXIXvXLfr}{\VERSE  Il leur répondit: Je vous dis, s'ils se taisent, les pierres crieront. \EVERSE}
\newcommand{\lcXIXvXLIfr}{\VERSE  Et comme Il approchait, voyant la ville, Il pleura sur elle, en disant: \EVERSE}
\newcommand{\lcXIXvXLIIfr}{\VERSE  Si tu connaissais, toi aussi, au moins en ce jour qui t'est donné, ce qui te procurerait la paix!  Mais maintenant cela est caché à tes yeux. \EVERSE}
\newcommand{\lcXIXvXLIIIfr}{\VERSE  Il viendra sur toi des jours où tes ennemis t'environneront de tranchées, ou ils t'enfermeront et te serreront de toutes parts; \EVERSE}
\newcommand{\lcXIXvXLIVfr}{\VERSE  et ils te renverseront à terre, toi et tes enfants qui sont au milieu de toi, et ils ne laisseront pas en toi pierre sur pierre, parce que tu n'as pas connu le temps où tu as été visitée. \EVERSE}
\newcommand{\lcXIXvXLVfr}{\VERSE  Et étant entré dans le temple, Il Se mit à chasser ceux qui y vendaient et ceux qui y achetaient, \EVERSE}
\newcommand{\lcXIXvXLVIfr}{\VERSE  leur disant: Il est écrit: Ma maison est une maison de prière; mais vous, vous en avez fait une caverne de voleurs. \EVERSE}
\newcommand{\lcXIXvXLVIIfr}{\VERSE  Et Il enseignait tous les jours dans le temple.  Et les princes des prêtres, les scribes et les principaux du peuple cherchaient à Le perdre; \EVERSE}
\newcommand{\lcXIXvXLVIIIfr}{\VERSE  mais ils ne trouvaient pas ce qu'ils pourraient Lui faire, car tout le peuple était suspendu d'admiration en L'écoutant. \EVERSE}
\newcommand{\lcXXvIfr}{\VERSE  Et il arriva qu'un de ces jours-là, comme Il enseignait le peuple dans le temple et lui annonçait l'Evangile, les princes des prêtres et les scribes survinrent avec les anciens, \EVERSE}
\newcommand{\lcXXvIIfr}{\VERSE  et Lui parlèrent en ces termes: Dites-nous par quelle autorité Vous faites ces choses, ou quel est celui qui Vous a donné ce pouvoir. \EVERSE}
\newcommand{\lcXXvIIIfr}{\VERSE  Jésus, répondant, leur dit: Je vous adresserai, Moi aussi, une question.  Répondez-Moi: \EVERSE}
\newcommand{\lcXXvIVfr}{\VERSE  Le baptême de Jean était-il du Ciel, ou des hommes? \EVERSE}
\newcommand{\lcXXvVfr}{\VERSE  Mais ils pensaient en eux-mêmes, disant: Si nous répondons: Du Ciel, Il dira: Pourquoi donc n'avez-vous pas cru en lui? \EVERSE}
\newcommand{\lcXXvVIfr}{\VERSE  Et si nous répondons: Des hommes, tout le peuple nous lapidera; car il est persuadé que Jean était un prophète. \EVERSE}
\newcommand{\lcXXvVIIfr}{\VERSE  Ils répondirent donc qu'ils ne savaient d'où il était. \EVERSE}
\newcommand{\lcXXvVIIIfr}{\VERSE  Et Jésus leur dit: Moi non plus, Je ne vous dis point par quelle autorité Je fais ces choses. \EVERSE}
\newcommand{\lcXXvIXfr}{\VERSE  Alors Il Se mit à dire au peuple cette parabole: un homme planta une vigne, et la loua à des vignerons; puis lui-même il fut pendant longtemps hors du pays. \EVERSE}
\newcommand{\lcXXvXfr}{\VERSE  Et, dans la saison, il envoya un serviteur vers les vignerons, pour qu'ils lui donnassent du fruit de la vigne.  Après l'avoir battu, ils le renvoyèrent les mains vides. \EVERSE}
\newcommand{\lcXXvXIfr}{\VERSE  Il envoya encore un autre serviteur; mais ils le battirent aussi, et, après l'avoir accablé d'outrages, ils le renvoyèrent les mains vides. \EVERSE}
\newcommand{\lcXXvXIIfr}{\VERSE  Il envoya encore un troisième, qu'ils blessèrent aussi et chassèrent. \EVERSE}
\newcommand{\lcXXvXIIIfr}{\VERSE  Alors le maître de la vigne dit: Que ferai-je?  J'enverrai mons fils bien-aimé; peut-être, en le voyant, éprouveront-ils du respect. \EVERSE}
\newcommand{\lcXXvXIVfr}{\VERSE  Mais lorsque les vignerons le virent, ils pensèrent en eux-mêmes, et dirent: Celui-ci est l'héritier; tuons-le, afin que l'héritage soit à nous. \EVERSE}
\newcommand{\lcXXvXVfr}{\VERSE  Et l'ayant chassé hors de la vigne, ils le tuèrent.  Que leur fera donc le maître de la vigne? \EVERSE}
\newcommand{\lcXXvXVIfr}{\VERSE  Il viendra, et il fera périr ces vignerons, et il donnera la vigne à d'autres.  Ayant entendu cela, ils Lui dirent: A Dieu ne plaise! \EVERSE}
\newcommand{\lcXXvXVIIfr}{\VERSE  Mais Lui, les regardant, dit: Qu'est-ce donc que ceci qui est écrit: La pierre qu'ont rejetée ceux qui bâtissaient est devenue la tête de l'angle? \EVERSE}
\newcommand{\lcXXvXVIIIfr}{\VERSE  Quiconque tombera sur cette pierre sera brisé; et celui sur qui elle tombera, elle l'écrasera. \EVERSE}
\newcommand{\lcXXvXIXfr}{\VERSE  Les princes des prêtres et les scribes cherchaient à mettre les mains sur Lui à cette heure même, mais ils craignirent le peuple: car ils avaient reconnu que c'était contre eux qu'Il avait dit cette parabole. \EVERSE}
\newcommand{\lcXXvXXfr}{\VERSE  Et L'épiant, ils envoyèrent des hommes artificieux, qui feindraient d'être justes, pour Le surprendre dans Ses paroles, afin de Le livrer à l'autorité et à la puissance du gouverneur. \EVERSE}
\newcommand{\lcXXvXXIfr}{\VERSE  Et ils L'interrogèrent, en disant: Maître, nous savons que Vous parlez et enseignez avec droiture, et que Vous n'avez pas d'égard aux personnes, mais que Vous enseignez la voie de Dieu dans la vérité. \EVERSE}
\newcommand{\lcXXvXXIIfr}{\VERSE  Nous est-il permis de payer le tribut à César, ou non? \EVERSE}
\newcommand{\lcXXvXXIIIfr}{\VERSE  Considérant leur ruse, Il leur dit: Pourquoi Me tentez-vous? \EVERSE}
\newcommand{\lcXXvXXIVfr}{\VERSE  Montrez-Moi un denier.  De qui porte-t-il l'image et l'inscription?  Ils Lui répondirent: De César. \EVERSE}
\newcommand{\lcXXvXXVfr}{\VERSE  Alors Il leur dit: Rendez donc à César ce qui est à César, et à Dieu ce qui est à Dieu. \EVERSE}
\newcommand{\lcXXvXXVIfr}{\VERSE  Et ils ne purent rien reprendre dans Ses paroles devant le peuple; mais, ayant admiré Sa réponse, ils se turent. \EVERSE}
\newcommand{\lcXXvXXVIIfr}{\VERSE  Quelques-uns des sadducéens, qui nient qu'il y ait une résurrection, s'approchèrent ensuite, et L'interrogèrent, \EVERSE}
\newcommand{\lcXXvXXVIIIfr}{\VERSE  en disant: Maître, Moïse a écrit pour nous: Si le frère de quelqu'un, ayant une femme, meurt sans laisser d'enfants, son frère épousera sa femme, et suscitera une postérité à son frère. \EVERSE}
\newcommand{\lcXXvXXIXfr}{\VERSE  Or il y avait sept frères; et le premier épousa une femme, et mourut sans enfants. \EVERSE}
\newcommand{\lcXXvXXXfr}{\VERSE  Le second la prit, et mourut lui-même sans enfants. \EVERSE}
\newcommand{\lcXXvXXXIfr}{\VERSE  Le troisième la prit aussi, et de même tous les sept; et ils ne laissèrent pas de postérité, et ils moururent. \EVERSE}
\newcommand{\lcXXvXXXIIfr}{\VERSE  Enfin, après eux tous, la femme mourut aussi. \EVERSE}
\newcommand{\lcXXvXXXIIIfr}{\VERSE  A la résurrection donc, duquel d'entre eux sera-t-elle l'épouse? car les sept l'ont eue pour femme. \EVERSE}
\newcommand{\lcXXvXXXIVfr}{\VERSE  Jésus leur dit: Les enfants de ce siécle se marient et sont donnés en mariage; \EVERSE}
\newcommand{\lcXXvXXXVfr}{\VERSE  mais ceux qui seront jugés dignes du siècle à venir et de la résurrection des morts ne se marieront pas, et ne prendront pas de femme; \EVERSE}
\newcommand{\lcXXvXXXVIfr}{\VERSE  car ils ne pourront plus mourir, parce qu'ils sont égaux aux Anges, et qu'ils sont fils de Dieu, étant fils de la résurrection. \EVERSE}
\newcommand{\lcXXvXXXVIIfr}{\VERSE  Mais que les morts ressuscitent, Moïse le montre lui-même, à l'endroit du Buisson, lorsqu'il appelle le Seigneur le Dieu d'Abraham, le Dieu d'Isaac, et le Dieu de Jacob. \EVERSE}
\newcommand{\lcXXvXXXVIIIfr}{\VERSE  Or Dieu n'est point le Dieu des morts, mais des vivants; car tous sont vivants pour Lui. \EVERSE}
\newcommand{\lcXXvXXXIXfr}{\VERSE  Alors quelques-uns des scribes, prenant la parole, Lui dirent: Maître, vous avez bien répondu. \EVERSE}
\newcommand{\lcXXvXLfr}{\VERSE  Et ils n'osaient plus Lui faire aucune question. \EVERSE}
\newcommand{\lcXXvXLIfr}{\VERSE  Mais Il leur dit: Comment dit-on que le Christ est fils de David, \EVERSE}
\newcommand{\lcXXvXLIIfr}{\VERSE  puisque David lui-même dit dans le livre des Psaumes: Le Seigneur a dit à mon Seigneur: Assieds-Toi à Ma droite, \EVERSE}
\newcommand{\lcXXvXLIIIfr}{\VERSE  jusqu'à ce que Je fasse de Tes ennemis l'escabeau de Tes pieds? \EVERSE}
\newcommand{\lcXXvXLIVfr}{\VERSE  David L'appelle donc Seigneur; alors, comment est-Il son fils? \EVERSE}
\newcommand{\lcXXvXLVfr}{\VERSE  Tandis que tout le peuple L'écoutait, Il dit à Ses disciples: \EVERSE}
\newcommand{\lcXXvXLVIfr}{\VERSE  Gardez-vous des scribes, qui affectent de se promener en robes longues, et qui aiment les salutations sur la place publique, les premières chaires dans les synagogues et les premières places dans les festins, \EVERSE}
\newcommand{\lcXXvXLVIIfr}{\VERSE  qui dévorent les maisons des veuves sous prétexte de longues prières.  Ils recevront une condamnation plus sévère. \EVERSE}
\newcommand{\lcXXIvIfr}{\VERSE  Jésus, regardant, vit les riches qui mettaient leurs offrandes dans le tronc. \EVERSE}
\newcommand{\lcXXIvIIfr}{\VERSE  Il vit aussi une pauvre veuve, qui y mit deux petites pièces de monnaie. \EVERSE}
\newcommand{\lcXXIvIIIfr}{\VERSE  Et Il dit: En vérité, Je vous le dis, cette pauvre veuve a mis plus que tous les autres. \EVERSE}
\newcommand{\lcXXIvIVfr}{\VERSE  Car tous ceux-là ont donné de leur superflu, pour faire des offrandes à Dieu; mais celle-ci a donné de son indigence, tout ce qu'elle avait pour vivre. \EVERSE}
\newcommand{\lcXXIvVfr}{\VERSE  Et comme quelques-uns disaient du temple qu'il était bâti de belles pierres, et orné de riches dons, Il dit: \EVERSE}
\newcommand{\lcXXIvVIfr}{\VERSE  Des jours viendront où, de ce que vous voyez, il ne restera pas pierre sur pierre qui ne soit détruite. \EVERSE}
\newcommand{\lcXXIvVIIfr}{\VERSE  Et ils L'interrogèrent, disant: Maître, quand ces choses arriveront-elles? et à quel signe connaîtra-t-on qu'elles vont s'accomplir? \EVERSE}
\newcommand{\lcXXIvVIIIfr}{\VERSE  Jésus dit: Prenez garde d'être séduits; car beaucoup viendront sous Mon nom, disant: C'est Moi, et le temps est proche.  Ne les suivez donc pas. \EVERSE}
\newcommand{\lcXXIvIXfr}{\VERSE  Et lorsque vous entendrez parler de guerres et de séditions, ne soyez pas effrayés; car il faut que ces choses arrivent d'abord, mais ce ne sera pas encore aussitôt la fin. \EVERSE}
\newcommand{\lcXXIvXfr}{\VERSE  Alors Il leur dit: Nation se soulèvera contre nation, et royaume contre royaume. \EVERSE}
\newcommand{\lcXXIvXIfr}{\VERSE  Et il y aura de grands tremblements de terre en divers lieux, et des pestes, et des famines, et des choses effrayantes dans le ciel, et de grands signes. \EVERSE}
\newcommand{\lcXXIvXIIfr}{\VERSE  Mais, avant tout cela, on mettra les mains sur vous, et on vous persécutera, vous livrant aux synagogues et aux prisons, vous traînant devant les rois et les gouverneurs, à cause de Mon nom; \EVERSE}
\newcommand{\lcXXIvXIIIfr}{\VERSE  et cela vous arrivera pour que vous rendiez témoignage. \EVERSE}
\newcommand{\lcXXIvXIVfr}{\VERSE  Mettez donc dans vos coeurs que vous n'aurez pas à méditer d'avance comment vous répondrez; \EVERSE}
\newcommand{\lcXXIvXVfr}{\VERSE  car Je vous donnerai une bouche et une sagesse auxquelles tous vos adversaires ne pourront résister et contredire. \EVERSE}
\newcommand{\lcXXIvXVIfr}{\VERSE  Vous serez livrés par vos parents, et par vos frères, et par vos proches, et par vos amis, et l'on fera mourir plusieurs d'entre vous; \EVERSE}
\newcommand{\lcXXIvXVIIfr}{\VERSE  et vous serez haïs de tous à cause de Mon nom. \EVERSE}
\newcommand{\lcXXIvXVIIIfr}{\VERSE  Mais pas un cheveu de votre tête ne périra. \EVERSE}
\newcommand{\lcXXIvXIXfr}{\VERSE  C'est par votre patience que vous sauverez vos vies. \EVERSE}
\newcommand{\lcXXIvXXfr}{\VERSE  Lorsque vous verrez Jérusalem entourée par une armée, alors sachez que sa désolation est proche. \EVERSE}
\newcommand{\lcXXIvXXIfr}{\VERSE  Alors, que ceux qui sont dans la Judée s'enfuient dans les montagnes, et que ceux qui sont au milieu d'elle en sortent, et que ceux qui sont dans les environs n'y entrent point. \EVERSE}
\newcommand{\lcXXIvXXIIfr}{\VERSE  Car ce seront des jours de vengeance, afin que s'accomplisse tout ce qui est écrit. \EVERSE}
\newcommand{\lcXXIvXXIIIfr}{\VERSE  Malheur à celles qui seront enceintes et qui allaiteront en ces jours-là!  Car il y aura une grande détresse dans le pays, et de la colère contre ce peuple. \EVERSE}
\newcommand{\lcXXIvXXIVfr}{\VERSE  Ils tomberont sous le tranchant du glaive, et ils seront emmenés captifs dans toutes les nations, et Jérusalem sera foulée aux pieds par les gentils, jusqu'à ce que le temps des nations soit accompli. \EVERSE}
\newcommand{\lcXXIvXXVfr}{\VERSE  Il y aura des signes dans le soleil, dans la lune et dans les étoiles, et, sur la terre, détresse des nations, à cause du bruit confus de la mer et des flots, \EVERSE}
\newcommand{\lcXXIvXXVIfr}{\VERSE  les hommes séchant de frayeur, dans l'attente de ce qui doit arriver à tout l'univers; car les puissances des cieux seront ébranlées. \EVERSE}
\newcommand{\lcXXIvXXVIIfr}{\VERSE  Et alors on verra le Fils de l'homme venant sur une nuée, avec une grande puissance et une grande majesté. \EVERSE}
\newcommand{\lcXXIvXXVIIIfr}{\VERSE  Or, lorsque ces choses commenceront à arriver, regardez et levez la tête, parce que votre rédemption approche. \EVERSE}
\newcommand{\lcXXIvXXIXfr}{\VERSE  Et Il leur proposa cette comparaison: Voyez le figuier et tous les arbres. \EVERSE}
\newcommand{\lcXXIvXXXfr}{\VERSE  Lorsqu'ils commencent à produire leur fruit, vous savez que l'été est proche. \EVERSE}
\newcommand{\lcXXIvXXXIfr}{\VERSE  De même, quand vous verrez arriver ces choses, sachez que le royaume de Dieu est proche. \EVERSE}
\newcommand{\lcXXIvXXXIIfr}{\VERSE  En vérité, Je vous le dis, cette race ne passera point, que tout ne s'accomplisse. \EVERSE}
\newcommand{\lcXXIvXXXIIIfr}{\VERSE  Le ciel et la terre passeront; mais Mes paroles ne passeront point. \EVERSE}
\newcommand{\lcXXIvXXXIVfr}{\VERSE  Prenez donc garde à vous, de peur que vos coeurs ne s'appesantissent par l'excès du manger et du boire, et par les soucis de cette vie, et que ce jour ne vienne sur vous à l'improviste; \EVERSE}
\newcommand{\lcXXIvXXXVfr}{\VERSE  car il viendra comme un filet sur tous ceux qui habitent sur la face de toute la terre. \EVERSE}
\newcommand{\lcXXIvXXXVIfr}{\VERSE  Veillez donc, priant en tout temps, afin que vous soyez trouvés dignes d'échapper à tous ces maux qui arriveront, et de paraître devant le Fils de l'homme. \EVERSE}
\newcommand{\lcXXIvXXXVIIfr}{\VERSE  Or, pendant le jour, Il enseignait dans le temple, et la nuit Il sortait, et demeurait sur la montagne appelée des Oliviers. \EVERSE}
\newcommand{\lcXXIvXXXVIIIfr}{\VERSE  Et tout le peuple venait à Lui de grand matin dans le temple pour L'écouter. \EVERSE}
\newcommand{\lcXXIIvIfr}{\VERSE  Cependant la fête des Azymes, appelée la Pâque, était proche, \EVERSE}
\newcommand{\lcXXIIvIIfr}{\VERSE  et les princes des prêtres et les scribes cherchaient comment ils feraient mourir Jésus; mais ils craignaient le peuple. \EVERSE}
\newcommand{\lcXXIIvIIIfr}{\VERSE  Or Satan entra dans Judas, surnommé Iscariote, l'un des douze. \EVERSE}
\newcommand{\lcXXIIvIVfr}{\VERSE  Et il alla, et s'entretint avec les princes des prêtres et les magistrats, de la manière dont il Le leur livrerait. \EVERSE}
\newcommand{\lcXXIIvVfr}{\VERSE  Ils se réjouirent, et convinrent de lui donner de l'argent. \EVERSE}
\newcommand{\lcXXIIvVIfr}{\VERSE  Il s'engagea, et il cherchait une occasion favorable pour Le livrer à l'insu des foules. \EVERSE}
\newcommand{\lcXXIIvVIIfr}{\VERSE  Cependant arriva le jour des Azymes, où il fallait immoler la pâque. \EVERSE}
\newcommand{\lcXXIIvVIIIfr}{\VERSE  Et Jésus envoya Pierre et Jean, en disant: Allez, et préparez-nous la pâque, afin que nous la mangions. \EVERSE}
\newcommand{\lcXXIIvIXfr}{\VERSE  Ils Lui dirent: Où voulez-vous que nous la préparions? \EVERSE}
\newcommand{\lcXXIIvXfr}{\VERSE  Il leur répondit: Voici, lorsque vous entrerez dans la ville, vous rencontrerez un homme portant une cruche d'eau; suivez-le dans la maison où il entrera, \EVERSE}
\newcommand{\lcXXIIvXIfr}{\VERSE  Et vous direz au père de famille de cette maison: Le Maître te dit: Où est la salle où Je pourrai manger la pâque avec Mes disciples? \EVERSE}
\newcommand{\lcXXIIvXIIfr}{\VERSE  Et il vous montrera une grande chambre haute, meublée; et là, faites les préparatifs. \EVERSE}
\newcommand{\lcXXIIvXIIIfr}{\VERSE  S'en allant donc ils trouvèrent comme Il leur avait dit, et ils préparèrent la pâque. \EVERSE}
\newcommand{\lcXXIIvXIVfr}{\VERSE  Quand l'heure fut venue, Il Se mit à table, et les douze Apôtres avec Lui. \EVERSE}
\newcommand{\lcXXIIvXVfr}{\VERSE  Et Il leur dit: J'ai désiré d'un grand désir de manger cette pâque avec vous, avant de souffrir. \EVERSE}
\newcommand{\lcXXIIvXVIfr}{\VERSE  Car, Je vous le dis, désormais Je ne la mangerai plus, jusqu'à ce qu'elle soit accomplie dans le royaume de Dieu. \EVERSE}
\newcommand{\lcXXIIvXVIIfr}{\VERSE  Et ayant pris le calice, Il rendit grâces, et dit: Prenez, et partagez entre vous. \EVERSE}
\newcommand{\lcXXIIvXVIIIfr}{\VERSE  Car, Je vous le dis, Je ne boirai plus du fruit de la vigne, jusqu'à ce que le règne de Dieu soit arrivé. \EVERSE}
\newcommand{\lcXXIIvXIXfr}{\VERSE  Puis, ayant pris du pain, Il rendit grâces, le rompit, et le leur donna, en disant: Ceci est Mon corps, qui est donné pour vous; faites ceci en mémoire de Moi. \EVERSE}
\newcommand{\lcXXIIvXXfr}{\VERSE  Il prit de même le calice, après qu'Il eut soupé, en disant: Ce calice est la nouvelle alliance en Mon sang, qui sera répandu pour vous. \EVERSE}
\newcommand{\lcXXIIvXXIfr}{\VERSE  Cependant, voici que la main de celui qui Me trahit est avec Moi à cette table. \EVERSE}
\newcommand{\lcXXIIvXXIIfr}{\VERSE  Quant au Fils de l'homme, Il S'en va, selon ce qui a été déterminé; mais malheur à l'homme par qui Il sera trahi! \EVERSE}
\newcommand{\lcXXIIvXXIIIfr}{\VERSE  Et ils commencèrent à se demander mutuellement quel était celui d'entre eux qui ferait cela. \EVERSE}
\newcommand{\lcXXIIvXXIVfr}{\VERSE  Il s'éleva aussi parmi eux une contestation, pour savoir lequel d'entre eux devait être estimé le plus grand. \EVERSE}
\newcommand{\lcXXIIvXXVfr}{\VERSE  Mais Il leur dit: Les rois des nations leur commandent en maîtres, et ceux qui ont l'autorité sur elles sont appelés leurs bienfaiteurs. \EVERSE}
\newcommand{\lcXXIIvXXVIfr}{\VERSE  Qu'il n'en soit pas ainsi de vous; mais que celui qui est le plus grand parmi vous devienne le plus petit; et celui qui gouverne, comme celui qui sert. \EVERSE}
\newcommand{\lcXXIIvXXVIIfr}{\VERSE  Car lequel est le plus grand?  celui qui est à table, ou celui qui sert?  N'est-ce pas celui qui est à table?  Moi, cependant, Je suis au milieu de vous comme celui qui sert. \EVERSE}
\newcommand{\lcXXIIvXXVIIIfr}{\VERSE  Vous, vous êtes demeurés avec Moi dans Mes tentations; \EVERSE}
\newcommand{\lcXXIIvXXIXfr}{\VERSE  et Moi, Je vous prépare le royaume, comme Mon Père Me l'a préparé, \EVERSE}
\newcommand{\lcXXIIvXXXfr}{\VERSE  afin que vous mangiez et buviez à Ma table dans Mon royaume, et que vous soyez assis sur des trônes, jugeant les douze tribus d'Israël. \EVERSE}
\newcommand{\lcXXIIvXXXIfr}{\VERSE  Le Seigneur dit encore: Simon, Simon, voici que Satan vous a réclamés, pour vous cribler comme le froment; \EVERSE}
\newcommand{\lcXXIIvXXXIIfr}{\VERSE  mais J'ai prié pour toi, afin que ta foi ne défaille point; et toi, lorsque tu seras converti, affermis tes frères. \EVERSE}
\newcommand{\lcXXIIvXXXIIIfr}{\VERSE  Pierre Lui dit: Seigneur, je suis prêt à aller, avec Vous, et en prison et à la mort. \EVERSE}
\newcommand{\lcXXIIvXXXIVfr}{\VERSE  Mais Jésus dit: Pierre, Je te le dis, le coq ne chantera pas aujourd'hui, que tu n'aies nié trois fois que tu Me connais.  Et Il leur dit: \EVERSE}
\newcommand{\lcXXIIvXXXVfr}{\VERSE  Lorsque Je vous ai envoyé sans sac, sans bourse et sans chaussures, vous a-t-il manqué quelque chose? \EVERSE}
\newcommand{\lcXXIIvXXXVIfr}{\VERSE  Ils répondirent: Rien.  Il ajouta: Mais maintenant, que celui qui a un sac le prenne, et une bourse également; et que celui qui n'en a point vende sa tunique, et achète une épée. \EVERSE}
\newcommand{\lcXXIIvXXXVIIfr}{\VERSE  Car, Je vous le dis, il faut encore que cette parole qui est écrite s'accomplisse en Moi: Il a étê mis au rang des scélérats.  En effet, ce qui Me concerne touche à sa fin. \EVERSE}
\newcommand{\lcXXIIvXXXVIIIfr}{\VERSE  Et ils dirent: Seigneur, voici deux épées.  Et Il leur dit: Cela suffit. \EVERSE}
\newcommand{\lcXXIIvXXXIXfr}{\VERSE  Et étant sorti, Il alla, selon Sa coutume, à la montagne des Oliviers, et Ses disciples Le suivirent. \EVERSE}
\newcommand{\lcXXIIvXLfr}{\VERSE  Lorsqu'Il fut arrivé dans ce lieu, Il leur dit: Priez, afin que vous ne succombiez pas à la tentation. \EVERSE}
\newcommand{\lcXXIIvXLIfr}{\VERSE  Puis Il S'éloigna d'eux à la distance d'un jet de pierre; et S'étant mis à genoux, Il priait, \EVERSE}
\newcommand{\lcXXIIvXLIIfr}{\VERSE  en disant: Père, si Vous le voulez, éloignez ce calice de Moi; cependant, que ce ne soit pas Ma volonté qui se fasse, mais la Vôtre. \EVERSE}
\newcommand{\lcXXIIvXLIIIfr}{\VERSE  Alors un Ange Lui apparut du Ciel, pour Le fortifier.  Et étant tombé en agonie, Il priait plus instamment. \EVERSE}
\newcommand{\lcXXIIvXLIVfr}{\VERSE  Et Sa sueur devint comme des gouttes de sang qui coulait jusqu'à terre. \EVERSE}
\newcommand{\lcXXIIvXLVfr}{\VERSE  S'étant levé après Sa prière, Il vint à Ses disciples, et Il les trouva endormis de tristesse. \EVERSE}
\newcommand{\lcXXIIvXLVIfr}{\VERSE  Et Il leur dit: Pourquoi dormez-vous?  Levez-vous et priez, afin que vous ne succombiez pas à la tentation. \EVERSE}
\newcommand{\lcXXIIvXLVIIfr}{\VERSE  Comme Il parlait encore, voici qu'une troupe parut, et celui qui s'appelait Judas, l'un des douze, marchait devant elle; et il s'approcha de Jésus pour Le baiser. \EVERSE}
\newcommand{\lcXXIIvXLVIIIfr}{\VERSE  Jésus lui dit: Judas, tu trahis le Fils de l'homme par un baiser? \EVERSE}
\newcommand{\lcXXIIvXLIXfr}{\VERSE  Ceux que étaient autour de Lui, voyant ce qui allait arriver, Lui dirent: Seigneur, frapperons-nous de l'épée? \EVERSE}
\newcommand{\lcXXIIvLfr}{\VERSE  Et l'un deux frappa le serviteur du grand prêtre, et lui coupa l'oreille droite. \EVERSE}
\newcommand{\lcXXIIvLIfr}{\VERSE  Mais Jésus, prenant la parole, dit: Restez-en là.  Et ayant touché l'oreille de cet homme, Il le guérit. \EVERSE}
\newcommand{\lcXXIIvLIIfr}{\VERSE  Puis Jésus dit à ceux qui étaient venus vers Lui, princes des prêtres, magistrats du temple, et anciens: Vous êtes sortis avec des épées et des bâtons, comme contre un brigand. \EVERSE}
\newcommand{\lcXXIIvLIIIfr}{\VERSE  Quand J'étais tous les jours avec vous dans le temple vous n'avez pas étendu les mains sur Moi; mais c'est ici votre heure, et la puissance des ténèbres. \EVERSE}
\newcommand{\lcXXIIvLIVfr}{\VERSE  Se saisissant alors de Lui, ils L'emmenèrent dans la maison du grand prêtre; et Pierre suivait de loin. \EVERSE}
\newcommand{\lcXXIIvLVfr}{\VERSE  Or, ayant allumé du feu au milieu de la cour, ils s'assirent autour; et Pierre était au milieu d'eux. \EVERSE}
\newcommand{\lcXXIIvLVIfr}{\VERSE  Une servante, qui le vit assis devant le feu, le fixa attentivement, et dit: Celui-ci était aussi avec Lui. \EVERSE}
\newcommand{\lcXXIIvLVIIfr}{\VERSE  Mais il renia Jésus, en disant: Femme, je ne Le connais pas. \EVERSE}
\newcommand{\lcXXIIvLVIIIfr}{\VERSE  Un peu après, un autre, le voyant, dit: Toi aussi, tu es de ces gens-là.  Mais Pierre dit: O homme, je n'en suis pas. \EVERSE}
\newcommand{\lcXXIIvLIXfr}{\VERSE  Et environ une heure plus tard, un autre affirmait la même chose, en disant: Certainement cet homme était aussi avec Lui; car il est Galiléen. \EVERSE}
\newcommand{\lcXXIIvLXfr}{\VERSE  Et Pierre dit: Homme, je ne sais pas ce que tu dis.  Et aussitôt, comme il parlait encore, le coq chanta. \EVERSE}
\newcommand{\lcXXIIvLXIfr}{\VERSE  Et le Seigneur, Se retournant, regarda Pierre.  Et Pierre se souvint de la parole que le Seigneur avait dite: Avant que le coq chante, tu Me renieras trois fois. \EVERSE}
\newcommand{\lcXXIIvLXIIfr}{\VERSE  Et Pierre, étant sorti dehors, pleura amèrement. \EVERSE}
\newcommand{\lcXXIIvLXIIIfr}{\VERSE  Ceux que tenaient Jésus se moquaient de lui, en Le frappant. \EVERSE}
\newcommand{\lcXXIIvLXIVfr}{\VERSE  Et ils Lui voilèrent la face, et ils Le frappaient au visage; et ils L'interrogeaient, en disant: Prophétise, qui est-ce qui T'a frappé? \EVERSE}
\newcommand{\lcXXIIvLXVfr}{\VERSE  Et ils proféraient contre Lui beaucoup d'autres blasphèmes. \EVERSE}
\newcommand{\lcXXIIvLXVIfr}{\VERSE  Lorsqu'il fit jour, les anciens du peuple, les princes des prêtres et les scribes s'assemblèrent; et L'ayant fait venir dans leur conseil, ils dirent: Si Tu es le Christ, dis-le-nous. \EVERSE}
\newcommand{\lcXXIIvLXVIIfr}{\VERSE  Il leur répondit: Si Je vous le dis, vous ne Me croirez pas; \EVERSE}
\newcommand{\lcXXIIvLXVIIIfr}{\VERSE  et si Je vous interroge, vous ne Me répondrez pas, et vous ne Me relâcherez pas. \EVERSE}
\newcommand{\lcXXIIvLXIXfr}{\VERSE  Mais désormais le Fils de l'homme sera assis à la droite de la puissance de Dieu. \EVERSE}
\newcommand{\lcXXIIvLXXfr}{\VERSE  Alors tous dirent: Tu es donc le Fils de Dieu?  Il répondit: Vous le dites, Je le suis. \EVERSE}
\newcommand{\lcXXIIvLXXIfr}{\VERSE  Et ils dirent: Qu'avons-nous encore besoin de témoignage?  Nous l'avons entendu nous-mêmes de Sa bouche. \EVERSE}
\newcommand{\lcXXIIIvIfr}{\VERSE  Et, s'étant tous levés, ils Le conduisirent à Pilate. \EVERSE}
\newcommand{\lcXXIIIvIIfr}{\VERSE  Et ils commencèrent à L'accuser, en disant: Nous avons trouvé cet homme pervertissant notre nation, empêchant de payer le tribut à César, et Se disant le Christ-roi. \EVERSE}
\newcommand{\lcXXIIIvIIIfr}{\VERSE  Pilate L'interrogea, en disant: Est-Tu le roi des Juifs?  Jésus répondit: tu le dis. \EVERSE}
\newcommand{\lcXXIIIvIVfr}{\VERSE  Alors Pilate dit aux princes des prêtres et aux foules: Je ne trouve rien de criminel dans cet homme. \EVERSE}
\newcommand{\lcXXIIIvVfr}{\VERSE  Mais ils insistaient, en disant: Il soulève le peuple, en enseignant par toute la Judée, depuis la Galilée, où Il a commencé, jusqu'ici. \EVERSE}
\newcommand{\lcXXIIIvVIfr}{\VERSE  Pilate, entendant parler de la Galilée, demanda si cet homme était Galiléen. \EVERSE}
\newcommand{\lcXXIIIvVIIfr}{\VERSE  Et ayant appris qu'Il était de la juridiction d'Hérode, il Le renvoya à Hérode, qui était aussi à Jérusalem en ces jours-là. \EVERSE}
\newcommand{\lcXXIIIvVIIIfr}{\VERSE  Hérode, voyant Jésus, en eut une grande joie; car il désirait depuis longtemps Le voir, parce qu'il avait entendu dire beaucoup de choses de Lui, et il espérait Lui voir faire quelque miracle. \EVERSE}
\newcommand{\lcXXIIIvIXfr}{\VERSE  Il Lui adressait donc de nombreuses questions; mais Jésus ne lui répondit rien. \EVERSE}
\newcommand{\lcXXIIIvXfr}{\VERSE  Cependant les princes des prêtres et les scribes étaient là, L'accusant sans relâche. \EVERSE}
\newcommand{\lcXXIIIvXIfr}{\VERSE  Or Hérode, avec ses gardes, Le méprisa, et il se moqua de Lui en Le revêtant d'une robe blanche; puis il Le renvoya à Pilate. \EVERSE}
\newcommand{\lcXXIIIvXIIfr}{\VERSE  Hérode et Pilate devinrent amis en ce jour même, d'ennemis qu'ils étaient auparavant. \EVERSE}
\newcommand{\lcXXIIIvXIIIfr}{\VERSE  Or Pilate, ayant convoqué les princes des prêtres, les magistrats et le peuple, \EVERSE}
\newcommand{\lcXXIIIvXIVfr}{\VERSE  leur dit: Vous m'avez présenté cet homme comme portant la nation à la révolte; et voici que, L'interrogeant devant vous, je ne L'ai trouvé coupable d'aucun des crimes dont vous L'accusez. \EVERSE}
\newcommand{\lcXXIIIvXVfr}{\VERSE  Ni Hérode non plus; car je vous ai renvoyés à lui, et on n'a rien fait à l'accusé qui montre qu'Il mérite la mort. \EVERSE}
\newcommand{\lcXXIIIvXVIfr}{\VERSE  Je Le renverrai donc, après L'avoir châtié. \EVERSE}
\newcommand{\lcXXIIIvXVIIfr}{\VERSE  Or il était obligé de leur délivrer un prisonnier le jour de la fête. \EVERSE}
\newcommand{\lcXXIIIvXVIIIfr}{\VERSE  Et la foule tout entière s'écria: Fais mourir Celui-ci, et délivre-nous Barabbas. \EVERSE}
\newcommand{\lcXXIIIvXIXfr}{\VERSE  Cet homme avait été mis en prison, à cause d'une sédition qui avait eu lieu dans la ville, et d'un meurtre. \EVERSE}
\newcommand{\lcXXIIIvXXfr}{\VERSE  Pilate leur parla de nouveau, voulant délivrer Jésus. \EVERSE}
\newcommand{\lcXXIIIvXXIfr}{\VERSE  Mais ils criaient plus fort, disant: Crucifie-Le, crucifie-Le! \EVERSE}
\newcommand{\lcXXIIIvXXIIfr}{\VERSE  Il leur dit pour la troisième fois: Mais quel mal a-t-Il fait?  Je ne trouve en Lui rien qui mérite la mort; je vais donc Le châtier, et je Le renverrai. \EVERSE}
\newcommand{\lcXXIIIvXXIIIfr}{\VERSE  Mais ils insistaient à grands cris, demandant qu'Il fût crucifié; et leurs clameurs redoublaient. \EVERSE}
\newcommand{\lcXXIIIvXXIVfr}{\VERSE  Alors Pilate ordonna que ce qu'ils demandaient fût exécuté. \EVERSE}
\newcommand{\lcXXIIIvXXVfr}{\VERSE  Il leur délivra celui qu'ils réclamaient, qui avait été mis en prison pour meurtre et sédition, et il livra Jésus à leur volonté. \EVERSE}
\newcommand{\lcXXIIIvXXVIfr}{\VERSE  Et comme ils L'emmenaient, ils prirent un certain Simon de Cyrène, qui revenait des champs, et ils le chargèrent de la croix, la lui faisant porter derrière Jésus. \EVERSE}
\newcommand{\lcXXIIIvXXVIIfr}{\VERSE  Or Il était suivi d'une grande foule de peuple, et de femmes qui se frappaient la poitrine et qui se lamentaient sur Lui. \EVERSE}
\newcommand{\lcXXIIIvXXVIIIfr}{\VERSE  Mais Jésus, Se retournant vers elles, dit: Filles de Jérusalem, ne pleurez pas sur Moi; mais pleurez sur vous-mêmes et sur vos enfants; \EVERSE}
\newcommand{\lcXXIIIvXXIXfr}{\VERSE  car voici qu'il viendra des jours où l'on dira: Heureuses les stériles, et les entrailles qui n'ont pas d'enfants, et les mamelles qui n'ont point allaité. \EVERSE}
\newcommand{\lcXXIIIvXXXfr}{\VERSE  Alors ils se mettront à dire aux montagnes: Tombez sur nous; et aux collines: Couvrez-nous. \EVERSE}
\newcommand{\lcXXIIIvXXXIfr}{\VERSE  Car s'ils traitent ainsi le bois vert, que fera-t-on au bois sec? \EVERSE}
\newcommand{\lcXXIIIvXXXIIfr}{\VERSE  On conduisait aussi avec Lui deux autres hommes, qui étaient des malfaiteurs, pour les mettre à mort. \EVERSE}
\newcommand{\lcXXIIIvXXXIIIfr}{\VERSE  Et lorsqu'ils furent arrivés au lieu appelé Calvaire, ils L'y crucifièrent, ainsi que des voleurs, l'un à droite et l'autre à gauche. \EVERSE}
\newcommand{\lcXXIIIvXXXIVfr}{\VERSE  Et Jésus disait: Père, pardonnez-leur, car ils ne savent ce qu'ils font.  Partageant ensuite Ses vêtements, ils les tirèrent au sort. \EVERSE}
\newcommand{\lcXXIIIvXXXVfr}{\VERSE  Et le peuple se tenait là, regardant; et, avec lui, les chefs se moquaient de Jésus, en disant: Il a sauvé les autres; qu'Il Se sauve Lui-même, s'Il est le Christ, l'élu de Dieu. \EVERSE}
\newcommand{\lcXXIIIvXXXVIfr}{\VERSE  Les soldats aussi L'insultaient, s'approchant de Lui, et Lui présentant du vinaigre, \EVERSE}
\newcommand{\lcXXIIIvXXXVIIfr}{\VERSE  et disant: Si Tu es le roi des Juifs, sauve-Toi. \EVERSE}
\newcommand{\lcXXIIIvXXXVIIIfr}{\VERSE  Il y avait aussi au-dessus de Lui une inscription, écrite en grec, en latin et en hébreu: Celui-ci est le roi des Juifs. \EVERSE}
\newcommand{\lcXXIIIvXXXIXfr}{\VERSE  Or l'un des voleurs suspendus en croix Le blasphémait, en disant: Si Tu es le Christ, sauve-Toi Toi-même, et nous avec Toi. \EVERSE}
\newcommand{\lcXXIIIvXLfr}{\VERSE  Mais l'autre le reprenait, en disant: Toi non plus, tu ne crains donc pas Dieu, toi qui es condamné au même supplice? \EVERSE}
\newcommand{\lcXXIIIvXLIfr}{\VERSE  Encore, pour nous, c'est justice, car nous recevons ce qu'ont mérité nos oeuvres; mais Celui-ci n'a fait aucun mal. \EVERSE}
\newcommand{\lcXXIIIvXLIIfr}{\VERSE  Et il disait à Jésus: Seigneur, souvenez-Vous de moi, lorsque Vous serez arrivé dans Votre royaume. \EVERSE}
\newcommand{\lcXXIIIvXLIIIfr}{\VERSE  Et Jésus lui dit: En vérité, Je te le dis, tu seras aujourd'hui avec Moi dans le paradis. \EVERSE}
\newcommand{\lcXXIIIvXLIVfr}{\VERSE  Il était environ la sixième heure, et les ténèbres couvrirent toute la terre jusqu'à la neuvième heure. \EVERSE}
\newcommand{\lcXXIIIvXLVfr}{\VERSE  Le soleil fut obscurci, et le voile du temple se déchira par le milieu. \EVERSE}
\newcommand{\lcXXIIIvXLVIfr}{\VERSE  Et criant d'une voix forte, Jésus dit: Père, Je remets Mon esprit entre Vos mains.  Et disant cela, Il expira. \EVERSE}
\newcommand{\lcXXIIIvXLVIIfr}{\VERSE  Or le centurion, voyant ce qui était arrivé, glorifia Dieu en disant: Certainement cet homme était juste. \EVERSE}
\newcommand{\lcXXIIIvXLVIIIfr}{\VERSE  Et toute la foule de ceux qui assistaient à ce spectacle, et qui voyaient ce qui se passait, s'en retournait en se frappant la poitrine. \EVERSE}
\newcommand{\lcXXIIIvXLIXfr}{\VERSE  Tous ceux qui avaient connu Jésus, et les femmes qui L'avaient suivi de Galailée, se tenaient à distance, regardant ces choses. \EVERSE}
\newcommand{\lcXXIIIvLfr}{\VERSE  Et voici qu'il y avait un homme nommé Joseph, membre du conseil, homme bon et juste, \EVERSE}
\newcommand{\lcXXIIIvLIfr}{\VERSE  qui n'avait pas consenti au dessein et aux actes des autres; il était d'Arimathie, ville de Judée, et il attendait aussi le royaume de Dieu. \EVERSE}
\newcommand{\lcXXIIIvLIIfr}{\VERSE  Cet homme alla trouver Pilate, et lui demanda le corps de Jésus. \EVERSE}
\newcommand{\lcXXIIIvLIIIfr}{\VERSE  Et l'ayant détaché de la croix, il l'enveloppa d'un linceul, et le plaça dans un sépulcre taillé dans le roc, où personne n'avait encore été mis. \EVERSE}
\newcommand{\lcXXIIIvLIVfr}{\VERSE  Or c'était le jour de la préparation, et le sabbat allait commencer. \EVERSE}
\newcommand{\lcXXIIIvLVfr}{\VERSE  Les femmes qui étaient venues de Galilée avec Jésus, ayant suivi Joseph, considérèrent le sépulcre, et comment le corps de Jésus y avait été mis. \EVERSE}
\newcommand{\lcXXIIIvLVIfr}{\VERSE  Et s'en retournant, elles préparèrent des aromates et des parfums; et, pendant le sabbat, elles se tinrent en repos, selon la loi. \EVERSE}
\newcommand{\lcXXIVvIfr}{\VERSE  Le premier jour après le sabbat, de grand matin, elles vinrent au sépulcre, apportant les aromates qu'elles avaient préparés; \EVERSE}
\newcommand{\lcXXIVvIIfr}{\VERSE  et elles trouvèrent la pierre roulée de devant le sépulcre. \EVERSE}
\newcommand{\lcXXIVvIIIfr}{\VERSE  Etant entrées, elles ne trouvèrent point le corps du Seigneur Jésus. \EVERSE}
\newcommand{\lcXXIVvIVfr}{\VERSE  Et tandis qu'elles étaient consternées de cela dans leur âme, voici que deux hommes parurent auprès d'elles, avec des vêtements resplendissants. \EVERSE}
\newcommand{\lcXXIVvVfr}{\VERSE  Et comme elles étaient saisies de frayeur, et qu'elles baissaient le visage vers la terre, ils leur dirent: Pourquoi cherchez-vous parmi les morts Celui qui est vivant? \EVERSE}
\newcommand{\lcXXIVvVIfr}{\VERSE  Il n'est point ici, mais Il est ressuscité.  Souvenez-vous de quelle manière Il vous a parlé, lorsqu'Il était encore en Galilée, \EVERSE}
\newcommand{\lcXXIVvVIIfr}{\VERSE  et qu'Il disait: Il faut que le Fils de l'homme soit livré entre les mains des pécheurs, qu'Il soit crucifié, et qu'Il ressuscite le troisième jour. \EVERSE}
\newcommand{\lcXXIVvVIIIfr}{\VERSE  Et elles se ressouvinrent de Ses paroles. \EVERSE}
\newcommand{\lcXXIVvIXfr}{\VERSE  De retour du sépulcre, elles racontèrent toutes ces choses aux onze et à tous les autres. \EVERSE}
\newcommand{\lcXXIVvXfr}{\VERSE  Ce furent Marie-Madeleine, Jeanne, et Marie mère de Jacques, et les autres qui étaient avec elles, qui rapportèrent ces choses aux Apôtres. \EVERSE}
\newcommand{\lcXXIVvXIfr}{\VERSE  Mais ces paroles leur parurent comme du délire, et ils ne les crurent point. \EVERSE}
\newcommand{\lcXXIVvXIIfr}{\VERSE  Cependant Pierre, se levant, courut au sépulcre; et s'étant baissé, il ne vit que les linges à terre; et il s'en alla, admirant en lui-même ce qui était arrivé. \EVERSE}
\newcommand{\lcXXIVvXIIIfr}{\VERSE  Et voici que ce même jour, deux d'entre eux allaient dans un bourg, nommé Emmaüs, éloigné de Jérusalem de soixante stades. \EVERSE}
\newcommand{\lcXXIVvXIVfr}{\VERSE  Et ils s'entretenaient de toutes ces choses qui s'étaient passées. \EVERSE}
\newcommand{\lcXXIVvXVfr}{\VERSE  Or il arriva, pendant qu'ils parlaient et conféraient ensemble, que Jésus Lui-même S'approcha, et marchait avec eux. \EVERSE}
\newcommand{\lcXXIVvXVIfr}{\VERSE  Mais une force empêchait leurs yeux de Le reconnaître. \EVERSE}
\newcommand{\lcXXIVvXVIIfr}{\VERSE  Et Il leur dit: Quelles sont ces paroles que vous échangez en marchant, et pourquoi êtes-vous tristes? \EVERSE}
\newcommand{\lcXXIVvXVIIIfr}{\VERSE  Prenant la parole, l'un d'eux, nommé Cléophas, Lui dit: Etes-vous seul étranger dans Jérusalem, et ne savez-vous pas ce qui s'y est passé ces jours-ci? \EVERSE}
\newcommand{\lcXXIVvXIXfr}{\VERSE  Quoi? leur dit-Il.  Et ils répondirent: Touchant Jésus de Nazareth, qui a été un prophète puissant en oeuvres et en paroles, devant Dieu et devant tout le peuple; \EVERSE}
\newcommand{\lcXXIVvXXfr}{\VERSE  et comment les princes des prêtres et nos chefs L'ont livré pour être condamné à mort, et L'ont crucifié. \EVERSE}
\newcommand{\lcXXIVvXXIfr}{\VERSE  Or nous espérions que c'était Lui qui rachèterait Israël; et maintenant, après tout cela, c'est aujourd'hui le troisième jour que ces choses se sont passées. \EVERSE}
\newcommand{\lcXXIVvXXIIfr}{\VERSE  Il est vrai que quelques femmes, qui sont des nôtres, nous ont effrayés.  Etant allées avant le jour au sépulcre, \EVERSE}
\newcommand{\lcXXIVvXXIIIfr}{\VERSE  et n'ayant pas trouvé Son corps, elles sont venues dire que des Anges leur ont apparu et ont affirmé qu'Il est vivant. \EVERSE}
\newcommand{\lcXXIVvXXIVfr}{\VERSE  Quelques-uns des nôtres sont aussi allés au sépulcre, et ont trouvé les choses comme les femmes avaient dit; mais Lui, ils ne L'ont pas trouvé. \EVERSE}
\newcommand{\lcXXIVvXXVfr}{\VERSE  Alors Il leur dit: O insensés, dont le coeur est lent à croire tout ce qu'ont dit les prophètes! \EVERSE}
\newcommand{\lcXXIVvXXVIfr}{\VERSE  Ne fallait-il pas que le Christ souffrît ces choses, et qu'Il entrât ainsi dans Sa gloire? \EVERSE}
\newcommand{\lcXXIVvXXVIIfr}{\VERSE  Et commençant par Moïse et par tous les prophètes, Il leur expliquait, dans toutes les Ecritures, ce qui Le concernait. \EVERSE}
\newcommand{\lcXXIVvXXVIIIfr}{\VERSE  Lorsqu'ils furent près du bourg où ils allaient, Il fit semblant d'aller plus loin. \EVERSE}
\newcommand{\lcXXIVvXXIXfr}{\VERSE  Mais ils Le pressèrent, en disant: Demeurez avec nous, car le soir arrive, et le jour est déjà sur son déclin.  Et Il entra avec eux. \EVERSE}
\newcommand{\lcXXIVvXXXfr}{\VERSE  Et il arriva, pendant qu'Il était à table avec eux, qu'Il prit du pain, et le bénit, et le rompit, et Il le leur présentait. \EVERSE}
\newcommand{\lcXXIVvXXXIfr}{\VERSE  Alors leurs yeux s'ouvrirent, et ils Le reconnurent; et Il disparut de devant eux. \EVERSE}
\newcommand{\lcXXIVvXXXIIfr}{\VERSE  Et ils se dirent l'un à l'autre: Est-ce que notre coeur n'était pas brûlant en nous, lorsqu'Il nous parlait sur le chemin, et qu'Il nous expliquait les Ecritures? \EVERSE}
\newcommand{\lcXXIVvXXXIIIfr}{\VERSE  Et se levant à l'heure même, ils retournèrent à Jérusalem; et ils trouvèrent les onze, et ceux qui étaient avec eux, assemblés, \EVERSE}
\newcommand{\lcXXIVvXXXIVfr}{\VERSE  et disant: Le Seigneur est vraiment ressuscité, et Il est apparu à Simon. \EVERSE}
\newcommand{\lcXXIVvXXXVfr}{\VERSE  Et ils racontaient eux-mêmes ce qui s'était passé en chemin, et comment ils L'avaient reconnu lorsqu'Il rompait le pain. \EVERSE}
\newcommand{\lcXXIVvXXXVIfr}{\VERSE  Or, pendant qu'ils parlaient ainsi, Jésus parut au milieu d'eux et leur dit: La paix soit avec vous!  C'est Moi, ne craignez point. \EVERSE}
\newcommand{\lcXXIVvXXXVIIfr}{\VERSE  Mais, troublés et épouvantés, ils croyaient voir un esprit. \EVERSE}
\newcommand{\lcXXIVvXXXVIIIfr}{\VERSE  Et Il leur dit: Pourquoi vous troublez-vous?  et pourquoi de telles pensées s'élèvent-elles dans vos coeurs? \EVERSE}
\newcommand{\lcXXIVvXXXIXfr}{\VERSE  Voyez Mes mains et Mes pieds; c'est bien Moi; touchez et voyez: un esprit n'a ni chair ni os, comme vous voyez que J'en ai. \EVERSE}
\newcommand{\lcXXIVvXLfr}{\VERSE  Et après avoir dit cela, Il leur montra Ses mains et Ses pieds. \EVERSE}
\newcommand{\lcXXIVvXLIfr}{\VERSE  Mais comme ils ne croyaient point encore et qu'ils s'étonnaient, transportés de joie, Il dit: Avez-vous ici quelque chose à manger? \EVERSE}
\newcommand{\lcXXIVvXLIIfr}{\VERSE  Ils Lui présentèrent un morceau de poisson rôti et un rayon de miel. \EVERSE}
\newcommand{\lcXXIVvXLIIIfr}{\VERSE  Et après qu'Il en eut mangé devant eux, prenant les restes, Il les leur donna. \EVERSE}
\newcommand{\lcXXIVvXLIVfr}{\VERSE  Et Il leur dit: C'est ce que Je vous disais lorsque J'étais encore avec vous, qu'il fallait que s'accomplît tout ce que a été écrit de Moi dans la loi de Moïse, dans les prophètes et dans les psaumes. \EVERSE}
\newcommand{\lcXXIVvXLVfr}{\VERSE  Alors Il leur ouvrit l'esprit, afin qu'ils comprissent les Ecritures. \EVERSE}
\newcommand{\lcXXIVvXLVIfr}{\VERSE  Et Il leur dit: C'est ainsi qu'il est écrit, et c'est ainsi qu'il fallait que le Christ souffrît, et qu'Il ressuscitât d'entre les morts le troisième jour, \EVERSE}
\newcommand{\lcXXIVvXLVIIfr}{\VERSE  et qu'on prêchât en Son nom la pénitence et la rémission des péchés dans toutes les nations, en commençant par Jérusalem. \EVERSE}
\newcommand{\lcXXIVvXLVIIIfr}{\VERSE  Or vous êtes témoins de ces choses. \EVERSE}
\newcommand{\lcXXIVvXLIXfr}{\VERSE  Et Moi, Je vais envoyer en vous le don promis par Mon Père; mais demeurez dans la ville, jusqu'à ce que vous soyez revêtus de la force d'en haut. \EVERSE}
\newcommand{\lcXXIVvLfr}{\VERSE  Puis Il les conduisit dehors, vers Béthanie; et ayant levé les mains, Il les bénit. \EVERSE}
\newcommand{\lcXXIVvLIfr}{\VERSE  Et il arriva, tandis qu'Il les bénissait, qu'Il Se sépara d'eux, et Il était enlevé au Ciel. \EVERSE}
\newcommand{\lcXXIVvLIIfr}{\VERSE  Et eux, L'ayant adoré, revinrent à Jérusalem avec une grande joie; \EVERSE}
\newcommand{\lcXXIVvLIIIfr}{\VERSE  et ils étaient sans cesse dans le temple, louant et bénissant Dieu.  Amen. \EVERSE}


\maketitle

% #############################################################################

\frontmatter

% INTRODUCTION
\include{document/intro}

\mainmatter

\part{Psautier}
\include{document/psautier}

\part{Propre du Temps}
% TEMPS DE L'AVENT
%\CHAPTER{Temps de l'Avent}

\SECTION{Premier dimanche de l'Avent}

\dominelabiamea

\ANTIPHON{\regemventurumdominum}{\regemventurumdominumfr}{Regem venturum Dominum}

\begin{quote}\emph{Venite, exultemus Domino ...}\end{quote}

\HYMNUS{\verbumsupernumprodiens}{\verbumsupernumprodiensfr}{Verbum supernum prodiens}

\begin{quote}\emph{Antiennes et psaumes au psautier selon la saison}\end{quote}

\VERSI{\exsionspecies}

\paternoster

\VERSI{\iubeillenos}

\VETUS{\isIvI\isIvII\isIvIII}{\isIvIfr\isIvIIfr\isIvIIIfr}{Isaïe I:1-3}{Lectio 1}
\VERSI{\aspiciensalonge}

\VERSI{\iubedivinum}

\VETUS[3]{\isIvIV\isIvV\isIvVI\isIvVII\isIvVIII\isIvIX}{\isIvIVfr\isIvVfr\isIvVIfr\isIvVIIfr\isIvVIIIfr\isIvIXfr}{Isaïe I:4-9}{Lectio 2}
\VERSI{\missusestgabriel}

\VERSI{\iubeevangelica}

\NOVUM{\lcXXIvXXV}{\lcXXIvXXVfr}{Luc XXI:25}{Lectio 3}
\LECTIO{\IAVi}{}{Saint Grégoire, homélie sur l'Évangile de Luc}
\VERSI{\eccediesveniunt}

\ORATIO{\excitaquaesumus}{\excitaquaesumusfr}

\conclusio

\SECTION{I\ier Lundi de l'Avent}


\backmatter
\part{Tables}

\printindex{antiphonae}{Antiennes}
\printindex{responsoria}{Répons brefs}
\printindex{orationes}{Oraisons}
\printindex{hymni}{Hymnes}
\printindex{psalmi}{Psaumes}
\printindex{vetus}{Ancien Testament}
\printindex{novum}{Nouveau Testament}
\printindex{lectiones}{Lectures}

\tableofcontents


\end{document}
